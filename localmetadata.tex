\author{Shelece Easterday}
\title{Highly complex syllable structure}
\subtitle{A typological and diachronic study}
\renewcommand{\lsSeries}{silp}
\renewcommand{\lsSeriesNumber}{9}
\dedication{For Elise, Ada, Astrid, and Maria:\\May you always be curious!}
\BookDOI{10.5281/zenodo.3268721}
\BackBody{The syllable is a natural unit of organization in spoken language whose strongest cross-linguistic patterns are often explained in terms of a universal preference for the CV structure. Syllable patterns involving long sequences of consonants are both typologically rare and theoretically marginalized, with few approaches treating these as natural or unproblematic structures. This book is an investigation of the properties of languages with highly complex syllable patterns. The two aims are (i) to establish whether these languages share other linguistic features in common such that they constitute a distinct linguistic type, and (ii) to identify possible diachronic paths and natural mechanisms by which these patterns come about in the history of a language. These issues are investigated in a diversified sample of 100 languages, 25 of which have highly complex syllable patterns.

Languages with highly complex syllable structure are characterized by a number of phonetic, phonological, and morphological features which serve to set them apart from languages with simpler syllable patterns. These include specific segmental and suprasegmental properties, a higher prevalence of vowel reduction processes with extreme outcomes, and higher average morpheme/word ratios. The results suggest that highly complex syllable structure is a linguistic type distinct from but sharing some characteristics with other proposed holistic phonological types, including stress-timed and consonantal languages. The results point to word stress and specific patterns of gestural organization as playing important roles in the diachronic development of these patterns out of simpler syllable structures.}
\renewcommand{\lsISBNdigital}{978-3-96110-194-8}
\renewcommand{\lsISBNhardcover}{978-3-96110-195-5}
\renewcommand{\lsID}{249}
