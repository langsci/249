% This file was converted to LaTeX by Writer2LaTeX ver. 1.4
% see http://writer2latex.sourceforge.net for more info
\documentclass[12pt]{article}
\usepackage[utf8]{inputenc}
\usepackage[T1]{fontenc}
\usepackage[english]{babel}
\usepackage{amsmath}
\usepackage{amssymb,amsfonts,textcomp}
\usepackage{array}
\usepackage{hhline}
\usepackage{hyperref}
\hypersetup{colorlinks=true, linkcolor=blue, citecolor=blue, filecolor=blue, urlcolor=blue}
\newcommand\textsubscript[1]{\ensuremath{{}_{\text{#1}}}}
% Headings and outline numbering
\makeatletter
\renewcommand\subsection{\@startsection{subsection}{2}{0.0in}{0in}{0.1mm}{\normalfont\normalsize\fontsize{18pt}{21.6pt}\selectfont\rmfamily\bfseries\upshape\raggedright}}
\renewcommand\@seccntformat[1]{\csname @textstyle#1\endcsname{\csname the#1\endcsname}\csname @distance#1\endcsname}
\setcounter{secnumdepth}{0}
\newcommand\@distancesection{}
\newcommand\@textstylesection[1]{#1}
\newcommand\@distancesubsection{}
\newcommand\@textstylesubsection[1]{#1}
\makeatother
\raggedbottom
% Paragraph styles
\renewcommand\familydefault{\rmdefault}
\renewcommand\seriesdefault{\mddefault}
\renewcommand\shapedefault{\updefault}
\newenvironment{styleBody}{\renewcommand\baselinestretch{1.0}\setlength\leftskip{0in}\setlength\rightskip{0in plus 1fil}\setlength\parindent{0in}\setlength\parfillskip{0pt plus 1fil}\setlength\parskip{0in plus 1pt}\writerlistparindent\writerlistleftskip\leavevmode\normalfont\normalsize\fontsize{11pt}{13.2pt}\selectfont\mdseries\upshape\writerlistlabel\ignorespaces}{\unskip\vspace{0in plus 1pt}\par}
% List styles
\newcommand\writerlistleftskip{}
\newcommand\writerlistparindent{}
\newcommand\writerlistlabel{}
\newcommand\writerlistremovelabel{\aftergroup\let\aftergroup\writerlistparindent\aftergroup\relax\aftergroup\let\aftergroup\writerlistlabel\aftergroup\relax}
\title{}
\author{}
\date{}
\begin{document}
\clearpage\setcounter{page}{1}\subsection[Appendix B: Data]{Appendix B: Data}
\begin{styleBody}
This appendix contains the coded data used for the various studies in the book. The languages are listed alphabetically by ISO 639-3 code.
\end{styleBody}

\begin{styleBody}
————
\end{styleBody}

\begin{styleBody}
\textbf{[alc] }\ \ \textbf{\textsc{Qawasqar}}\textbf{\ \ }Kawesqar, \textit{North Central Alacalufan} (Chile)
\end{styleBody}

\begin{styleBody}
\textbf{References consulted: }Aguilera (2001), Clairis (1977, 1985), Viegas Barros (1990)
\end{styleBody}

\begin{styleBody}
\textbf{Sound inventory}
\end{styleBody}

\begin{styleBody}
\textbf{C phoneme inventory:} /p p[2B0?] p’ t t[2B0?] t’ q q[2B0?] q’ t[361?]s t[361?]s’ s f x h m n l [27E?] w j/
\end{styleBody}

\begin{styleBody}
\textbf{N consonant phonemes:} 21
\end{styleBody}

\begin{styleBody}
\textbf{Geminates:} N/A
\end{styleBody}

\begin{styleBody}
\textbf{Voicing contrasts:} None
\end{styleBody}

\begin{styleBody}
\textbf{Places:} Bilabial, Alveolar, Velar, Uvular, Glottal
\end{styleBody}

\begin{styleBody}
\textbf{Manners:} Stop, Affricate, Fricative, Nasal, Flap/Tap, Central approximant, Lateral approximant
\end{styleBody}

\begin{styleBody}
\textbf{Elaborations:} Post-aspiration, Ejective, Uvular
\end{styleBody}

\begin{styleBody}
\textbf{N elaborations:} 3
\end{styleBody}

\begin{styleBody}
\textbf{N elaborated consonants: }11
\end{styleBody}

\begin{styleBody}
\textbf{V phoneme inventory:} /e a o/
\end{styleBody}

\begin{styleBody}
\textbf{N vowel qualities:} 3
\end{styleBody}

\begin{styleBody}
\textbf{Diphthongs or vowel sequences:} Diphthongs /aw ow/
\end{styleBody}

\begin{styleBody}
\textbf{Contrastive length:} None
\end{styleBody}

\begin{styleBody}
\textbf{Contrastive nasalization:} None
\end{styleBody}

\begin{styleBody}
\textbf{Other contrasts:} N/A
\end{styleBody}

\begin{styleBody}
\textbf{Notes: }Clairis gives minimal pairs for /p[2B0?] t[2B0?]/, gives /q’ q[2B0?]/ but not /k k’/. /e o/ vary quite widely. /e/ is [[259?]] 65.9\% of the time word-medially. Clairis and Viegas Barros both consider glides and high vowels to be in complementary distribution, but have chosen glides as lexical representation.
\end{styleBody}

\begin{styleBody}
\textbf{Syllable structure}
\end{styleBody}

\begin{styleBody}
\textbf{Complexity Category: }Highly Complex
\end{styleBody}

\begin{styleBody}
\textbf{Canonical syllable structure:} (C)(C)(C)(C)V(C)(C)(C)\textbf{ }(Clairis 1985: 391-401)
\end{styleBody}

\begin{styleBody}
\textbf{Size of maximal onset:} 4
\end{styleBody}

\begin{styleBody}
\textbf{Size of maximal coda:} 3
\end{styleBody}

\begin{styleBody}
\textbf{Onset obligatory:} No
\end{styleBody}

\begin{styleBody}
\textbf{Coda obligatory:} No
\end{styleBody}

\begin{styleBody}
\textbf{Vocalic nucleus patterns:} Short vowels, Diphthongs
\end{styleBody}

\begin{styleBody}
\textbf{Syllabic consonant patterns:} N/A
\end{styleBody}

\begin{styleBody}
\textbf{Size of maximal word{}-marginal sequences with syllabic obstruents:} N/A
\end{styleBody}

\begin{styleBody}
\textbf{Predictability of syllabic consonants:} N/A 
\end{styleBody}

\begin{styleBody}
\textbf{Morphological constituency of maximal syllable margin:} Morpheme-internal (Onset), Both patterns (Coda)
\end{styleBody}

\begin{styleBody}
\textbf{Morphological pattern of syllabic consonants:} N/A
\end{styleBody}

\begin{styleBody}
\textbf{Onset restrictions: }All consonants except /[27E?]/ occur in simple onsets. In biconsonantal onsets, C\textsubscript{1} may be /f q q' s t/ and C\textsubscript{2} may be /t[361?][283?] s t t' j w q/. Triconsonantal onsets have same restrictions for C1; include /qsq, qst, sqw/. Example given of four-consonant onset is /qsqj/.
\end{styleBody}

\begin{styleBody}
\textbf{Coda restrictions:} /f j l m n [27E?] s w/ do not appear in simple codas. In biconsonantal codas, C\textsubscript{1} is /f j l m n p q [27E?] t w s/ and C\textsubscript{2} is /s q/. Triconsonantal codas include /lqs, rqs, qsq/.
\end{styleBody}

\begin{styleBody}
\textbf{Notes: }Clairis notes that large clusters are “unstable in rapid speech”, e.g. \textit{qsqa[27E?] {\textgreater} sqa[27E?]}, ‘urine’ but that rapid speech can also produce clusters, e.g., future marker \textit{seqwe {\textgreater} sqwe} (1985: 393).
\end{styleBody}

\begin{styleBody}
\textbf{Suprasegmentals}
\end{styleBody}

\begin{styleBody}
\textbf{Tone: }No
\end{styleBody}

\begin{styleBody}
\textbf{Word stress: }Disagreement (Clairis 1977 claims stress, Clairis 1985 claims not, that it varies across different tokens of same word or doesn’t occur at all).
\end{styleBody}

\begin{styleBody}
\textbf{Vowel reduction processes}
\end{styleBody}

\begin{styleBody}
\textbf{alc-R1: }Low vowel /a/ and mid front vowel /e/, and to a lesser extent mid back vowel /o/, are frequently realized as [[259?]] (Clairis 1985: 382-4; conditioning environment not described).
\end{styleBody}

\begin{styleBody}
\textbf{alc-R2:} A word-initial vowel is often syncopated in rapid speech (Clairis 1985: 393).
\end{styleBody}

\begin{styleBody}
\textbf{alc-R3:} An interconsonantal vowel is often syncopated in rapid speech. Apparently only some consonants condition this process, but particulars are not described (Clairis 1985: 393).
\end{styleBody}

\begin{styleBody}
\textbf{Consonant allophony processes}
\end{styleBody}

\begin{styleBody}
\textbf{alc-C1:} Voiceless uvular stop [q] varies freely with affricated variant [qx] (Clairis 1985: 378).
\end{styleBody}

\begin{styleBody}
\textbf{alc-C2: }\ Bilabial stop may be realized as a fricative (Aguilera 2001).
\end{styleBody}

\begin{styleBody}
\textbf{alc-C3: }Voiceless alveolar fricative [s] may be realized as [h] word-finally (Clairis 1985: 372).
\end{styleBody}

\begin{styleBody}
\textbf{alc-C4: }Aspirated uvular stop [q\textsuperscript{h}] and velar fricative [x] vary freely with [h] (Clairis 1985: 377-8).
\end{styleBody}

\begin{styleBody}
\textbf{Morphology}
\end{styleBody}

\begin{styleBody}
(adequate texts unavailable)
\end{styleBody}

\clearpage\begin{styleBody}
\textbf{[als] }\ \ \textbf{\textsc{Albanian (Tosk dialect)}}\textbf{\ \ }Indo-European, \textit{Albanian} (Albania, Serbia and Montenegro)
\end{styleBody}

\begin{styleBody}
\textbf{References consulted:} Bevington (1974), Klippenstein (2010), Newmark (1957), Newmark et al. (1982)
\end{styleBody}

\begin{styleBody}
\textbf{Sound inventory}
\end{styleBody}

\begin{styleBody}
\textbf{C phoneme inventory:} /p b t d c [25F?] k [261?] t[361?]s d[361?]z t[361?][283?] d[361?][292?] f v $\theta $ ð s z [283?] [292?] h m n [272?] l [26B?] [27E?] r j/
\end{styleBody}

\begin{styleBody}
\textbf{N consonant phonemes:} 29
\end{styleBody}

\begin{styleBody}
\textbf{Geminates:} N/A
\end{styleBody}

\begin{styleBody}
\textbf{Voicing contrasts:} Obstruents
\end{styleBody}

\begin{styleBody}
\textbf{Places:} Bilabial, Dental, Alveolar, Palato-alveolar, Palatal, Velar, Glottal
\end{styleBody}

\begin{styleBody}
\textbf{Manners:} Stop, Affricate, Fricative, Nasal, Flap/Tap, Trill, Central approximant, Lateral approximant
\end{styleBody}

\begin{styleBody}
\textbf{N elaborations:} 4
\end{styleBody}

\begin{styleBody}
\textbf{Elaborations:} Voiced fricatives/affricates, Labiodental, Palato-alveolar, Velarization
\end{styleBody}

\begin{styleBody}
\textbf{V phoneme inventory:} /i y [25B?] [259?] a [254?] u/
\end{styleBody}

\begin{styleBody}
\textbf{N vowel qualities:} 7
\end{styleBody}

\begin{styleBody}
\textbf{Diphthongs or vowel sequences:} Diphthongs /ie ua ye ue/
\end{styleBody}

\begin{styleBody}
\textbf{Contrastive length:} None
\end{styleBody}

\begin{styleBody}
\textbf{Contrastive nasalization:} None
\end{styleBody}

\begin{styleBody}
\textbf{Other contrasts:} N/A
\end{styleBody}

\begin{styleBody}
\textbf{Notes:} Vowel length and nasalization contrasts occur in Gheg dialect.
\end{styleBody}

\begin{styleBody}
\textbf{Syllable structure}
\end{styleBody}

\begin{styleBody}
\textbf{Complexity Category:} Highly Complex
\end{styleBody}

\begin{styleBody}
\textbf{Canonical syllable structure: }(C)(C)(C)(C)V(C)(C)(C) (Newmark 1957: 24-9, Klippenstein 2010)
\end{styleBody}

\begin{styleBody}
\textbf{Size of maximal onset:} 4
\end{styleBody}

\begin{styleBody}
\textbf{Size of maximal coda:} 3
\end{styleBody}

\begin{styleBody}
\textbf{Onset obligatory:} No
\end{styleBody}

\begin{styleBody}
\textbf{Coda obligatory:} No
\end{styleBody}

\begin{styleBody}
\textbf{Vocalic nucleus patterns:} Short vowels, Diphthongs
\end{styleBody}

\begin{styleBody}
\textbf{Syllabic consonant patterns:} N/A
\end{styleBody}

\begin{styleBody}
\textbf{Size of maximal word{}-marginal sequences with syllabic obstruents:} N/A
\end{styleBody}

\begin{styleBody}
\textbf{Predictability of syllabic consonants:} N/A
\end{styleBody}

\begin{styleBody}
\textbf{Morphological constituency of maximal syllable margin:} Morpheme-internal (Coda), Morphologically Complex (Onset)
\end{styleBody}

\begin{styleBody}
\textbf{Morphological pattern of syllabic consonants:} N/A
\end{styleBody}

\begin{styleBody}
\textbf{Onset restrictions:} Apparently no restrictions for simple onsets. Biconsonantal onsets quite varied, include /t[361?][283?]c tk [292?]b fl t[27E?] pj zv mp rj/, with voicing mismatches typically avoided. Triconsonantal onsets include /[283?]pr, skl, skt, p[283?]t, [292?]vl, ndr, mbl/, Four-consonant onsets include /t[361?][283?]mpl, zmbr/.
\end{styleBody}

\begin{styleBody}
\textbf{Coda restrictions:} In simple codas, apparently /c h/ do not occur. Biconsonantal codas include /jt, [27E?]p, [27E?]f, mp, ls, fk, ps, tk, \ k$\theta $, t[361?]sk, [292?]d/. Triconsonantal codas always end in a voiceless sibilant plus /t/, include /p[283?]t, kst/.
\end{styleBody}

\begin{styleBody}
\textbf{Notes:} Klippenstein (2010) shows that there are some onset clusters not listed by Newmark which occur.
\end{styleBody}

\begin{styleBody}
\textbf{Suprasegmentals}
\end{styleBody}

\begin{styleBody}
\textbf{Tone:} No
\end{styleBody}

\begin{styleBody}
\textbf{Word stress: }Yes
\end{styleBody}

\begin{styleBody}
\textbf{Stress placement:} Morphologically or Lexically Conditioned
\end{styleBody}

\begin{styleBody}
\textbf{Phonetic processes conditioned by stress:} Vowel Reduction
\end{styleBody}

\begin{styleBody}
\textbf{Differences in phonological properties of stressed and unstressed syllables:} (None)
\end{styleBody}

\begin{styleBody}
\textbf{Phonetic correlates of stress: }Vowel duration (impressionistic), Intensity (impressionistic)
\end{styleBody}

\begin{styleBody}
\textbf{Notes:} In words without inflection, stress is final if that syllable is closed or ends in non-mid vowel, while stress falls on penultimate if final syllable ends in mid vowel (even if penultimate ends in mid vowel) (Trommer 2013). While vowel quality factors into stress assignment, it appears that there is no difference in the vowel quality contrasts in stressed and unstressed syllables.
\end{styleBody}

\begin{styleBody}
\textbf{Vowel reduction processes}
\end{styleBody}

\begin{styleBody}
\textbf{als-R1:} For many speakers in ordinary speech, unstressed /[259?]/ is not pronounced when word-final following a consonant (Newmark et al. 1982: 11; for older speakers in southern Tosk region, vowel is retained but pronounced as [[26A?]] in this context).
\end{styleBody}

\begin{styleBody}
\textbf{als-R2:} In rapid speech, mid central vowel /[259?]/ is optionally deleted when occurring between two consonants, of which C\textsubscript{1} is not /s z [292?]/. This deletion rarely occurs when both C\textsubscript{1} and C\textsubscript{2} are voiced (Klippenstein 2010: 21-2).
\end{styleBody}

\begin{styleBody}
\textbf{Consonant allophony processes}
\end{styleBody}

\begin{styleBody}
\textbf{als-C1}: Fricatives /f $\theta $ v ð/ have occasional homorganic stop allophones pre-juncture and preceding a consonant (Newmark 1957).
\end{styleBody}

\begin{styleBody}
\textbf{Morphology}
\end{styleBody}

\begin{styleBody}
(adequate texts unavailable)
\end{styleBody}

\clearpage\begin{styleBody}
\textbf{[aly] }\ \ \textbf{\textsc{Alyawarra}}\textbf{\ \ }Pama-Nyungan, \textit{Arandic-Thura-Yura} (Australia)
\end{styleBody}

\begin{styleBody}
\textbf{References consulted: }Yallop (1977)
\end{styleBody}

\begin{styleBody}
\textbf{Sound inventory}
\end{styleBody}

\begin{styleBody}
\textbf{C phoneme inventory:} /p t[32A?] t [288?] c k p\textsuperscript{m} t[32A?]\textsuperscript{n[32A?]} t\textsuperscript{n} [288?]\textsuperscript{[273?]} c\textsuperscript{[272?]} k\textsuperscript{ŋ} m n[32A?] n [273?] [272?] ŋ l[32A?] l [26D?] [28E?] r [27B?] w j [263?][31E?]/
\end{styleBody}

\begin{styleBody}
\textbf{N consonant phonemes:} 27
\end{styleBody}

\begin{styleBody}
\textbf{Geminates:} N/A
\end{styleBody}

\begin{styleBody}
\textbf{Voicing contrasts: }None
\end{styleBody}

\begin{styleBody}
\textbf{Places:} Bilabial, Dental, Alveolar, Retroflex, Palatal, Velar
\end{styleBody}

\begin{styleBody}
\textbf{Manners:} Stop, Nasal, Trill, Central approximant, Lateral approximant
\end{styleBody}

\begin{styleBody}
\textbf{N elaborations:} 2
\end{styleBody}

\begin{styleBody}
\textbf{Elaborations:} Nasal release, Retroflex
\end{styleBody}

\begin{styleBody}
\textbf{V phoneme inventory:} /i a u i[2D0?] u[2D0?]/
\end{styleBody}

\begin{styleBody}
\textbf{N vowel qualities:} 3
\end{styleBody}

\begin{styleBody}
\textbf{Diphthongs or vowel sequences:} Diphthongs /ai au/
\end{styleBody}

\begin{styleBody}
\textbf{Contrastive length:} Some
\end{styleBody}

\begin{styleBody}
\textbf{Contrastive nasalization:} None
\end{styleBody}

\begin{styleBody}
\textbf{Other contrasts:} N/A
\end{styleBody}

\begin{styleBody}
\textbf{Notes:} /[263?]/ doesn’t occur in W. Arrernte, but is retained in Alyawarra (Yallop 1977: 12).
\end{styleBody}

\begin{styleBody}
\textbf{Syllable structure}
\end{styleBody}

\begin{styleBody}
\textbf{Complexity Category: }Moderately Complex
\end{styleBody}

\begin{styleBody}
\textbf{Canonical syllable structure:} (C)(C)V(C)\textbf{ }(Yallop 1977: 41-5)
\end{styleBody}

\begin{styleBody}
\textbf{Size of maximal onset:} 2
\end{styleBody}

\begin{styleBody}
\textbf{Size of maximal coda:} 1
\end{styleBody}

\begin{styleBody}
\textbf{Onset obligatory:} No
\end{styleBody}

\begin{styleBody}
\textbf{Coda obligatory:} No
\end{styleBody}

\begin{styleBody}
\textbf{Vocalic nucleus patterns:} Short vowels, Long vowels
\end{styleBody}

\begin{styleBody}
\textbf{Syllabic consonant patterns:} Nasal
\end{styleBody}

\begin{styleBody}
\textbf{Size of maximal word{}-marginal sequences with syllabic obstruents:} N/A
\end{styleBody}

\begin{styleBody}
\textbf{Predictability of syllabic consonants:} Varies with VC sequence
\end{styleBody}

\begin{styleBody}
\textbf{Morphological constituency of maximal syllable margin:} Morpheme-internal (Onset)
\end{styleBody}

\begin{styleBody}
\textbf{Morphological pattern of syllabic consonants:} N/A
\end{styleBody}

\begin{styleBody}
\textbf{Onset restrictions:} C\textsubscript{1} may be a plosive, nasal-released plosive, or nasal. C\textsubscript{2} is always /w/.
\end{styleBody}

\begin{styleBody}
\textbf{Coda restrictions:} Nasals, laterals, and trills most common; less commonly, plosives and nasal-released plosives may occur. There are no word-final codas.
\end{styleBody}

\begin{styleBody}
\textbf{Notes:} The sequence /ŋkw/ seems to occur invariably as an onset cluster in the word \textit{ŋkwa[26D?]a} ‘sugar, sweetness’; however this is only true phrase-initially. In connected speech when following another word, words without an initial vowel always occur with a linking vowel (quality determined by initial consonant), which alters the syllable structure (Yallop 1977: 28-30). Since the 3-C onset seems to be a very marginal pattern, I take the canonical syllable structure of the language to be consistent with the (C)(C)V(C) pattern reported by Yallop later in the text.
\end{styleBody}

\begin{styleBody}
\textbf{Suprasegmentals}
\end{styleBody}

\begin{styleBody}
\textbf{Tone:} No
\end{styleBody}

\begin{styleBody}
\textbf{Word stress:} Yes
\end{styleBody}

\begin{styleBody}
\textbf{Stress placement:} Fixed
\end{styleBody}

\begin{styleBody}
\textbf{Phonetic processes conditioned by stress:} Vowel Reduction, Consonant Allophony in Stressed Syllables
\end{styleBody}

\begin{styleBody}
\textbf{Differences in phonological properties of stressed and unstressed syllables:} (None)
\end{styleBody}

\begin{styleBody}
\textbf{Phonetic correlates of stress: }Vowel duration (impressionistic), Pitch (impressionistic), Intensity (impressionistic)
\end{styleBody}

\begin{styleBody}
\textbf{Notes: }Duration is a correlate of stress for only some syllables.
\end{styleBody}

\begin{styleBody}
\textbf{Vowel reduction processes}
\end{styleBody}

\begin{styleBody}
\textbf{aly-R1:} High vowels /i u/ tend to be centralized and preceded by a glide in word-initial (unstressed) position (Yallop 1977: 25).
\end{styleBody}

\begin{styleBody}
\textbf{aly-R2:} Low short vowel /a/ is reduced to mid when occurring word-initially or -finally (and therefore unstressed) (Yallop 1977: 25).
\end{styleBody}

\begin{styleBody}
\textbf{aly-R3:} Low short vowel /a/ is often dropped word-initially (and therefore unstressed) before a single consonant (Yallop 1977: 28).
\end{styleBody}

\begin{styleBody}
\textbf{aly-R4:} In normal connected speech, short unstressed vowels are often elided altogether before continuants (Yallop 1977: 27).
\end{styleBody}

\begin{styleBody}
\textbf{aly-R5:} When low short vowel /a/ is dropped in word-initial, unstressed position before a sequence of consonants, the first may become syllabic (results in syllabic nasals, Yallop 1977: 19).
\end{styleBody}

\begin{styleBody}
\textbf{Consonant allophony processes}
\end{styleBody}

\begin{styleBody}
\textbf{aly-C1}: Palatal stop /c/ is often realized as affricate [cç] (Yallop 1977: 21).
\end{styleBody}

\begin{styleBody}
\textbf{aly-C2}: Lateral approximants may be realized as fricatives following a sequence of /ij/ or /aj/ and preceding a plosive (Yallop 1977: 19).
\end{styleBody}

\begin{styleBody}
\textbf{aly-C3}: A trill may be realized as palato-alveolar fricative [[292?]] following a dental or alveolar consonant (Yallop 1977: 19).
\end{styleBody}

\begin{styleBody}
\textbf{Morphology}
\end{styleBody}

\begin{styleBody}
(adequate texts unavailable)
\end{styleBody}

\clearpage\begin{styleBody}
\textbf{[amp] }\ \ \textbf{\textsc{Alamblak}}\textbf{\ \ }Sepik, \textit{Sepik Hill} (Papua New Guinea)
\end{styleBody}

\begin{styleBody}
\textbf{References consulted: }Bruce (1984), SIL (2004)
\end{styleBody}

\begin{styleBody}
\textbf{Sound inventory}
\end{styleBody}

\begin{styleBody}
\textbf{C phoneme inventory:} /p b t d k [261?] t[361?][283?] d[361?][292?] [278?] s [283?] x m n [272?] [27E?] w j/
\end{styleBody}

\begin{styleBody}
\textbf{N consonant phonemes:} 18
\end{styleBody}

\begin{styleBody}
\textbf{Geminates:} N/A
\end{styleBody}

\begin{styleBody}
\textbf{Voicing contrasts: }Obstruents
\end{styleBody}

\begin{styleBody}
\textbf{Places: }Bilabial, Alveolar, Palato-alveolar, Velar
\end{styleBody}

\begin{styleBody}
\textbf{Manners:} Stop, Affricate, Fricative, Nasal, Flap/Tap, Central approximant
\end{styleBody}

\begin{styleBody}
\textbf{N elaborations:} 2
\end{styleBody}

\begin{styleBody}
\textbf{Elaborations:} Voiced fricatives/affricates, Palato-alveolar
\end{styleBody}

\begin{styleBody}
\textbf{V phoneme inventory:} /i e [268?] [259?] a o u/
\end{styleBody}

\begin{styleBody}
\textbf{N vowel qualities:} 7
\end{styleBody}

\begin{styleBody}
\textbf{Diphthongs or vowel sequences:} Diphthongs /ai [268?]i ui oi au/
\end{styleBody}

\begin{styleBody}
\textbf{Contrastive length:} None
\end{styleBody}

\begin{styleBody}
\textbf{Contrastive nasalization:} None
\end{styleBody}

\begin{styleBody}
\textbf{Other contrasts:} N/A
\end{styleBody}

\begin{styleBody}
\textbf{Notes:} For affricates, SIL OPD gives only /d[361?][291?]/. /[268?]/ included in Bruce 1984, but not SIL OPD.
\end{styleBody}

\begin{styleBody}
\textbf{Syllable structure}
\end{styleBody}

\begin{styleBody}
\textbf{Complexity Category:} Highly Complex
\end{styleBody}

\begin{styleBody}
\textbf{Canonical syllable structure:} (C)(C)(C)V(C)(C)(C)\textbf{ }(Bruce 1984, SIL 2004)
\end{styleBody}

\begin{styleBody}
\textbf{Size of maximal onset:} 3
\end{styleBody}

\begin{styleBody}
\textbf{Size of maximal coda:} 3
\end{styleBody}

\begin{styleBody}
\textbf{Onset obligatory:} No
\end{styleBody}

\begin{styleBody}
\textbf{Coda obligatory:} No
\end{styleBody}

\begin{styleBody}
\textbf{Vocalic nucleus patterns:} Short vowels, Diphthongs
\end{styleBody}

\begin{styleBody}
\textbf{Syllabic consonant patterns:} Nasal
\end{styleBody}

\begin{styleBody}
\textbf{Size of maximal word{}-marginal sequences with syllabic obstruents:} N/A
\end{styleBody}

\begin{styleBody}
\textbf{Predictability of syllabic consonants:} Predictable from word/consonantal context
\end{styleBody}

\begin{styleBody}
\textbf{Morphological constituency of maximal syllable margin:} Morpheme-internal (Onset), Both patterns (Coda)
\end{styleBody}

\begin{styleBody}
\textbf{Morphological pattern of syllabic consonants:} Lexical items
\end{styleBody}

\begin{styleBody}
\textbf{Onset restrictions:} No restrictions on simple onsets. Biconsonantal onsets include /sk tw [261?]w [283?]w k[27E?] pk/. Triconsonantal onsets include /tkm, tkb, p[27E?]t, tkm, k[27E?]p, b[27E?]b, mxt/.
\end{styleBody}

\begin{styleBody}
\textbf{Coda restrictions:} It seems there are some restrictions on simple codas, including /b [261?] w j/. Biconsonantal codas include /nt, [27E?]t, s[27E?], [27E?]s, [261?]t/. Triconsonantal codas include /[272?][272?]t[361?][283?] ndt mbt/.
\end{styleBody}

\begin{styleBody}
\textbf{Notes:} SIL OPD lists some larger onsets, e.g. /kmb[27E?]/, and vowelless words, e.g. /kpt/. Unclear whether forms listed in SIL OPD have alternate forms with epenthetic vowel or if these are fully regular patterns. Syllabification in Bruce (1984) does give three-obstruent onset (\textit{jak[2C8?]tkb[259?]tk[268?]k[268?]b[259?]t} ‘to get and mash’, p. 60). Analysis of syllable structure dependent on analysis/status of high central vocoid /[268?]/. Bruce discusses possible history of this vowel and development of some of these clusters (1984: 69-70).
\end{styleBody}

\begin{styleBody}
\textbf{Suprasegmentals}
\end{styleBody}

\begin{styleBody}
\textbf{Tone:} No
\end{styleBody}

\begin{styleBody}
\textbf{Word stress:} Yes
\end{styleBody}

\begin{styleBody}
\textbf{Stress placement:} Weight-Sensitive
\end{styleBody}

\begin{styleBody}
\textbf{Phonetic processes conditioned by stress:} Vowel Reduction
\end{styleBody}

\begin{styleBody}
\textbf{Differences in phonological properties of stressed and unstressed syllables:} (None)
\end{styleBody}

\begin{styleBody}
\textbf{Phonetic correlates of stress: }Not described
\end{styleBody}

\begin{styleBody}
\textbf{Vowel reduction processes}
\end{styleBody}

\begin{styleBody}
\textbf{amp-R1:} Tense high front vowel /i/ is optionally realized as lax [[26A?]] when occurring after a palatal glide /j/ in an unstressed syllable (Bruce 1984: 37).
\end{styleBody}

\begin{styleBody}
\textbf{amp-R2:} A tense mid front vowel /e/ may be realized as lax [[25B?]] in unstressed syllables (Bruce 1984: 38).
\end{styleBody}

\begin{styleBody}
\textbf{amp-R3:} Mid back rounded vowel /o/ is shortened when preceded by a velar consonant and followed by an alveolar consonant (Bruce 1984: 39).
\end{styleBody}

\begin{styleBody}
\textbf{amp-R4:} Mid central vowel /[259?]/ may be realized as high central vowel [[268?]] preceding a consonant-initial stressed syllable (Bruce 1984: 41).
\end{styleBody}

\begin{styleBody}
\textbf{Consonant allophony processes}
\end{styleBody}

\begin{styleBody}
\textbf{amp-C1: }Alveolars may be realized as palatal or palato-alveolar following a palatal consonant (including glides) (Bruce 1984: 29).
\end{styleBody}

\begin{styleBody}
\textbf{amp-C2}: Fricatives are voiced when occurring after a voiced non-nasal and before a voiced consonant (Bruce 1984: 25).
\end{styleBody}

\begin{styleBody}
\textbf{amp-C3:} A labiovelar approximant is realized as a vocalic offglide [\textsuperscript{o}] following a mid or low vowel and preceding a peripheral consonant (Bruce 1984: 28).
\end{styleBody}

\begin{styleBody}
\textbf{Morphology}
\end{styleBody}

\begin{styleBody}
\textbf{Text: }“The spirit who turned into an animal” (Bruce 1984: 323-331)
\end{styleBody}

\begin{styleBody}
\textbf{Synthetic index: }2.5 morphemes/word (1264 morphemes, 502 words)
\end{styleBody}

\clearpage\begin{styleBody}
\textbf{[aot] }\ \ \textbf{\textsc{Atong\ \ }}\textbf{\ \ }Sino-Tibetan, \textit{Brahmaputran} (Bangladesh, India)
\end{styleBody}

\begin{styleBody}
\textbf{References consulted: }Van Breugel (2008)
\end{styleBody}

\begin{styleBody}
\textbf{Sound inventory}
\end{styleBody}

\begin{styleBody}
\textbf{C phoneme inventory:} /p p[2B0?] b t t[2B0?] d k k[2B0?] [261?] t[361?][255?] d[361?][291?] [255?] h m n ŋ [27E?] l w j/
\end{styleBody}

\begin{styleBody}
\textbf{N consonant phonemes:} 20
\end{styleBody}

\begin{styleBody}
\textbf{Geminates:} N/A
\end{styleBody}

\begin{styleBody}
\textbf{Voicing contrasts: }Obstruents
\end{styleBody}

\begin{styleBody}
\textbf{Places: }Bilabial, Alveolar, Alveolo-palatal, Velar, Glottal
\end{styleBody}

\begin{styleBody}
\textbf{Manners:} Stop, Affricate, Fricative, Nasal, Flap/tap, Central approximant, Lateral approximant
\end{styleBody}

\begin{styleBody}
\textbf{N elaborations:} 2
\end{styleBody}

\begin{styleBody}
\textbf{Elaborations:} Voiced fricatives/affricates, Post-aspiration
\end{styleBody}

\begin{styleBody}
\textbf{V phoneme inventory:} /i e [259?] a o u/
\end{styleBody}

\begin{styleBody}
\textbf{N vowel qualities:} 6
\end{styleBody}

\begin{styleBody}
\textbf{Diphthongs or vowel sequences:} None
\end{styleBody}

\begin{styleBody}
\textbf{Contrastive length:} None
\end{styleBody}

\begin{styleBody}
\textbf{Contrastive nasalization:} None
\end{styleBody}

\begin{styleBody}
\textbf{Other contrasts:} N/A
\end{styleBody}

\begin{styleBody}
\textbf{Notes:} /[27E?]/ has a trill variant. Van Breugel analyzes vowel-glides sequences as VC rather than diphthongs on the basis of distributional and perceptual evidence (2008: 48).
\end{styleBody}

\begin{styleBody}
\textbf{Syllable structure}
\end{styleBody}

\begin{styleBody}
\textbf{Complexity Category:} Moderately Complex
\end{styleBody}

\begin{styleBody}
\textbf{Canonical syllable structure:} (C)(C)V(C)\textbf{ }(van Breugel 2008: 43)
\end{styleBody}

\begin{styleBody}
\textbf{Size of maximal onset:} 2
\end{styleBody}

\begin{styleBody}
\textbf{Size of maximal coda: }1
\end{styleBody}

\begin{styleBody}
\textbf{Onset obligatory:} No
\end{styleBody}

\begin{styleBody}
\textbf{Coda obligatory:} No
\end{styleBody}

\begin{styleBody}
\textbf{Vocalic nucleus patterns: }Short vowels
\end{styleBody}

\begin{styleBody}
\textbf{Syllabic consonant patterns:} N/A
\end{styleBody}

\begin{styleBody}
\textbf{Size of maximal word{}-marginal sequences with syllabic obstruents:} N/A
\end{styleBody}

\begin{styleBody}
\textbf{Predictability of syllabic consonants:} N/A
\end{styleBody}

\begin{styleBody}
\textbf{Morphological constituency of maximal syllable margin:} N/A
\end{styleBody}

\begin{styleBody}
\textbf{Morphological pattern of syllabic consonants:} N/A
\end{styleBody}

\begin{styleBody}
\textbf{Onset restrictions:} All but /ŋ j/ may occur as simple onsets. C+[27E?] onsets may occur word-internally.
\end{styleBody}

\begin{styleBody}
\textbf{Coda restrictions:} Restricted to /p t k [255?] m n ŋ l w j/. 
\end{styleBody}

\begin{styleBody}
\textbf{Notes:} In non-initial syllables, C[259?][27E?]\~{}C[27E?] variation may occur, e.g. /ha[294?]b[259?][27E?]i/ {\textgreater} [ha[294?]b[259?][27E?]i]\~{}[ha[294?]b[27E?]i]. The variant with the schwa is most common, so Van Breugel analyzes this as a process of vowel reduction (2008: 43). However, comparative evidence from Boro-Garo, as well as language-internal examples given in the text, suggest that the /C[27E?]/ clusters are original and have been variably resolved in the modern language through consonant deletion and schwa insertion (2008: 30-32). Thus I take non-initial C+[27E?] clusters to be canonical onsets.
\end{styleBody}

\begin{styleBody}
\textbf{Suprasegmentals}
\end{styleBody}

\begin{styleBody}
\textbf{Tone:} No
\end{styleBody}

\begin{styleBody}
\textbf{Word stress: }Yes
\end{styleBody}

\begin{styleBody}
\textbf{Stress placement:} Unpredictable/Variable
\end{styleBody}

\begin{styleBody}
\textbf{Phonetic processes conditioned by stress:} Consonant Allophony in Stressed Syllables
\end{styleBody}

\begin{styleBody}
\textbf{Differences in phonological properties of stressed and unstressed syllables:} (None)
\end{styleBody}

\begin{styleBody}
\textbf{Phonetic correlates of stress: }Vowel duration (impressionistic), Pitch (impressionistic), Intensity (impressionistic)
\end{styleBody}

\begin{styleBody}
\textbf{Notes:} Stress, realized as a low pitch, is an “optional property” of the first syllable of a word. Otherwise, stress placement may vary by speaker and context (van Breugel 2008: 72-74; unclear from description whether word-level or higher-level stress is being described).
\end{styleBody}

\begin{styleBody}
\textbf{Vowel reduction processes}
\end{styleBody}

\begin{styleBody}
\textbf{aot-R1: }Vowels /i a o/ may be realized as [[26A?] [251?] [254?]] in closed syllables (van Breugel 2008: 53).
\end{styleBody}

\begin{styleBody}
\textbf{aot-R2: }The vowel /e/ has a free variant [[25B?]], especially word-finally (van Breugel 2008: 53).
\end{styleBody}

\begin{styleBody}
\textbf{aot-R3: }The vowel /u/ has free variant [[26F?]] (van Breugel 2008: 53).
\end{styleBody}

\begin{styleBody}
\textbf{aot-R4: }Vowels may be devoiced between a voiceless stop or affricate and another voiceless stop or affricate which is intervocalic (van Breugel 2008: 54).
\end{styleBody}

\begin{styleBody}
\textbf{aot-R5: }Vowels may be deleted between a /[255?][2B0?]/ or /t[361?][255?]/ and a voiceless stop or affricate which is intervocalic (van Breugel 2008: 54).
\end{styleBody}

\begin{styleBody}
\textbf{Consonant allophony processes}
\end{styleBody}

\begin{styleBody}
(none reported)
\end{styleBody}

\begin{styleBody}
\textbf{Morphology}
\end{styleBody}

\begin{styleBody}
(adequate texts unavailable)
\end{styleBody}

\clearpage\begin{styleBody}
\textbf{[apn] }\ \ \textbf{\textsc{Apinayé}}\textbf{\ \ }Nuclear-Macro-Je, \textit{Je} (Brazil)
\end{styleBody}

\begin{styleBody}
\textbf{References consulted: }Burgess \& Ham (1968), Ham (2009), Cunha de Oliveira (2005)
\end{styleBody}

\begin{styleBody}
\textbf{Sound inventory}
\end{styleBody}

\begin{styleBody}
\textbf{C phoneme inventory:} /p \textsuperscript{m}b t \textsuperscript{n}d k \textsuperscript{ŋ}[261?] [294?] t[361?][255?] \textsuperscript{[272?]}d[361?][292?] v s m n [272?] ŋ [27E?] j/
\end{styleBody}

\begin{styleBody}
\textbf{N consonant phonemes:} 17
\end{styleBody}

\begin{styleBody}
\textbf{Geminates:} N/A
\end{styleBody}

\begin{styleBody}
\textbf{Voicing contrasts:} Obstruents
\end{styleBody}

\begin{styleBody}
\textbf{Places:} Bilabial, Labiodental, Alveolar, Palato-alveolar, Velar, Glottal
\end{styleBody}

\begin{styleBody}
\textbf{Manners:} Stop, Affricate, Fricative, Nasal, Flap/Tap, Central approximant
\end{styleBody}

\begin{styleBody}
\textbf{N elaborations:} 4
\end{styleBody}

\begin{styleBody}
\textbf{Elaborations:} Voiced fricatives/affricates, Prenasalization, Labiodental, Palato-alveolar
\end{styleBody}

\begin{styleBody}
\textbf{V phoneme inventory:} /i e [25B?] a [28C?] [254?] [264?] o [26F?] u \~{i} \~{e} \~{a} [28C?] \~{o} [26F?] \~{u}/
\end{styleBody}

\begin{styleBody}
\textbf{N vowel qualities:} 10
\end{styleBody}

\begin{styleBody}
\textbf{Diphthongs or vowel sequences:} Diphthongs /ao u[259?]/
\end{styleBody}

\begin{styleBody}
\textbf{Contrastive length:} None
\end{styleBody}

\begin{styleBody}
\textbf{Contrastive nasalization:} Some
\end{styleBody}

\begin{styleBody}
\textbf{Other contrasts:} N/A
\end{styleBody}

\begin{styleBody}
\textbf{Notes:} Ham (2009) has /[291?]/ instead of /j/. Cunha de Oliveira has /w/ instead of /v/. Because of the allophonic distribution observed, /j/ and /v/ are selected for inclusion in the phoneme inventory above. Burgess \& Ham (1968) present a very different consonant inventory, considering prenasalized plosives to be predictable, and do not posit glides. /f/ occurs in loanwords. Discussion of typological unusualness of central vowel contrasts which are also attested in other Jê languages (Cunha de Oliveira 2005: 61-2). Diphthongs are not frequent and few instances have been attested.
\end{styleBody}

\begin{styleBody}
\textbf{Syllable structure}
\end{styleBody}

\begin{styleBody}
\textbf{Complexity Category:} Complex
\end{styleBody}

\begin{styleBody}
\textbf{Canonical syllable structure:} (C)(C)(C)V(C)\textbf{ }(Cunha de Oliveira 2005: 67-71)
\end{styleBody}

\begin{styleBody}
\textbf{Size of maximal onset:} 3
\end{styleBody}

\begin{styleBody}
\textbf{Size of maximal coda:} 1
\end{styleBody}

\begin{styleBody}
\textbf{Onset obligatory:} No
\end{styleBody}

\begin{styleBody}
\textbf{Coda obligatory:} No
\end{styleBody}

\begin{styleBody}
\textbf{Vocalic nucleus patterns:} Short vowels, Diphthongs
\end{styleBody}

\begin{styleBody}
\textbf{Syllabic consonant patterns:} N/A
\end{styleBody}

\begin{styleBody}
\textbf{Size of maximal word{}-marginal sequences with syllabic obstruents:} N/A
\end{styleBody}

\begin{styleBody}
\textbf{Predictability of syllabic consonants:} N/A 
\end{styleBody}

\begin{styleBody}
\textbf{Morphological constituency of maximal syllable margin:} Morpheme-internal (Onset)
\end{styleBody}

\begin{styleBody}
\textbf{Morphological pattern of syllabic consonants:} N/A
\end{styleBody}

\begin{styleBody}
\textbf{Onset restrictions:} Biconsonantal onsets limited to sequences of plosive or nasal + approximant or flap. In triconsonantal onsets, first consonant is a plosive, and others are limited to nasals, approximants, or flap. Each segment in a tautosyllabic sequence must be produced at a different place of articulation, and with a different manner of articulation.
\end{styleBody}

\begin{styleBody}
\textbf{Coda restrictions:} Limited to voiceless plosives or sonorants. Prenasalized stops and /ŋ/ do not occur.
\end{styleBody}

\begin{styleBody}
\textbf{Suprasegmentals}
\end{styleBody}

\begin{styleBody}
\textbf{Tone:} No
\end{styleBody}

\begin{styleBody}
\textbf{Word stress:} Yes
\end{styleBody}

\begin{styleBody}
\textbf{Stress placement:} Morphologically or Lexically Conditioned
\end{styleBody}

\begin{styleBody}
\textbf{Phonetic processes conditioned by stress:} Vowel Reduction, Consonant Allophony in Unstressed Syllables, Consonant Allophony in Stressed Syllables
\end{styleBody}

\begin{styleBody}
\textbf{Differences in phonological properties of stressed and unstressed syllables:} (None)
\end{styleBody}

\begin{styleBody}
\textbf{Phonetic correlates of stress: }Vowel duration (impressionistic), Intensity (impressionistic)
\end{styleBody}

\begin{styleBody}
\textbf{Vowel reduction processes}
\end{styleBody}

\begin{styleBody}
\textbf{apn-R1:} Vowels of unstressed syllables preceding the nucleus of a stress group are very short (Burgess and Ham 1968: 12).
\end{styleBody}

\begin{styleBody}
\textbf{apn-R2:} Vowels may be realized as devoiced utterance-finally, though nasalized vowels are devoiced less frequently than oral vowels (Ham 2009: 7).
\end{styleBody}

\begin{styleBody}
\textbf{Consonant allophony processes}
\end{styleBody}

\begin{styleBody}
\textbf{apn-C1:} Palatal glide /j/ is realized as alveolo-palatal [[291?]] in the simple onset of a stressed syllable (Cunha de Oliveira 2005: 58).
\end{styleBody}

\begin{styleBody}
\textbf{apn-C2:} /j/ is realized as [z] when occurring as the second consonant in a complex onset and directly preceding a vowel (Cunha de Oliveira 2005: 58).
\end{styleBody}

\begin{styleBody}
\textbf{apn-C3:} /j/ is realized as [d[361?][292?]] in syllable codas at word-final position, immediately followed by a vowel-initial morpheme (Cunha de Oliveira 2005: 58-9).
\end{styleBody}

\begin{styleBody}
\textbf{apn-C4:} /v/ is realized as [w] in syllable codas and in second position in complex syllable onsets (Cunha de Oliveira 2005: 59).
\end{styleBody}

\begin{styleBody}
\textbf{apn-C5: }Voiceless velar stop /k/ is palatalized preceding a front vowel (Cunha de Oliveira 2005: 50).
\end{styleBody}

\begin{styleBody}
\textbf{apn-C6: }Obstruents are optionally voiced in syllable codas (Cunha de Oliveira 2005: 44).
\end{styleBody}

\begin{styleBody}
\textbf{apn-C7: }Plosives are optionally voiced in the onset of unstressed syllables (Cunha de Oliveira 2005: 44).
\end{styleBody}

\begin{styleBody}
\textbf{apn-C8}: Voiceless bilabial stop /p/ is prenasalized when occurring word-finally after a nasalized vowel (Cunha de Oliveira 2005: 46).
\end{styleBody}

\begin{styleBody}
\textbf{apn-C9}: Voiceless alveolar stop /t/ is realized as a flap when occurring between two mid front vowels (Cunha de Oliveira 2005: 48).
\end{styleBody}

\begin{styleBody}
\textbf{Morphology}
\end{styleBody}

\begin{styleBody}
\textbf{Text:} “Sun and Moon” (first 8 pages, Cunha de Oliveira 2005: 304-311)
\end{styleBody}

\begin{styleBody}
\textbf{Synthetic index: }1.1 morphemes/word (445 morphemes, 409 words)
\end{styleBody}

\clearpage\begin{styleBody}
\textbf{[apu] }\ \ \textbf{\textsc{Apurinã}}\textbf{\ \ }Arawakan, \textit{Southern Maipuran} (Brazil)
\end{styleBody}

\begin{styleBody}
\textbf{References consulted: }Facundes (2000)
\end{styleBody}

\begin{styleBody}
\textbf{Sound inventory}
\end{styleBody}

\begin{styleBody}
\textbf{C phoneme inventory:} /p t k t[361?]s t[361?][283?] s [283?] h m n [272?] [27E?] j [270?]/
\end{styleBody}

\begin{styleBody}
\textbf{N consonant phonemes:} 14
\end{styleBody}

\begin{styleBody}
\textbf{Geminates:} N/A
\end{styleBody}

\begin{styleBody}
\textbf{Voicing contrasts:} None
\end{styleBody}

\begin{styleBody}
\textbf{Places:} Bilabial, Alveolar, Palato-alveolar, Palatal, Velar, Glottal
\end{styleBody}

\begin{styleBody}
\textbf{Manners:} Stop, Affricate, Fricative, Nasal, Flap/Tap, Central approximant
\end{styleBody}

\begin{styleBody}
\textbf{N elaborations:} 1
\end{styleBody}

\begin{styleBody}
\textbf{Elaborations:} Palato-alveolar
\end{styleBody}

\begin{styleBody}
\textbf{V phoneme inventory:} /i e [268?] a o i[2D0?] e[2D0?] [268?][2D0?] a[2D0?] o[2D0?] \~{i} \~{e} [268?] \~{a} \~{o} \~{i}[2D0?] \~{e}[2D0?] [268?][2D0?] \~{a}[2D0?] \~{o}[2D0?]/
\end{styleBody}

\begin{styleBody}
\textbf{N vowel qualities:} 5
\end{styleBody}

\begin{styleBody}
\textbf{Diphthongs or vowel sequences:} Vowel sequences /io ei ai ao oi/
\end{styleBody}

\begin{styleBody}
\textbf{Contrastive length:} All
\end{styleBody}

\begin{styleBody}
\textbf{Contrastive nasalization:} All
\end{styleBody}

\begin{styleBody}
\textbf{Other contrasts:} N/A
\end{styleBody}

\begin{styleBody}
\textbf{Notes:} /h/ occurs only word-initially. /o/ varies between [o] and [u]. 
\end{styleBody}

\begin{styleBody}
\textbf{Syllable structure}
\end{styleBody}

\begin{styleBody}
\textbf{Category:} Simple
\end{styleBody}

\begin{styleBody}
\textbf{Canonical syllable structure: }(C)V (Facundes 2000: 87-90)
\end{styleBody}

\begin{styleBody}
\textbf{Size of maximal onset:} 1
\end{styleBody}

\begin{styleBody}
\textbf{Size of maximal coda:} N/A
\end{styleBody}

\begin{styleBody}
\textbf{Onset obligatory:} No
\end{styleBody}

\begin{styleBody}
\textbf{Coda obligatory:} N/A
\end{styleBody}

\begin{styleBody}
\textbf{Vocalic nucleus patterns:} Short vowels, Long vowels, Vowel sequences
\end{styleBody}

\begin{styleBody}
\textbf{Syllabic consonant patterns:} N/A
\end{styleBody}

\begin{styleBody}
\textbf{Size of maximal word{}-marginal sequences with syllabic obstruents:} \textbf{\ }N/A
\end{styleBody}

\begin{styleBody}
\textbf{Predictability of syllabic consonants:} N/A
\end{styleBody}

\begin{styleBody}
\textbf{Morphological constituency of maximal syllable margin:} N/A
\end{styleBody}

\begin{styleBody}
\textbf{Morphological pattern of syllabic consonants:} N/A
\end{styleBody}

\begin{styleBody}
\textbf{Onset restrictions:} All consonants occur.
\end{styleBody}

\begin{styleBody}
\textbf{Notes:} It is possible to analyze diphthongs as coda glides.
\end{styleBody}

\begin{styleBody}
\textbf{Suprasegmentals}
\end{styleBody}

\begin{styleBody}
\textbf{Tone:} No
\end{styleBody}

\begin{styleBody}
\textbf{Word stress:} Yes
\end{styleBody}

\begin{styleBody}
\textbf{Stress placement:} Weight-Sensitive
\end{styleBody}

\begin{styleBody}
\textbf{Phonetic processes conditioned by stress:} Vowel Reduction, Consonant Allophony in Unstressed Syllables
\end{styleBody}

\begin{styleBody}
\textbf{Differences in phonological properties of stressed and unstressed syllables:} (None)
\end{styleBody}

\begin{styleBody}
\textbf{Phonetic correlates of stress: }Pitch (impressionistic), Intensity (impressionistic)
\end{styleBody}

\begin{styleBody}
\textbf{Vowel reduction processes}
\end{styleBody}

\begin{styleBody}
\textbf{apu-R1:} Vowels become devoiced in unstressed word-final position, especially in fast speech (Facundes 2000: 60-1). This process also causes aspiration of a preceding stop.
\end{styleBody}

\begin{styleBody}
\textbf{Consonant allophony processes}
\end{styleBody}

\begin{styleBody}
\textbf{apu-C1}: The voiceless velar stop is palatalized preceding mid front vowels (Facundes 2000: 76).
\end{styleBody}

\begin{styleBody}
\textbf{apu-C2: }Plosives are voiced following a nasalized vowel (Facundes 2000: 73).
\end{styleBody}

\begin{styleBody}
\textbf{Morphology}
\end{styleBody}

\begin{styleBody}
\textbf{Text:} “Apurina text sample” (Facundes 2000: 625-642)
\end{styleBody}

\begin{styleBody}
\textbf{Synthetic index: }2.1 morphemes/word (714 morphemes, 347 words)
\end{styleBody}

\clearpage\begin{styleBody}
\textbf{[ayz] }\ \ \textbf{\textsc{Maybrat}}\textbf{\ \ }Maybrat-Karon (Indonesia)
\end{styleBody}

\begin{styleBody}
\textbf{References consulted: }Dol (2007)
\end{styleBody}

\begin{styleBody}
\textbf{Sound inventory}
\end{styleBody}

\begin{styleBody}
\textbf{C phoneme inventory:} /p t k f s x m n r w j/
\end{styleBody}

\begin{styleBody}
\textbf{N consonant phonemes:} 11
\end{styleBody}

\begin{styleBody}
\textbf{Geminates:} N/A
\end{styleBody}

\begin{styleBody}
\textbf{Voicing contrasts:} None
\end{styleBody}

\begin{styleBody}
\textbf{Places:} Bilabial, Labiodental, Alveolar, Velar
\end{styleBody}

\begin{styleBody}
\textbf{Manners:} Stop, Fricative, Nasal, Trill, Central approximant
\end{styleBody}

\begin{styleBody}
\textbf{N elaborations:} 1
\end{styleBody}

\begin{styleBody}
\textbf{Elaborations:} Labiodental
\end{styleBody}

\begin{styleBody}
\textbf{V phoneme inventory:} /i e a [259?] o u/
\end{styleBody}

\begin{styleBody}
\textbf{N vowel qualities:} 6
\end{styleBody}

\begin{styleBody}
\textbf{Diphthongs or vowel sequences:} Vowel sequences /ii ie ia io ea eo ai ae ao au oi oa oo ua uo uu/
\end{styleBody}

\begin{styleBody}
\textbf{Contrastive length:} None
\end{styleBody}

\begin{styleBody}
\textbf{Contrastive nasalization:} None
\end{styleBody}

\begin{styleBody}
\textbf{Other contrasts:} N/A
\end{styleBody}

\begin{styleBody}
\textbf{Notes:} /o/ is described as lower than /e/. /[259?]/ occurs in some words as ‘optional’ phoneme, e.g. /te/\~{}/[259?]te/ ‘below’. It can’t take stress but is counted for syllabification, so I include it here as a phoneme (Dol 2007: 15-18).
\end{styleBody}

\begin{styleBody}
\textbf{Syllable structure}
\end{styleBody}

\begin{styleBody}
\textbf{Complexity Category:} Moderately Complex
\end{styleBody}

\begin{styleBody}
\textbf{Canonical syllable structure:} (C)V(C) (Dol 2007: 34-8)
\end{styleBody}

\begin{styleBody}
\textbf{Size of maximal onset:} 1
\end{styleBody}

\begin{styleBody}
\textbf{Size of maximal coda:} 1
\end{styleBody}

\begin{styleBody}
\textbf{Onset obligatory:} No
\end{styleBody}

\begin{styleBody}
\textbf{Coda obligatory:} No
\end{styleBody}

\begin{styleBody}
\textbf{Vocalic nucleus patterns:} Short vowels, Vowel sequences
\end{styleBody}

\begin{styleBody}
\textbf{Syllabic consonant patterns:} N/A
\end{styleBody}

\begin{styleBody}
\textbf{Size of maximal word{}-marginal sequences with syllabic obstruents:} N/A
\end{styleBody}

\begin{styleBody}
\textbf{Predictability of syllabic consonants:} N/A
\end{styleBody}

\begin{styleBody}
\textbf{Morphological constituency of maximal syllable margin:} N/A
\end{styleBody}

\begin{styleBody}
\textbf{Morphological pattern of syllabic consonants:} N/A
\end{styleBody}

\begin{styleBody}
\textbf{Onset restrictions: }All consonants occur.
\end{styleBody}

\begin{styleBody}
\textbf{Coda restrictions:} All consonants except /p w j/ occur.
\end{styleBody}

\begin{styleBody}
\textbf{Notes:} Initial consonant sequences are posited in Dol’s analysis as a result of morphology, but these are invariably broken up by an epenthetic schwa, such that phonetic onset clusters never occur. On the basis of both perceptual and acoustic evidence, Dol takes \textit{phonetic} structure of syllable to be canonical (2007: 35-7).
\end{styleBody}

\begin{styleBody}
\textbf{Suprasegmentals}
\end{styleBody}

\begin{styleBody}
\textbf{Tone: }No
\end{styleBody}

\begin{styleBody}
\textbf{Word stress: }Yes
\end{styleBody}

\begin{styleBody}
\textbf{Stress placement:} Fixed
\end{styleBody}

\begin{styleBody}
\textbf{Phonetic processes conditioned by stress:} (None)
\end{styleBody}

\begin{styleBody}
\textbf{Differences in phonological properties of stressed and unstressed syllables:} (None)
\end{styleBody}

\begin{styleBody}
\textbf{Phonetic correlates of stress: }Vowel duration (impressionistic)
\end{styleBody}

\begin{styleBody}
\textbf{Notes: }Duration a correlate in monosyllabic words that receive stress in connected speech.
\end{styleBody}

\begin{styleBody}
\textbf{Vowel reduction processes}
\end{styleBody}

\begin{styleBody}
(none reported)
\end{styleBody}

\begin{styleBody}
\textbf{Consonant allophony processes}
\end{styleBody}

\begin{styleBody}
\textbf{ayz-C1:} Plosives and /x/ vary with voiced variants freely (Dol 2007: 21-2).
\end{styleBody}

\begin{styleBody}
\textbf{ayz-C2}: Trill varies freely with flap in non-word-initial environments (Dol 2007: 24).
\end{styleBody}

\begin{styleBody}
\textbf{Morphology}
\end{styleBody}

\begin{styleBody}
\textbf{Text:} “Siwa and his brother Mafif” (Dol 2007: 284-291)
\end{styleBody}

\begin{styleBody}
\textbf{Synthetic index: }1.5 morphemes/word (689 morphemes, 453 words)
\end{styleBody}

\clearpage\begin{styleBody}
\textbf{[bak] }\ \ \textbf{\textsc{Bashkir}}\textbf{\ \ }Turkic, \textit{Common Turkic} (Russia)
\end{styleBody}

\begin{styleBody}
\textbf{References consulted: }Berkson et al. (2016), Matthew Carter (p.c.), Poppe (1964)
\end{styleBody}

\begin{styleBody}
\textbf{Sound inventory}
\end{styleBody}

\begin{styleBody}
\textbf{C phoneme inventory:} /p b t d k [261?] q $\theta $ ð s [283?] $\chi $ [281?] h m n ŋ l [27E?] w j/
\end{styleBody}

\begin{styleBody}
\textbf{N consonant phonemes:} 21
\end{styleBody}

\begin{styleBody}
\textbf{Geminates:} N/A
\end{styleBody}

\begin{styleBody}
\textbf{Voicing contrasts:} Obstruents
\end{styleBody}

\begin{styleBody}
\textbf{Places:} Bilabial, Dental, Alveolar, Palato-Alveolar, Velar, Uvular, Glottal
\end{styleBody}

\begin{styleBody}
\textbf{Manners:} Stop, Fricative, Nasal, Flap/Tap, Central approximant, Lateral approximant
\end{styleBody}

\begin{styleBody}
\textbf{N elaborations:} 2
\end{styleBody}

\begin{styleBody}
\textbf{Elaborations:} Voiced fricatives/affricates, Palato-alveolar, Uvular
\end{styleBody}

\begin{styleBody}
\textbf{V phoneme inventory:} /i y [26A?] [28F?] æ [251?] [26F?] [28A?] u/
\end{styleBody}

\begin{styleBody}
\textbf{N vowel qualities:} 9
\end{styleBody}

\begin{styleBody}
\textbf{Diphthongs or vowel sequences:} None
\end{styleBody}

\begin{styleBody}
\textbf{Contrastive length:} None
\end{styleBody}

\begin{styleBody}
\textbf{Contrastive nasalization:} None
\end{styleBody}

\begin{styleBody}
\textbf{Other contrasts:} N/A
\end{styleBody}

\begin{styleBody}
\textbf{Notes:} /[294?] t[361?]s t[361?][283?] f v z [292?]/ occur only in loans. Vowels are from Carter \& Robbins (2016) acoustic study. /[26F?]/ is the ‘canonical’ phoneme but quality is closer to [[28C?]]. /e [254?]/ occur only in loanwords from Russian.
\end{styleBody}

\begin{styleBody}
\textbf{Syllable structure}
\end{styleBody}

\begin{styleBody}
\textbf{Complexity Category:} Complex
\end{styleBody}

\begin{styleBody}
\textbf{Canonical syllable structure:} (C)V(C)(C)\textbf{ }(Poppe 1964: 12-18)
\end{styleBody}

\begin{styleBody}
\textbf{Size of maximal onset:} 1
\end{styleBody}

\begin{styleBody}
\textbf{Size of maximal coda:} 2
\end{styleBody}

\begin{styleBody}
\textbf{Onset obligatory:} No
\end{styleBody}

\begin{styleBody}
\textbf{Coda obligatory:} No
\end{styleBody}

\begin{styleBody}
\textbf{Vocalic nucleus patterns:} Short vowels, Long vowels, Diphthongs, Vowel sequences
\end{styleBody}

\begin{styleBody}
\textbf{Syllabic consonant patterns:} N/A
\end{styleBody}

\begin{styleBody}
\textbf{Size of maximal word{}-marginal sequences with syllabic obstruents:} N/A
\end{styleBody}

\begin{styleBody}
\textbf{Predictability of syllabic consonants:} N/A
\end{styleBody}

\begin{styleBody}
\textbf{Morphological constituency of maximal syllable margin:} Morpheme-internal (Coda)
\end{styleBody}

\begin{styleBody}
\textbf{Morphological pattern of syllabic consonants:} N/A
\end{styleBody}

\begin{styleBody}
\textbf{Onset restrictions:} All consonants except /$\theta $ ŋ/.
\end{styleBody}

\begin{styleBody}
\textbf{Coda restrictions:} All consonants except /b d [261?] [281?] h/ may occur as simple codas. Biconsonantal codas apparently have /r l/ as C\textsubscript{1} and a stop as C\textsubscript{2} (patterns inferred from examples).
\end{styleBody}

\begin{styleBody}
\textbf{Suprasegmentals}
\end{styleBody}

\begin{styleBody}
\textbf{Tone:} No
\end{styleBody}

\begin{styleBody}
\textbf{Word stress:} Yes
\end{styleBody}

\begin{styleBody}
\textbf{Stress placement:} Fixed
\end{styleBody}

\begin{styleBody}
\textbf{Phonetic processes conditioned by stress:} Vowel Reduction
\end{styleBody}

\begin{styleBody}
\textbf{Differences in phonological properties of stressed and unstressed syllables:} (None)
\end{styleBody}

\begin{styleBody}
\textbf{Phonetic correlates of stress: }Intensity (impressionistic)
\end{styleBody}

\begin{styleBody}
\textbf{Vowel reduction processes}
\end{styleBody}

\begin{styleBody}
\textbf{bak-R1:} High and mid vowels are lowered and centralized in pre-stressed position (Carter and Robbins 2015).
\end{styleBody}

\begin{styleBody}
\textbf{Consonant allophony processes}
\end{styleBody}

\begin{styleBody}
\textbf{bak-C1:} Velar fricative /x/ and nasal [ŋ] may be realized as uvulars adjacent to back vowels (Poppe 1964: 11).
\end{styleBody}

\begin{styleBody}
\textbf{bak-C2: }Lateral approximant [l] is velarized adjacent to back vowels (Poppe 1964: 10).
\end{styleBody}

\begin{styleBody}
\textbf{bak-C3: }Voiced bilabial stop [b] may be realized as a fricative in fast speech (Poppe 1964: 8).
\end{styleBody}

\begin{styleBody}
\textbf{bak-C4: }\ A labiovelar approximant is realized as a vocalic offglide syllable-finally and word-finally (Poppe 1964: 9).
\end{styleBody}

\begin{styleBody}
\textbf{bak-C5: }Labial semivowel /w/ is realized as [u] in syllable- and word-final position (Poppe 1964: 9)
\end{styleBody}

\begin{styleBody}
\textbf{Morphology}
\end{styleBody}

\begin{styleBody}
(adequate texts unavailable)
\end{styleBody}

\clearpage\begin{styleBody}
\textbf{[bbo] }\ \ \textbf{\textsc{Southern Bobo Madaré}}\textbf{\ \ }Mande, \textit{Western Mande} (Burkina Faso)
\end{styleBody}

\begin{styleBody}
\textbf{References consulted: }Morse (1976), Sanou (1978)
\end{styleBody}

\begin{styleBody}
\textbf{Sound inventory}
\end{styleBody}

\begin{styleBody}
\textbf{C phoneme inventory:} /k[361?]p [261?][361?]b p b t[32A?] d[32A?] k [261?] f v s[32A?] z[32A?] h m n[32A?] [272?] ŋ l[32A?] [27E?] w j/
\end{styleBody}

\begin{styleBody}
\textbf{N consonant phonemes:} 21
\end{styleBody}

\begin{styleBody}
\textbf{Geminates:} N/A
\end{styleBody}

\begin{styleBody}
\textbf{Voicing contrasts:} Obstruents
\end{styleBody}

\begin{styleBody}
\textbf{Places:} Labial-velar, Bilabial, Labiodental, Dental, Palatal, Velar, Glottal
\end{styleBody}

\begin{styleBody}
\textbf{Manners:} Stop, Fricative, Nasal, Flap/Tap, Lateral approximant, Central Approximant
\end{styleBody}

\begin{styleBody}
\textbf{N elaborations:} 2
\end{styleBody}

\begin{styleBody}
\textbf{Elaborations:} Voiced fricatives/affricates, Labiodental
\end{styleBody}

\begin{styleBody}
\textbf{V phoneme inventory:} /i e [25B?] [259?] a [254?] o u \~{i} [25B?] \~{a} [254?] \~{u}/
\end{styleBody}

\begin{styleBody}
\textbf{N vowel qualities:} 8
\end{styleBody}

\begin{styleBody}
\textbf{Diphthongs or vowel sequences:} None
\end{styleBody}

\begin{styleBody}
\textbf{Contrastive length:} None
\end{styleBody}

\begin{styleBody}
\textbf{Contrastive nasalization:} Some
\end{styleBody}

\begin{styleBody}
\textbf{Other contrasts: }N/A
\end{styleBody}

\begin{styleBody}
\textbf{Notes: }Phonetically long vowels analyzed as sequences (Morse 1976: 100-105). /[259?]/ is very reduced: in normal conversation, it sounds more like open transition than a vowel, but it does bear tone. Morse analyzes it as phoneme because in most cases it is unclear which vowel might have been reduced to produce this sound (Morse 1976: 42-5).
\end{styleBody}

\begin{styleBody}
\textbf{Syllable structure}
\end{styleBody}

\begin{styleBody}
\textbf{Complexity Category:} Simple
\end{styleBody}

\begin{styleBody}
\textbf{Canonical syllable structure:} (C)V\textbf{ }(Morse 1976: 112-114)
\end{styleBody}

\begin{styleBody}
\textbf{Size of maximal onset:} 1
\end{styleBody}

\begin{styleBody}
\textbf{Size of maximal coda:} N/A
\end{styleBody}

\begin{styleBody}
\textbf{Onset obligatory:} No
\end{styleBody}

\begin{styleBody}
\textbf{Coda obligatory:} N/A
\end{styleBody}

\begin{styleBody}
\textbf{Vocalic nucleus patterns:} Short vowels
\end{styleBody}

\begin{styleBody}
\textbf{Syllabic consonant patterns:} N/A
\end{styleBody}

\begin{styleBody}
\textbf{Size of maximal word{}-marginal sequences with syllabic obstruents:} N/A
\end{styleBody}

\begin{styleBody}
\textbf{Predictability of syllabic consonants:} N/A
\end{styleBody}

\begin{styleBody}
\textbf{Morphological constituency of maximal syllable margin:} N/A
\end{styleBody}

\begin{styleBody}
\textbf{Morphological pattern of syllabic consonants:} N/A
\end{styleBody}

\begin{styleBody}
\textbf{Onset restrictions:} All consonants occur.
\end{styleBody}

\begin{styleBody}
\textbf{Notes: }Occasionally CCV syllables occur in loanwords. The only cases of closed syllables are in a few French loans in well-educated speech (Morse 1976: 113).
\end{styleBody}

\begin{styleBody}
\textbf{Suprasegmentals}
\end{styleBody}

\begin{styleBody}
\textbf{Tone:} Yes
\end{styleBody}

\begin{styleBody}
\textbf{Word stress:} Yes
\end{styleBody}

\begin{styleBody}
\textbf{Stress placement:} Other (tone)
\end{styleBody}

\begin{styleBody}
\textbf{Phonetic processes conditioned by stress:} Consonant Allophony in Unstressed Syllables
\end{styleBody}

\begin{styleBody}
\textbf{Differences in phonological properties of stressed and unstressed syllables:} (None)
\end{styleBody}

\begin{styleBody}
\textbf{Phonetic correlates of stress: Not described}
\end{styleBody}

\begin{styleBody}
\textbf{Vowel reduction processes}
\end{styleBody}

\begin{styleBody}
\textbf{bbo-R1: }High front vowels /i i\~{ }/ are partially devoiced following /s/ (Morse 1976: 28-9).
\end{styleBody}

\begin{styleBody}
\textbf{Consonant allophony processes}
\end{styleBody}

\begin{styleBody}
\textbf{bbo-C1: }Bilabial stops are affricated preceding a high front vowel (Morse 1976: 20).
\end{styleBody}

\begin{styleBody}
\textbf{bbo-C2: }A flap [[27E?]] is realized with palato-alveolar fricative release [[27E?]\textsuperscript{[292?]}] preceding a high front vowel (Morse 1976: 25).
\end{styleBody}

\begin{styleBody}
\textbf{bbo-C3: }Voiced stops /b/ and /[261?]/ are realized as fricatives in intervocalic environments (Morse 1976: 22).
\end{styleBody}

\begin{styleBody}
\textbf{bbo-C4: }Alveolar and velar stops and fricatives, and /n/, are fronted preceding high and/or front vowels (Morse 1976: 20-23).
\end{styleBody}

\begin{styleBody}
\textbf{bbo-C5: }Stops, fricatives, nasals, and laterals are labialized preceding back vowels (Morse 1976: 20).
\end{styleBody}

\begin{styleBody}
\textbf{Morphology}
\end{styleBody}

\begin{styleBody}
(adequate texts unavailable)
\end{styleBody}

\clearpage\begin{styleBody}
\textbf{[bcj] }\ \ \textbf{\textsc{Bardi\ \ }}\textbf{\ \ }Nyulnyulan, \textit{Western Nyulnyulan} (Australia)
\end{styleBody}

\begin{styleBody}
\textbf{References consulted: }Bowern (2012)
\end{styleBody}

\begin{styleBody}
\textbf{Sound inventory}
\end{styleBody}

\begin{styleBody}
\textbf{C phoneme inventory:} /p t [288?] c [261?] m n [273?] [272?] ŋ l [26D?] [28E?] r [27B?] j w/
\end{styleBody}

\begin{styleBody}
\textbf{N consonant phonemes:} 17
\end{styleBody}

\begin{styleBody}
\textbf{Geminates:} N/A
\end{styleBody}

\begin{styleBody}
\textbf{Voicing contrasts:} None
\end{styleBody}

\begin{styleBody}
\textbf{Places:} Bilabial, Alveolar, Retroflex, Palatal, Velar
\end{styleBody}

\begin{styleBody}
\textbf{Manners:} Stop, Nasal, Trill, Central approximant, Lateral approximant
\end{styleBody}

\begin{styleBody}
\textbf{N elaborations:} 1
\end{styleBody}

\begin{styleBody}
\textbf{Elaborations:} Retroflex
\end{styleBody}

\begin{styleBody}
\textbf{V phoneme inventory:} /i a [254?] u i[2D0?] a[2D0?] u[2D0?]/
\end{styleBody}

\begin{styleBody}
\textbf{N vowel qualities:} 4
\end{styleBody}

\begin{styleBody}
\textbf{Diphthongs or vowel sequences:} None
\end{styleBody}

\begin{styleBody}
\textbf{Contrastive length:} Some
\end{styleBody}

\begin{styleBody}
\textbf{Contrastive nasalization:} None
\end{styleBody}

\begin{styleBody}
\textbf{Other contrasts:} N/A
\end{styleBody}

\begin{styleBody}
\textbf{Notes:} /e/ is marginally phonemic.
\end{styleBody}

\begin{styleBody}
\textbf{Syllable structure}
\end{styleBody}

\begin{styleBody}
\textbf{Complexity Category:} Complex
\end{styleBody}

\begin{styleBody}
\textbf{Canonical syllable structure:} (C)V(C)(C) (Bowern 2012: 94-104)
\end{styleBody}

\begin{styleBody}
\textbf{Size of maximal onset:} 1
\end{styleBody}

\begin{styleBody}
\textbf{Size of maximal coda:} 2
\end{styleBody}

\begin{styleBody}
\textbf{Onset obligatory:} No
\end{styleBody}

\begin{styleBody}
\textbf{Coda obligatory:} No
\end{styleBody}

\begin{styleBody}
\textbf{Vocalic nucleus patterns:} Short vowels, Long vowels
\end{styleBody}

\begin{styleBody}
\textbf{Syllabic consonant patterns:} N/A
\end{styleBody}

\begin{styleBody}
\textbf{Size of maximal word{}-marginal sequences with syllabic obstruents:} N/A
\end{styleBody}

\begin{styleBody}
\textbf{Predictability of syllabic consonants:} N/A
\end{styleBody}

\begin{styleBody}
\textbf{Morphological constituency of maximal syllable margin:} Both patterns (Coda)
\end{styleBody}

\begin{styleBody}
\textbf{Morphological pattern of syllabic consonants:} N/A
\end{styleBody}

\begin{styleBody}
\textbf{Onset restrictions:} Apparently none (though word-initially, /r/ and /[28E?]/ do not occur).
\end{styleBody}

\begin{styleBody}
\textbf{Coda restrictions:} Chart in Bowern (2012: 102) indicates that all consonants except /b/ may occur as a simple coda. Biconsonantal codas consist of /l/, /[27B?]/ or /r/ followed by a nasal which is homorganic with the following stop.
\end{styleBody}

\begin{styleBody}
\textbf{Suprasegmentals}
\end{styleBody}

\begin{styleBody}
\textbf{Tone:} No
\end{styleBody}

\begin{styleBody}
\textbf{Word stress:} Yes
\end{styleBody}

\begin{styleBody}
\textbf{Stress placement:} Fixed
\end{styleBody}

\begin{styleBody}
\textbf{Phonetic processes conditioned by stress:} Vowel Reduction, Consonant Allophony in Unstressed Syllables
\end{styleBody}

\begin{styleBody}
\textbf{Differences in phonological properties of stressed and unstressed syllables:} Vowel Length Contrasts (see notes)
\end{styleBody}

\begin{styleBody}
\textbf{Phonetic correlates of stress: }Vowel duration (instrumental), Pitch (instrumental), Intensity (instrumental)
\end{styleBody}

\begin{styleBody}
\textbf{Notes: }Duration is a correlate of stress for short vowels, and is slight. There are cases where post-tonic vowels are not neutralized, and examples where intensity peak does not coincide with the pitch peak. Stress is always word-initial; because some consonant contrasts do not occur word-initially, stressed syllables are associated with fewer consonant contrasts: trill /r/ and palatal lateral /[28E?]/ do not occur, while apico-dental and apico-alveolar (retroflex) consonants are neutralized in favor of retroflex series. Long vowels are rarely attested in unstressed positions. I take the consonant pattern to be reflective of general tendencies towards word-initial neutralization in Australian languages, but the vowel length pattern to be truly an effect of stress, since it is explicitly described in those terms.
\end{styleBody}

\begin{styleBody}
\textbf{Vowel reduction processes}
\end{styleBody}

\begin{styleBody}
\textbf{bcj-R1:} Short vowels are reduced in quality in unstressed syllables (centralized and lowered or raised to [[259?]] or [[25C?]], Bowern 2012: 88).
\end{styleBody}

\begin{styleBody}
\textbf{bcj-R2:} For some speakers, the vowel in an open medial syllable of a trisyllabic word is deleted, especially when it is /i/ and the third syllable is heavy (Bowern 2012: 91).
\end{styleBody}

\begin{styleBody}
\textbf{bcj-R3:} High front vowels /i u/ are rhoticized and reduced between a stop and a glide (Bowern 2012: 91).
\end{styleBody}

\begin{styleBody}
\textbf{bcj-R4}: Word-final vowels are often partially or fully devoiced (Bowern 2012: 92; in some dialects these vowels are omitted entirely).
\end{styleBody}

\begin{styleBody}
\textbf{bcj-R5:} A vowel in a syllable following a stressed syllable is characterized by both shortening and centralization, particularly when that syllable is open (Bowern 2012: 111; some sources consistently note this as vowel loss).
\end{styleBody}

\begin{styleBody}
\textbf{Consonant allophony processes}
\end{styleBody}

\begin{styleBody}
\textbf{bcj-C1: }Glide /j/ may be realized as [[25F?]] following a trill and preceding a vowel, while also following a stressed syllable (Bowern 2012: 80-1).
\end{styleBody}

\begin{styleBody}
\textbf{bcj-C2: }\ Stops are voiced intervocalically (Bowern 2012: 76).
\end{styleBody}

\begin{styleBody}
\textbf{bcj-C3: }A trill is realized as a flap intervocalically (Bowern 2012: 81).
\end{styleBody}

\begin{styleBody}
\textbf{bcj-C4: }Stops are realized with weak closure intervocalically (Bowern 2012: 78).
\end{styleBody}

\begin{styleBody}
\textbf{Morphology}
\end{styleBody}

\begin{styleBody}
\textbf{Text:} “Goolamana,” “Story about Mirrdiidi people” (Bowern 2012: 704-710)
\end{styleBody}

\begin{styleBody}
\textbf{Synthetic index: }2.0 morphemes/word (307 morphemes, 151 words)
\end{styleBody}

\clearpage\begin{styleBody}
\textbf{[bcq] }\ \ \textbf{\textsc{Bench\ \ }}\textbf{\ \ }Ta-Ne-Omotic (Ethiopia)
\end{styleBody}

\begin{styleBody}
\textbf{References consulted: }Rapold (2006)
\end{styleBody}

\begin{styleBody}
\textbf{Sound inventory}
\end{styleBody}

\begin{styleBody}
\textbf{C phoneme inventory:} /p b t d k [261?] [294?] p’ t’ k’ t[361?]s t[361?][283?] c[361?][255?] t[361?]s’ t[361?][283?]’ c[361?][255?]’ s z [283?] [292?] [255?] [291?] h m n l [27E?] j/
\end{styleBody}

\begin{styleBody}
\textbf{N consonant phonemes:} 28
\end{styleBody}

\begin{styleBody}
\textbf{Geminates:} N/A
\end{styleBody}

\begin{styleBody}
\textbf{Voicing contrasts: }Obstruents
\end{styleBody}

\begin{styleBody}
\textbf{Places: }Bilabial, Alveolar, Palato-alveolar, Alveolo-palatal, Velar, Glottal
\end{styleBody}

\begin{styleBody}
\textbf{Manners:} Stop, Affricate, Fricative, Nasal, Flap/tap, Central approximant, Lateral approximant
\end{styleBody}

\begin{styleBody}
\textbf{N elaborations:} 3
\end{styleBody}

\begin{styleBody}
\textbf{Elaborations:} Voiced fricatives/affricates, Ejective, Palato-alveolar
\end{styleBody}

\begin{styleBody}
\textbf{V phoneme inventory:} /i e a o u/
\end{styleBody}

\begin{styleBody}
\textbf{N vowel qualities:} 5
\end{styleBody}

\begin{styleBody}
\textbf{Diphthongs or vowel sequences:} None
\end{styleBody}

\begin{styleBody}
\textbf{Contrastive length:} None
\end{styleBody}

\begin{styleBody}
\textbf{Contrastive nasalization:} None
\end{styleBody}

\begin{styleBody}
\textbf{Other contrasts:} N/A
\end{styleBody}

\begin{styleBody}
\textbf{Notes:} Rapold notes that diphthongs are a possible analysis of certain glide-vowel patterns (2006: 100-102).
\end{styleBody}

\begin{styleBody}
\textbf{Syllable structure}
\end{styleBody}

\begin{styleBody}
\textbf{Complexity Category:} Highly Complex
\end{styleBody}

\begin{styleBody}
\textbf{Canonical syllable structure:} (C)CV(C)(C)(C)\textbf{ }(Rapold 2006: 91-112)
\end{styleBody}

\begin{styleBody}
\textbf{Size of maximal onset:} 2
\end{styleBody}

\begin{styleBody}
\textbf{Size of maximal coda: }3
\end{styleBody}

\begin{styleBody}
\textbf{Onset obligatory:} Yes
\end{styleBody}

\begin{styleBody}
\textbf{Coda obligatory:} No
\end{styleBody}

\begin{styleBody}
\textbf{Vocalic nucleus patterns: }Short vowels
\end{styleBody}

\begin{styleBody}
\textbf{Syllabic consonant patterns:} Nasal
\end{styleBody}

\begin{styleBody}
\textbf{Size of maximal word{}-marginal sequences with syllabic obstruents:} N/A
\end{styleBody}

\begin{styleBody}
\textbf{Predictability of syllabic consonants:} Phonemic
\end{styleBody}

\begin{styleBody}
\textbf{Morphological constituency of maximal syllable margin:} Morpheme-internal (Onset), Morphologically Complex (Coda)
\end{styleBody}

\begin{styleBody}
\textbf{Morphological pattern of syllabic consonants:} Both (Nasal)
\end{styleBody}

\begin{styleBody}
\textbf{Onset restrictions:} All but /l [27E?] t[361?]s t[361?][283?] c[361?]ç/ may occur as simple onsets word-initially. For CC onsets, C\textsubscript{1} may be any consonant except for /p’ h/, palato-alveolar and alveopalatal fricatives, or ejective affricates. C\textsubscript{2} is /j/.
\end{styleBody}

\begin{styleBody}
\textbf{Coda restrictions:} Almost any C may occur as simple coda. In CC codas, first C may be labial stop, fricative, liquid, nasal, or /j/ and second C may be buccal stop, ejective affricate or fricative (but fricatives do not form clusters with affricates or other fricatives). CCC codas highly restricted: /jnt/ or /p m [27E?] j/+/s/+/t/. /pst/ only HC pattern.
\end{styleBody}

\begin{styleBody}
\textbf{Notes:} Resyllabification of CCC codas is common in fluent speech when followed by vowel, but these codas do sometimes occur in speech. Note that C/j/ onsets are limited in their distribution, occurring only before /a/. Rapold discusses other possible phonological interpretations of this pattern (2006: 101-3), including palatalized C and C+falling diphthong analyses.
\end{styleBody}

\begin{styleBody}
\textbf{Suprasegmentals}
\end{styleBody}

\begin{styleBody}
\textbf{Tone:} Yes
\end{styleBody}

\begin{styleBody}
\textbf{Word stress: }Not reported
\end{styleBody}

\begin{styleBody}
\textbf{Vowel reduction processes}
\end{styleBody}

\begin{styleBody}
(none reported)
\end{styleBody}

\begin{styleBody}
\textbf{Consonant allophony processes}
\end{styleBody}

\begin{styleBody}
\textbf{bcq-C1: }An alveolar sibilant is realized as palato-alveolar following a preceding palato-alveolar sibilant, with intervening phonological material (long distance sibilant harmony) (Rapold 2006: 67).
\end{styleBody}

\begin{styleBody}
\textbf{bcq-C2: }A homorganic stop is inserted between a nasal and fricative, producing a nasal-affricate sequence (Rapold 2006: 69).
\end{styleBody}

\begin{styleBody}
\textbf{bcq-C3: }A voiceless bilabial stop is optionally realized as a bilabial or labiodental fricative, possibly in all contexts (Rapold 2006: 73).
\end{styleBody}

\begin{styleBody}
\textbf{bcq-C4: }Alveolar stops are realized as palato-alveolar preceding a palatal-alveolar sibilant (Rapold 2006: 74).
\end{styleBody}

\begin{styleBody}
\textbf{bcq-C5: }A syllabic nasal is realized as a nasalized high central vowel following a palato-alveolar or alveolo-palatal consonant (Rapold 2006: 76).
\end{styleBody}

\begin{styleBody}
\textbf{Morphology}
\end{styleBody}

\begin{styleBody}
\textbf{Text:} “B\=obt-\=agà b\=et — The skins of the baboons” (Rapold 2006: 594-599)
\end{styleBody}

\begin{styleBody}
\textbf{Synthetic index: }2.28 morphemes/word (594 morphemes, 261 words)
\end{styleBody}

\clearpage\begin{styleBody}
\textbf{[bsk] }\ \ \textbf{\textsc{Burushaski}}\textbf{\ \ }isolate (Pakistan)
\end{styleBody}

\begin{styleBody}
\textbf{References consulted: }Anderson (1997), Yoshioka (2012)
\end{styleBody}

\begin{styleBody}
\textbf{Sound inventory}
\end{styleBody}

\begin{styleBody}
\textbf{C phoneme inventory:} /p b t[32A?] d[32A?] [288?] [256?] k [261?] q p[2B0?] t[32A?][2B0?] [288?][2B0?] k[2B0?] q[2B0?] t[361?]s t[361?][255?] d[361?][291?] [288?][361?][282?] [256?][361?][290?] t[361?]s[2B0?] t[361?][255?][2B0?] [288?][361?][282?][2B0?] s z [255?] [282?] [263?] h m n ŋ [27E?] l w [270?][31F?] j/
\end{styleBody}

\begin{styleBody}
\textbf{N consonant phonemes:} 36
\end{styleBody}

\begin{styleBody}
\textbf{Geminates:} N/A
\end{styleBody}

\begin{styleBody}
\textbf{Voicing contrasts:} Obstruents
\end{styleBody}

\begin{styleBody}
\textbf{Places:} Bilabial, Dental, Alveolar, Retroflex, Alveolo-palatal, Velar, Uvular, Glottal
\end{styleBody}

\begin{styleBody}
\textbf{Manners:} Stop, Affricate, Fricative, Nasal, Flap/Tap, Central approximant, Lateral approximant
\end{styleBody}

\begin{styleBody}
\textbf{N elaborations:} 4
\end{styleBody}

\begin{styleBody}
\textbf{Elaborations:} Voiced fricatives/affricates, Post-aspiration, Retroflex, Uvular
\end{styleBody}

\begin{styleBody}
\textbf{V phoneme inventory:} /i [25B?] [28C?] o u i[2D0?] [25B?][2D0?] [28C?][2D0?] o[2D0?] u[2D0?]/
\end{styleBody}

\begin{styleBody}
\textbf{N vowel qualities:} 5
\end{styleBody}

\begin{styleBody}
\textbf{Diphthongs or vowel sequences:} Diphthongs /[28C?]i [28C?]u/
\end{styleBody}

\begin{styleBody}
\textbf{Contrastive length:} All
\end{styleBody}

\begin{styleBody}
\textbf{Contrastive nasalization:} None
\end{styleBody}

\begin{styleBody}
\textbf{Other contrasts:} N/A
\end{styleBody}

\begin{styleBody}
\textbf{Notes:} /[270?][31F?]/ is an advanced velar approximant.
\end{styleBody}

\begin{styleBody}
\textbf{Syllable structure}
\end{styleBody}

\begin{styleBody}
\textbf{Complexity Category:} Complex
\end{styleBody}

\begin{styleBody}
\textbf{Canonical syllable structure:} (C)(C)V(C)(C)\textbf{ }(Anderson 1997: 1024-5; Yoshioka 2012: 18-24)
\end{styleBody}

\begin{styleBody}
\textbf{Size of maximal onset:} 2
\end{styleBody}

\begin{styleBody}
\textbf{Size of maximal coda:} 2
\end{styleBody}

\begin{styleBody}
\textbf{Onset obligatory:} No
\end{styleBody}

\begin{styleBody}
\textbf{Coda obligatory:} No
\end{styleBody}

\begin{styleBody}
\textbf{Vocalic nucleus patterns:} Short vowels, Long vowels, Diphthongs
\end{styleBody}

\begin{styleBody}
\textbf{Syllabic consonant patterns:} N/A
\end{styleBody}

\begin{styleBody}
\textbf{Size of maximal word{}-marginal sequences with syllabic obstruents:} N/A
\end{styleBody}

\begin{styleBody}
\textbf{Predictability of syllabic consonants:} N/A
\end{styleBody}

\begin{styleBody}
\textbf{Morphological constituency of maximal syllable margin:} Morpheme-internal (Onset, Coda)
\end{styleBody}

\begin{styleBody}
\textbf{Morphological pattern of syllabic consonants:} N/A
\end{styleBody}

\begin{styleBody}
\textbf{Onset restrictions: }All consonants may occur as a simple onset, though /ŋ/ and /j/ do not occur word-initially. Biconsonantal onsets occur with /p b p\textsuperscript{h} t d t\textsuperscript{h} [261?]/ as C\textsubscript{1} and /[27E?] j/ as C\textsubscript{2}. Anderson also gives example of /[261?][263?]/ onset in Standard Burushaski.
\end{styleBody}

\begin{styleBody}
\textbf{Coda restrictions: }Any consonant except /w j/ can occur as simple coda, though voiced stops and affricates and some fricatives are not found word-finally. In biconsonantal codas, C\textsubscript{1} is a voiceless fricative and C\textsubscript{2} is /k/, or C\textsubscript{1} is a sonorant and C\textsubscript{2} is /t k [282?] [255?] t[361?]s t[361?][255?] [288?][361?][282?]/. 
\end{styleBody}

\begin{styleBody}
\textbf{Notes: }Yoshioka states that all word-initial Cr onsets are from loan words and onomatopoeia, but Anderson gives examples that appear to be native (e.g. \textit{pra:q} ‘completely’).
\end{styleBody}

\begin{styleBody}
\textbf{Suprasegmentals}
\end{styleBody}

\begin{styleBody}
\textbf{Tone:} No
\end{styleBody}

\begin{styleBody}
\textbf{Word stress:} Yes
\end{styleBody}

\begin{styleBody}
\textbf{Stress placement:} Morphologically or Lexically Conditioned
\end{styleBody}

\begin{styleBody}
\textbf{Phonetic processes conditioned by stress:} Vowel Reduction, Consonant Allophony in Unstressed Syllables
\end{styleBody}

\begin{styleBody}
\textbf{Differences in phonological properties of stressed and unstressed syllables:} Vowel Quality Contrasts
\end{styleBody}

\begin{styleBody}
\textbf{Phonetic correlates of stress: }Pitch (impressionistic)
\end{styleBody}

\begin{styleBody}
\textbf{Notes: }Stress is marked by pitch or pitch contour. Long vowels are only found in stressed syllables. Language is described as having pitch accent by Yoshioka. Vowel length is found only in stressed syllables in underived lexical items (Anderson 1028).
\end{styleBody}

\begin{styleBody}
\textbf{Vowel reduction processes}
\end{styleBody}

\begin{styleBody}
\textbf{bsk-R1:} High vowels /i u/ are realized as lax in unstressed syllables (Anderson 1997: 1029).
\end{styleBody}

\begin{styleBody}
\textbf{bsk-R2: }Mid front vowel /e/ fluctuates with [[25B?]] in unstressed syllables (Anderson 1997: 1029).
\end{styleBody}

\begin{styleBody}
\textbf{Notes: }In Yasin dialect, unstressed /o/ frequently raises to [u] (Anderson 1997: 1038).
\end{styleBody}

\begin{styleBody}
\textbf{Consonant allophony processes}
\end{styleBody}

\begin{styleBody}
\textbf{bsk-C1: }Voiceless velar stop [k] varies freely with uvular [q] preceding /a/ (Anderson 1997: 1025).
\end{styleBody}

\begin{styleBody}
\textbf{bsk-C2: }Voiced velar fricative varies freely with a velar affricate and a voiced uvular stop syllable-initially (Anderson 1997: 1025).
\end{styleBody}

\begin{styleBody}
\textbf{bsk-C3: }Aspirated stops may be realized as affricates or fricatives syllable-initially (Anderson 1997: 1025).
\end{styleBody}

\begin{styleBody}
\textbf{bsk-C4}: Alveolo-palatal [d[361?][291?]] varies freely with fricative variant (Anderson 1997: 1025).
\end{styleBody}

\begin{styleBody}
\textbf{bsk-C5: }Velar fricative /x/ may be realized as [h] preceding /u/ (Anderson 1997: 1025).
\end{styleBody}

\begin{styleBody}
\textbf{Morphology}
\end{styleBody}

\begin{styleBody}
(adequate texts unavailable)
\end{styleBody}

\clearpage\begin{styleBody}
\textbf{[cap] }\ \ \textbf{\textsc{Chipaya}}\textbf{\ \ }Uru-Chipaya (Bolivia)
\end{styleBody}

\begin{styleBody}
\textbf{References consulted: }Cerrón-Palomino (2006), Olson (1967)
\end{styleBody}

\begin{styleBody}
\textbf{Sound inventory}
\end{styleBody}

\begin{styleBody}
\textbf{C phoneme inventory:} /p p[2B0?] p’ t t[2B0?] t’ k k[2B0?] k’ k[2B7?] q q[2B0?] q’ q[2B7?] t[361?]s t[361?]s[2B0?] t[361?]s’ t[361?][283?] t[361?][283?][2B0?] t[361?][283?]’ [288?][361?][282?] [288?][361?][282?][2B0?] [288?][361?][282?]’ s[32A?] s [282?] x x[2B7?] $\chi $ $\chi [2B7?]$ m n [272?] ŋ l [28E?] [29F?] r w j/
\end{styleBody}

\begin{styleBody}
\textbf{N consonant phonemes:} 40
\end{styleBody}

\begin{styleBody}
\textbf{Geminates:} N/A
\end{styleBody}

\begin{styleBody}
\textbf{Voicing contrasts:} None
\end{styleBody}

\begin{styleBody}
\textbf{Places:} Bilabial, Dental, Alveolar, Palato-alveolar, Retroflex, Velar, Uvular
\end{styleBody}

\begin{styleBody}
\textbf{Manners:} Stop, Affricate, Fricative, Nasal, Trill, Central approximant, Lateral approximant
\end{styleBody}

\begin{styleBody}
\textbf{N elaborations:} 6
\end{styleBody}

\begin{styleBody}
\textbf{Elaborations:} Post-aspiration, Ejective, Palato-alveolar, Retroflex, Uvular, Labialization
\end{styleBody}

\begin{styleBody}
\textbf{V phoneme inventory:} /i e a o u i[2D0?] e[2D0?] a[2D0?] o[2D0?] u[2D0?]/
\end{styleBody}

\begin{styleBody}
\textbf{N vowel qualities:} 5
\end{styleBody}

\begin{styleBody}
\textbf{Diphthongs or vowel sequences:} None
\end{styleBody}

\begin{styleBody}
\textbf{Contrastive length:} All
\end{styleBody}

\begin{styleBody}
\textbf{Contrastive nasalization:} None
\end{styleBody}

\begin{styleBody}
\textbf{Other contrasts:} N/A
\end{styleBody}

\begin{styleBody}
\textbf{Notes:} /[294?]/ occurs in restricted sociolinguistic contexts in one morpheme (Cerrón-Palomino 2006: 55).
\end{styleBody}

\begin{styleBody}
\textbf{Syllable structure}
\end{styleBody}

\begin{styleBody}
\textbf{Complexity Category:} Complex
\end{styleBody}

\begin{styleBody}
\textbf{Canonical syllable structure:} (C)(C)V(C)(C)\textbf{ }(Cerrón Palomino 2006: 63-66).
\end{styleBody}

\begin{styleBody}
\textbf{Size of maximal onset:} 2
\end{styleBody}

\begin{styleBody}
\textbf{Size of maximal coda:} 2
\end{styleBody}

\begin{styleBody}
\textbf{Onset obligatory:} No
\end{styleBody}

\begin{styleBody}
\textbf{Coda obligatory:} No
\end{styleBody}

\begin{styleBody}
\textbf{Vocalic nucleus patterns:} Short vowels, Long vowels
\end{styleBody}

\begin{styleBody}
\textbf{Syllabic consonant patterns:} Obstruent (Conflicting reports)
\end{styleBody}

\begin{styleBody}
\textbf{Size of maximal word{}-marginal sequences with syllabic obstruents:} N/A
\end{styleBody}

\begin{styleBody}
\textbf{Predictability of syllabic consonants:} N/A
\end{styleBody}

\begin{styleBody}
\textbf{Morphological constituency of maximal syllable margin:} Morpheme-internal (Onset), Both patterns (Coda)
\end{styleBody}

\begin{styleBody}
\textbf{Morphological pattern of syllabic consonants:} N/A
\end{styleBody}

\begin{styleBody}
\textbf{Onset restrictions: }No restrictions on simple onsets. Biconsonantal onsets have /s s[32A?] [282?]/ as C\textsubscript{1}. Only presently attested triconsonantal onset is /x[282?]t[2B0?]/, pronounced [h[282?]t[2B0?]].
\end{styleBody}

\begin{styleBody}
\textbf{Coda restrictions: }No restrictions on simple codas. Biconsonantal codas end in /s[32A?]/.
\end{styleBody}

\begin{styleBody}
\textbf{Notes:} Triconsonantal onsets used to be more common, as they are derived from a combination of prefixes and a stem-initial consonant; however, these forms are now completely unproductive and “almost obsolete”. Speakers passively accept /x[282?]t[2B0?]/ in two forms, \textit{x[282?]t[2B0?]a[2D0?]} ‘give it to me!’ and \textit{x[282?]t[2B0?]a[2D0?][282?]la[28E?]a} ‘give it to me, please!’. (Cerrón Palomino 2006: 66). Because this is explicitly described as a marginal and rapidly obsolescing pattern, I classify this language as having Complex syllable structure, while noting that it has recently shifted from having Highly Complex syllable structure.
\end{styleBody}

\begin{styleBody}
\textbf{Suprasegmentals}
\end{styleBody}

\begin{styleBody}
\textbf{Tone:} No
\end{styleBody}

\begin{styleBody}
\textbf{Word stress: }Yes
\end{styleBody}

\begin{styleBody}
\textbf{Stress placement:} Fixed
\end{styleBody}

\begin{styleBody}
\textbf{Phonetic processes conditioned by stress:} Vowel Reduction
\end{styleBody}

\begin{styleBody}
\textbf{Differences in phonological properties of stressed and unstressed syllables:} (None)
\end{styleBody}

\begin{styleBody}
\textbf{Phonetic correlates of stress: }Not described
\end{styleBody}

\begin{styleBody}
\textbf{Vowel reduction processes}
\end{styleBody}

\begin{styleBody}
\textbf{cap-R1:} Low central vowel /a/ is realized as [[259?]] in unstressed open syllables (Olson 1967: 301).
\end{styleBody}

\begin{styleBody}
\textbf{cap-R2:} Short vowels /i e a o u/ are devoiced when preceded by an aspirated consonant and followed by a voiceless consonant (usually a non-sibilant fricative) (Cerrón-Palomino 2006: 62).
\end{styleBody}

\begin{styleBody}
\textbf{cap-R3:} Short vowels are truncated (deleted) before a pause (Cerrón-Palomino 2006: 67).
\end{styleBody}

\begin{styleBody}
\textbf{Notes: }Vowel devoicing is one of the most salient phonetic properties of the language (Cerrón-Palomino 2006: 62). Historical elision of pre-stress vowels is responsible for some of the onset sequences in Chipaya (Cerrón-Palomino 2006: 65).
\end{styleBody}

\begin{styleBody}
\textbf{Consonant allophony processes}
\end{styleBody}

\begin{styleBody}
\textbf{cap-C1: }Voiceless dental fricative is realized as a palato-alveolar when occurring between two high vowels (Cerrón-Palomino 2006: 48-9).
\end{styleBody}

\begin{styleBody}
\textbf{cap-C2: }Labiovelar approximant [w] may be realized as a fricative intervocalically, especially when the surrounding vowels are /i/ (Cerrón-Palomino 2006: 55).
\end{styleBody}

\begin{styleBody}
\textbf{cap-C3: }A trill is realized as a flap syllable-finally (Cerrón-Palomino 2006: 54).
\end{styleBody}

\begin{styleBody}
\textbf{cap-C4: }A palato-alveolar affricate is realized as a fricative word-finally (Cerrón-Palomino 2006: 49).
\end{styleBody}

\begin{styleBody}
\textbf{cap-C5: }Velar and uvular stops vary freely with fricative variants (Cerrón-Palomino 2006: 38).
\end{styleBody}

\begin{styleBody}
\textbf{Morphology}
\end{styleBody}

\begin{styleBody}
\textbf{Text: }“Tata Sabaya y el Sajama” (Cerrón Palomino 2006: 286-291)
\end{styleBody}

\begin{styleBody}
\textbf{Synthetic index: }2.1 morphemes/word (342 morphemes, 161 words)
\end{styleBody}

\clearpage\begin{styleBody}
\textbf{[car] }\ \ \textbf{\textsc{Carib\ \ }}\textbf{\ \ }Cariban, \textit{Guianan} (Suriname)
\end{styleBody}

\begin{styleBody}
\textbf{References consulted: }Courtz (2008), Hoff (1968), Peasgood (1972)
\end{styleBody}

\begin{styleBody}
\textbf{Sound inventory}
\end{styleBody}

\begin{styleBody}
\textbf{C phoneme inventory:} /p b t d k [261?] [294?] s h m n ŋ [27D?] w j/
\end{styleBody}

\begin{styleBody}
\textbf{N consonant phonemes:} 15
\end{styleBody}

\begin{styleBody}
\textbf{Geminates:} N/A
\end{styleBody}

\begin{styleBody}
\textbf{Voicing contrasts:} Obstruents
\end{styleBody}

\begin{styleBody}
\textbf{Places:} Bilabial, Alveolar, Retroflex, Velar, Glottal
\end{styleBody}

\begin{styleBody}
\textbf{Manners:} Stop, Fricative, Nasal, Flap/Tap, Velar, Glottal
\end{styleBody}

\begin{styleBody}
\textbf{N elaborations:} 1
\end{styleBody}

\begin{styleBody}
\textbf{Elaborations:} Retroflex
\end{styleBody}

\begin{styleBody}
\textbf{V phoneme inventory:} /i e [268?] a u o/
\end{styleBody}

\begin{styleBody}
\textbf{N vowel qualities:} 6
\end{styleBody}

\begin{styleBody}
\textbf{Diphthongs or vowel sequences:} Diphthongs /ei ui oi ii ai ou au/
\end{styleBody}

\begin{styleBody}
\textbf{Contrastive length:} None
\end{styleBody}

\begin{styleBody}
\textbf{Contrastive nasalization:} None
\end{styleBody}

\begin{styleBody}
\textbf{Other contrasts:} N/A
\end{styleBody}

\begin{styleBody}
\textbf{Notes: }Hoff shows phonemic length contrast in a very limited set of lexical items in 1968. Peasgood has vowel length distinction. Courtz and Yamada take vowel length to be prosodic.
\end{styleBody}

\begin{styleBody}
\textbf{Syllable structure}
\end{styleBody}

\begin{styleBody}
\textbf{Category:} Moderately Complex
\end{styleBody}

\begin{styleBody}
\textbf{Canonical syllable structure:} (C)V(C) (Courtz 2008: 22-7)
\end{styleBody}

\begin{styleBody}
\textbf{Size of maximal onset:} 1
\end{styleBody}

\begin{styleBody}
\textbf{Size of maximal coda:} 1
\end{styleBody}

\begin{styleBody}
\textbf{Onset obligatory:} No
\end{styleBody}

\begin{styleBody}
\textbf{Coda obligatory:} No
\end{styleBody}

\begin{styleBody}
\textbf{Vocalic nucleus patterns:} Short vowels, Diphthongs
\end{styleBody}

\begin{styleBody}
\textbf{Syllabic consonant patterns:} N/A
\end{styleBody}

\begin{styleBody}
\textbf{Size of maximal word{}-marginal sequences with syllabic obstruents:} N/A
\end{styleBody}

\begin{styleBody}
\textbf{Predictability of syllabic consonants:} N/A
\end{styleBody}

\begin{styleBody}
\textbf{Morphological constituency of maximal syllable margin:} N/A
\end{styleBody}

\begin{styleBody}
\textbf{Morphological pattern of syllabic consonants:} N/A
\end{styleBody}

\begin{styleBody}
\textbf{Onset restrictions: }All consonants occur.
\end{styleBody}

\begin{styleBody}
\textbf{Coda restrictions: }Only nasals and plosives occur.
\end{styleBody}

\begin{styleBody}
\textbf{Notes:} “Underlying” stop-C onsets are realized with epenthesized [[268?]] or stop isn’t pronounced at all when occurring sentence-initially. Author interprets most word-initial instances of /[268?]/ as “auxiliary vowels” needed to pronounce syllables that have lost their original vowel; e.g. /[268?]nta/ (Courtz 2008: 26).
\end{styleBody}

\begin{styleBody}
\textbf{Suprasegmentals}
\end{styleBody}

\begin{styleBody}
\textbf{Tone: }No
\end{styleBody}

\begin{styleBody}
\textbf{Word stress:} Yes
\end{styleBody}

\begin{styleBody}
\textbf{Stress placement:} Weight-Sensitive
\end{styleBody}

\begin{styleBody}
\textbf{Phonetic processes conditioned by stress:} Vowel Reduction, Consonant Allophony in Unstressed Syllables
\end{styleBody}

\begin{styleBody}
\textbf{Differences in phonological properties of stressed and unstressed syllables:} (None)
\end{styleBody}

\begin{styleBody}
\textbf{Phonetic correlates of stress: }Vowel duration (impressionistic), Pitch (impressionistic)
\end{styleBody}

\begin{styleBody}
\textbf{Vowel reduction processes}
\end{styleBody}

\begin{styleBody}
\textbf{car-R1:} A word-medial vowel is devoiced preceding syllable-initial /s/ (Peasgood 1972: 38).
\end{styleBody}

\begin{styleBody}
\textbf{car-R2:} An unstressed word-initial high central vowel /[268?]/ is deleted (Courtz 2008: 40).
\end{styleBody}

\begin{styleBody}
\textbf{car-R3:} An unstressed word-initial high front vowel /i/ is deleted unless it precedes /[27D?]/. The high and front features of the deleted vowel perseverate into the following consonant (Courtz 2008: 41).
\end{styleBody}

\begin{styleBody}
\textbf{Consonant allophony processes}
\end{styleBody}

\begin{styleBody}
\textbf{car-C1: }A voiceless alveolar fricative is realized as palato-alveolar adjacent to /i/ (Courtz 2008: 32).
\end{styleBody}

\begin{styleBody}
\textbf{car-C2: }Voiceless stops are realized as voiced following an unstressed CV sequence or following a nasal (Courtz 2008: 31).
\end{styleBody}

\begin{styleBody}
\textbf{Morphology}
\end{styleBody}

\begin{styleBody}
\textbf{Text: }“Kurupi’s haircut” (first 10 pages; Courtz 2008: 150-159)
\end{styleBody}

\begin{styleBody}
\textbf{Synthetic index: }1.8 morphemes/word (619 morphemes, 353 words)
\end{styleBody}

\clearpage\begin{styleBody}
\textbf{[cav] }\ \ \textbf{\textsc{Cavineña\ \ }}\textbf{\ \ }Pano-Tacanan, \textit{Tacanan} (Bolivia)
\end{styleBody}

\begin{styleBody}
\textbf{References consulted: }Guillaume (2008)
\end{styleBody}

\begin{styleBody}
\textbf{Sound inventory}
\end{styleBody}

\begin{styleBody}
\textbf{C phoneme inventory:} /p b t d c [25F?] k k[2B7?] t[361?]s t[361?][255?] s [255?] h m n [272?] [27A?] [28E?] w j/
\end{styleBody}

\begin{styleBody}
\textbf{N consonant phonemes:} 20
\end{styleBody}

\begin{styleBody}
\textbf{Geminates:} N/A
\end{styleBody}

\begin{styleBody}
\textbf{Voicing contrasts: }Obstruents
\end{styleBody}

\begin{styleBody}
\textbf{Places: }Bilabial, Alveolar, Alveolo-palatal, Palatal, Velar, Glottal
\end{styleBody}

\begin{styleBody}
\textbf{Manners:} Stop, Affricate, Fricative, Nasal, Flap/tap, Central approximant, Lateral flap, Lateral approximant
\end{styleBody}

\begin{styleBody}
\textbf{N elaborations:} 1
\end{styleBody}

\begin{styleBody}
\textbf{Elaborations:} Labialization
\end{styleBody}

\begin{styleBody}
\textbf{V phoneme inventory:} /i e a [28A?]/
\end{styleBody}

\begin{styleBody}
\textbf{N vowel qualities:} 4
\end{styleBody}

\begin{styleBody}
\textbf{Diphthongs or vowel sequences:} None
\end{styleBody}

\begin{styleBody}
\textbf{Contrastive length:} None
\end{styleBody}

\begin{styleBody}
\textbf{Contrastive nasalization:} None
\end{styleBody}

\begin{styleBody}
\textbf{Other contrasts:} N/A
\end{styleBody}

\begin{styleBody}
\textbf{Notes:} Vowel sequences occur as distinct syllables, sometimes with intervening glottal stop insertion. (Guillaume 2008: 28-9).
\end{styleBody}

\begin{styleBody}
\textbf{Syllable structure}
\end{styleBody}

\begin{styleBody}
\textbf{Complexity Category:} Simple
\end{styleBody}

\begin{styleBody}
\textbf{Canonical syllable structure:} (C)V(C)\textbf{ }(Guillaume 2008: 30-32)
\end{styleBody}

\begin{styleBody}
\textbf{Size of maximal onset:} 1
\end{styleBody}

\begin{styleBody}
\textbf{Size of maximal coda: }1
\end{styleBody}

\begin{styleBody}
\textbf{Onset obligatory:} No
\end{styleBody}

\begin{styleBody}
\textbf{Coda obligatory:} N/A
\end{styleBody}

\begin{styleBody}
\textbf{Vocalic nucleus patterns: }Short vowels
\end{styleBody}

\begin{styleBody}
\textbf{Syllabic consonant patterns:} N/A
\end{styleBody}

\begin{styleBody}
\textbf{Size of maximal word{}-marginal sequences with syllabic obstruents:} N/A
\end{styleBody}

\begin{styleBody}
\textbf{Predictability of syllabic consonants:} N/A
\end{styleBody}

\begin{styleBody}
\textbf{Morphological constituency of maximal syllable margin:} N/A
\end{styleBody}

\begin{styleBody}
\textbf{Morphological pattern of syllabic consonants:} N/A
\end{styleBody}

\begin{styleBody}
\textbf{Onset restrictions:} None.
\end{styleBody}

\begin{styleBody}
\textbf{Coda restrictions:} /s/ or /n/
\end{styleBody}

\begin{styleBody}
\textbf{Notes:} Codas occur word-medially in only five native words. In four of these cases it is clear that the coda has arisen from an “idiosyncratic process of vowel elision” (Guillaume 2008: 31). Codas may also occur in interjections and onomatopoeia.
\end{styleBody}

\begin{styleBody}
\textbf{Suprasegmentals}
\end{styleBody}

\begin{styleBody}
\textbf{Tone:} Yes
\end{styleBody}

\begin{styleBody}
\textbf{Word stress: }Yes
\end{styleBody}

\begin{styleBody}
\textbf{Stress placement:} Fixed
\end{styleBody}

\begin{styleBody}
\textbf{Phonetic processes conditioned by stress:} (None)
\end{styleBody}

\begin{styleBody}
\textbf{Differences in phonological properties of stressed and unstressed syllables:} Tonal Contrasts
\end{styleBody}

\begin{styleBody}
\textbf{Phonetic correlates of stress: }Pitch (impressionistic)
\end{styleBody}

\begin{styleBody}
\textbf{Vowel reduction processes}
\end{styleBody}

\begin{styleBody}
\textbf{cav-R1: }Vowels /e [28A?]/ occasionally have more open variants [[25B?] o] (Guillaume 2008: 29).
\end{styleBody}

\begin{styleBody}
\textbf{Notes: }Historical “idiosyncratic process of vowel elision” has created codas in a small set of words (Guillaume 2008: 29).
\end{styleBody}

\begin{styleBody}
\textbf{Consonant allophony processes}
\end{styleBody}

\begin{styleBody}
(none reported)
\end{styleBody}

\begin{styleBody}
\textbf{Morphology}
\end{styleBody}

\begin{styleBody}
\textbf{Text:} “When the Araonas became angry with each other” (first 6 pages), “The woman who was eaten up by giant mosquitoes” (Guillaume 2008: 773-8; 796-8)
\end{styleBody}

\begin{styleBody}
\textbf{Synthetic index: }1.73 morphemes/word (535 morphemes, 309 words)
\end{styleBody}

\clearpage\begin{styleBody}
\textbf{[cho] }\ \ \textbf{\textsc{Choctaw}}\textbf{\ \ }Muskogean, \textit{Western Muskogean} (United States)
\end{styleBody}

\begin{styleBody}
\textbf{References consulted: }Broadwell (2006)
\end{styleBody}

\begin{styleBody}
\textbf{Sound inventory}
\end{styleBody}

\begin{styleBody}
\textbf{C phoneme inventory:} /p b t k [294?] t[361?][283?] f s [283?] h m n l [26C?] w j/
\end{styleBody}

\begin{styleBody}
\textbf{N consonant phonemes:} 16
\end{styleBody}

\begin{styleBody}
\textbf{Geminates:} N/A
\end{styleBody}

\begin{styleBody}
\textbf{Voicing contrasts:} Obstruents
\end{styleBody}

\begin{styleBody}
\textbf{Places:} Bilabial, Labiodental, Alveolar, Palato-alveolar, Velar, Glottal
\end{styleBody}

\begin{styleBody}
\textbf{Manners:} Stop, Affricate, Fricative, Nasal, Central approximant, Lateral fricative, Lateral approximant
\end{styleBody}

\begin{styleBody}
\textbf{N elaborations:} 2
\end{styleBody}

\begin{styleBody}
\textbf{Elaborations:} Labiodental, Palato-alveolar
\end{styleBody}

\begin{styleBody}
\textbf{V phoneme inventory:} /i a u i[2D0?] a[2D0?] u[2D0?] \~{i} \~{a} \~{u}/
\end{styleBody}

\begin{styleBody}
\textbf{N vowel qualities:} 3
\end{styleBody}

\begin{styleBody}
\textbf{Diphthongs or vowel sequences:} None
\end{styleBody}

\begin{styleBody}
\textbf{Contrastive length:} Some
\end{styleBody}

\begin{styleBody}
\textbf{Contrastive nasalization:} Some
\end{styleBody}

\begin{styleBody}
\textbf{Other contrasts: }N/A
\end{styleBody}

\begin{styleBody}
\textbf{Notes:} Short vowels also have nasal counterparts.
\end{styleBody}

\begin{styleBody}
\textbf{Syllable structure}
\end{styleBody}

\begin{styleBody}
\textbf{Category:} Moderately Complex
\end{styleBody}

\begin{styleBody}
\textbf{Canonical syllable structure:} (C)V(C) (Broadwell 2006: 18-21).
\end{styleBody}

\begin{styleBody}
\textbf{Size of maximal onset:} 1
\end{styleBody}

\begin{styleBody}
\textbf{Size of maximal coda:} 1
\end{styleBody}

\begin{styleBody}
\textbf{Onset obligatory:} No
\end{styleBody}

\begin{styleBody}
\textbf{Coda obligatory:} No
\end{styleBody}

\begin{styleBody}
\textbf{Vocalic nucleus patterns:} Short vowels, Long vowels
\end{styleBody}

\begin{styleBody}
\textbf{Syllabic consonant patterns:} N/A
\end{styleBody}

\begin{styleBody}
\textbf{Size of maximal word{}-marginal sequences with syllabic obstruents:} N/A
\end{styleBody}

\begin{styleBody}
\textbf{Predictability of syllabic consonants:} N/A
\end{styleBody}

\begin{styleBody}
\textbf{Morphological constituency of maximal syllable margin:} N/A
\end{styleBody}

\begin{styleBody}
\textbf{Morphological pattern of syllabic consonants:} N/A
\end{styleBody}

\begin{styleBody}
\textbf{Onset restrictions: }All consonants except /[294?]/ occur.
\end{styleBody}

\begin{styleBody}
\textbf{Coda restrictions:} All consonants except /b [26C?] w j t[361?][283?]/ occur.
\end{styleBody}

\begin{styleBody}
\textbf{Suprasegmentals}
\end{styleBody}

\begin{styleBody}
\textbf{Tone:} Yes
\end{styleBody}

\begin{styleBody}
\textbf{Word stress:} Yes
\end{styleBody}

\begin{styleBody}
\textbf{Stress placement:} Morphologically or Lexically Conditioned
\end{styleBody}

\begin{styleBody}
\textbf{Phonetic processes conditioned by stress:} Vowel Reduction
\end{styleBody}

\begin{styleBody}
\textbf{Differences in phonological properties of stressed and unstressed syllables:} (None)
\end{styleBody}

\begin{styleBody}
\textbf{Phonetic correlates of stress: }Pitch (impressionistic)
\end{styleBody}

\begin{styleBody}
\textbf{Notes:} Language has a pitch accent system: final syllable of each word has high or rising pitch, while some stems have additional high pitch on penultimate or antepenultimate syllable. (Ulrich) Pitch is very minimally contrastive in the language (Broadwell 2006: 17).
\end{styleBody}

\begin{styleBody}
\textbf{Vowel reduction processes}
\end{styleBody}

\begin{styleBody}
\textbf{cho-R1:} A word-initial (and unstressed) high front vowel /i/ may be deleted before a sequence of /s/ or /[283?]/ and another consonant, in casual speech (Broadwell 2006: 19).
\end{styleBody}

\begin{styleBody}
\textbf{cho-R2: }A long high front vowel /i[2D0?]/ is often lowered to [e[2D0?]] when occurring word-finally (Broadwell 2006: 30).
\end{styleBody}

\begin{styleBody}
\textbf{Consonant allophony processes}
\end{styleBody}

\begin{styleBody}
\textbf{cho-C1:} A voiceless velar stop is voiced intervocalically. (Broadwell 2006: 15)
\end{styleBody}

\begin{styleBody}
\textbf{cho-C2: }\ A voiceless velar stop may be realized as a voiced fricative intervocalically (Broadwell 2006: 15)
\end{styleBody}

\begin{styleBody}
\textbf{Morphology}
\end{styleBody}

\begin{styleBody}
\textbf{Text: }“My first days in school,” “Life at the orphanage” (Broadwell 2006: 355-360)
\end{styleBody}

\begin{styleBody}
\textbf{Synthetic index: }2.1 morphemes/word (552 morphemes, 263 words)
\end{styleBody}

\clearpage\begin{styleBody}
\textbf{[coc] }\ \ \textbf{\textsc{Cocopa}}\textbf{\ \ \ \ }Cochimi-Yuman, \textit{Yuman }(Mexico, United States)
\end{styleBody}

\begin{styleBody}
\textbf{References consulted: }Bendixen (1980), Crawford (1966)
\end{styleBody}

\begin{styleBody}
\textbf{Sound inventory}
\end{styleBody}

\begin{styleBody}
\textbf{C phoneme inventory:} /p t [288?] k k[2B7?] q q[2B7?] [294?] t[361?][283?] s [26C?] [282?] [283?] [26C?][2B2?] x x[2B7?] m n n[2B2?] l l[2B2?] [27E?] w j/
\end{styleBody}

\begin{styleBody}
\textbf{N consonant phonemes:} 24
\end{styleBody}

\begin{styleBody}
\textbf{Geminates:} N/A
\end{styleBody}

\begin{styleBody}
\textbf{Voicing contrasts:} None
\end{styleBody}

\begin{styleBody}
\textbf{Places:} Bilabial, Alveolar, Palato-alveolar, Retroflex, Velar, Uvular, Glottal
\end{styleBody}

\begin{styleBody}
\textbf{Manners:} Stop, Affricate, Fricative, Nasal, Flap/Tap, Central approximant, Lateral fricative, Lateral approximant
\end{styleBody}

\begin{styleBody}
\textbf{N elaborations:} 5
\end{styleBody}

\begin{styleBody}
\textbf{Elaborations:} Palato-alveolar, Retroflex, Uvular, Palatalization, Labialization
\end{styleBody}

\begin{styleBody}
\textbf{V phoneme inventory:} /i a u i[2D0?] a[2D0?] u[2D0?]/
\end{styleBody}

\begin{styleBody}
\textbf{N vowel qualities:} 3
\end{styleBody}

\begin{styleBody}
\textbf{Diphthongs or vowel sequences:} Diphthongs /iw uj aj aw i[2D0?]w u[2D0?]j a[2D0?]j a[2D0?]w/, Vowel sequences /ia i[2D0?]a a[2D0?]a ua u[2D0?]a/
\end{styleBody}

\begin{styleBody}
\textbf{Contrastive length:} All
\end{styleBody}

\begin{styleBody}
\textbf{Contrastive nasalization:} None
\end{styleBody}

\begin{styleBody}
\textbf{Other contrasts:} N/A
\end{styleBody}

\begin{styleBody}
\textbf{Notes:} /e/ occurs only in loanwords from Spanish, English.
\end{styleBody}

\begin{styleBody}
\textbf{Syllable structure}
\end{styleBody}

\begin{styleBody}
\textbf{Complexity Category:} Highly Complex
\end{styleBody}

\begin{styleBody}
\textbf{Canonical syllable structure:} (C)(C)(C)(C)V(C)(C)(C)\textbf{ }(Crawford 1966: 35-48; Bendixen 1980: 218-19)
\end{styleBody}

\begin{styleBody}
\textbf{Size of maximal onset:} 4
\end{styleBody}

\begin{styleBody}
\textbf{Size of maximal coda:} 3
\end{styleBody}

\begin{styleBody}
\textbf{Onset obligatory:} No
\end{styleBody}

\begin{styleBody}
\textbf{Coda obligatory:} No
\end{styleBody}

\begin{styleBody}
\textbf{Vocalic nucleus patterns:} Short vowels, Long vowels, Diphthongs, Vowel sequences
\end{styleBody}

\begin{styleBody}
\textbf{Syllabic consonant patterns:} Nasal, Liquid, Obstruent
\end{styleBody}

\begin{styleBody}
\textbf{Size of maximal word{}-marginal sequences with syllabic obstruents:} 5 (initial), 3 (final)
\end{styleBody}

\begin{styleBody}
\textbf{Predictability of syllabic consonants:} Predictable from word/consonantal context
\end{styleBody}

\begin{styleBody}
\textbf{Morphological constituency of maximal syllable margin:} Morphologically Complex (Onset, Coda)
\end{styleBody}

\begin{styleBody}
\textbf{Morphological pattern of syllabic consonants:} Grammatical items (Nasal, Obstruent), Both (Liquid)
\end{styleBody}

\begin{styleBody}
\textbf{Onset restrictions: }Biconsonantal onsets include stop-fricative, fricative-fricative clusters. Triconsonantal onsets include /sx[288?] psk[2B7?] xps/. Four-consonant onsets include /[282?]t[361?][283?]x[294?] p[282?]t[361?][283?][294?] [283?]xlm/. Only glottal stops may be contiguous with other stops.
\end{styleBody}

\begin{styleBody}
\textbf{Coda restrictions: }Contiguous stops may not be identical in coda clusters. Biconsonantal codas include sonorant+obstruent, obstruent+obstruent, with no contiguous stops: /[283?]k [282?]x kp lp n[2B2?]x ms/. Triconsonantal codas include sonorant+obstruent+obstruent, or three obstruents: /qsk [282?]sk [27E?]sk/.
\end{styleBody}

\begin{styleBody}
\textbf{Notes: }Obstruent-sonorant and sonorant-obstruent onsets reported by Crawford, but Bendixen states these are predictably split by epenthesis (1980: 219-20). Crawford claims there are different combinatory patterns occurring in onsets of stressed and unstressed syllables (1966: 35-37), but the description is confusing and the examples don’t clarify. Both stressed and unstressed syllables have, e.g., /psk[2B7?]/ onsets. Both Crawford and Bendixen propose that fricatives may occur as syllable nuclei, though Bendixen states this occurs only in fastest rates of speech (1980: 34).
\end{styleBody}

\begin{styleBody}
\textbf{Suprasegmentals}
\end{styleBody}

\begin{styleBody}
\textbf{Tone: }No
\end{styleBody}

\begin{styleBody}
\textbf{Word stress:} Yes
\end{styleBody}

\begin{styleBody}
\textbf{Stress placement:} Morphologically or Lexically Conditioned
\end{styleBody}

\begin{styleBody}
\textbf{Phonetic processes conditioned by stress:} Vowel Reduction, Consonant Allophony in Unstressed Syllables
\end{styleBody}

\begin{styleBody}
\textbf{Differences in phonological properties of stressed and unstressed syllables:} (None)
\end{styleBody}

\begin{styleBody}
\textbf{Phonetic correlates of stress: }Vowel duration (impressionistic), Pitch (instrumental), Intensity (instrumental)
\end{styleBody}

\begin{styleBody}
\textbf{Notes:} Bendixen instrumentally confirms intensity/amplitude for standard word stress; Crawford reports (without instrumental evidence) duration and pitch as correlates. Bendixen reports these (instrumentally) for emphatic stress. Stress only targets syllables containing vowels (Crawford 28).
\end{styleBody}

\begin{styleBody}
\textbf{Vowel reduction processes}
\end{styleBody}

\begin{styleBody}
\textbf{coc-R1:} Vowels in unstressed syllables are somewhat less tense than those of stressed syllables (Crawford 1966: 22).
\end{styleBody}

\begin{styleBody}
\textbf{coc-R2:} A stressed vowel is shortened when preceded by /w/ and a morpheme boundary (Bendixen 1980: 67).
\end{styleBody}

\begin{styleBody}
\textbf{Notes: }Short /i/ is relatively rare in unstressed syllables (Crawford 1966: 32). In formal oration, unstressed syllables are barely audible (Bendixen 1980: 332-3).
\end{styleBody}

\begin{styleBody}
\textbf{Consonant allophony processes}
\end{styleBody}

\begin{styleBody}
\textbf{coc-C1: }A voiceless velar stop is fronted preceding /i/. (Crawford 1966: 15)
\end{styleBody}

\begin{styleBody}
\textbf{coc-C2: }Stops may be voiced following a long vowel word-finally when the following word begins with a nasal. (Bendixen 1980: 99-100)
\end{styleBody}

\begin{styleBody}
\textbf{Morphology}
\end{styleBody}

\begin{styleBody}
(adequate texts unavailable)
\end{styleBody}

\clearpage\begin{styleBody}
\textbf{[cod] }\ \ \textbf{\textsc{Cocama-Cocamilla}}\textbf{\ \ }Tupian, \textit{Maweti-Guarani }(Peru)
\end{styleBody}

\begin{styleBody}
\textbf{References consulted: }Vallejos Yopán (2010), Vallejos Yopán (p.c.)
\end{styleBody}

\begin{styleBody}
\textbf{Sound inventory}
\end{styleBody}

\begin{styleBody}
\textbf{C phoneme inventory:} /p t k t[361?]s t[361?][283?] x m n [27E?] w j/
\end{styleBody}

\begin{styleBody}
\textbf{N consonant phonemes:} 11
\end{styleBody}

\begin{styleBody}
\textbf{Geminates:} N/A
\end{styleBody}

\begin{styleBody}
\textbf{Voicing contrasts:} None
\end{styleBody}

\begin{styleBody}
\textbf{Places:} Bilabial, Alveolar, Palato-alveolar, Velar
\end{styleBody}

\begin{styleBody}
\textbf{Manners:} Stop, Affricate, Fricative, Nasal, Flap/Tap, Central approximant
\end{styleBody}

\begin{styleBody}
\textbf{N elaborations:} 1
\end{styleBody}

\begin{styleBody}
\textbf{Elaborations:} Palato-alveolar
\end{styleBody}

\begin{styleBody}
\textbf{V phoneme inventory:} /i e [268?] a u/
\end{styleBody}

\begin{styleBody}
\textbf{N vowel qualities:} 5
\end{styleBody}

\begin{styleBody}
\textbf{Diphthongs or vowel sequences:} None
\end{styleBody}

\begin{styleBody}
\textbf{Contrastive length:} None
\end{styleBody}

\begin{styleBody}
\textbf{Contrastive nasalization:} None
\end{styleBody}

\begin{styleBody}
\textbf{Other contrasts:} N/A
\end{styleBody}

\begin{styleBody}
\textbf{Syllable structure}
\end{styleBody}

\begin{styleBody}
\textbf{Category:} Moderately Complex
\end{styleBody}

\begin{styleBody}
\textbf{Canonical syllable structure:} (C)(C)V(C) (Vallejos Yopán 2010: 112-15)
\end{styleBody}

\begin{styleBody}
\textbf{Size of maximal onset:} 2
\end{styleBody}

\begin{styleBody}
\textbf{Size of maximal coda:} 1
\end{styleBody}

\begin{styleBody}
\textbf{Onset obligatory:} No
\end{styleBody}

\begin{styleBody}
\textbf{Coda obligatory:} No
\end{styleBody}

\begin{styleBody}
\textbf{Vocalic nucleus patterns:} Short vowels
\end{styleBody}

\begin{styleBody}
\textbf{Syllabic consonant patterns:} N/A
\end{styleBody}

\begin{styleBody}
\textbf{Size of maximal word{}-marginal sequences with syllabic obstruents:} N/A
\end{styleBody}

\begin{styleBody}
\textbf{Predictability of syllabic consonants:} N/A
\end{styleBody}

\begin{styleBody}
\textbf{Morphological constituency of maximal syllable margin:} Morpheme-internal (Onset)
\end{styleBody}

\begin{styleBody}
\textbf{Morphological pattern of syllabic consonants:} N/A
\end{styleBody}

\begin{styleBody}
\textbf{Onset restrictions:} All consonants occur in simple onsets; in complex onsets C\textsubscript{1} is limited to /p t k [27E?] n/ and C\textsubscript{2} to glides /w/ and /j/.
\end{styleBody}

\begin{styleBody}
\textbf{Coda restrictions:} Only /w j n/ occur.
\end{styleBody}

\begin{styleBody}
\textbf{Notes: }(none)
\end{styleBody}

\begin{styleBody}
\textbf{Suprasegmentals}
\end{styleBody}

\begin{styleBody}
\textbf{Tone:} No
\end{styleBody}

\begin{styleBody}
\textbf{Word stress:} Yes
\end{styleBody}

\begin{styleBody}
\textbf{Stress placement:} Fixed
\end{styleBody}

\begin{styleBody}
\textbf{Phonetic processes conditioned by stress:} Vowel Reduction
\end{styleBody}

\begin{styleBody}
\textbf{Differences in phonological properties of stressed and unstressed syllables:} (None)
\end{styleBody}

\begin{styleBody}
\textbf{Phonetic correlates of stress: }Not described
\end{styleBody}

\begin{styleBody}
\textbf{Vowel reduction processes}
\end{styleBody}

\begin{styleBody}
\textbf{cod-R1:} The high back vowel /u/ is produced as lax or [o] word-finally (and following stressed syllable) (Vallejos Yopán 2010: 109).
\end{styleBody}

\begin{styleBody}
\textbf{cod-R2:} The high front vowel /i/ is produced as lax or [e] word-finally (and following a stressed syllable) following an approximant segment (Vallejos Yopán 2010: 109).
\end{styleBody}

\begin{styleBody}
\textbf{cod-R3: }The mid vowel /e/ is slightly centralized word-medially, especially in fast pronunciation (Vallejos Yopán 2010: 110).
\end{styleBody}

\begin{styleBody}
\textbf{cod-R4: }In words of more than three syllables, the vowel of the antepenultimate syllable (preceding the stressed syllable) is deleted (Vallejos Yopán 2010: 110-11).
\end{styleBody}

\begin{styleBody}
\textbf{cod-R5:} Unstressed high vowels /i u/ are deleted word-initially preceding homorganic approximant /j/ or /w/ when the following syllable is stressed (Vallejos Yopán 2010: 111-12).
\end{styleBody}

\begin{styleBody}
\textbf{Consonant allophony processes}
\end{styleBody}

\begin{styleBody}
\textbf{cod-C1: }Alveolar affricate is realized as palato-alveolar preceding a high front vowel (Vallejos Yopán 2010: 101).
\end{styleBody}

\begin{styleBody}
\textbf{cod-C2: }A palatal glide may be realized as [z] word-initially and intervocalically (Vallejos Yopán 2010: 99).
\end{styleBody}

\begin{styleBody}
\textbf{cod-C3: }A labiovelar glide may be realized as a fricative intervocalically (Vallejos Yopán 2010: 99).
\end{styleBody}

\begin{styleBody}
\textbf{cod-C4: }An alveolar nasal is realized as palatal preceding a palatal glide (Vallejos Yopán 2010).
\end{styleBody}

\begin{styleBody}
\textbf{cod-C5: }Stops are voiced following a nasal (Vallejos Yopán 2010: 98).
\end{styleBody}

\begin{styleBody}
\textbf{cod-C6: }An alveolar affricate may be realized as a fricative preceding a non-high vowel (Vallejos Yopán 2010: 100).
\end{styleBody}

\begin{styleBody}
\textbf{Morphology}
\end{styleBody}

\begin{styleBody}
\textbf{Text:} “Bite of snake” (first 10 pages, Vallejos Yopán 2010: 883-892)
\end{styleBody}

\begin{styleBody}
\textbf{Synthetic index: }1.5 morphemes/word (489 morphemes, 329 words)
\end{styleBody}

\clearpage\begin{styleBody}
\textbf{[cub] }\ \ \textbf{\textsc{Cubeo}}\textbf{\ \ \ \ }Tucanoan, \textit{Eastern Tucanoan} (Colombia)
\end{styleBody}

\begin{styleBody}
\textbf{References consulted: }Chacon (2012), Morse \& Maxwell (1999)
\end{styleBody}

\begin{styleBody}
\textbf{Sound inventory}
\end{styleBody}

\begin{styleBody}
\textbf{C phoneme inventory:} /p b t d k t[361?][283?] h [27E?] w ð[31E?] j/
\end{styleBody}

\begin{styleBody}
\textbf{N consonant phonemes:} 11
\end{styleBody}

\begin{styleBody}
\textbf{Geminates:} N/A
\end{styleBody}

\begin{styleBody}
\textbf{Voicing contrasts:} Obstruents
\end{styleBody}

\begin{styleBody}
\textbf{Places:} Bilabial, Alveolar, Palato-alveolar, Velar, Glottal
\end{styleBody}

\begin{styleBody}
\textbf{Manners:} Stop, Affricate, Fricative, Flap/Tap, Central approximant
\end{styleBody}

\begin{styleBody}
\textbf{N elaborations:} 1
\end{styleBody}

\begin{styleBody}
\textbf{Elaborations:} Palato-alveolar
\end{styleBody}

\begin{styleBody}
\textbf{V phoneme inventory:} /i e [268?] a o u \~{i} \~{e} [268?] \~{a} \~{o} \~{u}/
\end{styleBody}

\begin{styleBody}
\textbf{N vowel qualities:} 6
\end{styleBody}

\begin{styleBody}
\textbf{Diphthongs or vowel sequences:} Vowel sequences /ea oa ue ao au ei ui/ and many more
\end{styleBody}

\begin{styleBody}
\textbf{Contrastive length:} None
\end{styleBody}

\begin{styleBody}
\textbf{Contrastive nasalization:} All
\end{styleBody}

\begin{styleBody}
\textbf{Other contrasts:} N/A
\end{styleBody}

\begin{styleBody}
\textbf{Notes:} Morse \& Maxwell give /x/ instead of /h/, don’t have /ð[31E?]/. Chacon states [ð[31E?]] often allophone of /j/, but does contrast with /j/ word-initially preceding /a/ in a highly frequent stem (‘make’). However, Chacon also gives minimal pairs for /ð[31E?]/, but the phoneme has very limited distribution. Morse \& Maxwell give /[25B?]/ instead of /e/.
\end{styleBody}

\begin{styleBody}
\textbf{Syllable structure}
\end{styleBody}

\begin{styleBody}
\textbf{Complexity Category:} Simple
\end{styleBody}

\begin{styleBody}
\textbf{Canonical syllable structure:} C(V) (Chacon 2012: 163-7)
\end{styleBody}

\begin{styleBody}
\textbf{Size of maximal onset:} 1
\end{styleBody}

\begin{styleBody}
\textbf{Size of maximal coda:} N/A
\end{styleBody}

\begin{styleBody}
\textbf{Onset obligatory:} No
\end{styleBody}

\begin{styleBody}
\textbf{Coda obligatory:} N/A
\end{styleBody}

\begin{styleBody}
\textbf{Vocalic nucleus patterns:} Short vowels, Vowel sequences
\end{styleBody}

\begin{styleBody}
\textbf{Syllabic consonant patterns:} N/A
\end{styleBody}

\begin{styleBody}
\textbf{Size of maximal word{}-marginal sequences with syllabic obstruents:} N/A
\end{styleBody}

\begin{styleBody}
\textbf{Predictability of syllabic consonants:} N/A
\end{styleBody}

\begin{styleBody}
\textbf{Morphological constituency of maximal syllable margin:} N/A
\end{styleBody}

\begin{styleBody}
\textbf{Morphological pattern of syllabic consonants:} N/A
\end{styleBody}

\begin{styleBody}
\textbf{Onset restrictions: }All consonants occur.
\end{styleBody}

\begin{styleBody}
\textbf{Notes: }The third vowel in a sequence, if /i/, is acoustically similar to [j] (Chacon 2012: 52).
\end{styleBody}

\begin{styleBody}
\textbf{Suprasegmentals}
\end{styleBody}

\begin{styleBody}
\textbf{Tone:} Yes
\end{styleBody}

\begin{styleBody}
\textbf{Word stress:} Yes
\end{styleBody}

\begin{styleBody}
\textbf{Stress placement:} Morphologically or Lexically Conditioned
\end{styleBody}

\begin{styleBody}
\textbf{Phonetic processes conditioned by stress:} Vowel Reduction, Consonant Allophony in Unstressed Syllables, Consonant Allophony in Stressed Syllables
\end{styleBody}

\begin{styleBody}
\textbf{Differences in phonological properties of stressed and unstressed syllables:} (None)
\end{styleBody}

\begin{styleBody}
\textbf{Phonetic correlates of stress: }Vowel duration (instrumental), Pitch (instrumental), Intensity (instrumental)
\end{styleBody}

\begin{styleBody}
\textbf{Notes: }Duration is the most consistent correlate of stress, intensity is less clear. “Kubeo tones are best seen as word-level contours, since they impose a particular pitch contour on a large section of an entire word, not only on individual syllables{\textquotedbl} (Chacon 134). Tones only occur on primary stressed syllables and syllables to the right of that.
\end{styleBody}

\begin{styleBody}
\textbf{Vowel reduction processes}
\end{styleBody}

\begin{styleBody}
\textbf{cub-R1:} Vowels in unstressed syllables are shorter, or may be deleted entirely (Chacon 2012: 109, 123; instrumental evidence pp. 155-9). Other segments in unstressed syllables can additionally be deleted.
\end{styleBody}

\begin{styleBody}
\textbf{Notes: }In sequences of three vowels analyzed as tautosyllabic by author, third vowel, if /i/, may be realized as [j] (Chacon 2012: 52).
\end{styleBody}

\begin{styleBody}
\textbf{Consonant allophony processes}
\end{styleBody}

\begin{styleBody}
\textbf{cub-C1: }A palatal glide may be realized as a palato-alveolar affricate, especially in word-initial stressed syllables, but also word-initially in unstressed syllables (Chacon 2012: 67).
\end{styleBody}

\begin{styleBody}
\textbf{cub-C2: }A labiovelar glide may be realized as a fricative preceding non-front vowels (Chacon 2012: 63).
\end{styleBody}

\begin{styleBody}
\textbf{cub-C3: }A\textbf{ }voiced alveolar stop is realized as an alveolar flap intervocalically (Chacon 2012: 63).
\end{styleBody}

\begin{styleBody}
\textbf{cub-C4: }A voiced alveolar stop is realized as a retroflex flap following any vowel and preceding a front vowel (Chacon 2012: 6).
\end{styleBody}

\begin{styleBody}
\textbf{cub-C5: }Voiceless bilabial and velar stops are sometimes realized as a glottal fricative when occurring in a post-stress syllable (Chacon 2012: 123).
\end{styleBody}

\begin{styleBody}
\textbf{Morphology}
\end{styleBody}

\begin{styleBody}
(adequate texts unavailable)
\end{styleBody}

\clearpage\begin{styleBody}
\textbf{[dow] }\ \ \textbf{\textsc{Doyayo}}\textbf{\ \ \ \ }Atlantic-Congo, \textit{Volta-Congo} (Cameroon)
\end{styleBody}

\begin{styleBody}
\textbf{References consulted: }Wiering \& Wiering (1994)
\end{styleBody}

\begin{styleBody}
\textbf{Sound inventory}
\end{styleBody}

\begin{styleBody}
\textbf{C phoneme inventory:} /p t k k[361?]p b [253?] d [257?] [261?] [261?][361?]b f v s z h m n ŋ l [27E?] w j/
\end{styleBody}

\begin{styleBody}
\textbf{N consonant phonemes:} 22
\end{styleBody}

\begin{styleBody}
\textbf{Geminates:} N/A
\end{styleBody}

\begin{styleBody}
\textbf{Voicing contrasts:} Obstruents
\end{styleBody}

\begin{styleBody}
\textbf{Places:} Labial-velar, Bilabial, Labiodental, Alveolar, Velar, Glottal
\end{styleBody}

\begin{styleBody}
\textbf{Manners:} Stop, Fricative, Nasal, Flap/Tap, Central approximant, Lateral approximant
\end{styleBody}

\begin{styleBody}
\textbf{N elaborations:} 3
\end{styleBody}

\begin{styleBody}
\textbf{Elaborations:} Voiced fricatives/affricates, Implosive, Labiodental
\end{styleBody}

\begin{styleBody}
\textbf{V phoneme inventory:} /i e [25B?] a [254?] o u i[2D0?] e[2D0?] [25B?][2D0?] a[2D0?] [254?][2D0?] o[2D0?] u[2D0?] \~{i} \~{e} [25B?] \~{a} [254?] \~{o} \~{u} \~{i}[2D0?] [25B?][2D0?] \~{a}[2D0?] [254?][2D0?] \~{u}[2D0?]/
\end{styleBody}

\begin{styleBody}
\textbf{N vowel qualities:} 7
\end{styleBody}

\begin{styleBody}
\textbf{Diphthongs or vowel sequences:} None
\end{styleBody}

\begin{styleBody}
\textbf{Contrastive length:} All
\end{styleBody}

\begin{styleBody}
\textbf{Contrastive nasalization:} Some
\end{styleBody}

\begin{styleBody}
\textbf{Other contrasts:} N/A
\end{styleBody}

\begin{styleBody}
\textbf{Notes:} There is a nasal contrast for all but /e e[2D0?] o o[2D0?]/.
\end{styleBody}

\begin{styleBody}
\textbf{Syllable structure}
\end{styleBody}

\begin{styleBody}
\textbf{Complexity Category:} Highly Complex
\end{styleBody}

\begin{styleBody}
\textbf{Canonical syllable structure:} (C)V(C)(C)(C)(C)\textbf{ }(Wiering \& Wiering 1994: 21-23, 37-43)
\end{styleBody}

\begin{styleBody}
\textbf{Size of maximal onset:} 1
\end{styleBody}

\begin{styleBody}
\textbf{Size of maximal coda:} 4
\end{styleBody}

\begin{styleBody}
\textbf{Onset obligatory:} No
\end{styleBody}

\begin{styleBody}
\textbf{Coda obligatory:} No
\end{styleBody}

\begin{styleBody}
\textbf{Vocalic nucleus patterns:} Short vowels, Long vowels
\end{styleBody}

\begin{styleBody}
\textbf{Syllabic consonant patterns:} Nasal
\end{styleBody}

\begin{styleBody}
\textbf{Size of maximal word{}-marginal sequences with syllabic obstruents:} N/A
\end{styleBody}

\begin{styleBody}
\textbf{Predictability of syllabic consonants:} Varies with CV sequence
\end{styleBody}

\begin{styleBody}
\textbf{Morphological constituency of maximal syllable margin:} Morphologically Complex (Coda)
\end{styleBody}

\begin{styleBody}
\textbf{Morphological pattern of syllabic consonants:} N/A
\end{styleBody}

\begin{styleBody}
\textbf{Onset restrictions: }All consonants but /ŋ/ may occur in onset.
\end{styleBody}

\begin{styleBody}
\textbf{Coda restrictions: }All consonants but /p [253?] [257?] k[361?]p [261?][361?]b $\beta $ h/ may occur as simple codas. Biconsonantal coda combinations are quite extensive, include /[27E?]k, pt, ts, kt, $\beta [27E?]$/. Triconsonantal codas include /b[27E?]t/ (phonetically [$\beta [27E?]$t]), /[261?]lt/ (phonetically [[263?]lt]), and more. Four-consonant codas include /blts/, /[261?]ldz/, /mnts/, /ŋ[27E?]dz/, and more. In largest clusters, C\textsubscript{1} is limited to /b [261?] m ŋ/, C\textsubscript{2} to /l [27E?] n/, C\textsubscript{3} to /d t/, and C\textsubscript{4} to /s z/. C\textsubscript{3} and C\textsubscript{4} must match in voicing. /b [261?]/ usually realized as fricatives in clusters.
\end{styleBody}

\begin{styleBody}
\textbf{Suprasegmentals}
\end{styleBody}

\begin{styleBody}
\textbf{Tone:} Yes
\end{styleBody}

\begin{styleBody}
\textbf{Word stress:} Not reported
\end{styleBody}

\begin{styleBody}
\textbf{Vowel reduction processes}
\end{styleBody}

\begin{styleBody}
\textbf{dow-R1:} A long vowel is optionally shortened preceding a coda of two or three consonants (Wiering \& Wiering 1994: 22).
\end{styleBody}

\begin{styleBody}
\textbf{dow-R2:} A long vowel is obligatorily shortened preceding a coda of four consonants (Wiering \& Wiering 1994: 22).
\end{styleBody}

\begin{styleBody}
\textbf{dow-R3:} Following any stop other than /b/, a sequence of vowel plus alveolar nasal consonant is realized as a syllabic nasal (Wiering \& Wiering 1994: 24).
\end{styleBody}

\begin{styleBody}
\textbf{Consonant allophony processes}
\end{styleBody}

\begin{styleBody}
\textbf{dow-C1: }Voiced bilabial and velar stops are spirantized initially in a voiced consonant cluster (Wiering \& Wiering 1994: 31-2).
\end{styleBody}

\begin{styleBody}
\textbf{Morphology}
\end{styleBody}

\begin{styleBody}
(adequate texts unavailable)
\end{styleBody}

\clearpage\begin{styleBody}
\textbf{[dru] }\ \ \textbf{\textsc{Rukai (Budai dialect)\ \ }}Austronesian (Taiwan)
\end{styleBody}

\begin{styleBody}
\textbf{References consulted: }Chen (2006)
\end{styleBody}

\begin{styleBody}
\textbf{Sound inventory}
\end{styleBody}

\begin{styleBody}
\textbf{C phoneme inventory:} /p b t d [256?] k [261?] t[361?]s v $\theta $ ð s m n ŋ r l [26D?] w j/
\end{styleBody}

\begin{styleBody}
\textbf{N consonant phonemes:} 20
\end{styleBody}

\begin{styleBody}
\textbf{Geminates:} N/A
\end{styleBody}

\begin{styleBody}
\textbf{Voicing contrasts:} Obstruents
\end{styleBody}

\begin{styleBody}
\textbf{Places:} Bilabial, Labiodental, Dental, Alveolar, Retroflex, Velar
\end{styleBody}

\begin{styleBody}
\textbf{Manners:} Stop, Affricate, Fricative, Nasal, Trill, Central approximant, Lateral approximant
\end{styleBody}

\begin{styleBody}
\textbf{N elaborations:} 3
\end{styleBody}

\begin{styleBody}
\textbf{Elaborations:} Voiced fricatives/affricates, Labiodental, Retroflex
\end{styleBody}

\begin{styleBody}
\textbf{V phoneme inventory:} /i [259?] a u i[2D0?] e[2D0?] a[2D0?] u[2D0?]/
\end{styleBody}

\begin{styleBody}
\textbf{N vowel qualities:} 4
\end{styleBody}

\begin{styleBody}
\textbf{Diphthongs or vowel sequences:} Diphthongs /au ai ia ua/
\end{styleBody}

\begin{styleBody}
\textbf{Contrastive length:} All
\end{styleBody}

\begin{styleBody}
\textbf{Contrastive nasalization:} None
\end{styleBody}

\begin{styleBody}
\textbf{Other contrasts:} N/A
\end{styleBody}

\begin{styleBody}
\textbf{Notes:} Long vowels are contrastive in monosyllabic words and first syllable of disyllabic words, but not in penultimate position.
\end{styleBody}

\begin{styleBody}
\textbf{Syllable structure}
\end{styleBody}

\begin{styleBody}
\textbf{Complexity Category:} Simple
\end{styleBody}

\begin{styleBody}
\textbf{Canonical syllable structure:} (C)V\textbf{ }(Chen 2006: 211-18)
\end{styleBody}

\begin{styleBody}
\textbf{Size of maximal onset:} 1
\end{styleBody}

\begin{styleBody}
\textbf{Size of maximal coda:} N/A
\end{styleBody}

\begin{styleBody}
\textbf{Onset obligatory:} No
\end{styleBody}

\begin{styleBody}
\textbf{Coda obligatory:} N/A
\end{styleBody}

\begin{styleBody}
\textbf{Vocalic nucleus patterns:} Short vowels, Long vowels, Diphthongs
\end{styleBody}

\begin{styleBody}
\textbf{Syllabic consonant patterns:} N/A
\end{styleBody}

\begin{styleBody}
\textbf{Size of maximal word{}-marginal sequences with syllabic obstruents:} N/A
\end{styleBody}

\begin{styleBody}
\textbf{Predictability of syllabic consonants:} N/A
\end{styleBody}

\begin{styleBody}
\textbf{Morphological constituency of maximal syllable margin:} N/A
\end{styleBody}

\begin{styleBody}
\textbf{Morphological pattern of syllabic consonants:} N/A
\end{styleBody}

\begin{styleBody}
\textbf{Onset restrictions: }All consonants occur.
\end{styleBody}

\begin{styleBody}
\textbf{Notes: }Most Formosan languages have canonical (C)V(C) structure; related language Paiwan also has (C)V(C). In Budai Rukai, a small number of sonorant codas such as nasals and laterals were attested in fast speech, but reconfirmation by author revealed (C)V forms for these (Chen 2006: 213).
\end{styleBody}

\begin{styleBody}
\textbf{Suprasegmentals}
\end{styleBody}

\begin{styleBody}
\textbf{Tone:} No
\end{styleBody}

\begin{styleBody}
\textbf{Word stress:} Yes
\end{styleBody}

\begin{styleBody}
\textbf{Stress placement:} Weight-Sensitive
\end{styleBody}

\begin{styleBody}
\textbf{Phonetic processes conditioned by stress:} Vowel Reduction
\end{styleBody}

\begin{styleBody}
\textbf{Differences in phonological properties of stressed and unstressed syllables:} (None)
\end{styleBody}

\begin{styleBody}
\textbf{Phonetic correlates of stress: }Vowel duration (instrumental), Pitch (instrumental)
\end{styleBody}

\begin{styleBody}
\textbf{Notes: }Pitch is a strong cue for stress in long and short vowels; duration a stronger cue for long vowels and is somewhat sensitive to word position of stress.
\end{styleBody}

\begin{styleBody}
\textbf{Vowel reduction processes}
\end{styleBody}

\begin{styleBody}
\textbf{dru-R1:} Long vowels are shortened when occurring in non-main stress position (Chen 2006: 257).
\end{styleBody}

\begin{styleBody}
\textbf{Consonant allophony processes}
\end{styleBody}

\begin{styleBody}
\textbf{dru-C1: }Voiceless alveolar fricative and affricate are realized as palato-alveolar preceding a high front vowel (Chen 2006: 230).
\end{styleBody}

\begin{styleBody}
\textbf{dru-C2: }A voiced labiodental fricative may be realized as a stop word-initially preceding schwa (Chen 2006: 227).
\end{styleBody}

\begin{styleBody}
\textbf{dru-C3: }A voiced labiodental fricative may be realized as a glide word-initially preceding a non-schwa vowel (Chen 2006: 227).
\end{styleBody}

\begin{styleBody}
\textbf{Morphology}
\end{styleBody}

\begin{styleBody}
(adequate texts unavailable)
\end{styleBody}

\clearpage\begin{styleBody}
\textbf{[dry] }\ \ \textbf{\textsc{Darai}}\textbf{\ \ \ \ }Indo-European, \textit{Indo-Iranian} (Nepal)
\end{styleBody}

\begin{styleBody}
\textbf{References consulted: }Dhakal (2012), Kotapish \& Kotapish (1978), Paudyal (2003), Netra P. Paudyal (p.c.)
\end{styleBody}

\begin{styleBody}
\textbf{Sound inventory}
\end{styleBody}

\begin{styleBody}
\textbf{C phoneme inventory:} /p b t[32A?] d[32A?] [288?] [256?] k [261?] p[2B0?] b[2B0?] t[32A?][2B0?] d[32A?][2B0?] [288?][2B0?] [256?][2B0?] k[2B0?] [261?][2B0?] t[361?]s d[361?]z t[361?]s[2B0?] d[361?]z[2B0?] s [266?] m n[32A?] ŋ r l $\beta [31E?]$ j/
\end{styleBody}

\begin{styleBody}
\textbf{N consonant phonemes:} 29
\end{styleBody}

\begin{styleBody}
\textbf{Geminates:} N/A
\end{styleBody}

\begin{styleBody}
\textbf{Voicing contrasts:} Obstruents
\end{styleBody}

\begin{styleBody}
\textbf{Places:} Bilabial, Dental, Alveolar, Retroflex, Velar, Glottal
\end{styleBody}

\begin{styleBody}
\textbf{Manners:} Stop, Affricate, Fricative, Nasal, Trill, Central approximant, Lateral approximant
\end{styleBody}

\begin{styleBody}
\textbf{N elaborations:} 4
\end{styleBody}

\begin{styleBody}
\textbf{Elaborations:} Breathy voice, Voiced fricatives/affricates, Post-aspiration, Retroflex
\end{styleBody}

\begin{styleBody}
\textbf{V phoneme inventory:} /i e [259?] a o u \~{i} \~{e} [259?] \~{a} \~{o} \~{u}/
\end{styleBody}

\begin{styleBody}
\textbf{N vowel qualities:} 6
\end{styleBody}

\begin{styleBody}
\textbf{Diphthongs or vowel sequences:} Diphthongs /iu eu au [259?]u ou ei ai ui [259?]i oi/
\end{styleBody}

\begin{styleBody}
\textbf{Contrastive length:} None
\end{styleBody}

\begin{styleBody}
\textbf{Contrastive nasalization:} All
\end{styleBody}

\begin{styleBody}
\textbf{Other contrasts:} N/A
\end{styleBody}

\begin{styleBody}
\textbf{Notes:} Kotapish \& Kotapish 1973 also report /[27D?]/. Paudyal gives /[28C?]/ instead of /[259?]/. All six vowels are marginally contrastive for nasality (Dhakal 2012: 7).
\end{styleBody}

\begin{styleBody}
\textbf{Syllable structure}
\end{styleBody}

\begin{styleBody}
\textbf{Complexity Category:} Moderately Complex
\end{styleBody}

\begin{styleBody}
\textbf{Canonical syllable structure:} (C)(C)V(C)\textbf{ }(Dhakal 2012: 17-20)
\end{styleBody}

\begin{styleBody}
\textbf{Size of maximal onset:} 2
\end{styleBody}

\begin{styleBody}
\textbf{Size of maximal coda:} 1
\end{styleBody}

\begin{styleBody}
\textbf{Onset obligatory:} No
\end{styleBody}

\begin{styleBody}
\textbf{Coda obligatory:} No
\end{styleBody}

\begin{styleBody}
\textbf{Vocalic nucleus patterns:} Short vowels, Diphthongs
\end{styleBody}

\begin{styleBody}
\textbf{Syllabic consonant patterns:} N/A
\end{styleBody}

\begin{styleBody}
\textbf{Size of maximal word{}-marginal sequences with syllabic obstruents:} N/A
\end{styleBody}

\begin{styleBody}
\textbf{Predictability of syllabic consonants:} N/A
\end{styleBody}

\begin{styleBody}
\textbf{Morphological constituency of maximal syllable margin:} Morpheme-internal (Onset)
\end{styleBody}

\begin{styleBody}
\textbf{Morphological pattern of syllabic consonants:} N/A
\end{styleBody}

\begin{styleBody}
\textbf{Onset restrictions: }C\textsubscript{1} may be any consonant except /ŋ/. C\textsubscript{2} is always a glide /$\beta [31E?]$/ or /j/.
\end{styleBody}

\begin{styleBody}
\textbf{Coda restrictions: }All consonants except for glides and glottal fricative /h/ are attested.
\end{styleBody}

\begin{styleBody}
Nucleus:
\end{styleBody}

\begin{styleBody}
\textbf{Notes: }Syllables lacking onsets are attested but very rare (Dhakal 2012: 19).
\end{styleBody}

\begin{styleBody}
\textbf{Suprasegmentals}
\end{styleBody}

\begin{styleBody}
\textbf{Tone: }No
\end{styleBody}

\begin{styleBody}
\textbf{Word stress:} Yes
\end{styleBody}

\begin{styleBody}
\textbf{Stress placement:} Fixed
\end{styleBody}

\begin{styleBody}
\textbf{Phonetic processes conditioned by stress:} Vowel Reduction
\end{styleBody}

\begin{styleBody}
\textbf{Differences in phonological properties of stressed and unstressed syllables:} Not described
\end{styleBody}

\begin{styleBody}
\textbf{Phonetic correlates of stress: }Not described
\end{styleBody}

\begin{styleBody}
\textbf{Vowel reduction processes}
\end{styleBody}

\begin{styleBody}
\textbf{dry-R1: }High back vowel /u/ may be deleted preceding a sequence of /wa/ and another consonant (Kotapish \& Kotapish 1973: 49).
\end{styleBody}

\begin{styleBody}
\textbf{Consonant allophony processes}
\end{styleBody}

\begin{styleBody}
\textbf{dry-C1: }Alveolar affricates and voiceless alveolar fricative are realized as palato-alveolar preceding a front vowel (Kotapish \& Kotapish 1978: 26).
\end{styleBody}

\begin{styleBody}
\textbf{dry-C2: }A voiceless aspirated velar fricative is realized as affricate [kx] following a vowel and preceding a schwa (Kotapish \& Kotapish 1978: 26).
\end{styleBody}

\begin{styleBody}
\textbf{dry-C3: }An alveolar flap is realized with palato-alveolar fricative release word finally (Kotapish \& Kotapish 1978: 24).
\end{styleBody}

\begin{styleBody}
\textbf{dry-C4: }Bilabial stops are realized as palatalized word-initially preceding /e/ (Kotapish \& Kotapish 1978: 27).
\end{styleBody}

\begin{styleBody}
\textbf{dry-C5: }A voiced bilabial stop is realized as prenasalized intervocalically (Kotapish \& Kotapish 1978: 27).
\end{styleBody}

\begin{styleBody}
\textbf{dry-C6: }A voiceless bilabial stop is spirantized intervocalically or between a vowel and a consonant (Kotapish \& Kotapish 1978: 17).
\end{styleBody}

\begin{styleBody}
\textbf{dry-C7: }A voiceless alveolar fricative varies with a glottal fricative preceding alveolar sonorants (Kotapish \& Kotapish 1978).
\end{styleBody}

\begin{styleBody}
\textbf{Morphology}
\end{styleBody}

\begin{styleBody}
\textbf{Text: }“Jackal and Hen” (Dhakal 2012: 180-192)
\end{styleBody}

\begin{styleBody}
\textbf{Synthetic index: }1.6 morphemes/word (734 morphemes, 472 words)
\end{styleBody}

\clearpage\begin{styleBody}
\textbf{[dyo] }\ \ \textbf{\textsc{Jola-Fonyi}}\textbf{\ \ }Atlantic-Congo, \textit{North-Central Atlantic} (Gambia, Senegal)
\end{styleBody}

\begin{styleBody}
\textbf{References consulted: }Lavergne (1979), Sapir (1965)
\end{styleBody}

\begin{styleBody}
\textbf{Sound inventory}
\end{styleBody}

\begin{styleBody}
\textbf{C phoneme inventory:} /p b t d c [25F?] k [261?] f s h m n [272?] ŋ l [279?] w j/
\end{styleBody}

\begin{styleBody}
\textbf{N consonant phonemes:} 19
\end{styleBody}

\begin{styleBody}
\textbf{Geminates:} N/A
\end{styleBody}

\begin{styleBody}
\textbf{Voicing contrasts:} Obstruents
\end{styleBody}

\begin{styleBody}
\textbf{Places:} Bilabial, Labiodental, Alveolar, Palatal, Velar, Glottal
\end{styleBody}

\begin{styleBody}
\textbf{Manners:} Stop, Fricative, Nasal, Central approximant, Lateral approximant
\end{styleBody}

\begin{styleBody}
\textbf{N elaborations:} 1
\end{styleBody}

\begin{styleBody}
\textbf{Elaborations:} Labiodental
\end{styleBody}

\begin{styleBody}
\textbf{V phoneme inventory:} /i [26A?] e [25B?] [258?] a [254?] o [28A?] u i[2D0?] [26A?][2D0?] e[2D0?] [25B?][2D0?] [258?][2D0?] a[2D0?] [254?][2D0?] o[2D0?] [28A?][2D0?] u[2D0?]/
\end{styleBody}

\begin{styleBody}
\textbf{N vowel qualities:} 10
\end{styleBody}

\begin{styleBody}
\textbf{Diphthongs or vowel sequences:} Diphthongs /e[28A?] iu [26A?]e [254?]a e[26A?]/
\end{styleBody}

\begin{styleBody}
\textbf{Contrastive length:} All
\end{styleBody}

\begin{styleBody}
\textbf{Contrastive nasalization:} None
\end{styleBody}

\begin{styleBody}
\textbf{Other contrasts:} N/A
\end{styleBody}

\begin{styleBody}
\textbf{Notes: }Diphthongs are rare.
\end{styleBody}

\begin{styleBody}
\textbf{Syllable structure}
\end{styleBody}

\begin{styleBody}
\textbf{Complexity Category:} Complex
\end{styleBody}

\begin{styleBody}
\textbf{Canonical syllable structure:} (C)V(C)(C)\textbf{ }(Sapir 1965: 6-9)
\end{styleBody}

\begin{styleBody}
\textbf{Size of maximal onset:} 1
\end{styleBody}

\begin{styleBody}
\textbf{Size of maximal coda:} 2
\end{styleBody}

\begin{styleBody}
\textbf{Onset obligatory:} No
\end{styleBody}

\begin{styleBody}
\textbf{Coda obligatory:} No
\end{styleBody}

\begin{styleBody}
\textbf{Vocalic nucleus patterns:} Short vowels, Long vowels, Diphthongs
\end{styleBody}

\begin{styleBody}
\textbf{Syllabic consonant patterns:} Nasal
\end{styleBody}

\begin{styleBody}
\textbf{Size of maximal word{}-marginal sequences with syllabic obstruents:} N/A
\end{styleBody}

\begin{styleBody}
\textbf{Predictability of syllabic consonants:} Predictable from word/consonantal context
\end{styleBody}

\begin{styleBody}
\textbf{Morphological constituency of maximal syllable margin:} Morpheme-internal (Coda)
\end{styleBody}

\begin{styleBody}
\textbf{Morphological pattern of syllabic consonants:} Grammatical items
\end{styleBody}

\begin{styleBody}
\textbf{Onset restrictions: }All consonants occur.
\end{styleBody}

\begin{styleBody}
\textbf{Coda restrictions: }For simple codas, all consonants except /d/ may occur. For complex codas, C\textsubscript{1} is a nasal, C\textsubscript{2} a stop.
\end{styleBody}

\begin{styleBody}
\textbf{Suprasegmentals}
\end{styleBody}

\begin{styleBody}
\textbf{Tone:} No
\end{styleBody}

\begin{styleBody}
\textbf{Word stress:} Yes
\end{styleBody}

\begin{styleBody}
\textbf{Stress placement:} Fixed
\end{styleBody}

\begin{styleBody}
\textbf{Phonetic processes conditioned by stress:} Vowel Reduction, Consonant Allophony in Unstressed Syllables, Consonant Allophony in Stressed Syllables
\end{styleBody}

\begin{styleBody}
\textbf{Differences in phonological properties of stressed and unstressed syllables:} None
\end{styleBody}

\begin{styleBody}
\textbf{Phonetic correlates of stress: }Not described
\end{styleBody}

\begin{styleBody}
\textbf{Vowel reduction processes}
\end{styleBody}

\begin{styleBody}
\textbf{dyo-R1:} High-mid central vowel /[258?]/ is lowered to the quality of English [[259?]] when unstressed (Sapir 1965: 6).
\end{styleBody}

\begin{styleBody}
\textbf{Consonant allophony processes}
\end{styleBody}

\begin{styleBody}
\textbf{dyo-C1: }Velar stops are realized as post velar preceding /u/ (Sapir 1965: 5).
\end{styleBody}

\begin{styleBody}
\textbf{dyo-C2: }A voiceless velar stop is realized as palatal preceding a front vowel (some speakers) (Sapir 1965).
\end{styleBody}

\begin{styleBody}
\textbf{Morphology}
\end{styleBody}

\begin{styleBody}
(adequate texts unavailable)
\end{styleBody}

\clearpage\begin{styleBody}
\textbf{[eus] }\ \ \textbf{\textsc{Basque (Central dialect)}}\textbf{\ \ }isolate (France, Spain)
\end{styleBody}

\begin{styleBody}
\textbf{References consulted: }Hualde (2003), Saltarelli et al. (1988)
\end{styleBody}

\begin{styleBody}
\textbf{Sound inventory}
\end{styleBody}

\begin{styleBody}
\textbf{C phoneme inventory:} /p b t[32A?] d[32A?] c [25F?] k [261?] t[32A?][361?]s[32A?] t[361?]s t[361?][283?] f s[32A?] s [283?] x m n[32A?] [272?] l [28E?] [27E?] r/
\end{styleBody}

\begin{styleBody}
\textbf{N consonant phonemes:} 23
\end{styleBody}

\begin{styleBody}
\textbf{Geminates:} N/A
\end{styleBody}

\begin{styleBody}
\textbf{Voicing contrasts:} Obstruents
\end{styleBody}

\begin{styleBody}
\textbf{Places:} Bilabial, Labiodental, Dental, Alveolar, Palato-alveolar, Palatal, Velar
\end{styleBody}

\begin{styleBody}
\textbf{Manners:} Stop, Affricate, Fricative, Nasal, Flap/Tap, Trill, Lateral approximant
\end{styleBody}

\begin{styleBody}
\textbf{N elaborations:} 2
\end{styleBody}

\begin{styleBody}
\textbf{Elaborations:} Labiodental, Palato-alveolar
\end{styleBody}

\begin{styleBody}
\textbf{V phoneme inventory:} /i e a o u/
\end{styleBody}

\begin{styleBody}
\textbf{N vowel qualities:} 5
\end{styleBody}

\begin{styleBody}
\textbf{Diphthongs or vowel sequences:} None
\end{styleBody}

\begin{styleBody}
\textbf{Contrastive length:} None
\end{styleBody}

\begin{styleBody}
\textbf{Contrastive nasalization:} None
\end{styleBody}

\begin{styleBody}
\textbf{Other contrasts:} N/A
\end{styleBody}

\begin{styleBody}
\textbf{Notes:} /c [25F?]/ are recently phonemic; Saltarelli et al. give these as /t[2B2?] d[2B2?]/. /x/ may be very retracted. Zuberoan dialect also has /y/ and phonemic nasalized vowels.
\end{styleBody}

\begin{styleBody}
\textbf{Syllable structure}
\end{styleBody}

\begin{styleBody}
\textbf{Complexity Category:} Complex
\end{styleBody}

\begin{styleBody}
\textbf{Canonical syllable structure:} (C)(C)V(C)(C)\textbf{ }(Saltarelli et al. 1988: 277-81; Hualde 2003: 33-7)
\end{styleBody}

\begin{styleBody}
\textbf{Size of maximal onset:} 1
\end{styleBody}

\begin{styleBody}
\textbf{Size of maximal coda:} 2
\end{styleBody}

\begin{styleBody}
\textbf{Onset obligatory:} No
\end{styleBody}

\begin{styleBody}
\textbf{Coda obligatory:} No
\end{styleBody}

\begin{styleBody}
\textbf{Vocalic nucleus patterns:} Short vowels
\end{styleBody}

\begin{styleBody}
\textbf{Syllabic consonant patterns:} N/A
\end{styleBody}

\begin{styleBody}
\textbf{Size of maximal word{}-marginal sequences with syllabic obstruents:} N/A
\end{styleBody}

\begin{styleBody}
\textbf{Predictability of syllabic consonants:} N/A
\end{styleBody}

\begin{styleBody}
\textbf{Morphological constituency of maximal syllable margin:} Morpheme-internal (Onset, Coda)
\end{styleBody}

\begin{styleBody}
\textbf{Morphological pattern of syllabic consonants:} N/A
\end{styleBody}

\begin{styleBody}
\textbf{Onset restrictions:} All consonants except for /[27E?] r/ occur as simple onsets. Biconsonantal onset clusters in language are non-native.
\end{styleBody}

\begin{styleBody}
\textbf{Coda restrictions: }All consonants except /b d [261?] p f m x/ occur as simple codas. Complex codas have /x s [283?]/, a liquid, or a nasal as C\textsubscript{1} and a plosive, affricate, or fricative as C\textsubscript{2}. Stops or affricates are not allowed in word-internal codas.
\end{styleBody}

\begin{styleBody}
\textbf{Notes: }Saltarelli et al. state that complex codas occur utterance-finally only; however, Hualde (2003) gives examples of nasal+fricative and liquid+fricative codas occurring word-internally, and a wide range of coda clusters occurring word-finally.
\end{styleBody}

\begin{styleBody}
\textbf{Suprasegmentals}
\end{styleBody}

\begin{styleBody}
\textbf{Tone:} No
\end{styleBody}

\begin{styleBody}
\textbf{Word stress:} Yes
\end{styleBody}

\begin{styleBody}
\textbf{Stress placement:} Fixed
\end{styleBody}

\begin{styleBody}
\textbf{Phonetic processes conditioned by stress:} (None)
\end{styleBody}

\begin{styleBody}
\textbf{Differences in phonological properties of stressed and unstressed syllables:} Nonee
\end{styleBody}

\begin{styleBody}
\textbf{Phonetic correlates of stress: }Pitch (instrumental)
\end{styleBody}

\begin{styleBody}
\textbf{Notes: }The Bizcaian dialects have pitch accent system. Saltarelli et al. describe five different accentual systems for dialects of Basque (1988: 282-3).
\end{styleBody}

\begin{styleBody}
\textbf{Vowel reduction processes}
\end{styleBody}

\begin{styleBody}
\textbf{Notes: }There are processes of unstressed vowel reduction and even deletion of post-tonic vowels in the High Navarrese dialects (Hualde \& Urbina 2003: 56-7).
\end{styleBody}

\begin{styleBody}
\textbf{Consonant allophony processes}
\end{styleBody}

\begin{styleBody}
\textbf{eus-C1: }A palatal lateral approximant is realized as a palatal fricative by some speakers (Hualde 2003: 29).
\end{styleBody}

\begin{styleBody}
\textbf{eus-C2: }Fricatives are realized as voiced preceding a voiced consonant (Hualde 2003: 24).
\end{styleBody}

\begin{styleBody}
\textbf{eus-C3: }A voiced stop may be realized as a fricative or approximant intervocalically (Hualde 2003: 19).
\end{styleBody}

\begin{styleBody}
\textbf{Morphology}
\end{styleBody}

\begin{styleBody}
\textbf{Text:} “Text 4” (Hualde \& Urbina 2003: 906-912)
\end{styleBody}

\begin{styleBody}
\textbf{Synthetic index: }1.6 morphemes/word (462 morphemes, 284 words)
\end{styleBody}

\clearpage\begin{styleBody}
\textbf{[ewe] }\ \ \textbf{\textsc{Ewe}}\textbf{\ \ }Atlantic-Congo, \textit{Volta-Congo} (Ghana, Togo)
\end{styleBody}

\begin{styleBody}
\textbf{References consulted: }Ameka (1991), Duthie (1996), Jalloh (2005), Stahlke (1971)
\end{styleBody}

\begin{styleBody}
\textbf{Sound inventory}
\end{styleBody}

\begin{styleBody}
\textbf{C phoneme inventory:} /p t[32A?] d[32A?] k [261?] k[361?]p [261?][361?]b t[361?]s d[361?]z [278?] $\beta $ f v s z x [266?] m n [272?] ŋ r l w/
\end{styleBody}

\begin{styleBody}
\textbf{N consonant phonemes:} 24
\end{styleBody}

\begin{styleBody}
\textbf{Geminates:} N/A
\end{styleBody}

\begin{styleBody}
\textbf{Voicing contrasts:} Obstruents
\end{styleBody}

\begin{styleBody}
\textbf{Places:} Labial-velar, Bilabial, Labiodental, Dental, Alveolar, Palatal, Velar, Glottal
\end{styleBody}

\begin{styleBody}
\textbf{Manners:} Stop, Affricate, Fricative, Nasal, Trill, Central approximant, Lateral approximant
\end{styleBody}

\begin{styleBody}
\textbf{N elaborations:} 1
\end{styleBody}

\begin{styleBody}
\textbf{Elaborations:} Labiodental
\end{styleBody}

\begin{styleBody}
\textbf{V phoneme inventory:} /i e [25B?] [259?] a [254?] o u \~{i} \~{e} [25B?] [259?] \~{a} [254?] \~{o} \~{u}/
\end{styleBody}

\begin{styleBody}
\textbf{N vowel qualities:} 8
\end{styleBody}

\begin{styleBody}
\textbf{Diphthongs or vowel sequences:} Vowel sequences /uu aa ao oo/
\end{styleBody}

\begin{styleBody}
\textbf{Contrastive length:} None
\end{styleBody}

\begin{styleBody}
\textbf{Contrastive nasalization:} All
\end{styleBody}

\begin{styleBody}
\textbf{Other contrasts:} N/A
\end{styleBody}

\begin{styleBody}
\textbf{Notes:} Here I’ve selected the most common/generally distributed allophone as the phoneme name, but the following pairs are in complementary distribution (phoneme label listed first): [ŋ]\~{}[[263?]], [m]\~{}[b], [n]\~{}[d], [[272?]]\~{}[j], [l]\~{}[\~{l}], [t[361?]s]\~{}[t[361?][283?]], [d[361?]z]\~{}[d[361?][292?]], and [w]\~{}[[263?]]. Duthie (1996: 10-18) gives description of allophonic variation. /r/ occurs only as C\textsubscript{2} in an onset cluster. /e/ now merging with /[25B?]/ (Duthie 1996: 19).
\end{styleBody}

\begin{styleBody}
\textbf{Syllable structure}
\end{styleBody}

\begin{styleBody}
\textbf{Complexity Category:} Moderately Complex
\end{styleBody}

\begin{styleBody}
\textbf{Canonical syllable structure:} (C)(C)V(C)\textbf{ }(Ameka 1991: 38-9)
\end{styleBody}

\begin{styleBody}
\textbf{Size of maximal onset:} 2
\end{styleBody}

\begin{styleBody}
\textbf{Size of maximal coda:} 1
\end{styleBody}

\begin{styleBody}
\textbf{Onset obligatory:} No
\end{styleBody}

\begin{styleBody}
\textbf{Coda obligatory:} No
\end{styleBody}

\begin{styleBody}
\textbf{Vocalic nucleus patterns:} Short vowels, Vowel sequences
\end{styleBody}

\begin{styleBody}
\textbf{Syllabic consonant patterns:} Nasal
\end{styleBody}

\begin{styleBody}
\textbf{Size of maximal word{}-marginal sequences with syllabic obstruents:} N/A
\end{styleBody}

\begin{styleBody}
\textbf{Predictability of syllabic consonants:} Phonemic
\end{styleBody}

\begin{styleBody}
\textbf{Morphological constituency of maximal syllable margin:} Morpheme-internal (Onset)
\end{styleBody}

\begin{styleBody}
\textbf{Morphological pattern of syllabic consonants:} Both
\end{styleBody}

\begin{styleBody}
\textbf{Onset restrictions: }C\textsubscript{1} may be any consonant except /r/. C\textsubscript{2} may be /l r w/ or [j], allophone of palatal nasal.
\end{styleBody}

\begin{styleBody}
\textbf{Coda restrictions: }Nasals only.
\end{styleBody}

\begin{styleBody}
\textbf{Notes: }Sequences such as /ŋk/ in \textit{ŋkeke} ‘day’ analyzed a belonging to different syllables, with [ŋ] being syllabic (1991: 39).
\end{styleBody}

\begin{styleBody}
\textbf{Suprasegmentals}
\end{styleBody}

\begin{styleBody}
\textbf{Tone:} Yes
\end{styleBody}

\begin{styleBody}
\textbf{Word stress:} Not reported
\end{styleBody}

\begin{styleBody}
\textbf{Vowel reduction processes}
\end{styleBody}

\begin{styleBody}
(none reported)
\end{styleBody}

\begin{styleBody}
\textbf{Consonant allophony processes}
\end{styleBody}

\begin{styleBody}
\textbf{ewe-C1: }Alveolar affricates and fricatives are realized as palato-alveolar preceding /i/ (Jalloh 2005: 9).
\end{styleBody}

\begin{styleBody}
\textbf{Morphology}
\end{styleBody}

\begin{styleBody}
(adequate texts unavailable)
\end{styleBody}

\clearpage\begin{styleBody}
\textbf{[fvr] }\ \ \textbf{\textsc{Fur}}\textbf{\ \ }Furan (Sudan)
\end{styleBody}

\begin{styleBody}
\textbf{References consulted: }Jakobi (1990), Kutsch Lojenga \& Waag (2004), Noel (2008)
\end{styleBody}

\begin{styleBody}
\textbf{Sound inventory}
\end{styleBody}

\begin{styleBody}
\textbf{C phoneme inventory:} /b t[32A?] d[32A?] [25F?] k [261?] f s m n [272?] ŋ l [27E?] w j/
\end{styleBody}

\begin{styleBody}
\textbf{N consonant phonemes:} 16
\end{styleBody}

\begin{styleBody}
\textbf{Geminates:} N/A
\end{styleBody}

\begin{styleBody}
\textbf{Voicing contrasts:} Obstruents
\end{styleBody}

\begin{styleBody}
\textbf{Places:} Bilabial, Labiodental, Dental, Alveolar, Palatal, Velar
\end{styleBody}

\begin{styleBody}
\textbf{Manners:} Stop, Fricative, Nasal, Flap/Tap, Central approximant, Lateral approximant
\end{styleBody}

\begin{styleBody}
\textbf{N elaborations:} 1
\end{styleBody}

\begin{styleBody}
\textbf{Elaborations:} Labiodental
\end{styleBody}

\begin{styleBody}
\textbf{V phoneme inventory:} /i [26A?] [25B?] [259?] a [254?] [28A?] u/
\end{styleBody}

\begin{styleBody}
\textbf{N vowel qualities:} 8
\end{styleBody}

\begin{styleBody}
\textbf{Diphthongs or vowel sequences:} Vowel sequences /ii [26A?][26A?] [25B?][25B?] aa [254?][254?] [28A?][28A?] uu [26A?]a i[254?] i[25B?] a[26A?] ai [28A?]a u[254?] u[25B?]/
\end{styleBody}

\begin{styleBody}
\textbf{Contrastive length:} None
\end{styleBody}

\begin{styleBody}
\textbf{Contrastive nasalization:} None
\end{styleBody}

\begin{styleBody}
\textbf{Other contrasts:} N/A
\end{styleBody}

\begin{styleBody}
\textbf{Notes:} /h/ occurs in only two lexical items and is variable in one of them, so I have omitted it here. /f/ is classified by Jakobi as voiceless bilabial stop by phonological criteria, but its actual realization is [f] in most contexts. [j] alternates with [z]. Noel give /d[361?][292?]/ instead of /[25F?]/. Vowel system is from Kutsch Lojenga and Waag (2004); Jakobi and Noel each give 5 vowels, /a [25B?] i [254?] u/. Long vowels are analyzed as sequences by Jakobi.
\end{styleBody}

\begin{styleBody}
\textbf{Syllable structure}
\end{styleBody}

\begin{styleBody}
\textbf{Complexity Category:} Moderately Complex
\end{styleBody}

\begin{styleBody}
\textbf{Canonical syllable structure:} (C)V(C) (Jakobi 1990: 53-8)
\end{styleBody}

\begin{styleBody}
\textbf{Size of maximal onset:} 1
\end{styleBody}

\begin{styleBody}
\textbf{Size of maximal coda:} 1
\end{styleBody}

\begin{styleBody}
\textbf{Onset obligatory: }No
\end{styleBody}

\begin{styleBody}
\textbf{Coda obligatory:} No
\end{styleBody}

\begin{styleBody}
\textbf{Vocalic nucleus patterns:} Short vowels
\end{styleBody}

\begin{styleBody}
\textbf{Syllabic consonant patterns:} N/A
\end{styleBody}

\begin{styleBody}
\textbf{Size of maximal word{}-marginal sequences with syllabic obstruents:} N/A
\end{styleBody}

\begin{styleBody}
\textbf{Predictability of syllabic consonants:} N/A
\end{styleBody}

\begin{styleBody}
\textbf{Morphological constituency of maximal syllable margin:} N/A
\end{styleBody}

\begin{styleBody}
\textbf{Morphological pattern of syllabic consonants:} N/A
\end{styleBody}

\begin{styleBody}
\textbf{Onset restrictions: }All consonants occur. Onset [j] occurs only as allophone of /i/ (as analyzed by author).
\end{styleBody}

\begin{styleBody}
\textbf{Coda restrictions: }All consonants except voiced obstruents /b d [25F?] [261?]/ and [z] (allophone of /j/) may occur.
\end{styleBody}

\begin{styleBody}
\textbf{Suprasegmentals}
\end{styleBody}

\begin{styleBody}
\textbf{Tone:} Yes
\end{styleBody}

\begin{styleBody}
\textbf{Word stress:} Not reported
\end{styleBody}

\begin{styleBody}
\textbf{Vowel reduction processes}
\end{styleBody}

\begin{styleBody}
\textbf{fvr-R1:} In 3-syllable words with the structure (C\textsubscript{1})V\textsubscript{1}C\textsubscript{2}V\textsubscript{2}C\textsubscript{3}V\textsubscript{3}, where C\textsubscript{2} is /l/ or /r/, C\textsubscript{3} is /l/, /r/, or nasal /m n [272?] ŋ/, and V\textsubscript{1} and V\textsubscript{2} are identical, V\textsubscript{2} may optionally be deleted (Jakobi 1990: 60-61).
\end{styleBody}

\begin{styleBody}
\textbf{Consonant allophony processes}
\end{styleBody}

\begin{styleBody}
\textbf{fvr-C1: }A palatal glide is realized as [z] word-initially (Jakobi 1990: 19).
\end{styleBody}

\begin{styleBody}
\textbf{Morphology}
\end{styleBody}

\begin{styleBody}
\textbf{Text: }“A Fur text” (Jakobi 1990: 125-127)
\end{styleBody}

\begin{styleBody}
\textbf{Synthetic index: }1.2 morphemes/word (234 morphemes, 202 words)
\end{styleBody}

\clearpage\begin{styleBody}
\textbf{[grj] }\ \ \textbf{\textsc{Southern Grebo}}\textbf{\ \ }Atlantic-Congo, \textit{Volta-Congo }(Liberia)
\end{styleBody}

\begin{styleBody}
\textbf{References consulted: }Innes (1966, 1981), Newman (1986)
\end{styleBody}

\begin{styleBody}
\textbf{Sound inventory}
\end{styleBody}

\begin{styleBody}
\textbf{C phoneme inventory:} /p b t d c [25F?] k [261?] k[361?]p [261?][361?]b f s h m[325?] m n[325?] n [272?] ŋ ŋ[361?]m l[325?] l w[325?] w j/
\end{styleBody}

\begin{styleBody}
\textbf{N consonant phonemes:} 25
\end{styleBody}

\begin{styleBody}
\textbf{Geminates:} N/A
\end{styleBody}

\begin{styleBody}
\textbf{Voicing contrasts:} Obstruent, Sonorants
\end{styleBody}

\begin{styleBody}
\textbf{Places:} Labial-velar, Bilabial, Dental/Alveolar, Palatal, Velar, Glottal
\end{styleBody}

\begin{styleBody}
\textbf{Manners:} Stop, Fricative, Nasal, Lateral approximant, Central approximant
\end{styleBody}

\begin{styleBody}
\textbf{N elaborations: }2
\end{styleBody}

\begin{styleBody}
\textbf{Elaborations:} Devoiced sonorants, Labiodental
\end{styleBody}

\begin{styleBody}
\textbf{V phoneme inventory:} /i [26A?] e [25B?] a [254?] o [28A?] u \~{i} \~{e} [25B?] \~{a} [254?] \~{o} \~{u}/
\end{styleBody}

\begin{styleBody}
\textbf{N vowel qualities: }9
\end{styleBody}

\begin{styleBody}
\textbf{Diphthongs or vowel sequences: }None
\end{styleBody}

\begin{styleBody}
\textbf{Contrastive length: }None
\end{styleBody}

\begin{styleBody}
\textbf{Contrastive nasalization: }Some
\end{styleBody}

\begin{styleBody}
\textbf{Other contrasts: }N/A
\end{styleBody}

\begin{styleBody}
\textbf{Notes:} Voiced nasal stops may be better analyzed as allophones of voiced stops (Newman 1986: 176).
\end{styleBody}

\begin{styleBody}
\textbf{Syllable structure}
\end{styleBody}

\begin{styleBody}
\textbf{Complexity Category:} Simple
\end{styleBody}

\begin{styleBody}
\textbf{Canonical syllable structure:} (C)V\textbf{ }(Innes 1981: 130, 1966: 15-16)
\end{styleBody}

\begin{styleBody}
\textbf{Size of maximal onset:} 1
\end{styleBody}

\begin{styleBody}
\textbf{Size of maximal coda:} N/A
\end{styleBody}

\begin{styleBody}
\textbf{Onset obligatory:} No
\end{styleBody}

\begin{styleBody}
\textbf{Coda obligatory:} N/A
\end{styleBody}

\begin{styleBody}
\textbf{Vocalic nucleus patterns:} Short vowels
\end{styleBody}

\begin{styleBody}
\textbf{Syllabic consonant patterns:} N/A
\end{styleBody}

\begin{styleBody}
\textbf{Size of maximal word{}-marginal sequences with syllabic obstruents:} N/A
\end{styleBody}

\begin{styleBody}
\textbf{Predictability of syllabic consonants:} N/A
\end{styleBody}

\begin{styleBody}
\textbf{Morphological constituency of maximal syllable margin:} N/A
\end{styleBody}

\begin{styleBody}
\textbf{Morphological pattern of syllabic consonants:} N/A
\end{styleBody}

\begin{styleBody}
\textbf{Onset restrictions: }All consonants occur.
\end{styleBody}

\begin{styleBody}
\textbf{Notes: }CCV shapes occur in loans; C\textsubscript{2} is always [l] or [w] (Innes 1981: 130, 1966: 15-16). Innes reports that some words of the form CVCV contract to CCV in rapid speech; this is when the medial consonant is /d/ or /n/, and it results in C+[l] clusters (e.g., \textit{pone} {\textgreater} \textit{pl\={e}} ‘rat’, 1981: 130). Newman reports that there are many such words for which \textbf{only} ClV forms occur and there are no corresponding CVCV forms, but he gives examples which Innes lists alternating forms for. Because Innes provides evidence, I adopt his analysis. Newman also reports that C\textsubscript{2} in such clusters is generally pronounced as an ‘r-like tap’ rather than a lateral (1986: 177).
\end{styleBody}

\begin{styleBody}
\textbf{Suprasegmentals}
\end{styleBody}

\begin{styleBody}
\textbf{Tone:} Yes
\end{styleBody}

\begin{styleBody}
\textbf{Word stress:} Disagreement (Innes reports stress; Newman 1986 reports he could not verify this)
\end{styleBody}

\begin{styleBody}
\textbf{Vowel reduction processes}
\end{styleBody}

\begin{styleBody}
\textbf{grj-R1:} In words of the form CVCV in rapid speech, V\textsubscript{1} is deleted if C\textsubscript{1} is a non-alveolar stop or /f, m, m[325?], ŋ/ and C\textsubscript{2} is /d/ or /n/; C\textsubscript{2} is realized phonetically as [l] when this process occurs (Innes 1981: 130, 1966: 15-16).
\end{styleBody}

\begin{styleBody}
\textbf{Consonant allophony processes}
\end{styleBody}

\begin{styleBody}
(none reported)
\end{styleBody}

\begin{styleBody}
\textbf{Morphology}
\end{styleBody}

\begin{styleBody}
(adequate texts unavailable)
\end{styleBody}

\clearpage\begin{styleBody}
\textbf{[hts] }\ \ \textbf{\textsc{Hadza}}\textbf{\ \ }isolate (Tanzania)
\end{styleBody}

\begin{styleBody}
\textbf{References consulted: }Kirk Miller (p.c.), Sands (2013), Sands et al. (2012), Bonny Sands (p.c.)
\end{styleBody}

\begin{styleBody}
\textbf{Sound inventory}
\end{styleBody}

\begin{styleBody}
\textbf{C phoneme inventory:} /p[2B0?] p b t[2B0?] t d k[2B0?] k [261?] k[2B0?][2B7?] k[2B7?] [261?][2B7?] [294?] p’ k’ k’[2B7?] k[1C0?] k[1C3?] k[1C1?] m n [272?] ŋ ŋ[2B7?] ŋ[325?][1C0?]’ ŋ[1C0?] ŋ[325?][1C3?]’ ŋ[1C3?] ŋ[325?][1C1?]’ ŋ[1C1?] \textsuperscript{m}p[2B0?] \textsuperscript{m}b \textsuperscript{n}t[2B0?] \textsuperscript{n}d \textsuperscript{ŋ}k[2B0?] \textsuperscript{ŋ}[261?] \textsuperscript{n}t[361?]s \textsuperscript{n}d[361?]z \textsuperscript{[272?]}d[361?][292?] t[361?]s d[361?]z t[361?][283?] t[361?][28E?][325?] d[361?][292?] t[361?]s’ t[361?][283?]’ t[361?][28E?][325?]’ f s [26C?] [283?] l j w [266?]/
\end{styleBody}

\begin{styleBody}
\textbf{N consonant phonemes:} 55
\end{styleBody}

\begin{styleBody}
\textbf{Geminates:} N/A
\end{styleBody}

\begin{styleBody}
\textbf{Voicing contrasts:} Obstruents, Sonorants
\end{styleBody}

\begin{styleBody}
\textbf{Places:} Bilabial, Labiodental, Dental, Alveolar, Palato-alveolar, Velar, Glottal
\end{styleBody}

\begin{styleBody}
\textbf{Manners:} Stop, Affricate, Fricative, Nasal, Central approximant, Lateral affricate, Lateral fricative, Lateral approximant
\end{styleBody}

\begin{styleBody}
\textbf{N elaborations:} 10
\end{styleBody}

\begin{styleBody}
\textbf{Elaborations:} Voiced fricatives/affricates, Devoiced sonorants, Prenasalization, Post-aspiration, Lateral release, Ejective, Click, Labiodental, Palato-alveolar, Labialization
\end{styleBody}

\begin{styleBody}
\textbf{V phoneme inventory:} /i e a o u \~{i} \~{u}/
\end{styleBody}

\begin{styleBody}
\textbf{N vowel qualities: }5
\end{styleBody}

\begin{styleBody}
\textbf{Diphthongs or vowel sequences: }None
\end{styleBody}

\begin{styleBody}
\textbf{Contrastive length: }None
\end{styleBody}

\begin{styleBody}
\textbf{Contrastive nasalization: }Some
\end{styleBody}

\begin{styleBody}
\textbf{Other contrasts: }N/A
\end{styleBody}

\begin{styleBody}
\textbf{Notes:} All voiced obstruents borrowed except for /b/, medial prenasalized plosives, and nasals apart from /m/ and /n/; however, sources unknown (Kirk Miller, p.c.). If we take this analysis to be accurate, then language has 49 consonant phonemes instead of 55. Vowel nasalization is marginally contrastive (Sands 2013: 38).
\end{styleBody}

\begin{styleBody}
\textbf{Syllable structure}
\end{styleBody}

\begin{styleBody}
\textbf{Complexity Category:} Simple
\end{styleBody}

\begin{styleBody}
\textbf{Canonical syllable structure:} (C)V\textbf{ }(Tucker et al. 1977: 309; Sands et al. 1996; Sands 2013)
\end{styleBody}

\begin{styleBody}
\textbf{Size of maximal onset:} 1
\end{styleBody}

\begin{styleBody}
\textbf{Size of maximal coda:} N/A
\end{styleBody}

\begin{styleBody}
\textbf{Onset obligatory:} Yes
\end{styleBody}

\begin{styleBody}
\textbf{Coda obligatory:} N/A
\end{styleBody}

\begin{styleBody}
\textbf{Vocalic nucleus patterns:} Short vowels
\end{styleBody}

\begin{styleBody}
\textbf{Syllabic consonant patterns:} N/A
\end{styleBody}

\begin{styleBody}
\textbf{Size of maximal word{}-marginal sequences with syllabic obstruents:} N/A
\end{styleBody}

\begin{styleBody}
\textbf{Predictability of syllabic consonants:} N/A
\end{styleBody}

\begin{styleBody}
\textbf{Morphological constituency of maximal syllable margin:} N/A
\end{styleBody}

\begin{styleBody}
\textbf{Morphological pattern of syllabic consonants:} N/A
\end{styleBody}

\begin{styleBody}
\textbf{Onset restrictions: }All consonants occur.
\end{styleBody}

\begin{styleBody}
\textbf{Notes: }Sands (2013) analyzes syllable structure as CV, where C includes prenasalized obstruents and V may be a nasal vowel. Kirk Miller (p.c.) analyzes syllable structure as CV(N), without nasal vowels. Miller also notes that onsets are obligatory, with predictable /h/ occurring in otherwise vowel-initial syllables.
\end{styleBody}

\begin{styleBody}
\textbf{Suprasegmentals}
\end{styleBody}

\begin{styleBody}
\textbf{Tone:} Yes
\end{styleBody}

\begin{styleBody}
\textbf{Word stress:} No
\end{styleBody}

\begin{styleBody}
\textbf{Notes:} Prominence may shift syllables according to context. Thus this may be a pitch accent language, but there is not enough information to characterize the stress pattern.
\end{styleBody}

\begin{styleBody}
\textbf{Vowel reduction processes}
\end{styleBody}

\begin{styleBody}
\textbf{hts-R1:} Final vowels frequently become voiceless, especially when preceded by /[294?]/ or other voiceless stops. This devoicing can extend to penultimate vowels, such that the final two syllables of a word in utterance-final position can become whispered. (Sands et al. 1996: 177; Tucker et al. 1977: 309)
\end{styleBody}

\begin{styleBody}
\textbf{Consonant allophony processes}
\end{styleBody}

\begin{styleBody}
\textbf{hts-C1: }An ejective velar stop is realized as an affricate [kx’] by some speakers (Sands 2013: 41).
\end{styleBody}

\begin{styleBody}
\textbf{hts-C2: }An alveolar lateral approximant is realized as a flap intervocalically (Sands 2013: 41).
\end{styleBody}

\begin{styleBody}
\textbf{Morphology}
\end{styleBody}

\begin{styleBody}
(adequate texts unavailable)
\end{styleBody}

\clearpage\begin{styleBody}
\textbf{[huu] }\ \ \textbf{\textsc{Murui Huitoto\ \ }}\textbf{\ \ }Huitotoan, \textit{Nuclear Witotoan} (Colombia, Peru)
\end{styleBody}

\begin{styleBody}
\textbf{References consulted: }Wojtylak (2017), Katarzyna Wojtylak (p.c.)
\end{styleBody}

\begin{styleBody}
\textbf{Sound inventory}
\end{styleBody}

\begin{styleBody}
\textbf{C phoneme inventory:} /p b t d k [261?] t[361?][283?] d[361?][292?] [278?] $\beta $ $\theta $ h m n [272?] [27E?]/
\end{styleBody}

\begin{styleBody}
\textbf{N consonant phonemes:} 16
\end{styleBody}

\begin{styleBody}
\textbf{Geminates:} N/A
\end{styleBody}

\begin{styleBody}
\textbf{Voicing contrasts: }Obstruents
\end{styleBody}

\begin{styleBody}
\textbf{Places: }Bilabial, Dental, Alveolar, Palato-alveolar, Palatal, Velar, Glottal
\end{styleBody}

\begin{styleBody}
\textbf{Manners:} Stop, Affricate, Fricative, Nasal, Flap/tap
\end{styleBody}

\begin{styleBody}
\textbf{N elaborations:} 2
\end{styleBody}

\begin{styleBody}
\textbf{Elaborations:} Voiced fricatives/affricates, Palato-alveolar
\end{styleBody}

\begin{styleBody}
\textbf{V phoneme inventory:} /i [25B?] a o [26F?] u i[2D0?] [25B?][2D0?] a[2D0?] o[2D0?] [26F?][2D0?] u[2D0?]/
\end{styleBody}

\begin{styleBody}
\textbf{N vowel qualities:} 6
\end{styleBody}

\begin{styleBody}
\textbf{Diphthongs or vowel sequences:} Diphthongs /ai [25B?]i ui a[26F?] o[26F?]/
\end{styleBody}

\begin{styleBody}
\textbf{Contrastive length:} All
\end{styleBody}

\begin{styleBody}
\textbf{Contrastive nasalization:} None
\end{styleBody}

\begin{styleBody}
\textbf{Other contrasts:} N/A
\end{styleBody}

\begin{styleBody}
\textbf{Notes:} /p/ occurs only marginally. Wojtylak labels /f v/ as labiodental but gives their typical realizations as [[278?] $\beta $]. Approximants [w j [270?]] occur but are not contrastive, occurring as allophones of vowels (2017: 75). /s/ occurs in speech of younger people bilingual in Spanish. Vowel sequences also occur, and may be realized as phonetic diphthongs, but they are transcribed as belonging to separate syllables and are described as being realized as such in slow/normal speech (2017: 90-93).
\end{styleBody}

\begin{styleBody}
\textbf{Syllable structure}
\end{styleBody}

\begin{styleBody}
\textbf{Complexity Category:} Simple
\end{styleBody}

\begin{styleBody}
\textbf{Canonical syllable structure:} (C)V\textbf{ }(Wojtylak 2017: 93-95)
\end{styleBody}

\begin{styleBody}
\textbf{Size of maximal onset:} 1
\end{styleBody}

\begin{styleBody}
\textbf{Size of maximal coda: }N/A
\end{styleBody}

\begin{styleBody}
\textbf{Onset obligatory:} No
\end{styleBody}

\begin{styleBody}
\textbf{Coda obligatory:} N/A
\end{styleBody}

\begin{styleBody}
\textbf{Vocalic nucleus patterns: }Short vowels, Long vowels, Diphthongs
\end{styleBody}

\begin{styleBody}
\textbf{Syllabic consonant patterns:} N/A
\end{styleBody}

\begin{styleBody}
\textbf{Size of maximal word{}-marginal sequences with syllabic obstruents:} N/A
\end{styleBody}

\begin{styleBody}
\textbf{Predictability of syllabic consonants:} N/A
\end{styleBody}

\begin{styleBody}
\textbf{Morphological constituency of maximal syllable margin:} N/A
\end{styleBody}

\begin{styleBody}
\textbf{Morphological pattern of syllabic consonants:} N/A
\end{styleBody}

\begin{styleBody}
\textbf{Onset restrictions:} None.
\end{styleBody}

\begin{styleBody}
\textbf{Coda restrictions:} N/A
\end{styleBody}

\begin{styleBody}
\textbf{Notes:} /[294?]/ may occur as a coda in restricted contexts.
\end{styleBody}

\begin{styleBody}
\textbf{Suprasegmentals}
\end{styleBody}

\begin{styleBody}
\textbf{Tone:} No
\end{styleBody}

\begin{styleBody}
\textbf{Word stress: }Yes
\end{styleBody}

\begin{styleBody}
\textbf{Stress placement:} Fixed
\end{styleBody}

\begin{styleBody}
\textbf{Phonetic processes conditioned by stress:}
\end{styleBody}

\begin{styleBody}
\textbf{Differences in phonological properties of stressed and unstressed syllables:} Vowel Length Contrasts
\end{styleBody}

\begin{styleBody}
\textbf{Phonetic correlates of stress: }Pitch (impressionistic)
\end{styleBody}

\begin{styleBody}
\textbf{Vowel reduction processes}
\end{styleBody}

\begin{styleBody}
(none reported)
\end{styleBody}

\begin{styleBody}
\textbf{Consonant allophony processes}
\end{styleBody}

\begin{styleBody}
(none reported)
\end{styleBody}

\begin{styleBody}
\textbf{Morphology}
\end{styleBody}

\begin{styleBody}
\textbf{Text:} “Jiyakino — The Murui Origin Myth” (first 8 pages, Wojtylak 2017: 578-585)
\end{styleBody}

\begin{styleBody}
\textbf{Synthetic index: }1.93 morphemes/word (729 morphemes, 377 words)
\end{styleBody}

\clearpage\begin{styleBody}
\textbf{[iii] }\ \ \textbf{\textsc{Sichuan Yi}}\textbf{\ \ }Sino-Tibetan, \textit{Burmo-Qiangic} (China)
\end{styleBody}

\begin{styleBody}
\textbf{References consulted: }Gerner (2013), Maoji (1997), Merrifield (2012)
\end{styleBody}

\begin{styleBody}
\textbf{Sound inventory}
\end{styleBody}

\begin{styleBody}
\textbf{C phoneme inventory:} /p p[2B0?] b \textsuperscript{m}b t t[2B0?] d \textsuperscript{n}d k k[2B0?] [261?] \textsuperscript{ŋ}[261?] t[361?]s t[361?]s[2B0?] d[361?]z \textsuperscript{n}d[361?]z [288?][361?][282?] [288?][361?][282?][2B0?] [256?][361?][290?] \textsuperscript{[273?]}[256?][361?][290?] t[361?][255?] t[361?][255?][2B0?] d[361?][291?] \textsuperscript{[272?]}d[361?][291?] f v s z [282?] [290?] [255?] [291?] x [263?] h m[325?] m n[325?] n [272?] ŋ l[325?] l/
\end{styleBody}

\begin{styleBody}
\textbf{N consonant phonemes:} 43
\end{styleBody}

\begin{styleBody}
\textbf{Geminates:} N/A
\end{styleBody}

\begin{styleBody}
\textbf{Voicing contrasts:} Obstruents, Sonorants
\end{styleBody}

\begin{styleBody}
\textbf{Places:} Bilabial, Labiodental, Dental, Alveolar, Retroflex, Alveolo-palatal, Velar, Glottal
\end{styleBody}

\begin{styleBody}
\textbf{Manners:} Stop, Affricate, Fricative, Nasal, Lateral approximant
\end{styleBody}

\begin{styleBody}
\textbf{N elaborations:} 6
\end{styleBody}

\begin{styleBody}
\textbf{Elaborations:} Voiced fricatives/affricates, Devoiced sonorants, Prenasalization, Post-aspiration, Labiodental, Retroflex
\end{styleBody}

\begin{styleBody}
\textbf{V phoneme inventory:} /i [25B?] [268?] a [254?] o [26F?] u [268?][330?] u[330?]/
\end{styleBody}

\begin{styleBody}
\textbf{N vowel qualities:} 8
\end{styleBody}

\begin{styleBody}
\textbf{Diphthongs or vowel sequences:} None
\end{styleBody}

\begin{styleBody}
\textbf{Contrastive length:} None
\end{styleBody}

\begin{styleBody}
\textbf{Contrastive nasalization:} None
\end{styleBody}

\begin{styleBody}
\textbf{Other contrasts:} Creaky voice (some)
\end{styleBody}

\begin{styleBody}
\textbf{Notes:} /[272?]/ represents alveolo-palatal nasal. Pei-Shan dialect additionally has a series of palato-alveolars derived from velars: /t[361?][283?] t[361?][283?][2B0?] d[361?][292?] \textsuperscript{[272?]}d[361?][292?] [283?] [292?]/ (Maoji 1997: 68-9). /[268?] u/ have contrastive creaky voice counterparts.
\end{styleBody}

\begin{styleBody}
\textbf{Syllable structure}
\end{styleBody}

\begin{styleBody}
\textbf{Complexity Category:} Simple
\end{styleBody}

\begin{styleBody}
\textbf{Canonical syllable structure:} (C)V\textbf{ }(Gerner 2013: 30-2)
\end{styleBody}

\begin{styleBody}
\textbf{Size of maximal onset:} 1
\end{styleBody}

\begin{styleBody}
\textbf{Size of maximal coda:} N/A
\end{styleBody}

\begin{styleBody}
\textbf{Onset obligatory:} No
\end{styleBody}

\begin{styleBody}
\textbf{Coda obligatory:} N/A
\end{styleBody}

\begin{styleBody}
\textbf{Vocalic nucleus patterns:} Short vowels
\end{styleBody}

\begin{styleBody}
\textbf{Syllabic consonant patterns:} Nasal, Liquid
\end{styleBody}

\begin{styleBody}
\textbf{Size of maximal word{}-marginal sequences with syllabic obstruents:} N/A
\end{styleBody}

\begin{styleBody}
\textbf{Predictability of syllabic consonants:} Varies with CV sequence
\end{styleBody}

\begin{styleBody}
\textbf{Morphological constituency of maximal syllable margin:} N/A
\end{styleBody}

\begin{styleBody}
\textbf{Morphological pattern of syllabic consonants:} N/A
\end{styleBody}

\begin{styleBody}
\textbf{Onset restrictions: }All consonants occur.
\end{styleBody}

\begin{styleBody}
\textbf{Suprasegmentals}
\end{styleBody}

\begin{styleBody}
\textbf{Tone: }Yes
\end{styleBody}

\begin{styleBody}
\textbf{Word stress: }Not reported
\end{styleBody}

\begin{styleBody}
\textbf{Vowel reduction processes}
\end{styleBody}

\begin{styleBody}
\textbf{iii-R1:} When nasals and lateral approximants co-occur with central vowel /[268?]/, these sequences are in free variation with syllabic consonants (Gerner 2013: 31).
\end{styleBody}

\begin{styleBody}
\textbf{Consonant allophony processes}
\end{styleBody}

\begin{styleBody}
(none reported)
\end{styleBody}

\begin{styleBody}
\textbf{Morphology}
\end{styleBody}

\begin{styleBody}
\textbf{Text:} “Why do men have their livestock stay close to home?” (Gerner 2013: 525-530)
\end{styleBody}

\begin{styleBody}
\textbf{Synthetic index: }1.0 morphemes/word (465 morphemes, 455 words)
\end{styleBody}

\clearpage\begin{styleBody}
\textbf{[itl] }\ \ \textbf{\textsc{Itelmen}}\textbf{\ \ }Chukotko-Kamchatkan (Russia)
\end{styleBody}

\begin{styleBody}
\textbf{References consulted:} Bobaljik (2006), Jonathan Bobaljik (p.c.), Georg \& Volodin (1999), Volodin (1976), Volodin \& Zhukova (1968)
\end{styleBody}

\begin{styleBody}
\textbf{Sound inventory}
\end{styleBody}

\begin{styleBody}
\textbf{C phoneme inventory:} /p t k q [294?] p’ t’ k’ q’ t[361?][283?] t[361?][283?]’ [278?] $\beta $ s z [26C?] x $\chi $ m n ŋ l j/
\end{styleBody}

\begin{styleBody}
\textbf{N consonant phonemes:} 23
\end{styleBody}

\begin{styleBody}
\textbf{Geminates:} N/A
\end{styleBody}

\begin{styleBody}
\textbf{Voicing contrasts:} Obstruents
\end{styleBody}

\begin{styleBody}
\textbf{Places:} Bilabial, Alveolar, Palato-alveolar, Velar, Uvular, Glottal
\end{styleBody}

\begin{styleBody}
\textbf{Manners:} Stop, Affricate, Fricative, Nasal, Central approximant, Lateral fricative, Lateral approximant
\end{styleBody}

\begin{styleBody}
\textbf{N elaborations:} 4
\end{styleBody}

\begin{styleBody}
\textbf{Elaborations:} Voiced fricatives/affricates, Ejective, Palato-alveolar, Uvular
\end{styleBody}

\begin{styleBody}
\textbf{V phoneme inventory:} /i e [259?] a o u/
\end{styleBody}

\begin{styleBody}
\textbf{N vowel qualities:} 6
\end{styleBody}

\begin{styleBody}
\textbf{Diphthongs or vowel sequences:} None
\end{styleBody}

\begin{styleBody}
\textbf{Contrastive length:} None
\end{styleBody}

\begin{styleBody}
\textbf{Contrastive nasalization:} None
\end{styleBody}

\begin{styleBody}
\textbf{Other contrasts:} N/A
\end{styleBody}

\begin{styleBody}
\textbf{Notes:} /r [272?] [28E?]/ occur in Russian, Koryak loans. Some sources have /[294?]/ as a suprasegmental phenomenon, but Georg \& Volodin consider it a segment. 
\end{styleBody}

\begin{styleBody}
\textbf{Syllable structure}
\end{styleBody}

\begin{styleBody}
\textbf{Complexity Category:} Highly Complex
\end{styleBody}

\begin{styleBody}
\textbf{Canonical syllable structure:} (C)(C)(C)(C)(C)(C)(C)V(C)(C)(C)(C)(C)\textbf{ }(Georg \& Volodin 1999: 38-44)
\end{styleBody}

\begin{styleBody}
\textbf{Size of maximal onset:} 7
\end{styleBody}

\begin{styleBody}
\textbf{Size of maximal coda:} 5
\end{styleBody}

\begin{styleBody}
\textbf{Onset obligatory:} No
\end{styleBody}

\begin{styleBody}
\textbf{Coda obligatory:} No
\end{styleBody}

\begin{styleBody}
\textbf{Vocalic nucleus patterns:} Short vowels
\end{styleBody}

\begin{styleBody}
\textbf{Syllabic consonant patterns:} Nasal, Liquid
\end{styleBody}

\begin{styleBody}
\textbf{Size of maximal word{}-marginal sequences with syllabic obstruents:} N/A
\end{styleBody}

\begin{styleBody}
\textbf{Predictability of syllabic consonants:} Predictable from word/consonantal context
\end{styleBody}

\begin{styleBody}
\textbf{Morphological constituency of maximal syllable margin:} Morphologically Complex (Onset, Coda)
\end{styleBody}

\begin{styleBody}
\textbf{Morphological pattern of syllabic consonants:} Lexical items (Liquid), Both (Nasal)
\end{styleBody}

\begin{styleBody}
\textbf{Onset restrictions: }Apparently all consonants may occur as simple onsets. It seems almost any biconsonantal onset may occur. In triconsonantal onsets, generally any consonant may be added to a permissible biconsonantal onset, so long as one of the consonants is a voiceless stop or sonorant, but consecutive sequences of three voiceless stops, fricatives, or sonants are not allowed within lexical morphemes. There are many examples of 4-consonant onsets, which are combinations of two permissible biconsonantal onsets. Examples include /ttxn, ksxw, ktxl/. Five-consonant onsets include /kp[26C?]kn, kskqz/. Six-consonant onsets include /tksxqz/. The one example of a seven-consonant onset given is /kstk’[26C?]kn/.
\end{styleBody}

\begin{styleBody}
\textbf{Coda restrictions: }There seem to be restrictions on simple codas; examples not given for /p’ t’ z j/ in this environment. Biconsonantal codas include /mx [26C?]q sx/. Triconsonantal codas include /p[26C?]h m[26C?]x/. Four-consonant codas include /nt[361?][283?]px mp[26C?]x [26C?]txt[361?][283?]/. Five-consonant codas include /nx[26C?]xt[361?][283?] mstxt[361?][283?]/.
\end{styleBody}

\begin{styleBody}
\textbf{Notes: }Combinability of consonants within clusters is subject to few constraints. “Das häufige Auftreten komplexer Konsonantengruppen gehört zu den auffällingsten Zügen der itelmenischen Phonologie.” (“The frequent occurrence of complex consonant clusters is one of the most notable traits of Itelmen phonology.” Georg \& Volodin 1999:38)
\end{styleBody}

\begin{styleBody}
\textbf{Suprasegmentals}
\end{styleBody}

\begin{styleBody}
\textbf{Tone:} No
\end{styleBody}

\begin{styleBody}
\textbf{Word stress:} Yes
\end{styleBody}

\begin{styleBody}
\textbf{Stress placement:} Fixed
\end{styleBody}

\begin{styleBody}
\textbf{Phonetic processes conditioned by stress:} (None)
\end{styleBody}

\begin{styleBody}
\textbf{Differences in phonological properties of stressed and unstressed syllables:} Not described
\end{styleBody}

\begin{styleBody}
\textbf{Phonetic correlates of stress: }Intensity (impressionistic)
\end{styleBody}

\begin{styleBody}
\textbf{Vowel reduction processes}
\end{styleBody}

\begin{styleBody}
\textbf{itl-R1:} Vowels occurring in closed syllables ‘cluttered with consonants’ (\textsubscript{n}CVC\textsubscript{n}) are less clear and reduced in quality (Volodin 1976: 73).
\end{styleBody}

\begin{styleBody}
\textbf{itl-R2:} Mid central vowel /[259?]/ may be realized as a high back unrounded vowel [[26F?]] or drop entirely in some contexts where the consonantal environment has no effect (Georg \& Volodin 1999: 13).
\end{styleBody}

\begin{styleBody}
\textbf{Consonant allophony processes}
\end{styleBody}

\begin{styleBody}
\textbf{itl-C1: }Some stops and affricates are labialized preceding a rounded vowel (Georg \& Volodin 1999: 16).
\end{styleBody}

\begin{styleBody}
\textbf{itl-C2: }A voiceless bilabial stop is spirantized intervocalically (Georg \& Volodin 1999: 14-15).
\end{styleBody}

\begin{styleBody}
\textbf{itl-C3: }A voiceless bilabial fricative is realized as an approximant preceding a consonant (Georg \& Volodin 1999).
\end{styleBody}

\begin{styleBody}
\textbf{Morphology}
\end{styleBody}

\begin{styleBody}
\textbf{Text:} “Süddialekt” (Georg \& Volodin 1999: 250-262)
\end{styleBody}

\begin{styleBody}
\textbf{Synthetic index: }2.0 morphemes/word (876 morphemes, 438 words)
\end{styleBody}

\clearpage\begin{styleBody}
\textbf{[kal] }\ \ \textbf{\textsc{Kalaallisut}}\textbf{\ \ \ \ }Eskimo-Aleut, \textit{Eskimo} (Greenland)
\end{styleBody}

\begin{styleBody}
\textbf{References consulted: }Fortescue (1984), Hagerup (2011), Jacobsen (2000)
\end{styleBody}

\begin{styleBody}
\textbf{Sound inventory}
\end{styleBody}

\begin{styleBody}
\textbf{C phoneme inventory:} /p t k q f[2D0?] v s [26C?][2D0?] [29D?] x[2D0?] [263?] $\chi [2D0?]$ [281?] m n ŋ [274?] l/
\end{styleBody}

\begin{styleBody}
\textbf{N consonant phonemes:} 18
\end{styleBody}

\begin{styleBody}
\textbf{Geminates:} /[274?][2D0?]/, many others in morphophonological contexts
\end{styleBody}

\begin{styleBody}
\textbf{Voicing contrasts:} None
\end{styleBody}

\begin{styleBody}
\textbf{Places:} Bilabial, Labiodental, Alveolar, Palatal, Velar, Uvular
\end{styleBody}

\begin{styleBody}
\textbf{Manners:} Stop, Fricative, Nasal, Lateral approximant, Lateral fricative 
\end{styleBody}

\begin{styleBody}
\textbf{N elaborations:} 3
\end{styleBody}

\begin{styleBody}
\textbf{Elaborations:} Voiced fricative/affricates, Labiodental, Uvular
\end{styleBody}

\begin{styleBody}
\textbf{V phoneme inventory:} /i a u i[2D0?] a[2D0?] u[2D0?]/
\end{styleBody}

\begin{styleBody}
\textbf{N vowel qualities:} 3
\end{styleBody}

\begin{styleBody}
\textbf{Diphthongs or vowel sequences:} Diphthong /ai/
\end{styleBody}

\begin{styleBody}
\textbf{Contrastive length:} All
\end{styleBody}

\begin{styleBody}
\textbf{Contrastive nasalization:} None
\end{styleBody}

\begin{styleBody}
\textbf{Other contrasts:} N/A
\end{styleBody}

\begin{styleBody}
\textbf{Notes: }Geminate versions of /v [263?] [281?]/ are also voiceless, so are treated as separate phonemes here. Similarly, geminate version of /l/ — /[26C?][2D0?]/ — differs in both voicing and manner of articulation, so it is included in the phoneme inventory here. /[274?]/ usually occurs as a geminate except for in some morphophonological contexts; Jacobsen states that the geminate is only marginally contrastive. /[282?]/ found only in central dialect region and is described as rapidly receding and merging with /s/, with merger complete in younger speakers. /f h/ occur in loanwords. /a[2D0?]/ much more common than other long vowels. Other historical diphthongs have merged into long vowels.
\end{styleBody}

\begin{styleBody}
\textbf{Syllable structure}
\end{styleBody}

\begin{styleBody}
\textbf{Category: }Moderately Complex
\end{styleBody}

\begin{styleBody}
\textbf{Canonical syllable structure: }(C)V(C) (Fortescue 1984: 338-9)
\end{styleBody}

\begin{styleBody}
\textbf{Size of maximal onset: }1
\end{styleBody}

\begin{styleBody}
\textbf{Size of maximal coda:} 1
\end{styleBody}

\begin{styleBody}
\textbf{Onset obligatory:} No
\end{styleBody}

\begin{styleBody}
\textbf{Coda obligatory }No
\end{styleBody}

\begin{styleBody}
\textbf{Vocalic nucleus patterns:} Short vowels, Long vowels, Diphthongs
\end{styleBody}

\begin{styleBody}
\textbf{Syllabic consonant patterns:} N/A
\end{styleBody}

\begin{styleBody}
\textbf{Size of maximal word{}-marginal sequences with syllabic obstruents:} N/A
\end{styleBody}

\begin{styleBody}
\textbf{Predictability of syllabic consonants:} N/A
\end{styleBody}

\begin{styleBody}
\textbf{Morphological constituency of maximal syllable margin:} N/A
\end{styleBody}

\begin{styleBody}
\textbf{Morphological pattern of syllabic consonants:} N/A
\end{styleBody}

\begin{styleBody}
\textbf{Onset restrictions: }All consonants occur.
\end{styleBody}

\begin{styleBody}
\textbf{Coda restrictions: }All(?) consonants occur.
\end{styleBody}

\begin{styleBody}
\textbf{Notes: }Final geminates occur in syncopated exclamations.
\end{styleBody}

\begin{styleBody}
\textbf{Suprasegmentals}
\end{styleBody}

\begin{styleBody}
\textbf{Tone:} No
\end{styleBody}

\begin{styleBody}
\textbf{Word stress:} No
\end{styleBody}

\begin{styleBody}
\textbf{Vowel reduction processes}
\end{styleBody}

\begin{styleBody}
\textbf{kal-R1: }\ Short high vowels /i u/ are produced as lax word-finally (Hagerup 2011: 56-63).
\end{styleBody}

\begin{styleBody}
\textbf{kal-R2: }Short high vowels /i u/ tend to be devoiced between voiceless consonants in open syllables (Fortescue 1984: 335).
\end{styleBody}

\begin{styleBody}
\textbf{kal-R3: }Long vowels are realized as shorter when preceding long consonants than they are preceding singleton consonants (Jacobsen 2000: 65).
\end{styleBody}

\begin{styleBody}
\textbf{Consonant allophony processes}
\end{styleBody}

\begin{styleBody}
\textbf{kal-C1: }A voiceless alveolar stop is affricated preceding a high front vowel (Fortescue 1984: 333).
\end{styleBody}

\begin{styleBody}
\textbf{kal-C2: }Some stops and fricatives are realized with secondary palatalization adjacent to a high front vowel (Fortescue 1984: 333).
\end{styleBody}

\begin{styleBody}
\textbf{kal-C3: }A voiceless alveolar fricative is somewhat voiced intervocalically (Fortescue 1984: 334).
\end{styleBody}

\begin{styleBody}
\textbf{kal-C4: }A uvular stop may be realized as a fricative intervocalically (Fortescue 1984: 333).
\end{styleBody}

\begin{styleBody}
\textbf{kal-C5: }A voiced velar fricative is realized as a glide intervocalically (Fortescue 1984: 334).
\end{styleBody}

\begin{styleBody}
\textbf{kal-C6: }A velar nasal may be realized as a nasalized vowel intervocalically (Fortescue 1984: 334).
\end{styleBody}

\begin{styleBody}
\textbf{Morphology}
\end{styleBody}

\begin{styleBody}
(adequate texts unavailable)
\end{styleBody}

\clearpage\begin{styleBody}
\textbf{[kat] }\ \ \textbf{\textsc{Georgian}}\textbf{\ \ \ \ }Kartvelian, \textit{Georgian-Zan} (Georgia)
\end{styleBody}

\begin{styleBody}
\textbf{References consulted: }Aronson (1990, 1991), Chitoran (1998), Hewitt (1995), Jun et al. (2006), Shosted \& Chikovani (2006), Skopeteas \& Féry (2010), Vicenek (2010), Vogt (1958)
\end{styleBody}

\begin{styleBody}
\textbf{Sound inventory}
\end{styleBody}

\begin{styleBody}
\textbf{C phoneme inventory:} /b p[2B0?] d[32A?] t[32A?][2B0?] [261?] k[2B0?] p’ t[32A?]’ k’ q’ d[361?]z t[361?]s[2B0?] d[361?][292?] t[361?][283?][2B0?] t[361?]s’ t[361?][283?]’ $\beta $ z s [292?] [283?] [281?] $\chi $ h m n [27E?] l/
\end{styleBody}

\begin{styleBody}
\textbf{N consonant phonemes:} 28
\end{styleBody}

\begin{styleBody}
\textbf{Geminates:} N/A
\end{styleBody}

\begin{styleBody}
\textbf{Voicing contrasts:} Obstruents
\end{styleBody}

\begin{styleBody}
\textbf{Places:} Bilabial, Dental, Alveolar, Palato-alveolar, Velar, Uvular, Glottal
\end{styleBody}

\begin{styleBody}
\textbf{Manners:} Stop, Affricate, Fricative, Nasal, Flap/Tap, Lateral approximant
\end{styleBody}

\begin{styleBody}
\textbf{N elaborations:} 5
\end{styleBody}

\begin{styleBody}
\textbf{Elaborations:} Voiced fricatives/affricates, Post-aspiration, Ejective, Palato-alveolar, Uvular
\end{styleBody}

\begin{styleBody}
\textbf{V phoneme inventory:} /i e a o u/
\end{styleBody}

\begin{styleBody}
\textbf{N vowel qualities:} 5
\end{styleBody}

\begin{styleBody}
\textbf{Diphthongs or vowel sequences:} None
\end{styleBody}

\begin{styleBody}
\textbf{Contrastive length:} None
\end{styleBody}

\begin{styleBody}
\textbf{Contrastive nasalization:} None
\end{styleBody}

\begin{styleBody}
\textbf{Other contrasts:} N/A
\end{styleBody}

\begin{styleBody}
\textbf{Notes:} [$\beta $] alternates with [$\beta [31E?]$].Vicenik gives instrumental evidence that /b d [261?]/ are voiced. Shosted \& Chikovani have /v/ for the glide, as well as velars instead of /[281?] $\chi $/. Robins \& Waterson have a velarized lateral instead of a plain lateral approximant.
\end{styleBody}

\begin{styleBody}
\textbf{Syllable structure}
\end{styleBody}

\begin{styleBody}
\textbf{Category:} Highly Complex
\end{styleBody}

\begin{styleBody}
\textbf{Canonical syllable structure: }(C)(C)(C)(C)(C)(C)(C)(C)V(C)(C)(C)(C)(C) (Hewitt 1995: 19-20; Vogt 1958; Butskhrikidze 2002: 197-205)
\end{styleBody}

\begin{styleBody}
\textbf{Size of maximal onset:} 8
\end{styleBody}

\begin{styleBody}
\textbf{Size of maximal coda:} 5
\end{styleBody}

\begin{styleBody}
\textbf{Onset obligatory:} No
\end{styleBody}

\begin{styleBody}
\textbf{Coda obligatory:} No
\end{styleBody}

\begin{styleBody}
\textbf{Vocalic nucleus patterns:} Short vowels
\end{styleBody}

\begin{styleBody}
\textbf{Syllabic consonant patterns:} N/A
\end{styleBody}

\begin{styleBody}
\textbf{Size of maximal word{}-marginal sequences with syllabic obstruents:} N/A
\end{styleBody}

\begin{styleBody}
\textbf{Predictability of syllabic consonants:} N/A
\end{styleBody}

\begin{styleBody}
\textbf{Morphological constituency of maximal syllable margin:} Morphologically Complex (Onset, Coda)
\end{styleBody}

\begin{styleBody}
\textbf{Morphological pattern of syllabic consonants:} N/A
\end{styleBody}

\begin{styleBody}
\textbf{Onset restrictions: }All consonants occur as simple onsets. Biconsonantal onsets include stop+stop, stop+affricate, stop+fricative, stop+sonorant, affricate+stop, affricate+fricative, affricate+sonorant, fricative+stop, and so on. Triconsonantal onsets include stop+stop+stop, stop+stop+sonorant, stop+affricate+stop, stop+sonorant+stop, fricative+stop+sonorant etc. stem-initially, and more when prefixes are involved. All larger onsets include sonorants such that there are no obstruent strings of more than three; e.g. /p’[27E?]t[361?]s’k'$\beta [31E?]$, t[361?]s’q’[27E?]t, brt[361?]s'q{}'/. Seven-consonant onsets include /[261?]$\beta [31E?]$t[361?]s’$\beta [31E?][27E?]$tn/. Eight-consonant onsets include /[261?]$\beta [31E?]$prt[361?]sk$\beta [31E?]$n/.
\end{styleBody}

\begin{styleBody}
\textbf{Coda restrictions: }All(?) consonants but /h/ occur in simple codas. Biconsonantal codas include /[27E?]t bs nd ds ls bt mt pt/. Triconsonantal codas include /[261?]ns $\chi $ls/. Five-member codas include /nt[361?][283?]xls, [27E?]t[361?]s’q’$\beta [31E?]$s, [27E?]t'k'ls/.
\end{styleBody}

\begin{styleBody}
\textbf{Notes: }Vogt lists 740 onset clusters (of up to six members) and 244 stem-final clusters (of up to four members); however, he does not include morphologically complex clusters. True word-final clusters seem to be much more restricted than stem-final clusters, which are always followed by a vowel, which resyllabifies the cluster. However it does seem to be the case that sonorants are required in all onsets of more than three consonants and all codas of three consonant or more. A subset of clusters are known as ‘harmonic’ and consist of a non-velar stop or affricate followed by a homogeneous velar or uvular consonant. These have been analyzed as single segment, but Chitoran (1998) shows through instrumental analysis that they have phonetic and timing characteristics of sequences.
\end{styleBody}

\begin{styleBody}
\textbf{Suprasegmentals}
\end{styleBody}

\begin{styleBody}
\textbf{Tone:} No
\end{styleBody}

\begin{styleBody}
\textbf{Word stress:} Yes
\end{styleBody}

\begin{styleBody}
\textbf{Stress placement:} Fixed
\end{styleBody}

\begin{styleBody}
\textbf{Phonetic processes conditioned by stress:} (None)
\end{styleBody}

\begin{styleBody}
\textbf{Differences in phonological properties of stressed and unstressed syllables:} (None)
\end{styleBody}

\begin{styleBody}
\textbf{Phonetic correlates of stress: }Not described
\end{styleBody}

\begin{styleBody}
\textbf{Vowel reduction processes}
\end{styleBody}

\begin{styleBody}
(none reported)
\end{styleBody}

\begin{styleBody}
\textbf{Consonant allophony processes}
\end{styleBody}

\begin{styleBody}
\textbf{kat-C1: }The uvular ejective stop may vary with an ejective uvular affricate variant (Aronson 1991).
\end{styleBody}

\begin{styleBody}
\textbf{kat-C2: }A uvular ejective stop may vary with an ejective uvular fricative (Shosted \& Chikovani 2006).
\end{styleBody}

\begin{styleBody}
\textbf{Morphology}
\end{styleBody}

\begin{styleBody}
\textbf{Text:} “The destiny of Kartli” (Hewitt 1995: 655-663)
\end{styleBody}

\begin{styleBody}
\textbf{Synthetic index: }2.4 morphemes/word (594 morphemes, 246 words)
\end{styleBody}

\clearpage\begin{styleBody}
\textbf{[kbc] }\ \ \textbf{\textsc{Kadiwéu}}\textbf{\ \ }Guaicuruan (Brazil)
\end{styleBody}

\begin{styleBody}
\textbf{References consulted: }Braggio (1981), Sandalo (1997)
\end{styleBody}

\begin{styleBody}
\textbf{Sound inventory}
\end{styleBody}

\begin{styleBody}
\textbf{C phoneme inventory:} /p b t d k [261?] q t[361?][283?] d[361?][292?] [281?] m n l w j/
\end{styleBody}

\begin{styleBody}
\textbf{N consonant phonemes:} 15
\end{styleBody}

\begin{styleBody}
\textbf{Geminates:} /b[2D0?] d[2D0?] [261?][2D0?] m[2D0?] n[2D0?] l[2D0?] w[2D0?] j[2D0?]/ (Some)
\end{styleBody}

\begin{styleBody}
\textbf{Voicing contrasts:} Obstruents
\end{styleBody}

\begin{styleBody}
\textbf{Places:} Bilabial, Dental, Palato-Alveolar, Velar, Uvular
\end{styleBody}

\begin{styleBody}
\textbf{Manners:} Stop, Affricate, Fricative, Nasal, Central approximant, Lateral approximant
\end{styleBody}

\begin{styleBody}
\textbf{N elaborations:} 3
\end{styleBody}

\begin{styleBody}
\textbf{Elaborations:} Voiced fricatives/affricates, Palato-alveolar, Uvular
\end{styleBody}

\begin{styleBody}
\textbf{V phoneme inventory:} /i e a o i[2D0?] e[2D0?] a[2D0?] o[2D0?]/
\end{styleBody}

\begin{styleBody}
\textbf{N vowel qualities:} 4
\end{styleBody}

\begin{styleBody}
\textbf{Diphthongs or vowel sequences:} None
\end{styleBody}

\begin{styleBody}
\textbf{Contrastive length:} All
\end{styleBody}

\begin{styleBody}
\textbf{Contrastive nasalization:} None
\end{styleBody}

\begin{styleBody}
\textbf{Other contrasts:} N/A
\end{styleBody}

\begin{styleBody}
\textbf{Notes:} Sandalo gives geminate counterparts of /m n l w j/ in the phoneme inventory. Sandalo analyzes /[281?]/ as a uvular stop phonologically, but since it is realized as a fricative in most positions, I use this symbol.
\end{styleBody}

\begin{styleBody}
\textbf{Syllable structure}
\end{styleBody}

\begin{styleBody}
\textbf{Complexity Category:} Complex
\end{styleBody}

\begin{styleBody}
\textbf{Canonical syllable structure:} (C)(C)V\textbf{ }(Sandalo 1997: 17-18)
\end{styleBody}

\begin{styleBody}
\textbf{Size of maximal onset:} 2
\end{styleBody}

\begin{styleBody}
\textbf{Size of maximal coda:} N/A
\end{styleBody}

\begin{styleBody}
\textbf{Onset obligatory: }No
\end{styleBody}

\begin{styleBody}
\textbf{Coda obligatory:} N/A
\end{styleBody}

\begin{styleBody}
\textbf{Vocalic nucleus patterns:} Short vowels, Long vowels, Diphthongs
\end{styleBody}

\begin{styleBody}
\textbf{Syllabic consonant patterns:} N/A
\end{styleBody}

\begin{styleBody}
\textbf{Size of maximal word{}-marginal sequences with syllabic obstruents:} N/A
\end{styleBody}

\begin{styleBody}
\textbf{Predictability of syllabic consonants:} N/A
\end{styleBody}

\begin{styleBody}
\textbf{Morphological constituency of maximal syllable margin:} Morpheme-internal (Onset)
\end{styleBody}

\begin{styleBody}
\textbf{Morphological pattern of syllabic consonants:} N/A
\end{styleBody}

\begin{styleBody}
\textbf{Onset restrictions:} All consonants may occur in simple onsets. In biconsonantal onets, C\textsubscript{2} is always /[281?]/, C\textsubscript{1} may be stop or nasal (perhaps others too).
\end{styleBody}

\begin{styleBody}
\textbf{Notes: }Sandalo’s analysis has /[281?] d d[2D0?]/ occurring as codas in clitics, but apparently these never surface as such phonetically, being deleted preceding consonants and resyllabified as onsets preceding vowels (1997: 16).
\end{styleBody}

\begin{styleBody}
\textbf{Suprasegmentals}
\end{styleBody}

\begin{styleBody}
\textbf{Tone:} No
\end{styleBody}

\begin{styleBody}
\textbf{Word stress:} Yes
\end{styleBody}

\begin{styleBody}
\textbf{Stress placement:} Weight-Sensitive
\end{styleBody}

\begin{styleBody}
\textbf{Phonetic processes conditioned by stress:} Consonant Allophony in Unstressed Syllables
\end{styleBody}

\begin{styleBody}
\textbf{Differences in phonological properties of stressed and unstressed syllables:} (None)
\end{styleBody}

\begin{styleBody}
\textbf{Phonetic correlates of stress: }Intensity (impressionistic)
\end{styleBody}

\begin{styleBody}
\textbf{Vowel reduction processes}
\end{styleBody}

\begin{styleBody}
\textbf{kbc-R1:} Long vowels are optionally reduced to short vowels preceding a voiceless stop onset of a following syllable. Example given shows that following stop is lengthened (Sandalo 1997: 17).
\end{styleBody}

\begin{styleBody}
\textbf{Consonant allophony processes}
\end{styleBody}

\begin{styleBody}
\textbf{kbc-C1: }A voiced uvular fricative may be realized as a stop word-initially (Sandalo 1997: 16).
\end{styleBody}

\begin{styleBody}
\textbf{kbc-C2: }A voiced alveolar stop is realized as a flap intervocalically (Sandalo 1997: 16).
\end{styleBody}

\begin{styleBody}
\textbf{kbc-C3: }A voiced palato-alveolar affricate is realized as a fricative by some speakers (Sandalo 1997: 15-16).
\end{styleBody}

\begin{styleBody}
\textbf{Morphology}
\end{styleBody}

\begin{styleBody}
(adequate texts unavailable)
\end{styleBody}

\clearpage\begin{styleBody}
\textbf{[kbd] }\ \ \textbf{\textsc{Kabardian}}\textbf{\ \ }Abkhaz-Adyge, \textit{Circassian} (Russia, Turkey)
\end{styleBody}

\begin{styleBody}
\textbf{References consulted: }Applebaum (2013), Colarusso (2006), Gordon \& Applebaum (2010), Kuipers (1960), Matasović (2010)
\end{styleBody}

\begin{styleBody}
\textbf{Sound inventory}
\end{styleBody}

\begin{styleBody}
\textbf{C phoneme inventory:} /p b t d k[2B7?] [261?][2B7?] q q[2B7?] [294?] [294?][2B7?] p’ t’ k’[2B7?] q’ q’[2B7?] t[361?]s d[361?]z t[361?][283?] d[361?][292?] t[361?]s’ t[361?][283?]’ f v s z [26C?] [26E?] [255?] [291?] [283?] [292?] x x[2B7?] [263?] $\chi $ [281?] $\chi [2B7?]$ [281?][2B7?] [127?] h f' [26C?]’ [255?]' m n r w j/
\end{styleBody}

\begin{styleBody}
\textbf{N consonant phonemes:} 48
\end{styleBody}

\begin{styleBody}
\textbf{Geminates:} N/A
\end{styleBody}

\begin{styleBody}
\textbf{Voicing contrasts:} Obstruents
\end{styleBody}

\begin{styleBody}
\textbf{Places:} Bilabial, Labiodental, Dental, Alveolar, Palato-alveolar, Alveolo-palatal, Velar, Uvular, Pharyngeal, Glottal
\end{styleBody}

\begin{styleBody}
\textbf{Manners:} Stop, Affricate, Fricative, Nasal, Trill, Central approximant, Lateral fricative, Ejective
\end{styleBody}

\begin{styleBody}
\textbf{N elaborations:} 7
\end{styleBody}

\begin{styleBody}
\textbf{Elaborations:} Voiced fricatives/affricates, Ejective, Labiodental, Palato-alveolar, Uvular, Pharyngeal, Labialization
\end{styleBody}

\begin{styleBody}
\textbf{V phoneme inventory:} /[259?] a a[2D0?]/
\end{styleBody}

\begin{styleBody}
\textbf{N vowel qualities:} 2
\end{styleBody}

\begin{styleBody}
\textbf{Diphthongs or vowel sequences:} Diphthongs /aw j[259?]/
\end{styleBody}

\begin{styleBody}
\textbf{Contrastive length:} Some
\end{styleBody}

\begin{styleBody}
\textbf{Contrastive nasalization:} None
\end{styleBody}

\begin{styleBody}
\textbf{Other contrasts:} N/A
\end{styleBody}

\begin{styleBody}
\textbf{Notes:} /[127?]/ is marginal; exists in the speech of older generations, mostly in Arabic loans (Matasovic 10). Colarusso has /c [25F?]/ or /t[320?] d[320?]/ for /t[361?][283?] d[361?][292?]/. Colarusso doesn’t have /[263?]/. Other accounts posit two short vowels ([259?] a) and five long vowels (a[2D0?] e[2D0?] i[2D0?] o[2D0?] u[2D0?]). There is a length contrast for /a/: /a[2D0?]/ is back open, /a/ is central open.
\end{styleBody}

\begin{styleBody}
\textbf{Syllable structure}
\end{styleBody}

\begin{styleBody}
\textbf{Complexity Category:} Highly Complex
\end{styleBody}

\begin{styleBody}
\textbf{Canonical syllable structure:} (C)(C)(C)V(C)(C)\textbf{ }(Colarusso 2006: 4-20; Matasović 2010: 13; Applebaum 2013)
\end{styleBody}

\begin{styleBody}
\textbf{Size of maximal onset:} 3
\end{styleBody}

\begin{styleBody}
\textbf{Size of maximal coda:} 2
\end{styleBody}

\begin{styleBody}
\textbf{Onset obligatory: }No
\end{styleBody}

\begin{styleBody}
\textbf{Coda obligatory:} N/A
\end{styleBody}

\begin{styleBody}
\textbf{Vocalic nucleus patterns:} Short vowels, Long vowels
\end{styleBody}

\begin{styleBody}
\textbf{Syllabic consonant patterns:} Nasal, Liquid, Obstruent
\end{styleBody}

\begin{styleBody}
\textbf{Size of maximal word{}-marginal sequences with syllabic obstruents:} N/A
\end{styleBody}

\begin{styleBody}
\textbf{Predictability of syllabic consonants:} Predictable from word/consonantal context (Nasal, Liquid), Varies with VC sequence (Nasal, Liquid, Obstruent)
\end{styleBody}

\begin{styleBody}
\textbf{Morphological constituency of maximal syllable margin:} Morpheme-internal (Coda), Both patterns (Onset)
\end{styleBody}

\begin{styleBody}
\textbf{Morphological pattern of syllabic consonants:} Unclear
\end{styleBody}

\begin{styleBody}
\textbf{Onset restrictions: }All consonants occur as simple onsets. Biconsonantal onsets consist mostly of stop+fricative, e.g. /th b[281?][2B7?] p[255?]/, but also rarely include two stops, e.g. /pq/. These clusters tend to be regressive, and clusters with labial first element are especially frequent. Examples of triconsonantal onsets include /bzw zb[263?] p[255?]t/.
\end{styleBody}

\begin{styleBody}
\textbf{Coda restrictions: }Unclear whether there are restrictions on simple codas. Biconsonantal codas include /bz wf p[26C?] rt/.
\end{styleBody}

\begin{styleBody}
\textbf{Notes: }Colarusso analyzes initial sequence in \textit{zb[263?]\'{a}[255?]} ‘I covered/thatched it’ as \textit{z.b[263?]\'{a}[255?]}, but gives no articulatory/perceptual evidence for this (2006: 17). This analysis seems to be influenced by formal models of syllable structure. Matasović describes such sequences as onset clusters (2010: 13). Applebaum (2013) gives examples of complex codas.
\end{styleBody}

\begin{styleBody}
\textbf{Suprasegmentals}
\end{styleBody}

\begin{styleBody}
\textbf{Tone: }No
\end{styleBody}

\begin{styleBody}
\textbf{Word stress:} Yes
\end{styleBody}

\begin{styleBody}
\textbf{Stress placement:} Weight-Sensitive
\end{styleBody}

\begin{styleBody}
\textbf{Phonetic processes conditioned by stress:} Vowel Reduction
\end{styleBody}

\begin{styleBody}
\textbf{Differences in phonological properties of stressed and unstressed syllables:} (None)
\end{styleBody}

\begin{styleBody}
\textbf{Phonetic correlates of stress: }Vowel duration (instrumental), Pitch (instrumental), Intensity (instrumental)
\end{styleBody}

\begin{styleBody}
\textbf{Notes: }Duration and intensity are correlates of stress for most speakers.
\end{styleBody}

\begin{styleBody}
\textbf{Vowel reduction processes}
\end{styleBody}

\begin{styleBody}
\textbf{kbd-R1:} Low vowel /a/ is realized as higher [[250?]] when unstressed (Applebaum 2013: 98-9).
\end{styleBody}

\begin{styleBody}
\textbf{kbd-R2:} Frequently a sequence of a short high vowel and a consonant is replaced by a syllabic consonant (results in syllabic nasals, liquids, and obstruents; Kuipers 1960: 24, 42-3).
\end{styleBody}

\begin{styleBody}
\textbf{kbd-R3:} Word-final /[259?]/ is deleted after a stressed syllable (Kuipers 1960: 34, 42).
\end{styleBody}

\begin{styleBody}
\textbf{kbd-R4: }Unstressed /[259?]/ preceding a stressed syllable is often deleted, so long as it does not produce an initial consonant cluster (Gordon \& Applebaum 2010: 42).
\end{styleBody}

\begin{styleBody}
\textbf{Consonant allophony processes}
\end{styleBody}

\begin{styleBody}
\textbf{kbd-C1: }Voiceless plosives may have affricated release preceding a vowel (Kuipers 1960: 17).
\end{styleBody}

\begin{styleBody}
\textbf{kbd-C2: }Labiovelar and palatal glides are realized with slight glottal friction word-initially (Kuipers 1960: 22).
\end{styleBody}

\begin{styleBody}
\textbf{kbd-C3: }Stops are voiced preceding a voiced stop or fricative (Matasović 2010: 11).
\end{styleBody}

\begin{styleBody}
\textbf{kbd-C4: }Voiceless ejective palato-alveolar affricate and fricative are realized as voiced word-medially (Kuipers 1960: 19).
\end{styleBody}

\begin{styleBody}
\textbf{Morphology}
\end{styleBody}

\begin{styleBody}
\textbf{Text:} “Nart story” (Applebaum 2013: 223-231)
\end{styleBody}

\begin{styleBody}
\textbf{Synthetic index: }2.5 morphemes/word (571 morphemes, 229 words)
\end{styleBody}

\clearpage\begin{styleBody}
\textbf{[kbh] }\ \ \textbf{\textsc{Camsá}}\textbf{\ \ }isolate (Colombia)
\end{styleBody}

\begin{styleBody}
\textbf{References consulted: }Fabre (2002), Howard (1967, 1972), Juajibioy Chindoy (1962), Monguí Sánchez (1981)
\end{styleBody}

\begin{styleBody}
\textbf{Sound inventory}
\end{styleBody}

\begin{styleBody}
\textbf{C phoneme inventory:} /p b t d k [261?] t[361?]s [288?][361?][282?] t[361?][283?] [278?] s [282?] [283?] x m n [272?] l [27E?] [28E?] w j/
\end{styleBody}

\begin{styleBody}
\textbf{N consonant phonemes:} 22
\end{styleBody}

\begin{styleBody}
\textbf{Geminates:} N/A
\end{styleBody}

\begin{styleBody}
\textbf{Voicing contrasts:} Obstruents
\end{styleBody}

\begin{styleBody}
\textbf{Places:} Bilabial, Alveolar, Palato-alveolar, Retroflex, Velar
\end{styleBody}

\begin{styleBody}
\textbf{Manners:} Stop, Affricate, Fricative, Nasal, Flap/Tap, Central approximant, Lateral approximant
\end{styleBody}

\begin{styleBody}
\textbf{N elaborations:} 2
\end{styleBody}

\begin{styleBody}
\textbf{Elaborations:} Palato-alveolar, Retroflex
\end{styleBody}

\begin{styleBody}
\textbf{V phoneme inventory:} /i e [268?] a o u/
\end{styleBody}

\begin{styleBody}
\textbf{N vowel qualities:} 6
\end{styleBody}

\begin{styleBody}
\textbf{Diphthongs or vowel sequences:} Diphthongs /ai oi ui ia io ie ua ue/
\end{styleBody}

\begin{styleBody}
\textbf{Contrastive length:} None
\end{styleBody}

\begin{styleBody}
\textbf{Contrastive nasalization:} None
\end{styleBody}

\begin{styleBody}
\textbf{Other contrasts:} N/A
\end{styleBody}

\begin{styleBody}
\textbf{Notes:} Monguí Sánchez give a very different consonant phoneme inventory than the others. Juajibioy Chindoy concurs with Howard and the others, but additionally lists affricate /pf/. Howard lists {\textless}ë{\textgreater} for what others list as /[268?]/; Monguí Sánchez gives /[259?]/ for this vowel.
\end{styleBody}

\begin{styleBody}
\textbf{Syllable structure}
\end{styleBody}

\begin{styleBody}
\textbf{Complexity Category:} Highly Complex
\end{styleBody}

\begin{styleBody}
\textbf{Canonical syllable structure:} (C)(C)(C)(C)V\textbf{ }(Howard 1967: 81-5, Howard 1972: 84-9)
\end{styleBody}

\begin{styleBody}
\textbf{Size of maximal onset:} 4
\end{styleBody}

\begin{styleBody}
\textbf{Size of maximal coda:} N/A
\end{styleBody}

\begin{styleBody}
\textbf{Onset obligatory:} No
\end{styleBody}

\begin{styleBody}
\textbf{Coda obligatory:} N/A
\end{styleBody}

\begin{styleBody}
\textbf{Vocalic nucleus patterns:} Short vowels, Diphthongs
\end{styleBody}

\begin{styleBody}
\textbf{Syllabic consonant patterns:} N/A
\end{styleBody}

\begin{styleBody}
\textbf{Size of maximal word{}-marginal sequences with syllabic obstruents:} N/A
\end{styleBody}

\begin{styleBody}
\textbf{Predictability of syllabic consonants:} N/A
\end{styleBody}

\begin{styleBody}
\textbf{Morphological constituency of maximal syllable margin:} Morphologically Complex (Onset)
\end{styleBody}

\begin{styleBody}
\textbf{Morphological pattern of syllabic consonants:} N/A
\end{styleBody}

\begin{styleBody}
\textbf{Onset restrictions: }Biconsonantal onsets are most commonly two voiceless consonants at different places of articulation, e.g. /xt st ft[361?]s sb t[361?][283?]t, tk tm [283?]l nj/. Many combinations occur, but liquids and glides are restricted to C\textsubscript{2} position. In triconsonantal onsets, C\textsubscript{1} is /b t s [283?] n/ and apparently /[278?]/, C\textsubscript{2} is /d t k t[361?][283?] t[361?]s x [283?] [282?] m j/, and C\textsubscript{3} is /b k j m n [27E?]/. Examples include /stx ndm [278?]xn st[361?][283?]b s[283?]t[361?]s/ . 4-consonant onsets include /[278?]stx/.
\end{styleBody}

\begin{styleBody}
\textbf{Notes: }“Consonant clusters are very common in Camsá” (Howard 1967: 81). Howard (1967) gives canonical syllable structure as (C)(C)(C)V, but updates it to (C)(C)(C)(C)V in Howard (1972), saying onsets may consist of four consonants when subject is 1st person plural (/[278?]{}-/). Some biconsonantal onsets (stop+stop, C+nasal sequences at different places of articulation) appear with brief transitional vocoid [\textsuperscript{[259?]}] between consonants; similarly there are affects on length of fricatives in first versus second position of biconsonantal onsets, and sometimes an associated vocoid or offlgide with those (1967: 82).
\end{styleBody}

\begin{styleBody}
\textbf{Suprasegmentals}
\end{styleBody}

\begin{styleBody}
\textbf{Tone:} No
\end{styleBody}

\begin{styleBody}
\textbf{Word stress:} Yes
\end{styleBody}

\begin{styleBody}
\textbf{Stress placement:} Fixed
\end{styleBody}

\begin{styleBody}
\textbf{Phonetic processes conditioned by stress:} Vowel Reduction
\end{styleBody}

\begin{styleBody}
\textbf{Differences in phonological properties of stressed and unstressed syllables:} (None)
\end{styleBody}

\begin{styleBody}
\textbf{Phonetic correlates of stress: }Vowel duration (impressionistic)
\end{styleBody}

\begin{styleBody}
\textbf{Vowel reduction processes}
\end{styleBody}

\begin{styleBody}
\textbf{kbh-R1:} Word-final vowels occurring after a penultimate stress are optionally devoiced or deleted. This may occur in isolation but generally occurs in the middle of a clause (Howard 1967: 86).
\end{styleBody}

\begin{styleBody}
\textbf{kbh-R2:} Word-medial vowels are ‘practically eliminated,’ with syllables between the first syllable and the stressed syllable being ‘squeezed together’ (Howard 1967: 86-7).
\end{styleBody}

\begin{styleBody}
\textbf{Notes: }“Words are pronounced rapidly with vowels practically eliminated word medially. A degree of emphasis is placed on the vowel of the first syllable with the following syllables squeezed together before the stressed syllable.” (Howard 1967: 86-7).
\end{styleBody}

\begin{styleBody}
\textbf{Consonant allophony processes}
\end{styleBody}

\begin{styleBody}
\textbf{kbh-C1: }A palatal glide is realized as a voiced palato-alveolar affricate following an alveolar nasal (Howard 1967).
\end{styleBody}

\begin{styleBody}
\textbf{kbh-C2: }An alveolar flap is realized as [[290?]] word-initially (Howard 1967: 78).
\end{styleBody}

\begin{styleBody}
\textbf{kbh-C3: }A voiceless alveolar fricative is realized as voiced adjacent to a voiced alveolar stop (Howard 1967: 78).
\end{styleBody}

\begin{styleBody}
\textbf{kbh-C4: }A voiced bilabial stop may be spirantized in all environments (Howard 1967: 77).
\end{styleBody}

\begin{styleBody}
\textbf{Morphology}
\end{styleBody}

\begin{styleBody}
(adequate texts unavailable)
\end{styleBody}

\clearpage\begin{styleBody}
\textbf{[kbk] }\ \ \textbf{\textsc{Grass Koiari}}\textbf{\ \ \ \ }Koiarian, \textit{Koiaric} (Papua New Guinea)
\end{styleBody}

\begin{styleBody}
\textbf{References consulted: }Dutton (1996)
\end{styleBody}

\begin{styleBody}
\textbf{Sound inventory}
\end{styleBody}

\begin{styleBody}
\textbf{C phoneme inventory:} /b t d k [261?] [278?] $\beta $ s h m n l j/
\end{styleBody}

\begin{styleBody}
\textbf{N consonant phonemes:} 13
\end{styleBody}

\begin{styleBody}
\textbf{Geminates:} N/A
\end{styleBody}

\begin{styleBody}
\textbf{Voicing contrasts:} Obstruents
\end{styleBody}

\begin{styleBody}
\textbf{Places:} Bilabial, Alveolar, Velar, Glottal
\end{styleBody}

\begin{styleBody}
\textbf{Manners:} Stop, Fricative, Nasal, Lateral approximant, Central approximant
\end{styleBody}

\begin{styleBody}
\textbf{N elaborations:} 1
\end{styleBody}

\begin{styleBody}
\textbf{Elaborations:} Voiced fricatives/affricates
\end{styleBody}

\begin{styleBody}
\textbf{V phoneme inventory:} /i e a o u/
\end{styleBody}

\begin{styleBody}
\textbf{N vowel qualities:} 5
\end{styleBody}

\begin{styleBody}
\textbf{Diphthongs or vowel sequences:} None
\end{styleBody}

\begin{styleBody}
\textbf{Contrastive length:} None
\end{styleBody}

\begin{styleBody}
\textbf{Contrastive nasalization:} None
\end{styleBody}

\begin{styleBody}
\textbf{Other contrasts:} N/A
\end{styleBody}

\begin{styleBody}
\textbf{Notes:} [$\beta $] alternates with [w], with [w] occurring before back vowels; perhaps it would be better analyzed as /w/.
\end{styleBody}

\begin{styleBody}
\textbf{Syllable structure}
\end{styleBody}

\begin{styleBody}
\textbf{Complexity Category:} Simple
\end{styleBody}

\begin{styleBody}
\textbf{Canonical syllable structure:} (C)V\textbf{ }(Dutton 1996: 7)
\end{styleBody}

\begin{styleBody}
\textbf{Size of maximal onset:} 1
\end{styleBody}

\begin{styleBody}
\textbf{Size of maximal coda:} N/A
\end{styleBody}

\begin{styleBody}
\textbf{Onset obligatory:} No
\end{styleBody}

\begin{styleBody}
\textbf{Coda obligatory:} N/A
\end{styleBody}

\begin{styleBody}
\textbf{Vocalic nucleus patterns:} Short vowels
\end{styleBody}

\begin{styleBody}
\textbf{Syllabic consonant patterns:} N/A
\end{styleBody}

\begin{styleBody}
\textbf{Size of maximal word{}-marginal sequences with syllabic obstruents:} N/A
\end{styleBody}

\begin{styleBody}
\textbf{Predictability of syllabic consonants:} N/A
\end{styleBody}

\begin{styleBody}
\textbf{Morphological constituency of maximal syllable margin:} N/A
\end{styleBody}

\begin{styleBody}
\textbf{Morphological pattern of syllabic consonants:} N/A
\end{styleBody}

\begin{styleBody}
\textbf{Onset restrictions: }All consonants occur.
\end{styleBody}

\begin{styleBody}
\textbf{Suprasegmentals}
\end{styleBody}

\begin{styleBody}
\textbf{Tone:} No
\end{styleBody}

\begin{styleBody}
\textbf{Word stress:} Yes
\end{styleBody}

\begin{styleBody}
\textbf{Stress placement:} Morphologically or Lexically Conditioned
\end{styleBody}

\begin{styleBody}
\textbf{Phonetic processes conditioned by stress:} Consonant Allophony in Stressed Syllables
\end{styleBody}

\begin{styleBody}
\textbf{Differences in phonological properties of stressed and unstressed syllables:} (None) 
\end{styleBody}

\begin{styleBody}
\textbf{Phonetic correlates of stress: }Vowel duration (impressionistic), Pitch (impressionistic)
\end{styleBody}

\begin{styleBody}
\textbf{Vowel reduction processes}
\end{styleBody}

\begin{styleBody}
(none reported)
\end{styleBody}

\begin{styleBody}
\textbf{Consonant allophony processes}
\end{styleBody}

\begin{styleBody}
\textbf{kbk-C1: }A voiced bilabial fricative is realized as a glide preceding non-front vowels (Dutton 1996).
\end{styleBody}

\begin{styleBody}
\textbf{kbk-C2: }A voiceless bilabial fricative may be realized as [p] word-initially preceding a back vowel (Dutton 1996).
\end{styleBody}

\begin{styleBody}
\textbf{kbk-C3: }An alveolar lateral approximant is realized as a flap preceding front vowels (Dutton 1996).
\end{styleBody}

\begin{styleBody}
\textbf{Morphology}
\end{styleBody}

\begin{styleBody}
\textbf{Text:} “Maruba” (Dutton 1996: 72-76)
\end{styleBody}

\begin{styleBody}
\textbf{Synthetic index: }1.5 morphemes/word (488 morphemes, 318 words)
\end{styleBody}

\clearpage\begin{styleBody}
\textbf{[kca] }\ \ \textbf{\textsc{Eastern Khanty}}\textbf{\ \ }Uralic, \textit{Khantyic }(Russia)
\end{styleBody}

\begin{styleBody}
\textbf{References consulted: }Filchenko (2007), Andrey Filchenko (p.c.)
\end{styleBody}

\begin{styleBody}
\textbf{Sound inventory}
\end{styleBody}

\begin{styleBody}
\textbf{C phoneme inventory:} /p t c k q t[361?][283?] s [263?] m n n[320?] [272?] ŋ r l [28E?] [29F?] w j/
\end{styleBody}

\begin{styleBody}
\textbf{N consonant phonemes:} 19
\end{styleBody}

\begin{styleBody}
\textbf{Geminates:} N/A
\end{styleBody}

\begin{styleBody}
\textbf{Voicing contrasts:} None
\end{styleBody}

\begin{styleBody}
\textbf{Places:} Bilabial, Alveolar, Palato-alveolar, Palatal, Velar, Uvular
\end{styleBody}

\begin{styleBody}
\textbf{Manners:} Stop, Affricate, Fricative, Nasal, Trill, Central approximant, Lateral approximant
\end{styleBody}

\begin{styleBody}
\textbf{N elaborations:} 4
\end{styleBody}

\begin{styleBody}
\textbf{Elaborations:} Voiced fricatives/affricates, Palato-alveolar, Uvular
\end{styleBody}

\begin{styleBody}
\textbf{V phoneme inventory:} /i y e ø œ æ [268?] [259?][318?] [259?] a [254?] o u/
\end{styleBody}

\begin{styleBody}
\textbf{N vowel qualities:} 13
\end{styleBody}

\begin{styleBody}
\textbf{Diphthongs or vowel sequences:} None
\end{styleBody}

\begin{styleBody}
\textbf{Contrastive length:} None
\end{styleBody}

\begin{styleBody}
\textbf{Contrastive nasalization:} None
\end{styleBody}

\begin{styleBody}
\textbf{Other contrasts:} N/A
\end{styleBody}

\begin{styleBody}
\textbf{Notes:} /r/ is described as an ‘alveolar-palatal trill’. /k q [263?] ŋ/ are described as ‘cacuminal’ (retroflex?). /t[361?][26C?]/ occurs in Upper Yugan dialect only. /[259?] [259?][318?] ø [254?]/ are ‘reduced’ vowels, produced as lax, weak, and short, commonly occurring in unstressed syllables.
\end{styleBody}

\begin{styleBody}
\textbf{Syllable structure}
\end{styleBody}

\begin{styleBody}
\textbf{Complexity Category:} Moderately Complex
\end{styleBody}

\begin{styleBody}
\textbf{Canonical syllable structure:} (C)V(C)(C)\textbf{ }(Filchenko 2007: 53-7)
\end{styleBody}

\begin{styleBody}
\textbf{Size of maximal onset:} 1
\end{styleBody}

\begin{styleBody}
\textbf{Size of maximal coda:} 2
\end{styleBody}

\begin{styleBody}
\textbf{Onset obligatory:} No
\end{styleBody}

\begin{styleBody}
\textbf{Coda obligatory:} No
\end{styleBody}

\begin{styleBody}
\textbf{Vocalic nucleus patterns:} Short vowels
\end{styleBody}

\begin{styleBody}
\textbf{Syllabic consonant patterns:} N/A
\end{styleBody}

\begin{styleBody}
\textbf{Size of maximal word{}-marginal sequences with syllabic obstruents:} N/A
\end{styleBody}

\begin{styleBody}
\textbf{Predictability of syllabic consonants:} N/A
\end{styleBody}

\begin{styleBody}
\textbf{Morphological constituency of maximal syllable margin:} Morphologically Complex (Coda)
\end{styleBody}

\begin{styleBody}
\textbf{Morphological pattern of syllabic consonants:} N/A
\end{styleBody}

\begin{styleBody}
\textbf{Onset restrictions: }C\textsubscript{1} may be any consonant except /ŋ/. C\textsubscript{2} is always a glide /$\beta [31E?]$/ or /j/.
\end{styleBody}

\begin{styleBody}
\textbf{Coda restrictions: }All consonants except for glides and glottal fricative /h/ are attested. 
\end{styleBody}

\begin{styleBody}
\textbf{Notes: }Canonical syllable structure includes coda clusters. These come about through derivation or inflection, and vowel epenthesis is employed “robustly and productively” such that most are not realized as coda clusters. However, derived coda clusters with a sonorant preceding a homorganic stop are more likely to be retained (e.g. \textit{lol-t} ‘crack, dent’-PL). Description suggests that occurrence of clusters is a matter of probability, and there is an “extremely low probability of consonant clusters at the morphemic edges, word-initial, and word-final position” (Filchenko 2007: 55). I therefore analyze this language as having Moderately Complex syllable structure.
\end{styleBody}

\begin{styleBody}
\textbf{Suprasegmentals}
\end{styleBody}

\begin{styleBody}
\textbf{Tone:} No
\end{styleBody}

\begin{styleBody}
\textbf{Word stress:} Yes
\end{styleBody}

\begin{styleBody}
\textbf{Stress placement:} Weight-Sensitive
\end{styleBody}

\begin{styleBody}
\textbf{Phonetic processes conditioned by stress:} Vowel Reduction
\end{styleBody}

\begin{styleBody}
\textbf{Phonetic correlates of stress: }Not described
\end{styleBody}

\begin{styleBody}
\textbf{Differences in phonological properties of stressed and unstressed syllables:} Vowel Quality Contrasts (see notes)
\end{styleBody}

\begin{styleBody}
\textbf{Notes:} /[259?] [259?][318?] ø [254?]/ commonly occur in unstressed syllables, and their occurrence in a word may complicate typical patterns of stress assignment (if initial syllable in bisyllabic word has reduced vowel, stress shifts to next syllable).
\end{styleBody}

\begin{styleBody}
\textbf{Vowel reduction processes}
\end{styleBody}

\begin{styleBody}
\textbf{kca-R1:} Word-final vowels, particularly /[259?]/ and /ø/, are under-articulated, reduced, devoiced, or deleted (Filchenko 2007: 56).
\end{styleBody}

\begin{styleBody}
\textbf{Consonant allophony processes}
\end{styleBody}

\begin{styleBody}
\textbf{kca-C1: }Voiceless velar stop and voiced velar fricative are realized as uvulars adjacent to back vowels (Filchenko 2007: 41).
\end{styleBody}

\begin{styleBody}
\textbf{kca-C2: }A labiovelar approximant may be realized as a bilabial stop following /m/ (Filchenko 2007: 44-45).
\end{styleBody}

\begin{styleBody}
\textbf{kca-C3: }A voiced velar fricative may be realized as a velar stop adjacent to /t k q t[361?][283?]/ (Filchenko 2007: 45).
\end{styleBody}

\begin{styleBody}
\textbf{kca-C4: }A voiced velar fricative may be realized as a velar stop intervocalically (Filchenko 2007: 45).
\end{styleBody}

\begin{styleBody}
\textbf{kca-C5: }Labial and dorsal consonants are palatalized preceding front vowels (Filchenko 2007: 37).
\end{styleBody}

\begin{styleBody}
\textbf{kca-C6: }A voiced velar fricative is realized as a labiovelar approximant following /u/ (Filchenko 2007: 45-6).
\end{styleBody}

\begin{styleBody}
\textbf{Morphology}
\end{styleBody}

\begin{styleBody}
\textbf{Text:} “A bear in the river” (Filchenko 2007: 582-588)
\end{styleBody}

\begin{styleBody}
\textbf{Synthetic index: }1.9 morphemes/word (649 morphemes, 342 words)
\end{styleBody}

\clearpage\begin{styleBody}
\textbf{[ket] }\ \ \textbf{\textsc{Ket}}\textbf{\ \ }Yeniseian, \textit{Northern Yeniseian} (Russia)
\end{styleBody}

\begin{styleBody}
\textbf{References consulted: }Georg (2007), Vajda (2000)
\end{styleBody}

\begin{styleBody}
\textbf{Sound inventory}
\end{styleBody}

\begin{styleBody}
\textbf{C phoneme inventory:} /b t d k q s h m n ŋ l j/
\end{styleBody}

\begin{styleBody}
\textbf{N consonant phonemes:} 12
\end{styleBody}

\begin{styleBody}
\textbf{Geminates:} N/A
\end{styleBody}

\begin{styleBody}
\textbf{Voicing contrasts:} Obstruents
\end{styleBody}

\begin{styleBody}
\textbf{Places:} Bilabial, Alveolar, Velar, Uvular, Glottal
\end{styleBody}

\begin{styleBody}
\textbf{Manners:} Stop, Fricative, Nasal, Central approximant, Lateral approximant
\end{styleBody}

\begin{styleBody}
\textbf{N elaborations:} 1
\end{styleBody}

\begin{styleBody}
\textbf{Elaborations:} Uvular
\end{styleBody}

\begin{styleBody}
\textbf{V phoneme inventory:} /i e [268?] [259?] a o u/
\end{styleBody}

\begin{styleBody}
\textbf{N vowel qualities:} 7
\end{styleBody}

\begin{styleBody}
\textbf{Diphthongs or vowel sequences:} None
\end{styleBody}

\begin{styleBody}
\textbf{Contrastive length:} None
\end{styleBody}

\begin{styleBody}
\textbf{Contrastive nasalization:} None
\end{styleBody}

\begin{styleBody}
\textbf{Other contrasts:} N/A
\end{styleBody}

\begin{styleBody}
\textbf{Syllable structure}
\end{styleBody}

\begin{styleBody}
\textbf{Complexity Category:} Complex
\end{styleBody}

\begin{styleBody}
\textbf{Canonical syllable structure:} (C)(C)V(C)(C)(C)\textbf{ }(Georg 2007: 80-4)
\end{styleBody}

\begin{styleBody}
\textbf{Size of maximal onset:} 2
\end{styleBody}

\begin{styleBody}
\textbf{Size of maximal coda:} 3
\end{styleBody}

\begin{styleBody}
\textbf{Onset obligatory:} No
\end{styleBody}

\begin{styleBody}
\textbf{Coda obligatory:} No
\end{styleBody}

\begin{styleBody}
\textbf{Vocalic nucleus patterns:} Short vowels
\end{styleBody}

\begin{styleBody}
\textbf{Syllabic consonant patterns:} Nasal
\end{styleBody}

\begin{styleBody}
\textbf{Size of maximal word{}-marginal sequences with syllabic obstruents:} N/A
\end{styleBody}

\begin{styleBody}
\textbf{Predictability of syllabic consonants:} Predictable from word/consonantal context
\end{styleBody}

\begin{styleBody}
\textbf{Morphological constituency of maximal syllable margin:} Morphologically Complex (Onset), or Both patterns (Coda)
\end{styleBody}

\begin{styleBody}
\textbf{Morphological pattern of syllabic consonants:} Grammatical items
\end{styleBody}

\begin{styleBody}
\textbf{Onset restrictions: }Apparently all consonants occur in simple onsets. Onset clusters have /b k d/ as C\textsubscript{1} and apparently any (?) consonant as C\textsubscript{2}.
\end{styleBody}

\begin{styleBody}
\textbf{Coda restrictions: }All consonants except /h/ occur as simple codas. Biconsonantal codas seem fairly unrestricted, though most end in /s/ (nominalizing suffix), /n/, or /ŋ/ (plural suffixes). Other biconsonantal codas such as /tl/, /ŋl/, /nt/, /kt/, and /qt/ may occur within roots. Triconsonantal codas always have a continuant as the second member and /s/ as the third member.
\end{styleBody}

\begin{styleBody}
\textbf{Notes: }Canonical syllable structure differs here from Georg’s reported patterns, which include two-consonant codas. In discussion on p. 84 he gives example of triconsonantal coda, which may occur when the nominalizer suffix \textit{{}-s} is added to a coda ending in a continuant. All examples of biconsonantal onsets have stops as C\textsubscript{2} but it would seem based on patterns reported that any stem-initial C could occur in this position. What is written as /[294?]/ in Georg’s transcriptions marks Tone 2 and shouldn’t be analyzed as a consonant.
\end{styleBody}

\begin{styleBody}
\textbf{Suprasegmentals}
\end{styleBody}

\begin{styleBody}
\textbf{Tone:} Yes
\end{styleBody}

\begin{styleBody}
\textbf{Word stress:} No
\end{styleBody}

\begin{styleBody}
\textbf{Notes: }Falling tones are ‘acoustically close to a dynamic stress’.
\end{styleBody}

\begin{styleBody}
\textbf{Vowel reduction processes}
\end{styleBody}

\begin{styleBody}
\textbf{ket-R1:} An unstressed high front vowel /i/ in the sequence VCiCV is syncopated, if no non-permitted consonant cluster results (Georg 2007: 214; “stress” here refers to tonal contour).
\end{styleBody}

\begin{styleBody}
\textbf{ket-R2:} Vowels with second tone lose their tone except in absolute final position in phrase (Vajda 2000: 15-16).
\end{styleBody}

\begin{styleBody}
\textbf{ket-R3:} In post-tonal (non-initial) syllables and the second syllable of a disyllabic pitch contour, there is free variation between vowels and higher counterparts (Vajda 2000: 11).
\end{styleBody}

\begin{styleBody}
\textbf{Consonant allophony processes}
\end{styleBody}

\begin{styleBody}
\textbf{ket-C1: }A voiceless alveolar fricative is sometimes realized as a palato-alveolar fricative or affricate preceding front vowels (Georg 2007: 78).
\end{styleBody}

\begin{styleBody}
\textbf{ket-C2: }A voiceless velar stop is realized as a voiced velar fricative intervocalically (Georg 2007: 75).
\end{styleBody}

\begin{styleBody}
\textbf{ket-C3: }A consonant is voiced preceding another consonant (Georg 2007: 75).
\end{styleBody}

\begin{styleBody}
\textbf{ket-C4: }A voiced alveolar stop is realized as a flap intervocalically in some dialects (Georg 2007: 76).
\end{styleBody}

\begin{styleBody}
\textbf{ket-C5: }Voiced bilabial stop, voiceless velar and uvular stops are spirantized intervocalically (Georg 2007: 75-8).
\end{styleBody}

\begin{styleBody}
\textbf{Morphology}
\end{styleBody}

\begin{styleBody}
\textbf{Text:} “Two brothers” (Vajda 2004: 92-97)
\end{styleBody}

\begin{styleBody}
\textbf{Synthetic index: }2.3 morphemes/word (602 morphemes, 267 words)
\end{styleBody}

\clearpage\begin{styleBody}
\textbf{[kew] }\ \ \textbf{\textsc{East Kewa}}\textbf{\ \ }Nuclear Trans New Guinea, \textit{Enga-Kewa-Huli} (Papua New Guinea)
\end{styleBody}

\begin{styleBody}
\textbf{References consulted: }Franklin (1971), Franklin \& Franklin (1962, 1978)
\end{styleBody}

\begin{styleBody}
\textbf{Sound inventory}
\end{styleBody}

\begin{styleBody}
\textbf{C phoneme inventory:} /\textsuperscript{m}b t \textsuperscript{n}d c k [278?] s x m n [272?] [27A?] [27E?] w j/
\end{styleBody}

\begin{styleBody}
\textbf{N consonant phonemes:} 15
\end{styleBody}

\begin{styleBody}
\textbf{Geminates:} N/A
\end{styleBody}

\begin{styleBody}
\textbf{Voicing contrasts:} Obstruents
\end{styleBody}

\begin{styleBody}
\textbf{Places:} Bilabial, Alveolar, Palatal, Velar
\end{styleBody}

\begin{styleBody}
\textbf{Manners:} Stop, Fricative, Nasal, Flap/Tap, Lateral Flap/Tap, Central Approximant
\end{styleBody}

\begin{styleBody}
\textbf{N elaborations:} 1
\end{styleBody}

\begin{styleBody}
\textbf{Elaborations:} Prenasalization
\end{styleBody}

\begin{styleBody}
\textbf{V phoneme inventory:} /i e [259?] a o u/
\end{styleBody}

\begin{styleBody}
\textbf{N vowel qualities:} 6
\end{styleBody}

\begin{styleBody}
\textbf{Diphthongs or vowel sequences:} None
\end{styleBody}

\begin{styleBody}
\textbf{Contrastive length:} None
\end{styleBody}

\begin{styleBody}
\textbf{Contrastive nasalization:} None
\end{styleBody}

\begin{styleBody}
\textbf{Other contrasts:} N/A
\end{styleBody}

\begin{styleBody}
\textbf{Notes:} Palatal consonants are present in Eastern Kewa only, and do not occur before high vowels (Franklin \& Franklin 1978: 21). /e/ reported in 1971, 1978 references, but not 1962 references. /a/ produced slightly longer, but has no short counterpart.
\end{styleBody}

\begin{styleBody}
\textbf{Syllable structure}
\end{styleBody}

\begin{styleBody}
\textbf{Complexity Category:} Simple
\end{styleBody}

\begin{styleBody}
\textbf{Canonical syllable structure:} (C)V\textbf{ }(Franklin 1971: 11-12)
\end{styleBody}

\begin{styleBody}
\textbf{Size of maximal onset:} 1
\end{styleBody}

\begin{styleBody}
\textbf{Size of maximal coda:} N/A
\end{styleBody}

\begin{styleBody}
\textbf{Onset obligatory:} No
\end{styleBody}

\begin{styleBody}
\textbf{Coda obligatory:} N/A
\end{styleBody}

\begin{styleBody}
\textbf{Vocalic nucleus patterns:} Short vowels
\end{styleBody}

\begin{styleBody}
\textbf{Syllabic consonant patterns:} N/A
\end{styleBody}

\begin{styleBody}
\textbf{Size of maximal word{}-marginal sequences with syllabic obstruents:} N/A
\end{styleBody}

\begin{styleBody}
\textbf{Predictability of syllabic consonants:} N/A
\end{styleBody}

\begin{styleBody}
\textbf{Morphological constituency of maximal syllable margin:} N/A
\end{styleBody}

\begin{styleBody}
\textbf{Morphological pattern of syllabic consonants:} N/A
\end{styleBody}

\begin{styleBody}
\textbf{Onset restrictions: }All consonants occur.
\end{styleBody}

\begin{styleBody}
\textbf{Suprasegmentals}
\end{styleBody}

\begin{styleBody}
\textbf{Tone:} Yes
\end{styleBody}

\begin{styleBody}
\textbf{Word stress:} Yes
\end{styleBody}

\begin{styleBody}
\textbf{Stress placement:} Morphologically or Lexically Conditioned
\end{styleBody}

\begin{styleBody}
\textbf{Phonetic processes conditioned by stress:} (None)
\end{styleBody}

\begin{styleBody}
\textbf{Differences in phonological properties of stressed and unstressed syllables:} Vowel Quality Contrasts
\end{styleBody}

\begin{styleBody}
\textbf{Phonetic correlates of stress: }Vowel duration (impressionistic), Pitch (impressionistic)
\end{styleBody}

\begin{styleBody}
\textbf{Notes: }Pitch as a correlate of stress here indicates that perceptual or auditory height of a tone may be conditioned by stress placement.
\end{styleBody}

\begin{styleBody}
\textbf{Vowel reduction processes}
\end{styleBody}

\begin{styleBody}
(none reported)
\end{styleBody}

\begin{styleBody}
\textbf{Consonant allophony processes}
\end{styleBody}

\begin{styleBody}
\textbf{kew-C1: }A voiceless alveolar fricative is realized as palato-alveolar preceding a high vowel (Franklin 1971: 24).
\end{styleBody}

\begin{styleBody}
\textbf{kew-C2: }A voiceless bilabial or velar fricative may be realized as an affricate utterance-initially (Franklin 1971: 24).
\end{styleBody}

\begin{styleBody}
\textbf{kew-C3: }Fricatives may be voiced in fast speech (Franklin 1971: 24).
\end{styleBody}

\begin{styleBody}
\textbf{Morphology}
\end{styleBody}

\begin{styleBody}
\textbf{Text: }“East Kewa” (lines 1-13, 32-58, Franklin \& Franklin 1978: 483-487)
\end{styleBody}

\begin{styleBody}
\textbf{Synthetic index: }1.4 morphemes/word (399 morphemes, 278 words)
\end{styleBody}

\clearpage\begin{styleBody}
\textbf{[khc] }\ \ \textbf{\textsc{Tukang Besi North}}\textbf{\ \ }Austronesian, \textit{Malayo-Polynesian} (Indonesia)
\end{styleBody}

\begin{styleBody}
\textbf{References consulted: }Donohue (1999)
\end{styleBody}

\begin{styleBody}
\textbf{Sound inventory}
\end{styleBody}

\begin{styleBody}
\textbf{C phoneme inventory:} /p t[32A?] k [261?] [294?] \textsuperscript{m}p \textsuperscript{m}b \textsuperscript{n}t[32A?] \textsuperscript{n}d[32A?] \textsuperscript{ŋ}k \textsuperscript{ŋ}[261?] [253?] [257?][32A?] $\beta $ s h \textsuperscript{n}s m n[32A?] ŋ r l/
\end{styleBody}

\begin{styleBody}
\textbf{N consonant phonemes:} 22
\end{styleBody}

\begin{styleBody}
\textbf{Geminates:} N/A
\end{styleBody}

\begin{styleBody}
\textbf{Voicing contrasts:} Obstruents
\end{styleBody}

\begin{styleBody}
\textbf{Places:} Bilabial, Dental, Alveolar, Velar, Glottal
\end{styleBody}

\begin{styleBody}
\textbf{Manners:} Stop, Fricative, Nasal, Trill, Lateral Approximant
\end{styleBody}

\begin{styleBody}
\textbf{N elaborations:} 3
\end{styleBody}

\begin{styleBody}
\textbf{Elaborations:} Voiced fricatives/affricates, Prenasalization, Implosive
\end{styleBody}

\begin{styleBody}
\textbf{V phoneme inventory:} /i [25B?] a o [26F?]/
\end{styleBody}

\begin{styleBody}
\textbf{N vowel qualities:} 5
\end{styleBody}

\begin{styleBody}
\textbf{Diphthongs or vowel sequences:} None
\end{styleBody}

\begin{styleBody}
\textbf{Contrastive length:} None
\end{styleBody}

\begin{styleBody}
\textbf{Contrastive nasalization:} None
\end{styleBody}

\begin{styleBody}
\textbf{Other contrasts:} N/A
\end{styleBody}

\begin{styleBody}
\textbf{Notes:} There are also palatal phonemes loaned from Indonesian/Trade Malay. Author presents distributional/reduplication/syllabification evidence for analyzing prenasalized stops as unitary rather than sequences in reduplication processes.
\end{styleBody}

\begin{styleBody}
\textbf{Syllable structure}
\end{styleBody}

\begin{styleBody}
\textbf{Complexity Category:} Simple
\end{styleBody}

\begin{styleBody}
\textbf{Canonical syllable structure:} (C)V\textbf{ }(Donohue 1999: 30-1)
\end{styleBody}

\begin{styleBody}
\textbf{Size of maximal onset:} 1
\end{styleBody}

\begin{styleBody}
\textbf{Size of maximal coda:} N/A
\end{styleBody}

\begin{styleBody}
\textbf{Onset obligatory:} No
\end{styleBody}

\begin{styleBody}
\textbf{Coda obligatory:} N/A
\end{styleBody}

\begin{styleBody}
\textbf{Vocalic nucleus patterns:} Short vowels
\end{styleBody}

\begin{styleBody}
\textbf{Syllabic consonant patterns:} N/A
\end{styleBody}

\begin{styleBody}
\textbf{Size of maximal word{}-marginal sequences with syllabic obstruents:} N/A
\end{styleBody}

\begin{styleBody}
\textbf{Predictability of syllabic consonants:} N/A
\end{styleBody}

\begin{styleBody}
\textbf{Morphological constituency of maximal syllable margin:} N/A
\end{styleBody}

\begin{styleBody}
\textbf{Morphological pattern of syllabic consonants:} N/A
\end{styleBody}

\begin{styleBody}
\textbf{Onset restrictions: }All consonants occur.
\end{styleBody}

\begin{styleBody}
\textbf{Suprasegmentals}
\end{styleBody}

\begin{styleBody}
\textbf{Tone:} No
\end{styleBody}

\begin{styleBody}
\textbf{Word stress:} Yes
\end{styleBody}

\begin{styleBody}
\textbf{Stress placement:} Fixed
\end{styleBody}

\begin{styleBody}
\textbf{Phonetic processes conditioned by stress:} Consonant Allophony in Unstressed Syllables
\end{styleBody}

\begin{styleBody}
\textbf{Differences in phonological properties of stressed and unstressed syllables:} (None)
\end{styleBody}

\begin{styleBody}
\textbf{Phonetic correlates of stress: }Pitch (impressionistic)
\end{styleBody}

\begin{styleBody}
\textbf{Notes:} Donohue speculates that Tukang Besi has an incipient pitch accent system that’s developing through the regularization of phonetic properties of older non-contrastive stress system (1999: 34).
\end{styleBody}

\begin{styleBody}
\textbf{Vowel reduction processes}
\end{styleBody}

\begin{styleBody}
\textbf{khc-R1:} In casual speech, any word-final vowel can delete or become voiceless after a voiceless consonant (Donohue 1999: 23).
\end{styleBody}

\begin{styleBody}
\textbf{Consonant allophony processes}
\end{styleBody}

\begin{styleBody}
\textbf{khc-C1: }Non-implosive bilabial stops may be realized as affricates preceding /a o/ (Donohue 1999: 16).
\end{styleBody}

\begin{styleBody}
\textbf{khc-C2: }Voiceless glottal fricative is realized as voiceless bilabial fricative preceding /u/ (Donohue 1999: 19).
\end{styleBody}

\begin{styleBody}
\textbf{khc-C3: }A voiceless velar stop is realized as fronted preceding /i/ (Donohue 1999: 19).
\end{styleBody}

\begin{styleBody}
\textbf{khc-C4: }An alveolar trill may be realized as an alveolar, lateral, or retroflex flap intervocalically in some dialects and in casual speech (Donohue 1999: 18).
\end{styleBody}

\begin{styleBody}
\textbf{khc-C5: }An alveolar lateral approximant my be realized as a lateral or retroflex flap following a non-front vowel in some dialects and in casual speech (Donohue 1999: 18).
\end{styleBody}

\begin{styleBody}
\textbf{khc-C6: }A voiced velar stop is spirantized in lax environments, including between two unstressed vowels (Donohue 1999: 27).
\end{styleBody}

\begin{styleBody}
\textbf{khc-C7: }An implosive bilabial stop is realized as a fricative intervocalically (Donohue 1999: 16).
\end{styleBody}

\begin{styleBody}
\textbf{khc-C8: }Non-implosive bilabial stops may be spirantized preceding non-high back vowels (Donohue 1999: 16).
\end{styleBody}

\begin{styleBody}
\textbf{Morphology}
\end{styleBody}

\begin{styleBody}
\textbf{Text:} “The heron and the monkey” (Donohue 1999: 516-520)
\end{styleBody}

\begin{styleBody}
\textbf{Synthetic index: }1.5 morphemes/word (605 morphemes, 398 words)
\end{styleBody}

\clearpage\begin{styleBody}
\textbf{[khr] }\ \ \textbf{\textsc{Kharia}}\textbf{\ \ \ \ }Austroasiatic, \textit{Mundaic }(India)
\end{styleBody}

\begin{styleBody}
\textbf{References consulted: }Peterson (2011)
\end{styleBody}

\begin{styleBody}
\textbf{Sound inventory}
\end{styleBody}

\begin{styleBody}
\textbf{C phoneme inventory:} /p b t[32A?] d[32A?] [288?] [256?] c [25F?] k [261?] [294?] b[2B0?] t[32A?][2B0?] d[32A?][2B0?] [288?][2B0?] [256?][2B0?] c[2B0?] [25F?][2B0?] k[2B0?] [261?][2B0?] f s h m n[32A?] [272?] ŋ [27E?][32A?] [27D?] l[32A?] w j/
\end{styleBody}

\begin{styleBody}
\textbf{N consonant phonemes:} 32
\end{styleBody}

\begin{styleBody}
\textbf{Geminates:} N/A
\end{styleBody}

\begin{styleBody}
\textbf{Voicing contrasts:} Obstruents
\end{styleBody}

\begin{styleBody}
\textbf{Places:} Bilabial, Labiodental, Dental, Retroflex, Palatal, Velar, Glottal
\end{styleBody}

\begin{styleBody}
\textbf{Manners:} Stop, Fricative, Nasal, Flap/Tap, Central approximant, Lateral approximant
\end{styleBody}

\begin{styleBody}
\textbf{N elaborations:} 4
\end{styleBody}

\begin{styleBody}
\textbf{Elaborations:} Breathy voice, Post-aspiration, Labiodental, Retroflex
\end{styleBody}

\begin{styleBody}
\textbf{V phoneme inventory:} /i [25B?] a [254?] u/
\end{styleBody}

\begin{styleBody}
\textbf{N vowel qualities:} 5
\end{styleBody}

\begin{styleBody}
\textbf{Diphthongs or vowel sequences:} None
\end{styleBody}

\begin{styleBody}
\textbf{Contrastive length:} None
\end{styleBody}

\begin{styleBody}
\textbf{Contrastive nasalization:} None
\end{styleBody}

\begin{styleBody}
\textbf{Other contrasts:} N/A
\end{styleBody}

\begin{styleBody}
\textbf{Notes:} Palatal stops often realized as affricates. The retroflex consonants are most often realized as post-alveolars. /[27D?]/ is marginally phonemic, but there is a minimal pair distinguishing it from /[256?]/. [[294?]] is also described as extremely marginal, does not seem to contrast with anything and is predictable in its distribution. /[25B?] [254?]/ raise to /e o/ when lengthened. Status of diphthongs /ae ao ou oi ui/ doubtful to Peterson, as they do not occur before codas in the native vocabulary. Therefore he analyzes these as V+glide. Other authors consider nasalization to be marginally phonemic, but Peterson does not (2011: 27).
\end{styleBody}

\begin{styleBody}
\textbf{Syllable structure}
\end{styleBody}

\begin{styleBody}
\textbf{Complexity Category:} Moderately Complex
\end{styleBody}

\begin{styleBody}
\textbf{Canonical syllable structure:} (C)V(C)\textbf{ }(Peterson 2011: 32-3)
\end{styleBody}

\begin{styleBody}
\textbf{Size of maximal onset:} 1
\end{styleBody}

\begin{styleBody}
\textbf{Size of maximal coda:} 1
\end{styleBody}

\begin{styleBody}
\textbf{Onset obligatory:} No
\end{styleBody}

\begin{styleBody}
\textbf{Coda obligatory:} No
\end{styleBody}

\begin{styleBody}
\textbf{Vocalic nucleus patterns:} Short vowels, Diphthongs
\end{styleBody}

\begin{styleBody}
\textbf{Syllabic consonant patterns:} Nasal
\end{styleBody}

\begin{styleBody}
\textbf{Size of maximal word{}-marginal sequences with syllabic obstruents:} N/A
\end{styleBody}

\begin{styleBody}
\textbf{Predictability of syllabic consonants:} Predictable from word/consonantal context
\end{styleBody}

\begin{styleBody}
\textbf{Morphological constituency of maximal syllable margin:} N/A
\end{styleBody}

\begin{styleBody}
\textbf{Morphological pattern of syllabic consonants:} Lexical items
\end{styleBody}

\begin{styleBody}
\textbf{Onset restrictions: }C\textsubscript{1} may be any consonant except /[27D?]/ and /ŋ/.
\end{styleBody}

\begin{styleBody}
\textbf{Coda restrictions: }/s/ and /h/ do not occur in native codas. Voicing, aspiration, and dental/retroflex contrasts neutralized in coda.
\end{styleBody}

\begin{styleBody}
\textbf{Suprasegmentals}
\end{styleBody}

\begin{styleBody}
\textbf{Tone:} No
\end{styleBody}

\begin{styleBody}
\textbf{Word stress:} No
\end{styleBody}

\begin{styleBody}
\textbf{Notes: }Word-level rising prosodic pattern defines the phonological word, but any syllable may be more prominent with respect to intensity.
\end{styleBody}

\begin{styleBody}
\textbf{Vowel reduction processes}
\end{styleBody}

\begin{styleBody}
(none reported)
\end{styleBody}

\begin{styleBody}
\textbf{Consonant allophony processes}
\end{styleBody}

\begin{styleBody}
\textbf{khr-C1: }A voiced velar stop is realized as a glottal stop syllable-finally (Peterson 2011: 29).
\end{styleBody}

\begin{styleBody}
\textbf{Morphology}
\end{styleBody}

\begin{styleBody}
\textbf{Text:} “The nine totems” (first 8 pages, Peterson 2011: 439-446)
\end{styleBody}

\begin{styleBody}
\textbf{Synthetic index: }1.5 morphemes/word (604 morphemes, 399 words)
\end{styleBody}

\clearpage\begin{styleBody}
\textbf{[kjn] }\ \ \textbf{\textsc{Kunjen (Oykangand dialect)}}\textbf{\ \ }Pama-Nyungan, \textit{Paman} (Australia)
\end{styleBody}

\begin{styleBody}
\textbf{References consulted: }Sommer (1969, 1981)
\end{styleBody}

\begin{styleBody}
\textbf{Sound inventory}
\end{styleBody}

\begin{styleBody}
\textbf{C phoneme inventory:} /p t[32A?] t c k p[2B0?] t[32A?][2B0?] t[2B0?] c[2B0?] k[2B0?] f ð [263?] m n[32A?] n [272?] ŋ r[325?] [279?] l w j/
\end{styleBody}

\begin{styleBody}
\textbf{N consonant phonemes:} 23
\end{styleBody}

\begin{styleBody}
\textbf{Geminates:} N/A
\end{styleBody}

\begin{styleBody}
\textbf{Voicing contrasts:} None
\end{styleBody}

\begin{styleBody}
\textbf{Places:} Bilabial, Labiodental, Dental, Alveolar, Palatal, Velar
\end{styleBody}

\begin{styleBody}
\textbf{Manners:} Stop, Fricative, Nasal, Trill, Central approximant, Lateral approximant
\end{styleBody}

\begin{styleBody}
\textbf{N elaborations:} 4
\end{styleBody}

\begin{styleBody}
\textbf{Elaborations:} Voiced fricatives/affricates, Devoiced sonorants, Post-aspiration, Labiodental
\end{styleBody}

\begin{styleBody}
\textbf{V phoneme inventory:} /[26A?] e a o u/
\end{styleBody}

\begin{styleBody}
\textbf{N vowel qualities:} 5
\end{styleBody}

\begin{styleBody}
\textbf{Diphthongs or vowel sequences:} None
\end{styleBody}

\begin{styleBody}
\textbf{Contrastive length:} None
\end{styleBody}

\begin{styleBody}
\textbf{Contrastive nasalization:} None
\end{styleBody}

\begin{styleBody}
\textbf{Other contrasts:} N/A
\end{styleBody}

\begin{styleBody}
\textbf{Notes:} Prenasalized stops interpreted as a cluster on the basis of occurrence of reverse sequences and separate occurrence of component segments (1969: 34).
\end{styleBody}

\begin{styleBody}
\textbf{Syllable structure}
\end{styleBody}

\begin{styleBody}
\textbf{Complexity Category:} Highly Complex
\end{styleBody}

\begin{styleBody}
\textbf{Canonical syllable structure:} (C)VC(C)(C)(C)\textbf{ }(Summer 1969: 33-35; Sommer 1981; Dixon 1970)
\end{styleBody}

\begin{styleBody}
\textbf{Size of maximal onset:} 1
\end{styleBody}

\begin{styleBody}
\textbf{Size of maximal coda:} 4
\end{styleBody}

\begin{styleBody}
\textbf{Onset obligatory:} No
\end{styleBody}

\begin{styleBody}
\textbf{Coda obligatory:} Yes
\end{styleBody}

\begin{styleBody}
\textbf{Vocalic nucleus patterns:} Short vowels
\end{styleBody}

\begin{styleBody}
\textbf{Syllabic consonant patterns:} N/A
\end{styleBody}

\begin{styleBody}
\textbf{Size of maximal word{}-marginal sequences with syllabic obstruents:} N/A
\end{styleBody}

\begin{styleBody}
\textbf{Predictability of syllabic consonants:} N/A
\end{styleBody}

\begin{styleBody}
\textbf{Morphological constituency of maximal syllable margin:} Morpheme-internal (Coda)
\end{styleBody}

\begin{styleBody}
\textbf{Morphological pattern of syllabic consonants:} N/A
\end{styleBody}

\begin{styleBody}
\textbf{Onset restrictions:} Unclear.
\end{styleBody}

\begin{styleBody}
\textbf{Coda restrictions: }Simple codas unrestricted. Biconsonantal codas include nasal-nasal, lateral-stop, lateral-fricative, stop-nasal, rhotic-stop, rhotic-nasal, rhotic-glide, glide-glide. Triconsonantal codas have liquid as C\textsubscript{1} followed by stop-nasal or nasal-stop sequence, or stop-nasal-stop sequence in which first two members are homorganic. Four-consonant codas consist of /l [279?] j/ followed by homorganic sequence of stop, nasal, and stop, e.g. /lbmb/.
\end{styleBody}

\begin{styleBody}
\textbf{Notes: }This language is typologically unusual in that it is claimed to have no onsets. Sommer (1970, 1981) argues for this analysis using evidence from phonological processes in the language. Consonant-initial syllables are reported to occur in a few lexical items when these are sentence-initial: Sommer (1969: 16, 33) indicates that this is optional and limited to words which occur with high frequency in that environment, but Sommer (1981) suggests that this is an invariant pattern. Dixon (1970) disagrees with Sommer’s analysis; in work with Olgolo he observed many invariant word-final vowels in the language. He analyzes the language as having V(C)(C) structure in initial syllables and CV(C)(C) syllables following that, with the limitation that a stem-final syllable can have at most one final consonant (1970: 274). Sommer criticizes Dixon for using data from the more distantly related Olgolo rather than from closely related Olgol to argue against patterns in Oykangand. Dixon also analyzes the language as having a series of pre-stopped nasals; this would affect the canonical syllable structure proposed by Sommer. Sommer argues for his sequential analysis of these structures in (1981: 242 f1)
\end{styleBody}

\begin{styleBody}
\textbf{Suprasegmentals}
\end{styleBody}

\begin{styleBody}
\textbf{Tone:} No
\end{styleBody}

\begin{styleBody}
\textbf{Word stress:} Yes
\end{styleBody}

\begin{styleBody}
\textbf{Stress placement:} Morphologically or Lexically Conditioned
\end{styleBody}

\begin{styleBody}
\textbf{Phonetic processes conditioned by stress:} Consonant Allophony in Stressed Syllables
\end{styleBody}

\begin{styleBody}
\textbf{Differences in phonological properties of stressed and unstressed syllables:} Not described
\end{styleBody}

\begin{styleBody}
\textbf{Phonetic correlates of stress: }Vowel duration (instrumental), Intensity (impressionistic)
\end{styleBody}

\begin{styleBody}
\textbf{Vowel reduction processes}
\end{styleBody}

\begin{styleBody}
\textbf{kjn-R1:} In fast speech, vowels tend toward an indeterminate central position resembling [[259?]] but maintain their rounding characteristics (Sommer 1969: 41).
\end{styleBody}

\begin{styleBody}
\textbf{kjn-R2:} High front vowel is realized as lax in unstressed, non-word-initial position (Sommer 1969:41).
\end{styleBody}

\begin{styleBody}
\textbf{Consonant allophony processes}
\end{styleBody}

\begin{styleBody}
\textbf{kjn-C1: }Unaspirated voiceless stops are voiced preceding a nasal (Sommer 1969: 39).
\end{styleBody}

\begin{styleBody}
\textbf{kjn-C2: }Unaspirated voiceless stops are sometimes voiced following a liquid (Sommer 1969: 39).
\end{styleBody}

\begin{styleBody}
\textbf{Morphology}
\end{styleBody}

\begin{styleBody}
(adequate texts unavailable)
\end{styleBody}

\clearpage\begin{styleBody}
\textbf{[kms] }\ \ \textbf{\textsc{Kamasau}}\textbf{\ \ }Nuclear Torricelli, \textit{Marienberg} (Papua New Guinea)
\end{styleBody}

\begin{styleBody}
\textbf{References consulted: }Sanders \& Sanders (1980)
\end{styleBody}

\begin{styleBody}
\textbf{Sound inventory}
\end{styleBody}

\begin{styleBody}
\textbf{C phoneme inventory:} /b t d t[361?][283?] d[361?][292?] k [261?] [294?] \textsuperscript{m}b \textsuperscript{n}d \textsuperscript{[272?]}d[361?][292?] \textsuperscript{ŋ}[261?] [278?] $\beta $ s [263?] m n [272?] ŋ [27E?] w j/
\end{styleBody}

\begin{styleBody}
\textbf{N consonant phonemes:} 23
\end{styleBody}

\begin{styleBody}
\textbf{Geminates:} N/A
\end{styleBody}

\begin{styleBody}
\textbf{Voicing contrasts:} Obstruents
\end{styleBody}

\begin{styleBody}
\textbf{Places:} Bilabial, Alveolar, Palato-alveolar, Velar, Glottal
\end{styleBody}

\begin{styleBody}
\textbf{Manners:} Stop, Affricate, Fricative, Nasal, Flap/Tap, Central approximant
\end{styleBody}

\begin{styleBody}
\textbf{N elaborations:} 3
\end{styleBody}

\begin{styleBody}
\textbf{Elaborations:} Voiced fricatives/affricates, Prenasalization, Palato-alveolar
\end{styleBody}

\begin{styleBody}
\textbf{V phoneme inventory:} /i e [268?] a o u/
\end{styleBody}

\begin{styleBody}
\textbf{N vowel qualities:} 6
\end{styleBody}

\begin{styleBody}
\textbf{Diphthongs or vowel sequences:} Vowel sequences /iu ia ie io ui ua ue uo ai au ao ei eu ea eo oi ou/
\end{styleBody}

\begin{styleBody}
\textbf{Contrastive length:} None
\end{styleBody}

\begin{styleBody}
\textbf{Contrastive nasalization:} None
\end{styleBody}

\begin{styleBody}
\textbf{Other contrasts:} N/A
\end{styleBody}

\begin{styleBody}
\textbf{Syllable structure}
\end{styleBody}

\begin{styleBody}
\textbf{Complexity Category:} Moderately Complex
\end{styleBody}

\begin{styleBody}
\textbf{Canonical syllable structure:} (C)(C)V(C) (Sanders \& Sanders 1980: 116-121)
\end{styleBody}

\begin{styleBody}
\textbf{Size of maximal onset:} 2
\end{styleBody}

\begin{styleBody}
\textbf{Size of maximal coda:} 1
\end{styleBody}

\begin{styleBody}
\textbf{Onset obligatory:} No
\end{styleBody}

\begin{styleBody}
\textbf{Coda obligatory:} No
\end{styleBody}

\begin{styleBody}
\textbf{Vocalic nucleus patterns:} Short vowels, Vowel sequences
\end{styleBody}

\begin{styleBody}
\textbf{Syllabic consonant patterns:} Nasal
\end{styleBody}

\begin{styleBody}
\textbf{Size of maximal word{}-marginal sequences with syllabic obstruents:} N/A
\end{styleBody}

\begin{styleBody}
\textbf{Predictability of syllabic consonants:} Varies with CV sequence
\end{styleBody}

\begin{styleBody}
\textbf{Morphological constituency of maximal syllable margin:} Both patterns (Onset)
\end{styleBody}

\begin{styleBody}
\textbf{Morphological pattern of syllabic consonants:} N/A
\end{styleBody}

\begin{styleBody}
\textbf{Onset restrictions: }C\textsubscript{1} may be a plosive, /\textsuperscript{m}b/, /s/, /[27E?]/, or nasal. C\textsubscript{2} is always /j/, /w/, or /[27E?]/.
\end{styleBody}

\begin{styleBody}
\textbf{Coda restrictions: }All consonants except /w j/ may occur. CCVVC syllables are always closed by /[294?]/.
\end{styleBody}

\begin{styleBody}
\textbf{Suprasegmentals}
\end{styleBody}

\begin{styleBody}
\textbf{Tone:} No
\end{styleBody}

\begin{styleBody}
\textbf{Word stress:} Yes
\end{styleBody}

\begin{styleBody}
\textbf{Stress placement:} Unpredictable/Variable
\end{styleBody}

\begin{styleBody}
\textbf{Phonetic processes conditioned by stress:} Vowel Reduction
\end{styleBody}

\begin{styleBody}
\textbf{Differences in phonological properties of stressed and unstressed syllables:} (None)
\end{styleBody}

\begin{styleBody}
\textbf{Phonetic correlates of stress: }Not described
\end{styleBody}

\begin{styleBody}
\textbf{Vowel reduction processes}
\end{styleBody}

\begin{styleBody}
\textbf{kms-R1:} The vowel in a word-initial syllable preceding a stressed syllable has a tendency to be reduced, being deleted or overlapped with a preceding (nasal) consonant to produce a syllabic consonant (Sanders \& Sanders 1980: 114-115).
\end{styleBody}

\begin{styleBody}
\textbf{kms-R2:} A low central vowel /a/ occurs as mid in an unstressed syllable (Sanders \& Sanders 1980: 122).
\end{styleBody}

\begin{styleBody}
\textbf{Consonant allophony processes}
\end{styleBody}

\begin{styleBody}
(none reported)
\end{styleBody}

\begin{styleBody}
\textbf{Morphology}
\end{styleBody}

\begin{styleBody}
\textbf{Text:} “Amu2 Text” (Sanders \& Sanders 1994: 85-94)
\end{styleBody}

\begin{styleBody}
\textbf{Synthetic index: }1.4 morphemes/word (639 morphemes, 455 words)
\end{styleBody}

\clearpage\begin{styleBody}
\textbf{[knc] }\ \ \textbf{\textsc{Kanuri}}\textbf{\ \ \ \ }Saharan, \textit{Western Saharan} (Chad, Niger, Nigeria, Sudan)
\end{styleBody}

\begin{styleBody}
\textbf{References consulted: }Cyffer (1998), Hutchison (1981)
\end{styleBody}

\begin{styleBody}
\textbf{Sound inventory}
\end{styleBody}

\begin{styleBody}
\textbf{C phoneme inventory:} /b t d k [261?] \textsuperscript{m}b \textsuperscript{n}d \textsuperscript{ŋ}[261?] t[361?][283?] d[361?][292?] f s z [283?] h m n [27E?] l w j/
\end{styleBody}

\begin{styleBody}
\textbf{N consonant phonemes:} 21
\end{styleBody}

\begin{styleBody}
\textbf{Geminates:} N/A
\end{styleBody}

\begin{styleBody}
\textbf{Voicing contrasts:} Obstruents
\end{styleBody}

\begin{styleBody}
\textbf{Places:} Bilabial, Labiodental, Alveolar, Palato-alveolar, Velar, Glottal
\end{styleBody}

\begin{styleBody}
\textbf{Manners:} Stop, Affricate, Fricative, Nasal, Flap/Tap, Central approximant, Lateral approximant
\end{styleBody}

\begin{styleBody}
\textbf{N elaborations:} 4
\end{styleBody}

\begin{styleBody}
\textbf{Elaborations:} Voiced fricatives/affricates, Prenasalization, Labiodental, Palato-alveolar
\end{styleBody}

\begin{styleBody}
\textbf{V phoneme inventory:} /i e [259?] a [28C?] o u/
\end{styleBody}

\begin{styleBody}
\textbf{N vowel qualities:} 7
\end{styleBody}

\begin{styleBody}
\textbf{Diphthongs or vowel sequences: }None
\end{styleBody}

\begin{styleBody}
\textbf{Contrastive length:} None
\end{styleBody}

\begin{styleBody}
\textbf{Contrastive nasalization:} None
\end{styleBody}

\begin{styleBody}
\textbf{Other contrasts:} None
\end{styleBody}

\begin{styleBody}
\textbf{Notes: }Vowel sequences /aa ii uu ai au ia iu oi/ appear to be variable realizations in predictable contexts in which an intervocalic /[261?]/ may be weakened or entirely lost.
\end{styleBody}

\begin{styleBody}
\textbf{Syllable structure}
\end{styleBody}

\begin{styleBody}
\textbf{Category:} Moderately Complex
\end{styleBody}

\begin{styleBody}
\textbf{Canonical syllable structure:} (C)V(C) (Hutchison 1981: 15-17, Cyffer 1998)
\end{styleBody}

\begin{styleBody}
\textbf{Size of maximal onset:} 1
\end{styleBody}

\begin{styleBody}
\textbf{Size of maximal coda:} 1
\end{styleBody}

\begin{styleBody}
\textbf{Onset obligatory:} No
\end{styleBody}

\begin{styleBody}
\textbf{Coda obligatory:} No
\end{styleBody}

\begin{styleBody}
\textbf{Vocalic nucleus patterns:} Short vowels, Vowel sequences
\end{styleBody}

\begin{styleBody}
\textbf{Syllabic consonant patterns:} Nasal
\end{styleBody}

\begin{styleBody}
\textbf{Size of maximal word{}-marginal sequences with syllabic obstruents:} N/A
\end{styleBody}

\begin{styleBody}
\textbf{Predictability of syllabic consonants:} Predictable from word/consonantal context
\end{styleBody}

\begin{styleBody}
\textbf{Morphological constituency of maximal syllable margin:} N/A
\end{styleBody}

\begin{styleBody}
\textbf{Morphological pattern of syllabic consonants:} Lexical items
\end{styleBody}

\begin{styleBody}
\textbf{Onset restrictions: }None.
\end{styleBody}

\begin{styleBody}
\textbf{Coda restrictions:} only sonorant consonants /l [27E?] m n/ occur (Hutchison 1981: 15)
\end{styleBody}

\begin{styleBody}
\textbf{Notes: }Hutchison states that onsetless syllables occur only in borrowings (1981: 15), but both Hutchison and Cyffer (1998) give examples of V-initial nouns, verbs, and demonstratives which seem unlikely to be borrowed (e.g., verb paradigm for ‘come’, demonstratives \textit{\'{a}d[259?]} and \textit{\'{a}nj\`{i}}).
\end{styleBody}

\begin{styleBody}
\textbf{Suprasegmentals}
\end{styleBody}

\begin{styleBody}
\textbf{Tone:} Yes
\end{styleBody}

\begin{styleBody}
\textbf{Word stress:} No
\end{styleBody}

\begin{styleBody}
\textbf{Vowel reduction processes}
\end{styleBody}

\begin{styleBody}
(none reported)
\end{styleBody}

\begin{styleBody}
\textbf{Consonant allophony processes}
\end{styleBody}

\begin{styleBody}
\textbf{knc-C1: }A voiceless alveolar fricative is realized as palato-alveolar preceding front vowels (Cyffer 1998: 20).
\end{styleBody}

\begin{styleBody}
\textbf{knc-C2: }A voiceless alveolar fricative is realized as a palatal stop when occurring after a consonant and before a front vowel (Cyffer 1998: 21).
\end{styleBody}

\begin{styleBody}
\textbf{knc-C3: }A voiceless labiodental fricative is realized as labial preceding a back rounded vowel (Cyffer 1998: 23).
\end{styleBody}

\begin{styleBody}
\textbf{knc-C4: }Voiceless consonants are voiced when occurring after a sonorant and preceding a vowel (Cyffer 1998: 22).
\end{styleBody}

\begin{styleBody}
\textbf{knc-C5: }A voiced velar stop is spirantized intervocalically (Cyffer 1998: 22).
\end{styleBody}

\begin{styleBody}
\textbf{knc-C6: }A voiced bilabial stop may be realized as a labiovelar glide when occurring after a vowel or liquid (Cyffer 1998: 22).
\end{styleBody}

\begin{styleBody}
\textbf{knc-C7: }A voiceless alveolar fricative may be realized as a palato-alveolar affricate when occurring after a sonorant and preceding a front vowel (Cyffer 1998: 21).
\end{styleBody}

\begin{styleBody}
\textbf{knc-C8: }Velar stops are realized as corresponding glides when adjacent to front and back vowels, respectively (Cyffer 1998: 22).
\end{styleBody}

\begin{styleBody}
\textbf{Morphology}
\end{styleBody}

\begin{styleBody}
(adequate texts unavailable)
\end{styleBody}

\clearpage\begin{styleBody}
\textbf{[kpm] }\ \ \textbf{\textsc{Koho (Sre dialect)}}\textbf{\ \ }Austroasiatic, \textit{Bahnaric} (Vietnam)
\end{styleBody}

\begin{styleBody}
\textbf{References consulted: }Ladefoged \& Maddieson (1997), Manley (1972), Olsen (2014)
\end{styleBody}

\begin{styleBody}
\textbf{Sound inventory}
\end{styleBody}

\begin{styleBody}
\textbf{C phoneme inventory:} /p b t d c [25F?] k [261?] [294?] p[2B0?] t[2B0?] c[2B0?] k[2B0?] [253?] [257?] s h m n [272?] ŋ m[2B0?] n[2B0?] [272?][2B0?] r r[2B0?] l l[2B0?] w j/
\end{styleBody}

\begin{styleBody}
\textbf{N consonant phonemes:} 30
\end{styleBody}

\begin{styleBody}
\textbf{Geminates:} N/A
\end{styleBody}

\begin{styleBody}
\textbf{Voicing contrasts:} Obstruents
\end{styleBody}

\begin{styleBody}
\textbf{Places:} Bilabial, Alveolar, Palatal, Velar, Glottal
\end{styleBody}

\begin{styleBody}
\textbf{Manners:} Stop, Fricative, Nasal, Trill, Central approximant, Lateral approximant
\end{styleBody}

\begin{styleBody}
\textbf{N elaborations:} 3
\end{styleBody}

\begin{styleBody}
\textbf{Elaborations:} Post-aspiration, Implosive
\end{styleBody}

\begin{styleBody}
\textbf{V phoneme inventory:} /i e [25B?] [268?] a [251?] [254?] o [264?] u i[2D0?] e[2D0?] [25B?][2D0?] [268?][2D0?] a[2D0?] [251?][2D0?] [254?][2D0?] [264?][2D0?] o[2D0?] u[2D0?]/
\end{styleBody}

\begin{styleBody}
\textbf{N vowel qualities:} 10
\end{styleBody}

\begin{styleBody}
\textbf{Diphthongs or vowel sequences:} None
\end{styleBody}

\begin{styleBody}
\textbf{Contrastive length:} All
\end{styleBody}

\begin{styleBody}
\textbf{Contrastive nasalization:} None
\end{styleBody}

\begin{styleBody}
\textbf{Other contrasts:} N/A
\end{styleBody}

\begin{styleBody}
\textbf{Notes:} Ladefoged \& Maddieson (1996:116) suspect the aspirated nasal are actually voiceless. Olsen argues that aspirated trill, lateral approximants are units, using morphological evidence. Manley doesn’t list these, but does have /j[2B0?] w[2B0?]/ instead. Olsen shows VOT for aspirated sonorants is 2-3 times longer than for unaspirated stops, concluding that this indicates aspiration rather than voicelessness. [[268?]] varies with [[26F?]]. /[251?]/ occurs in subdialects A \& B, but not C. /e o a/ almost always occur long. Long vowels are associated with pitch fall or rise; Manley analyzes pitch, not length as the conditioned feature (Manley 1972: 15).
\end{styleBody}

\begin{styleBody}
\textbf{Syllable structure}
\end{styleBody}

\begin{styleBody}
\textbf{Complexity Category:} Complex
\end{styleBody}

\begin{styleBody}
\textbf{Canonical syllable structure:} C(C)(C)V(C)(C)\textbf{ }(Olsen 2014: 30-40, Manley 1972: 23-7)
\end{styleBody}

\begin{styleBody}
\textbf{Size of maximal onset:} 3
\end{styleBody}

\begin{styleBody}
\textbf{Size of maximal coda:} 2
\end{styleBody}

\begin{styleBody}
\textbf{Onset obligatory:} Yes
\end{styleBody}

\begin{styleBody}
\textbf{Coda obligatory:} No
\end{styleBody}

\begin{styleBody}
\textbf{Vocalic nucleus patterns:} Short vowels, Long vowels
\end{styleBody}

\begin{styleBody}
\textbf{Syllabic consonant patterns:} Nasals
\end{styleBody}

\begin{styleBody}
\textbf{Size of maximal word{}-marginal sequences with syllabic obstruents:} N/A
\end{styleBody}

\begin{styleBody}
\textbf{Predictability of syllabic consonants:} Phonemic
\end{styleBody}

\begin{styleBody}
\textbf{Morphological constituency of maximal syllable margin:} Morpheme-internal (Onset, Coda)
\end{styleBody}

\begin{styleBody}
\textbf{Morphological pattern of syllabic consonants:} Lexical
\end{styleBody}

\begin{styleBody}
\textbf{Onset restrictions: }All consonants may occur in simple onsets, though presyllable onsets are limited to unaspirated, unimploded obstruents. Biconsonantal onsets have a liquid or glide as C\textsubscript{2}. Triconsonantal onsets are limited to a stop or /s m/ as C\textsubscript{1}, /r l/ as C\textsubscript{2}, and /w j/ as C\textsubscript{3}.
\end{styleBody}

\begin{styleBody}
\textbf{Coda restrictions: }In presyllables, simple codas are limited to liquids and /n/. In main syllables, simple coda may be liquid, nasal, glide, or glottal. Biconsonantal codas are glide + /[294?] h/.
\end{styleBody}

\begin{styleBody}
\textbf{Notes: }This language has presyllables/main syllable distinction.
\end{styleBody}

\begin{styleBody}
\textbf{Suprasegmentals}
\end{styleBody}

\begin{styleBody}
\textbf{Tone:} Yes
\end{styleBody}

\begin{styleBody}
\textbf{Word stress: }Yes
\end{styleBody}

\begin{styleBody}
\textbf{Stress placement:} Fixed
\end{styleBody}

\begin{styleBody}
\textbf{Phonetic processes conditioned by stress:} Vowel Reduction
\end{styleBody}

\begin{styleBody}
\textbf{Differences in phonological properties of stressed and unstressed syllables:} Vowel Quality Contrasts, Vowel Length Contrasts, Consonant Contrasts, Tonal Contrasts
\end{styleBody}

\begin{styleBody}
\textbf{Phonetic correlates of stress: }Intensity (impressionistic)
\end{styleBody}

\begin{styleBody}
\textbf{Notes: }Duration and pitch seem to be interdependent correlates of tone. Main syllable vowels have “attendant pitch length” (Olsen 2014: 32).
\end{styleBody}

\begin{styleBody}
\textbf{Vowel reduction processes}
\end{styleBody}

\begin{styleBody}
\textbf{kpm-R1: }\ Long vowels decrease in duration if not occurring word-finally, particularly if unstressed (Olsen 2014: 33).
\end{styleBody}

\begin{styleBody}
\textbf{kpm-R2: }Presyllables tend to weaken or disappear in many environments (Olsen 2014: 31).
\end{styleBody}

\begin{styleBody}
\textbf{Consonant allophony processes}
\end{styleBody}

\begin{styleBody}
\textbf{kpm-C1: }An alveolar trill is realized as a flap when occurring as second consonant of onset and preceding a vowel (Olsen 2014: 24).
\end{styleBody}

\begin{styleBody}
\textbf{Morphology}
\end{styleBody}

\begin{styleBody}
\textbf{Text:} “Traditional village work” (Olsen 2014: 106-107)
\end{styleBody}

\begin{styleBody}
\textbf{Synthetic index: }1.0 morphemes/word (90 morphemes, 89 words)
\end{styleBody}

\clearpage\begin{styleBody}
\textbf{[ktb] }\ \ \textbf{\textsc{Kambaata}}\textbf{\ \ }Afro-Asiatic, \textit{Cushitic} (Ethiopia)
\end{styleBody}

\begin{styleBody}
\textbf{References consulted: }Treis (2008)
\end{styleBody}

\begin{styleBody}
\textbf{Sound inventory}
\end{styleBody}

\begin{styleBody}
\textbf{C phoneme inventory:} /b t d k [261?] [294?] p’ t’ k’ t[361?][283?] d[361?][292?] t[361?][283?]’ f s z [283?] [292?][2D0?] h m n [272?][2D0?] r [27E?]\textsuperscript{[294?]} l l\textsuperscript{[294?]} w j/
\end{styleBody}

\begin{styleBody}
\textbf{N consonant phonemes:} 27
\end{styleBody}

\begin{styleBody}
\textbf{Geminates:} /b[2D0?] t[2D0?] d[2D0?] k[2D0?] [261?][2D0?] [294?][2D0?] p’[2D0?] t’[2D0?] k’[2D0?] t[361?][283?][2D0?] d[361?][292?][2D0?] t[361?][283?]’[2D0?] f[2D0?] s[2D0?] z[2D0?] [283?][2D0?] [292?][2D0?] h[2D0?] m[2D0?] n[2D0?] [272?][2D0?] r[2D0?] [27E?]\textsuperscript{[294?]}[2D0?] l[2D0?] l\textsuperscript{[294?]}[2D0?] w[2D0?] j[2D0?]/ (All, including some that don’t have singleton counterparts)
\end{styleBody}

\begin{styleBody}
\textbf{Voicing contrasts: }Obstruents
\end{styleBody}

\begin{styleBody}
\textbf{Places: }Bilabial, Labiodental, Alveolar, Palato-alveolar, Palatal, Velar, Glottal
\end{styleBody}

\begin{styleBody}
\textbf{Manners:} Stop, Affricate, Fricative, Nasal, Flap/tap, Trill, Central approximant, Lateral approximant
\end{styleBody}

\begin{styleBody}
\textbf{N elaborations: }5
\end{styleBody}

\begin{styleBody}
\textbf{Elaborations:} Creaky voice, Voiced fricatives/affricates, Ejective, Labiodental, Palato-alveolar
\end{styleBody}

\begin{styleBody}
\textbf{V phoneme inventory:} /i e a o u/
\end{styleBody}

\begin{styleBody}
\textbf{N vowel qualities:} 5
\end{styleBody}

\begin{styleBody}
\textbf{Diphthongs or vowel sequences:} Vowel sequences /ii ee aa oo uu/
\end{styleBody}

\begin{styleBody}
\textbf{Contrastive length:} None
\end{styleBody}

\begin{styleBody}
\textbf{Contrastive nasalization:} None
\end{styleBody}

\begin{styleBody}
\textbf{Other contrasts:} Voicing?
\end{styleBody}

\begin{styleBody}
\textbf{Notes:} Singleton/geminate contrasts occur for all consonants intervocalically, except for /[292?][2D0?] [272?][2D0?]/ which are always geminate. Glottalized liquids /[27E?]\textsuperscript{[294?]} l\textsuperscript{[294?]}/ are rare and have ‘defective’ distribution. Treis analyzes phonetically long vowels as sequences of identical vowels. Phonemic status of voiceless vowels is not fully determined. These seem to be mostly predictable variants of voiced vowels, but there are exceptions to these patterns in a few grammatical contexts. There are no minimal pairs of words with the same stress pattern but with final vowels differing in voicing only (Treis 2008: 20-22). Nasalized vowels are marginally phonemic, occurring in very few lexical items, most of which are ideophonic. I do not count nasalization as a contrastive feature here.
\end{styleBody}

\begin{styleBody}
\textbf{Syllable structure}
\end{styleBody}

\begin{styleBody}
\textbf{Complexity Category:} Moderately Complex
\end{styleBody}

\begin{styleBody}
\textbf{Canonical syllable structure:} CV(C)\textbf{ }(Treis 2008: 41)
\end{styleBody}

\begin{styleBody}
\textbf{Size of maximal onset:} 1
\end{styleBody}

\begin{styleBody}
\textbf{Size of maximal coda: }N/A
\end{styleBody}

\begin{styleBody}
\textbf{Onset obligatory:} Yes
\end{styleBody}

\begin{styleBody}
\textbf{Coda obligatory:} No
\end{styleBody}

\begin{styleBody}
\textbf{Vocalic nucleus patterns: }Short vowels, Long vowels, Diphthongs
\end{styleBody}

\begin{styleBody}
\textbf{Syllabic consonant patterns:} N/A
\end{styleBody}

\begin{styleBody}
\textbf{Size of maximal word{}-marginal sequences with syllabic obstruents:} N/A
\end{styleBody}

\begin{styleBody}
\textbf{Predictability of syllabic consonants:} N/A
\end{styleBody}

\begin{styleBody}
\textbf{Morphological constituency of maximal syllable margin:} N/A
\end{styleBody}

\begin{styleBody}
\textbf{Morphological pattern of syllabic consonants:} N/A
\end{styleBody}

\begin{styleBody}
\textbf{Onset restrictions:} None.
\end{styleBody}

\begin{styleBody}
\textbf{Coda restrictions:} Word-finally, no restrictions.
\end{styleBody}

\begin{styleBody}
\textbf{Notes: }Many words end in ‘hardly audible final [i[325?]]’ (p. 48). Phonetic diphthongs occur as a result of morphophonological processes.
\end{styleBody}

\begin{styleBody}
\textbf{Suprasegmentals}
\end{styleBody}

\begin{styleBody}
\textbf{Tone:} No
\end{styleBody}

\begin{styleBody}
\textbf{Word stress: }Yes
\end{styleBody}

\begin{styleBody}
\textbf{Stress placement:} Morphologically or Lexically Conditioned
\end{styleBody}

\begin{styleBody}
\textbf{Phonetic processes conditioned by stress:} Vowel Reduction
\end{styleBody}

\begin{styleBody}
\textbf{Differences in phonological properties of stressed and unstressed syllables:} (None)
\end{styleBody}

\begin{styleBody}
\textbf{Phonetic correlates of stress: }Pitch (impressionistic)
\end{styleBody}

\begin{styleBody}
\textbf{Vowel reduction processes}
\end{styleBody}

\begin{styleBody}
\textbf{ktb-R1: }In some closed syllables, short vowels tend to be slightly centralized (Treis 2008: 18).
\end{styleBody}

\begin{styleBody}
\textbf{ktb-R2: }Unstressed word-final long vowels are at most half-long (Treis 2008: 19).
\end{styleBody}

\begin{styleBody}
\textbf{ktb-R3: }Unstressed word-final short vowels /i e a o u/ are subject to extra shortening (Treis 2008: 20).
\end{styleBody}

\begin{styleBody}
\textbf{ktb-R4: }Unstressed word-final short vowels /i a u/ are subject to devoicing (Treis 2008: 20).
\end{styleBody}

\begin{styleBody}
\textbf{ktb-R5: }Unstressed word-final /i/ may be deleted in rapid speech (Treis 2008: 20).
\end{styleBody}

\begin{styleBody}
\textbf{Notes: }There are two morphemes ending in /a/ which do not undergo the devoicing process in ktb-R4.
\end{styleBody}

\begin{styleBody}
\textbf{Consonant allophony processes}
\end{styleBody}

\begin{styleBody}
\textbf{ktb-C1:} A voiced bilabial stop is realized as an approximant intervocalically (Treis 2008: 24).
\end{styleBody}

\begin{styleBody}
\textbf{ktb-C2:} A nasal is realized as palato-alveolar preceding a palato-alveolar consonants (Treis 2008: 34).
\end{styleBody}

\begin{styleBody}
\textbf{ktb-C3:} An alveolar trill is realized as a flap intervocalically (Treis 2008: 35).
\end{styleBody}

\begin{styleBody}
\textbf{Morphology}
\end{styleBody}

\begin{styleBody}
(adequate texts unavailable)
\end{styleBody}

\clearpage\begin{styleBody}
\textbf{[kyh] }\ \ \textbf{\textsc{Karok}}\textbf{\ \ }isolate (United States)
\end{styleBody}

\begin{styleBody}
\textbf{References consulted: }Bright (1957), Sandy (2014)
\end{styleBody}

\begin{styleBody}
\textbf{Sound inventory}
\end{styleBody}

\begin{styleBody}
\textbf{C phoneme inventory:} /p t k [294?] t[361?][283?] $\beta $ f $\theta $ s [283?] x h m n [27E?] j/
\end{styleBody}

\begin{styleBody}
\textbf{N consonant phonemes:} 16
\end{styleBody}

\begin{styleBody}
\textbf{Geminates:} N/A
\end{styleBody}

\begin{styleBody}
\textbf{Voicing contrasts:} None
\end{styleBody}

\begin{styleBody}
\textbf{Places:} Bilabial, Labiodental, Dental, Alveolar, Palato-alveolar, Velar, Glottal
\end{styleBody}

\begin{styleBody}
\textbf{Manners:} Stop, Affricate, Fricative, Nasal, Flap/Tap, Central approximant
\end{styleBody}

\begin{styleBody}
\textbf{N elaborations:} 2
\end{styleBody}

\begin{styleBody}
\textbf{Elaborations:} Labiodental, Palato-alveolar
\end{styleBody}

\begin{styleBody}
\textbf{V phoneme inventory:} /i a o u i[2D0?] e[2D0?] a[2D0?] o[2D0?] u[2D0?]/
\end{styleBody}

\begin{styleBody}
\textbf{N vowel qualities:} 5
\end{styleBody}

\begin{styleBody}
\textbf{Diphthongs or vowel sequences:} Diphthong /ui/, perhaps more
\end{styleBody}

\begin{styleBody}
\textbf{Contrastive length:} Some
\end{styleBody}

\begin{styleBody}
\textbf{Contrastive nasalization:} None
\end{styleBody}

\begin{styleBody}
\textbf{Other contrasts:} N/A
\end{styleBody}

\begin{styleBody}
\textbf{Notes:} /[283?]/ only marginally contrastive with /s/ (Bright 1957: 17). No length distinction in /e[2D0?]/.
\end{styleBody}

\begin{styleBody}
\textbf{Syllable structure}
\end{styleBody}

\begin{styleBody}
\textbf{Category:} Moderately Complex
\end{styleBody}

\begin{styleBody}
\textbf{Canonical syllable structure: }CV(C) (Bright 1957: 11)
\end{styleBody}

\begin{styleBody}
\textbf{Size of maximal onset:} 1
\end{styleBody}

\begin{styleBody}
\textbf{Size of maximal coda:} 1
\end{styleBody}

\begin{styleBody}
\textbf{Onset obligatory:} Yes
\end{styleBody}

\begin{styleBody}
\textbf{Coda obligatory:} No
\end{styleBody}

\begin{styleBody}
\textbf{Vocalic nucleus patterns:} Short vowels, Long vowels, Diphthongs
\end{styleBody}

\begin{styleBody}
\textbf{Syllabic consonant patterns:} N/A
\end{styleBody}

\begin{styleBody}
\textbf{Size of maximal word{}-marginal sequences with syllabic obstruents:} N/A
\end{styleBody}

\begin{styleBody}
\textbf{Predictability of syllabic consonants:} N/A
\end{styleBody}

\begin{styleBody}
\textbf{Morphological constituency of maximal syllable margin:} N/A
\end{styleBody}

\begin{styleBody}
\textbf{Morphological pattern of syllabic consonants:} N/A
\end{styleBody}

\begin{styleBody}
\textbf{Onset restrictions: }All consonants occur.
\end{styleBody}

\begin{styleBody}
\textbf{Coda restrictions: }All consonants occur.
\end{styleBody}

\begin{styleBody}
\textbf{Suprasegmentals}
\end{styleBody}

\begin{styleBody}
\textbf{Tone:} Yes
\end{styleBody}

\begin{styleBody}
\textbf{Word stress:} Yes
\end{styleBody}

\begin{styleBody}
\textbf{Stress placement:} Other (tone and weight)
\end{styleBody}

\begin{styleBody}
\textbf{Phonetic processes conditioned by stress:} Vowel Reduction, Consonant Allophony in Unstressed Syllables
\end{styleBody}

\begin{styleBody}
\textbf{Differences in phonological properties of stressed and unstressed syllables:} Not described
\end{styleBody}

\begin{styleBody}
\textbf{Phonetic correlates of stress: }Not described
\end{styleBody}

\begin{styleBody}
\textbf{Notes: }Sandy argues that stress placement is predictable and phonologically conditiond by requirements of tone, which in turn is determined by syllable weight/structure. Stress coincides with tone-bearing mora (Sandy 2014: 40).
\end{styleBody}

\begin{styleBody}
\textbf{Vowel reduction processes}
\end{styleBody}

\begin{styleBody}
\textbf{kyh-R1: }An unstressed low central vowel /a/ without tone is realized as [[259?]] (Bright 1957: 11).
\end{styleBody}

\begin{styleBody}
\textbf{kyh-R2:} An unaccented word-initial short vowel preceding two consonants may be lost following a pause (Bright 1957: 53).
\end{styleBody}

\begin{styleBody}
\textbf{kyh-R3:} Long vowels in post-tonic syllables followed by a pause are realized with lower pitch and glottalization (Bright 1957: 13).
\end{styleBody}

\begin{styleBody}
\textbf{kyh-R4:} Syllables with short vowels may be realized with whispered voice in post-tonic position preceding a pause (Bright 1957: 13).
\end{styleBody}

\begin{styleBody}
\textbf{Consonant allophony processes}
\end{styleBody}

\begin{styleBody}
\textbf{kyh-C1: }A voiceless velar fricative is realized with uvular trill release when occurring before a front vowel (Bright 1957).
\end{styleBody}

\begin{styleBody}
\textbf{kyh-C2: }A voiceless velar fricative is realized as labialized when occurring after a back vowel (Bright 1957: 8)
\end{styleBody}

\begin{styleBody}
\textbf{Morphology}
\end{styleBody}

\begin{styleBody}
\textbf{Text:} “How salmon was given to mankind” (Angulo \& Freeland 1971: 202-4)
\end{styleBody}

\begin{styleBody}
\textbf{Synthetic index: }2.4 morphemes/word (614 morphemes, 252 words)
\end{styleBody}

\clearpage\begin{styleBody}
\textbf{[lao] }\ \ \textbf{\textsc{Lao}}\textbf{\ \ }Tai-Kadai, \textit{Kam-Tai} (Laos, Thailand)
\end{styleBody}

\begin{styleBody}
\textbf{References consulted: }Enfield (2004, 2007), Erickson (2001), Morev et al. (1979)
\end{styleBody}

\begin{styleBody}
\textbf{Sound inventory}
\end{styleBody}

\begin{styleBody}
\textbf{C phoneme inventory:} /p p[2B0?] b t t[2B0?] d t[2B7?] k k[2B0?] k[2B7?] k[2B7?][2B0?] [294?] [294?][2B7?] t[361?][255?] t[361?][255?][2B7?] f s s[2B7?] h m n [272?] ŋ ŋ[2B7?] l l[2B7?] [28B?] j/
\end{styleBody}

\begin{styleBody}
\textbf{N consonant phonemes:} 28
\end{styleBody}

\begin{styleBody}
\textbf{Geminates:} N/A
\end{styleBody}

\begin{styleBody}
\textbf{Voicing contrasts:} Obstruents
\end{styleBody}

\begin{styleBody}
\textbf{Places:} Bilabial, Labiodental, Alveolar, Alveolo-palatal, Velar, Glottal
\end{styleBody}

\begin{styleBody}
\textbf{Manners:} Stop, Affricate, Fricative, Nasal, Central approximant, Lateral approximant
\end{styleBody}

\begin{styleBody}
\textbf{N elaborations:} 3
\end{styleBody}

\begin{styleBody}
\textbf{Elaborations:} Post-aspiration, Labiodental, Labialization
\end{styleBody}

\begin{styleBody}
\textbf{V phoneme inventory:} /i e [25B?] [259?] a [254?] o [26F?] u i[2D0?] e[2D0?] [25B?][2D0?] [259?][2D0?] a[2D0?] [254?][2D0?] o[2D0?] [26F?][2D0?] u[2D0?]/
\end{styleBody}

\begin{styleBody}
\textbf{N vowel qualities:} 9
\end{styleBody}

\begin{styleBody}
\textbf{Diphthongs or vowel sequences:} Diphthongs /ia ua [26F?]a/
\end{styleBody}

\begin{styleBody}
\textbf{Contrastive length:} All
\end{styleBody}

\begin{styleBody}
\textbf{Contrastive nasalization:} None
\end{styleBody}

\begin{styleBody}
\textbf{Other contrasts:} N/A
\end{styleBody}

\begin{styleBody}
\textbf{Notes: }/[28B?]/ varies between fricative and approximant, but approximant more common. Labialized consonants don’t occur before rounded vowels. Because there are no Cj sequences in the language, the CG analysis is rejected for these (Erickson 2001: 138). Diphthong /a[26F?]/ occurs in Northern varieties.
\end{styleBody}

\begin{styleBody}
\textbf{Syllable structure}
\end{styleBody}

\begin{styleBody}
\textbf{Complexity Category:} Moderately Complex
\end{styleBody}

\begin{styleBody}
\textbf{Canonical syllable structure:} CV(C)\textbf{ }(Enfield 2007: 33-5; Morev et al. 1979: 20)
\end{styleBody}

\begin{styleBody}
\textbf{Size of maximal onset:} 1
\end{styleBody}

\begin{styleBody}
\textbf{Size of maximal coda:} 1
\end{styleBody}

\begin{styleBody}
\textbf{Onset obligatory:} Yes
\end{styleBody}

\begin{styleBody}
\textbf{Coda obligatory:} No
\end{styleBody}

\begin{styleBody}
\textbf{Vocalic nucleus patterns:} Short vowels, Long vowels, Diphthongs
\end{styleBody}

\begin{styleBody}
\textbf{Syllabic consonant patterns:} N/A
\end{styleBody}

\begin{styleBody}
\textbf{Size of maximal word{}-marginal sequences with syllabic obstruents:} N/A
\end{styleBody}

\begin{styleBody}
\textbf{Predictability of syllabic consonants:} N/A
\end{styleBody}

\begin{styleBody}
\textbf{Morphological constituency of maximal syllable margin:} N/A
\end{styleBody}

\begin{styleBody}
\textbf{Morphological pattern of syllabic consonants:} N/A
\end{styleBody}

\begin{styleBody}
\textbf{Onset restrictions: }All consonants occur. 
\end{styleBody}

\begin{styleBody}
\textbf{Coda restrictions: }Limited to /p t k [294?] m n ŋ w j/.
\end{styleBody}

\begin{styleBody}
\textbf{Suprasegmentals}
\end{styleBody}

\begin{styleBody}
\textbf{Tone:} Yes
\end{styleBody}

\begin{styleBody}
\textbf{Word stress:} Yes
\end{styleBody}

\begin{styleBody}
\textbf{Stress placement:} Fixed
\end{styleBody}

\begin{styleBody}
\textbf{Phonetic processes conditioned by stress:} (None)
\end{styleBody}

\begin{styleBody}
\textbf{Differences in phonological properties of stressed and unstressed syllables:} Vowel Quality Contrasts, Vowel Length Contrasts, Consonant Contrasts, Tonal Contrasts
\end{styleBody}

\begin{styleBody}
\textbf{Phonetic correlates of stress: }Not described
\end{styleBody}

\begin{styleBody}
\textbf{Vowel reduction processes}
\end{styleBody}

\begin{styleBody}
(none reported)
\end{styleBody}

\begin{styleBody}
\textbf{Consonant allophony processes}
\end{styleBody}

\begin{styleBody}
(none reported)
\end{styleBody}

\begin{styleBody}
\textbf{Morphology}
\end{styleBody}

\begin{styleBody}
\textbf{Text:} “A grammar of Lao” (Enfield 2007: 488-497)
\end{styleBody}

\begin{styleBody}
\textbf{Synthetic index: }1.1 morphemes/word (381 morphemes, 362 words)
\end{styleBody}

\clearpage\begin{styleBody}
\textbf{[lep] }\ \ \textbf{\textsc{Lepcha}}\textbf{\ \ \ \ }Sino-Tibetan, \textit{Himalayish} (Bhutan, India, Nepal)
\end{styleBody}

\begin{styleBody}
\textbf{References consulted: }Plaisier (2007), Sprigg (1966)
\end{styleBody}

\begin{styleBody}
\textbf{Sound inventory}
\end{styleBody}

\begin{styleBody}
\textbf{C phoneme inventory:} /p p[2B0?] b t t[2B0?] d [288?] [288?][2B0?] [256?] c c[2B0?] k k[2B0?] [261?] [294?] t[361?]s t[361?]s[2B0?] f v s z [283?] [292?] h m n [272?] ŋ r l $\beta [31E?]$ j/
\end{styleBody}

\begin{styleBody}
\textbf{N consonant phonemes:} 32
\end{styleBody}

\begin{styleBody}
\textbf{Geminates:} N/A
\end{styleBody}

\begin{styleBody}
\textbf{Voicing contrasts:} Obstruents
\end{styleBody}

\begin{styleBody}
\textbf{Places:} Bilabial, Labiodental, Dental, Alveolar, Palato-alveolar, Retroflex, Palatal, Velar, Glottal
\end{styleBody}

\begin{styleBody}
\textbf{Manners:} Stop, Affricate, Fricative, Nasal, Trill, Central approximant, Lateral approximant
\end{styleBody}

\begin{styleBody}
\textbf{N elaborations:} 5
\end{styleBody}

\begin{styleBody}
\textbf{Elaborations:} Voiced fricatives/affricates, Post-aspiration, Labiodental, Palato-alveolar, Retroflex
\end{styleBody}

\begin{styleBody}
\textbf{V phoneme inventory:} /i e [259?] a [254?] o [26F?] u/
\end{styleBody}

\begin{styleBody}
\textbf{N vowel qualities:} 8
\end{styleBody}

\begin{styleBody}
\textbf{Diphthongs or vowel sequences:} None
\end{styleBody}

\begin{styleBody}
\textbf{Contrastive length:} None
\end{styleBody}

\begin{styleBody}
\textbf{Contrastive nasalization:} None
\end{styleBody}

\begin{styleBody}
\textbf{Other contrasts:} N/A
\end{styleBody}

\begin{styleBody}
\textbf{Notes:} [e] varies with [[25B?]]. /[259?]/ approaches [[28C?]] or [[26F?]] in some contexts.
\end{styleBody}

\begin{styleBody}
\textbf{Syllable structure}
\end{styleBody}

\begin{styleBody}
\textbf{Complexity Category:} Complex
\end{styleBody}

\begin{styleBody}
\textbf{Canonical syllable structure:} C(C)(C)V(C)\textbf{ }(Plaisier 2007: 30-32)
\end{styleBody}

\begin{styleBody}
\textbf{Size of maximal onset:} 3
\end{styleBody}

\begin{styleBody}
\textbf{Size of maximal coda:} 1
\end{styleBody}

\begin{styleBody}
\textbf{Onset obligatory:} Yes
\end{styleBody}

\begin{styleBody}
\textbf{Coda obligatory:} No
\end{styleBody}

\begin{styleBody}
\textbf{Vocalic nucleus patterns:} Short vowels
\end{styleBody}

\begin{styleBody}
\textbf{Syllabic consonant patterns:} N/A
\end{styleBody}

\begin{styleBody}
\textbf{Size of maximal word{}-marginal sequences with syllabic obstruents:} N/A
\end{styleBody}

\begin{styleBody}
\textbf{Predictability of syllabic consonants:} N/A
\end{styleBody}

\begin{styleBody}
\textbf{Morphological constituency of maximal syllable margin:} Morpheme-internal (Onset)
\end{styleBody}

\begin{styleBody}
\textbf{Morphological pattern of syllabic consonants:} N/A
\end{styleBody}

\begin{styleBody}
\textbf{Onset restrictions: }All consonants occur in simple onsets. Biconsonantal onsets have /j r l/ as C\textsubscript{2}. Triconsonantal onsets have /k [261?] p b f m l t[2B0?]/ as C\textsubscript{1}, /r l/ as C\textsubscript{2}, and /j/ as C\textsubscript{3}.
\end{styleBody}

\begin{styleBody}
\textbf{Coda restrictions: }Limited to /p t k m n ŋ r l/.
\end{styleBody}

\begin{styleBody}
\textbf{Suprasegmentals}
\end{styleBody}

\begin{styleBody}
\textbf{Tone:} Yes
\end{styleBody}

\begin{styleBody}
\textbf{Word stress:} Yes
\end{styleBody}

\begin{styleBody}
\textbf{Stress placement:} Weight-Sensitive
\end{styleBody}

\begin{styleBody}
\textbf{Phonetic processes conditioned by stress:} (None)
\end{styleBody}

\begin{styleBody}
\textbf{Differences in phonological properties of stressed and unstressed syllables:} Not described
\end{styleBody}

\begin{styleBody}
\textbf{Phonetic correlates of stress: }Vowel duration (impressionistic), Intensity (impressionistic)
\end{styleBody}

\begin{styleBody}
\textbf{Notes: }Duration a correlate of stress in open syllables. Pitch seems to be correlate not of stress but weak tonal system.
\end{styleBody}

\begin{styleBody}
\textbf{Vowel reduction processes}
\end{styleBody}

\begin{styleBody}
(none reported)
\end{styleBody}

\begin{styleBody}
\textbf{Consonant allophony processes}
\end{styleBody}

\begin{styleBody}
\textbf{lep-C1: }A voiceless alveolar fricative is realized as palato-alveolar preceding a high front vowel (Plaisier 2007: 27).
\end{styleBody}

\begin{styleBody}
\textbf{lep-C2: }Velar stops are realized as palatalized preceding a front vowel (Plaisier 2007: 21).
\end{styleBody}

\begin{styleBody}
\textbf{lep-C3: }An alveolar trill varies (apparently?) freely with a flap (Plaisier 2007: 28).
\end{styleBody}

\begin{styleBody}
\textbf{Morphology}
\end{styleBody}

\begin{styleBody}
\textbf{Text:} “The story of the jackal” (Plaisier 2007: 165-168)
\end{styleBody}

\begin{styleBody}
\textbf{Synthetic index: }1.7 morphemes/word (249 morphemes, 144 words)
\end{styleBody}

\clearpage\begin{styleBody}
\textbf{[lez] }\ \ \textbf{\textsc{Lezgian}}\textbf{\ \ }Nakh-Daghestanian, \textit{Daghestanian} (Azerbaijan, Russia)
\end{styleBody}

\begin{styleBody}
\textbf{References consulted: }Chitoran \& Babaliyeva (2007), Haspelmath (1993), Kodzasov (1990), Yu (2004)
\end{styleBody}

\begin{styleBody}
\textbf{Sound inventory}
\end{styleBody}

\begin{styleBody}
\textbf{C phoneme inventory:} /p p[2B0?] b t t[2B0?] d t[2B7?] t[2B7?][2B0?] k k[2B0?] [261?] k[2B7?] k[2B7?][2B0?] [261?][2B7?] q q[2B0?] q[2B7?] q[2B7?][2B0?] [294?] p’ t’ t’[2B7?] k’ k’[2B7?] q’ q’[2B7?] t[361?]s t[361?]s[2B0?] t[361?]s[2B7?] t[361?]s[2B7?][2B0?] t[361?][283?] t[361?][283?][2B0?] t[361?]s’ t[361?]s’[2B7?] t[361?][283?]’ f s z s[2B7?] z[2B7?] [283?] [292?] x $\chi $ [281?] $\chi [2B7?]$ [281?][2B7?] h m n l r j w/
\end{styleBody}

\begin{styleBody}
\textbf{N consonant phonemes:} 54
\end{styleBody}

\begin{styleBody}
\textbf{Geminates:} N/A
\end{styleBody}

\begin{styleBody}
\textbf{Voicing contrasts:} Obstruents
\end{styleBody}

\begin{styleBody}
\textbf{Places:} Bilabial, Labiodental, Dental, Palato-alveolar, Velar, Uvular, Glottal
\end{styleBody}

\begin{styleBody}
\textbf{Manners:} Stop, Affricate, Fricative, Nasal, Trill, Central approximant, Lateral approximant
\end{styleBody}

\begin{styleBody}
\textbf{N elaborations:} 7
\end{styleBody}

\begin{styleBody}
\textbf{Elaborations:} Voiced fricatives/affricates, Post-aspiration, Ejective, Labiodental, Palato-alveolar, Uvular, Labialization
\end{styleBody}

\begin{styleBody}
\textbf{V phoneme inventory:} /i y e æ a u æ[2D0?] a[2D0?]/
\end{styleBody}

\begin{styleBody}
\textbf{N vowel qualities:} 6
\end{styleBody}

\begin{styleBody}
\textbf{Diphthongs or vowel sequences:} None
\end{styleBody}

\begin{styleBody}
\textbf{Contrastive length:} Some
\end{styleBody}

\begin{styleBody}
\textbf{Contrastive nasalization:} None
\end{styleBody}

\begin{styleBody}
\textbf{Other contrasts:} N/A
\end{styleBody}

\begin{styleBody}
\textbf{Notes:} Kodzasov has /$\beta $/ instead of /w/. /æ[2D0?] a[2D0?]/ are rather marginally contrastive with other vowels. Some dialects have /[26F?]/.
\end{styleBody}

\begin{styleBody}
\textbf{Syllable structure}
\end{styleBody}

\begin{styleBody}
\textbf{Complexity Category:} Highly Complex
\end{styleBody}

\begin{styleBody}
\textbf{Canonical syllable structure:} (C)(C)(C)V(C)(C)\textbf{ }(Haspelmath 1993: 40-46)
\end{styleBody}

\begin{styleBody}
\textbf{Size of maximal onset:} 3
\end{styleBody}

\begin{styleBody}
\textbf{Size of maximal coda:} 2
\end{styleBody}

\begin{styleBody}
\textbf{Onset obligatory:} No
\end{styleBody}

\begin{styleBody}
\textbf{Coda obligatory:} No
\end{styleBody}

\begin{styleBody}
\textbf{Vocalic nucleus patterns:} Short vowels
\end{styleBody}

\begin{styleBody}
\textbf{Syllabic consonant patterns:} N/A
\end{styleBody}

\begin{styleBody}
\textbf{Size of maximal word{}-marginal sequences with syllabic obstruents:} N/A
\end{styleBody}

\begin{styleBody}
\textbf{Predictability of syllabic consonants:} N/A
\end{styleBody}

\begin{styleBody}
\textbf{Morphological constituency of maximal syllable margin:} Morpheme-internal (Onset, Coda)
\end{styleBody}

\begin{styleBody}
\textbf{Morphological pattern of syllabic consonants:} N/A
\end{styleBody}

\begin{styleBody}
\textbf{Onset restrictions: }Biconsonantal onsets include sequences of voiceless obstruents or voiceless obstruent+sonorant. Triconsonantal onsets consist of three voiceless obstruents or two voiceless obstruents and an /r/ or /l/, include /krt[361?][283?], t[2B0?][2B7?]rp, [283?]tk, kk'l t[361?][283?]xr kst ktk/.
\end{styleBody}

\begin{styleBody}
\textbf{Coda restrictions: }Biconsonantal codas have no restrictions, include /rd, st, mp, xt, lt, rk/.
\end{styleBody}

\begin{styleBody}
\textbf{Notes: }Syllable structure has undergone changes recently and used to be canonically (C)V (Haspelmath 1993: 46).
\end{styleBody}

\begin{styleBody}
\textbf{Suprasegmentals}
\end{styleBody}

\begin{styleBody}
\textbf{Tone:} No
\end{styleBody}

\begin{styleBody}
\textbf{Word stress:} Yes
\end{styleBody}

\begin{styleBody}
\textbf{Stress placement:} Morphologically or Lexically Conditioned
\end{styleBody}

\begin{styleBody}
\textbf{Phonetic processes conditioned by stress:} Vowel Reduction
\end{styleBody}

\begin{styleBody}
\textbf{Differences in phonological properties of stressed and unstressed syllables:} Not described
\end{styleBody}

\begin{styleBody}
\textbf{Phonetic correlates of stress: }Vowel duration (instrumental)
\end{styleBody}

\begin{styleBody}
\textbf{Vowel reduction processes}
\end{styleBody}

\begin{styleBody}
\textbf{lez-R1:} High vowels /i y u/ are devoiced and shortened, or deleted when occurring pre-tonically between voiceless obstruents, if both are not fricatives. Process occurs even if there is an intervening /r/ before the second obstruent (Haspelmath 1993: 36-40; Chitoran \& Babaliyeva 2007).
\end{styleBody}

\begin{styleBody}
\textbf{lez-R2:} High vowels /i y u/ are optionally devoiced and shortened, or deleted when occurring between an obstruent and a sonorant followed by a stressed vowel (Haspelmath 1993: 36-40; Chitoran \& Babaliyeva 2007).
\end{styleBody}

\begin{styleBody}
\textbf{lez-R3:} Mid front vowel /e/ is produced with higher quality in pre-stress syllables, especially when followed by /i/ in the next syllable (Haspelmath 1993: 32).
\end{styleBody}

\begin{styleBody}
\textbf{Consonant allophony processes}
\end{styleBody}

\begin{styleBody}
\textbf{lez-C1: }A labiovelar approximant varies freely with a labial fricative variant (Haspelmath 1993: 35).
\end{styleBody}

\begin{styleBody}
\textbf{lez-C2: }An alveolar lateral approximant is velarized syllable-finally following a back vowel (Haspelmath 1993: 35).
\end{styleBody}

\begin{styleBody}
\textbf{Morphology}
\end{styleBody}

\begin{styleBody}
\textbf{Text:} “Who is stealing the melons?”, “The magpie and the wolf” (Haspelmath 1993: 448-456)
\end{styleBody}

\begin{styleBody}
\textbf{Synthetic index: }1.7 morphemes/word (249 morphemes, 144 words)
\end{styleBody}

\clearpage\begin{styleBody}
\textbf{[lkt] }\ \ \textbf{\textsc{Lakota}}\textbf{\ \ \ \ }Siouan, \textit{Core Siouan} (United States)
\end{styleBody}

\begin{styleBody}
\textbf{References consulted: }Ingham (2003), Lakota Language Consortium (2008), Mirzayan (2010), Rood \& Taylor (1996)
\end{styleBody}

\begin{styleBody}
\textbf{Sound inventory}
\end{styleBody}

\begin{styleBody}
\textbf{C phoneme inventory:} /p p[2B0?] b t t[2B0?] k k[2B0?] [294?] p’ t’ k’ t[361?][283?] t[361?][283?][2B0?] t[361?][283?]’ s z [283?] [292?] x [263?] h m n l w j/
\end{styleBody}

\begin{styleBody}
\textbf{N consonant phonemes:} 26
\end{styleBody}

\begin{styleBody}
\textbf{Geminates:} N/A
\end{styleBody}

\begin{styleBody}
\textbf{Voicing contrasts:} Obstruents
\end{styleBody}

\begin{styleBody}
\textbf{Places:} Bilabial, Alveolar, Palato-alveolar, Velar, Glottal
\end{styleBody}

\begin{styleBody}
\textbf{Manners:} Stop, Affricate, Fricative, Nasal, Central approximant, Lateral approximant
\end{styleBody}

\begin{styleBody}
\textbf{N elaborations:} 4
\end{styleBody}

\begin{styleBody}
\textbf{Elaborations:} Voiced fricatives/affricates, Post-aspiration, Ejective, Palato-alveolar
\end{styleBody}

\begin{styleBody}
\textbf{V phoneme inventory:} /i e a o u \~{i} \~{a} \~{u}/
\end{styleBody}

\begin{styleBody}
\textbf{N vowel qualities:} 5
\end{styleBody}

\begin{styleBody}
\textbf{Diphthongs or vowel sequences:} None
\end{styleBody}

\begin{styleBody}
\textbf{Contrastive length:} None
\end{styleBody}

\begin{styleBody}
\textbf{Contrastive nasalization:} Some
\end{styleBody}

\begin{styleBody}
\textbf{Other contrasts:} N/A
\end{styleBody}

\begin{styleBody}
\textbf{Notes:} Rood \& Taylor call /x [263?]/ velar; Mirzayan call these post-velar. /b/ has limited distribution but is unpredictable in some words. Nasal contrast of /i a u/. Only diphthong, /au/, used solely by men in greeting \textit{hau}.
\end{styleBody}

\begin{styleBody}
\textbf{Syllable structure}
\end{styleBody}

\begin{styleBody}
\textbf{Complexity Category:} Complex
\end{styleBody}

\begin{styleBody}
\textbf{Canonical syllable structure:} (C)(C)V(C) (Taylor \& Rood 1996: 446-7, Mirzayan 2010: 39, Ingham 2003: 5)
\end{styleBody}

\begin{styleBody}
\textbf{Size of maximal onset:} 2
\end{styleBody}

\begin{styleBody}
\textbf{Size of maximal coda:} 1
\end{styleBody}

\begin{styleBody}
\textbf{Onset obligatory:} No
\end{styleBody}

\begin{styleBody}
\textbf{Coda obligatory:} No
\end{styleBody}

\begin{styleBody}
\textbf{Vocalic nucleus patterns:} Short vowels, Diphthongs, Vowel sequences
\end{styleBody}

\begin{styleBody}
\textbf{Syllabic consonant patterns:} N/A
\end{styleBody}

\begin{styleBody}
\textbf{Size of maximal word{}-marginal sequences with syllabic obstruents:} N/A
\end{styleBody}

\begin{styleBody}
\textbf{Predictability of syllabic consonants:} N/A
\end{styleBody}

\begin{styleBody}
\textbf{Morphological constituency of maximal syllable margin:} Morpheme-internal (Onset)
\end{styleBody}

\begin{styleBody}
\textbf{Morphological pattern of syllabic consonants:} N/A
\end{styleBody}

\begin{styleBody}
\textbf{Onset restrictions:} Apparently any consonant may function as simple onset. Biconsonatal onsets include sequences of two plosives, plosive+fricative, fricative+plosive, obstruent+voiced continuant sequences, and sequences of two voiced continuants.
\end{styleBody}

\begin{styleBody}
\textbf{Coda restrictions: }Limited to /s [283?] h l b [261?]/ word-internally, and /n m/ word-finally.
\end{styleBody}

\begin{styleBody}
\textbf{Notes: }Syllabification usually follows morpheme boundaries
\end{styleBody}

\begin{styleBody}
\textbf{Suprasegmentals}
\end{styleBody}

\begin{styleBody}
\textbf{Tone:} No
\end{styleBody}

\begin{styleBody}
\textbf{Word stress:} Yes
\end{styleBody}

\begin{styleBody}
\textbf{Stress placement:} Fixed
\end{styleBody}

\begin{styleBody}
\textbf{Phonetic processes conditioned by stress:} Vowel Reduction
\end{styleBody}

\begin{styleBody}
\textbf{Differences in phonological properties of stressed and unstressed syllables:} (None)
\end{styleBody}

\begin{styleBody}
\textbf{Phonetic correlates of stress: }Vowel duration (instrumental), Pitch (impressionistic), Intensity (impressionistic)
\end{styleBody}

\begin{styleBody}
\textbf{Notes: }Duration is a significant correlate in certain segmental contexts. Intensity reported by Ullrich et al (2008) (’greater loudness’), but not instrumentally confirmed.
\end{styleBody}

\begin{styleBody}
\textbf{Vowel reduction processes}
\end{styleBody}

\begin{styleBody}
\textbf{lkt-R1: }In rapid speech any unstressed word-final vowel may be dropped. This process is very frequent but more common in certain morphosyntactic constructions (Mirzayan 2010: 155-6, Taylor \& Rood 1996: 447).
\end{styleBody}

\begin{styleBody}
\textbf{Consonant allophony processes}
\end{styleBody}

\begin{styleBody}
\textbf{lkt-C1: }Velar stops are realized as palato-alveolar affricates following a high front vowel (Ingham 2005: 6).
\end{styleBody}

\begin{styleBody}
\textbf{lkt-C2: }A voiceless glottal fricative is sometimes realized as a palatal glide (Ingham 2005).
\end{styleBody}

\begin{styleBody}
\textbf{Morphology}
\end{styleBody}

\begin{styleBody}
\textbf{Text:} “Hunting eggs in the spring” (Ingham 2003: 95-96)
\end{styleBody}

\begin{styleBody}
\textbf{Synthetic index: }1.3 morphemes/word (282 morphemes, 215 words)
\end{styleBody}

\clearpage\begin{styleBody}
\textbf{[lpa] }\ \ \textbf{\textsc{Lelepa}}\textbf{\ \ }Austronesian, \textit{Malayo-Polynesian} (Vanuatu)
\end{styleBody}

\begin{styleBody}
\textbf{References consulted: }Lacrampe (2014)
\end{styleBody}

\begin{styleBody}
\textbf{Sound inventory}
\end{styleBody}

\begin{styleBody}
\textbf{C phoneme inventory:} /k[361?]p[2B7?] p t k f s ŋ[361?]m[2B7?] m n ŋ l r w j/
\end{styleBody}

\begin{styleBody}
\textbf{N consonant phonemes:} 14
\end{styleBody}

\begin{styleBody}
\textbf{Geminates:} N/A
\end{styleBody}

\begin{styleBody}
\textbf{Voicing contrasts:} None
\end{styleBody}

\begin{styleBody}
\textbf{Places:} Labial-velar, Bilabial, Labiodental, Alveolar, Velar
\end{styleBody}

\begin{styleBody}
\textbf{Manners:} Stop, Fricative, Nasal, Trill, Central approximant, Lateral approximant
\end{styleBody}

\begin{styleBody}
\textbf{N elaborations:} 2
\end{styleBody}

\begin{styleBody}
\textbf{Elaborations:} Labiodental, Labialization
\end{styleBody}

\begin{styleBody}
\textbf{V phoneme inventory:} /i e a o u a[2D0?]/
\end{styleBody}

\begin{styleBody}
\textbf{N vowel qualities:} 5
\end{styleBody}

\begin{styleBody}
\textbf{Diphthongs or vowel sequences:} Diphthongs /ej aj aw ow/
\end{styleBody}

\begin{styleBody}
\textbf{Contrastive length:} Some
\end{styleBody}

\begin{styleBody}
\textbf{Contrastive nasalization:} None
\end{styleBody}

\begin{styleBody}
\textbf{Other contrasts:} N/A
\end{styleBody}

\begin{styleBody}
\textbf{Notes:} Length distinction for /a/ only.
\end{styleBody}

\begin{styleBody}
\textbf{Syllable structure}
\end{styleBody}

\begin{styleBody}
\textbf{Complexity Category:} Complex
\end{styleBody}

\begin{styleBody}
\textbf{Canonical syllable structure:} (C)(C)(C)V(C)(C)\textbf{ }(Lacrampe 2014: 41-8)
\end{styleBody}

\begin{styleBody}
\textbf{Size of maximal onset:} 3
\end{styleBody}

\begin{styleBody}
\textbf{Size of maximal coda:} 2
\end{styleBody}

\begin{styleBody}
\textbf{Onset obligatory:} No
\end{styleBody}

\begin{styleBody}
\textbf{Coda obligatory:} No
\end{styleBody}

\begin{styleBody}
\textbf{Vocalic nucleus patterns:} Short vowels, Long vowels, Diphthongs
\end{styleBody}

\begin{styleBody}
\textbf{Syllabic consonant patterns:} Nasal, Liquid
\end{styleBody}

\begin{styleBody}
\textbf{Size of maximal word{}-marginal sequences with syllabic obstruents:} N/A
\end{styleBody}

\begin{styleBody}
\textbf{Predictability of syllabic consonants:} Predictable from word/consonantal context
\end{styleBody}

\begin{styleBody}
\textbf{Morphological constituency of maximal syllable margin:} Morpheme-internal (Onset, Coda)
\end{styleBody}

\begin{styleBody}
\textbf{Morphological pattern of syllabic consonants:} Lexical items
\end{styleBody}

\begin{styleBody}
\textbf{Onset restrictions: }All consonants may occur in simple onsets. In biconsonantal onsets, C\textsubscript{1} is a plosive, nasal, or fricative and C\textsubscript{2} is usually a liquid but an also be a fricative, stop, or glide. Triconsonantal onsets are /fsr, psr/).
\end{styleBody}

\begin{styleBody}
\textbf{Coda restrictions: }Any consonant may occur in simple codas. The specific patterns for biconsonantal codas are unclear, but they include /lf, rk, ŋs nt ŋk lp/ and appear to be limited to sonorant+obstruent.
\end{styleBody}

\begin{styleBody}
\textbf{Notes: }Syllable structure is in process of becoming more complex in this language, with rampant vowel reduction producing many codas and clusters, though many clusters are also invariant.
\end{styleBody}

\begin{styleBody}
\textbf{Suprasegmentals}
\end{styleBody}

\begin{styleBody}
\textbf{Tone:} No
\end{styleBody}

\begin{styleBody}
\textbf{Word stress:} Yes
\end{styleBody}

\begin{styleBody}
\textbf{Stress placement:} Fixed
\end{styleBody}

\begin{styleBody}
\textbf{Phonetic processes conditioned by stress:} Vowel Reduction
\end{styleBody}

\begin{styleBody}
\textbf{Differences in phonological properties of stressed and unstressed syllables:} (None)
\end{styleBody}

\begin{styleBody}
\textbf{Phonetic correlates of stress: }Vowel duration (impressionistic), Pitch (impressionistic), Intensity (impressionistic)
\end{styleBody}

\begin{styleBody}
\textbf{Notes:} These correlates of stress do not necessarily co-occur; length especially is reduced in rapid speech.
\end{styleBody}

\begin{styleBody}
\textbf{Vowel reduction processes}
\end{styleBody}

\begin{styleBody}
\textbf{lpa-R1:} Mid front vowel /e/ is optionally reduced to [[259?]] when occurring in open unstressed syllables (Lacrampe 2014: 34).
\end{styleBody}

\begin{styleBody}
\textbf{lpa-R2:} Low central vowel /a/ is reduced to [[250?]] in unstressed syllables (Lacrampe 2014: 34-5).
\end{styleBody}

\begin{styleBody}
\textbf{lpa-R3}: After a consonant, word-final high vowels /i u/ and mid back vowel /o/ may be deleted or devoiced (Lacrampe 2014: 15, 64-5).
\end{styleBody}

\begin{styleBody}
\textbf{lpa-R4:} After a consonant, word-final mid front vowel /e/ and low vowel /a/ are reduced in quality, devoiced, or deleted (Lacrampe 2014: 15, 64-5).
\end{styleBody}

\begin{styleBody}
\textbf{lpa-R5:} A vowel filling the nucleus of a syllable preceding the syllable receiving primary stress is reduced in quality, when the word is three syllables or fewer and the stressed syllable has an onset (Lacrampe 2014: 66; process does not occur if it produces an unattested consonant cluster).
\end{styleBody}

\begin{styleBody}
\textbf{lpa-R6:} A vowel filling the nucleus of a syllable preceding the syllable receiving primary stress is deleted, when the word is four or more syllables, the stressed syllable is CV, and the reduced syllable is CV or V (Lacrampe 2014: 66-7; process does not occur if it produces an unattested consonant cluster).
\end{styleBody}

\begin{styleBody}
\textbf{Notes:} Processes R3-R6 are said to be more common in the speech of younger speakers.
\end{styleBody}

\begin{styleBody}
\textbf{Consonant allophony processes}
\end{styleBody}

\begin{styleBody}
\textbf{lpa-C1: }A voiceless velar stop is realized as uvular following a back vowel or /a/ (Lacrampe 2014: 19).
\end{styleBody}

\begin{styleBody}
\textbf{lpa-C2: }Stops and fricatives are optionally voiced intervocalically (Lacrampe 2014: 17).
\end{styleBody}

\begin{styleBody}
\textbf{lpa-C3: }A voiceless velar fricative may be spirantized following a back vowel or /a/ (Lacrampe 2014: 20).
\end{styleBody}

\begin{styleBody}
\textbf{Morphology}
\end{styleBody}

\begin{styleBody}
\textbf{Text:} “Text 1” (Lacrampe 2014: 495-500)
\end{styleBody}

\begin{styleBody}
\textbf{Synthetic index: }1.4 morphemes/word (586 morphemes, 406 words)
\end{styleBody}

\clearpage\begin{styleBody}
\textbf{[lun] }\ \ \textbf{\textsc{Lunda\ \ }}\textbf{\ \ }Atlantic-Congo, \textit{Volta-Congo} (Democratic Republic of Congo, Angola, Zambia)
\end{styleBody}

\begin{styleBody}
\textbf{References consulted: }Kawasha (2003)
\end{styleBody}

\begin{styleBody}
\textbf{Sound inventory}
\end{styleBody}

\begin{styleBody}
\textbf{C phoneme inventory:} /p b t d k [261?] t[361?][283?] d[361?][292?] f v s z [283?] [292?] h m n [272?] ŋ l w j/
\end{styleBody}

\begin{styleBody}
\textbf{N consonant phonemes:} 22
\end{styleBody}

\begin{styleBody}
\textbf{Geminates:} N/A
\end{styleBody}

\begin{styleBody}
\textbf{Voicing contrasts:} Obstruents
\end{styleBody}

\begin{styleBody}
\textbf{Places:} Bilabial, Labiodental, Alveolar, Palato-alveolar, Velar, Glottal
\end{styleBody}

\begin{styleBody}
\textbf{Manners:} Stop, Affricate, Fricative, Nasal, Central approximant, Lateral approximant
\end{styleBody}

\begin{styleBody}
\textbf{N elaborations:} 3
\end{styleBody}

\begin{styleBody}
\textbf{Elaborations:} Voiced fricatives/affricates, Labiodental, Palato-alveolar
\end{styleBody}

\begin{styleBody}
\textbf{V phoneme inventory:} /i e a o u i[2D0?] e[2D0?] a[2D0?] o[2D0?] u[2D0?]/
\end{styleBody}

\begin{styleBody}
\textbf{N vowel qualities:} 5
\end{styleBody}

\begin{styleBody}
\textbf{Diphthongs or vowel sequences:} None
\end{styleBody}

\begin{styleBody}
\textbf{Contrastive length:} All
\end{styleBody}

\begin{styleBody}
\textbf{Contrastive nasalization:} None
\end{styleBody}

\begin{styleBody}
\textbf{Other contrasts:} N/A
\end{styleBody}

\begin{styleBody}
\textbf{Notes:} Reasoning for not considering nasal+C sequences to be prenasalized stops given (2003: 24). Vowel length contrastive in just a few cases.
\end{styleBody}

\begin{styleBody}
\textbf{Syllable structure}
\end{styleBody}

\begin{styleBody}
\textbf{Complexity Category:} Complex
\end{styleBody}

\begin{styleBody}
\textbf{Canonical syllable structure:} (C)(C)(C)V\textbf{ }(Kawasha 2003: 20-21)
\end{styleBody}

\begin{styleBody}
\textbf{Size of maximal onset:} 3
\end{styleBody}

\begin{styleBody}
\textbf{Size of maximal coda:} N/A
\end{styleBody}

\begin{styleBody}
\textbf{Onset obligatory:} No
\end{styleBody}

\begin{styleBody}
\textbf{Coda obligatory:} N/A
\end{styleBody}

\begin{styleBody}
\textbf{Vocalic nucleus patterns:} Short vowels, Long vowels
\end{styleBody}

\begin{styleBody}
\textbf{Syllabic consonant patterns:} N/A
\end{styleBody}

\begin{styleBody}
\textbf{Size of maximal word{}-marginal sequences with syllabic obstruents:} N/A
\end{styleBody}

\begin{styleBody}
\textbf{Predictability of syllabic consonants:} N/A
\end{styleBody}

\begin{styleBody}
\textbf{Morphological constituency of maximal syllable margin:} Morphologically Complex (Onset)
\end{styleBody}

\begin{styleBody}
\textbf{Morphological pattern of syllabic consonants:} N/A
\end{styleBody}

\begin{styleBody}
\textbf{Onset restrictions: }Simple onsets are apparently unrestricted. In biconsonantal onsets, any non-glide consonant may occur as C\textsubscript{1} if followed by a bilabial glide /w/ as C\textsubscript{2}. Nasal + consonant sequences also occur as biconsonantal onsets. Triconsonantal onsets have a nasal as C\textsubscript{1}, any non-nasal, non-glide consonant as C\textsubscript{2}, and a glide (bilabial) as C\textsubscript{3}.
\end{styleBody}

\begin{styleBody}
\textbf{Notes: }Kawasha discusses onset restrictions in terms of glides, of which there are two (/w j/) in the language, but in examples only /w/ occurs in clusters.
\end{styleBody}

\begin{styleBody}
\textbf{Suprasegmentals}
\end{styleBody}

\begin{styleBody}
\textbf{Tone:} Yes
\end{styleBody}

\begin{styleBody}
\textbf{Word stress: }Not reported
\end{styleBody}

\begin{styleBody}
\textbf{Vowel reduction processes}
\end{styleBody}

\begin{styleBody}
\textbf{lun-R1}: A word-final high vowel /i/ is realized as “voiceless and muted” when following semi-vowels /w j/, glottal consonant /h/, or nasal /m/ in continuous speech. Examples show that this is syncope, not devoicing (Kawasha 2003: 37-8).
\end{styleBody}

\begin{styleBody}
\textbf{Consonant allophony processes}
\end{styleBody}

\begin{styleBody}
(none reported)
\end{styleBody}

\begin{styleBody}
\textbf{Morphology}
\end{styleBody}

\begin{styleBody}
(adequate texts unavailable)
\end{styleBody}

\clearpage\begin{styleBody}
\textbf{[mcr] }\ \ \textbf{\textsc{Menya}}\textbf{\ \ \ \ }Angan, \textit{Nuclear Angan} (Papua New Guinea)
\end{styleBody}

\begin{styleBody}
\textbf{References consulted: }Whitehead (1992, 2004)
\end{styleBody}

\begin{styleBody}
\textbf{Sound inventory}
\end{styleBody}

\begin{styleBody}
\textbf{C phoneme inventory:} /p t[32A?] k q \textsuperscript{m}b \textsuperscript{n}d[32A?] \textsuperscript{ŋ}[261?] \textsuperscript{[274?]}[262?] t[361?][283?] \textsuperscript{[272?]}d[361?][292?] h m n[32A?] [272?] ŋ w j/
\end{styleBody}

\begin{styleBody}
\textbf{N consonant phonemes:} 17
\end{styleBody}

\begin{styleBody}
\textbf{Geminates:} N/A
\end{styleBody}

\begin{styleBody}
\textbf{Voicing contrasts:} Obstruents
\end{styleBody}

\begin{styleBody}
\textbf{Places:} Bilabial, Dental, Palato-alveolar, Velar, Uvular, Glottal
\end{styleBody}

\begin{styleBody}
\textbf{Manners:} Stop, Affricate, Fricative, Nasal, Central approximant
\end{styleBody}

\begin{styleBody}
\textbf{N elaborations:} 4
\end{styleBody}

\begin{styleBody}
\textbf{Elaborations:} Voiced fricatives/affricates, Prenasalization, Palato-alveolar, Uvular
\end{styleBody}

\begin{styleBody}
\textbf{V phoneme inventory:} /i e [259?] a o u/
\end{styleBody}

\begin{styleBody}
\textbf{N vowel qualities:} 6
\end{styleBody}

\begin{styleBody}
\textbf{Diphthongs or vowel sequences:} Vowel sequences /u[259?] io ue u[259?] ua ai uau/ (perhaps more)
\end{styleBody}

\begin{styleBody}
\textbf{Contrastive length:} None
\end{styleBody}

\begin{styleBody}
\textbf{Contrastive nasalization:} None
\end{styleBody}

\begin{styleBody}
\textbf{Other contrasts:} N/A
\end{styleBody}

\begin{styleBody}
\textbf{Notes:} /q/ most frequent consonant phoneme in this language.
\end{styleBody}

\begin{styleBody}
\textbf{Syllable structure}
\end{styleBody}

\begin{styleBody}
\textbf{Complexity Category:} Highly Complex
\end{styleBody}

\begin{styleBody}
\textbf{Canonical syllable structure:} (C)(C)(C)V(C)\textbf{ }(Whitehead 2004: 226; SIL OPD)
\end{styleBody}

\begin{styleBody}
\textbf{Size of maximal onset:} 3
\end{styleBody}

\begin{styleBody}
\textbf{Size of maximal coda:} 1
\end{styleBody}

\begin{styleBody}
\textbf{Onset obligatory:} No
\end{styleBody}

\begin{styleBody}
\textbf{Coda obligatory:} No
\end{styleBody}

\begin{styleBody}
\textbf{Vocalic nucleus patterns:} Short vowels, Vowel sequences
\end{styleBody}

\begin{styleBody}
\textbf{Syllabic consonant patterns:} Nasal
\end{styleBody}

\begin{styleBody}
\textbf{Size of maximal word{}-marginal sequences with syllabic obstruents:} N/A
\end{styleBody}

\begin{styleBody}
\textbf{Predictability of syllabic consonants:} Predictable from word/consonantal context
\end{styleBody}

\begin{styleBody}
\textbf{Morphological constituency of maximal syllable margin:} Morpheme-internal (Onset)
\end{styleBody}

\begin{styleBody}
\textbf{Morphological pattern of syllabic consonants:} Both
\end{styleBody}

\begin{styleBody}
\textbf{Onset restrictions: }No restrictions on simple onsets. Biconsonantal onsets include /tq, pk, pq/, at least, with suggestions of nasals occurring as well (2004: 9). Triconsonantal onsets include /tpq, ptq/ (/q/ typically produced as [[281?]] or [[263?]] after a plosive in clusters). 
\end{styleBody}

\begin{styleBody}
\textbf{Coda restrictions: }Nasals /m n/ occur.
\end{styleBody}

\begin{styleBody}
\textbf{Notes: }Non-homorganic consonants are separated by extremely short vocalic segments which are inconsistently produced and represented, “more and more not being written” (Whitehead 2004: 226). Quality seems to be conditioned by vowel harmony and/or surrounding consonants. When three plosives come together, there is a greater likelihood of one vowel being written but inconsistency as to which one (2004: 9).
\end{styleBody}

\begin{styleBody}
\textbf{Suprasegmentals}
\end{styleBody}

\begin{styleBody}
\textbf{Tone:} Yes
\end{styleBody}

\begin{styleBody}
\textbf{Word stress:} Yes
\end{styleBody}

\begin{styleBody}
\textbf{Stress placement:} Not described
\end{styleBody}

\begin{styleBody}
\textbf{Phonetic processes conditioned by stress:} (None)
\end{styleBody}

\begin{styleBody}
\textbf{Differences in phonological properties of stressed and unstressed syllables:} Not described
\end{styleBody}

\begin{styleBody}
\textbf{Phonetic correlates of stress: }Not described
\end{styleBody}

\begin{styleBody}
\textbf{Notes:} Tone and stress described as interdependent in language, with tone being phonemic but having low functional load (Whitehead 2004: 226). 
\end{styleBody}

\begin{styleBody}
\textbf{Vowel reduction processes}
\end{styleBody}

\begin{styleBody}
(none reported)
\end{styleBody}

\begin{styleBody}
\textbf{Consonant allophony processes}
\end{styleBody}

\begin{styleBody}
\textbf{mcr-C1: }A voiceless dental stop is in free variation with a flap and a lateral approximant in intervocalic position (Whitehead 2004: 9).
\end{styleBody}

\begin{styleBody}
\textbf{mcr-C2: }A voiceless uvular stop varies with voiced uvular or velar fricatives in intervocalic position (Whitehead 2004: 9).
\end{styleBody}

\begin{styleBody}
\textbf{Morphology}
\end{styleBody}

\begin{styleBody}
\textbf{Text: }“Hunting expedition” (first 20 pages, Whitehead 2004: 238-257)
\end{styleBody}

\begin{styleBody}
\textbf{Synthetic index: }2.5 morphemes/word (745 morphemes, 301 words)
\end{styleBody}

\clearpage\begin{styleBody}
\textbf{[mdx] }\ \ \textbf{\textsc{Dizin (Central dialect)}}\textbf{\ \ }Dizoid (Ethiopia)
\end{styleBody}

\begin{styleBody}
\textbf{References consulted: }Allan (1976), Beachy (2005), Breeze (1988)
\end{styleBody}

\begin{styleBody}
\textbf{Sound inventory}
\end{styleBody}

\begin{styleBody}
\textbf{C phoneme inventory:} /p b t d k [261?] p’ t’ k’ t[361?]s t[361?][283?] d[361?][292?] t[361?]s’ t[361?][283?]’ [278?] s z [283?] [292?] h m n ŋ [27E?] l w j/
\end{styleBody}

\begin{styleBody}
\textbf{N consonant phonemes:} 27
\end{styleBody}

\begin{styleBody}
\textbf{Geminates:} N/A
\end{styleBody}

\begin{styleBody}
\textbf{Voicing contrasts:} Obstruents
\end{styleBody}

\begin{styleBody}
\textbf{Places:} Bilabial, Alveolar, Palato-alveolar, Velar, Glottal
\end{styleBody}

\begin{styleBody}
\textbf{Manners:} Stop, Affricate, Fricative, Nasal, Flap/Tap, Central approximant, Lateral approximant
\end{styleBody}

\begin{styleBody}
\textbf{N elaborations:} 3
\end{styleBody}

\begin{styleBody}
\textbf{Elaborations:} Voiced fricatives/affricates, Ejective, Palato-alveolar
\end{styleBody}

\begin{styleBody}
\textbf{V phoneme inventory:} /i e [25B?] [268?] [251?] o u i[2D0?] e[2D0?] [251?][2D0?] o[2D0?] u[2D0?]/
\end{styleBody}

\begin{styleBody}
\textbf{N vowel qualities:} 7
\end{styleBody}

\begin{styleBody}
\textbf{Diphthongs or vowel sequences:} None
\end{styleBody}

\begin{styleBody}
\textbf{Contrastive length:} Some
\end{styleBody}

\begin{styleBody}
\textbf{Contrastive nasalization:} None
\end{styleBody}

\begin{styleBody}
\textbf{Other contrasts:} N/A
\end{styleBody}

\begin{styleBody}
\textbf{Notes:} /[294?]/ is posited to avoid underlying syllabic nasals, otherwise its occurrence is completely predictable. I do not include it. /[288?][361?][282?] [288?][361?][282?]’ [282?] [290?]/ occur only in Western Dizin (Beach 2005). Allan gives 24 consonant phonemes, lists an inventory that is quite divergent from those posited by Beachy and Breeze.
\end{styleBody}

\begin{styleBody}
\textbf{Syllable structure}
\end{styleBody}

\begin{styleBody}
\textbf{Complexity Category:} Complex
\end{styleBody}

\begin{styleBody}
\textbf{Canonical syllable structure:} (C)V(C)(C)(C)\textbf{ }(Beachy 2005: 38-46)
\end{styleBody}

\begin{styleBody}
\textbf{Size of maximal onset:} 1
\end{styleBody}

\begin{styleBody}
\textbf{Size of maximal coda:} 3
\end{styleBody}

\begin{styleBody}
\textbf{Onset obligatory:} No
\end{styleBody}

\begin{styleBody}
\textbf{Coda obligatory:} No
\end{styleBody}

\begin{styleBody}
\textbf{Vocalic nucleus patterns:} Short vowels, Long vowels
\end{styleBody}

\begin{styleBody}
\textbf{Syllabic consonant patterns:} Nasal
\end{styleBody}

\begin{styleBody}
\textbf{Size of maximal word{}-marginal sequences with syllabic obstruents:} N/A
\end{styleBody}

\begin{styleBody}
\textbf{Predictability of syllabic consonants:} Phonemic
\end{styleBody}

\begin{styleBody}
\textbf{Morphological constituency of maximal syllable margin:} Morpheme-internal (Coda)
\end{styleBody}

\begin{styleBody}
\textbf{Morphological pattern of syllabic consonants:} Both
\end{styleBody}

\begin{styleBody}
\textbf{Onset restrictions: }All consonants except /p/ may occur (though /p/ occurs as onset in loanwords).
\end{styleBody}

\begin{styleBody}
\textbf{Coda restrictions: }For simple codas, all consonants except /[294?], k’, d[361?][292?]/ may occur. Biconsonantal coda combinations are fairly free, though not all possible combinations occur and most follow a rising sonority contour according to a standard six-point hierarchy . In tri-consonantal codas, C\textsubscript{1} is /j/, C\textsubscript{2} is /n/, and C\textsubscript{3} is /t, d, s, [283?]/.
\end{styleBody}

\begin{styleBody}
\textbf{Notes: }Syllabic nasal nuclei occur only in syllables with predictable obligatory onset of [[294?]] and optional coda of /t/ (Beachy 2005: 41).
\end{styleBody}

\begin{styleBody}
\textbf{Suprasegmentals}
\end{styleBody}

\begin{styleBody}
\textbf{Tone: }Yes
\end{styleBody}

\begin{styleBody}
\textbf{Word stress:} Yes
\end{styleBody}

\begin{styleBody}
\textbf{Stress placement:} Fixed
\end{styleBody}

\begin{styleBody}
\textbf{Phonetic processes conditioned by stress:} Vowel Reduction
\end{styleBody}

\begin{styleBody}
\textbf{Differences in phonological properties of stressed and unstressed syllables:} Not described
\end{styleBody}

\begin{styleBody}
\textbf{Phonetic correlates of stress: }Not described
\end{styleBody}

\begin{styleBody}
\textbf{Vowel reduction processes}
\end{styleBody}

\begin{styleBody}
\textbf{mdx-R1:} Short high vowels /i u/ are sometimes realized as voiceless when word-final (Beachy 2005: 35-6).
\end{styleBody}

\begin{styleBody}
\textbf{mdx-R2:} The phoneme /[25B?]/ is optionally realized as [[259?]], but no conditioning environment given (Beachy 2005: 37).
\end{styleBody}

\begin{styleBody}
\textbf{Consonant allophony processes}
\end{styleBody}

\begin{styleBody}
\textbf{mdx-C1: }A voiceless alveolar fricative is realized as voiced preceding /d/ (Beachy 2005: 26).
\end{styleBody}

\begin{styleBody}
\textbf{mdx-C2: }Voiced bilabial and velar stops are realized as fricatives word-finally (Beachy 2005: 17).
\end{styleBody}

\begin{styleBody}
\textbf{mdx-C3: }A voiceless bilabial stop varies with a bilabial fricative and a labiodental fricative word-internally and word-finally (Beachy 2005: 17).
\end{styleBody}

\begin{styleBody}
\textbf{Morphology}
\end{styleBody}

\begin{styleBody}
\textbf{Text: }“A lion and a fox” (Beachy 2005: 154-158)
\end{styleBody}

\begin{styleBody}
\textbf{Synthetic index: }1.9 morphemes/word (485 morphemes, 251 words)
\end{styleBody}

\clearpage\begin{styleBody}
\textbf{[mhi] }\ \ \textbf{\textsc{Ma’di}}\textbf{\ \ }Central Sudanic, \textit{Moru-Madi} (South Sudan, Uganda)
\end{styleBody}

\begin{styleBody}
\textbf{References consulted: }Blackings \& Fabb (2003)
\end{styleBody}

\begin{styleBody}
\textbf{Sound inventory}
\end{styleBody}

\begin{styleBody}
\textbf{C phoneme inventory:} /p b t d t[2B7?] d[2B7?] k [261?] k[2B7?] [261?][2B7?] k[361?]p [261?][361?]b [294?] [294?][2B7?] \textsuperscript{m}b \textsuperscript{n}d \textsuperscript{n}d[2B7?] \textsuperscript{ŋ}[261?] \textsuperscript{ŋ}[261?][2B7?] \textsuperscript{ŋ[361?]m}[261?][361?]b [253?] [257?] [284?] [260?][361?][253?] t[361?][283?][2B7?] t[361?][283?] d[361?][292?] \textsuperscript{[272?]}d[361?][292?] f v s z h \textsuperscript{[271?]}v m n [272?] ŋ[361?]m r r[2B7?] l l[2B7?] j w/
\end{styleBody}

\begin{styleBody}
\textbf{N consonant phonemes:} 44
\end{styleBody}

\begin{styleBody}
\textbf{Geminates:} N/A
\end{styleBody}

\begin{styleBody}
\textbf{Voicing contrasts: }Obstruents
\end{styleBody}

\begin{styleBody}
\textbf{Places: }Labial-velar, Bilabial, Labiodental, Alveolar, Palato-alveolar, Palatal, Velar, Glottal
\end{styleBody}

\begin{styleBody}
\textbf{Manners:} Stop, Affricate, Fricative, Nasal, Trill, Central approximant, Lateral approximant
\end{styleBody}

\begin{styleBody}
\textbf{N elaborations:} 6
\end{styleBody}

\begin{styleBody}
\textbf{Elaborations:} Voiced fricatives/affricates, Prenasalization, Implosive, Labiodental, Palato-alveolar, Labialization
\end{styleBody}

\begin{styleBody}
\textbf{V phoneme inventory:} /i [26A?] e [25B?] a [254?] o [28A?] u/
\end{styleBody}

\begin{styleBody}
\textbf{N vowel qualities:} 9
\end{styleBody}

\begin{styleBody}
\textbf{Diphthongs or vowel sequences:} None
\end{styleBody}

\begin{styleBody}
\textbf{Contrastive length:} None
\end{styleBody}

\begin{styleBody}
\textbf{Contrastive nasalization:} None
\end{styleBody}

\begin{styleBody}
\textbf{Other contrasts:} N/A
\end{styleBody}

\begin{styleBody}
\textbf{Notes:}
\end{styleBody}

\begin{styleBody}
\textbf{Syllable structure}
\end{styleBody}

\begin{styleBody}
\textbf{Complexity Category:} Simple
\end{styleBody}

\begin{styleBody}
\textbf{Canonical syllable structure:} (C)V (Blackings \& Fabb 2003: 34-35)
\end{styleBody}

\begin{styleBody}
\textbf{Size of maximal onset:} 1
\end{styleBody}

\begin{styleBody}
\textbf{Size of maximal coda: }N/A
\end{styleBody}

\begin{styleBody}
\textbf{Onset obligatory:} No
\end{styleBody}

\begin{styleBody}
\textbf{Coda obligatory:} N/A
\end{styleBody}

\begin{styleBody}
\textbf{Vocalic nucleus patterns: }Short vowels
\end{styleBody}

\begin{styleBody}
\textbf{Syllabic consonant patterns:} N/A
\end{styleBody}

\begin{styleBody}
\textbf{Size of maximal word{}-marginal sequences with syllabic obstruents:} N/A
\end{styleBody}

\begin{styleBody}
\textbf{Predictability of syllabic consonants:} N/A
\end{styleBody}

\begin{styleBody}
\textbf{Morphological constituency of maximal syllable margin:} N/A
\end{styleBody}

\begin{styleBody}
\textbf{Morphological pattern of syllabic consonants:} N/A
\end{styleBody}

\begin{styleBody}
\textbf{Onset restrictions:} None. CC onsets may occur in loanwords in speech of educated people.
\end{styleBody}

\begin{styleBody}
\textbf{Coda restrictions:} N/A
\end{styleBody}

\begin{styleBody}
\textbf{Notes: }Codas may occur in ideophones and in educated variants.
\end{styleBody}

\begin{styleBody}
\textbf{Suprasegmentals}
\end{styleBody}

\begin{styleBody}
\textbf{Tone:} Yes
\end{styleBody}

\begin{styleBody}
\textbf{Word stress: }No
\end{styleBody}

\begin{styleBody}
\textbf{Vowel reduction processes}
\end{styleBody}

\begin{styleBody}
(none)
\end{styleBody}

\begin{styleBody}
\textbf{Consonant allophony processes}
\end{styleBody}

\begin{styleBody}
(none)
\end{styleBody}

\begin{styleBody}
\textbf{Morphology}
\end{styleBody}

\begin{styleBody}
\textbf{Text:} “Hare, Caragule, and the water dance” (Blackings \& Fabb 2003: 671-677)
\end{styleBody}

\begin{styleBody}
\textbf{Synthetic index: }1.09 morphemes/word (440 morphemes, 405 words)
\end{styleBody}

\clearpage\begin{styleBody}
\textbf{[mio] }\ \ \textbf{\textsc{Pinotepa Mixtec}}\textbf{\ \ }Otomanguean, \textit{Eastern Otomanguean} (Mexico)
\end{styleBody}

\begin{styleBody}
\textbf{References consulted: }Bradley (1971), Costello (2014)
\end{styleBody}

\begin{styleBody}
\textbf{Sound inventory}
\end{styleBody}

\begin{styleBody}
\textbf{C phoneme inventory:} /p t[32A?] t[32A?][2B2?] k k[2B7?] [294?] \textsuperscript{m}b \textsuperscript{n}d[32A?] \textsuperscript{n}d[2B2?] \textsuperscript{ŋ}[261?] t[361?][283?] s [283?] m n[32A?] [272?] l[32A?] [27E?] w j/
\end{styleBody}

\begin{styleBody}
\textbf{N consonant phonemes:} 20
\end{styleBody}

\begin{styleBody}
\textbf{Geminates:} N/A
\end{styleBody}

\begin{styleBody}
\textbf{Voicing contrasts:} Obstruents
\end{styleBody}

\begin{styleBody}
\textbf{Places:} Bilabial, Dental, Alveolar, Palato-alveolar, Velar, Glottal
\end{styleBody}

\begin{styleBody}
\textbf{Manners:} Stop, Affricate, Fricative, Nasal, Flap/Tap, Central approximant, Lateral approximant
\end{styleBody}

\begin{styleBody}
\textbf{N elaborations:} 4
\end{styleBody}

\begin{styleBody}
\textbf{Elaborations:} Prenasalization, Palato-alveolar, Palatalization, Labialization
\end{styleBody}

\begin{styleBody}
\textbf{V phoneme inventory:} /i e a o u \~{i} \~{e} \~{a} \~{o} \~{u} i[330?] e[330?] a[330?] o[330?] u[330?] i[330?] e[330?] a[330?] o[330?] u[330?]/
\end{styleBody}

\begin{styleBody}
\textbf{N vowel qualities:} 5
\end{styleBody}

\begin{styleBody}
\textbf{Diphthongs or vowel sequences:} None
\end{styleBody}

\begin{styleBody}
\textbf{Contrastive length:} None
\end{styleBody}

\begin{styleBody}
\textbf{Contrastive nasalization:} All
\end{styleBody}

\begin{styleBody}
\textbf{Other contrasts: }Glottalization (All)
\end{styleBody}

\begin{styleBody}
\textbf{Notes:} Author calls /[294?]/ a ‘semiconsonant’. /[278?] s[2B2?]/ occur only in Spanish loans. /x/ occurs in diminutive speech style.
\end{styleBody}

\begin{styleBody}
\textbf{Syllable structure}
\end{styleBody}

\begin{styleBody}
\textbf{Complexity Category: }Simple
\end{styleBody}

\begin{styleBody}
\textbf{Canonical syllable structure:} (C)V\textbf{ }(Bradley 1970: 14)
\end{styleBody}

\begin{styleBody}
\textbf{Size of maximal onset:} 1
\end{styleBody}

\begin{styleBody}
\textbf{Size of maximal coda:} N/A
\end{styleBody}

\begin{styleBody}
\textbf{Onset obligatory:} No
\end{styleBody}

\begin{styleBody}
\textbf{Coda obligatory:} N/A
\end{styleBody}

\begin{styleBody}
\textbf{Vocalic nucleus patterns:} Short vowels
\end{styleBody}

\begin{styleBody}
\textbf{Syllabic consonant patterns:} N/A
\end{styleBody}

\begin{styleBody}
\textbf{Size of maximal word{}-marginal sequences with syllabic obstruents:} N/A
\end{styleBody}

\begin{styleBody}
\textbf{Predictability of syllabic consonants:} N/A
\end{styleBody}

\begin{styleBody}
\textbf{Morphological constituency of maximal syllable margin:} N/A
\end{styleBody}

\begin{styleBody}
\textbf{Morphological pattern of syllabic consonants:} N/A
\end{styleBody}

\begin{styleBody}
\textbf{Onset restrictions: }All consonants occur.
\end{styleBody}

\begin{styleBody}
\textbf{Notes: }Bradley describes glottal stop as a semiconsonant which may attach to a vowel to form a complex nucleus and thus a ‘checked’ syllable (1970: 14). Costello analyzes syllable template as (C)V(C) with glottal stop as the only acceptable coda (2014: 24-5). Both authors note that checked syllables/glottal codas occur only in stressed/tonic syllables. Since the glottal stop has a very limited distribution and does not behave like a prototypical coda, I consider this language to have Simple syllable structure. The analysis of the glottal stop as a laryngeal feature of the syllable has been proposed for other Mixtecan languages as well (e.g., Juchitán Zapotec, Marlett \& Pickett 1987).
\end{styleBody}

\begin{styleBody}
\textbf{Suprasegmentals}
\end{styleBody}

\begin{styleBody}
\textbf{Tone:} Yes
\end{styleBody}

\begin{styleBody}
\textbf{Word stress:} Yes
\end{styleBody}

\begin{styleBody}
\textbf{Stress placement:} Fixed
\end{styleBody}

\begin{styleBody}
\textbf{Phonetic processes conditioned by stress:} Vowel Reduction, Consonant Allophony in Unstressed Syllables, Consonant Allophony in Stressed Syllables
\end{styleBody}

\begin{styleBody}
\textbf{Differences in phonological properties of stressed and unstressed syllables:} Not described
\end{styleBody}

\begin{styleBody}
\textbf{Phonetic correlates of stress: }Vowel duration (impressionistic), Intensity (impressionistic)
\end{styleBody}

\begin{styleBody}
\textbf{Vowel reduction processes}
\end{styleBody}

\begin{styleBody}
\textbf{mio-R1:} A final unstressed vowel decays at the end of a terminal contour, following a pause (Bradley 1970: 13).
\end{styleBody}

\begin{styleBody}
\textbf{Consonant allophony processes}
\end{styleBody}

\begin{styleBody}
\textbf{mio-C1: }Prenasalized dental stop may be realized as palato-alveolar in a post-tonic syllable immediately following the tonic syllable (Bradley 1970: 6).
\end{styleBody}

\begin{styleBody}
\textbf{mio-C2: }Labiovelar and palatal glides are fricated in tonic syllables (Bradley 1970: 8).
\end{styleBody}

\begin{styleBody}
\textbf{mio-C3: }/t[361?][283?] k k[2B7?]/ may be realized as voiced in post-tonic syllables (Bradley 1970: 5).
\end{styleBody}

\begin{styleBody}
\textbf{mio-C4: }/k[2B7?]/ is occasionally voiced in pretonic syllables (Bradley 1970: 5).
\end{styleBody}

\begin{styleBody}
\textbf{Morphology}
\end{styleBody}

\begin{styleBody}
(adequate texts unavailable)
\end{styleBody}

\clearpage\begin{styleBody}
\textbf{[mjg] }\ \ \textbf{\textsc{Tu}}\textbf{\ \ \ \ }Mongolic, \textit{Southern Periphery Mongolic} (China)
\end{styleBody}

\begin{styleBody}
\textbf{References consulted: }Slater (2003)
\end{styleBody}

\begin{styleBody}
\textbf{Sound inventory}
\end{styleBody}

\begin{styleBody}
\textbf{C phoneme inventory:} /p[2B0?] p t[2B0?] t k[2B0?] k q[2B0?] q t[361?]s[2B0?] t[361?]s t[361?][255?][2B0?] t[361?][255?] t[361?][282?][2B0?] t[361?][282?] f s [255?] [282?] $\chi $ m n ŋ l [27B?] w j/
\end{styleBody}

\begin{styleBody}
\textbf{N consonant phonemes:} 26
\end{styleBody}

\begin{styleBody}
\textbf{Geminates:} N/A
\end{styleBody}

\begin{styleBody}
\textbf{Voicing contrasts:} None
\end{styleBody}

\begin{styleBody}
\textbf{Places:} Bilabial, Labiodental, Alveolar, Retroflex, Alveolo-palatal, Velar, Uvular
\end{styleBody}

\begin{styleBody}
\textbf{Manners:} Stop, Affricate, Fricative, Nasal, Central approximant, Lateral approximant
\end{styleBody}

\begin{styleBody}
\textbf{N elaborations:} 4
\end{styleBody}

\begin{styleBody}
\textbf{Elaborations:} Post-aspiration, Labiodental, Retroflex, Uvular
\end{styleBody}

\begin{styleBody}
\textbf{V phoneme inventory:} /i e a o u/
\end{styleBody}

\begin{styleBody}
\textbf{N vowel qualities:} 5
\end{styleBody}

\begin{styleBody}
\textbf{Diphthongs or vowel sequences:} None
\end{styleBody}

\begin{styleBody}
\textbf{Contrastive length:} None
\end{styleBody}

\begin{styleBody}
\textbf{Contrastive nasalization:} None
\end{styleBody}

\begin{styleBody}
\textbf{Other contrasts:} N/A
\end{styleBody}

\begin{styleBody}
\textbf{Notes:} /t[361?][255?][2B0?] t[361?][255?] [255?]/ are described as being post-alveolar most often, but symbols indicate alveolo-palatal. Absence of contrastive vowel length is unusual for a Mongolic language.
\end{styleBody}

\begin{styleBody}
\textbf{Syllable structure}
\end{styleBody}

\begin{styleBody}
\textbf{Complexity Category:} Moderately Complex
\end{styleBody}

\begin{styleBody}
\textbf{Canonical syllable structure:} (C)(C)V(C)\textbf{ }(Slater 2003: 54-72)
\end{styleBody}

\begin{styleBody}
\textbf{Size of maximal onset:} 2
\end{styleBody}

\begin{styleBody}
\textbf{Size of maximal coda:} 1
\end{styleBody}

\begin{styleBody}
\textbf{Onset obligatory:} No
\end{styleBody}

\begin{styleBody}
\textbf{Coda obligatory:} No
\end{styleBody}

\begin{styleBody}
\textbf{Vocalic nucleus patterns:} Short vowels
\end{styleBody}

\begin{styleBody}
\textbf{Syllabic consonant patterns:} Liquid
\end{styleBody}

\begin{styleBody}
\textbf{Size of maximal word{}-marginal sequences with syllabic obstruents:} N/A
\end{styleBody}

\begin{styleBody}
\textbf{Predictability of syllabic consonants:} Phonemic
\end{styleBody}

\begin{styleBody}
\textbf{Morphological constituency of maximal syllable margin:} Morpheme-internal (Onset)
\end{styleBody}

\begin{styleBody}
\textbf{Morphological pattern of syllabic consonants:} Lexical items
\end{styleBody}

\begin{styleBody}
\textbf{Onset restrictions: }C\textsubscript{1} may be any consonant except /ŋ/, but may not be identical to C\textsubscript{2}. C\textsubscript{2} must be glide /j/ or /w/. 
\end{styleBody}

\begin{styleBody}
\textbf{Coda restrictions: }Restricted to /[27B?] ŋ n j w/.
\end{styleBody}

\begin{styleBody}
\textbf{Suprasegmentals}
\end{styleBody}

\begin{styleBody}
\textbf{Tone:} No
\end{styleBody}

\begin{styleBody}
\textbf{Word stress:} Yes
\end{styleBody}

\begin{styleBody}
\textbf{Stress placement:} Fixed
\end{styleBody}

\begin{styleBody}
\textbf{Phonetic processes conditioned by stress:} Vowel Reduction, Consonant Allophony in Unstressed Syllables, Consonant Allophony in Stressed Syllables
\end{styleBody}

\begin{styleBody}
\textbf{Differences in phonological properties of stressed and unstressed syllables:} (None)
\end{styleBody}

\begin{styleBody}
\textbf{Phonetic correlates of stress: }Pitch (impressionistic), Intensity (impressionistic)
\end{styleBody}

\begin{styleBody}
\textbf{Notes: }Different outcomes for /i/ and /e/. /i/ generally realized as somewhat central, but may move towards quality [i], especially in stressed syllables. /e/ is [[25B?]]\~{}[[259?]] in most contexts, but [[259?]] generally appears in stressed syllables without onset clusters or codas. I’m not coding this as vowel reduction because it seems there is free variation even within stressed syllables.
\end{styleBody}

\begin{styleBody}
\textbf{Vowel reduction processes}
\end{styleBody}

\begin{styleBody}
\textbf{mjg-R1:} High vowels /i u/ are realized as lax in unstressed syllables (Slater 2003:35).
\end{styleBody}

\begin{styleBody}
\textbf{mjg-R2:} High vowels /i u/ and mid front vowel /e/ are often devoiced following a voiceless consonant. This typically occurs in medial unstressed syllables and is most regular following a voiceless fricative (Slater 2003: 36).
\end{styleBody}

\begin{styleBody}
\textbf{Consonant allophony processes}
\end{styleBody}

\begin{styleBody}
\textbf{mjg-C1: }A palatal glide is realized as a fricative in the onset of a stressed syllable (Slater 2003: 31-2).
\end{styleBody}

\begin{styleBody}
\textbf{mjg-C2: }A retroflex approximant is realized as fricative [[290?]] in the onset of a stressed syllable (Slater 2003: 30).
\end{styleBody}

\begin{styleBody}
\textbf{mjg-C3: }A retroflex approximant is realized as an alveolar flap intervocalically before an unstressed vowel (Slater 2003: 31).
\end{styleBody}

\begin{styleBody}
\textbf{Morphology}
\end{styleBody}

\begin{styleBody}
\textbf{Text: }“Rabbit’s trick” (Slater 2003: 343-350)
\end{styleBody}

\begin{styleBody}
\textbf{Synthetic index: }1.5 morphemes/word (547 morphemes, 377 words)
\end{styleBody}

\clearpage\begin{styleBody}
\textbf{[mji] }\ \ \textbf{\textsc{Kim Mun (Vietnam dialect)}}\textbf{\ \ }Hmong-Mien, \textit{Mienic} (Vietnam)
\end{styleBody}

\begin{styleBody}
\textbf{References consulted: }Clark (2008)
\end{styleBody}

\begin{styleBody}
\textbf{Sound inventory}
\end{styleBody}

\begin{styleBody}
\textbf{C phoneme inventory:} /p b t d c [25F?] k [261?] f v $\theta $ s h m n [272?] ŋ l [28E?] w j/
\end{styleBody}

\begin{styleBody}
\textbf{N consonant phonemes:} 21
\end{styleBody}

\begin{styleBody}
\textbf{Geminates:} N/A
\end{styleBody}

\begin{styleBody}
\textbf{Voicing contrasts:} Obstruents
\end{styleBody}

\begin{styleBody}
\textbf{Places:} Bilabial, Labiodental, Dental, Alveolar, Palatal, Velar, Glottal
\end{styleBody}

\begin{styleBody}
\textbf{Manners:} Stop, Fricative, Nasal, Central approximant, Lateral approximant
\end{styleBody}

\begin{styleBody}
\textbf{N elaborations:} 2
\end{styleBody}

\begin{styleBody}
\textbf{Elaborations:} Voiced fricatives/affricates, Labiodental
\end{styleBody}

\begin{styleBody}
\textbf{V phoneme inventory:} /i e [25B?] [250?] a [254?] o u a[2D0?]/
\end{styleBody}

\begin{styleBody}
\textbf{N vowel qualities:} 8
\end{styleBody}

\begin{styleBody}
\textbf{Diphthongs or vowel sequences:} None
\end{styleBody}

\begin{styleBody}
\textbf{Contrastive length:} Some
\end{styleBody}

\begin{styleBody}
\textbf{Contrastive nasalization:} None
\end{styleBody}

\begin{styleBody}
\textbf{Other contrasts:} N/A
\end{styleBody}

\begin{styleBody}
\textbf{Notes:} Length contrastive for /a/ only.
\end{styleBody}

\begin{styleBody}
\textbf{Syllable structure}
\end{styleBody}

\begin{styleBody}
\textbf{Complexity Category:} Moderately Complex
\end{styleBody}

\begin{styleBody}
\textbf{Canonical syllable structure:} (C)(C)V(C)\textbf{ }(Clark 2008: 123-7)
\end{styleBody}

\begin{styleBody}
\textbf{Size of maximal onset:} 2
\end{styleBody}

\begin{styleBody}
\textbf{Size of maximal coda:} 1
\end{styleBody}

\begin{styleBody}
\textbf{Onset obligatory:} No
\end{styleBody}

\begin{styleBody}
\textbf{Coda obligatory:} No
\end{styleBody}

\begin{styleBody}
\textbf{Vocalic nucleus patterns:} Short vowels, Long vowels
\end{styleBody}

\begin{styleBody}
\textbf{Syllabic consonant patterns:} N/A
\end{styleBody}

\begin{styleBody}
\textbf{Size of maximal word{}-marginal sequences with syllabic obstruents:} N/A
\end{styleBody}

\begin{styleBody}
\textbf{Predictability of syllabic consonants:} N/A
\end{styleBody}

\begin{styleBody}
\textbf{Morphological constituency of maximal syllable margin:} Morpheme-internal (Onset)
\end{styleBody}

\begin{styleBody}
\textbf{Morphological pattern of syllabic consonants:} N/A
\end{styleBody}

\begin{styleBody}
\textbf{Onset restrictions:} Simple onsets unrestricted. For complex onsets, C\textsubscript{1} must be /p b t k [261?]/, and C\textsubscript{2} must be /l w j/. 
\end{styleBody}

\begin{styleBody}
\textbf{Coda restrictions: }Restricted to nasals, glides, and /p t/.
\end{styleBody}

\begin{styleBody}
\textbf{Notes: }It is possible Vietnam Kim Mun is in the process of losing onset clusters, as vowel epenthesis sometimes occurs in /kl/ sequences (Clark 2008: 127).
\end{styleBody}

\begin{styleBody}
\textbf{Suprasegmentals}
\end{styleBody}

\begin{styleBody}
\textbf{Tone:} Yes
\end{styleBody}

\begin{styleBody}
\textbf{Word stress:} Not reported
\end{styleBody}

\begin{styleBody}
\textbf{Vowel reduction processes}
\end{styleBody}

\begin{styleBody}
\textbf{mji-R1:} Long vowels are shortened and produced with level tone in non-word-final syllables (Clark 2008: 117).
\end{styleBody}

\begin{styleBody}
\textbf{Consonant allophony processes}
\end{styleBody}

\begin{styleBody}
(none reported)
\end{styleBody}

\begin{styleBody}
\textbf{Morphology}
\end{styleBody}

\begin{styleBody}
(adequate texts unavailable)
\end{styleBody}

\clearpage\begin{styleBody}
\textbf{[moh] }\ \ \textbf{\textsc{Mohawk}}\textbf{\ \ }Iroquoian, \textit{Northern Iroquoian} (Canada, United States)
\end{styleBody}

\begin{styleBody}
\textbf{References consulted: }Bonvillain (1973), Michelson (1981, 1988)
\end{styleBody}

\begin{styleBody}
\textbf{Sound inventory}
\end{styleBody}

\begin{styleBody}
\textbf{C phoneme inventory:} /t k [294?] d[361?][292?] s h n l j w/
\end{styleBody}

\begin{styleBody}
\textbf{N consonant phonemes:} 10
\end{styleBody}

\begin{styleBody}
\textbf{Geminates:} N/A
\end{styleBody}

\begin{styleBody}
\textbf{Voicing contrasts:} None
\end{styleBody}

\begin{styleBody}
\textbf{Places: }Alveolar, Palato-alveolar, Velar, Glottal
\end{styleBody}

\begin{styleBody}
\textbf{Manners:} Stop, Affricate, Fricative, Nasal, Central approximant, Lateral approximant
\end{styleBody}

\begin{styleBody}
\textbf{N elaborations:} 2
\end{styleBody}

\begin{styleBody}
\textbf{Elaborations:} Voiced fricatives/affricates, Palato-alveolar
\end{styleBody}

\begin{styleBody}
\textbf{V phoneme inventory:} /i [26A?] e [28C?] a o u\~{ }/
\end{styleBody}

\begin{styleBody}
\textbf{N vowel qualities:} 7
\end{styleBody}

\begin{styleBody}
\textbf{Diphthongs or vowel sequences:} None
\end{styleBody}

\begin{styleBody}
\textbf{Contrastive length:} None
\end{styleBody}

\begin{styleBody}
\textbf{Contrastive nasalization:} None
\end{styleBody}

\begin{styleBody}
\textbf{Other contrasts:} None
\end{styleBody}

\begin{styleBody}
\textbf{Notes:} /[28C?] u/ are nasalized. Peripheral phonemic vowel /[26A?]/ occurs in two basic words (Bonvillain 1973: 43). Bonvillain states vowel length is predictable.
\end{styleBody}

\begin{styleBody}
\textbf{Syllable structure}
\end{styleBody}

\begin{styleBody}
\textbf{Complexity Category:} Highly Complex
\end{styleBody}

\begin{styleBody}
\textbf{Canonical syllable structure:} (C)(C)(C)(C)V(C)(C)(C)\textbf{ }(Bonvillain 1973: 21-23; Michelson 1981, 1988: 12)
\end{styleBody}

\begin{styleBody}
\textbf{Size of maximal onset:} 4
\end{styleBody}

\begin{styleBody}
\textbf{Size of maximal coda:} 3
\end{styleBody}

\begin{styleBody}
\textbf{Onset obligatory:} No
\end{styleBody}

\begin{styleBody}
\textbf{Coda obligatory:} No
\end{styleBody}

\begin{styleBody}
\textbf{Vocalic nucleus patterns:} Short vowels, Vowel sequences
\end{styleBody}

\begin{styleBody}
\textbf{Syllabic consonant patterns:} N/A
\end{styleBody}

\begin{styleBody}
\textbf{Size of maximal word{}-marginal sequences with syllabic obstruents:} N/A
\end{styleBody}

\begin{styleBody}
\textbf{Predictability of syllabic consonants:} N/A
\end{styleBody}

\begin{styleBody}
\textbf{Morphological constituency of maximal syllable margin:} Morphologically Complex (Onset, Coda)
\end{styleBody}

\begin{styleBody}
\textbf{Morphological pattern of syllabic consonants:} N/A
\end{styleBody}

\begin{styleBody}
\textbf{Onset restrictions: }All consonants may occur in simple onsets. Biconsonantal onsets are /nj tj kj kw ts ks st kt sk tk sh th kh/. Triconsonantal onsets always have /j s w h/ as a member, e.g. /tsj, ksk, kts, shw, shr, khn/. Four-consonant onsets are /shnj khnj/.
\end{styleBody}

\begin{styleBody}
\textbf{Coda restrictions: }All consonants except /d[361?][292?]/ may occur in simple codas. Biconsonantal codas include /ks [294?]s ts/. Triconsonantal codas are rare and highly restricted, include /[294?]ks [294?]ts kst/.
\end{styleBody}

\begin{styleBody}
\textbf{Notes: }Michelson writes that vowel epenthesis predictably breaks up triconsonantal onsets (1981), but lists many surface word-initial onsets in 1988 (p. 12).
\end{styleBody}

\begin{styleBody}
\textbf{Suprasegmentals}
\end{styleBody}

\begin{styleBody}
\textbf{Tone:} Yes
\end{styleBody}

\begin{styleBody}
\textbf{Word stress:} Yes
\end{styleBody}

\begin{styleBody}
\textbf{Stress placement:} Fixed
\end{styleBody}

\begin{styleBody}
\textbf{Phonetic processes conditioned by stress:} (None)
\end{styleBody}

\begin{styleBody}
\textbf{Differences in phonological properties of stressed and unstressed syllables:} (None)
\end{styleBody}

\begin{styleBody}
\textbf{Phonetic correlates of stress: }Pitch (impressionistic)
\end{styleBody}

\begin{styleBody}
\textbf{Notes: }Some co-occurrence of length with stress: all long vowels stressed, but not all stressed vowels long; lengthening is thus dependent on accent.
\end{styleBody}

\begin{styleBody}
\textbf{Vowel reduction processes}
\end{styleBody}

\begin{styleBody}
\textbf{moh-R1:} The length of a long vowel may be somewhat diminished in keeping with phrasal and sentence contours (Bonvillain 1973: 46).
\end{styleBody}

\begin{styleBody}
\textbf{Consonant allophony processes}
\end{styleBody}

\begin{styleBody}
\textbf{moh-C1: }A voiceless alveolar fricative is realized as palato-alveolar preceding /i/, by some speakers (Bonvillain 1973: 31).
\end{styleBody}

\begin{styleBody}
\textbf{moh-C2: }A labiovelar glide is realized as a labiodental fricative preceding /h/ (Bonvillain 1973: 34).
\end{styleBody}

\begin{styleBody}
\textbf{moh-C3: }A voiceless alveolar fricative is realized as voiced intervocalically (Bonvillain 1973).
\end{styleBody}

\begin{styleBody}
\textbf{moh-C4: }A voiceless alveolar fricative is realized as voiced word-initially preceding a vowel (Bonvillain 1973).
\end{styleBody}

\begin{styleBody}
\textbf{moh-C5: }Voiceless stops /t k/ are realized as voiced preceding a vowel with an optional intervening glide (Bonvillain 1973: 28).
\end{styleBody}

\begin{styleBody}
\textbf{Morphology}
\end{styleBody}

\begin{styleBody}
(adequate texts unavailable)
\end{styleBody}

\clearpage\begin{styleBody}
\textbf{[mpc] }\ \ \textbf{\textsc{Mangarrayi\ \ }}\textbf{\ \ }Mangarrayi-Maran (Australia)
\end{styleBody}

\begin{styleBody}
\textbf{References consulted: }Merlan (1989)
\end{styleBody}

\begin{styleBody}
\textbf{Sound inventory}
\end{styleBody}

\begin{styleBody}
\textbf{C phoneme inventory:} /b d [256?] [25F?] [261?] [294?] m n [273?] [272?] ŋ l [26D?] [279?] [27B?] w j/
\end{styleBody}

\begin{styleBody}
\textbf{N consonant phonemes:} 17
\end{styleBody}

\begin{styleBody}
\textbf{Geminates:} N/A
\end{styleBody}

\begin{styleBody}
\textbf{Voicing contrasts:} None
\end{styleBody}

\begin{styleBody}
\textbf{Places:} Bilabial, Alveolar, Retroflex, Palatal, Velar, Glottal
\end{styleBody}

\begin{styleBody}
\textbf{Manners:} Stop, Nasal, Central approximant, Lateral approximant
\end{styleBody}

\begin{styleBody}
\textbf{N elaborations:} 1
\end{styleBody}

\begin{styleBody}
\textbf{Elaborations:} Retroflex
\end{styleBody}

\begin{styleBody}
\textbf{V phoneme inventory:} /i e a o u/
\end{styleBody}

\begin{styleBody}
\textbf{N vowel qualities:} 5
\end{styleBody}

\begin{styleBody}
\textbf{Diphthongs or vowel sequences:} None
\end{styleBody}

\begin{styleBody}
\textbf{Contrastive length:} None
\end{styleBody}

\begin{styleBody}
\textbf{Contrastive nasalization:} None
\end{styleBody}

\begin{styleBody}
\textbf{Other contrasts:} N/A
\end{styleBody}

\begin{styleBody}
\textbf{Notes:} Voiced stop symbols used for single stop series.
\end{styleBody}

\begin{styleBody}
\textbf{Syllable structure}
\end{styleBody}

\begin{styleBody}
\textbf{Complexity Category:} Complex
\end{styleBody}

\begin{styleBody}
\textbf{Canonical syllable structure:} CV(C)(C)\textbf{ }(Merlan 1989: 186-96)
\end{styleBody}

\begin{styleBody}
\textbf{Size of maximal onset:} 1
\end{styleBody}

\begin{styleBody}
\textbf{Size of maximal coda:} 2
\end{styleBody}

\begin{styleBody}
\textbf{Onset obligatory:} Yes
\end{styleBody}

\begin{styleBody}
\textbf{Coda obligatory:} No
\end{styleBody}

\begin{styleBody}
\textbf{Vocalic nucleus patterns:} Short vowels
\end{styleBody}

\begin{styleBody}
\textbf{Syllabic consonant patterns:} N/A
\end{styleBody}

\begin{styleBody}
\textbf{Size of maximal word{}-marginal sequences with syllabic obstruents:} N/A
\end{styleBody}

\begin{styleBody}
\textbf{Predictability of syllabic consonants:} N/A
\end{styleBody}

\begin{styleBody}
\textbf{Morphological constituency of maximal syllable margin:} Both patterns (Coda)
\end{styleBody}

\begin{styleBody}
\textbf{Morphological pattern of syllabic consonants:} N/A
\end{styleBody}

\begin{styleBody}
\textbf{Onset restrictions: }All consonants except for rhotics /[279?] [27B?]/ may occur.
\end{styleBody}

\begin{styleBody}
\textbf{Coda restrictions: }Any consonant may occur as simple coda. Biconsonantal codas consist of a non-nasal sonorant /l [26D?] [279?] [27B?]/ followed by a stop or nasal, or nasal followed by glottal stop (e.g. /[272?][294?]/, p. 182).
\end{styleBody}

\begin{styleBody}
\textbf{Notes: }V syllables result from the reduction of irrealis prefix forms \textit{wa-} and \textit{ja-} to \textit{a-} (p. 196).
\end{styleBody}

\begin{styleBody}
\textbf{Suprasegmentals}
\end{styleBody}

\begin{styleBody}
\textbf{Tone:} No
\end{styleBody}

\begin{styleBody}
\textbf{Word stress:} Yes
\end{styleBody}

\begin{styleBody}
\textbf{Stress placement:} Fixed
\end{styleBody}

\begin{styleBody}
\textbf{Phonetic processes conditioned by stress:} (None)
\end{styleBody}

\begin{styleBody}
\textbf{Differences in phonological properties of stressed and unstressed syllables:} (None)
\end{styleBody}

\begin{styleBody}
\textbf{Phonetic correlates of stress: }Pitch (impressionistic), Intensity (impressionistic)
\end{styleBody}

\begin{styleBody}
\textbf{Vowel reduction processes}
\end{styleBody}

\begin{styleBody}
(none reported)
\end{styleBody}

\begin{styleBody}
\textbf{Consonant allophony processes}
\end{styleBody}

\begin{styleBody}
\textbf{mpi-C1: }Velar stops are realized as palato-alveolar affricates preceding /i/ (Allison 2012).
\end{styleBody}

\begin{styleBody}
\textbf{Morphology}
\end{styleBody}

\begin{styleBody}
(adequate texts unavailable)
\end{styleBody}

\clearpage\begin{styleBody}
\textbf{[mpi] }\ \ \textbf{\textsc{Mpade (Makari dialect)}}\textbf{\ \ }Afro-Asiatic, \textit{Chadic} (Cameroon)
\end{styleBody}

\begin{styleBody}
\textbf{References consulted: }Allison (2012), Mahamat (2005)
\end{styleBody}

\begin{styleBody}
\textbf{Sound inventory}
\end{styleBody}

\begin{styleBody}
\textbf{C phoneme inventory:} /p b t d k [261?] \textsuperscript{m}b \textsuperscript{n}d \textsuperscript{ŋ}[261?] [253?] [257?] k’ t[361?][283?] d[361?][292?] t[361?]s’ t[361?][283?]’ f s z [283?] h m n [27E?] l w j/
\end{styleBody}

\begin{styleBody}
\textbf{N consonant phonemes:} 27
\end{styleBody}

\begin{styleBody}
\textbf{Geminates:} N/A
\end{styleBody}

\begin{styleBody}
\textbf{Voicing contrasts:} Obstruents
\end{styleBody}

\begin{styleBody}
\textbf{Places:} Bilabial, Labiodental, Alveolar, Palato-alveolar, Velar, Glottal
\end{styleBody}

\begin{styleBody}
\textbf{Manners:} Stop, Affricate, Fricative, Nasal, Flap/Tap, Central approximant, Lateral approximant
\end{styleBody}

\begin{styleBody}
\textbf{N elaborations:} 6
\end{styleBody}

\begin{styleBody}
\textbf{Elaborations:} Voiced fricatives/affricates, Prenasalization, Ejective, Implosive, Labiodental, Palato-alveolar
\end{styleBody}

\begin{styleBody}
\textbf{V phoneme inventory:} /i e [268?] a o u/
\end{styleBody}

\begin{styleBody}
\textbf{N vowel qualities:} 6
\end{styleBody}

\begin{styleBody}
\textbf{Diphthongs or vowel sequences:} None
\end{styleBody}

\begin{styleBody}
\textbf{Contrastive length:} None
\end{styleBody}

\begin{styleBody}
\textbf{Contrastive nasalization:} None
\end{styleBody}

\begin{styleBody}
\textbf{Other contrasts:} N/A
\end{styleBody}

\begin{styleBody}
\textbf{Notes:} Mahamat (2005) does not give /\textsuperscript{m}b \textsuperscript{n}d \textsuperscript{ŋ}[261?]/. Allison gives reasoning for differences (2012: 17-20).
\end{styleBody}

\begin{styleBody}
\textbf{Syllable structure}
\end{styleBody}

\begin{styleBody}
\textbf{Complexity Category:} Complex
\end{styleBody}

\begin{styleBody}
\textbf{Canonical syllable structure:} (C)(C)(C)V(C)\textbf{ }(Allison 2012: 23-24)
\end{styleBody}

\begin{styleBody}
\textbf{Size of maximal onset:} 3
\end{styleBody}

\begin{styleBody}
\textbf{Size of maximal coda:} 1
\end{styleBody}

\begin{styleBody}
\textbf{Onset obligatory:} No
\end{styleBody}

\begin{styleBody}
\textbf{Coda obligatory:} No
\end{styleBody}

\begin{styleBody}
\textbf{Vocalic nucleus patterns:} Short vowels
\end{styleBody}

\begin{styleBody}
\textbf{Syllabic consonant patterns:} Nasal
\end{styleBody}

\begin{styleBody}
\textbf{Size of maximal word{}-marginal sequences with syllabic obstruents:} N/A
\end{styleBody}

\begin{styleBody}
\textbf{Predictability of syllabic consonants:} Predictable from word/consonantal context
\end{styleBody}

\begin{styleBody}
\textbf{Morphological constituency of maximal syllable margin:} Morpheme-internal (Onset)
\end{styleBody}

\begin{styleBody}
\textbf{Morphological pattern of syllabic consonants:} Lexical items
\end{styleBody}

\begin{styleBody}
\textbf{Onset restrictions: }Apparently no restrictions on simple onsets. In biconsonantal onsets, the most common pattern is for C\textsubscript{1} to be a stop or fricative, and C\textsubscript{2} to be /[27E?] l w j/. /sk sk’ ft/ onsets also occur. The only triconsonantal onset is /skw/.
\end{styleBody}

\begin{styleBody}
\textbf{Coda restrictions: }Only sonorants /m n l [27E?] w j/ occur.
\end{styleBody}

\begin{styleBody}
\textbf{Suprasegmentals}
\end{styleBody}

\begin{styleBody}
\textbf{Tone:} Yes
\end{styleBody}

\begin{styleBody}
\textbf{Word stress:} Not reported
\end{styleBody}

\begin{styleBody}
\textbf{Vowel reduction processes}
\end{styleBody}

\begin{styleBody}
(none reported)
\end{styleBody}

\begin{styleBody}
\textbf{Consonant allophony processes}
\end{styleBody}

\begin{styleBody}
(none reported)
\end{styleBody}

\begin{styleBody}
\textbf{Morphology}
\end{styleBody}

\begin{styleBody}
(adequate texts unavailable)
\end{styleBody}

\clearpage\begin{styleBody}
\textbf{[mri] }\ \ \textbf{\textsc{Maori}}\textbf{\ \ }Austronesian, \textit{Malayo-Polynesian} (New Zealand)
\end{styleBody}

\begin{styleBody}
\textbf{References consulted: }Bauer (1999)
\end{styleBody}

\begin{styleBody}
\textbf{Sound inventory}
\end{styleBody}

\begin{styleBody}
\textbf{C phoneme inventory:} /p t k [278?] h m n ŋ [27E?] w/
\end{styleBody}

\begin{styleBody}
\textbf{N consonant phonemes:} 10
\end{styleBody}

\begin{styleBody}
\textbf{Geminates:} N/A
\end{styleBody}

\begin{styleBody}
\textbf{Voicing contrasts:} None
\end{styleBody}

\begin{styleBody}
\textbf{Places:} Bilabial, Alveolar, Velar, Glottal
\end{styleBody}

\begin{styleBody}
\textbf{Manners:} Stop, Fricative, Nasal, Flap/tap, Central approximant
\end{styleBody}

\begin{styleBody}
\textbf{N elaborations:} 0
\end{styleBody}

\begin{styleBody}
\textbf{Elaborations:} N/A
\end{styleBody}

\begin{styleBody}
\textbf{V phoneme inventory:} /i [25B?] a [254?] u a[2D0?]/
\end{styleBody}

\begin{styleBody}
\textbf{N vowel qualities:} 5
\end{styleBody}

\begin{styleBody}
\textbf{Diphthongs or vowel sequences:} Vowel sequences /ii [25B?][25B?] [254?][254?] uu a[25B?] ai a[254?] au [254?]a [254?][25B?] [254?]i [254?][254?] [254?]u [25B?]a [25B?]i [25B?][254?] [25B?]u ua u[25B?] ui u[254?] ia i[25B?] i[254?] iu/.
\end{styleBody}

\begin{styleBody}
\textbf{Contrastive length:} Some
\end{styleBody}

\begin{styleBody}
\textbf{Contrastive nasalization:} None
\end{styleBody}

\begin{styleBody}
\textbf{Other contrasts:} N/A
\end{styleBody}

\begin{styleBody}
\textbf{Notes:} /[278?]/ is variable in realization, was likely /f/ in past.
\end{styleBody}

\begin{styleBody}
\textbf{Syllable structure}
\end{styleBody}

\begin{styleBody}
\textbf{Complexity Category:} Simple
\end{styleBody}

\begin{styleBody}
\textbf{Canonical syllable structure:} (C)V\textbf{ }(Bauer 1999: 533-8)
\end{styleBody}

\begin{styleBody}
\textbf{Size of maximal onset:} 1
\end{styleBody}

\begin{styleBody}
\textbf{Size of maximal coda:} N/A
\end{styleBody}

\begin{styleBody}
\textbf{Onset obligatory:} No
\end{styleBody}

\begin{styleBody}
\textbf{Coda obligatory:} N/A
\end{styleBody}

\begin{styleBody}
\textbf{Vocalic nucleus patterns:} Short vowels, Long vowels, Diphthongs
\end{styleBody}

\begin{styleBody}
\textbf{Syllabic consonant patterns:} N/A
\end{styleBody}

\begin{styleBody}
\textbf{Size of maximal word{}-marginal sequences with syllabic obstruents:} N/A
\end{styleBody}

\begin{styleBody}
\textbf{Predictability of syllabic consonants:} N/A
\end{styleBody}

\begin{styleBody}
\textbf{Morphological constituency of maximal syllable margin:} N/A
\end{styleBody}

\begin{styleBody}
\textbf{Morphological pattern of syllabic consonants:} N/A
\end{styleBody}

\begin{styleBody}
\textbf{Onset restrictions: }All consonants occur.
\end{styleBody}

\begin{styleBody}
\textbf{Suprasegmentals}
\end{styleBody}

\begin{styleBody}
\textbf{Tone:} No
\end{styleBody}

\begin{styleBody}
\textbf{Word stress:} Yes
\end{styleBody}

\begin{styleBody}
\textbf{Stress placement:} Morphologically or Lexically Conditioned
\end{styleBody}

\begin{styleBody}
\textbf{Phonetic processes conditioned by stress:} Consonant Allophony in Unstressed Syllables, Consonant Allophony in Stressed Syllables
\end{styleBody}

\begin{styleBody}
\textbf{Differences in phonological properties of stressed and unstressed syllables:} (None)
\end{styleBody}

\begin{styleBody}
\textbf{Phonetic correlates of stress: }Vowel duration (impressionistic), Pitch (impressionistic), Intensity (impressionistic)
\end{styleBody}

\begin{styleBody}
\textbf{Notes: }Intensity is optionally a correlate of stress. Pitch here is a pitch fall. Secondary stress marked only by length.
\end{styleBody}

\begin{styleBody}
\textbf{Vowel reduction processes}
\end{styleBody}

\begin{styleBody}
\textbf{mri-R1:} The final vowel of a word spoken in isolation is frequently devoiced (Bauer 1999: 546).
\end{styleBody}

\begin{styleBody}
\textbf{Consonant allophony processes}
\end{styleBody}

\begin{styleBody}
\textbf{mri-C1: }Stops may vary freely with affricates in stressed syllables (Bauer 1999: 545).
\end{styleBody}

\begin{styleBody}
\textbf{mri-C2: }A voiceless alveolar stop may be affricated preceding an unstressed, devoiced vowel; sometimes this process involves palatalization of the stop too (Bauer 1999).
\end{styleBody}

\begin{styleBody}
\textbf{mri-C3: }In stressed syllables, /w/ may be produced with closer approximation (Bauer 1999: 545).
\end{styleBody}

\begin{styleBody}
\textbf{Morphology}
\end{styleBody}

\begin{styleBody}
(adequate texts unavailable)
\end{styleBody}

\clearpage\begin{styleBody}
\textbf{[nir] }\ \ \textbf{\textsc{Nimboran}}\textbf{\ \ }Nimboranic (Indonesia)
\end{styleBody}

\begin{styleBody}
\textbf{References consulted: }Anceaux (1965), May \& May (1981), May (1997)
\end{styleBody}

\begin{styleBody}
\textbf{Sound inventory}
\end{styleBody}

\begin{styleBody}
\textbf{C phoneme inventory:} /p t k b \textsuperscript{m}b d \textsuperscript{n}d \textsuperscript{ŋ}[261?] s h m n ŋ [26D?] w j/
\end{styleBody}

\begin{styleBody}
\textbf{N consonant phonemes:} 16
\end{styleBody}

\begin{styleBody}
\textbf{Geminates:} N/A
\end{styleBody}

\begin{styleBody}
\textbf{Voicing contrasts:} Obstruents
\end{styleBody}

\begin{styleBody}
\textbf{Places:} Bilabial, Dental, Retroflex, Velar, Glottal
\end{styleBody}

\begin{styleBody}
\textbf{Manners:} Stop, Fricative, Nasal, Central approximant, Lateral flap
\end{styleBody}

\begin{styleBody}
\textbf{N elaborations:} 2
\end{styleBody}

\begin{styleBody}
\textbf{Elaborations:} Prenasalization, Retroflex
\end{styleBody}

\begin{styleBody}
\textbf{V phoneme inventory:} /i e [289?] a o u/
\end{styleBody}

\begin{styleBody}
\textbf{N vowel qualities:} 6
\end{styleBody}

\begin{styleBody}
\textbf{Diphthongs or vowel sequences:} Vowel sequences /ii ee [289?][289?] aa oo uu ai a[289?] ei ao ou/
\end{styleBody}

\begin{styleBody}
\textbf{Contrastive length:} None
\end{styleBody}

\begin{styleBody}
\textbf{Contrastive nasalization:} None
\end{styleBody}

\begin{styleBody}
\textbf{Other contrasts:} N/A
\end{styleBody}

\begin{styleBody}
\textbf{Notes:} /[26D?]/ is used for retroflexed lateral flap. Prenasalized stops given by May with distributional justification. Extrasystematical phonemes /$\beta $/ and /x/ occur in one lexical item each (Anceaux 1965: 9). Anceaux gives /[268?]/ instead of /[289?]/. Long vowels are analyzed as vowel clusters.
\end{styleBody}

\begin{styleBody}
\textbf{Syllable structure}
\end{styleBody}

\begin{styleBody}
\textbf{Complexity Category:} Complex
\end{styleBody}

\begin{styleBody}
\textbf{Canonical syllable structure:} (C)(C)(C)V(C)\textbf{ }(Anceaux 1965: 31-6; May 1997: 12-19; May \& May 1981: 12)
\end{styleBody}

\begin{styleBody}
\textbf{Size of maximal onset:} 3
\end{styleBody}

\begin{styleBody}
\textbf{Size of maximal coda:} 1
\end{styleBody}

\begin{styleBody}
\textbf{Onset obligatory:} No
\end{styleBody}

\begin{styleBody}
\textbf{Coda obligatory: }No
\end{styleBody}

\begin{styleBody}
\textbf{Vocalic nucleus patterns:} Short vowels, Vowel sequences
\end{styleBody}

\begin{styleBody}
\textbf{Syllabic consonant patterns:} Obstruent (Conflicting reports)
\end{styleBody}

\begin{styleBody}
\textbf{Size of maximal word{}-marginal sequences with syllabic obstruents:} N/A
\end{styleBody}

\begin{styleBody}
\textbf{Predictability of syllabic consonants:} N/A
\end{styleBody}

\begin{styleBody}
\textbf{Morphological constituency of maximal syllable margin:} Morpheme-internal (Onset)
\end{styleBody}

\begin{styleBody}
\textbf{Morphological pattern of syllabic consonants:} N/A
\end{styleBody}

\begin{styleBody}
\textbf{Onset restrictions: }All consonants may occur as simple onsets. In biconsonantal onsets, all consonants except for /w j/ may occur as C\textsubscript{1}. If C\textsubscript{1} is nasal or stop, then C\textsubscript{2} is /l j w/. Biconsonantal onsets /nt sp sw sk hm hn/ additionally occur. Triconsonantal onset patterns are limited to /skw skl sk \textsuperscript{ŋ}[261?]lw blw/.
\end{styleBody}

\begin{styleBody}
\textbf{Coda restrictions:} Limited to /m n ŋ p/. For a few speakers, a word-final vowel sequence /ii/ may be realized as [ik].
\end{styleBody}

\begin{styleBody}
\textbf{Notes: }May notes that triconsonantal onsets with /w j/ as third member could be alternatively interpreted as biconsonantal onsets followed by a vowel sequence starting with /u/ or /i/; however he adopts the former analysis due to syllable peak patterns observed in the language (May 1997: 17-18). May \& May note that the initial fricative in /skl/ onsets may be syllabic ([s[329?].kl]), based on speaker reaction to syllable division in words with this cluster (1981: 29); however, reconsideration of the data in May 1997 leads to the triconsonantal analysis.
\end{styleBody}

\begin{styleBody}
\textbf{Suprasegmentals}
\end{styleBody}

\begin{styleBody}
\textbf{Tone:} No
\end{styleBody}

\begin{styleBody}
\textbf{Word stress:} Yes
\end{styleBody}

\begin{styleBody}
\textbf{Stress placement:} Morphologically or Lexically Conditioned
\end{styleBody}

\begin{styleBody}
\textbf{Phonetic processes conditioned by stress:} Vowel Reduction, Consonant Allophony in Unstressed Syllables
\end{styleBody}

\begin{styleBody}
\textbf{Differences in phonological properties of stressed and unstressed syllables:} Not described
\end{styleBody}

\begin{styleBody}
\textbf{Phonetic correlates of stress: }Pitch (impressionistic), Intensity (impressionistic)
\end{styleBody}

\begin{styleBody}
\textbf{Notes: }Vowel quality correlate for /[289?]/, /a/ may vary in very complex combinations of word, syllable, stress, and vowel contexts.
\end{styleBody}

\begin{styleBody}
\textbf{Vowel reduction processes}
\end{styleBody}

\begin{styleBody}
\textbf{nir-R1:} High front vowel /i/ is lowered when unaccented and preceding a word-final /ŋ/ (Anceaux 1965: 10).
\end{styleBody}

\begin{styleBody}
\textbf{nir-R2:} The low central vowel /a/ is in free variation with a higher variant if it precedes a consonant and an accented vowel (Anceaux 1965: 13).
\end{styleBody}

\begin{styleBody}
\textbf{nir-R3: }Mid back rounded vowel /o/ is realized as higher and unrounded when occurring word-finally and without accent (Anceaux 1965: 14).
\end{styleBody}

\begin{styleBody}
\textbf{Consonant allophony processes}
\end{styleBody}

\begin{styleBody}
\textbf{nir-C1: }Sequences of alveolar stops, fricative, and nasal and /i/ vary with palatalized variants of the consonants when /i/ is unstressed (May \& May 1981: 18).
\end{styleBody}

\begin{styleBody}
\textbf{nir-C2: }A voiceless bilabial stop is voiced preceding a voiced consonant (May 1997: 30).
\end{styleBody}

\begin{styleBody}
\textbf{nir-C3: }A voiceless bilabial stop varies with a fricative syllable-initially (May \& May 1981: 16).
\end{styleBody}

\begin{styleBody}
\textbf{nir-C4: }A voiceless bilabial stop is spirantized intervocalically (May \& May 1981: 16).
\end{styleBody}

\begin{styleBody}
\textbf{Morphology}
\end{styleBody}

\begin{styleBody}
\textbf{Text:} “Sample text” (May 1997: 172-177)
\end{styleBody}

\begin{styleBody}
\textbf{Synthetic index: }1.7 morphemes/word (334 morphemes, 198 words)
\end{styleBody}

\clearpage\begin{styleBody}
\textbf{[niv] }\ \ \textbf{\textsc{Nivkh (West Sakhalin dialect)}}\textbf{\ \ }isolate (Russia)
\end{styleBody}

\begin{styleBody}
\textbf{References consulted: }Gruzdeva (1998), Kreinovich (1979), Shiraishi (2006)
\end{styleBody}

\begin{styleBody}
\textbf{Sound inventory}
\end{styleBody}

\begin{styleBody}
\textbf{C phoneme inventory:} /p p[2B0?] t t[2B0?] c c[2B0?] k k[2B0?] q q[2B0?] [278?] $\beta $ s z x [263?] $\chi $ [281?] h m n [272?] ŋ l r[325?] r w j/
\end{styleBody}

\begin{styleBody}
\textbf{N consonant phonemes:} 28
\end{styleBody}

\begin{styleBody}
\textbf{Geminates:} N/A
\end{styleBody}

\begin{styleBody}
\textbf{Voicing contrasts:} Obstruents, Sonorants
\end{styleBody}

\begin{styleBody}
\textbf{Places:} Bilabial, Dental, Palatal, Velar, Uvular, Glottal
\end{styleBody}

\begin{styleBody}
\textbf{Manners:} Stop, Fricative, Nasal, Trill, Central approximant, Lateral approximant
\end{styleBody}

\begin{styleBody}
\textbf{N elaborations:} 4
\end{styleBody}

\begin{styleBody}
\textbf{Elaborations:} Voiced fricatives/affricates, Devoiced sonorants, Post-aspiration, Uvular
\end{styleBody}

\begin{styleBody}
\textbf{V phoneme inventory:} /i e [268?] a o u/
\end{styleBody}

\begin{styleBody}
\textbf{N vowel qualities:} 6
\end{styleBody}

\begin{styleBody}
\textbf{Diphthongs or vowel sequences: }None
\end{styleBody}

\begin{styleBody}
\textbf{Contrastive length:} None
\end{styleBody}

\begin{styleBody}
\textbf{Contrastive nasalization:} None
\end{styleBody}

\begin{styleBody}
\textbf{Other contrasts:} N/A
\end{styleBody}

\begin{styleBody}
\textbf{Notes:} The uvular/velar distinction is ‘nearly allophonic’. Gruzdeva posits a 3-way stop contrast between voiced, voiceless, and aspirated stops. I take Shiraishi’s analysis here.
\end{styleBody}

\begin{styleBody}
\textbf{Syllable structure}
\end{styleBody}

\begin{styleBody}
\textbf{Complexity Category:} Complex
\end{styleBody}

\begin{styleBody}
\textbf{Canonical syllable structure:} (C)(C)V(C)(C)(C)\textbf{ }(Shiraishi 2006: 29-30)
\end{styleBody}

\begin{styleBody}
\textbf{Size of maximal onset:} 2
\end{styleBody}

\begin{styleBody}
\textbf{Size of maximal coda:} 3
\end{styleBody}

\begin{styleBody}
\textbf{Onset obligatory:} No
\end{styleBody}

\begin{styleBody}
\textbf{Coda obligatory:} No
\end{styleBody}

\begin{styleBody}
\textbf{Vocalic nucleus patterns:} Short vowels
\end{styleBody}

\begin{styleBody}
\textbf{Syllabic consonant patterns:} N/A
\end{styleBody}

\begin{styleBody}
\textbf{Size of maximal word{}-marginal sequences with syllabic obstruents:} N/A
\end{styleBody}

\begin{styleBody}
\textbf{Predictability of syllabic consonants:} N/A
\end{styleBody}

\begin{styleBody}
\textbf{Morphological constituency of maximal syllable margin:} Both patterns (Onset, Coda)
\end{styleBody}

\begin{styleBody}
\textbf{Morphological pattern of syllabic consonants:} N/A
\end{styleBody}

\begin{styleBody}
\textbf{Onset restrictions: }Apparently no restrictions on simple onsets. Biconsonantal onsets may not have plosive or /j/ as C\textsubscript{2}.
\end{styleBody}

\begin{styleBody}
\textbf{Coda restrictions: }Simple codas apparently unrestricted. Biconsonantal codas include /sk [263?]s wk [272?][278?]/. Examples of triconsonantal codas include /ntq/ and /nt$\chi $/.
\end{styleBody}

\begin{styleBody}
\textbf{Notes: }Gruzdeva (1998, for Amur and E. Sakhalin dialects) also lists /lms/, /lmr/, and /vdr/ codas.
\end{styleBody}

\begin{styleBody}
\textbf{Suprasegmentals}
\end{styleBody}

\begin{styleBody}
\textbf{Tone: }No
\end{styleBody}

\begin{styleBody}
\textbf{Word stress:} Yes
\end{styleBody}

\begin{styleBody}
\textbf{Stress placement:} Fixed
\end{styleBody}

\begin{styleBody}
\textbf{Phonetic processes conditioned by stress:} Consonant Allophony in Stressed Syllables
\end{styleBody}

\begin{styleBody}
\textbf{Differences in phonological properties of stressed and unstressed syllables:} (None)
\end{styleBody}

\begin{styleBody}
\textbf{Phonetic correlates of stress: }Pitch (impressionistic)
\end{styleBody}

\begin{styleBody}
\textbf{Vowel reduction processes}
\end{styleBody}

\begin{styleBody}
(none reported)
\end{styleBody}

\begin{styleBody}
\textbf{Notes:} in related Amur dialect, stress shift from 2nd to 1st syllable contributed to loss and reduction of vowels and distinct phonological character of this dialect.
\end{styleBody}

\begin{styleBody}
\textbf{Consonant allophony processes}
\end{styleBody}

\begin{styleBody}
\textbf{niv-C1: }A voiceless alveolar trill may be produced with palato-alveolar fricative release or vary with a voiceless palato-alveolar fricative (Shiraishi 2006: 26).
\end{styleBody}

\begin{styleBody}
\textbf{niv-C2: }Consonants become palatalized preceding front vowels, especially when stressed (Shiraishi 2006: 23).
\end{styleBody}

\begin{styleBody}
\textbf{niv-C3: }Non-aspirated plosives are realized as voiced following sonorants (Shiraishi 2006: 25).
\end{styleBody}

\begin{styleBody}
\textbf{Morphology}
\end{styleBody}

\begin{styleBody}
\textbf{Text:} “A frog and a rat” (Gruzdeva 1998: 58-61)
\end{styleBody}

\begin{styleBody}
\textbf{Synthetic index: }1.7 morphemes/word (408 morphemes, 240 words)
\end{styleBody}

\clearpage\begin{styleBody}
\textbf{[nsm] }\ \ \textbf{\textsc{Sumi Naga}}\textbf{\ \ }Sino-Tibetan, \textit{Kuki-Chin-Naga} (India)
\end{styleBody}

\begin{styleBody}
\textbf{References consulted: }Teo (2009), Teo (2012)
\end{styleBody}

\begin{styleBody}
\textbf{Sound inventory}
\end{styleBody}

\begin{styleBody}
\textbf{C phoneme inventory:} /p p[2B0?] b t t[2B0?] d k k[2B0?] [261?] q q[2B0?] t[361?][283?] t[361?][283?][2B0?] f v [283?] [292?] x [263?] h m m\textsuperscript{[266?]} n n\textsuperscript{[266?]} ŋ l l\textsuperscript{[266?]} j/
\end{styleBody}

\begin{styleBody}
\textbf{N consonant phonemes:} 28
\end{styleBody}

\begin{styleBody}
\textbf{Geminates:} N/A
\end{styleBody}

\begin{styleBody}
\textbf{Voicing contrasts: }Obstruents, Sonorants
\end{styleBody}

\begin{styleBody}
\textbf{Places: }Bilabial, Labiodental, Alveolar, Palato-alveolar, Velar, Uvular, Glottal
\end{styleBody}

\begin{styleBody}
\textbf{Manners:} Stop, Affricate, Fricative, Nasal, Central approximant, Lateral approximant
\end{styleBody}

\begin{styleBody}
\textbf{N elaborations:} 6
\end{styleBody}

\begin{styleBody}
\textbf{Elaborations:} Breathy voice, Voiced fricatives/affricates, Post-aspiration, Labiodental, Palato-alveolar, Uvular
\end{styleBody}

\begin{styleBody}
\textbf{V phoneme inventory:} /i e [268?] a o u/
\end{styleBody}

\begin{styleBody}
\textbf{N vowel qualities:} 6
\end{styleBody}

\begin{styleBody}
\textbf{Diphthongs or vowel sequences:} Vowel sequences /iu ia ua uo oo ai oi/
\end{styleBody}

\begin{styleBody}
\textbf{Contrastive length:} None
\end{styleBody}

\begin{styleBody}
\textbf{Contrastive nasalization:} None
\end{styleBody}

\begin{styleBody}
\textbf{Other contrasts:} N/A
\end{styleBody}

\begin{styleBody}
\textbf{Notes:} Sreedhar (1980) has [[283?] [292?]] not as contrastive but as allophones of /s z/. /[279?]/, which occurs in some recent loans, is argued by Teo to be nativized but marginal (2009: 36, 2012: 366). The language does not have phonologically contrastive vowel length, but phonetic long vowels and diphthongs result from phonological vowel sequences arising through morphological concatenation (Teo 2009: 58-9). In at least some of these cases (perhaps all in the case of long vowels?), the long vowel/diphthong is one variant, where another might have an intervening glottal stop. In some cases long vowels may additionally occur through the variable deletion of an intervocalic glide. The list of vowel sequences given by the author does not distinguish the invariant forms from the variable forms, and may not be an exhaustive list for the language.
\end{styleBody}

\begin{styleBody}
\textbf{Syllable structure}
\end{styleBody}

\begin{styleBody}
\textbf{Complexity Category:} Simple
\end{styleBody}

\begin{styleBody}
\textbf{Canonical syllable structure:} (C)V (Teo 2009: 57-64)
\end{styleBody}

\begin{styleBody}
\textbf{Size of maximal onset:} 1
\end{styleBody}

\begin{styleBody}
\textbf{Size of maximal coda: }N/A
\end{styleBody}

\begin{styleBody}
\textbf{Onset obligatory:} No
\end{styleBody}

\begin{styleBody}
\textbf{Coda obligatory:} N/A
\end{styleBody}

\begin{styleBody}
\textbf{Vocalic nucleus patterns: }Short vowels, Diphthongs
\end{styleBody}

\begin{styleBody}
\textbf{Syllabic consonant patterns:} Nasal
\end{styleBody}

\begin{styleBody}
\textbf{Size of maximal word{}-marginal sequences with syllabic obstruents:} N/A
\end{styleBody}

\begin{styleBody}
\textbf{Predictability of syllabic consonants:} Predictable from word/consonantal context
\end{styleBody}

\begin{styleBody}
\textbf{Morphological constituency of maximal syllable margin:} N/A
\end{styleBody}

\begin{styleBody}
\textbf{Morphological pattern of syllabic consonants:} Lexical items.
\end{styleBody}

\begin{styleBody}
\textbf{Onset restrictions:} None in main syllables, only /p t k m/ occur in sesquisyllables.
\end{styleBody}

\begin{styleBody}
\textbf{Coda restrictions:} N/A.
\end{styleBody}

\begin{styleBody}
\textbf{Notes: }CC onsets occur in variation with sesquisyllabic cvC sequences. Simple codas occur in natural speech when a prefix precedes a sesquisyllable; this is likely the result of recent/ongoing vowel deletion (Teo 2009: 62-4). Diphthongs are often the result of morphological concatenation.
\end{styleBody}

\begin{styleBody}
\textbf{Suprasegmentals}
\end{styleBody}

\begin{styleBody}
\textbf{Tone:} Yes
\end{styleBody}

\begin{styleBody}
\textbf{Word stress: }No
\end{styleBody}

\begin{styleBody}
\textbf{Notes: }Some parts of speech (monomorphemic verbs, numerals in isolation, some noun roots) have sesquisyllabic patterns \ (minor syllable with restricted set of consonants, vowels, and tones followed by a full syllable). However, this pattern is not pervasive throughout the language (other disyllabic verb and noun roots do not show sesquisyllabic patterns). Teo notes that minor syllables in sesquisyllabic structures could be argued to receive less prominence than full syllables, but that stress is “not phonemic” (2012: 371-372).
\end{styleBody}

\begin{styleBody}
\textbf{Vowel reduction processes}
\end{styleBody}

\begin{styleBody}
\textbf{nsm-R1: }Mid vowels /e o/ have free variants [[25B?] [254?]] (Teo 2009: 45-46, Teo 2012: 369).
\end{styleBody}

\begin{styleBody}
\textbf{nsm-R2: }High central vowel /[268?]/ is sometimes realized as [[259?]] word-medially (Teo 2009: 45).
\end{styleBody}

\begin{styleBody}
\textbf{nsm-R3: }Word-medial and word-final high vowels /i [268?] u/ are prone to deletion following a fricative or /q[2B0?]/ (Teo 2009: 66).
\end{styleBody}

\begin{styleBody}
\textbf{nsm-R4: }Word-final high vowels are prone to deletion following a nasal (Teo 2012: 369).
\end{styleBody}

\begin{styleBody}
\textbf{nsm-R5:} Vowels in minor syllables may be altogether deleted between a stop and a lateral approximant (Teo 2012: 370).
\end{styleBody}

\begin{styleBody}
\textbf{Consonant allophony processes}
\end{styleBody}

\begin{styleBody}
\textbf{nsm-C1:} An aspirated voiceless uvular stop /q[2B0?]/ often has an affricated release [q\textsuperscript{$\chi $}] (Teo 2009: 39).
\end{styleBody}

\begin{styleBody}
\textbf{nsm-C2:} Labiodental fricative /v/ is realized as approximant [w] preceding back vowels /u o/ (Teo 2009: 39).
\end{styleBody}

\begin{styleBody}
\textbf{nsm-C3:} Labiodental fricative, alveolar nasals, and lateral approximant /v n n\textsuperscript{[266?] }l\textsuperscript{[266?]}/ are realized as palatalized preceding front vowels /i e/ (Teo 2009: 40, 42).
\end{styleBody}

\begin{styleBody}
\textbf{nsm-C4:} The voiced palato-alveolar fricative has a free affricated variant [d[361?][292?]] (Teo 2009: 40).
\end{styleBody}

\begin{styleBody}
\textbf{nsm-C5:} Velar obstruents are realized as palatal preceding front vowels /i e/ (Teo 2012: 368).
\end{styleBody}

\begin{styleBody}
\textbf{Morphology}
\end{styleBody}

\begin{styleBody}
(adequate texts unavailable)
\end{styleBody}

\clearpage\begin{styleBody}
\textbf{[nuk] }\ \ \textbf{\textsc{Nuu-chah-nulth}}\textbf{\ \ }Wakashan, \textit{Southern Wakashan} (Canada)
\end{styleBody}

\begin{styleBody}
\textbf{References consulted: }Carlson et al. (2001), Kim (2003), Rose (1981), Stonham (1999)
\end{styleBody}

\begin{styleBody}
\textbf{Sound inventory}
\end{styleBody}

\begin{styleBody}
\textbf{C phoneme inventory:} /p t k k[2B7?] q q[2B7?] [295?] [294?] p’ t’ k’ k’[2B7?] t[361?]s t[361?][283?] t[361?][26C?] t[361?]s’ t[361?][283?]’ t[361?][26C?]’ s [26C?] [283?] x x[2B7?] $\chi $ $\chi [2B7?]$ [127?] h m n m’ n’ j w j’ w’/
\end{styleBody}

\begin{styleBody}
\textbf{N consonant phonemes:} 35
\end{styleBody}

\begin{styleBody}
\textbf{Geminates:} N/A
\end{styleBody}

\begin{styleBody}
\textbf{Voicing contrasts:} None
\end{styleBody}

\begin{styleBody}
\textbf{Places:} Bilabial, Alveolar, Palato-alveolar, Velar, Uvular, Pharyngeal, Glottal
\end{styleBody}

\begin{styleBody}
\textbf{Manners:} Stop, Affricate, Fricative, Nasal, Central approximant, Lateral affricate, Lateral fricative
\end{styleBody}

\begin{styleBody}
\textbf{N elaborations:} 7
\end{styleBody}

\begin{styleBody}
\textbf{Elaborations:} Creaky voice, Lateral release, Ejective, Palato-alveolar, Uvular, Pharyngeal, Labialization
\end{styleBody}

\begin{styleBody}
\textbf{V phoneme inventory:} /i a u i[2D0?] a[2D0?] u[2D0?]/
\end{styleBody}

\begin{styleBody}
\textbf{N vowel qualities:} 3
\end{styleBody}

\begin{styleBody}
\textbf{Diphthongs or vowel sequences:} None
\end{styleBody}

\begin{styleBody}
\textbf{Contrastive length:} All
\end{styleBody}

\begin{styleBody}
\textbf{Contrastive nasalization:} None
\end{styleBody}

\begin{styleBody}
\textbf{Other contrasts:} N/A
\end{styleBody}

\begin{styleBody}
\textbf{Notes:} Stonham reports 39 consonants, including /q’ q’[2B7?] [127?][2B7?] [26C?][2B7?]/. Davidson has /q’ q’[2B7?]/ but Kim shows these have merged with /[295?]/ in present language. /o e/ appear phonemically only in loanwords, vocative constructions, and expressions for speech act.
\end{styleBody}

\begin{styleBody}
\textbf{Syllable structure}
\end{styleBody}

\begin{styleBody}
\textbf{Complexity Category:} Highly Complex
\end{styleBody}

\begin{styleBody}
\textbf{Canonical syllable structure:} CV(C)(C)(C)(C)\textbf{ }(Kim 2003: 161-6; Stonham 1999: 47-55)
\end{styleBody}

\begin{styleBody}
\textbf{Size of maximal onset:} 1
\end{styleBody}

\begin{styleBody}
\textbf{Size of maximal coda:} 4
\end{styleBody}

\begin{styleBody}
\textbf{Onset obligatory:} Yes
\end{styleBody}

\begin{styleBody}
\textbf{Coda obligatory:} No
\end{styleBody}

\begin{styleBody}
\textbf{Vocalic nucleus patterns:} Short vowels, Long vowels
\end{styleBody}

\begin{styleBody}
\textbf{Syllabic consonant patterns:} N/A
\end{styleBody}

\begin{styleBody}
\textbf{Size of maximal word{}-marginal sequences with syllabic obstruents:} N/A
\end{styleBody}

\begin{styleBody}
\textbf{Predictability of syllabic consonants:} N/A
\end{styleBody}

\begin{styleBody}
\textbf{Morphological constituency of maximal syllable margin:} Morphologically Complex (Coda)
\end{styleBody}

\begin{styleBody}
\textbf{Morphological pattern of syllabic consonants:} N/A
\end{styleBody}

\begin{styleBody}
\textbf{Onset restrictions: }All consonants may occur as simple onsets.
\end{styleBody}

\begin{styleBody}
\textbf{Coda restrictions: }Glottal(ized) and pharyngeal consonants do not occur as simple codas. Biconsonantal codas include /t[361?]sk, ks, tq, mt[361?]s/. Triconsonantal codas include /t[361?]s[283?]tq t[127?]t[361?]s m$\chi $s p[26C?]t[361?]s qt[361?][26C?]s/. Four-consonant codas are rare; C\textsubscript{1} must be a nasal, or the sequence /q[127?]/ must occur: /mtq[283?] [127?]sq[127?] nkq[127?] t[127?]q[127?]/. Sonorants do not follow obstruents in coda clusters, but there seem to be few manner/place restrictions on obstruent sequences.
\end{styleBody}

\begin{styleBody}
\textbf{Notes: }Kim and Stonham both report canonical CV(C)(C)(C) structure, but Stonham lists a few cases of 4-consonant codas (1999: 48). Sequences of identical consonants occur only across morpheme boundaries.
\end{styleBody}

\begin{styleBody}
\textbf{Suprasegmentals}
\end{styleBody}

\begin{styleBody}
\textbf{Tone:} No
\end{styleBody}

\begin{styleBody}
\textbf{Word stress:} Yes
\end{styleBody}

\begin{styleBody}
\textbf{Stress placement:} Weight-Sensitive
\end{styleBody}

\begin{styleBody}
\textbf{Phonetic processes conditioned by stress:} Vowel Reduction
\end{styleBody}

\begin{styleBody}
\textbf{Differences in phonological properties of stressed and unstressed syllables:} (None)
\end{styleBody}

\begin{styleBody}
\textbf{Phonetic correlates of stress: }Vowel duration (impressionistic), Intensity (impressionistic)
\end{styleBody}

\begin{styleBody}
\textbf{Notes: }Pitch may vary independently of the correlates of stress.
\end{styleBody}

\begin{styleBody}
\textbf{Vowel reduction processes}
\end{styleBody}

\begin{styleBody}
\textbf{nuk-R1:} Word-final short vowels are deleted (Rose 1981: 25).
\end{styleBody}

\begin{styleBody}
\textbf{nuk-R2:} Preceding a word-final coda, the rightmost vowel may be deleted if it is in a third or later syllable, is not obligatorily long, and is not already flanked by consonant clusters. If the rightmost vowel does not fit these conditions, then the rightmost vowel which is capable of deleting will do so, given that it is in third or later syllable (Rose 1981: 25).
\end{styleBody}

\begin{styleBody}
\textbf{nuk-R3}: A vowel two syllables leftward of a deleted vowel is optionally deleted, if it is not in an inflectional suffix and is in a third or later syllable of the word (Rose 1981: 25).
\end{styleBody}

\begin{styleBody}
\textbf{nuk-R4:} Word-final long vowels are shortened (Rose 1981: 27).
\end{styleBody}

\begin{styleBody}
\textbf{Notes: }Interaction of processes in noo-R1, noo-R2, and noo-R3 may produce long consonant sequences, but only when fricatives are present between any occurring stops (Rose 1981: 26).
\end{styleBody}

\begin{styleBody}
\textbf{Consonant allophony processes}
\end{styleBody}

\begin{styleBody}
\textbf{nuk-C1: }A consonant is labialized following /u/ and preceding another vowel (Stonham 1999: 27).
\end{styleBody}

\begin{styleBody}
\textbf{Morphology}
\end{styleBody}

\begin{styleBody}
\textbf{Text:} “What mosquitoes are made of” (Stonham 1999: 133-143)
\end{styleBody}

\begin{styleBody}
\textbf{Synthetic index: }2.6 morphemes/word (545 morphemes, 212 words)
\end{styleBody}

\clearpage\begin{styleBody}
\textbf{[ood] }\ \ \textbf{\textsc{Tohono O’odham}}\textbf{\ \ }Uto-Aztecan, \textit{Southern Uto-Aztecan} (Mexico, United States)
\end{styleBody}

\begin{styleBody}
\textbf{References consulted: }Dolores \& Mathiot (1991), Fitzgerald (1994), Hale (1959), Hill \& Zepeda (1992), Saxton (1963, 1982), Albert Alvarez Gonzalez (p.c.)
\end{styleBody}

\begin{styleBody}
\textbf{Sound inventory}
\end{styleBody}

\begin{styleBody}
\textbf{C phoneme inventory:} /p b t[32A?] d[32A?] [256?] k [261?] [294?] t[361?][283?] d[361?][292?] s[32A?] [282?] h m n[32A?] [272?] ŋ [26D?] $\beta [31E?]$ j/
\end{styleBody}

\begin{styleBody}
\textbf{N consonant phonemes:} 20
\end{styleBody}

\begin{styleBody}
\textbf{Geminates:} N/A
\end{styleBody}

\begin{styleBody}
\textbf{Voicing contrasts:} Obstruents
\end{styleBody}

\begin{styleBody}
\textbf{Places:} Bilabial, Dental, Retroflex, Palatal, Velar, Glottal
\end{styleBody}

\begin{styleBody}
\textbf{Manners:} Stop, Affricate, Fricative, Nasal, Central approximant, Lateral flap
\end{styleBody}

\begin{styleBody}
\textbf{N elaborations:} 3
\end{styleBody}

\begin{styleBody}
\textbf{Elaborations:} Voiced fricatives/affricates, Palato-alveolar, Retroflex
\end{styleBody}

\begin{styleBody}
\textbf{V phoneme inventory:} /i [268?] a o u i[2D0?] [268?][2D0?] a[2D0?] o[2D0?] u[2D0?] i[325?]/
\end{styleBody}

\begin{styleBody}
\textbf{N vowel qualities:} 5
\end{styleBody}

\begin{styleBody}
\textbf{Diphthongs or vowel sequences:} Diphthongs / i[268?] iu io ia [268?]i [268?]u [268?]o [268?]a ui u[268?] uo ua oi o[268?] ou oa ai a[268?] au oa/
\end{styleBody}

\begin{styleBody}
\textbf{Contrastive length:} All
\end{styleBody}

\begin{styleBody}
\textbf{Contrastive nasalization:} None
\end{styleBody}

\begin{styleBody}
\textbf{Other contrasts:} Voicing (Some)
\end{styleBody}

\begin{styleBody}
\textbf{Notes:} /[26D?]/ is a retroflex lateral flap. /o/ is often realized as [[254?]]. ‘Extra-short’ (voiceless) vowels sometimes represented in the orthography with a breve \ \u{ }, but it is unclear whether these are the same vowels predicted by the rules below, or other vowels altogether. Dolores \& Mathiot (1991: 236) state that /i[325?]/ is phonemic.
\end{styleBody}

\begin{styleBody}
\textbf{Syllable structure}
\end{styleBody}

\begin{styleBody}
\textbf{Complexity Category:} Highly Complex
\end{styleBody}

\begin{styleBody}
\textbf{Canonical syllable structure:} C(C)(C)(C)V(C)(C)(C)(C)\textbf{ }(Saxton 1982: 100-102, Hale 1959: 24-30, Hill \& Zepeda 1992)
\end{styleBody}

\begin{styleBody}
\textbf{Size of maximal onset:} 4
\end{styleBody}

\begin{styleBody}
\textbf{Size of maximal coda:} 4
\end{styleBody}

\begin{styleBody}
\textbf{Onset obligatory:} Yes
\end{styleBody}

\begin{styleBody}
\textbf{Coda obligatory:} No
\end{styleBody}

\begin{styleBody}
\textbf{Vocalic nucleus patterns:} Short vowels, Long vowels, Vowel sequences
\end{styleBody}

\begin{styleBody}
\textbf{Syllabic consonant patterns:} Obstruents
\end{styleBody}

\begin{styleBody}
\textbf{Size of maximal word{}-marginal sequences with syllabic obstruents:} N/A (grammatical particles are independent, not phonologically bound to adjacent word)
\end{styleBody}

\begin{styleBody}
\textbf{Predictability of syllabic consonants:} Unpredictable
\end{styleBody}

\begin{styleBody}
\textbf{Morphological constituency of maximal syllable margin:} Morphologically Complex (Onset, Coda)
\end{styleBody}

\begin{styleBody}
\textbf{Morphological pattern of syllabic consonants:} Grammatical
\end{styleBody}

\begin{styleBody}
\textbf{Onset restrictions: }All consonants may occur as simple onsets. Biconsonantal onsets are varied and governed by several complex patterns, but include all stop+spirant, spirant+unvoiced stop, nasal+homorganic nonnasal sequences, in addition to others, e.g. /[282?]k st[361?][283?] bp mp dt d[361?][292?]t[361?][283?] kk n[261?]/. Triconsonantal onsets include complex combination of biconsonantal patterns such as /s[272?]k/. 4-consonant onsets also include complex combination of biconsonantal patterns, include /nd[282?][294?]/.
\end{styleBody}

\begin{styleBody}
\textbf{Coda restrictions: }Biconsonantal codas include /[261?]s dk ms/. Triconsonantal codas include /kpn [261?][282?]p tpk bst[361?][283?]/. 4-consonant codas include /[283?]t[361?][283?]kt[361?][283?] t[361?][283?]spk/.
\end{styleBody}

\begin{styleBody}
\textbf{Notes: }Saxton gives maximal onset of three consonants; however, Hale gives example of 4-consonant onset. Hale gives specific rules for consonant combinations, but these are difficult to interpret and include medial clusters. Description of phonetic characteristics is for clusters, not vowels that undergo predictable devoicing in certain environments. “Except in the case of a few words that drop an initial /v/ or /h/ […] there are no words that possess an initial vowel in O’odham” (Dolores \& Mathiot 1991: 238).
\end{styleBody}

\begin{styleBody}
\textbf{Suprasegmentals}
\end{styleBody}

\begin{styleBody}
\textbf{Tone:} No
\end{styleBody}

\begin{styleBody}
\textbf{Word stress:} Yes
\end{styleBody}

\begin{styleBody}
\textbf{Stress placement:} Fixed
\end{styleBody}

\begin{styleBody}
\textbf{Phonetic processes conditioned by stress:} Vowel Reduction, Consonant Allophony in Stressed Syllables
\end{styleBody}

\begin{styleBody}
\textbf{Differences in phonological properties of stressed and unstressed syllables:} Vowel Quality Contrasts, Vowel Length Contrasts
\end{styleBody}

\begin{styleBody}
\textbf{Phonetic correlates of stress: }Vowel duration (impressionistic), Pitch (impressionistic), Intensity (impressionistic)
\end{styleBody}

\begin{styleBody}
\textbf{Notes: }Dolores \& Mathiot (1991) report that there is no stress, but others report it does occur.
\end{styleBody}

\begin{styleBody}
\textbf{Vowel reduction processes}
\end{styleBody}

\begin{styleBody}
\textbf{ood-R1:} Word-final unstressed short vowels /i [268?]/ are devoiced when occurring between a non-laryngeal consonant and a pause (Saxton 1963: 31).
\end{styleBody}

\begin{styleBody}
\textbf{ood-R2:} Unstressed short vowels /i [268?]/ are devoiced when occurring between a stop and a voiceless stop (Saxton 1963: 31).
\end{styleBody}

\begin{styleBody}
\textbf{ood-R3:} Word-final unstressed short vowels /a o u/ are devoiced following a stressed vowel (Saxton 1963: 31).
\end{styleBody}

\begin{styleBody}
\textbf{ood-R4:} Vowels have a voiceless offglide when preceding voiceless or devoiced consonants (Saxton 1963: 31).
\end{styleBody}

\begin{styleBody}
\textbf{ood-R5:} Vowels optionally have a voiceless offglide preceding a pause (Saxton 1963: 31).
\end{styleBody}

\begin{styleBody}
\textbf{ood-R6:} An unstressed vowel is deleted when flanked by consonants that form a permitted consonant cluster (Saxton 1982: 103).
\end{styleBody}

\begin{styleBody}
\textbf{ood-R7:} Unstressed vowels are reduced to [[259?]], except for noncentral vowels following consonants that are not /t t[361?][283?]/ (Saxton 1982: 104).
\end{styleBody}

\begin{styleBody}
\textbf{Consonant allophony processes}
\end{styleBody}

\begin{styleBody}
\textbf{ood-C1:} Labial glide /$\beta [31E?]$/ is realized as a fricative [$\beta $] or [[278?]] before /i/ or /a/ (Saxton 1963: 31).
\end{styleBody}

\begin{styleBody}
\textbf{Morphology}
\end{styleBody}

\begin{styleBody}
\textbf{Text: }“The coyote and the jackrabbit” (Saxton 1982: 263-266)
\end{styleBody}

\begin{styleBody}
\textbf{Synthetic index: }1.4 morphemes/word (353 morphemes, 250 words)
\end{styleBody}

\clearpage\begin{styleBody}
\textbf{[opm] }\ \ \textbf{\textsc{Oksapmin}}\textbf{\ \ }Nuclear Trans New Guinea, \textit{Asman-Awyu-Ok} (Papua New Guinea)
\end{styleBody}

\begin{styleBody}
\textbf{References consulted: }Loughnane (2009)
\end{styleBody}

\begin{styleBody}
\textbf{Sound inventory}
\end{styleBody}

\begin{styleBody}
\textbf{C phoneme inventory:} /t k k[2B7?] \textsuperscript{m}b \textsuperscript{n}d \textsuperscript{ŋ}[261?] \textsuperscript{ŋ}[261?][2B7?] [278?] s x x[2B7?] m n l w j/
\end{styleBody}

\begin{styleBody}
\textbf{N consonant phonemes:} 16
\end{styleBody}

\begin{styleBody}
\textbf{Geminates:} N/A
\end{styleBody}

\begin{styleBody}
\textbf{Voicing contrasts:} Obstruent
\end{styleBody}

\begin{styleBody}
\textbf{Places:} Bilabial, Alveolar, Velar
\end{styleBody}

\begin{styleBody}
\textbf{Manners:} Stop, Fricative, Nasal, Central approximant, Lateral approximant
\end{styleBody}

\begin{styleBody}
\textbf{N elaborations:} 2
\end{styleBody}

\begin{styleBody}
\textbf{Elaborations:} Prenasalization, Labialization
\end{styleBody}

\begin{styleBody}
\textbf{V phoneme inventory:} /i e [259?] a o u/
\end{styleBody}

\begin{styleBody}
\textbf{N vowel qualities:} 6
\end{styleBody}

\begin{styleBody}
\textbf{Diphthongs or vowel sequences:} None
\end{styleBody}

\begin{styleBody}
\textbf{Contrastive length:} None
\end{styleBody}

\begin{styleBody}
\textbf{Contrastive nasalization:} None
\end{styleBody}

\begin{styleBody}
\textbf{Other contrasts:} N/A
\end{styleBody}

\begin{styleBody}
\textbf{Notes:} The Lawrences also propose /[259?]i [28A?]/ for vowel inventory.
\end{styleBody}

\begin{styleBody}
\textbf{Syllable structure}
\end{styleBody}

\begin{styleBody}
\textbf{Complexity Category:} Complex
\end{styleBody}

\begin{styleBody}
\textbf{Canonical syllable structure:} (C)(C)V(C) (Loughnane 2009: 63-73)
\end{styleBody}

\begin{styleBody}
\textbf{Size of maximal onset:} 2
\end{styleBody}

\begin{styleBody}
\textbf{Size of maximal coda:} 1
\end{styleBody}

\begin{styleBody}
\textbf{Onset obligatory:} Yes
\end{styleBody}

\begin{styleBody}
\textbf{Coda obligatory:} No
\end{styleBody}

\begin{styleBody}
\textbf{Vocalic nucleus patterns:} Short vowels
\end{styleBody}

\begin{styleBody}
\textbf{Syllabic consonant patterns:} Nasal
\end{styleBody}

\begin{styleBody}
\textbf{Size of maximal word{}-marginal sequences with syllabic obstruents:} N/A
\end{styleBody}

\begin{styleBody}
\textbf{Predictability of syllabic consonants:} Varies with VC sequence (Nasal)
\end{styleBody}

\begin{styleBody}
\textbf{Morphological constituency of maximal syllable margin:} Both patterns (Onset)
\end{styleBody}

\begin{styleBody}
\textbf{Morphological pattern of syllabic consonants:} N/A
\end{styleBody}

\begin{styleBody}
\textbf{Onset restrictions: }Any consonant may occur as a simple onset. In biconsonantal onsets, C\textsubscript{1} may be any consonant except for a glide /w j/ or labialized stop or fricative /k\textsuperscript{w} \textsuperscript{ŋ}[261?]\textsuperscript{w} x\textsuperscript{w}/, and C\textsubscript{2} may be /j w l x/. /sk/ onsets also occur.
\end{styleBody}

\begin{styleBody}
\textbf{Coda restrictions: }All consonants occur except for prenasalized stops.
\end{styleBody}

\begin{styleBody}
\textbf{Notes: }The biconsonantal onset patterns described above include what Loughnane considers to be ‘marginal’ clusters: those that are realized for some speakers as clusters and for other speakers with a very short or full schwa vowel between the consonants (Loughnane 2009: 64-5). Since these are regular patterns for some speakers I include them here.
\end{styleBody}

\begin{styleBody}
\textbf{Suprasegmentals}
\end{styleBody}

\begin{styleBody}
\textbf{Tone:} No
\end{styleBody}

\begin{styleBody}
\textbf{Word stress:} No
\end{styleBody}

\begin{styleBody}
\textbf{Vowel reduction processes}
\end{styleBody}

\begin{styleBody}
\textbf{opm-R1:} Nasals may become syllabic in the fast speech of some speakers; example given (m[259?]m[263?]an {\textgreater} m[329?][263?]an) involves either deletion of initial CV or deletion of interconsonantal V and syllabification of nasal (Loughnane 2009: 64).
\end{styleBody}

\begin{styleBody}
\textbf{Consonant allophony processes}
\end{styleBody}

\begin{styleBody}
\textbf{opm-C1: }A voiceless bilabial fricative is realized as a bilabial stop preceding a syllable boundary followed by a consonant (Loughnane 2009: 33).
\end{styleBody}

\begin{styleBody}
\textbf{opm-C2: }A voiceless bilabial fricative is realized as a stop word-finally (Loughnane 2009: 33).
\end{styleBody}

\begin{styleBody}
\textbf{opm-C3: }A voiceless velar fricative is realized as voiceless palatal fricative syllable-initially preceding a high front vowel or syllable-finally following a high front vowel (Loughnane 2009: 42).
\end{styleBody}

\begin{styleBody}
\textbf{opm-C4: }A voiceless velar fricative is realized as a voiced palatal fricative following /i/ and preceding another vowel (Loughnane 2009: 42).
\end{styleBody}

\begin{styleBody}
\textbf{opm-C5: }Voiceless fricatives are voiced intervocalically (Loughnane 2009: 42).
\end{styleBody}

\begin{styleBody}
\textbf{Morphology}
\end{styleBody}

\begin{styleBody}
\textbf{Text:} “Echidna, laxjan bird, and bat” (Loughnane 2009: 493-502)
\end{styleBody}

\begin{styleBody}
\textbf{Synthetic index: }1.7 morphemes/word (843 morphemes, 482 words)
\end{styleBody}

\clearpage\begin{styleBody}
\textbf{[pac] }\ \ \textbf{\textsc{Pacoh}}\textbf{\ \ }Austroasiatic, \textit{Katuic} (Vietnam)
\end{styleBody}

\begin{styleBody}
\textbf{References consulted: }Alves (2000, 2006)
\end{styleBody}

\begin{styleBody}
\textbf{Sound inventory}
\end{styleBody}

\begin{styleBody}
\textbf{C phoneme inventory:} /p b t d c [25F?] k [294?] p[2B0?] t[2B0?] k[2B0?] m n [272?] ŋ ç h r l w j w\textsuperscript{[294?]} j\textsuperscript{[294?]}/
\end{styleBody}

\begin{styleBody}
\textbf{N consonant phonemes:} 23
\end{styleBody}

\begin{styleBody}
\textbf{Geminates:} N/A
\end{styleBody}

\begin{styleBody}
\textbf{Voicing contrasts:} Obstruents
\end{styleBody}

\begin{styleBody}
\textbf{Places:} Bilabial, Alveolar, Palatal, Velar, Glottal
\end{styleBody}

\begin{styleBody}
\textbf{Manners:} Stop, Fricative, Nasal, Trill, Central approximant, Lateral approximant
\end{styleBody}

\begin{styleBody}
\textbf{N elaborations:} 2
\end{styleBody}

\begin{styleBody}
\textbf{Elaborations:} Creaky voice, Post-aspiration
\end{styleBody}

\begin{styleBody}
\textbf{V phoneme inventory:} /i e[330?] [25B?] æ [268?] [259?] [259?][330?] a [252?] [254?] o[330?] u i[2D0?] e[330?][2D0?] [25B?][2D0?] æ[2D0?] [268?][2D0?] [259?][2D0?] [259?][330?][2D0?] a[2D0?] [252?][2D0?] [254?][2D0?] o[330?][2D0?] u[2D0?]/
\end{styleBody}

\begin{styleBody}
\textbf{N vowel qualities:} 12
\end{styleBody}

\begin{styleBody}
\textbf{Diphthongs or vowel sequences:} Diphthongs /i[259?] [268?][259?] u[259?] i[259?][330?] [268?][259?][330?] u[259?][330?]/
\end{styleBody}

\begin{styleBody}
\textbf{Contrastive length:} All
\end{styleBody}

\begin{styleBody}
\textbf{Contrastive nasalization:} None
\end{styleBody}

\begin{styleBody}
\textbf{Other contrasts:} Glottalization (some)
\end{styleBody}

\begin{styleBody}
\textbf{Notes:} /ç/ ranges from alveolar to palatal fricative. The vowels transcribed as creaky voice differ in [RTR] value ([+RTR]), which manifests as both lower vowel quality and glottalic or ‘slight degree of raspiness’. This distinction is common in Mon-Khmer language and generally has phonation effects such as “breathiness, creakiness, or raspiness” (Alves 2006: 14).
\end{styleBody}

\begin{styleBody}
\textbf{Syllable structure}
\end{styleBody}

\begin{styleBody}
\textbf{Complexity Category: }Moderately Complex
\end{styleBody}

\begin{styleBody}
\textbf{Canonical syllable structure:} C(C)V(C)\textbf{ }(Alves 2006: 17-21)
\end{styleBody}

\begin{styleBody}
\textbf{Size of maximal onset:} 2
\end{styleBody}

\begin{styleBody}
\textbf{Size of maximal coda:} 1
\end{styleBody}

\begin{styleBody}
\textbf{Onset obligatory:} No
\end{styleBody}

\begin{styleBody}
\textbf{Coda obligatory:} No
\end{styleBody}

\begin{styleBody}
\textbf{Vocalic nucleus patterns:} Short vowels, Long vowels
\end{styleBody}

\begin{styleBody}
\textbf{Syllabic consonant patterns:} Nasal, Liquid
\end{styleBody}

\begin{styleBody}
\textbf{Size of maximal word{}-marginal sequences with syllabic obstruents:} N/A
\end{styleBody}

\begin{styleBody}
\textbf{Predictability of syllabic consonants:} Phonemic
\end{styleBody}

\begin{styleBody}
\textbf{Morphological constituency of maximal syllable margin:} Morpheme-internal (Onset)
\end{styleBody}

\begin{styleBody}
\textbf{Morphological pattern of syllabic consonants:} Lexical items
\end{styleBody}

\begin{styleBody}
\textbf{Onset restrictions: }Simple onsets unrestricted. For complex onsets, C\textsubscript{1} must be a stop, and C\textsubscript{2} must be /l r/. 
\end{styleBody}

\begin{styleBody}
\textbf{Coda restrictions: }In main syllables, apparently all consonants occur. In presyllables, only sonorants occur.
\end{styleBody}

\begin{styleBody}
\textbf{Suprasegmentals}
\end{styleBody}

\begin{styleBody}
\textbf{Tone:} No
\end{styleBody}

\begin{styleBody}
\textbf{Word stress:} Yes
\end{styleBody}

\begin{styleBody}
\textbf{Stress placement:} Fixed
\end{styleBody}

\begin{styleBody}
\textbf{Phonetic processes conditioned by stress:} (None)
\end{styleBody}

\begin{styleBody}
\textbf{Differences in phonological properties of stressed and unstressed syllables:} Vowel Quality Contrasts, Vowel Length Contrasts, Consonant Contrasts
\end{styleBody}

\begin{styleBody}
\textbf{Phonetic correlates of stress: }Vowel duration (impressionistic), Intensity (impressionistic)
\end{styleBody}

\begin{styleBody}
\textbf{Vowel reduction processes}
\end{styleBody}

\begin{styleBody}
(none reported)
\end{styleBody}

\begin{styleBody}
\textbf{Notes: }Alves (2000) notes that only [[259?]] occurs as a vocalic nucleus in closed presyllables; however, this does not appear to be a currently productive process. Likewise syllabic nasals and liquids can only occur in presyllables with glottal-stop onsets. “Clearly, some kind of phonetic reduction is resulting in the loss of vowel distinctions in closed presyllables and in the complete loss of vowels in presyllables with nasals as the sonorant peaks …” (2000: 22)
\end{styleBody}

\begin{styleBody}
\textbf{Consonant allophony processes}
\end{styleBody}

\begin{styleBody}
\textbf{pac-C1: }A labiovelar approximant is realized as a labiodental fricative word-initially (Alves 2006: 11).
\end{styleBody}

\begin{styleBody}
\textbf{pac-C2: }Velar consonants are labialized following /u/ or /o/ (Alves 2006: 12).
\end{styleBody}

\begin{styleBody}
\textbf{Morphology}
\end{styleBody}

\begin{styleBody}
\textbf{Text:} “The Old Days” and “Pacoh Fellows and Girls” (Watson 1980: 86-7, 182-4)
\end{styleBody}

\begin{styleBody}
\textbf{Synthetic index: }1.1 morphemes/word (557 morphemes, 520 words)
\end{styleBody}

\clearpage\begin{styleBody}
\textbf{[pay] }\ \ \textbf{\textsc{Pech}}\textbf{\ \ }Chibchan (Honduras)
\end{styleBody}

\begin{styleBody}
\textbf{References consulted: }Holt (1986, 1999)
\end{styleBody}

\begin{styleBody}
\textbf{Sound inventory}
\end{styleBody}

\begin{styleBody}
\textbf{C phoneme inventory:} /p b t[32A?] k k[2B7?] [294?] s[32A?] [283?] h m n [27E?] r l[32A?] j w/
\end{styleBody}

\begin{styleBody}
\textbf{N consonant phonemes:} 16
\end{styleBody}

\begin{styleBody}
\textbf{Geminates:} N/A
\end{styleBody}

\begin{styleBody}
\textbf{Voicing contrasts:} Obstruents
\end{styleBody}

\begin{styleBody}
\textbf{Places:} Bilabial, Dental, Palato-alveolar, Velar, Glottal
\end{styleBody}

\begin{styleBody}
\textbf{Manners:} Stop, Fricative, Nasal, Flap/Tap, Trill, Central approximant, Lateral approximant
\end{styleBody}

\begin{styleBody}
\textbf{N elaborations:} 2
\end{styleBody}

\begin{styleBody}
\textbf{Elaborations:} Palato-alveolar, Labialization
\end{styleBody}

\begin{styleBody}
\textbf{V phoneme inventory:} /i e a o u i[2D0?] e[2D0?] a[2D0?] o[2D0?] u[2D0?] \~{i} \~{e} \~{a} \~{o} \~{u} \~{i}[2D0?] \~{e}[2D0?] \~{a}[2D0?] \~{o}[2D0?] \~{u}[2D0?] /
\end{styleBody}

\begin{styleBody}
\textbf{N vowel qualities:} 5
\end{styleBody}

\begin{styleBody}
\textbf{Diphthongs or vowel sequences:} Diphthongs /aj aw ej/
\end{styleBody}

\begin{styleBody}
\textbf{Contrastive length:} All
\end{styleBody}

\begin{styleBody}
\textbf{Contrastive nasalization:} All
\end{styleBody}

\begin{styleBody}
\textbf{Other contrasts:} N/A
\end{styleBody}

\begin{styleBody}
\textbf{Syllable structure}
\end{styleBody}

\begin{styleBody}
\textbf{Complexity Category:} Complex
\end{styleBody}

\begin{styleBody}
\textbf{Canonical syllable structure:} (C)(C)V(C)(C)\textbf{ }(Holt 1999: 20-21)
\end{styleBody}

\begin{styleBody}
\textbf{Size of maximal onset:} 2
\end{styleBody}

\begin{styleBody}
\textbf{Size of maximal coda:} 2
\end{styleBody}

\begin{styleBody}
\textbf{Onset obligatory:} No
\end{styleBody}

\begin{styleBody}
\textbf{Coda obligatory:} No
\end{styleBody}

\begin{styleBody}
\textbf{Vocalic nucleus patterns:} Short vowels, Long vowels, Diphthongs
\end{styleBody}

\begin{styleBody}
\textbf{Syllabic consonant patterns:} N/A
\end{styleBody}

\begin{styleBody}
\textbf{Size of maximal word{}-marginal sequences with syllabic obstruents:} N/A
\end{styleBody}

\begin{styleBody}
\textbf{Predictability of syllabic consonants:} N/A
\end{styleBody}

\begin{styleBody}
\textbf{Morphological constituency of maximal syllable margin:} Morpheme-internal (Coda), Both patterns (Onset)
\end{styleBody}

\begin{styleBody}
\textbf{Morphological pattern of syllabic consonants:} N/A
\end{styleBody}

\begin{styleBody}
\textbf{Onset restrictions: }Apparently none for simple onsets. C\textsubscript{2} is /r/ in biconsonantal onsets.
\end{styleBody}

\begin{styleBody}
\textbf{Coda restrictions: }In simple codas, all consonants except /p t k[2B7?] b l[32A?] m/ occur. Biconsonantal codas have /[27E?]/ as the first member, and only occur medially.
\end{styleBody}

\begin{styleBody}
\textbf{Notes: }/p[27E?], t[27E?], k[27E?], b[27E?]/ onsets appear to be a recent development as a result of syncope of historical or underlying vowels. These vowels “often reappear in extremely slow, careful speech” (Holt 1999: 20).
\end{styleBody}

\begin{styleBody}
\textbf{Suprasegmentals}
\end{styleBody}

\begin{styleBody}
\textbf{Tone:} Yes
\end{styleBody}

\begin{styleBody}
\textbf{Word stress:} Yes
\end{styleBody}

\begin{styleBody}
\textbf{Stress placement:} Morphologically or Lexically Conditioned
\end{styleBody}

\begin{styleBody}
\textbf{Phonetic processes conditioned by stress:} Vowel Reduction, Consonant Allophony in Unstressed Syllables
\end{styleBody}

\begin{styleBody}
\textbf{Differences in phonological properties of stressed and unstressed syllables:} Not described
\end{styleBody}

\begin{styleBody}
\textbf{Phonetic correlates of stress: }Not described
\end{styleBody}

\begin{styleBody}
\textbf{Notes:} Pitch only hinted to be a correlate of stress (Holt 1986: 238). Stress is apparently predictable on the basis of underlying tone (1999: 19). Holt describes tone system as being relatively simple, with tones associated with certain marked syllables and distributed to unmarked syllables through assimilation or prosodic patterns (so patterns somewhat predictable?).
\end{styleBody}

\begin{styleBody}
\textbf{Vowel reduction processes}
\end{styleBody}

\begin{styleBody}
\textbf{pay-R1:} In closed syllables, long vowels are realized as relatively short (Holt 1999: 18).
\end{styleBody}

\begin{styleBody}
\textbf{pay-R2}: Short vowels are usually open and lax in closed syllables and when unstressed (Holt 1999: 18).
\end{styleBody}

\begin{styleBody}
\textbf{pay-R3: }In rapid speech, unstressed high front vowel /i/ is sometimes realized as [[259?]] (Holt 1999: 18).
\end{styleBody}

\begin{styleBody}
\textbf{pay-R4:} In rapid speech, vowels in unstressed syllables are sometimes voiceless between voiceless consonants (Holt 1999: 18).
\end{styleBody}

\begin{styleBody}
\textbf{pay-R5:} Unstressed vowels are usually deleted between any consonant and a following /[27E?]/ (Holt 1999: 23).
\end{styleBody}

\begin{styleBody}
\textbf{pay-R6}: An unstressed interconsonantal vowel is often lost between two stressed syllables (Holt 1999: 23).
\end{styleBody}

\begin{styleBody}
\textbf{pay-R7:} The length of a long vowel can metathesize with that of a following consonant (usually /[283?]/ or /k/), shortening the vowel and lengthening the following consonant (unclear if following C is in same syllable) (Holt 1999: 24-5).
\end{styleBody}

\begin{styleBody}
\textbf{Consonant allophony processes}
\end{styleBody}

\begin{styleBody}
\textbf{pay-C1: }A voiceless palato-alveolar fricative may be realized as an affricate preceding a vowel (Holt 1999: 16).
\end{styleBody}

\begin{styleBody}
\textbf{pay-C2: }Glides may be realized as pre-stopped [\textsuperscript{d}j \textsuperscript{[261?]}w] when occurring word-initially (Holt 1999: 16).
\end{styleBody}

\begin{styleBody}
\textbf{pay-C3: }A voiceless velar stop is realized as voiced following a long vowel (Holt 1999: 15-16).
\end{styleBody}

\begin{styleBody}
\textbf{pay-C4: }A voiced bilabial stop is spirantized intervocalically (Holt 1999: 16).
\end{styleBody}

\begin{styleBody}
\textbf{Morphology}
\end{styleBody}

\begin{styleBody}
\textbf{Text:} “Sample text” (Holt 1999: 79-80)
\end{styleBody}

\begin{styleBody}
\textbf{Synthetic index: }2.8 morphemes/word (196 morphemes, 69 words)
\end{styleBody}

\clearpage\begin{styleBody}
\textbf{[pib] }\ \ \textbf{\textsc{Yine}}\textbf{\ \ }Arawakan, \textit{Southern Maipuran} (Peru)
\end{styleBody}

\begin{styleBody}
\textbf{References consulted: }Hanson (2010), Lin (1997), Matteson (1965), Parker (1989), Urquía Sebastián \& Marlett (2008)
\end{styleBody}

\begin{styleBody}
\textbf{Sound inventory}
\end{styleBody}

\begin{styleBody}
\textbf{C phoneme inventory:} /p t c k t[361?]s t[361?][283?] s [283?] ç [266?] m n l [27E?] w j/
\end{styleBody}

\begin{styleBody}
\textbf{N consonant phonemes:} 16
\end{styleBody}

\begin{styleBody}
\textbf{Geminates:} N/A
\end{styleBody}

\begin{styleBody}
\textbf{Voicing contrasts:} None
\end{styleBody}

\begin{styleBody}
\textbf{Places:} Bilabial, Alveolar, Palato-alveolar, Palatal, Velar, Glottal
\end{styleBody}

\begin{styleBody}
\textbf{Manners:} Stop, Affricate, Fricative, Nasal, Flap/Tap, Central approximant, Lateral approximant
\end{styleBody}

\begin{styleBody}
\textbf{N elaborations:} 1
\end{styleBody}

\begin{styleBody}
\textbf{Elaborations:} Palato-alveolar
\end{styleBody}

\begin{styleBody}
\textbf{V phoneme inventory:} /i e [268?] a o/
\end{styleBody}

\begin{styleBody}
\textbf{N vowel qualities:} 5
\end{styleBody}

\begin{styleBody}
\textbf{Diphthongs or vowel sequences:} None
\end{styleBody}

\begin{styleBody}
\textbf{Contrastive length:} None
\end{styleBody}

\begin{styleBody}
\textbf{Contrastive nasalization:} None
\end{styleBody}

\begin{styleBody}
\textbf{Other contrasts:} N/A
\end{styleBody}

\begin{styleBody}
\textbf{Notes:} Urquia Sebastian reports /c[361?]ç/ intead of /c/; Hanson notes these vary freely. /[266?]/ has very wide range of variation (Hanson 2010: 20-23). Matteson gives /[26F?]/ for Hanson’s /[268?]/. Urquía Sebastián, Matteson report vowel length distinctions for all vowel qualities but Lin, Hanson report no contrastive vowel length.
\end{styleBody}

\begin{styleBody}
\textbf{Syllable structure}
\end{styleBody}

\begin{styleBody}
\textbf{Complexity Category:} Highly Complex
\end{styleBody}

\begin{styleBody}
\textbf{Canonical syllable structure:} C(C)(C)V\textbf{ }(Hanson 2010: 25; Matteson 1965: 22-32; Matteson \& Pike 1958; Lin 1997: 404-6; Lin 1993)
\end{styleBody}

\begin{styleBody}
\textbf{Size of maximal onset:} 3
\end{styleBody}

\begin{styleBody}
\textbf{Size of maximal coda:} N/A
\end{styleBody}

\begin{styleBody}
\textbf{Onset obligatory:} Yes
\end{styleBody}

\begin{styleBody}
\textbf{Coda obligatory:} N/A
\end{styleBody}

\begin{styleBody}
\textbf{Vocalic nucleus patterns:} Short vowels, Long vowels
\end{styleBody}

\begin{styleBody}
\textbf{Syllabic consonant patterns:} Nasal (Conflicting), Liquid (Conflicting), Obstruent (Conflicting)
\end{styleBody}

\begin{styleBody}
\textbf{Size of maximal word{}-marginal sequences with syllabic obstruents:} N/A
\end{styleBody}

\begin{styleBody}
\textbf{Predictability of syllabic consonants:} N/A
\end{styleBody}

\begin{styleBody}
\textbf{Morphological constituency of maximal syllable margin:} Morphologically Complex (Onset)
\end{styleBody}

\begin{styleBody}
\textbf{Morphological pattern of syllabic consonants:} N/A
\end{styleBody}

\begin{styleBody}
\textbf{Onset restrictions: }All consonants occur as simple onsets. Biconsonantal onsets have no sonority constraints, though most identical clusters do not occur, and combinations of obstruents with similar place/manner rare. Examples of occurring biconsonantal onsets include /pt mw çp jw ks tm mt sm ms nn kn tl/. Triconsonantal onsets include /pc[27E?] nkn wt[361?][283?]k nt[361?]sp nt[361?][283?]k mtn/.
\end{styleBody}

\begin{styleBody}
\textbf{Notes: }Hanson notes that word-internal nasals are often (but inconsistently) treated as ambisyllabic, but never treated as only codas (2010: 25). No other explicit examples of 3-obstruent clusters, but all triconsonantal clusters are morphologically complex and have consonantal prefix as first constituent (including pronominal prefixes /n-/, /t-/, /p-/, /[27E?]{}-/, /w-/). Matteson states that C\textsubscript{1} may not be a fricative or affricate, but that there are no general restrictions on C\textsubscript{2} and C\textsubscript{3 }(1965: 29-30). Combinatory restrictions on place/manner would still seem to allow three-obstruent codas. Additionally, Matteson states that the frequency of triconsonantal onsets was lower in 1965 than a count made a decade before. Therefore it could be that syllable structure in the language is simplifying. I classify it as having Highly Complex syllable structure, while acknowledging that such patterns may be marginal in the language.
\end{styleBody}

\begin{styleBody}
\textbf{Suprasegmentals}
\end{styleBody}

\begin{styleBody}
\textbf{Tone:} No
\end{styleBody}

\begin{styleBody}
\textbf{Word stress:} Yes
\end{styleBody}

\begin{styleBody}
\textbf{Stress placement:} Fixed
\end{styleBody}

\begin{styleBody}
\textbf{Phonetic processes conditioned by stress:} Vowel Reduction
\end{styleBody}

\begin{styleBody}
\textbf{Differences in phonological properties of stressed and unstressed syllables:} (None)
\end{styleBody}

\begin{styleBody}
\textbf{Phonetic correlates of stress: }Pitch (impressionistic)
\end{styleBody}

\begin{styleBody}
\textbf{Vowel reduction processes}
\end{styleBody}

\begin{styleBody}
\textbf{pib-R1:} A vowel in an (unstressed) utterance-final syllable may be wholly or partly voiceless (Matteson 1965: 23).
\end{styleBody}

\begin{styleBody}
\textbf{pib-R2:} Short low vowel /a/ may be realized as a neutral vowel in an (unstressed) utterance-final syllable (Matteson 1965: 23).
\end{styleBody}

\begin{styleBody}
\textbf{pib-R3:} Unstressed vowels are ‘somewhat more centralized but without any significant reduction’ (Hanson 2010: 16).
\end{styleBody}

\begin{styleBody}
\textbf{Consonant allophony processes}
\end{styleBody}

\begin{styleBody}
\textbf{pib-C1: }A voiceless palatal stop varies freely with affricated variant [cç] (Hanson 2010: 17).
\end{styleBody}

\begin{styleBody}
\textbf{pib-C2: }An alveolar lateral approximant may be realized as a stop following a nasal consonant (Hanson 2010: 24).
\end{styleBody}

\begin{styleBody}
\textbf{pib-C3: }Stops are realized as voiced intervocalically (Hanson 2010: 17).
\end{styleBody}

\begin{styleBody}
\textbf{pib-C4: }An alveolar lateral approximant is realized as an alveolar flap following a front or high central vowel (Hanson 2010: 24).
\end{styleBody}

\begin{styleBody}
\textbf{Morphology}
\end{styleBody}

\begin{styleBody}
\textbf{Text: }“The anteater and the jaguars” (Hanson 2010: 379-386)
\end{styleBody}

\begin{styleBody}
\textbf{Synthetic index: }2.1 morphemes/word (539 morphemes, 257 words)
\end{styleBody}

\clearpage\begin{styleBody}
\textbf{[pol] }\ \ \textbf{\textsc{Polish}}\textbf{\ \ }Indo-European, \textit{Balto-Slavic} (Poland)
\end{styleBody}

\begin{styleBody}
\textbf{References consulted: }Gussman (2007), Jassem (2003), Newlin-Łokowicz (2012), Nowak (2006), Rocławski (1976), Zydorowicz (2010)
\end{styleBody}

\begin{styleBody}
\textbf{Sound inventory}
\end{styleBody}

\begin{styleBody}
\textbf{C phoneme inventory:} /p b p[2B2?] b[2B2?] t[32A?] d[32A?] k [261?] k[2B2?] [261?][2B2?] t[32A?][361?]s[32A?] d[32A?][361?]z[32A?] t[361?][283?] d[361?][292?] t[361?][255?] d[361?][291?] f v f[2B2?] v[2B2?] s[32A?] z[32A?] [283?] [292?] [255?] [291?] x m m[2B2?] n[32A?] [272?] r l j w/
\end{styleBody}

\begin{styleBody}
\textbf{N consonant phonemes:} 35
\end{styleBody}

\begin{styleBody}
\textbf{Geminates:} N/A
\end{styleBody}

\begin{styleBody}
\textbf{Voicing contrasts:} Obstruents
\end{styleBody}

\begin{styleBody}
\textbf{Places:} Bilabial, Labiodental, Dental, Alveolar, Palato-alveolar, Alveolo-palatal, Velar
\end{styleBody}

\begin{styleBody}
\textbf{Manners:} Stop, Affricate, Fricative, Nasal, Trill, Central approximant, Lateral approximant
\end{styleBody}

\begin{styleBody}
\textbf{N elaborations:} 4
\end{styleBody}

\begin{styleBody}
\textbf{Elaborations:} Voiced fricatives/affricates, Labiodental, Palato-alveolar, Palatalization
\end{styleBody}

\begin{styleBody}
\textbf{V phoneme inventory:} /i [25B?] [268?] a [254?] u [25B?] [254?]/
\end{styleBody}

\begin{styleBody}
\textbf{N vowel qualities:} 6
\end{styleBody}

\begin{styleBody}
\textbf{Diphthongs or vowel sequences:} None
\end{styleBody}

\begin{styleBody}
\textbf{Contrastive length:} None
\end{styleBody}

\begin{styleBody}
\textbf{Contrastive nasalization:} Some
\end{styleBody}

\begin{styleBody}
\textbf{Other contrasts:} N/A
\end{styleBody}

\begin{styleBody}
\textbf{Notes:} Jassem and Gussman differ quite a bit in their C inventories. Palatalized labials are often realized a sequences of C\textsuperscript{j}j before vowels that are not /i/. Some of the consonants Gussman listed have been omitted here because they are predictable (/s[32A?][2B2?] z[32A?][2B2?] x[2B2?]/) or occur only in loanwords (/[283?][2B2?] [292?][2B2?]/, etc.). /[25B?] [254?]/ have nasalized counterparts, which may be perhaps better analyzed as diphthongs: /[25B?][2B7?] [254?][2B7?]/.
\end{styleBody}

\begin{styleBody}
\textbf{Syllable structure}
\end{styleBody}

\begin{styleBody}
\textbf{Complexity Category:} Highly Complex
\end{styleBody}

\begin{styleBody}
\textbf{Canonical syllable structure:} (C)(C)(C)(C)(C)V(C)(C)(C)(C)(C)\textbf{ }(Gussman 2007: 200-224; Jassem 2003: 103; Zydorowicz 2010, Bargiełowna 1950)
\end{styleBody}

\begin{styleBody}
\textbf{Size of maximal onset:} 5
\end{styleBody}

\begin{styleBody}
\textbf{Size of maximal coda:} 5
\end{styleBody}

\begin{styleBody}
\textbf{Onset obligatory:} No
\end{styleBody}

\begin{styleBody}
\textbf{Coda obligatory:} No
\end{styleBody}

\begin{styleBody}
\textbf{Vocalic nucleus patterns:} Short vowels
\end{styleBody}

\begin{styleBody}
\textbf{Syllabic consonant patterns:} Nasal, Liquid
\end{styleBody}

\begin{styleBody}
\textbf{Size of maximal word{}-marginal sequences with syllabic obstruents:} N/A
\end{styleBody}

\begin{styleBody}
\textbf{Predictability of syllabic consonants:} Varies with CV sequence
\end{styleBody}

\begin{styleBody}
\textbf{Morphological constituency of maximal syllable margin:} Morphologically Complex (Onset, Coda)
\end{styleBody}

\begin{styleBody}
\textbf{Morphological pattern of syllabic consonants:} N/A
\end{styleBody}

\begin{styleBody}
\textbf{Onset restrictions: }Biconsonantal onsets quite varied, include /kt ft[361?]s t[283?] nd jm [261?]d[361?][291?] p[283?] dv dr pl [261?]w sm mn wd/. Triconsonantal onsets include /sxl [283?]kw p[283?]t b[292?]m[2B2?] x[283?]t tkf[2B2?]/. Four-consonant onsets include /pstr pst[283?] fstr dr[261?]n fks[283?] vz[261?]l/. Five-consonant onsets like /spstr/ may occur in phonological words.
\end{styleBody}

\begin{styleBody}
\textbf{Coda restrictions: }Biconsonantal codas include /[255?]t[361?][255?] st kt st rf wn nt[361?]s lk wf/. Triconsonantal codas include /n[283?]t l[255?][272?] jsk psk stf rtf xtr/. Four-consonant codas include /[272?]stf tstf rstf/, have strict limitations on final three consonants.
\end{styleBody}

\begin{styleBody}
\textbf{Notes: }The examples presented in Bargiełowna (1950: 21) suggest that 5-C codas may occur in rare cases when phonemic nasalized vowel precedes 4-C cluster, e.g. \textit{przestępstw }[p[283?]estempstf].
\end{styleBody}

\begin{styleBody}
\textbf{Suprasegmentals}
\end{styleBody}

\begin{styleBody}
\textbf{Tone:} No
\end{styleBody}

\begin{styleBody}
\textbf{Word stress:} Yes
\end{styleBody}

\begin{styleBody}
\textbf{Stress placement:} Fixed
\end{styleBody}

\begin{styleBody}
\textbf{Phonetic processes conditioned by stress:} Vowel Reduction
\end{styleBody}

\begin{styleBody}
\textbf{Differences in phonological properties of stressed and unstressed syllables:} (None)
\end{styleBody}

\begin{styleBody}
\textbf{Phonetic correlates of stress: }Vowel duration (instrumental), Pitch (instrumental), Intensity (instrumental)
\end{styleBody}

\begin{styleBody}
\textbf{Notes: }The pitch correlate here is a pitch slope.
\end{styleBody}

\begin{styleBody}
\textbf{Vowel reduction processes}
\end{styleBody}

\begin{styleBody}
\textbf{pol-R1: }Non-high vowels /[25B?] [254?] a/ are more reduced in F1 and F2 domains in non-stressable positions (Nowak 2006: 378-9).
\end{styleBody}

\begin{styleBody}
\textbf{pol-R2: }In rapid speech, syllabic nasals and liquids may occur as optional variants of vowel-consonant sequences (occurs more often in grammatical elements; Rubach 1974).
\end{styleBody}

\begin{styleBody}
\textbf{Consonant allophony processes}
\end{styleBody}

\begin{styleBody}
\textbf{pol-C1: }An alveolar trill is realized as a flap in rapid speech (Rocławski 1976: 132).
\end{styleBody}

\begin{styleBody}
\textbf{Morphology}
\end{styleBody}

\begin{styleBody}
(adequate texts unavailable)
\end{styleBody}

\clearpage\begin{styleBody}
\textbf{[pqm] }\ \ \textbf{\textsc{Passamaquoddy-Maliseet}}\textbf{\ \ }Algic, \textit{Algonquian} (Canada, United States)
\end{styleBody}

\begin{styleBody}
\textbf{References consulted: }Leavitt (1996), LeSourd (1993)
\end{styleBody}

\begin{styleBody}
\textbf{Sound inventory}
\end{styleBody}

\begin{styleBody}
\textbf{C phoneme inventory:} /p t k k[2B7?] t[361?][283?] s h m n l w j/
\end{styleBody}

\begin{styleBody}
\textbf{N consonant phonemes:} 12
\end{styleBody}

\begin{styleBody}
\textbf{Geminates:} N/A
\end{styleBody}

\begin{styleBody}
\textbf{Voicing contrasts:} None
\end{styleBody}

\begin{styleBody}
\textbf{Places:} Bilabial, Alveolar, Palato-alveolar, Velar, Glottal
\end{styleBody}

\begin{styleBody}
\textbf{Manners:} Stop, Affricate, Fricative, Nasal, Central approximant, Lateral approximant
\end{styleBody}

\begin{styleBody}
\textbf{N elaborations:} 1
\end{styleBody}

\begin{styleBody}
\textbf{Elaborations:} Palato-alveolar
\end{styleBody}

\begin{styleBody}
\textbf{V phoneme inventory:} /i [25B?] [259?] a o o[2D0?]/
\end{styleBody}

\begin{styleBody}
\textbf{N vowel qualities:} 5
\end{styleBody}

\begin{styleBody}
\textbf{Diphthongs or vowel sequences:} None
\end{styleBody}

\begin{styleBody}
\textbf{Contrastive length:} Some
\end{styleBody}

\begin{styleBody}
\textbf{Contrastive nasalization:} None
\end{styleBody}

\begin{styleBody}
\textbf{Other contrasts:} N/A
\end{styleBody}

\begin{styleBody}
\textbf{Notes:} Aspirated stops /p[2B0?] t[2B0?] k[2B0?]/ occur word-initially as a result of some morphophonemic contrasts and contrast with stop+/h/ (Sherwood 1986). LeSourd argues that preaspirated stops are clusters (1996: 38-41). /o/ here is intermediate between [o] and [u]. /[259?]/ is never lengthened under stress, but is contrastive word-initially. According to LeSourd, vowel length is predictable in both dialects, except for /o/ and /o[2D0?]/ before /w/.
\end{styleBody}

\begin{styleBody}
\textbf{Syllable structure}
\end{styleBody}

\begin{styleBody}
\textbf{Complexity Category:} Highly Complex
\end{styleBody}

\begin{styleBody}
\textbf{Canonical syllable structure:} (C)(C)(C)V(C)(C)(C)\textbf{ }(LeSourd 1993: 58-61, 121-160)
\end{styleBody}

\begin{styleBody}
\textbf{Size of maximal onset:} 3
\end{styleBody}

\begin{styleBody}
\textbf{Size of maximal coda:} 3
\end{styleBody}

\begin{styleBody}
\textbf{Onset obligatory:} No
\end{styleBody}

\begin{styleBody}
\textbf{Coda obligatory:} No
\end{styleBody}

\begin{styleBody}
\textbf{Vocalic nucleus patterns:} Short vowels, Long vowels
\end{styleBody}

\begin{styleBody}
\textbf{Syllabic consonant patterns:} N/A
\end{styleBody}

\begin{styleBody}
\textbf{Size of maximal word{}-marginal sequences with syllabic obstruents:} N/A
\end{styleBody}

\begin{styleBody}
\textbf{Predictability of syllabic consonants:} N/A
\end{styleBody}

\begin{styleBody}
\textbf{Morphological constituency of maximal syllable margin:} Morpheme-internal (Onset, Coda)
\end{styleBody}

\begin{styleBody}
\textbf{Morphological pattern of syllabic consonants:} N/A
\end{styleBody}

\begin{styleBody}
\textbf{Onset restrictions:} All consonants occur as simple onsets. Biconsonantal onset combinations are fairly unrestricted and include /pt tp pt[361?][283?] sk[2B7?] hs/. Triconsonantal onsets are usually of the form CsC: /psk psk[2B7?] ksp ksk[2B7?]/, though /nkh/, /nsp/ occur in some stems.
\end{styleBody}

\begin{styleBody}
\textbf{Coda restrictions: }All(?) consonants occur as simple codas (/h/ does not occur word-finally). Biconsonantal codas similar in form to onsets and are fairly unrestricted: /pt[361?][283?] tk[2B7?] t[361?][283?]k kp sk[2B7?] ts st hk[2B7?]/. Triconsonantal codas usually of form CsC, include /psk[2B7?] ksk[2B7?] nsk wsk/.
\end{styleBody}

\begin{styleBody}
\textbf{Suprasegmentals}
\end{styleBody}

\begin{styleBody}
\textbf{Tone:} Yes
\end{styleBody}

\begin{styleBody}
\textbf{Word stress:} Yes
\end{styleBody}

\begin{styleBody}
\textbf{Stress placement:} Weight-Sensitive
\end{styleBody}

\begin{styleBody}
\textbf{Phonetic processes conditioned by stress:} Vowel Reduction
\end{styleBody}

\begin{styleBody}
\textbf{Differences in phonological properties of stressed and unstressed syllables:} (None)
\end{styleBody}

\begin{styleBody}
\textbf{Phonetic correlates of stress: }Vowel duration (impressionistic)
\end{styleBody}

\begin{styleBody}
\textbf{Notes: }Duration a correlate of stress in open syllables. Language described as having pitch accent: a stressed syllable may bear high or low pitch/contour (LeSourd 1993: 62).
\end{styleBody}

\begin{styleBody}
\textbf{Vowel reduction processes}
\end{styleBody}

\begin{styleBody}
\textbf{pqm-R1:} Unstressed/unstressable /[259?]/ is omitted more often than not after sequences of /h/C, /ss/, or between non-syllabic sonorants, except in slow or deliberate speech (LeSourd 1996: 36).
\end{styleBody}

\begin{styleBody}
\textbf{pqm-R2: }Unstressable vowels which are not eliminated by phonological syncope are often subject to phonetic reduction or deletion (LeSourd 1996: 104).
\end{styleBody}

\begin{styleBody}
\textbf{Consonant allophony processes}
\end{styleBody}

\begin{styleBody}
\textbf{pqm-C1: }Voiceless stops may be voiced intervocalically (LeSourd 1993: 37).
\end{styleBody}

\begin{styleBody}
\textbf{pqm-C2: }Velar stops may be spirantized intervocalically (LeSourd 1993: 37)\textbf{.}
\end{styleBody}

\begin{styleBody}
\textbf{Morphology}
\end{styleBody}

\begin{styleBody}
\textbf{Text:} “A sample text” (Leavitt 1996: 55-58)
\end{styleBody}

\begin{styleBody}
\textbf{Synthetic index: }2.1 morphemes/word (262 morphemes, 125 words)
\end{styleBody}

\clearpage\begin{styleBody}
\textbf{[pwn] }\ \ \textbf{\textsc{Paiwan (Saichia dialect)}}\textbf{\ \ }Austronesian (Taiwan)
\end{styleBody}

\begin{styleBody}
\textbf{References consulted: }Chang (2006)
\end{styleBody}

\begin{styleBody}
\textbf{Sound inventory}
\end{styleBody}

\begin{styleBody}
\textbf{C phoneme inventory:} /p b t d t[2B2?] d[2B2?] [256?] k [261?] [294?] t[361?]s v s z m n ŋ [27E?] r [28E?] $\beta [31E?]$ j/
\end{styleBody}

\begin{styleBody}
\textbf{N consonant phonemes:} 22
\end{styleBody}

\begin{styleBody}
\textbf{Geminates:} N/A
\end{styleBody}

\begin{styleBody}
\textbf{Voicing contrasts: }Obstruents
\end{styleBody}

\begin{styleBody}
\textbf{Places: }Bilabial, Labiodental, Alveolar, Retroflex, Palatal, Velar, Glottal
\end{styleBody}

\begin{styleBody}
\textbf{Manners:} Stop, Affricate, Fricative, Nasal, Flap/tap, Trill, Central approximant, Lateral approximant
\end{styleBody}

\begin{styleBody}
\textbf{N elaborations:} 4
\end{styleBody}

\begin{styleBody}
\textbf{Elaborations:} Voiced fricatives/affricates, Labiodental, Retroflex, Palatalization
\end{styleBody}

\begin{styleBody}
\textbf{V phoneme inventory:} /i [259?] a u/
\end{styleBody}

\begin{styleBody}
\textbf{N vowel qualities:} 4
\end{styleBody}

\begin{styleBody}
\textbf{Diphthongs or vowel sequences:} None
\end{styleBody}

\begin{styleBody}
\textbf{Contrastive length:} None
\end{styleBody}

\begin{styleBody}
\textbf{Contrastive nasalization:} None
\end{styleBody}

\begin{styleBody}
\textbf{Other contrasts:} N/A
\end{styleBody}

\begin{styleBody}
\textbf{Notes:} Palatalized alveolars occur only in Saichia dialect. In Santimen, /[27E?] r/ have merged. /h/ occurs in loans from Japanese in Saichia dialect.
\end{styleBody}

\begin{styleBody}
\textbf{Syllable structure}
\end{styleBody}

\begin{styleBody}
\textbf{Complexity Category:} Moderately Complex
\end{styleBody}

\begin{styleBody}
\textbf{Canonical syllable structure:} (C)V(C) (Chang 2006: 31-34)
\end{styleBody}

\begin{styleBody}
\textbf{Size of maximal onset:} 1
\end{styleBody}

\begin{styleBody}
\textbf{Size of maximal coda: }1
\end{styleBody}

\begin{styleBody}
\textbf{Onset obligatory:} No
\end{styleBody}

\begin{styleBody}
\textbf{Coda obligatory:} No
\end{styleBody}

\begin{styleBody}
\textbf{Vocalic nucleus patterns: }Short vowels
\end{styleBody}

\begin{styleBody}
\textbf{Syllabic consonant patterns:} Nasal, Obstruent
\end{styleBody}

\begin{styleBody}
\textbf{Size of maximal word{}-marginal sequences with syllabic obstruents:} 2
\end{styleBody}

\begin{styleBody}
\textbf{Predictability of syllabic consonants:} Varies with CV sequence
\end{styleBody}

\begin{styleBody}
\textbf{Morphological constituency of maximal syllable margin:} N/A
\end{styleBody}

\begin{styleBody}
\textbf{Morphological pattern of syllabic consonants:} Lexical items
\end{styleBody}

\begin{styleBody}
\textbf{Onset restrictions:} None?
\end{styleBody}

\begin{styleBody}
\textbf{Coda restrictions:} None?
\end{styleBody}

\begin{styleBody}
\textbf{Notes:} Syllabic nasals and [s[329?] z[329?] t[361?]s[329?]] occur as result of variation with C[259?] sequences.
\end{styleBody}

\begin{styleBody}
\textbf{Suprasegmentals}
\end{styleBody}

\begin{styleBody}
\textbf{Tone:} No
\end{styleBody}

\begin{styleBody}
\textbf{Word stress: }Yes
\end{styleBody}

\begin{styleBody}
\textbf{Stress placement:} Morphologically or Lexically Conditioned
\end{styleBody}

\begin{styleBody}
\textbf{Phonetic processes conditioned by stress:}
\end{styleBody}

\begin{styleBody}
\textbf{Differences in phonological properties of stressed and unstressed syllables:} (None)
\end{styleBody}

\begin{styleBody}
\textbf{Phonetic correlates of stress:} Pitch (instrumental), Intensity (instrumental)
\end{styleBody}

\begin{styleBody}
\textbf{Vowel reduction processes}
\end{styleBody}

\begin{styleBody}
\textbf{pwn-R1:} In rapid speech, /[259?]/ may be dropped following sibilants /s z t[361?]s/, often resulting in a syllabic consonant (Chang 2006: 40-41).
\end{styleBody}

\begin{styleBody}
\textbf{pwn-R2:} In rapid speech, /[259?]/ may be dropped following nasals /m n ŋ/ (Chang 2006: 41).
\end{styleBody}

\begin{styleBody}
\textbf{pwn-R3:} In rapid speech, /[259?]/ may be dropped preceding nasals /m n ŋ/ resulting in a syllabic nasal (occurs with schwa preceding nasal coda produced by pwn-R2; Chang 2006: 41-42).
\end{styleBody}

\begin{styleBody}
\textbf{Consonant allophony processes}
\end{styleBody}

\begin{styleBody}
\textbf{pwn-C1:} Velar stops are fronted preceding high front vowel /i/ (Chang 2006: 22).
\end{styleBody}

\begin{styleBody}
\textbf{pwn-C2:} Labiovelar approximant /w/ is realized as a labiodental fricative [v] word-finally preceding a vowel (Chang 2006: 40).
\end{styleBody}

\begin{styleBody}
\textbf{Morphology}
\end{styleBody}

\begin{styleBody}
\textbf{Text:} “\textit{tjuvak — }Sea shells” (first 8 pages, Chang 2006: 431-438)
\end{styleBody}

\begin{styleBody}
\textbf{Synthetic index: }1.26 morphemes/word (544 morphemes, 433 words)
\end{styleBody}

\clearpage\begin{styleBody}
\textbf{[qvi] }\ \ \textbf{\textsc{Imbabura Highland Quichua}}\textbf{\ \ \ \ }Quechuan, \textit{Quechua II} (Ecuador)
\end{styleBody}

\begin{styleBody}
\textbf{References consulted: }Carpenter (1982), Cole (1982)
\end{styleBody}

\begin{styleBody}
\textbf{Sound inventory}
\end{styleBody}

\begin{styleBody}
\textbf{C phoneme inventory:} /p b t d k [261?] t[361?]s t[361?][283?] [278?] $\beta $ s z [283?] [292?] x m n [272?] l [27E?] w j/
\end{styleBody}

\begin{styleBody}
\textbf{N consonant phonemes:} 22
\end{styleBody}

\begin{styleBody}
\textbf{Geminates:} N/A
\end{styleBody}

\begin{styleBody}
\textbf{Voicing contrasts:} Obstruents
\end{styleBody}

\begin{styleBody}
\textbf{Places:} Bilabial, Alveolar, Palato-alveolar, Palatal, Velar
\end{styleBody}

\begin{styleBody}
\textbf{Manners:} Stop, Affricate, Fricative, Nasal, Flap/Tap, Central approximant, Lateral approximant
\end{styleBody}

\begin{styleBody}
\textbf{N elaborations:} 2
\end{styleBody}

\begin{styleBody}
\textbf{Elaborations:} Voiced fricatives/affricates, Palato-alveolar
\end{styleBody}

\begin{styleBody}
\textbf{V phoneme inventory:} /i a u/
\end{styleBody}

\begin{styleBody}
\textbf{N vowel qualities:} 3
\end{styleBody}

\begin{styleBody}
\textbf{Diphthongs or vowel sequences:} None
\end{styleBody}

\begin{styleBody}
\textbf{Contrastive length:} None
\end{styleBody}

\begin{styleBody}
\textbf{Contrastive nasalization:} None
\end{styleBody}

\begin{styleBody}
\textbf{Other contrasts:} N/A
\end{styleBody}

\begin{styleBody}
\textbf{Notes:} Carpenter gives series of aspirated stops, but Cole shows these have fricativized in Imbabura Quechua. /b d [261?] $\beta $ z/ are not indigenous but are now fully integrated/nativized (e.g., occur in suffixes). /r/ contrasts with flap in some dialect areas.
\end{styleBody}

\begin{styleBody}
\textbf{Syllable structure}
\end{styleBody}

\begin{styleBody}
\textbf{Complexity Category:} Moderately Complex
\end{styleBody}

\begin{styleBody}
\textbf{Canonical syllable structure:} (C)V(C)\textbf{ }(Cole 1982: 203-5)
\end{styleBody}

\begin{styleBody}
\textbf{Size of maximal onset:} 1
\end{styleBody}

\begin{styleBody}
\textbf{Size of maximal coda:} 1
\end{styleBody}

\begin{styleBody}
\textbf{Onset obligatory:} No
\end{styleBody}

\begin{styleBody}
\textbf{Coda obligatory:} No
\end{styleBody}

\begin{styleBody}
\textbf{Vocalic nucleus patterns:} Short vowels
\end{styleBody}

\begin{styleBody}
\textbf{Syllabic consonant patterns:} N/A
\end{styleBody}

\begin{styleBody}
\textbf{Size of maximal word{}-marginal sequences with syllabic obstruents:} N/A
\end{styleBody}

\begin{styleBody}
\textbf{Predictability of syllabic consonants:} N/A
\end{styleBody}

\begin{styleBody}
\textbf{Morphological constituency of maximal syllable margin:} N/A
\end{styleBody}

\begin{styleBody}
\textbf{Morphological pattern of syllabic consonants:} N/A
\end{styleBody}

\begin{styleBody}
\textbf{Onset restrictions:} All consonants occur.
\end{styleBody}

\begin{styleBody}
\textbf{Coda restrictions: }In native words, restricted to voiceless fricatives, liquids, and semivowels.
\end{styleBody}

\begin{styleBody}
\textbf{Suprasegmentals}
\end{styleBody}

\begin{styleBody}
\textbf{Tone:} No
\end{styleBody}

\begin{styleBody}
\textbf{Word stress:} Yes
\end{styleBody}

\begin{styleBody}
\textbf{Stress placement:} Fixed
\end{styleBody}

\begin{styleBody}
\textbf{Phonetic processes conditioned by stress:} Vowel Reduction
\end{styleBody}

\begin{styleBody}
\textbf{Differences in phonological properties of stressed and unstressed syllables:} Not described
\end{styleBody}

\begin{styleBody}
\textbf{Phonetic correlates of stress: }Pitch (impressionistic), Intensity (impressionistic)
\end{styleBody}

\begin{styleBody}
\textbf{Vowel reduction processes}
\end{styleBody}

\begin{styleBody}
\textbf{qvi-R1:} Vowels /i a u/ appear in lax form when unstressed (Jake 1983: 17; Cole 1982: 203 reports this for word-final unstressed vowels only).
\end{styleBody}

\begin{styleBody}
\textbf{Consonant allophony processes}
\end{styleBody}

\begin{styleBody}
\textbf{qvi-C1: }A voiced alveolar flap is realized as retroflex fricative [[290?]] word-initially (Cole 1982: 202).
\end{styleBody}

\begin{styleBody}
\textbf{qvi-C2: }A voiceless velar fricative may be realized as [[261?]] preceding a voiced consonant (Cole 1982: 201).
\end{styleBody}

\begin{styleBody}
\textbf{qvi-C3: }Voiceless stops and affricates are voiced following a nasal (Cole 1982: 200).
\end{styleBody}

\begin{styleBody}
\textbf{Morphology}
\end{styleBody}

\begin{styleBody}
\textbf{Text:} “Minkaymanta” (Carpenter 1982: 442-55) ****Ecuadorian dialect****
\end{styleBody}

\begin{styleBody}
\textbf{Synthetic index: }2.1 morphemes/word (206 morphemes, 97 words)
\end{styleBody}

\clearpage\begin{styleBody}
\textbf{[roo] }\ \ \textbf{\textsc{Rotokas}}\textbf{\ \ }North Bougainville, \textit{Rotokas-Askopan} (Papua New Guinea)
\end{styleBody}

\begin{styleBody}
\textbf{References consulted: }Firchow \& Firchow (1969), Firchow et al. (1973), Robinson (2006, 2011) 
\end{styleBody}

\begin{styleBody}
\textbf{Sound inventory}
\end{styleBody}

\begin{styleBody}
\textbf{C phoneme inventory:} /p t k [261?] $\beta $ [27E?]/
\end{styleBody}

\begin{styleBody}
\textbf{N consonant phonemes:} 6
\end{styleBody}

\begin{styleBody}
\textbf{Geminates:} N/A
\end{styleBody}

\begin{styleBody}
\textbf{Voicing contrasts:} Obstruents
\end{styleBody}

\begin{styleBody}
\textbf{Places:} Bilabial, Alveolar, Velar
\end{styleBody}

\begin{styleBody}
\textbf{Manners:} Stop, Fricative, Flap/Tap
\end{styleBody}

\begin{styleBody}
\textbf{N elaborations:} 1
\end{styleBody}

\begin{styleBody}
\textbf{Elaborations:} Voiced fricatives/affricates
\end{styleBody}

\begin{styleBody}
\textbf{V phoneme inventory:} /i e a o u i[2D0?] e[2D0?] a[2D0?] o[2D0?] u[2D0?]/
\end{styleBody}

\begin{styleBody}
\textbf{N vowel qualities:} 5
\end{styleBody}

\begin{styleBody}
\textbf{Diphthongs or vowel sequences:} None
\end{styleBody}

\begin{styleBody}
\textbf{Contrastive length:} All
\end{styleBody}

\begin{styleBody}
\textbf{Contrastive nasalization:} None
\end{styleBody}

\begin{styleBody}
\textbf{Other contrasts:} N/A
\end{styleBody}

\begin{styleBody}
\textbf{Notes:} Consonant phoneme representations are given as most characteristic allophonic realization. The contrast between voiceless, voiced, and nasal stops appears to have collapsed in Central Rotokas, producing this system, but these contrasts can still be found in Aita Rotokas (Robinson 2006).
\end{styleBody}

\begin{styleBody}
\textbf{Syllable structure}
\end{styleBody}

\begin{styleBody}
\textbf{Complexity Category:} Simple
\end{styleBody}

\begin{styleBody}
\textbf{Canonical syllable structure:} (C)V\textbf{ }(Robinson 2011: 28-9)
\end{styleBody}

\begin{styleBody}
\textbf{Size of maximal onset:} 1
\end{styleBody}

\begin{styleBody}
\textbf{Size of maximal coda:} N/A
\end{styleBody}

\begin{styleBody}
\textbf{Onset obligatory:} No
\end{styleBody}

\begin{styleBody}
\textbf{Coda obligatory:} N/A
\end{styleBody}

\begin{styleBody}
\textbf{Vocalic nucleus patterns:} Short vowels, Long vowels
\end{styleBody}

\begin{styleBody}
\textbf{Syllabic consonant patterns:} N/A
\end{styleBody}

\begin{styleBody}
\textbf{Size of maximal word{}-marginal sequences with syllabic obstruents:} N/A
\end{styleBody}

\begin{styleBody}
\textbf{Predictability of syllabic consonants:} N/A
\end{styleBody}

\begin{styleBody}
\textbf{Morphological constituency of maximal syllable margin:} N/A
\end{styleBody}

\begin{styleBody}
\textbf{Morphological pattern of syllabic consonants:} N/A
\end{styleBody}

\begin{styleBody}
\textbf{Onset restrictions: }All consonants occur.
\end{styleBody}

\begin{styleBody}
\textbf{Suprasegmentals}
\end{styleBody}

\begin{styleBody}
\textbf{Tone:} No
\end{styleBody}

\begin{styleBody}
\textbf{Word stress:} Yes
\end{styleBody}

\begin{styleBody}
\textbf{Stress placement:} Weight-Sensitive
\end{styleBody}

\begin{styleBody}
\textbf{Phonetic processes conditioned by stress:} (None)
\end{styleBody}

\begin{styleBody}
\textbf{Differences in phonological properties of stressed and unstressed syllables:} Not described
\end{styleBody}

\begin{styleBody}
\textbf{Phonetic correlates of stress: }Not described
\end{styleBody}

\begin{styleBody}
\textbf{Notes: }Firchow and Firchow (1969: 271) hint at interrelationship between length and stress, but do not elaborate.
\end{styleBody}

\begin{styleBody}
\textbf{Vowel reduction processes}
\end{styleBody}

\begin{styleBody}
(none reported)
\end{styleBody}

\begin{styleBody}
\textbf{Consonant allophony processes}
\end{styleBody}

\begin{styleBody}
\textbf{roo-C1: }A voiceless alveolar stop may be realized as an affricate preceding /i/ (Robinson 2011: 28).
\end{styleBody}

\begin{styleBody}
\textbf{roo-C2: }A voiced alveolar flap varies freely with [n], [l], [d] (Firchow \& Firchow 1969: 274).
\end{styleBody}

\begin{styleBody}
\textbf{roo-C3: }A voiced bilabial fricative varies freely with a voiced bilabial stop (Firchow \& Firchow 1969: 274).
\end{styleBody}

\begin{styleBody}
\textbf{roo-C4: }A voiceless alveolar stop may be realized as a fricative preceding /i/ (Robinson 2011: 28).
\end{styleBody}

\begin{styleBody}
\textbf{roo-C5: }A voiced velar stop may be spirantized medially (Firchow \& Firchow 1969: 274).
\end{styleBody}

\begin{styleBody}
\textbf{Morphology}
\end{styleBody}

\begin{styleBody}
\textbf{Text:} “Matevu (version 2)” (Robinson 2011: 293-304)
\end{styleBody}

\begin{styleBody}
\textbf{Synthetic index: }2.2 morphemes/word (642 morphemes, 293 words)
\end{styleBody}

\clearpage\begin{styleBody}
\textbf{[scs] }\ \ \textbf{\textsc{North Slavey (Hare dialect)}}\textbf{\ \ }Athabaskan-Eyak-Tlingit, \textit{Athabaskan-Eyak} (Canada)
\end{styleBody}

\begin{styleBody}
\textbf{References consulted: }Rice (1989, 2005)
\end{styleBody}

\begin{styleBody}
\textbf{Sound inventory}
\end{styleBody}

\begin{styleBody}
\textbf{C phoneme inventory:} /p t t[2B0?] k k[2B0?] k[2B7?] [294?] t’ k’ t[361?]s t[361?][26C?] t[361?][283?] t[361?]s’ t[361?][26C?]’ t[361?][283?]’ f s z [26C?] [26E?] [283?] [292?] x [263?] h m n [27E?] j w [294?]w/
\end{styleBody}

\begin{styleBody}
\textbf{N consonant phonemes:} 31
\end{styleBody}

\begin{styleBody}
\textbf{Geminates:} N/A
\end{styleBody}

\begin{styleBody}
\textbf{Voicing contrasts:} Obstruents
\end{styleBody}

\begin{styleBody}
\textbf{Places:} Bilabial, Labiodental, Alveolar, Palato-alveolar, Velar, Glottal
\end{styleBody}

\begin{styleBody}
\textbf{Manners:} Stop, Affricate, Fricative, Nasal, Flap/Tap, Central approximant, Lateral affricate, Lateral fricative
\end{styleBody}

\begin{styleBody}
\textbf{N elaborations:} 8
\end{styleBody}

\begin{styleBody}
\textbf{Elaborations:} Creaky voice, Voiced fricatives/affricates, Post-aspiration, Lateral release, Ejective, Labiodental, Palato-alveolar, Labialization
\end{styleBody}

\begin{styleBody}
\textbf{V phoneme inventory:} /i e [25B?] a o u \~{i} [25B?] \~{o}/
\end{styleBody}

\begin{styleBody}
\textbf{N vowel qualities:} 6
\end{styleBody}

\begin{styleBody}
\textbf{Diphthongs or vowel sequences:} None
\end{styleBody}

\begin{styleBody}
\textbf{Contrastive length:} None
\end{styleBody}

\begin{styleBody}
\textbf{Contrastive nasalization:} Some
\end{styleBody}

\begin{styleBody}
\textbf{Other contrasts:} N/A
\end{styleBody}

\begin{styleBody}
\textbf{Notes:} The distinction between /e/ and /[25B?]/ is contrastive in Hare. Nasal vowels are said to be derived by rules in Rice (1989), but Rice (2005) lists them as phonemic.
\end{styleBody}

\begin{styleBody}
\textbf{Syllable structure}
\end{styleBody}

\begin{styleBody}
\textbf{Category: }Moderately Complex
\end{styleBody}

\begin{styleBody}
\textbf{Canonical syllable structure: }(C)V(C) (Rice 1989: 143-53)
\end{styleBody}

\begin{styleBody}
\textbf{Size of maximal onset:} 1
\end{styleBody}

\begin{styleBody}
\textbf{Size of maximal coda:} 1
\end{styleBody}

\begin{styleBody}
\textbf{Onset obligatory:} No
\end{styleBody}

\begin{styleBody}
\textbf{Coda obligatory:} No
\end{styleBody}

\begin{styleBody}
\textbf{Vocalic nucleus patterns:} Short vowels, Long vowels
\end{styleBody}

\begin{styleBody}
\textbf{Syllabic consonant patterns:} N/A
\end{styleBody}

\begin{styleBody}
\textbf{Size of maximal word{}-marginal sequences with syllabic obstruents:} N/A
\end{styleBody}

\begin{styleBody}
\textbf{Predictability of syllabic consonants:} N/A
\end{styleBody}

\begin{styleBody}
\textbf{Morphological constituency of maximal syllable margin:} N/A
\end{styleBody}

\begin{styleBody}
\textbf{Morphological pattern of syllabic consonants:} N/A
\end{styleBody}

\begin{styleBody}
\textbf{Onset restrictions:} All consonants occur.
\end{styleBody}

\begin{styleBody}
\textbf{Coda restrictions: }Only [[294?] h j] occur, though others occur ‘underlyingly’.
\end{styleBody}

\begin{styleBody}
\textbf{Suprasegmentals}
\end{styleBody}

\begin{styleBody}
\textbf{Tone:} Yes
\end{styleBody}

\begin{styleBody}
\textbf{Word stress:} Yes
\end{styleBody}

\begin{styleBody}
\textbf{Stress placement:} Morphologically or Lexically Conditioned
\end{styleBody}

\begin{styleBody}
\textbf{Phonetic processes conditioned by stress:} (None)
\end{styleBody}

\begin{styleBody}
\textbf{Differences in phonological properties of stressed and unstressed syllables:} (None)
\end{styleBody}

\begin{styleBody}
\textbf{Phonetic correlates of stress: }Vowel duration (impressionistic), Pitch (impressionistic), Intensity (impressionistic)
\end{styleBody}

\begin{styleBody}
\textbf{Notes:} Predictable stress-like properties occur, falling on V immediately preceding stem of verb or stem vowel of noun. A high tone on a vowel already bearing a high tone for some other reason gives the syllable extra prominence by increasing the pitch (Rice 2005: 362).
\end{styleBody}

\begin{styleBody}
\textbf{Vowel reduction processes}
\end{styleBody}

\begin{styleBody}
(none reported)
\end{styleBody}

\begin{styleBody}
\textbf{Consonant allophony processes}
\end{styleBody}

\begin{styleBody}
\textbf{scs-C1: }Velar fricatives are labialized preceding a rounded vowel (Rice 1989: 31).
\end{styleBody}

\begin{styleBody}
\textbf{scs-C2: }Plain velar stops and the voiceless velar fricative are realized as palatal preceding a front vowel (Rice 1989: 31).
\end{styleBody}

\begin{styleBody}
\textbf{scs-C3: }Ejectives and plain consonants may be voiced intervocalically (Rice 1989: 31).
\end{styleBody}

\begin{styleBody}
\textbf{scs-C4: }Voiceless palato-alveolar affricate may vary freely with [[283?]] (Rice 1989: 35).
\end{styleBody}

\begin{styleBody}
\textbf{scs-C5: }A voiceless velar fricative varies freely with a glottal fricative (Rice 1989: 32).
\end{styleBody}

\begin{styleBody}
\textbf{scs-C6: }A voiced velar fricative may be realized as a labiovelar approximant preceding a round vowel (Rice 1989: 32).
\end{styleBody}

\begin{styleBody}
\textbf{Morphology}
\end{styleBody}

\begin{styleBody}
(adequate texts unavailable)
\end{styleBody}

\clearpage\begin{styleBody}
\textbf{[sea] }\ \ \textbf{\textsc{Semai\ \ }}\textbf{\ \ }Austroasiatic, \textit{Aslian} (Malaysia)
\end{styleBody}

\begin{styleBody}
\textbf{References consulted: }Dentan (2003), Diffloth (1976a, b), Philips (2007), Sloan (1988), Sylvia Tufvesson (p.c.)
\end{styleBody}

\begin{styleBody}
\textbf{Sound inventory}
\end{styleBody}

\begin{styleBody}
\textbf{C phoneme inventory:} /p b t d c [25F?] k [261?] [294?] m n [272?] ŋ ç h [27E?] l w j/
\end{styleBody}

\begin{styleBody}
\textbf{N consonant phonemes:} 19
\end{styleBody}

\begin{styleBody}
\textbf{Geminates:} N/A
\end{styleBody}

\begin{styleBody}
\textbf{Voicing contrasts:} Obstruents
\end{styleBody}

\begin{styleBody}
\textbf{Places:} Bilabial, Alveolar, Palatal, Velar, Glottal
\end{styleBody}

\begin{styleBody}
\textbf{Manners:} Stop, Fricative, Nasal, Flap/Tap, Central approximant, Lateral approximant
\end{styleBody}

\begin{styleBody}
\textbf{N elaborations:} 0
\end{styleBody}

\begin{styleBody}
\textbf{Elaborations:} N/A
\end{styleBody}

\begin{styleBody}
\textbf{V phoneme inventory:} /i [25B?] [259?] [251?] [254?] u i[2D0?] e[2D0?] [25B?][2D0?] [268?][2D0?] [251?][2D0?] [254?][2D0?] o[2D0?] u[2D0?] \~{i} [25B?] [259?] [251?] [254?] \~{u} \~{i}[2D0?] [25B?][2D0?] [268?][2D0?] [251?][2D0?] [254?][2D0?] \~{u}[2D0?]/
\end{styleBody}

\begin{styleBody}
\textbf{N vowel qualities:} 9
\end{styleBody}

\begin{styleBody}
\textbf{Diphthongs or vowel sequences:} None
\end{styleBody}

\begin{styleBody}
\textbf{Contrastive length:} Some
\end{styleBody}

\begin{styleBody}
\textbf{Contrastive nasalization:} Some
\end{styleBody}

\begin{styleBody}
\textbf{Other contrasts:} N/A
\end{styleBody}

\begin{styleBody}
\textbf{Notes:} /e[2D0?] [268?][2D0?] o[2D0?]/ occur only long; there is contrastive nasalization for all but these.
\end{styleBody}

\begin{styleBody}
\textbf{Syllable structure}
\end{styleBody}

\begin{styleBody}
\textbf{Complexity Category: }Highly Complex 
\end{styleBody}

\begin{styleBody}
\textbf{Canonical syllable structure:} C(C)V(C)\textbf{ }(Diffloth 1976a, 1976b; Sloan 1988; Philips 2007)
\end{styleBody}

\begin{styleBody}
\textbf{Size of maximal onset:} 2
\end{styleBody}

\begin{styleBody}
\textbf{Size of maximal coda:} 1
\end{styleBody}

\begin{styleBody}
\textbf{Onset obligatory:} Yes
\end{styleBody}

\begin{styleBody}
\textbf{Coda obligatory:} No
\end{styleBody}

\begin{styleBody}
\textbf{Vocalic nucleus patterns:} Short vowels, Long vowels
\end{styleBody}

\begin{styleBody}
\textbf{Syllabic consonant patterns:} Nasal, Liquid, Obstruent
\end{styleBody}

\begin{styleBody}
\textbf{Size of maximal word{}-marginal sequences with syllabic obstruents:} 4 (initial)
\end{styleBody}

\begin{styleBody}
\textbf{Predictability of syllabic consonants:} Predictable from word/consonantal context
\end{styleBody}

\begin{styleBody}
\textbf{Morphological constituency of maximal syllable margin:} Morpheme-internal (Onset)
\end{styleBody}

\begin{styleBody}
\textbf{Morphological pattern of syllabic consonants:} Grammatical items
\end{styleBody}

\begin{styleBody}
\textbf{Onset restrictions:} All consonants may occur in simple onsets. Examples of biconsonantal onsets include /dn [261?]h c[27E?]/.
\end{styleBody}

\begin{styleBody}
\textbf{Coda restrictions: }All consonants except voiced stops may occur.
\end{styleBody}

\begin{styleBody}
\textbf{Notes: }Philips (2007) reports CVC maximum for both major and minor syllables, with optional elision of /[259?]/ in minor syllable. However, both Diffloth and Sloan report that some roots have two initial consonants before the main vowel. Sloan’s data (from Diffloth) includes reduplicated (expressive) forms, in which C\textsubscript{1}(V)C\textsubscript{2}VC\textsubscript{3} {\textgreater} C\textsubscript{1}C\textsubscript{3}C\textsubscript{1}C\textsubscript{2}VC\textsubscript{3}, e.g. dŋ[254?]h {\textgreater} dhdŋ[254?]h. Sloan takes canonical major syllable to be C(C)V(V)(C), with minor syllables being of the shape C or CC, where obstruents such as /p/ or /c/, as well as more sonorous consonants, may occur (1988: 320-1). This results in word-initial sequences such as /sts/, /krk/, /pnpr/, and larger strings as in \textit{kckmr[294?][25B?][2D0?]c} ‘short, fat arms’, syllabified /kc.km.r.[294?][25B?][2D0?]c/. In the Kampar dialect, long sequences of up to five consonants may occur in intensification constructions formed through reduplication, and some roots have word-initial sequences of three consonants (Sylvia Tufvesson, p.c.). It appears that the longest string of obstruents word-initially is 4: /gpgh/. Much like syllables without vowels in Tashlhiyt, Semai minor syllables “are clearly heard and perceived as distinct syllables” (1988: 320). Because the unusual syllable patterns of this language produce strings of 3 obstruents or greater word-initially, I classify this language as Highly Complex.
\end{styleBody}

\begin{styleBody}
\textbf{Suprasegmentals}
\end{styleBody}

\begin{styleBody}
\textbf{Tone:} No
\end{styleBody}

\begin{styleBody}
\textbf{Word stress:} Yes
\end{styleBody}

\begin{styleBody}
\textbf{Stress placement:} Fixed
\end{styleBody}

\begin{styleBody}
\textbf{Phonetic processes conditioned by stress:} (None)
\end{styleBody}

\begin{styleBody}
\textbf{Differences in phonological properties of stressed and unstressed syllables:} Vowel Quality Contrasts, Vowel Length Contrasts
\end{styleBody}

\begin{styleBody}
\textbf{Phonetic correlates of stress: }Not described
\end{styleBody}

\begin{styleBody}
\textbf{Vowel reduction processes}
\end{styleBody}

\begin{styleBody}
\textbf{sea-R1:} Preceding glottal consonants, some long vowels (/e[2D0?] [268?][2D0?] o[2D0?]/) are produced as short (Philips 2007: 10-11).
\end{styleBody}

\begin{styleBody}
\textbf{Consonant allophony processes}
\end{styleBody}

\begin{styleBody}
\textbf{sea-C1: }Palatal stops have affricated release syllable initially (Philips 2007: 5).
\end{styleBody}

\begin{styleBody}
\textbf{Morphology}
\end{styleBody}

\begin{styleBody}
(adequate texts unavailable)
\end{styleBody}

\clearpage\begin{styleBody}
\textbf{[shi] }\ \ \textbf{\textsc{Tashlhiyt}}\textbf{\ \ }Afro-Asiatic, \textit{Berber} (Morocco)
\end{styleBody}

\begin{styleBody}
\textbf{References consulted: }Coleman (2001), Dell \& Elmedlaoui (2002), Gordon \& Nafi (2012), Louali \& Puech (1999), Ridouane (2002, 2007, 2008, 2014), Roettger et al. (2015)
\end{styleBody}

\begin{styleBody}
\textbf{Sound inventory}
\end{styleBody}

\begin{styleBody}
\textbf{C phoneme inventory:} /b t d t[2E4?] d[2E4?] k [261?] k[2B7?] [261?][2B7?] f s s[2E4?] [283?] $\chi $ $\chi [2B7?]$ [29C?] h z z[2E4?] [292?] [281?] [281?][2B7?] [2A2?] m n w l l[2E4?] r r[2E4?] j/
\end{styleBody}

\begin{styleBody}
\textbf{N consonant phonemes: }34
\end{styleBody}

\begin{styleBody}
\textbf{Geminates:} /b[2D0?] t[2D0?] d[2D0?] t[2E4?][2D0?] d[2E4?][2D0?] k[2D0?] [261?][2D0?] k[2B7?][2D0?] [261?][2B7?][2D0?] q[2D0?] q[2B7?][2D0?] f[2D0?] s[2D0?] s[2E4?][2D0?] [283?][2D0?] $\chi [2D0?]$ $\chi [2B7?][2D0?]$ [29C?][2D0?] h[2D0?] z[2D0?] z[2E4?][2D0?] [292?][2D0?] [292?][2D0?][2E4?] [281?][2D0?] [281?][2B7?][2D0?] [2A2?][2D0?] m[2D0?] n[2D0?] w[2D0?] l[2D0?] l[2E4?][2D0?] r[2D0?] r[2E4?][2D0?] j[2D0?]/ (All, including some that don’t have singleton counterparts)
\end{styleBody}

\begin{styleBody}
\textbf{Voicing contrasts:} Obstruents
\end{styleBody}

\begin{styleBody}
\textbf{Places:} Bilabial, Labiodental, Dental, Palato-alveolar, Velar, Uvular, Pharyngeal, Glottal
\end{styleBody}

\begin{styleBody}
\textbf{Manners:} Stop, Fricative, Nasal, Trill, Central approximant, Lateral approximant
\end{styleBody}

\begin{styleBody}
\textbf{N elaborations:} 7
\end{styleBody}

\begin{styleBody}
\textbf{Elaborations:} Voiced fricatives/affricates, Labiodental, Palato-alveolar, Uvular, Pharyngeal, Labialization, Pharyngealization
\end{styleBody}

\begin{styleBody}
\textbf{V phoneme inventory:} /i a u/
\end{styleBody}

\begin{styleBody}
\textbf{N vowel qualities:} 3
\end{styleBody}

\begin{styleBody}
\textbf{Diphthongs or vowel sequences:} None
\end{styleBody}

\begin{styleBody}
\textbf{Contrastive length:} None
\end{styleBody}

\begin{styleBody}
\textbf{Contrastive nasalization:} None
\end{styleBody}

\begin{styleBody}
\textbf{Other contrasts:} N/A
\end{styleBody}

\begin{styleBody}
\textbf{Notes:} All consonants have short/long counterparts except for /q[2D0?] q[2B7?][2D0?] [292?][2D0?][2E4?]/, which are analyzed by Ridouane as long. /n[2E4?] [283?][2E4?]/ are extremely marginal, according to Ridouane.
\end{styleBody}

\begin{styleBody}
\textbf{Syllable structure}
\end{styleBody}

\begin{styleBody}
\textbf{Complexity Category:} Highly Complex
\end{styleBody}

\begin{styleBody}
\textbf{Canonical syllable structure: }(C)V(C)\textbf{ }(Dell \& Elmedlouai 2002, Ridouane 2008)
\end{styleBody}

\begin{styleBody}
\textbf{Size of maximal onset:} 1
\end{styleBody}

\begin{styleBody}
\textbf{Size of maximal coda:} 1
\end{styleBody}

\begin{styleBody}
\textbf{Onset obligatory:} No
\end{styleBody}

\begin{styleBody}
\textbf{Coda obligatory:} No
\end{styleBody}

\begin{styleBody}
\textbf{Vocalic nucleus patterns:} Short vowels, Long vowels
\end{styleBody}

\begin{styleBody}
\textbf{Syllabic consonant patterns:} Nasal, Liquid, Obstruent
\end{styleBody}

\begin{styleBody}
\textbf{Size of maximal word{}-marginal sequences with syllabic obstruents:} ? (words without vowels)
\end{styleBody}

\begin{styleBody}
\textbf{Predictability of syllabic consonants:} Predictable from word/consonantal context
\end{styleBody}

\begin{styleBody}
\textbf{Morphological constituency of maximal syllable margin:} N/A
\end{styleBody}

\begin{styleBody}
\textbf{Morphological pattern of syllabic consonants:} Both
\end{styleBody}

\begin{styleBody}
\textbf{Onset restrictions: }None for simple onsets.
\end{styleBody}

\begin{styleBody}
\textbf{Coda restrictions:} Unclear.
\end{styleBody}

\begin{styleBody}
\textbf{Notes: }Dell \& Elmedlaoui propose (C)V(C) structure with consonantal nuclei allowed. Puech \& Louali (1999) present experimental acoustic and perceptual data which suggest biconsonantal onsets, at least; Ridouane (2008) argues against this using a variety of phonetic experiments and phonological processes. Regardless of the analysis of syllable structure, the phonetic patterns have long sequences of consonants (including obstruents) word-initially, medially, and finally, and therefore fits our definition of Highly Complex.
\end{styleBody}

\begin{styleBody}
\textbf{Suprasegmentals}
\end{styleBody}

\begin{styleBody}
\textbf{Tone:} No
\end{styleBody}

\begin{styleBody}
\textbf{Word stress:} No
\end{styleBody}

\begin{styleBody}
\textbf{Stress placement:} Fixed
\end{styleBody}

\begin{styleBody}
\textbf{Phonetic processes conditioned by stress:} (None)
\end{styleBody}

\begin{styleBody}
\textbf{Differences in phonological properties of stressed and unstressed syllables:} (None)
\end{styleBody}

\begin{styleBody}
\textbf{Phonetic correlates of stress: }Vowel duration (instrumental), Intensity (instrumental)
\end{styleBody}

\begin{styleBody}
\textbf{Notes: }Gordon \& Nafi (2012) argue for word-level stress, but Roettger et al. (2015) show evidence that it is a phrase-level phenomenon.
\end{styleBody}

\begin{styleBody}
\textbf{Vowel reduction processes}
\end{styleBody}

\begin{styleBody}
\textbf{shi-R1}: Vowels are shortened preceding a geminate consonant (Dell \& Elmedlouai 2002).
\end{styleBody}

\begin{styleBody}
\textbf{Consonant allophony processes}
\end{styleBody}

\begin{styleBody}
(none reported)
\end{styleBody}

\begin{styleBody}
\textbf{Morphology}
\end{styleBody}

\begin{styleBody}
\textbf{Text:} “The north wind and the sun” (Ridouane 2014: 219)
\end{styleBody}

\begin{styleBody}
\textbf{Synthetic index: }1.9 morphemes/word (144 morphemes, 76 words)
\end{styleBody}

\clearpage\begin{styleBody}
\textbf{[spl] }\ \ \textbf{\textsc{Selepet}}\textbf{\ \ }Nuclear Trans New Guinea, \textit{Finisterre-Huon} (Papua New Guinea)
\end{styleBody}

\begin{styleBody}
\textbf{References consulted: }McElhanon (1970)
\end{styleBody}

\begin{styleBody}
\textbf{Sound inventory}
\end{styleBody}

\begin{styleBody}
\textbf{C phoneme inventory:} /p \textsuperscript{m}b t[32A?] \textsuperscript{n[32A?]}d[32A?] k \textsuperscript{ŋ}[261?] s h m n[32A?] ŋ l [27E?] w j/
\end{styleBody}

\begin{styleBody}
\textbf{N consonant phonemes:} 15
\end{styleBody}

\begin{styleBody}
\textbf{Geminates:} N/A
\end{styleBody}

\begin{styleBody}
\textbf{Voicing contrasts:} Obstruents
\end{styleBody}

\begin{styleBody}
\textbf{Places:} Bilabial, Dental, Alveolar, Velar, Glottal
\end{styleBody}

\begin{styleBody}
\textbf{Manners:} Stops, Fricative, Nasal, Flap/Tap, Central approximant, Lateral approximant
\end{styleBody}

\begin{styleBody}
\textbf{N elaborations:} 1
\end{styleBody}

\begin{styleBody}
\textbf{Elaborations:} Prenasalization
\end{styleBody}

\begin{styleBody}
\textbf{V phoneme inventory:} /i e a [254?] o u/
\end{styleBody}

\begin{styleBody}
\textbf{N vowel qualities:} 6
\end{styleBody}

\begin{styleBody}
\textbf{Diphthongs or vowel sequences:} Diphthongs /ii ie ia i[254?] io iu ei eu ai ae ao au [254?]i [254?]e [254?]o [254?]u oi oe ou ui ue ua u[254?] uo uu/
\end{styleBody}

\begin{styleBody}
\textbf{Contrastive length:} None
\end{styleBody}

\begin{styleBody}
\textbf{Contrastive nasalization:} None
\end{styleBody}

\begin{styleBody}
\textbf{Other contrasts:} N/A
\end{styleBody}

\begin{styleBody}
\textbf{Notes:} Prenasalization in voiced stops is weak in initial position, but always prenasalized intervocalically, so I take this variant to be phoneme label. No significant difference in length between simple and complex nuclei.
\end{styleBody}

\begin{styleBody}
\textbf{Syllable structure}
\end{styleBody}

\begin{styleBody}
\textbf{Complexity Category:} Moderately Complex
\end{styleBody}

\begin{styleBody}
\textbf{Canonical syllable structure: }(C)V(C) (McElhanon 1970: 14-18)
\end{styleBody}

\begin{styleBody}
\textbf{Size of maximal onset:} 1
\end{styleBody}

\begin{styleBody}
\textbf{Size of maximal coda:} 1
\end{styleBody}

\begin{styleBody}
\textbf{Onset obligatory:} No
\end{styleBody}

\begin{styleBody}
\textbf{Coda obligatory:} No
\end{styleBody}

\begin{styleBody}
\textbf{Vocalic nucleus patterns:} Short vowels, Vowel sequences
\end{styleBody}

\begin{styleBody}
\textbf{Syllabic consonant patterns:} N/A
\end{styleBody}

\begin{styleBody}
\textbf{Size of maximal word{}-marginal sequences with syllabic obstruents:} N/A
\end{styleBody}

\begin{styleBody}
\textbf{Predictability of syllabic consonants:} N/A
\end{styleBody}

\begin{styleBody}
\textbf{Morphological constituency of maximal syllable margin:} N/A
\end{styleBody}

\begin{styleBody}
\textbf{Morphological pattern of syllabic consonants:} N/A
\end{styleBody}

\begin{styleBody}
\textbf{Onset restrictions: }All consonants occur.
\end{styleBody}

\begin{styleBody}
\textbf{Coda restrictions:} Only voiceless stops and nasals occur.
\end{styleBody}

\begin{styleBody}
\textbf{Suprasegmentals}
\end{styleBody}

\begin{styleBody}
\textbf{Tone:} Not reported
\end{styleBody}

\begin{styleBody}
\textbf{Word stress:} Yes
\end{styleBody}

\begin{styleBody}
\textbf{Stress placement:} Fixed
\end{styleBody}

\begin{styleBody}
\textbf{Phonetic processes conditioned by stress:} Vowel Reduction
\end{styleBody}

\begin{styleBody}
\textbf{Differences in phonological properties of stressed and unstressed syllables:} (None)
\end{styleBody}

\begin{styleBody}
\textbf{Phonetic correlates of stress: }Intensity (impressionistic)
\end{styleBody}

\begin{styleBody}
\textbf{Vowel reduction processes}
\end{styleBody}

\begin{styleBody}
\textbf{spl-R1}: Unstressed syllables or vowels in postnuclear slope (in phrases) tend to elision nearer the nucleus (McElhanon 1970: 6).
\end{styleBody}

\begin{styleBody}
\textbf{spl-R2: }Length of syllables or vowels decreases nearer the nucleus in the prenuclear slope (in phrases) (McElhanon 1970: 6).
\end{styleBody}

\begin{styleBody}
\textbf{Consonant allophony processes}
\end{styleBody}

\begin{styleBody}
\textbf{spl-C1: }A palatal glide may be realized as a palatal alveolar fricative [z[2B2?]]\~{}[s[2B2?]] word-initially preceding /i/ (McElhanon 1970).
\end{styleBody}

\begin{styleBody}
\textbf{spl-C2: }A palatal glide may be realized as a palatal alveolar fricative [z[2B2?]]\~{}[s[2B2?]] when following a consonant (McElhanon 1970).
\end{styleBody}

\begin{styleBody}
\textbf{Morphology}
\end{styleBody}

\begin{styleBody}
(adequate texts unavailable)
\end{styleBody}

\clearpage\begin{styleBody}
\textbf{[svs] }\ \ \textbf{\textsc{Savosavo}}\textbf{\ \ }isolate (Solomon Islands)
\end{styleBody}

\begin{styleBody}
\textbf{References consulted: }Wegener (2008)
\end{styleBody}

\begin{styleBody}
\textbf{Sound inventory}
\end{styleBody}

\begin{styleBody}
\textbf{C phoneme inventory:} /p \textsuperscript{m}b t \textsuperscript{n}d \textsuperscript{[272?]}[25F?] k \textsuperscript{ŋ}[261?] m n [272?] ŋ s z r l $\beta [31E?]$ [270?]/
\end{styleBody}

\begin{styleBody}
\textbf{N consonant phonemes:} 17
\end{styleBody}

\begin{styleBody}
\textbf{Geminates:} N/A
\end{styleBody}

\begin{styleBody}
\textbf{Voicing contrasts:} Obstruent
\end{styleBody}

\begin{styleBody}
\textbf{Places:} Bilabials, Alveolar, Palatal, Velar
\end{styleBody}

\begin{styleBody}
\textbf{Manners:} Stop, Fricative, Nasal, Trill, Central Approximant, Lateral Approximant
\end{styleBody}

\begin{styleBody}
\textbf{N elaborations:} 2
\end{styleBody}

\begin{styleBody}
\textbf{Elaborations:} Voiced fricatives/affricates, Prenasalization
\end{styleBody}

\begin{styleBody}
\textbf{V phoneme inventory:} /i e a o u/
\end{styleBody}

\begin{styleBody}
\textbf{N vowel qualities:} 5
\end{styleBody}

\begin{styleBody}
\textbf{Diphthongs or vowel sequences:} Diphthong /ai/
\end{styleBody}

\begin{styleBody}
\textbf{Contrastive length:} None
\end{styleBody}

\begin{styleBody}
\textbf{Contrastive nasalization:} None
\end{styleBody}

\begin{styleBody}
\textbf{Other contrasts:} None
\end{styleBody}

\begin{styleBody}
\textbf{Notes:} Prenasalized stops have both plain voiced and prenasalized allophones, but latter much more frequent. /ai/ combination is diphthong in some cases, disyllabic vowel sequence in others (Wegener 2008: 22).
\end{styleBody}

\begin{styleBody}
\textbf{Syllable structure}
\end{styleBody}

\begin{styleBody}
\textbf{Complexity Category:} Simple
\end{styleBody}

\begin{styleBody}
\textbf{Canonical syllable structure:} (C)V\textbf{ }(Wegener 2008: 23-4)
\end{styleBody}

\begin{styleBody}
\textbf{Size of maximal onset:} 1
\end{styleBody}

\begin{styleBody}
\textbf{Size of maximal coda:} N/A
\end{styleBody}

\begin{styleBody}
\textbf{Onset obligatory:} No
\end{styleBody}

\begin{styleBody}
\textbf{Coda obligatory:} N/A
\end{styleBody}

\begin{styleBody}
\textbf{Vocalic nucleus patterns:} Short vowels
\end{styleBody}

\begin{styleBody}
\textbf{Syllabic consonant patterns:} N/A
\end{styleBody}

\begin{styleBody}
\textbf{Size of maximal word{}-marginal sequences with syllabic obstruents:} N/A
\end{styleBody}

\begin{styleBody}
\textbf{Predictability of syllabic consonants:} N/A
\end{styleBody}

\begin{styleBody}
\textbf{Morphological constituency of maximal syllable margin:} N/A
\end{styleBody}

\begin{styleBody}
\textbf{Morphological pattern of syllabic consonants:} N/A
\end{styleBody}

\begin{styleBody}
\textbf{Onset restrictions:} All consonants occur.
\end{styleBody}

\begin{styleBody}
\textbf{Suprasegmentals}
\end{styleBody}

\begin{styleBody}
\textbf{Tone:} No
\end{styleBody}

\begin{styleBody}
\textbf{Word stress:} Yes
\end{styleBody}

\begin{styleBody}
\textbf{Stress placement:} Morphologically or Lexically Conditioned
\end{styleBody}

\begin{styleBody}
\textbf{Phonetic processes conditioned by stress:} (None)
\end{styleBody}

\begin{styleBody}
\textbf{Differences in phonological properties of stressed and unstressed syllables:} (None)
\end{styleBody}

\begin{styleBody}
\textbf{Phonetic correlates of stress: }Vowel duration (impressionistic), Pitch (instrumental), Intensity (instrumental)
\end{styleBody}

\begin{styleBody}
\textbf{Notes: }Only pitch and intensity instrumentally confirmed as correlates of stress, and then only qualitatively in Praat. Pitch only sometimes a correlate of stress (p. 24).
\end{styleBody}

\begin{styleBody}
\textbf{Vowel reduction processes}
\end{styleBody}

\begin{styleBody}
(none reported)
\end{styleBody}

\begin{styleBody}
\textbf{Consonant allophony processes}
\end{styleBody}

\begin{styleBody}
\textbf{svs-C1: }An alveolar trill may be realized as a flap in rapid speech (Wegener 2010: 17).
\end{styleBody}

\begin{styleBody}
\textbf{Morphology}
\end{styleBody}

\begin{styleBody}
\textbf{Text:} “Koi polupolu” (lines 1-46, Wegener 2008: 331-336)
\end{styleBody}

\begin{styleBody}
\textbf{Synthetic index: }1.6 morphemes/word (630 morphemes, 396 words)
\end{styleBody}

\clearpage\begin{styleBody}
\textbf{[sxr] }\ \ \textbf{\textsc{Saaroa\ \ }}\textbf{\ \ }Austronesian, \textit{Tsouic} (Taiwan)
\end{styleBody}

\begin{styleBody}
\textbf{References consulted: }Pan (2012)
\end{styleBody}

\begin{styleBody}
\textbf{Sound inventory}
\end{styleBody}

\begin{styleBody}
\textbf{C phoneme inventory:} /p t k [294?] t[361?]s v s m n ŋ [27E?] r [26C?]/
\end{styleBody}

\begin{styleBody}
\textbf{N consonant phonemes:} 13
\end{styleBody}

\begin{styleBody}
\textbf{Geminates:} N/A
\end{styleBody}

\begin{styleBody}
\textbf{Voicing contrasts: }None
\end{styleBody}

\begin{styleBody}
\textbf{Places: }Bilabial, Labiodental, Alveolar, Velar, Glottal
\end{styleBody}

\begin{styleBody}
\textbf{Manners:} Stop, Affricate, Fricative, Nasal, Flap/tap, Trill, Lateral fricative
\end{styleBody}

\begin{styleBody}
\textbf{N elaborations:} 2
\end{styleBody}

\begin{styleBody}
\textbf{Elaborations:} Voiced fricatives/affricates, Labiodental
\end{styleBody}

\begin{styleBody}
\textbf{V phoneme inventory:} /i [268?] a u i[2D0?] [268?][2D0?] a[2D0?] u[2D0?]/
\end{styleBody}

\begin{styleBody}
\textbf{N vowel qualities:} 4
\end{styleBody}

\begin{styleBody}
\textbf{Diphthongs or vowel sequences:} None
\end{styleBody}

\begin{styleBody}
\textbf{Contrastive length:} All
\end{styleBody}

\begin{styleBody}
\textbf{Contrastive nasalization:} None
\end{styleBody}

\begin{styleBody}
\textbf{Other contrasts:} N/A
\end{styleBody}

\begin{styleBody}
\textbf{Notes:} /b t[361?][255?] d[361?]z [261?]/ occur in Japanese loans, /p[2B0?] t[2B0?] t[361?]s[2B0?]/ in Mandarin loans, /k[2B0?]/ in Southern Min loans, and /h/ in Bunun loans.
\end{styleBody}

\begin{styleBody}
\textbf{Syllable structure}
\end{styleBody}

\begin{styleBody}
\textbf{Complexity Category:} Simple
\end{styleBody}

\begin{styleBody}
\textbf{Canonical syllable structure:} (C)V(C) (Pan 2012: 32-33)
\end{styleBody}

\begin{styleBody}
\textbf{Size of maximal onset:} 1
\end{styleBody}

\begin{styleBody}
\textbf{Size of maximal coda: }1
\end{styleBody}

\begin{styleBody}
\textbf{Onset obligatory:} No
\end{styleBody}

\begin{styleBody}
\textbf{Coda obligatory:} N/A
\end{styleBody}

\begin{styleBody}
\textbf{Vocalic nucleus patterns: }Short vowels, Long vowels
\end{styleBody}

\begin{styleBody}
\textbf{Syllabic consonant patterns:} N/A
\end{styleBody}

\begin{styleBody}
\textbf{Size of maximal word{}-marginal sequences with syllabic obstruents:} N/A
\end{styleBody}

\begin{styleBody}
\textbf{Predictability of syllabic consonants:} N/A
\end{styleBody}

\begin{styleBody}
\textbf{Morphological constituency of maximal syllable margin:} N/A
\end{styleBody}

\begin{styleBody}
\textbf{Morphological pattern of syllabic consonants:} N/A
\end{styleBody}

\begin{styleBody}
\textbf{Onset restrictions:} None
\end{styleBody}

\begin{styleBody}
\textbf{Coda restrictions:} /m n/
\end{styleBody}

\begin{styleBody}
\textbf{Notes:} There are only two native words which have invariant codas, one word-final and one word-medial (Pan 2012: 32f).
\end{styleBody}

\begin{styleBody}
\textbf{Suprasegmentals}
\end{styleBody}

\begin{styleBody}
\textbf{Tone:} No
\end{styleBody}

\begin{styleBody}
\textbf{Word stress: }Yes
\end{styleBody}

\begin{styleBody}
\textbf{Stress placement:} Fixed
\end{styleBody}

\begin{styleBody}
\textbf{Phonetic processes conditioned by stress:} Vowel Reduction
\end{styleBody}

\begin{styleBody}
\textbf{Differences in phonological properties of stressed and unstressed syllables:} (None)
\end{styleBody}

\begin{styleBody}
\textbf{Phonetic correlates of stress: }Pitch (impressionistic), Intensity (impressionistic)
\end{styleBody}

\begin{styleBody}
\textbf{Vowel reduction processes}
\end{styleBody}

\begin{styleBody}
\textbf{sxr-R1: }Unstressed high vowels /i [268?] u/ are typically deleted in normal and rapid speech word-finally following /ŋ/ (Pan 2012: 38).
\end{styleBody}

\begin{styleBody}
\textbf{sxr-R2: }Unstressed high central vowel /[268?]/ is typically deleted word-finally following /m/ (Pan 2012: 39).
\end{styleBody}

\begin{styleBody}
\textbf{Consonant allophony processes}
\end{styleBody}

\begin{styleBody}
\textbf{sxr-C1:} Alveolar fricative and affricate /s t[361?]s/ are realized as palato-alveolar preceding high front vowel /i/ (Pan 2012: 28).
\end{styleBody}

\begin{styleBody}
\textbf{Morphology}
\end{styleBody}

\begin{styleBody}
\textbf{Text:} “Extract from text 1: Introducing myself and my children,” “Extract from test 2: Daily life of the past,” “Extract from text 3: How to make a mat,” “Extract from text 4: How to make sticky rice cakes” (Pan 2012: 365-372)
\end{styleBody}

\begin{styleBody}
\textbf{Synthetic index: }1.7 morphemes/word (493 morphemes, 288 words)
\end{styleBody}

\clearpage\begin{styleBody}
\textbf{[tbi] }\ \ \textbf{\textsc{Gaam}}\textbf{\ \ }Eastern Jebel (Sudan)
\end{styleBody}

\begin{styleBody}
\textbf{References consulted: }Bender (1983), Crewe (1975), Stirtz (2011)
\end{styleBody}

\begin{styleBody}
\textbf{Sound inventory}
\end{styleBody}

\begin{styleBody}
\textbf{C phoneme inventory:} /p b t[32A?] d[32A?] t d c [25F?] k [261?] f s m n [272?] ŋ l [27E?] w ð[31E?] j/
\end{styleBody}

\begin{styleBody}
\textbf{N consonant phonemes:} 21
\end{styleBody}

\begin{styleBody}
\textbf{Geminates:} /f[2D0?] s[2D0?] m[2D0?] n[2D0?] [272?][2D0?] ŋ[2D0?] l[2D0?] r[2D0?]/ (Some)
\end{styleBody}

\begin{styleBody}
\textbf{Voicing contrasts:} Obstruents
\end{styleBody}

\begin{styleBody}
\textbf{Places:} Bilabial, Labiodental, Dental, Alveolar, Palatal, Velar
\end{styleBody}

\begin{styleBody}
\textbf{Manners:} Stop, Fricative, Nasal, Flap/Tap, Central approximant, Lateral approximant
\end{styleBody}

\begin{styleBody}
\textbf{N elaborations:} 1
\end{styleBody}

\begin{styleBody}
\textbf{Elaborations:} Labiodental
\end{styleBody}

\begin{styleBody}
\textbf{V phoneme inventory:} /i [25B?] [259?] a [254?] u i[2D0?] [25B?][2D0?] [259?][2D0?] a[2D0?] [254?][2D0?] u[2D0?]/
\end{styleBody}

\begin{styleBody}
\textbf{N vowel qualities:} 6
\end{styleBody}

\begin{styleBody}
\textbf{Diphthongs or vowel sequences:} Vowel sequences /[25B?][254?] a[25B?] a[254?] [254?][25B?] [254?]a i[259?] iu [259?]i [259?]u ui u[259?]/
\end{styleBody}

\begin{styleBody}
\textbf{Contrastive length:} All
\end{styleBody}

\begin{styleBody}
\textbf{Contrastive nasalization:} None
\end{styleBody}

\begin{styleBody}
\textbf{Other contrasts:} N/A
\end{styleBody}

\begin{styleBody}
\textbf{Notes:} Geminate /f s n [272?] ŋ l [27E?] (trill)/ occur in intervocalic position.
\end{styleBody}

\begin{styleBody}
\textbf{Syllable structure}
\end{styleBody}

\begin{styleBody}
\textbf{Complexity Category:} Complex
\end{styleBody}

\begin{styleBody}
\textbf{Canonical syllable structure:} (C)V(C)(C)\textbf{ }(Stirtz 2011: 36-43)
\end{styleBody}

\begin{styleBody}
\textbf{Size of maximal onset:} 1
\end{styleBody}

\begin{styleBody}
\textbf{Size of maximal coda:} 2
\end{styleBody}

\begin{styleBody}
\textbf{Onset obligatory:} No
\end{styleBody}

\begin{styleBody}
\textbf{Coda obligatory:} No
\end{styleBody}

\begin{styleBody}
\textbf{Vocalic nucleus patterns:} Short vowels, Long vowels, Vowel sequences
\end{styleBody}

\begin{styleBody}
\textbf{Syllabic consonant patterns:} N/A
\end{styleBody}

\begin{styleBody}
\textbf{Size of maximal word{}-marginal sequences with syllabic obstruents:} N/A
\end{styleBody}

\begin{styleBody}
\textbf{Predictability of syllabic consonants:} N/A
\end{styleBody}

\begin{styleBody}
\textbf{Morphological constituency of maximal syllable margin:} Both patterns (Coda)
\end{styleBody}

\begin{styleBody}
\textbf{Morphological pattern of syllabic consonants:} N/A
\end{styleBody}

\begin{styleBody}
\textbf{Onset restrictions: }Apparently none.
\end{styleBody}

\begin{styleBody}
\textbf{Coda restrictions: }For simple codas, all consonants except /[294?], k’, d[361?][292?]/ may occur. Biconsonantal codas occur only word-finally. C\textsubscript{1} is /r, l/, or nasal. C\textsubscript{2} is an obstruent.
\end{styleBody}

\begin{styleBody}
\textbf{Suprasegmentals}
\end{styleBody}

\begin{styleBody}
\textbf{Tone:} Yes
\end{styleBody}

\begin{styleBody}
\textbf{Word stress:} Yes
\end{styleBody}

\begin{styleBody}
\textbf{Stress placement:} Unpredictable/Variable
\end{styleBody}

\begin{styleBody}
\textbf{Phonetic processes conditioned by stress:} (None)
\end{styleBody}

\begin{styleBody}
\textbf{Differences in phonological properties of stressed and unstressed syllables:} (None)
\end{styleBody}

\begin{styleBody}
\textbf{Phonetic correlates of stress: }Intensity (impressionistic)
\end{styleBody}

\begin{styleBody}
\textbf{Notes:} In connected speech, stress patterns subject to ‘largely unpredictable rhythmic variation’ (Crewe 1975: 12-13).
\end{styleBody}

\begin{styleBody}
\textbf{Vowel reduction processes}
\end{styleBody}

\begin{styleBody}
(none reported)
\end{styleBody}

\begin{styleBody}
\textbf{Consonant allophony processes}
\end{styleBody}

\begin{styleBody}
\textbf{tbi-C1: }Voiced bilabial and palatal stops are realized as approximants intervocalically (Stirtz 2011: 25).
\end{styleBody}

\begin{styleBody}
\textbf{tbi-C2: }Voiced bilabial and palatal stops and glides are realized as the corresponding vowel when occurring word-finally (Stirtz 2011).
\end{styleBody}

\begin{styleBody}
\textbf{Morphology}
\end{styleBody}

\begin{styleBody}
\textbf{Text:} “The goat and the fox,” “The Nyeerma and the fox” (Stirtz 2011: 319-326)
\end{styleBody}

\begin{styleBody}
\textbf{Synthetic index: }1.5 morphemes/word (503 morphemes, 339 words)
\end{styleBody}

\clearpage\begin{styleBody}
\textbf{[teh] }\ \ \textbf{\textsc{Tehuelche}}\textbf{\ \ }Chonan, \textit{Continental Chonan} (Argentina)
\end{styleBody}

\begin{styleBody}
\textbf{References consulted: }Fernández Garay (1998), Fernández Garay \& Hernández (2006)
\end{styleBody}

\begin{styleBody}
\textbf{Sound inventory}
\end{styleBody}

\begin{styleBody}
\textbf{C phoneme inventory:} /p b t[32A?] d[32A?] k [261?] q [262?] [294?] p’ t[32A?]’ k’ q’ t[361?][283?] t[361?][283?]’ s [283?] x $\chi $ m n l r w j/
\end{styleBody}

\begin{styleBody}
\textbf{N consonant phonemes:} 25
\end{styleBody}

\begin{styleBody}
\textbf{Geminates:} N/A
\end{styleBody}

\begin{styleBody}
\textbf{Voicing contrasts:} Obstruents
\end{styleBody}

\begin{styleBody}
\textbf{Places:} Bilabial, Dental, Palato-alveolar, Velar, Uvular, Glottal
\end{styleBody}

\begin{styleBody}
\textbf{Manners:} Stop, Affricate, Fricative, Nasal, Trill, Central approximant, Lateral approximant
\end{styleBody}

\begin{styleBody}
\textbf{N elaborations:} 3
\end{styleBody}

\begin{styleBody}
\textbf{Elaborations:} Ejective, Palato-alveolar, Uvular
\end{styleBody}

\begin{styleBody}
\textbf{V phoneme inventory:} /e a o e[2D0?] a[2D0?] o[2D0?]/
\end{styleBody}

\begin{styleBody}
\textbf{N vowel qualities:} 3
\end{styleBody}

\begin{styleBody}
\textbf{Diphthongs or vowel sequences:} None
\end{styleBody}

\begin{styleBody}
\textbf{Contrastive length:} All
\end{styleBody}

\begin{styleBody}
\textbf{Contrastive nasalization:} None
\end{styleBody}

\begin{styleBody}
\textbf{Other contrasts:} N/A
\end{styleBody}

\begin{styleBody}
\textbf{Syllable structure}
\end{styleBody}

\begin{styleBody}
\textbf{Complexity Category:} Highly Complex
\end{styleBody}

\begin{styleBody}
\textbf{Canonical syllable structure:} (C)(C)V(C)(C)(C)\textbf{ }(Fernández Garay 1998: 93-103; Fernández Garay \& Hernández 2006: 13-14)
\end{styleBody}

\begin{styleBody}
\textbf{Size of maximal onset:} 2
\end{styleBody}

\begin{styleBody}
\textbf{Size of maximal coda:} 3
\end{styleBody}

\begin{styleBody}
\textbf{Onset obligatory:} No
\end{styleBody}

\begin{styleBody}
\textbf{Coda obligatory:} No
\end{styleBody}

\begin{styleBody}
\textbf{Vocalic nucleus patterns:} Short vowels, Long vowels
\end{styleBody}

\begin{styleBody}
\textbf{Syllabic consonant patterns:} Nasal, Obstruent
\end{styleBody}

\begin{styleBody}
\textbf{Size of maximal word{}-marginal sequences with syllabic obstruents:} 3 (initial), {\textgreater}3 (final)
\end{styleBody}

\begin{styleBody}
\textbf{Predictability of syllabic consonants:} Phonemic, Predictable from word/consonantal context, or Varies with CV sequence
\end{styleBody}

\begin{styleBody}
\textbf{Morphological constituency of maximal syllable margin:} Both patterns (Onset, Coda)
\end{styleBody}

\begin{styleBody}
\textbf{Morphological pattern of syllabic consonants:} Grammatical items
\end{styleBody}

\begin{styleBody}
\textbf{Onset restrictions:} Biconsonantal onsets include /pl/, /k, q/+obstruent, and /m/+consonant.
\end{styleBody}

\begin{styleBody}
\textbf{Coda restrictions:} There are no apparent restrictions for biconsonantal codas, which include /m[283?] p[283?] t[283?] [294?]r lk' rt[361?][283?] rn/. Triconsonantal clusters include at least /[294?][283?]p mnk rnk [294?]nk/.
\end{styleBody}

\begin{styleBody}
\textbf{Notes:} Canonical patterns listed above are for the root morpheme and word levels (Fernández Garay \& Hernández explicitly state that triconsonantal clusters occur word-finally, but don’t include this shape in their list). In addition to clusters, Fernández Garay (1998) and Fernández Garay \& Hernández (2006) posit syllables of the shape C and CC, which may consist of obstruents (and some other Cs) and correspond to grammatical morphemes. These can be added at word margins to the canonical clusters to form larger sequences of consonants at the “phrase” level, an example being k{}-t[361?][283?]a[294?][283?]p{}-[283?]{}-kn {\textgreater} /kt[361?][283?]a[294?][283?]p[283?]k’n/ ‘it is being washed’ (2006: 13). Examples throughout the description show maxima of three consonants word/phrase-initially and six consonants word/phrase-finally resulting from affixation processes. The authors state that “this accumulation of consonants is made possible by the development of epenthetic vowels.” These \textit{supporting vowels} play the role of “lubricator” in sequences of consonants and are described as having a neutral vowel quality corresponding to the neutralization of all other vowels (2006: 13). However, in the (1998) reference they are transcribed alternately as [[259?]] and [[28A?]], seemingly as a response to the consonantal environment.
\end{styleBody}

\begin{styleBody}
\textbf{Suprasegmentals}
\end{styleBody}

\begin{styleBody}
\textbf{Tone:} No
\end{styleBody}

\begin{styleBody}
\textbf{Word stress:} Yes
\end{styleBody}

\begin{styleBody}
\textbf{Stress placement:} Morphologically or Lexically Conditioned
\end{styleBody}

\begin{styleBody}
\textbf{Phonetic processes conditioned by stress:} Vowel Reduction
\end{styleBody}

\begin{styleBody}
\textbf{Differences in phonological properties of stressed and unstressed syllables:} Not described
\end{styleBody}

\begin{styleBody}
\textbf{Phonetic correlates of stress: }Not described
\end{styleBody}

\begin{styleBody}
\textbf{Vowel reduction processes}
\end{styleBody}

\begin{styleBody}
\textbf{teh-R1:} Mid front vowel /e/ is realized as [[259?]] when unstressed (Fernandez Garay 1998: 82).
\end{styleBody}

\begin{styleBody}
\textbf{teh-R2: }Vowels may be elided in word-internal and word-final (unstressed) syllables. Entire unstressed syllables may delete as well (Fernandez Garay 1998: 104-5).
\end{styleBody}

\begin{styleBody}
\textbf{Consonant allophony processes}
\end{styleBody}

\begin{styleBody}
\textbf{teh-C1: }A palatal glide may be realized as a voiced palato-alveolar fricative by some speakers (Fernández Garay 1998: 73).
\end{styleBody}

\begin{styleBody}
\textbf{teh-C2: }Uvular stops and fricatives are fronted preceding a front vowel (Fernández Garay 1998: 77-78).
\end{styleBody}

\begin{styleBody}
\textbf{Notes: }There is a great deal of unpredictable allophonic variation due to the obsolescence of the language.
\end{styleBody}

\begin{styleBody}
\textbf{Morphology}
\end{styleBody}

\begin{styleBody}
\textbf{Text:} “Monologo 2,” “Monologo 4” (Fernandez Garay and Hernandez 2006: 269-286)
\end{styleBody}

\begin{styleBody}
\textbf{Synthetic index: }1.7 morphemes/word (412 morphemes, 249 words)
\end{styleBody}

\clearpage\begin{styleBody}
\textbf{[tel] }\ \ \textbf{\textsc{Telugu}}\textbf{\ \ \ \ }Dravidian, \textit{South Dravidian} (India)
\end{styleBody}

\begin{styleBody}
\textbf{References consulted: }Kelley (1963), Kostić et al. (1977), Krishnamurti (1998), Bhaskararao \& Ray (2017)
\end{styleBody}

\begin{styleBody}
\textbf{Sound inventory}
\end{styleBody}

\begin{styleBody}
\textbf{C phoneme inventory:} /p b t[32A?] d[32A?] [288?] [256?] k [261?] p[2B0?] b[2B0?] d[32A?][2B0?] [288?][2B0?] [256?][2B0?] k[2B0?] [261?][2B0?] t[361?][255?] t[361?][255?][2B0?] d[361?][291?] d[361?][291?][2B0?] f s [282?] [255?] x m n [273?] r l [26D?] w j/
\end{styleBody}

\begin{styleBody}
\textbf{N consonant phonemes:} 32
\end{styleBody}

\begin{styleBody}
\textbf{Geminates:} N/A
\end{styleBody}

\begin{styleBody}
\textbf{Voicing contrasts:} Obstruents
\end{styleBody}

\begin{styleBody}
\textbf{Places:} Bilabial, Dental, Alveolar, Retroflex, Alveolo-palatal, Velar
\end{styleBody}

\begin{styleBody}
\textbf{Manners:} Stop, Affricate, Fricative, Nasal, Trill, Central approximant, Lateral approximant
\end{styleBody}

\begin{styleBody}
\textbf{N elaborations:} 5
\end{styleBody}

\begin{styleBody}
\textbf{Elaborations:} Breathy voice, Voiced fricatives/affricates, Post-aspiration, Labiodental, Retroflex
\end{styleBody}

\begin{styleBody}
\textbf{V phoneme inventory:} /i e a o u i[2D0?] e[2D0?] æ[2D0?] a[2D0?] o[2D0?] u[2D0?]/
\end{styleBody}

\begin{styleBody}
\textbf{N vowel qualities:} 6
\end{styleBody}

\begin{styleBody}
\textbf{Diphthongs or vowel sequences:} Diphthongs /ai au/
\end{styleBody}

\begin{styleBody}
\textbf{Contrastive length:} Some
\end{styleBody}

\begin{styleBody}
\textbf{Contrastive nasalization:} None
\end{styleBody}

\begin{styleBody}
\textbf{Other contrasts:} N/A
\end{styleBody}

\begin{styleBody}
\textbf{Notes:} Vowel length distinction for all but /æ[2D0?]/, which is always long. Place distinctions in consonant inventory from Kostić et al. (1977:191).
\end{styleBody}

\begin{styleBody}
\textbf{Syllable structure}
\end{styleBody}

\begin{styleBody}
\textbf{Complexity Category:} Moderately Complex
\end{styleBody}

\begin{styleBody}
\textbf{Canonical syllable structure:} (C)V(C)\textbf{ }(Kostić et al. 1977: 199)
\end{styleBody}

\begin{styleBody}
\textbf{Size of maximal onset:} 1
\end{styleBody}

\begin{styleBody}
\textbf{Size of maximal coda:} 1
\end{styleBody}

\begin{styleBody}
\textbf{Onset obligatory:} No
\end{styleBody}

\begin{styleBody}
\textbf{Coda obligatory:} No
\end{styleBody}

\begin{styleBody}
\textbf{Vocalic nucleus patterns:} Short vowels, Long vowels, Diphthongs
\end{styleBody}

\begin{styleBody}
\textbf{Syllabic consonant patterns:} N/A
\end{styleBody}

\begin{styleBody}
\textbf{Size of maximal word{}-marginal sequences with syllabic obstruents:} N/A
\end{styleBody}

\begin{styleBody}
\textbf{Predictability of syllabic consonants:} N/A
\end{styleBody}

\begin{styleBody}
\textbf{Morphological constituency of maximal syllable margin:} N/A
\end{styleBody}

\begin{styleBody}
\textbf{Morphological pattern of syllabic consonants:} N/A
\end{styleBody}

\begin{styleBody}
\textbf{Onset restrictions:} All consonants occur. 
\end{styleBody}

\begin{styleBody}
\textbf{Coda restrictions:} None? Speaker judgment data suggests that word-medial geminate clusters (e.g. \textit{pennu}) are typically syllabified as onset of syllable, while other word-medial CC clusters (e.g. \textit{gampa}) are split across syllables (Sailaja 1999).
\end{styleBody}

\begin{styleBody}
\textbf{Suprasegmentals}
\end{styleBody}

\begin{styleBody}
\textbf{Tone:} No
\end{styleBody}

\begin{styleBody}
\textbf{Word stress:} Yes
\end{styleBody}

\begin{styleBody}
\textbf{Stress placement:} Fixed
\end{styleBody}

\begin{styleBody}
\textbf{Phonetic processes conditioned by stress:} Vowel Reduction, Consonant Allophony in Unstressed Syllables, Consonant Allophony in Stressed Syllables
\end{styleBody}

\begin{styleBody}
\textbf{Differences in phonological properties of stressed and unstressed syllables:} (None)
\end{styleBody}

\begin{styleBody}
\textbf{Phonetic correlates of stress: }Not described
\end{styleBody}

\begin{styleBody}
\textbf{Vowel reduction processes}
\end{styleBody}

\begin{styleBody}
\textbf{tel-R1:} A short vowel is lost between two consonants which have the same place of articulation, or of which C\textsubscript{1} is an apical and C\textsubscript{2} is an apical or laminal, or between dentals and affricates (Kostić et al. 1977: 9).
\end{styleBody}

\begin{styleBody}
\textbf{tel-R2:} Long vowels /i[2D0?] a[2D0?] o[2D0?] u[2D0?]/ in word-final (and unstressed) position may be reduced to the length of a short vowel (Kostić et al. 1977: 11-52).
\end{styleBody}

\begin{styleBody}
\textbf{tel-R3:} [[259?]] is reported to be an allophone or perhaps free variant of /a, e/ but no further description is given as to its distribution (Kostić et al. 1977).
\end{styleBody}

\begin{styleBody}
\textbf{Consonant allophony processes}
\end{styleBody}

\begin{styleBody}
\textbf{tel-C1: }A voiceless aspirated stop may be realized as an affricate [kx] in a stressed syllable (Kostić et al. 1977: 105).
\end{styleBody}

\begin{styleBody}
\textbf{tel-C2: }Stops are partially fricated in unstressed position (Kostić et al. 1977: 89).
\end{styleBody}

\begin{styleBody}
\textbf{tel-C3: }Stops are realized as fricatives intervocalically (Krishnamurti 1998: 207).
\end{styleBody}

\begin{styleBody}
\textbf{tel-C4: }A bilabial stop or nasal varies with a (nasalized) labial glide intervocalically (Krishnamurti 1998: 207, Bhaskararao \& Ray 2017: 234).
\end{styleBody}

\begin{styleBody}
\textbf{tel-C5: }A bilabial stop or nasal is realized as a (nasalized) labial glide word-finally (Krishnamurti 1998: 207, Bhaskararao \& Ray 2017: 234).
\end{styleBody}

\begin{styleBody}
\textbf{tel-C6: }A bilabial stop is realized as a labial glide preceding /w s h/ (Krishnamurti 1998: 207).
\end{styleBody}

\begin{styleBody}
\textbf{tel-C7: }In intervocalic position /r/ and /[273?]/ are realized as taps/flaps (Bhaskararao \& Ray 2017: 235).
\end{styleBody}

\begin{styleBody}
\textbf{Morphology}
\end{styleBody}

\begin{styleBody}
(adequate texts unavailable)
\end{styleBody}

\clearpage\begin{styleBody}
\textbf{[thp] }\ \ \textbf{\textsc{Thompson}}\textbf{\ \ }Salishan, \textit{Interior Salish} (Canada)
\end{styleBody}

\begin{styleBody}
\textbf{References consulted: }Koch (2008), Thompson \& Thompson (1992, 1996)
\end{styleBody}

\begin{styleBody}
\textbf{Sound inventory}
\end{styleBody}

\begin{styleBody}
\textbf{C phoneme inventory:} /p t[32A?] k k[2B7?] q q[2B7?] [294?] p’ k’ k’[2B7?] q’ q’[2B7?] t[361?]s t[361?][283?] t[32A?][361?][26C?]’ t[361?]s’ s [283?] x x[2B7?] $\chi $ $\chi [2B7?]$ h [26C?][32A?] m n[32A?] m’ n[32A?]’ l l’ ð[31E?] j [270?] w [2A2?] [2A2?][2B7?] ð[31E?]’ j’ [270?]’ w’ [2A2?]’ [2A2?]’[2B7?]/
\end{styleBody}

\begin{styleBody}
\textbf{N consonant phonemes:} 42
\end{styleBody}

\begin{styleBody}
\textbf{Geminates:} N/A
\end{styleBody}

\begin{styleBody}
\textbf{Voicing contrasts:} None
\end{styleBody}

\begin{styleBody}
\textbf{Places:} Bilabial, Dental, Alveolar, Palato-alveolar, Velar, Uvular, Glottal 
\end{styleBody}

\begin{styleBody}
\textbf{Manners:} Stop, Affricate, Fricative, Nasal, Central approximant, Lateral affricate, Lateral fricative, Lateral approximant
\end{styleBody}

\begin{styleBody}
\textbf{N elaborations:} 7
\end{styleBody}

\begin{styleBody}
\textbf{Elaborations:} Creaky voice, Lateral release, Ejective, Palato-alveolar, Uvular, Pharyngeal, Labialization
\end{styleBody}

\begin{styleBody}
\textbf{V phoneme inventory:} /i [26A?] [25B?] [259?] [28C?] a [254?] u/
\end{styleBody}

\begin{styleBody}
\textbf{N vowel qualities:} 8
\end{styleBody}

\begin{styleBody}
\textbf{Diphthongs or vowel sequences:} None
\end{styleBody}

\begin{styleBody}
\textbf{Contrastive length:} None
\end{styleBody}

\begin{styleBody}
\textbf{Contrastive nasalization:} None
\end{styleBody}

\begin{styleBody}
\textbf{Other contrasts:} N/A
\end{styleBody}

\begin{styleBody}
\textbf{Notes:} /[270?] [270?]’/ are exceedingly rare. /t[32A?]’/ occurs rarely and in apparent loanwords. [[26A?]] varies with [[258?]], but this phoneme is generally very rare.
\end{styleBody}

\begin{styleBody}
\textbf{Syllable structure}
\end{styleBody}

\begin{styleBody}
\textbf{Complexity Category:} Highly Complex
\end{styleBody}

\begin{styleBody}
\textbf{Canonical syllable structure:} C(C)(C)V(C)(C)(C)(C)(C)(C)\textbf{ }(Thompson \& Thompson 1992: 25-43; Thompson \& Thompson 1996; Thompson et al. 1996)
\end{styleBody}

\begin{styleBody}
\textbf{Size of maximal onset:} 3
\end{styleBody}

\begin{styleBody}
\textbf{Size of maximal coda:} 6
\end{styleBody}

\begin{styleBody}
\textbf{Onset obligatory:} Yes
\end{styleBody}

\begin{styleBody}
\textbf{Coda obligatory:} No
\end{styleBody}

\begin{styleBody}
\textbf{Vocalic nucleus patterns:} Short vowels
\end{styleBody}

\begin{styleBody}
\textbf{Syllabic consonant patterns:} Nasal, Liquid
\end{styleBody}

\begin{styleBody}
\textbf{Size of maximal word{}-marginal sequences with syllabic obstruents:} N/A
\end{styleBody}

\begin{styleBody}
\textbf{Predictability of syllabic consonants:} Predictable from word/consonantal context
\end{styleBody}

\begin{styleBody}
\textbf{Morphological constituency of maximal syllable margin:} Morphologically Complex (Onset, Coda)
\end{styleBody}

\begin{styleBody}
\textbf{Morphological pattern of syllabic consonants:} Lexical items (Liquid), Both (Nasal)
\end{styleBody}

\begin{styleBody}
\textbf{Onset restrictions:} Apparently all Cs may occur initially. Biconsonantal onsets include /k[26C?], q[2B7?]n, sx[2B7?], st[361?][26C?]’, t[361?]sk[2B7?]’/. Triconsonantal onsets include /ns[294?], s$\chi $ð[31E?], spt, st[361?]s’k/.
\end{styleBody}

\begin{styleBody}
\textbf{Coda restrictions:} Unclear if all Cs occur in codas. Biconsonantal codas quite varied, include /mx[2B7?] [283?]t q[2B7?]m [294?]t/. Triconsonantal codas include /x[2B7?]kt x[2B7?]st[361?]s pst[361?]s kst t[361?]sms/. Six-consonant codas include obstruent-only sequences such as /[26C?]qsxtx[2B7?]/.
\end{styleBody}

\begin{styleBody}
\textbf{Notes:} Canonical syllable patterns not explicitly stated, but based on examples given. Authors state that sequences of six obstruents “not uncommon” (1992: 25). Authors analyze some clusters as having underlying vowels intervening between consonants. As far as I can tell from the description, unstressed underlying vowels do not occur in actual production unless explicitly transcribed in the surface form. Coda clusters larger than six have been observed, however, example is transcribed with optional intervening schwa (\textit{n\'{i}k’kstkpt[361?][26C?]’([259?])[26C?]} ‘you people already got your hands cut’, 1992:25). “Study of the grammatical system shows that such words [with long obstruent sequences] are made up of strings of meaningful subparts, morphemes, many of which have vowels when they fall under stress. But each word has just a single main stress, and vowels mostly drop out of the unstressed morphemes.” (Thompson et al 1996: 612)
\end{styleBody}

\begin{styleBody}
\textbf{Suprasegmentals}
\end{styleBody}

\begin{styleBody}
\textbf{Tone:} No
\end{styleBody}

\begin{styleBody}
\textbf{Word stress:} Yes
\end{styleBody}

\begin{styleBody}
\textbf{Stress placement:} Morphologically or Lexically Conditioned
\end{styleBody}

\begin{styleBody}
\textbf{Phonetic processes conditioned by stress:} Vowel Reduction, Consonant Allophony in Unstressed Syllables, Consonant Allophony in Stressed Syllables
\end{styleBody}

\begin{styleBody}
\textbf{Differences in phonological properties of stressed and unstressed syllables:} Vowel Quality Contrasts
\end{styleBody}

\begin{styleBody}
\textbf{Phonetic correlates of stress: }Vowel duration (instrumental), Pitch (instrumental), Intensity (instrumental)
\end{styleBody}

\begin{styleBody}
\textbf{Vowel reduction processes}
\end{styleBody}

\begin{styleBody}
\textbf{thp-R1: }High vowels /i u/ are nearly always realized as [[259?]] when preceding the main stress, except for when /u/ occurs between two velar consonants (Thompson \& Thompson 1992: 32).
\end{styleBody}

\begin{styleBody}
\textbf{thp-R2:} In allegro speech, unstressed /i [26A?] [25B?] a o u/ tend to be realized as [[259?]] (Thompson \& Thompson 1992: 45).
\end{styleBody}

\begin{styleBody}
\textbf{thp-R3}: In allegro speech, unstressed /[259?]/ is deleted between obstruents (Thompson \& Thompson 1992: 45).
\end{styleBody}

\begin{styleBody}
\textbf{thp-R4:} Unstressed /[28C?]/ is reduced to [[259?]] (Thompson \& Thompson 1992: 45).
\end{styleBody}

\begin{styleBody}
\textbf{thp-R5:} In allegro speech, stressed /a, o/ and sometimes /[25B?]/ are frequently replaced by [[28C?]] (Thompson \& Thompson 1992: 46).
\end{styleBody}

\begin{styleBody}
The processes below have quite a few phonological and some morphological complications, and their productivity is unclear. I have not included them as phonetically- or phonologically-conditioned vowel reduction in my analysis, but they should be noted:
\end{styleBody}

\begin{styleBody}
\textbf{thp-R6:} Most post-tonic vowels are deleted (Thompson \& Thompson 1992: 23, 33-4).
\end{styleBody}

\begin{styleBody}
\textbf{thp-R7:} In successive syllables preceding main stress, /[259?]/ is deleted (Thompson \& Thompson 1992: 31-2).
\end{styleBody}

\begin{styleBody}
\textbf{thp-R8:} In successive syllables preceding main stress, vowels /i [26A?] [25B?] a o u/ are deleted when adjacent to a laryngeal, pharyngeal, or homorganic semivowel and preceding the main stress (Thompson \& Thompson 1992: 31-2).
\end{styleBody}

\begin{styleBody}
\textbf{Consonant allophony processes}
\end{styleBody}

\begin{styleBody}
\textbf{thp-C1: }Velars are rounded following /u/ in a stressed or closed syllable (Thompson \& Thompson 1992: 36).
\end{styleBody}

\begin{styleBody}
\textbf{Morphology}
\end{styleBody}

\begin{styleBody}
\textbf{Text: }“The man who went to the moon” (lines 1-65; Thompson \& Thompson 1992: 200-5)
\end{styleBody}

\begin{styleBody}
\textbf{Synthetic index: }1.7 morphemes/word (506 morphemes, 297 words)
\end{styleBody}

\clearpage\begin{styleBody}
\textbf{[tow] }\ \ \textbf{\textsc{Towa\ \ }}\textbf{\ \ }Kiowa-Tanoan (United States)
\end{styleBody}

\begin{styleBody}
\textbf{References consulted: }Bell (1993), Logan Sutton (p.c.), Yumitani (1998)
\end{styleBody}

\begin{styleBody}
\textbf{Sound inventory}
\end{styleBody}

\begin{styleBody}
\textbf{C phoneme inventory:} /p b t d c k[2B2?][2B0?] [261?][2B2?] k[2B0?] [261?] k[2B7?] [294?] p’ t’ k[2B2?]’ k’ t[361?][283?] [278?] f v s z [283?] h [266?] m n [27E?] l j w/
\end{styleBody}

\begin{styleBody}
\textbf{N consonant phonemes:} 30
\end{styleBody}

\begin{styleBody}
\textbf{Geminates:} N/A
\end{styleBody}

\begin{styleBody}
\textbf{Voicing contrasts:} Obstruents
\end{styleBody}

\begin{styleBody}
\textbf{Places:} Bilabial, Labiodental, Alveolar, Palato-Alveolar, Palatal, Velar, Glottal
\end{styleBody}

\begin{styleBody}
\textbf{Manners:} Stop, Affricate, Fricative, Nasal, Flap/Tap, Central Approximant, Lateral Approximant
\end{styleBody}

\begin{styleBody}
\textbf{N elaborations:} 6
\end{styleBody}

\begin{styleBody}
\textbf{Elaborations:} Voiced fricative/affricate, Ejective, Labiodental, Palato-alveolar, Palatalization, Labialization
\end{styleBody}

\begin{styleBody}
\textbf{V phoneme inventory:} /i e æ [268?] [251?] o i[2D0?] e[2D0?] æ[2D0?] [268?][2D0?] o[2D0?] [251?][2D0?] i\~{ } \~{æ} [268?] [251?] \~{o} i\~{ }[2D0?] \~{æ}[2D0?] [268?][2D0?] [251?][2D0?] \~{o}[2D0?]/
\end{styleBody}

\begin{styleBody}
\textbf{N vowel qualities:} 6
\end{styleBody}

\begin{styleBody}
\textbf{Diphthongs or vowel sequences:} None
\end{styleBody}

\begin{styleBody}
\textbf{Contrastive length:} All
\end{styleBody}

\begin{styleBody}
\textbf{Contrastive nasalization:} Some
\end{styleBody}

\begin{styleBody}
\textbf{Other contrasts:} None
\end{styleBody}

\begin{styleBody}
\textbf{Notes:} Yumitani lists voiceless velar stops as aspirated, but it aspiration is not contrastive. Nasalization contrast for all but /e e[2D0?]/.
\end{styleBody}

\begin{styleBody}
\textbf{Syllable structure}
\end{styleBody}

\begin{styleBody}
\textbf{Complexity Category:} Simple
\end{styleBody}

\begin{styleBody}
\textbf{Canonical syllable structure:} (C)V(C)\textbf{ }(Yumitani 1998: 21-22)
\end{styleBody}

\begin{styleBody}
\textbf{Size of maximal onset:} 1
\end{styleBody}

\begin{styleBody}
\textbf{Size of maximal coda:} 1
\end{styleBody}

\begin{styleBody}
\textbf{Onset obligatory:} No
\end{styleBody}

\begin{styleBody}
\textbf{Coda obligatory:} No
\end{styleBody}

\begin{styleBody}
\textbf{Vocalic nucleus patterns:} Short vowels, Long vowels
\end{styleBody}

\begin{styleBody}
\textbf{Syllabic consonant patterns:} N/A
\end{styleBody}

\begin{styleBody}
\textbf{Size of maximal word{}-marginal sequences with syllabic obstruents:} N/A
\end{styleBody}

\begin{styleBody}
\textbf{Predictability of syllabic consonants:} N/A
\end{styleBody}

\begin{styleBody}
\textbf{Morphological constituency of maximal syllable margin:} N/A
\end{styleBody}

\begin{styleBody}
\textbf{Morphological pattern of syllabic consonants:} N/A
\end{styleBody}

\begin{styleBody}
\textbf{Onset restrictions: }All consonants occur.
\end{styleBody}

\begin{styleBody}
\textbf{Coda restrictions: }/l/, /[283?]/ in limited environments.
\end{styleBody}

\begin{styleBody}
\textbf{Phonetic correlates of stress: }Vowel duration (instrumental), Pitch (instrumental)
\end{styleBody}

\begin{styleBody}
\textbf{Notes:} The issue of whether Towa has true vowel-initial syllables is still up for debate; a voiced glottal fricative is often heard before the vowel (Yumitani 1998: 22-23). Though the canonical syllable structure has an optional coda, in speech there is a tendency toward a CV template. Closed syllables do not occur word-internally, and when occurring word-finally are often resyllabified. See discussion in Chapter 3.
\end{styleBody}

\begin{styleBody}
\textbf{Suprasegmentals}
\end{styleBody}

\begin{styleBody}
\textbf{Tone:} Yes
\end{styleBody}

\begin{styleBody}
\textbf{Word stress:} Yes
\end{styleBody}

\begin{styleBody}
\textbf{Stress placement:} Morphologically or Lexically Conditioned
\end{styleBody}

\begin{styleBody}
\textbf{Phonetic processes conditioned by stress:} Vowel Reduction, Consonant Allophony in Unstressed Syllables
\end{styleBody}

\begin{styleBody}
\textbf{Differences in phonological properties of stressed and unstressed syllables:} (None)
\end{styleBody}

\begin{styleBody}
\textbf{Phonetic correlates of stress: }Vowel duration (instrumental), Pitch (instrumental)
\end{styleBody}

\begin{styleBody}
\textbf{Notes: }Pitch is not as strong an indicator as length for stress.
\end{styleBody}

\begin{styleBody}
\textbf{Vowel reduction processes}
\end{styleBody}

\begin{styleBody}
\textbf{tow-R1:} Long vowels become glottalized (V[294?]V) word-finally before a pause (Yumitani 1998: 20; vowels not specified).
\end{styleBody}

\begin{styleBody}
\textbf{tow-R2: }Some long vowels have short variants word-finally (Yumitani 1998: 20; vowels not specified).
\end{styleBody}

\begin{styleBody}
\textbf{tow-R3: }Non-initial (unstressed) vowels are more central and harder to identify than corresponding initial (stressed) vowels (Yumitani 1998: 31, Bell 1993: 29).
\end{styleBody}

\begin{styleBody}
\textbf{Consonant allophony processes}
\end{styleBody}

\begin{styleBody}
\textbf{tow-C1: }Palatal stops are realized as palato-alveolar affricates preceding a high front vowel (Yumitani 1998: 13).
\end{styleBody}

\begin{styleBody}
\textbf{tow-C2: }A palato-alveolar fricative is realized as a voiced palato-alveolar affricate preceding a high front vowel (Yumitani 1998: 13).
\end{styleBody}

\begin{styleBody}
\textbf{tow-C3: }A voiceless alveolar fricative is realized as a glottal fricative syllable-initially in a syllable carrying low tone, especially among younger speakers (Yumitani 1998: 13).
\end{styleBody}

\begin{styleBody}
\textbf{Morphology}
\end{styleBody}

\begin{styleBody}
\textbf{Text:} “About my childhood” (Yumitani 1998: 248-250)
\end{styleBody}

\begin{styleBody}
\textbf{Synthetic index: }1.7 morphemes/word (408 morphemes, 244 words)
\end{styleBody}

\clearpage\begin{styleBody}
\textbf{[tzh] }\ \ \textbf{\textsc{Tzeltal (Aguacatenango dialect)}}\textbf{\ \ }Mayan, \textit{Core Mayan} (Mexico)
\end{styleBody}

\begin{styleBody}
\textbf{References consulted: }Kaufman (1971), Polian (2006), Smith (2007)
\end{styleBody}

\begin{styleBody}
\textbf{Sound inventory}
\end{styleBody}

\begin{styleBody}
\textbf{C phoneme inventory:} /p b t[32A?] k [294?] p’ t[32A?]’ k’ t[32A?]s[32A?] t[361?][283?] t[32A?]s[32A?]’ t[361?][283?]’ s [283?] h m n l[32A?] [27E?] w j/
\end{styleBody}

\begin{styleBody}
\textbf{N consonant phonemes:} 21
\end{styleBody}

\begin{styleBody}
\textbf{Geminates:} N/A
\end{styleBody}

\begin{styleBody}
\textbf{Voicing contrasts:} Obstruents
\end{styleBody}

\begin{styleBody}
\textbf{Places:} Bilabial, Dental, Alveolar, Palato-alveolar, Velar, Glottal
\end{styleBody}

\begin{styleBody}
\textbf{Manners:} Stop, Affricate, Fricative, Nasal, Flap/Tap, Central approximant, Lateral approximant
\end{styleBody}

\begin{styleBody}
\textbf{N elaborations:} 2
\end{styleBody}

\begin{styleBody}
\textbf{Elaborations:} Ejective, Palato-alveolar
\end{styleBody}

\begin{styleBody}
\textbf{V phoneme inventory:} /i e a o u e[2D0?]/
\end{styleBody}

\begin{styleBody}
\textbf{N vowel qualities: }5
\end{styleBody}

\begin{styleBody}
\textbf{Diphthongs or vowel sequences:} None
\end{styleBody}

\begin{styleBody}
\textbf{Contrastive length:} Yes
\end{styleBody}

\begin{styleBody}
\textbf{Contrastive nasalization:} Some
\end{styleBody}

\begin{styleBody}
\textbf{Other contrasts:} N/A
\end{styleBody}

\begin{styleBody}
\textbf{Notes:} /f [27E?]/ occur only in loanwords in speech of ‘acculturated’ speakers (Kaufman 1971: 13). Kaufman also gives /d [261?]/. Smith calls /j/ a fricative. The few cases of apparent long vowels are often morphologically complex; however, /e[2D0?]/ is contrastive in one monomorphemic word.
\end{styleBody}

\begin{styleBody}
\textbf{Syllable structure}
\end{styleBody}

\begin{styleBody}
\textbf{Complexity Category:} Complex
\end{styleBody}

\begin{styleBody}
\textbf{Canonical syllable structure:} (C)(C)V(C)(C)\textbf{ }(Kaufman 1971: 9-15)
\end{styleBody}

\begin{styleBody}
\textbf{Size of maximal onset:} 2
\end{styleBody}

\begin{styleBody}
\textbf{Size of maximal coda:} 2
\end{styleBody}

\begin{styleBody}
\textbf{Onset obligatory:} No
\end{styleBody}

\begin{styleBody}
\textbf{Coda obligatory:} No
\end{styleBody}

\begin{styleBody}
\textbf{Vocalic nucleus patterns:} Short vowels
\end{styleBody}

\begin{styleBody}
\textbf{Syllabic consonant patterns:} N/A
\end{styleBody}

\begin{styleBody}
\textbf{Size of maximal word{}-marginal sequences with syllabic obstruents:} N/A
\end{styleBody}

\begin{styleBody}
\textbf{Predictability of syllabic consonants:} N/A
\end{styleBody}

\begin{styleBody}
\textbf{Morphological constituency of maximal syllable margin:} Morpheme-internal (Coda), Morphologically Complex (Onset)
\end{styleBody}

\begin{styleBody}
\textbf{Morphological pattern of syllabic consonants:} N/A
\end{styleBody}

\begin{styleBody}
\textbf{Onset restrictions:} Apparently no restrictions on simple onsets. Biconsonantal onsets consist of /s [283?] h/ plus any (?) consonant.
\end{styleBody}

\begin{styleBody}
\textbf{Coda restrictions: }Apparently no restrictions on simple codas. Biconsonantal codas limited to sequence of /h/ + voiceless stop or affricate.
\end{styleBody}

\begin{styleBody}
\textbf{Notes: }Initial clusters in Spanish loans may be prefixed with /s [283?] h/, resulting in triconsonantal onset.
\end{styleBody}

\begin{styleBody}
\textbf{Suprasegmentals}
\end{styleBody}

\begin{styleBody}
\textbf{Tone:} No
\end{styleBody}

\begin{styleBody}
\textbf{Word stress:} Yes
\end{styleBody}

\begin{styleBody}
\textbf{Stress placement:} Fixed
\end{styleBody}

\begin{styleBody}
\textbf{Phonetic processes conditioned by stress:} Vowel Reduction
\end{styleBody}

\begin{styleBody}
\textbf{Differences in phonological properties of stressed and unstressed syllables:} (None)
\end{styleBody}

\begin{styleBody}
\textbf{Phonetic correlates of stress: }Vowel duration (impressionistic)
\end{styleBody}

\begin{styleBody}
\textbf{Notes: }Length is a correlate of stress in some dialects (Polian 2006: 23).
\end{styleBody}

\begin{styleBody}
\textbf{Vowel reduction processes}
\end{styleBody}

\begin{styleBody}
\textbf{tzh-R1:} Vowels are realized as extra short when unstressed before a consonant following a stressed vowel (Kaufman 1971: 12).
\end{styleBody}

\begin{styleBody}
\textbf{tzh-R2:} Vowels are short when unstressed before a phrasal (? ‘caret’) juncture (Kaufman 1971: 12).
\end{styleBody}

\begin{styleBody}
\textbf{tzh-R3:} In casual speech, reducible vowels (=post-tonic, if also followed by at least one more vowel and not more than two consonants before a juncture intervenes) are replaced by /a/ or /e/ (Kaufman 1971: 26-7).
\end{styleBody}

\begin{styleBody}
\textbf{Notes: }There are also several productive vowel reduction processes taking place in particular speech styles. In Assimilative Speech, used by unmarried children who are living at home or not yet economically dependent, reducible vowels as well as some other vowels are replaced by echo vowels. In Clipped Speech, used by men between the ages of 18 and 40 who are married or economically independent of their parents, reducible vowels are ‘zeroed’ wherever possible, or otherwise replaced by [[259?]] (Kaufman 1971: 26-7).
\end{styleBody}

\begin{styleBody}
\textbf{Consonant allophony processes}
\end{styleBody}

\begin{styleBody}
\textbf{tzh-C1: }Voiced stops are spirantized following a vowel. (Kaufman 1971: 11)
\end{styleBody}

\begin{styleBody}
\textbf{Morphology}
\end{styleBody}

\begin{styleBody}
\textbf{Text:} “Le voyage à la finca” (first 5 pages, Polian 2006: 235-239) ****Central dialect****
\end{styleBody}

\begin{styleBody}
\textbf{Synthetic index: }1.5 morphemes/word (446 morphemes, 299 words)
\end{styleBody}

\clearpage\begin{styleBody}
\textbf{[ung] }\ \ \textbf{\textsc{Ngarinyin}}\textbf{\ \ }Worrorran (Australia)
\end{styleBody}

\begin{styleBody}
\textbf{References consulted: }Coate \& Oates (1970), Rumsey (1978)
\end{styleBody}

\begin{styleBody}
\textbf{Sound inventory}
\end{styleBody}

\begin{styleBody}
\textbf{C phoneme inventory:} /p t [288?] c k m n [273?] [272?] ŋ l [26D?] [28E?] r [27B?] w j/
\end{styleBody}

\begin{styleBody}
\textbf{N consonant phonemes:} 17
\end{styleBody}

\begin{styleBody}
\textbf{Geminates:} N/A
\end{styleBody}

\begin{styleBody}
\textbf{Voicing contrasts:} None
\end{styleBody}

\begin{styleBody}
\textbf{Places:} Bilabial, Alveolar, Retroflex, Palatal, Velar
\end{styleBody}

\begin{styleBody}
\textbf{Manners:} Stop, Nasal, Trill, Central approximant, Lateral approximant
\end{styleBody}

\begin{styleBody}
\textbf{N elaborations:} 1
\end{styleBody}

\begin{styleBody}
\textbf{Elaborations:} Retroflex
\end{styleBody}

\begin{styleBody}
\textbf{V phoneme inventory:} /[26A?] e a o [28A?] a[2D0?]/
\end{styleBody}

\begin{styleBody}
\textbf{N vowel qualities:} 5
\end{styleBody}

\begin{styleBody}
\textbf{Diphthongs or vowel sequences:} None
\end{styleBody}

\begin{styleBody}
\textbf{Contrastive length:} Some
\end{styleBody}

\begin{styleBody}
\textbf{Contrastive nasalization:} None
\end{styleBody}

\begin{styleBody}
\textbf{Other contrasts:} N/A
\end{styleBody}

\begin{styleBody}
\textbf{Notes:} Vowel length generally not phonemic.
\end{styleBody}

\begin{styleBody}
\textbf{Syllable structure}
\end{styleBody}

\begin{styleBody}
\textbf{Complexity Category:} Complex
\end{styleBody}

\begin{styleBody}
\textbf{Canonical syllable structure:} (C)(C)(C)V(C)(C)\textbf{ }(Rumsey 1978: 23-6)
\end{styleBody}

\begin{styleBody}
\textbf{Size of maximal onset:} 3
\end{styleBody}

\begin{styleBody}
\textbf{Size of maximal coda:} 2
\end{styleBody}

\begin{styleBody}
\textbf{Onset obligatory:} No
\end{styleBody}

\begin{styleBody}
\textbf{Coda obligatory:} No
\end{styleBody}

\begin{styleBody}
\textbf{Vocalic nucleus patterns:} Short vowels
\end{styleBody}

\begin{styleBody}
\textbf{Syllabic consonant patterns:} N/A
\end{styleBody}

\begin{styleBody}
\textbf{Size of maximal word{}-marginal sequences with syllabic obstruents:} N/A
\end{styleBody}

\begin{styleBody}
\textbf{Predictability of syllabic consonants:} N/A
\end{styleBody}

\begin{styleBody}
\textbf{Morphological constituency of maximal syllable margin:} Morpheme-internal (Onset, Coda)
\end{styleBody}

\begin{styleBody}
\textbf{Morphological pattern of syllabic consonants:} N/A
\end{styleBody}

\begin{styleBody}
\textbf{Onset restrictions: }All consonants except for /r [28E?]/ may occur in simple onsets. Biconsonantal onsets are /p[27B?]/, /[288?][27B?]/, /m[27B?]/, /k[27B?]/, and /pr/. Triconsonantal onset limited to /pr[27B?]/.
\end{styleBody}

\begin{styleBody}
\textbf{Coda restrictions: }All consonants except for /p m [28E?] [27B?]/ may occur as simple codas. Biconsonantal codas have lateral /l [26D?]/ as C\textsubscript{1} and nasal /ŋ n [273?] m/ as C\textsubscript{2}. These codas are always followed by an onset which is /k/ or /p/ in the following syllable.
\end{styleBody}

\begin{styleBody}
\textbf{Notes: }Apparently /a/ is the only V syllable.
\end{styleBody}

\begin{styleBody}
\textbf{Suprasegmentals}
\end{styleBody}

\begin{styleBody}
\textbf{Tone:} No
\end{styleBody}

\begin{styleBody}
\textbf{Word stress:} Yes
\end{styleBody}

\begin{styleBody}
\textbf{Stress placement:} Morphologically or Lexically Conditioned
\end{styleBody}

\begin{styleBody}
\textbf{Phonetic processes conditioned by stress:} Vowel Reduction
\end{styleBody}

\begin{styleBody}
\textbf{Differences in phonological properties of stressed and unstressed syllables:} (None)
\end{styleBody}

\begin{styleBody}
\textbf{Phonetic correlates of stress: }see below
\end{styleBody}

\begin{styleBody}
\textbf{Notes:} Here it’s REDUCED duration that is a correlate of stress, and for /a/ only. When carrying primary stress in polysyllabic words, /a/ is higher and shorter than long low back vowel, by a degree which depends on which consonant follows it.
\end{styleBody}

\begin{styleBody}
\textbf{Vowel reduction processes}
\end{styleBody}

\begin{styleBody}
\textbf{ung-R1}: Lax high vowels /[26A?] [28A?]/ are produced as tense in word-final position or before a loose juncture (Rumsey 1978: 13-16).
\end{styleBody}

\begin{styleBody}
\textbf{ung-R2}: Low central vowel /a/ is realized as [[259?]] when unstressed (Rumsey 1978: 17-18).
\end{styleBody}

\begin{styleBody}
\textbf{Consonant allophony processes}
\end{styleBody}

\begin{styleBody}
\textbf{ung-C1: }An alveolar lateral approximant is velarized adjacent to back vowels (Rumsey 1978: 11).
\end{styleBody}

\begin{styleBody}
\textbf{ung-C2: }A voiced bilabial stop is labialized preceding /u/ (Rumsey 1978: 9-10).
\end{styleBody}

\begin{styleBody}
\textbf{ung-C3: }Velars are fronted preceding front vowels (Rumsey 1978: 11).
\end{styleBody}

\begin{styleBody}
\textbf{ung-C4: }Stops are voiced following a nasal and preceding any sound (Rumsey 1978: 9).
\end{styleBody}

\begin{styleBody}
\textbf{ung-C5: }A trill may be realized as a flap intervocalically (Rumsey 1978: 12).
\end{styleBody}

\begin{styleBody}
\textbf{ung-C6: }A trill may be realized as a flap word-finally (Rumsey 1978: 12).
\end{styleBody}

\begin{styleBody}
\textbf{Morphology}
\end{styleBody}

\begin{styleBody}
\textbf{Text:} “Ngunbangguwe ‘Mt. Trafalgar’” (Coate \& Oates 1970: 104-110)
\end{styleBody}

\begin{styleBody}
\textbf{Synthetic index: }2.0 morphemes/word (1142 morphemes, 573 words)
\end{styleBody}

\clearpage\begin{styleBody}
\textbf{[ura] }\ \ \textbf{\textsc{Urarina}}\textbf{\ \ }isolate (Peru)
\end{styleBody}

\begin{styleBody}
\textbf{References consulted: }Olawsky (2006)
\end{styleBody}

\begin{styleBody}
\textbf{Sound inventory}
\end{styleBody}

\begin{styleBody}
\textbf{C phoneme inventory:} /b t d k k[2B7?] t[255?] f[2B7?] s [283?] h h[2B2?] m n [272?] l [27D?]/
\end{styleBody}

\begin{styleBody}
\textbf{N consonant phonemes:} 16
\end{styleBody}

\begin{styleBody}
\textbf{Geminates:} N/A
\end{styleBody}

\begin{styleBody}
\textbf{Voicing contrasts:} Obstruents
\end{styleBody}

\begin{styleBody}
\textbf{Places:} Bilabial, Labiodental, Alveolar, Retroflex, Alveolo-palatal, Velar, Glottal
\end{styleBody}

\begin{styleBody}
\textbf{Manners:} Stop, Affricate, Fricative, Nasal, Flap/tap, Lateral approximant
\end{styleBody}

\begin{styleBody}
\textbf{N elaborations:} 4
\end{styleBody}

\begin{styleBody}
\textbf{Elaborations:} Labiodental, Retroflex, Palatalization, Labialization
\end{styleBody}

\begin{styleBody}
\textbf{V phoneme inventory:} /i e [289?] a u \~{i} \~{e} [289?] \~{a} \~{u}/
\end{styleBody}

\begin{styleBody}
\textbf{N vowel qualities:} 5
\end{styleBody}

\begin{styleBody}
\textbf{Diphthongs or vowel sequences:} Diphthongs /ae aj ej au a[289?]/
\end{styleBody}

\begin{styleBody}
\textbf{Contrastive length:} None
\end{styleBody}

\begin{styleBody}
\textbf{Contrastive nasalization:} All
\end{styleBody}

\begin{styleBody}
\textbf{Other contrasts:} N/A
\end{styleBody}

\begin{styleBody}
\textbf{Notes:} /f\textsuperscript{w}/ has merged with former /h\textsuperscript{w}/ for most younger speakers; most common pronunciation of the latter is now [f\textsuperscript{w}]. Distribution of /h\textsuperscript{j}/ is mostly restricted to word-initial position preceding /a u [289?]/. Most occurrences of /[272?]/ are predictable, but minimal pairs occur word-initially.. Author considers [d[361?][292?]] to be predictable, but there are a few near-minimal pairs with ‘complex conditioning’ (Olawsky 2006: 30-49). Vowel length distinction exists at grammatical level, but minimal pairs do not exist except for loans and morphologically complex forms; also vowel length in these contexts may be variable (Olawsky 2006: 56-7). /u/ varies with [o].
\end{styleBody}

\begin{styleBody}
\textbf{Syllable structure}
\end{styleBody}

\begin{styleBody}
\textbf{Complexity Category:} Simple
\end{styleBody}

\begin{styleBody}
\textbf{Canonical syllable structure:} (C)V (Olawsky 2006: 75-6)
\end{styleBody}

\begin{styleBody}
\textbf{Size of maximal onset: }1
\end{styleBody}

\begin{styleBody}
\textbf{Size of maximal coda:} N/A
\end{styleBody}

\begin{styleBody}
\textbf{Onset obligatory:} No
\end{styleBody}

\begin{styleBody}
\textbf{Coda obligatory:} N/A
\end{styleBody}

\begin{styleBody}
\textbf{Vocalic nucleus patterns:} Short vowels, Long vowels, Diphthongs
\end{styleBody}

\begin{styleBody}
\textbf{Syllabic consonant patterns:} N/A
\end{styleBody}

\begin{styleBody}
\textbf{Size of maximal word{}-marginal sequences with syllabic obstruents:} N/A
\end{styleBody}

\begin{styleBody}
\textbf{Predictability of syllabic consonants:} N/A
\end{styleBody}

\begin{styleBody}
\textbf{Morphological constituency of maximal syllable margin:} N/A
\end{styleBody}

\begin{styleBody}
\textbf{Morphological pattern of syllabic consonants:} N/A
\end{styleBody}

\begin{styleBody}
\textbf{Onset restrictions: }All consonants occur.
\end{styleBody}

\begin{styleBody}
\textbf{Coda restrictions: }N/A
\end{styleBody}

\begin{styleBody}
\textbf{Notes: }CCV syllables sometimes occur in Spanish loanwords in the speech of bilinguals, but these are usually split up by epenthesis (Olawsky 2006: 76).
\end{styleBody}

\begin{styleBody}
\textbf{Suprasegmentals}
\end{styleBody}

\begin{styleBody}
\textbf{Tone:} Yes
\end{styleBody}

\begin{styleBody}
\textbf{Word stress: }Yes
\end{styleBody}

\begin{styleBody}
\textbf{Stress placement:} Weight-Sensitive
\end{styleBody}

\begin{styleBody}
\textbf{Phonetic processes conditioned by stress:} (None)
\end{styleBody}

\begin{styleBody}
\textbf{Differences in phonological properties of stressed and unstressed syllables:} (None)
\end{styleBody}

\begin{styleBody}
\textbf{Phonetic correlates of stress: }Intensity (impressionistic)
\end{styleBody}

\begin{styleBody}
\textbf{Vowel reduction processes}
\end{styleBody}

\begin{styleBody}
(none reported)
\end{styleBody}

\begin{styleBody}
\textbf{Consonant allophony processes}
\end{styleBody}

\begin{styleBody}
\textbf{ura-C1: }A voiceless alveolar fricative is realized as palato-alveolar following /i/ in some dialects (Olawsky 2006: 38).
\end{styleBody}

\begin{styleBody}
\textbf{ura-C2: }An alveolo-palatal affricate [t[361?][255?]] is realized as a palato-alveolar affricate [d[361?][292?]] word-initially preceding /a/ or /[289?]/ (Olawsky 2006: 39).
\end{styleBody}

\begin{styleBody}
\textbf{ura-C3: }A sequence of /[27D?]i/ may vary freely with palatalized retroflex flap [[27D?]\textsuperscript{j}] or palato-alveolar affricate [d[361?][292?]] (Olawsky 2006: 71).
\end{styleBody}

\begin{styleBody}
\textbf{ura-C4: }A sequence of /ku/ may vary with [k\textsuperscript{w}] (Olawsky 2006: 37).
\end{styleBody}

\begin{styleBody}
\textbf{ura-C5: }Voiceless glottal fricative and alveolar nasal may be realized as palatalized following /i/ (Olawsky 2006: 47).
\end{styleBody}

\begin{styleBody}
\textbf{Morphology}
\end{styleBody}

\begin{styleBody}
\textbf{Text:} “Text 8,” “Text 10 (Olawsky 2006: 902-905)
\end{styleBody}

\begin{styleBody}
\textbf{Synthetic index: }1.6 morphemes/word (270 morphemes, 169 words)
\end{styleBody}

\clearpage\begin{styleBody}
\textbf{[ute] }\ \ \textbf{\textsc{Ute\ \ }}\textbf{\ \ }Uto-Aztecan, \textit{Northern Uto-Aztecan} (United States)
\end{styleBody}

\begin{styleBody}
\textbf{References consulted: }Givón (2011), Harms (1966), Oberly (2013)
\end{styleBody}

\begin{styleBody}
\textbf{Sound inventory}
\end{styleBody}

\begin{styleBody}
\textbf{C phoneme inventory:} /p t k [294?] t[361?][283?] $\beta $ s [263?] m n [27E?] w j/
\end{styleBody}

\begin{styleBody}
\textbf{N consonant phonemes:} 13
\end{styleBody}

\begin{styleBody}
\textbf{Geminates:} N/A
\end{styleBody}

\begin{styleBody}
\textbf{Voicing contrasts:} None
\end{styleBody}

\begin{styleBody}
\textbf{Places:} Bilabial, Dental, Palato-Alveolar, Velar, Glottal
\end{styleBody}

\begin{styleBody}
\textbf{Manners:} Stop, Affricate, Fricative, Nasal, Flap/Tap, Central approximant
\end{styleBody}

\begin{styleBody}
\textbf{N elaborations:} 2
\end{styleBody}

\begin{styleBody}
\textbf{Elaborations:} Voiced fricatives/affricates, Palato-alveolar
\end{styleBody}

\begin{styleBody}
\textbf{V phoneme inventory:} /i œ a [26F?] u i[2D0?] œ[2D0?] a[2D0?] [26F?][2D0?] u[2D0?] i[325?] œ[325?] a[325?] [26F?][325?] u[325?]/
\end{styleBody}

\begin{styleBody}
\textbf{N vowel qualities:} 5
\end{styleBody}

\begin{styleBody}
\textbf{Diphthongs or vowel sequences:} None
\end{styleBody}

\begin{styleBody}
\textbf{Contrastive length:} All
\end{styleBody}

\begin{styleBody}
\textbf{Contrastive nasalization:} None
\end{styleBody}

\begin{styleBody}
\textbf{Other contrasts:} Voicing
\end{styleBody}

\begin{styleBody}
\textbf{Notes:} Uvulars are allophones (at least historically) of /k/ or /[263?]/ (Givón 2011: 26). The voiced/voiceless distinction in vowels is a recent development, which does seem to be distinctive in certain grammatical contexts.
\end{styleBody}

\begin{styleBody}
\textbf{Syllable structure}
\end{styleBody}

\begin{styleBody}
\textbf{Complexity Category:} Simple
\end{styleBody}

\begin{styleBody}
\textbf{Canonical syllable structure:} CV(C)\textbf{ }(Givón 2011: 27-28)
\end{styleBody}

\begin{styleBody}
\textbf{Size of maximal onset:} 1
\end{styleBody}

\begin{styleBody}
\textbf{Size of maximal coda:} 1
\end{styleBody}

\begin{styleBody}
\textbf{Onset obligatory:} Yes
\end{styleBody}

\begin{styleBody}
\textbf{Coda obligatory:} No
\end{styleBody}

\begin{styleBody}
\textbf{Vocalic nucleus patterns:} Short vowels, Long vowels
\end{styleBody}

\begin{styleBody}
\textbf{Syllabic consonant patterns:} N/A
\end{styleBody}

\begin{styleBody}
\textbf{Size of maximal word{}-marginal sequences with syllabic obstruents:} N/A
\end{styleBody}

\begin{styleBody}
\textbf{Predictability of syllabic consonants:} N/A
\end{styleBody}

\begin{styleBody}
\textbf{Morphological constituency of maximal syllable margin:} N/A
\end{styleBody}

\begin{styleBody}
\textbf{Morphological pattern of syllabic consonants:} N/A
\end{styleBody}

\begin{styleBody}
\textbf{Onset restrictions:} All consonants occur.
\end{styleBody}

\begin{styleBody}
\textbf{Coda restrictions: }Optional coda may be /j/ or result from the recent deletion of a word-final vowel. 
\end{styleBody}

\begin{styleBody}
\textbf{Notes: }Givón argues that all apparent vowel-initial words are actually glottal stop-initial (2011: 27). See discussion of syllable patterns in Chapter 3.
\end{styleBody}

\begin{styleBody}
\textbf{Suprasegmentals}
\end{styleBody}

\begin{styleBody}
\textbf{Tone: }No
\end{styleBody}

\begin{styleBody}
\textbf{Word stress:} Yes
\end{styleBody}

\begin{styleBody}
\textbf{Stress placement:} Fixed
\end{styleBody}

\begin{styleBody}
\textbf{Phonetic processes conditioned by stress:} Vowel Reduction
\end{styleBody}

\begin{styleBody}
\textbf{Differences in phonological properties of stressed and unstressed syllables:} (None)
\end{styleBody}

\begin{styleBody}
\textbf{Phonetic correlates of stress: }Vowel duration (instrumental), Pitch (instrumental)
\end{styleBody}

\begin{styleBody}
\textbf{Notes: }Duration is a correlate of stress for short vowels only.
\end{styleBody}

\begin{styleBody}
\textbf{Vowel reduction processes}
\end{styleBody}

\begin{styleBody}
\textbf{ute-R1:} Unstressed and de-stressed vowels are devoiced word-finally (Givón 2011: 20-23; environments both phonological and grammatical).
\end{styleBody}

\begin{styleBody}
\textbf{ute-R2:} Unstressed and de-stressed vowels are sometimes devoiced word-initially (Givón 2011: 20-23; environments both phonological and grammatical).
\end{styleBody}

\begin{styleBody}
\textbf{ute-R3:} Vowels may become devoiced in unstressed syllables beginning with a voiceless consonant /k p t s t[361?][283?]/, a nasal /n m/, or a glide /w/ (Givón 2011: 21).
\end{styleBody}

\begin{styleBody}
\textbf{Consonant allophony processes}
\end{styleBody}

\begin{styleBody}
\textbf{ute-C1: }Velar stops are labialized following back rounded vowels (Givón 2011: 29).
\end{styleBody}

\begin{styleBody}
\textbf{ute-C2: }Velar stops are palatalized following high vowels (Givón 2011: 29).
\end{styleBody}

\begin{styleBody}
\textbf{ute-C3: }A voiceless bilabial stop is realized as a voiced labiodental fricative intervocalically (Givón 2011: 24; process rapidly phonemicizing).
\end{styleBody}

\begin{styleBody}
\textbf{ute-C4: }A voiceless alveolar stop is realized as a flap intervocalically (Givón 2011: 25; process rapidly phonemicizing).
\end{styleBody}

\begin{styleBody}
\textbf{ute-C5: }Velar stops are spirantized intervocalically (Givón 2011: 26-7).
\end{styleBody}

\begin{styleBody}
\textbf{ute-C6:} Velar stop /k/ has variant [q] and velar fricative /[263?]/ has variant [[281?]], between two low vowels and adjacent to mid back vowels.
\end{styleBody}

\begin{styleBody}
\textbf{ute-C7: }Velar stop /k/ has variant [$\chi $] between two low vowels and adjacent to mid back vowels.
\end{styleBody}

\begin{styleBody}
\textbf{Morphology}
\end{styleBody}

\begin{styleBody}
\textbf{Text:} “Porcupine tricks Coyote” (first 5 pages, Givón 2013: 107-111)
\end{styleBody}

\begin{styleBody}
\textbf{Synthetic index: }2.3 morphemes/word (593 morphemes, 255 words)
\end{styleBody}

\clearpage\begin{styleBody}
\textbf{[wba] }\textbf{\textsc{Warao}}\textbf{\ \ }isolate (Venezuela)
\end{styleBody}

\begin{styleBody}
\textbf{References consulted: }Arinterol (2000), Osborn (1966), Romero-Figeroa (1997)
\end{styleBody}

\begin{styleBody}
\textbf{Sound inventory}
\end{styleBody}

\begin{styleBody}
\textbf{C phoneme inventory:} /p t k k[2B7?] s h m n [27A?] w j/
\end{styleBody}

\begin{styleBody}
\textbf{N consonant phonemes:} 11
\end{styleBody}

\begin{styleBody}
\textbf{Geminates:} N/A
\end{styleBody}

\begin{styleBody}
\textbf{Voicing contrasts:} None
\end{styleBody}

\begin{styleBody}
\textbf{Places:} Bilabial, Alveolar, Velar, Glottal
\end{styleBody}

\begin{styleBody}
\textbf{Manners:} Stop, Fricative, Nasal, Central approximant, Lateral flap
\end{styleBody}

\begin{styleBody}
\textbf{N elaborations:} 1
\end{styleBody}

\begin{styleBody}
\textbf{Elaborations:} Labialization
\end{styleBody}

\begin{styleBody}
\textbf{V phoneme inventory:} /i e a o u/
\end{styleBody}

\begin{styleBody}
\textbf{N vowel qualities:} 5
\end{styleBody}

\begin{styleBody}
\textbf{Diphthongs or vowel sequences:} None
\end{styleBody}

\begin{styleBody}
\textbf{Contrastive length: }None
\end{styleBody}

\begin{styleBody}
\textbf{Contrastive nasalization:} None
\end{styleBody}

\begin{styleBody}
\textbf{Other contrasts:} N/A
\end{styleBody}

\begin{styleBody}
\textbf{Notes:} /[27A?]/ has [[27E?]] and [d] variants. Osborn reports phonemic nasal contrasts for each vowel, but Romero-Figueroa states these are phonologically conditioned (1997: 108).
\end{styleBody}

\begin{styleBody}
\textbf{Syllable structure}
\end{styleBody}

\begin{styleBody}
\textbf{Complexity Category:} Simple
\end{styleBody}

\begin{styleBody}
\textbf{Canonical syllable structure:} (C)V\textbf{ }(Romero-Figeroa 1997: 109-112)
\end{styleBody}

\begin{styleBody}
\textbf{Size of maximal onset:} 1
\end{styleBody}

\begin{styleBody}
\textbf{Size of maximal coda:} N/A
\end{styleBody}

\begin{styleBody}
\textbf{Onset obligatory:} No
\end{styleBody}

\begin{styleBody}
\textbf{Coda obligatory:} N/A
\end{styleBody}

\begin{styleBody}
\textbf{Vocalic nucleus patterns:} Short vowels
\end{styleBody}

\begin{styleBody}
\textbf{Syllabic consonant patterns:} N/A
\end{styleBody}

\begin{styleBody}
\textbf{Size of maximal word{}-marginal sequences with syllabic obstruents:} N/A
\end{styleBody}

\begin{styleBody}
\textbf{Predictability of syllabic consonants:} N/A
\end{styleBody}

\begin{styleBody}
\textbf{Morphological constituency of maximal syllable margin:} N/A
\end{styleBody}

\begin{styleBody}
\textbf{Morphological pattern of syllabic consonants:} N/A
\end{styleBody}

\begin{styleBody}
\textbf{Onset restrictions:} All consonants occur.
\end{styleBody}

\begin{styleBody}
\textbf{Suprasegmentals}
\end{styleBody}

\begin{styleBody}
\textbf{Tone:} No
\end{styleBody}

\begin{styleBody}
\textbf{Word stress:} Yes
\end{styleBody}

\begin{styleBody}
\textbf{Stress placement:} Fixed
\end{styleBody}

\begin{styleBody}
\textbf{Phonetic processes conditioned by stress:} (None)
\end{styleBody}

\begin{styleBody}
\textbf{Differences in phonological properties of stressed and unstressed syllables:} (None)
\end{styleBody}

\begin{styleBody}
\textbf{Phonetic correlates of stress: }Intensity (impressionistic)
\end{styleBody}

\begin{styleBody}
\textbf{Vowel reduction processes}
\end{styleBody}

\begin{styleBody}
(none reported)
\end{styleBody}

\begin{styleBody}
\textbf{Consonant allophony processes}
\end{styleBody}

\begin{styleBody}
\textbf{wba-C1: }Voiced alveolar stop and voiceless alveolar fricative are realized as voiced palato-alveolar affricate, and voiceless palato-alveolar fricative, respectively, following /i/ (Arinterol 2000: 121).
\end{styleBody}

\begin{styleBody}
\textbf{wba-C2: }Labial and palatal glides may vary freely with corresponding fricatives (Arinterol 2000: 122).
\end{styleBody}

\begin{styleBody}
\textbf{wba-C3: }Stops may become voiced preceding a vowel (Arinterol 1997: 107).
\end{styleBody}

\begin{styleBody}
\textbf{wba-C4: }Voiced alveolar stops are realized as flaps intervocalically (Arinterol 1997: 107).
\end{styleBody}

\begin{styleBody}
\textbf{Morphology}
\end{styleBody}

\begin{styleBody}
\textbf{Text:} (fragments of texts, Romero-Figeroa 1997: 118-123)
\end{styleBody}

\begin{styleBody}
\textbf{Synthetic index: }1.6 morphemes/word (182 morphemes, 116 words)
\end{styleBody}

\clearpage\begin{styleBody}
\textbf{[wmd] }\textbf{\textsc{Mamaindê}}\textbf{\ \ }Nambiquaran, \textit{Nambikwara Complex} (Brazil)
\end{styleBody}

\begin{styleBody}
\textbf{References consulted: }Eberhard (2009)
\end{styleBody}

\begin{styleBody}
\textbf{Sound inventory}
\end{styleBody}

\begin{styleBody}
\textbf{C phoneme inventory:} /p t k [294?] p[2B0?] t[2B0?] k[2B0?] s h m n l w j/
\end{styleBody}

\begin{styleBody}
\textbf{N consonant phonemes:} 14
\end{styleBody}

\begin{styleBody}
\textbf{Geminates:} N/A
\end{styleBody}

\begin{styleBody}
\textbf{Voicing contrasts:} None
\end{styleBody}

\begin{styleBody}
\textbf{Places:} Bilabial, Alveolar, Velar, Glottal
\end{styleBody}

\begin{styleBody}
\textbf{Manners:} Stop, Fricative, Nasal, Central approximant, Lateral approximant
\end{styleBody}

\begin{styleBody}
\textbf{N elaborations:} 1
\end{styleBody}

\begin{styleBody}
\textbf{Elaborations:} Post-aspiration
\end{styleBody}

\begin{styleBody}
\textbf{V phoneme inventory:} /i e a o u \~{i} \~{e} \~{a} \~{o} \~{u} i[330?] e[330?] a[330?] o[330?] u[330?] \~{i}[330?] \~{a}[330?] \~{u}[330?]/
\end{styleBody}

\begin{styleBody}
\textbf{N vowel qualities:} 5
\end{styleBody}

\begin{styleBody}
\textbf{Diphthongs or vowel sequences:} Diphthongs /iu i[330?]u[330?] ei e[330?]i[330?] eu ai a[330?]i[330?] au a[330?]u[330?] \~{i}\~{u} \~{i}[330?]\~{u}[330?] \~{e}\~{i} \~{e}\~{u} \~{a}\~{i} \~{a}[330?]\~{i}[330?] \~{a}\~{u} \~{a}[330?]u[330?]/
\end{styleBody}

\begin{styleBody}
\textbf{Contrastive length:} None
\end{styleBody}

\begin{styleBody}
\textbf{Contrastive nasalization:} Some
\end{styleBody}

\begin{styleBody}
\textbf{Other contrasts:} Creaky (Some)
\end{styleBody}

\begin{styleBody}
\textbf{Syllable structure}
\end{styleBody}

\begin{styleBody}
\textbf{Complexity Category:} Complex
\end{styleBody}

\begin{styleBody}
\textbf{Canonical syllable structure:} (C)(C)V(C)(C)\textbf{ }(Eberhard 2009: 124-34)
\end{styleBody}

\begin{styleBody}
\textbf{Size of maximal onset:} 2
\end{styleBody}

\begin{styleBody}
\textbf{Size of maximal coda:} 2
\end{styleBody}

\begin{styleBody}
\textbf{Onset obligatory:} No
\end{styleBody}

\begin{styleBody}
\textbf{Coda obligatory:} No
\end{styleBody}

\begin{styleBody}
\textbf{Vocalic nucleus patterns:} Short vowels, Diphthongs
\end{styleBody}

\begin{styleBody}
\textbf{Syllabic consonant patterns:} Nasal
\end{styleBody}

\begin{styleBody}
\textbf{Size of maximal word{}-marginal sequences with syllabic obstruents:} N/A
\end{styleBody}

\begin{styleBody}
\textbf{Predictability of syllabic consonants:} Varies with CV sequence
\end{styleBody}

\begin{styleBody}
\textbf{Morphological constituency of maximal syllable margin:} Morpheme-internal (Onset, Coda)
\end{styleBody}

\begin{styleBody}
\textbf{Morphological pattern of syllabic consonants:} N/A
\end{styleBody}

\begin{styleBody}
\textbf{Onset restrictions:} Any consonant may occur as simple onset. Biconsonantal onsets are /k[2B0?] t[2B0?] k h/+ /w/, or /h [294?]/+/l n j w s/ or /[294?]m/.
\end{styleBody}

\begin{styleBody}
\textbf{Coda restrictions:} Simple codas are stops and nasals. Biconsonantal codas consist of stop or nasal + /[294?]/.
\end{styleBody}

\begin{styleBody}
\textbf{Notes:} Complex onsets and codas (besides C+w onsets) always involve glottal, so this language doesn’t have ‘prototypically’ Complex syllable structure.
\end{styleBody}

\begin{styleBody}
\textbf{Suprasegmentals}
\end{styleBody}

\begin{styleBody}
\textbf{Tone:} Yes
\end{styleBody}

\begin{styleBody}
\textbf{Word stress:} Yes
\end{styleBody}

\begin{styleBody}
\textbf{Stress placement:} Morphologically or Lexically Conditioned
\end{styleBody}

\begin{styleBody}
\textbf{Phonetic processes conditioned by stress:} Vowel Reduction, Consonant Allophony in Unstressed Syllables, Consonant Allophony in Stressed Syllables
\end{styleBody}

\begin{styleBody}
\textbf{Differences in phonological properties of stressed and unstressed syllables:} (None)
\end{styleBody}

\begin{styleBody}
\textbf{Phonetic correlates of stress: }Vowel duration (instrumental), Intensity (instrumental)
\end{styleBody}

\begin{styleBody}
\textbf{Vowel reduction processes}
\end{styleBody}

\begin{styleBody}
\textbf{wmd-R1:} Unstressed syllables may optionally lose their vowel. This can result in syllabic consonants (Eberhard 2009: 262-3; only illustrated for nasals).
\end{styleBody}

\begin{styleBody}
\textbf{wmd-R2: }Non-front vowels /a o u/ are usually weakened in unstressed syllables (Eberhard 2009: 271; there is one morphological exception).
\end{styleBody}

\begin{styleBody}
\textbf{Consonant allophony processes}
\end{styleBody}

\begin{styleBody}
\textbf{wmd-C1: }A palatal glide is realized as a voiceless palato-alveolar affricate following a syllable-initial plosive (Eberhard 2009: 94).
\end{styleBody}

\begin{styleBody}
\textbf{wmd-C2: }An alveolar lateral approximant is realized as a voiceless alveolar stop following a syllable-initial oral obstruent (Eberhard 2009: 92).
\end{styleBody}

\begin{styleBody}
\textbf{wmd-C3: }Voiceless stops may become voiced adjacent to voiced sounds, especially word-initally preceding an unstressed vowel (Eberhard 2009).
\end{styleBody}

\begin{styleBody}
\textbf{wmd-C4: }Voiceless stops are voiced in a stressed onset (Eberhard 2009: 55).
\end{styleBody}

\begin{styleBody}
\textbf{wmd-C5: }Voiceless alveolar stop is realized as a voiced implosive when occurring preceding a stressed back vowel and occurring word-initially or following a glottal stop (Eberhard 2009: 58).
\end{styleBody}

\begin{styleBody}
\textbf{wmd-C6: }A voiceless alveolar stop is realized a a flap intervocalically preceding an unstressed vowel (Eberhard 2009: 55).
\end{styleBody}

\begin{styleBody}
\textbf{Morphology}
\end{styleBody}

\begin{styleBody}
(adequate texts unavailable)
\end{styleBody}

\clearpage\begin{styleBody}
\textbf{[wut] }\ \ \textbf{\textsc{Wutung}}\textbf{\ \ }Sko, \textit{Nuclear Skou-Serra-Piore} (Papua New Guinea)
\end{styleBody}

\begin{styleBody}
\textbf{References consulted: }Marmion (2010), Doug Marmion (p.c.)
\end{styleBody}

\begin{styleBody}
\textbf{Sound inventory}
\end{styleBody}

\begin{styleBody}
\textbf{C phoneme inventory:} /p b t d [294?] t[361?][283?] d[361?][292?] f s h m n [272?] l w/
\end{styleBody}

\begin{styleBody}
\textbf{N consonant phonemes:} 15
\end{styleBody}

\begin{styleBody}
\textbf{Geminates:} N/A
\end{styleBody}

\begin{styleBody}
\textbf{Voicing contrasts:} Obstruents
\end{styleBody}

\begin{styleBody}
\textbf{Places:} Bilabial, Labiodental, Alveolar, Palato-alveolar, Palatal, Glottal
\end{styleBody}

\begin{styleBody}
\textbf{Manners:} Stop, Affricate, Fricative, Nasal, Central approximant, Lateral approximant
\end{styleBody}

\begin{styleBody}
\textbf{N elaborations:} 3
\end{styleBody}

\begin{styleBody}
\textbf{Elaborations:} Voiced fricatives/affricates, Labiodental, Palato-alveolar
\end{styleBody}

\begin{styleBody}
\textbf{V phoneme inventory:} /i e [25B?] [275?] [250?] o [28A?] \~{i} \~{e} [25B?] [275?] [250?] \~{o} [28A?]/
\end{styleBody}

\begin{styleBody}
\textbf{N vowel qualities:} 7
\end{styleBody}

\begin{styleBody}
\textbf{Diphthongs or vowel sequences:} None
\end{styleBody}

\begin{styleBody}
\textbf{Contrastive length:} None
\end{styleBody}

\begin{styleBody}
\textbf{Contrastive nasalization:} Some
\end{styleBody}

\begin{styleBody}
\textbf{Other contrasts:} N/A
\end{styleBody}

\begin{styleBody}
\textbf{Notes:} /k/ occurs in one (possibly recent) borrowing. Contrastive nasalization for /i e [25B?] [250?] o [28A?]/.
\end{styleBody}

\begin{styleBody}
\textbf{Syllable structure}
\end{styleBody}

\begin{styleBody}
\textbf{Complexity Category: }Highly Complex
\end{styleBody}

\begin{styleBody}
\textbf{Canonical syllable structure:} (C)(C)(C)(C)V(C)\textbf{ }(Marmion 2010: 68-76)
\end{styleBody}

\begin{styleBody}
\textbf{Size of maximal onset:} 4
\end{styleBody}

\begin{styleBody}
\textbf{Size of maximal coda:} 1
\end{styleBody}

\begin{styleBody}
\textbf{Onset obligatory:} No
\end{styleBody}

\begin{styleBody}
\textbf{Coda obligatory:} No
\end{styleBody}

\begin{styleBody}
\textbf{Vocalic nucleus patterns:} Short vowels
\end{styleBody}

\begin{styleBody}
\textbf{Syllabic consonant patterns:} N/A
\end{styleBody}

\begin{styleBody}
\textbf{Size of maximal word{}-marginal sequences with syllabic obstruents:} N/A
\end{styleBody}

\begin{styleBody}
\textbf{Predictability of syllabic consonants:} N/A
\end{styleBody}

\begin{styleBody}
\textbf{Morphological constituency of maximal syllable margin:} Morpheme-internal (Onset)
\end{styleBody}

\begin{styleBody}
\textbf{Morphological pattern of syllabic consonants:} N/A
\end{styleBody}

\begin{styleBody}
\textbf{Onset restrictions:} All consonants occur as simple onsets. Biconsonantal onsets include /h [294?]/ + voiced consonant, or labial consonant + /l/. Triconsonantal onsets are /h [294?]/ + /b m/ + /l/, also /h[272?]d[361?][292?], hmb/. Only known example of four-consonant onset is /hmbl/.
\end{styleBody}

\begin{styleBody}
\textbf{Coda restrictions:} Codas occur in rare circumstances and are always nasals /m n/.
\end{styleBody}

\begin{styleBody}
\textbf{Suprasegmentals}
\end{styleBody}

\begin{styleBody}
\textbf{Tone:} Yes
\end{styleBody}

\begin{styleBody}
\textbf{Word stress:} Yes
\end{styleBody}

\begin{styleBody}
\textbf{Stress placement:} Unpredictable/Variable
\end{styleBody}

\begin{styleBody}
\textbf{Phonetic processes conditioned by stress:} Consonant Allophony in Unstressed Syllables
\end{styleBody}

\begin{styleBody}
\textbf{Differences in phonological properties of stressed and unstressed syllables:} (None)
\end{styleBody}

\begin{styleBody}
\textbf{Phonetic correlates of stress: }Vowel duration (impressionistic), Intensity (impressionistic)
\end{styleBody}

\begin{styleBody}
\textbf{Vowel reduction processes}
\end{styleBody}

\begin{styleBody}
(none reported)
\end{styleBody}

\begin{styleBody}
\textbf{Consonant allophony processes}
\end{styleBody}

\begin{styleBody}
\textbf{wut-C1: }A labiovelar glide may be realized as a bilabial fricative word-initially preceding a vowel (Marmion 2010: 57-8).
\end{styleBody}

\begin{styleBody}
\textbf{wut-C2: }A labiovelar glide may be realized as a bilabial fricative intervocalically (Marmion 2010: 57-8).
\end{styleBody}

\begin{styleBody}
\textbf{wut-C3: }A voiced palato-alveolar affricate may be realized as a glide intervocalically (Marmion 2010: 55).
\end{styleBody}

\begin{styleBody}
\textbf{Morphology}
\end{styleBody}

\begin{styleBody}
\textbf{Text:} “Crow and white cockatoo,” “Womia the mermaid” (Marmion 2010: 378-382)
\end{styleBody}

\begin{styleBody}
\textbf{Synthetic index: }1.1 morphemes/word (340 morphemes, 313 words)
\end{styleBody}

\clearpage\begin{styleBody}
\textbf{[yak] }\ \ \textbf{\textsc{Yakima Sahaptin}}\textbf{\ \ }Sahaptian, \textit{Sahaptin} (United States)
\end{styleBody}

\begin{styleBody}
\textbf{References consulted: }Hargus \& Beavert (2002, 2005, 2006), Jansen (2010), Minthorn (2005), Rigsby \& Rude (1996), Rude (2009), Rude \& CTUIR (2014)
\end{styleBody}

\begin{styleBody}
\textbf{Sound inventory}
\end{styleBody}

\begin{styleBody}
\textbf{C phoneme inventory:} /p t k k[2B7?] q q[2B7?] [294?] p’ t’ k’ k’[2B7?] q’ q’[2B7?] t[361?][26C?] t[361?]s t[361?][283?] t[361?][26C?]’ t[361?]s’ t[361?][283?]’ [26C?] s [283?] x x[2B7?] $\chi $ $\chi [2B7?]$ h m n l w j/
\end{styleBody}

\begin{styleBody}
\textbf{N consonant phonemes:} 32
\end{styleBody}

\begin{styleBody}
\textbf{Geminates:} N/A
\end{styleBody}

\begin{styleBody}
\textbf{Voicing contrasts:} None
\end{styleBody}

\begin{styleBody}
\textbf{Places:} Bilabial, Dental, Palato-alveolar, Velar, Uvular, Glottal
\end{styleBody}

\begin{styleBody}
\textbf{Manners:} Stop, Affricate, Fricative, Nasal, Central approximant, Lateral affricate, Lateral fricative, Lateral approximant
\end{styleBody}

\begin{styleBody}
\textbf{N elaborations:} 5
\end{styleBody}

\begin{styleBody}
\textbf{Elaborations:} Lateral release, Ejective, Palato-alveolar, Uvular, Labialization
\end{styleBody}

\begin{styleBody}
\textbf{V phoneme inventory:} /i [268?] a u i[2D0?] a[2D0?] u[2D0?]/
\end{styleBody}

\begin{styleBody}
\textbf{N vowel qualities:} 4
\end{styleBody}

\begin{styleBody}
\textbf{Diphthongs or vowel sequences:} None
\end{styleBody}

\begin{styleBody}
\textbf{Contrastive length:} Some
\end{styleBody}

\begin{styleBody}
\textbf{Contrastive nasalization:} None
\end{styleBody}

\begin{styleBody}
\textbf{Other contrasts:} N/A
\end{styleBody}

\begin{styleBody}
\textbf{Notes:} Vowel length contrast for /i a u/. Note that Jansen, Rigsby \& Rude posit diphthongs; Hargus \& Beavert (2006) argue against this on evidence from phonological processes.
\end{styleBody}

\begin{styleBody}
\textbf{Syllable structure}
\end{styleBody}

\begin{styleBody}
\textbf{Complexity Category:} Highly Complex
\end{styleBody}

\begin{styleBody}
\textbf{Canonical syllable structure:} C(C)(C)(C)V(C)(C)(C)(C)\textbf{ }(Hargus \& Beavert 2006, 2002; Rigsby \& Rude 1996: 671)
\end{styleBody}

\begin{styleBody}
\textbf{Size of maximal onset:} 4
\end{styleBody}

\begin{styleBody}
\textbf{Size of maximal coda:} 4
\end{styleBody}

\begin{styleBody}
\textbf{Onset obligatory:} Yes
\end{styleBody}

\begin{styleBody}
\textbf{Coda obligatory:} No
\end{styleBody}

\begin{styleBody}
\textbf{Vocalic nucleus patterns:} Short vowels, Long vowels, Diphthongs
\end{styleBody}

\begin{styleBody}
\textbf{Syllabic consonant patterns:} N/A
\end{styleBody}

\begin{styleBody}
\textbf{Size of maximal word{}-marginal sequences with syllabic obstruents:} N/A
\end{styleBody}

\begin{styleBody}
\textbf{Predictability of syllabic consonants:} N/A
\end{styleBody}

\begin{styleBody}
\textbf{Morphological constituency of maximal syllable margin:} Morpheme-internal (Onset), Both patterns (Coda)
\end{styleBody}

\begin{styleBody}
\textbf{Morphological pattern of syllabic consonants:} N/A
\end{styleBody}

\begin{styleBody}
\textbf{Onset restrictions:} All consonants occur as simple onsets. Biconsonantal onsets may be obstruent+sonorant, obstruent+obstruent, sonorant+sonorant, sonorant+obstruent and include /$\chi $n pt qt[361?][26C?] mj tw qn lt t[361?][283?]t[361?][283?]/. Triconsonantal onsets are quite numerous and include sequences of three obstruents, e.g. /p[283?]$\chi $, tk[2B7?]s/, as well as combinations with sonorants, e.g. /tmt, t$\chi $n/. Onsets of four consonants occurring morpheme-internally include /[283?]t$\chi $n, ksks/, with perhaps more when morphologically complex forms are considered (but there are few candidate prefixes).
\end{styleBody}

\begin{styleBody}
\textbf{Coda restrictions:} /h [294?]/ do not occur in simple codas. Biconsonantal codas are apparently unrestricted, include /tk t’k q[2B7?]'k ms wn/. Triconsonantal codas include /tks stk pt[361?][26C?]’k/. Hargus \& Beavert (2006) list /wtk[2B7?][283?] wq’$\chi [283?]$ jlps/ as four-consonant codas.
\end{styleBody}

\begin{styleBody}
\textbf{Notes: }Clusters of glottalized or labialized obstruents do not occur (in reduplication contexts, their plain forms appear), but clusters of identical rearticulated plain consonants, e.g. /pp/, /qq/, do appear and are common. There seem to be place restrictions on initial obstruent sequences: except for identical rearticulated clusters, sequences of homorganic consonants generally don’t occur, and dorsal+labial sequences are common but labial+dorsal sequences occur in one item (Hargus \& Beavert 2002: 237). Meanwhile, there are no coherent place restrictions on YS final clusters (2002: 239). Rigsby \& Rude (1996) state that initial clusters are maximally three members, but Hargus \& Beavert (2006) give evidence from phonological processes that glide is a consonant and not part of previous vowel (diphthong). Minimal words in YS are CCV or CVC (Hargus \& Beavert 2006). Hargus \& Beavert reject syllabic obstruents for YS but Minthorn (2005) posits syllabic obstruents for related dialect Umatilla based on instrumental evidence and speaker judgments.
\end{styleBody}

\begin{styleBody}
\textbf{Suprasegmentals}
\end{styleBody}

\begin{styleBody}
\textbf{Tone:} No
\end{styleBody}

\begin{styleBody}
\textbf{Word stress:} Yes
\end{styleBody}

\begin{styleBody}
\textbf{Stress placement:} Morphologically or Lexically Conditioned
\end{styleBody}

\begin{styleBody}
\textbf{Phonetic processes conditioned by stress:} Vowel Reduction
\end{styleBody}

\begin{styleBody}
\textbf{Differences in phonological properties of stressed and unstressed syllables:} (None)
\end{styleBody}

\begin{styleBody}
\textbf{Phonetic correlates of stress: }Pitch (instrumental), Intensity (instrumental)
\end{styleBody}

\begin{styleBody}
\textbf{Notes: }Hargus and Beavert (2005) describe language as having pitch accent.
\end{styleBody}

\begin{styleBody}
\textbf{Vowel reduction processes}
\end{styleBody}

\begin{styleBody}
\textbf{yak-R1}: Short vowels may be realized as lax when unstressed (Jansen 2010: 40).
\end{styleBody}

\begin{styleBody}
\textbf{yak-R2}: Short vowels may be realized as lax in rapid speech (Jansen 2010: 40).
\end{styleBody}

\begin{styleBody}
\textbf{Consonant allophony processes}
\end{styleBody}

\begin{styleBody}
\textbf{yak-C1: }An ejective dental stop varies with affricate variant [t[361?]$\theta $’] in all environments (Rigsby \& Rude 1996: 669).
\end{styleBody}

\begin{styleBody}
\textbf{yak-C2: }Voiceless velar stops are fronted preceding high front vowels and palatal glides (Rigsby \& Rude 1996: 667).
\end{styleBody}

\begin{styleBody}
\textbf{Morphology}
\end{styleBody}

\begin{styleBody}
\textbf{Text:} “Coyote and Prairie Chicken” (first 15 pages, Jansen 2010: 444-458)
\end{styleBody}

\begin{styleBody}
\textbf{Synthetic index: }1.8 morphemes/word (575 morphemes, 324 words)
\end{styleBody}

\clearpage\begin{styleBody}
\textbf{[yor] }\ \ \textbf{\textsc{Yoruba}}\textbf{\ \ }Atlantic-Congo, \textit{Volta-Congo} (Benin, Nigeria)
\end{styleBody}

\begin{styleBody}
\textbf{References consulted: }Bamgbose (1966), Rowlands (1969), Seidl (2000), Siertsema (1959)
\end{styleBody}

\begin{styleBody}
\textbf{Sound inventory}
\end{styleBody}

\begin{styleBody}
\textbf{C phoneme inventory:} /b t d [25F?] k [261?] k[361?]p [261?][361?]b f s [283?] h m l [27E?] j w/
\end{styleBody}

\begin{styleBody}
\textbf{N consonant phonemes:} 17
\end{styleBody}

\begin{styleBody}
\textbf{Geminates:} N/A
\end{styleBody}

\begin{styleBody}
\textbf{Voicing contrasts:} Obstruents
\end{styleBody}

\begin{styleBody}
\textbf{Places:} Labial-velar, Bilabial, Labiodental, Alveolar, Palato-Alveolar, Velar, Glottal
\end{styleBody}

\begin{styleBody}
\textbf{Manners:} Stop, Fricative, Nasal, Flap/Tap, Lateral approximant, Central approximant
\end{styleBody}

\begin{styleBody}
\textbf{N elaborations:} 2
\end{styleBody}

\begin{styleBody}
\textbf{Elaborations:} Labiodental, Palato-alveolar
\end{styleBody}

\begin{styleBody}
\textbf{V phoneme inventory:} /i e [25B?] a [254?] o u \~{i} [25B?] [254?] \~{u}/
\end{styleBody}

\begin{styleBody}
\textbf{N vowel qualities:} 7
\end{styleBody}

\begin{styleBody}
\textbf{Diphthongs or vowel sequences:} None
\end{styleBody}

\begin{styleBody}
\textbf{Contrastive length:} None
\end{styleBody}

\begin{styleBody}
\textbf{Contrastive nasalization:} Some
\end{styleBody}

\begin{styleBody}
\textbf{Other contrasts:} None
\end{styleBody}

\begin{styleBody}
\textbf{Syllable structure}
\end{styleBody}

\begin{styleBody}
\textbf{Complexity Category:} Simple
\end{styleBody}

\begin{styleBody}
\textbf{Canonical syllable structure:} (C)V\textbf{ }(Bamgbose 1966: 6)
\end{styleBody}

\begin{styleBody}
\textbf{Size of maximal onset:} 1
\end{styleBody}

\begin{styleBody}
\textbf{Size of maximal coda:} N/A
\end{styleBody}

\begin{styleBody}
\textbf{Onset obligatory:} No
\end{styleBody}

\begin{styleBody}
\textbf{Coda obligatory:} N/A
\end{styleBody}

\begin{styleBody}
\textbf{Vocalic nucleus patterns:} Short vowels
\end{styleBody}

\begin{styleBody}
\textbf{Syllabic consonant patterns:} Nasal
\end{styleBody}

\begin{styleBody}
\textbf{Size of maximal word{}-marginal sequences with syllabic obstruents:} N/A
\end{styleBody}

\begin{styleBody}
\textbf{Predictability of syllabic consonants:} Phonemic
\end{styleBody}

\begin{styleBody}
\textbf{Morphological constituency of maximal syllable margin:} N/A
\end{styleBody}

\begin{styleBody}
\textbf{Morphological pattern of syllabic consonants:} Grammatical items
\end{styleBody}

\begin{styleBody}
\textbf{Onset restrictions: }All consonants occur.
\end{styleBody}

\begin{styleBody}
\textbf{Suprasegmentals}
\end{styleBody}

\begin{styleBody}
\textbf{Tone: }Yes
\end{styleBody}

\begin{styleBody}
\textbf{Word stress:} No
\end{styleBody}

\begin{styleBody}
\textbf{Vowel reduction processes}
\end{styleBody}

\begin{styleBody}
(none reported)
\end{styleBody}

\begin{styleBody}
\textbf{Consonant allophony processes}
\end{styleBody}

\begin{styleBody}
(none reported) 
\end{styleBody}

\begin{styleBody}
\textbf{Morphology}
\end{styleBody}

\begin{styleBody}
(adequate texts unavailable)
\end{styleBody}

\clearpage\begin{styleBody}
\textbf{[yue] }\ \ \textbf{\textsc{Cantonese}}\textbf{\ \ }Sino-Tibetan, \textit{Sinitic} (China)
\end{styleBody}

\begin{styleBody}
\textbf{References consulted: }Bauer \& Benedict (1997), Matthews \& Yip (1994)
\end{styleBody}

\begin{styleBody}
\textbf{Sound inventory}
\end{styleBody}

\begin{styleBody}
\textbf{N consonant phonemes: }/p t k k[2B7?] p[2B0?] t[2B0?] k[2B0?] k[2B7?][2B0?] t[361?]s t[361?]s[2B0?] f s h m n ŋ l j w/
\end{styleBody}

\begin{styleBody}
\textbf{Geminates: }N/A
\end{styleBody}

\begin{styleBody}
\textbf{Voicing contrasts: }None
\end{styleBody}

\begin{styleBody}
\textbf{Places: }Bilabial, Labiodental, Dental/Alveolar, Velar, Glottal
\end{styleBody}

\begin{styleBody}
\textbf{Manners: }Stop, Affricate, Fricative, Nasal, Central approximant, Lateral approximant
\end{styleBody}

\begin{styleBody}
\textbf{N elaborations:} 3
\end{styleBody}

\begin{styleBody}
\textbf{Elaborations: }Post-aspiration, Labiodental, Labialization
\end{styleBody}

\begin{styleBody}
\textbf{V phoneme inventory:} /i y e ø a [251?] [254?] u/
\end{styleBody}

\begin{styleBody}
\textbf{N vowel qualities:} 8
\end{styleBody}

\begin{styleBody}
\textbf{Diphthongs or vowel sequences:} Diphthongs /ai [251?]i au [251?]u ei øi iu ui oi ou/
\end{styleBody}

\begin{styleBody}
\textbf{Contrastive length:} None
\end{styleBody}

\begin{styleBody}
\textbf{Contrastive nasalization:} None
\end{styleBody}

\begin{styleBody}
\textbf{Other contrasts:} N/A
\end{styleBody}

\begin{styleBody}
\textbf{Notes:} /[251?]/ longer than /a/ but there is also a quality distinction. Bauer \& Benedict disagree with this vowel system and propose a 14-vowel system with length distinctions (1997: 45-48).
\end{styleBody}

\begin{styleBody}
\textbf{Syllable structure}
\end{styleBody}

\begin{styleBody}
\textbf{Complexity Category:} Moderately Complex
\end{styleBody}

\begin{styleBody}
\textbf{Canonical syllable structure:} (C)V(C)\textbf{ }(Matthews \& Yip 1994: 16-20, Bauer \& Benedict 1997)
\end{styleBody}

\begin{styleBody}
\textbf{Size of maximal onset:} 1
\end{styleBody}

\begin{styleBody}
\textbf{Size of maximal coda:} 1
\end{styleBody}

\begin{styleBody}
\textbf{Onset obligatory:} No
\end{styleBody}

\begin{styleBody}
\textbf{Coda obligatory:} No
\end{styleBody}

\begin{styleBody}
\textbf{Vocalic nucleus patterns:} Short vowels, Long vowels, Diphthongs
\end{styleBody}

\begin{styleBody}
\textbf{Syllabic consonant patterns:} Nasal
\end{styleBody}

\begin{styleBody}
\textbf{Size of maximal word{}-marginal sequences with syllabic obstruents:} N/A
\end{styleBody}

\begin{styleBody}
\textbf{Predictability of syllabic consonants:} Phonemic
\end{styleBody}

\begin{styleBody}
\textbf{Morphological constituency of maximal syllable margin:} N/A
\end{styleBody}

\begin{styleBody}
\textbf{Morphological pattern of syllabic consonants:} Both
\end{styleBody}

\begin{styleBody}
\textbf{Onset restrictions:} All consonants occur.
\end{styleBody}

\begin{styleBody}
\textbf{Coda restrictions:} Limited to /m n ŋ p t k/ (Matthews \& Yip) and semi-vowels (Bauer \& Benedict).
\end{styleBody}

\begin{styleBody}
\textbf{Suprasegmentals}
\end{styleBody}

\begin{styleBody}
\textbf{Tone: }Yes
\end{styleBody}

\begin{styleBody}
\textbf{Word stress: }No
\end{styleBody}

\begin{styleBody}
\textbf{Vowel reduction processes}
\end{styleBody}

\begin{styleBody}
(none reported)
\end{styleBody}

\begin{styleBody}
\textbf{Consonant allophony processes}
\end{styleBody}

\begin{styleBody}
\textbf{yue-C1: }A voiceless alveolar fricative is realized as a palatal or palato-alveolar fricative preceding /y/ (also preceding /i[2D0?]/ in Guangzhou dialect) (Bauer \& Benedict 1997: 28-9).
\end{styleBody}

\begin{styleBody}
\textbf{yue-C2: }Alveolar affricates are realized as palato-alveolar preceding front vowels, front and central rounded vowels /i[2D0?] y[2D0?] œ[2D0?] [275?]/ (Bauer \& Benedict 1997: 29-30).
\end{styleBody}

\begin{styleBody}
\textbf{Morphology}
\end{styleBody}

\begin{styleBody}
(adequate texts unavailable)
\end{styleBody}
\end{document}
