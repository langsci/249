% This file was converted to LaTeX by Writer2LaTeX ver. 1.4
% see http://writer2latex.sourceforge.net for more info
\documentclass[12pt]{article}
\usepackage[utf8]{inputenc}
\usepackage[T1]{fontenc}
\usepackage[english]{babel}
\usepackage{amsmath}
\usepackage{amssymb,amsfonts,textcomp}
\usepackage{array}
\usepackage{supertabular}
\usepackage{hhline}
\usepackage{hyperref}
\hypersetup{colorlinks=true, linkcolor=blue, citecolor=blue, filecolor=blue, urlcolor=blue}
\usepackage{graphicx}
% footnotes configuration
\makeatletter
\renewcommand\thefootnote{\arabic{footnote}}
\renewcommand\@makefnmark{\mbox{\textstyleFootnoteanchor{\@thefnmark}}}
\makeatother
\newcommand\textsubscript[1]{\ensuremath{{}_{\text{#1}}}}
% Text styles
\newcommand\textstyleFootnoteanchor[1]{\textsuperscript{#1}}
% Headings and outline numbering
\makeatletter
\renewcommand\section{\@startsection{section}{1}{0.0in}{0in}{0.1mm}{\normalfont\normalsize\fontsize{18pt}{21.6pt}\selectfont\rmfamily\bfseries\upshape\raggedright}}
\renewcommand\subsection{\@startsection{subsection}{2}{0.0in}{0in}{0.1mm}{\normalfont\normalsize\fontsize{30pt}{36.0pt}\selectfont\rmfamily\bfseries\upshape\raggedright}}
\renewcommand\@seccntformat[1]{\csname @textstyle#1\endcsname{\csname the#1\endcsname}\csname @distance#1\endcsname}
\setcounter{secnumdepth}{0}
\newcommand\@distancesection{}
\newcommand\@textstylesection[1]{#1}
\newcommand\@distancesubsection{}
\newcommand\@textstylesubsection[1]{#1}
\makeatother
\makeatletter
\newcommand\arraybslash{\let\\\@arraycr}
\makeatother
\raggedbottom
% Paragraph styles
\renewcommand\familydefault{\rmdefault}
\renewcommand\seriesdefault{\mddefault}
\renewcommand\shapedefault{\updefault}
\newenvironment{styleSubtitle}{\renewcommand\baselinestretch{1.0}\setlength\leftskip{0in}\setlength\rightskip{0in plus 1fil}\setlength\parindent{0in}\setlength\parfillskip{0pt plus 1fil}\setlength\parskip{0in plus 1pt}\writerlistparindent\writerlistleftskip\leavevmode\normalfont\normalsize\fontsize{20pt}{24.0pt}\selectfont\mdseries\upshape\writerlistlabel\ignorespaces}{\unskip\vspace{0in plus 1pt}\par}
\newenvironment{styleBody}{\renewcommand\baselinestretch{1.0}\setlength\leftskip{0in}\setlength\rightskip{0in plus 1fil}\setlength\parindent{0in}\setlength\parfillskip{0pt plus 1fil}\setlength\parskip{0in plus 1pt}\writerlistparindent\writerlistleftskip\leavevmode\normalfont\normalsize\fontsize{11pt}{13.2pt}\selectfont\mdseries\upshape\writerlistlabel\ignorespaces}{\unskip\vspace{0in plus 1pt}\par}
\newenvironment{styleContentsii}{\renewcommand\baselinestretch{1.0}\setlength\leftskip{0in}\setlength\rightskip{0in plus 1fil}\setlength\parindent{0in}\setlength\parfillskip{0pt plus 1fil}\setlength\parskip{0in plus 1pt}\writerlistparindent\writerlistleftskip\leavevmode\normalfont\normalsize\fontsize{14pt}{16.8pt}\selectfont\mdseries\upshape\writerlistlabel\ignorespaces}{\unskip\vspace{0.0835in plus 0.00835in}\par}
\newenvironment{styleContentsi}{\renewcommand\baselinestretch{1.0}\setlength\leftskip{0.1665in}\setlength\rightskip{0in plus 1fil}\setlength\parindent{0in}\setlength\parfillskip{0pt plus 1fil}\setlength\parskip{0in plus 1pt}\writerlistparindent\writerlistleftskip\leavevmode\normalfont\normalsize\mdseries\upshape\writerlistlabel\ignorespaces}{\unskip\vspace{0.0835in plus 0.00835in}\par}
\newenvironment{styleTableStyleii}{\renewcommand\baselinestretch{1.0}\setlength\leftskip{0in}\setlength\rightskip{0in plus 1fil}\setlength\parindent{0in}\setlength\parfillskip{0pt plus 1fil}\setlength\parskip{0in plus 1pt}\writerlistparindent\writerlistleftskip\leavevmode\normalfont\normalsize\fontsize{10pt}{12.0pt}\selectfont\mdseries\upshape\writerlistlabel\ignorespaces}{\unskip\vspace{0in plus 1pt}\par}
\newenvironment{styleStandard}{\renewcommand\baselinestretch{1.0}\setlength\leftskip{0in}\setlength\rightskip{0in plus 1fil}\setlength\parindent{0in}\setlength\parfillskip{0pt plus 1fil}\setlength\parskip{0in plus 1pt}\writerlistparindent\writerlistleftskip\leavevmode\normalfont\normalsize\writerlistlabel\ignorespaces}{\unskip\vspace{0in plus 1pt}\par}
% List styles
\newcommand\writerlistleftskip{}
\newcommand\writerlistparindent{}
\newcommand\writerlistlabel{}
\newcommand\writerlistremovelabel{\aftergroup\let\aftergroup\writerlistparindent\aftergroup\relax\aftergroup\let\aftergroup\writerlistlabel\aftergroup\relax}
\setlength\tabcolsep{1mm}
\renewcommand\arraystretch{1.3}
\title{}
\author{}
\date{2019-06-14}
\begin{document}
\begin{styleSubtitle}\bfseries
Highly complex syllable structure: a typological and diachronic study
\end{styleSubtitle}

\begin{styleBody}
Shelece Easterday
\end{styleBody}

\begin{styleBody}
Manuscript, version 14 June 2019
\end{styleBody}

\clearpage\begin{styleBody}
For Elise, Ada, Astrid, and Maria:\newline
May you always be curious!
\end{styleBody}

\begin{styleContentsii}
Abbreviations\ \ 10
\end{styleContentsii}

\begin{styleContentsii}
Acknowledgements\ \ 11
\end{styleContentsii}

\begin{styleContentsii}
Chapter 1: Syllables and syllable structure\ \ 12
\end{styleContentsii}

\begin{styleContentsi}
1.1 Background\ \ 13
\end{styleContentsi}

\begin{styleContentsi}
1.1.1 The syllable\ \ 13
\end{styleContentsi}

\begin{styleContentsi}
1.1.2 Crosslinguistic patterns in syllable structure\ \ 14
\end{styleContentsi}

\begin{styleContentsi}
1.1.2.1 CV as a universal syllable type\ \ 15
\end{styleContentsi}

\begin{styleContentsi}
1.1.2.2 Asymmetries in onset and coda patterns\ \ 15
\end{styleContentsi}

\begin{styleContentsi}
1.1.2.3 Consonant clusters\ \ 17
\end{styleContentsi}

\begin{styleContentsi}
1.1.2.4 Nucleus patterns\ \ 19
\end{styleContentsi}

\begin{styleContentsi}
1.1.2.5 Syllable structure and morphology\ \ 20
\end{styleContentsi}

\begin{styleContentsi}
1.1.3 Theoretical models and crosslinguistic patterns of syllable structure\ \ 21
\end{styleContentsi}

\begin{styleContentsi}
1.2 Highly complex syllable structure: typological outlier, theoretical problem\ \ 22
\end{styleContentsi}

\begin{styleContentsi}
1.3 Syllable structure complexity: accounts and correlations\ \ 24
\end{styleContentsi}

\begin{styleContentsi}
1.3.1 Speech rhythm typologies\ \ 24
\end{styleContentsi}

\begin{styleContentsi}
1.3.2 Other holistic typologies\ \ 25
\end{styleContentsi}

\begin{styleContentsi}
1.3.3 Consonantal and vocalic languages\ \ 26
\end{styleContentsi}

\begin{styleContentsi}
1.4 The current study\ \ 28
\end{styleContentsi}

\begin{styleContentsi}
1.4.1 Research questions\ \ 29
\end{styleContentsi}

\begin{styleContentsi}
1.4.2 Proposed analyses and framework\ \ 29
\end{styleContentsi}

\begin{styleContentsii}
Chapter 2: Language sample\ \ 32
\end{styleContentsii}

\begin{styleContentsi}
2.1 Language sampling\ \ 32
\end{styleContentsi}

\begin{styleContentsi}
2.1.1 Common sources of bias in language sampling\ \ 32
\end{styleContentsi}

\begin{styleContentsi}
2.1.2 Other factors which may influence phonological structure and syllable complexity\ \ 34
\end{styleContentsi}

\begin{styleContentsi}
2.1.2.1 Population\ \ 34
\end{styleContentsi}

\begin{styleContentsi}
2.1.2.2 Language endangerment\ \ 34
\end{styleContentsi}

\begin{styleContentsi}
2.1.2.3 Ecological factors\ \ 35
\end{styleContentsi}

\begin{styleContentsi}
2.1.3 Specific considerations in the current study\ \ 35
\end{styleContentsi}

\begin{styleContentsi}
2.2 Defining the categories of syllable structure complexity\ \ 36
\end{styleContentsi}

\begin{styleContentsi}
2.3 Constructing the language sample\ \ 42
\end{styleContentsi}

\begin{styleContentsi}
2.4 Language sample for current study\ \ 43
\end{styleContentsi}

\begin{styleContentsi}
2.4.1 Areal features of sample\ \ 46
\end{styleContentsi}

\begin{styleContentsi}
2.4.2 Genealogical features of sample\ \ 47
\end{styleContentsi}

\begin{styleContentsi}
2.4.3 Sociolinguistic features of sample\ \ 48
\end{styleContentsi}

\begin{styleContentsi}
2.5 Data collection\ \ 49
\end{styleContentsi}

\begin{styleContentsii}
Chapter 3: Syllable structure patterns in sample\ \ 50
\end{styleContentsii}

\begin{styleContentsi}
3.1 Introduction\ \ 50
\end{styleContentsi}

\begin{styleContentsi}
3.1.1 Crosslinguistic studies of syllable structure\ \ 50
\end{styleContentsi}

\begin{styleContentsi}
3.1.2 Considerations in the current chapter\ \ 52
\end{styleContentsi}

\begin{styleContentsi}
3.2 Methodology\ \ 54
\end{styleContentsi}

\begin{styleContentsi}
3.2.1 Patterns considered\ \ 54
\end{styleContentsi}

\begin{styleContentsi}
3.2.2 Status of inserted vowels\ \ 56
\end{styleContentsi}

\begin{styleContentsi}
3.2.3 Edges of categories\ \ 59
\end{styleContentsi}

\begin{styleContentsi}
3.2.4 Coding\ \ 61
\end{styleContentsi}

\begin{styleContentsi}
3.3 Results for language sample\ \ 62
\end{styleContentsi}

\begin{styleContentsi}
3.3.1 Maximal onset and coda sizes\ \ 62
\end{styleContentsi}

\begin{styleContentsi}
3.3.2 Relationship between onset and coda complexity\ \ 64
\end{styleContentsi}

\begin{styleContentsi}
3.3.3 Syllable structure complexity and obligatoriness of syllable margins\ \ 65
\end{styleContentsi}

\begin{styleContentsi}
3.3.4 Vocalic nucleus patterns\ \ 67
\end{styleContentsi}

\begin{styleContentsi}
3.3.5 Syllabic consonants\ \ 68
\end{styleContentsi}

\begin{styleContentsi}
3.3.6 Morphological patterns\ \ 72
\end{styleContentsi}

\begin{styleContentsi}
3.4 Properties of highly complex syllable structure\ \ 79
\end{styleContentsi}

\begin{styleContentsi}
3.4.1 Examples of Highly Complex syllable patterns in sample\ \ 79
\end{styleContentsi}

\begin{styleContentsi}
3.4.2 Prevalence of Highly Complex syllable patterns within languages\ \ 82
\end{styleContentsi}

\begin{styleContentsi}
3.4.3 Acoustic and perceptual characteristics\ \ 88
\end{styleContentsi}

\begin{styleContentsi}
3.5. Discussion\ \ 91
\end{styleContentsi}

\begin{styleContentsii}
Chapter 4: Phoneme inventories and syllable structure complexity\ \ 95
\end{styleContentsii}

\begin{styleContentsi}
4.1 Introduction\ \ 95
\end{styleContentsi}

\begin{styleContentsi}
4.1.1 Crosslinguistic patterns in consonant inventories\ \ 96
\end{styleContentsi}

\begin{styleContentsi}
4.1.2 Crosslinguistic patterns in vowel inventories\ \ 99
\end{styleContentsi}

\begin{styleContentsi}
4.1.3 Segmental inventories and syllable structure complexity\ \ 100
\end{styleContentsi}

\begin{styleContentsi}
4.1.4 The current study and hypotheses\ \ 101
\end{styleContentsi}

\begin{styleContentsi}
4.2 Methodology\ \ 103
\end{styleContentsi}

\begin{styleContentsi}
4.2.1 Patterns considered\ \ 103
\end{styleContentsi}

\begin{styleContentsi}
4.2.2 Coding\ \ 107
\end{styleContentsi}

\begin{styleContentsi}
4.3 Results: Vowel inventories\ \ 110
\end{styleContentsi}

\begin{styleContentsi}
4.3.1 Vowel quality inventory size\ \ 110
\end{styleContentsi}

\begin{styleContentsi}
4.3.2 Contrastive vowel length\ \ 111
\end{styleContentsi}

\begin{styleContentsi}
4.3.3 Other vowel contrasts\ \ 113
\end{styleContentsi}

\begin{styleContentsi}
4.3.4 Diphthongs and vowel sequences\ \ 115
\end{styleContentsi}

\begin{styleContentsi}
4.3.5 Vocalic nucleus inventories and syllable structure complexity\ \ 116
\end{styleContentsi}

\begin{styleContentsi}
4.3.6 Summary of vowel patterns in sample\ \ 117
\end{styleContentsi}

\begin{styleContentsi}
4.4 Results: Consonant inventories\ \ 118
\end{styleContentsi}

\begin{styleContentsi}
4.4.1 Consonant phoneme inventory size\ \ 118
\end{styleContentsi}

\begin{styleContentsi}
4.4.2 Elaborations\ \ 121
\end{styleContentsi}

\begin{styleContentsi}
4.4.3 Phonation features\ \ 125
\end{styleContentsi}

\begin{styleContentsi}
4.4.4 Place features\ \ 127
\end{styleContentsi}

\begin{styleContentsi}
4.4.5 Manner features\ \ 133
\end{styleContentsi}

\begin{styleContentsi}
4.4.6 Summary of consonant patterns in sample\ \ 136
\end{styleContentsi}

\begin{styleContentsi}
4.5 Discussion\ \ 138
\end{styleContentsi}

\begin{styleContentsi}
4.5.1 Segmental inventory patterns and syllable structure complexity\ \ 138
\end{styleContentsi}

\begin{styleContentsi}
4.5.2 Articulations and contrasts characteristic of the Highly Complex category\ \ 142
\end{styleContentsi}

\begin{styleContentsi}
4.5.2.1 Palato-alveolars\ \ 142
\end{styleContentsi}

\begin{styleContentsi}
4.5.2.2 Uvulars\ \ 144
\end{styleContentsi}

\begin{styleContentsi}
4.5.2.3 Ejectives\ \ 146
\end{styleContentsi}

\begin{styleContentsi}
4.5.2.4 Affricates\ \ 148
\end{styleContentsi}

\begin{styleContentsi}
4.5.2.5 More post-velar distinctions and pharyngeals\ \ 150
\end{styleContentsi}

\begin{styleContentsi}
4.5.3 Articulations and consonantal contrasts characteristic of the Simple category\ \ 151
\end{styleContentsi}

\begin{styleContentsi}
4.5.3.1 Prenasalization\ \ 151
\end{styleContentsi}

\begin{styleContentsi}
4.5.3.2 Flaps/taps\ \ 153
\end{styleContentsi}

\begin{styleContentsi}
4.5.4 Segmental patterns, sound change, and syllable structure complexity\ \ 154
\end{styleContentsi}

\begin{styleContentsii}
Chapter 5: Suprasegmental patterns\ \ 157
\end{styleContentsii}

\begin{styleContentsi}
5.1.1 Word stress and tone\ \ 157
\end{styleContentsi}

\begin{styleContentsi}
5.1.2 Suprasegmentals and syllable structure complexity\ \ 159
\end{styleContentsi}

\begin{styleContentsi}
5.1.3 The current study and hypotheses\ \ 162
\end{styleContentsi}

\begin{styleContentsi}
5.2 Methodology\ \ 164
\end{styleContentsi}

\begin{styleContentsi}
5.2.1 Patterns considered\ \ 164
\end{styleContentsi}

\begin{styleContentsi}
5.2.2 Coding\ \ 172
\end{styleContentsi}

\begin{styleContentsi}
5.3 Results: Tone\ \ 173
\end{styleContentsi}

\begin{styleContentsi}
5.4 Results: Stress\ \ 173
\end{styleContentsi}

\begin{styleContentsi}
5.4.1 Presence of word stress and syllable structure complexity\ \ 173
\end{styleContentsi}

\begin{styleContentsi}
5.4.2 Stress assignment\ \ 175
\end{styleContentsi}

\begin{styleContentsi}
5.4.3 Phonetic processes conditioned by word stress\ \ 177
\end{styleContentsi}

\begin{styleContentsi}
5.4.3.1 Unstressed vowel reduction\ \ 178
\end{styleContentsi}

\begin{styleContentsi}
5.4.3.2 Processes affecting consonants in unstressed syllables\ \ 179
\end{styleContentsi}

\begin{styleContentsi}
5.4.3.3 Processes affecting consonants in stressed syllables\ \ 180
\end{styleContentsi}

\begin{styleContentsi}
5.4.3.4 Implicational relationships between phonetic processes conditioned by stress\ \ 181
\end{styleContentsi}

\begin{styleContentsi}
5.4.4 Phonological properties of stressed and unstressed syllables\ \ 182
\end{styleContentsi}

\begin{styleContentsi}
5.4.5 Phonetic correlates of stress\ \ 185
\end{styleContentsi}

\begin{styleContentsi}
5.5 Discussion\ \ 189
\end{styleContentsi}

\begin{styleContentsi}
5.5.1 Suprasegmental patterns and highly complex syllable structure\ \ 189
\end{styleContentsi}

\begin{styleContentsi}
5.5.2 Word stress and the development of syllable structure patterns\ \ 192
\end{styleContentsi}

\begin{styleContentsii}
Chapter 6: Vowel reduction and syllable structure complexity\ \ 195
\end{styleContentsii}

\begin{styleContentsi}
6.1 Introduction and hypothesis\ \ 195
\end{styleContentsi}

\begin{styleContentsi}
6.2 \ Methodology\ \ 197
\end{styleContentsi}

\begin{styleContentsi}
6.2.1 \ Patterns considered\ \ 197
\end{styleContentsi}

\begin{styleContentsi}
6.2.2 Determining what constitutes a process\ \ 200
\end{styleContentsi}

\begin{styleContentsi}
6.2.3 Coding\ \ 202
\end{styleContentsi}

\begin{styleContentsi}
6.3 \ Results\ \ 203
\end{styleContentsi}

\begin{styleContentsi}
6.3.1 Languages with vowel reduction\ \ 203
\end{styleContentsi}

\begin{styleContentsi}
6.3.2 Number of distinct vowel reduction processes present\ \ 204
\end{styleContentsi}

\begin{styleContentsi}
6.3.3 Affected vowels\ \ 205
\end{styleContentsi}

\begin{styleContentsi}
6.3.4 Conditioning environments\ \ 208
\end{styleContentsi}

\begin{styleContentsi}
6.3.5 Outcomes\ \ 210
\end{styleContentsi}

\begin{styleContentsi}
6.3.6 Holistic analysis of vowel reduction processes\ \ 215
\end{styleContentsi}

\begin{styleContentsi}
6.3.7 Summary of vowel reduction patterns\ \ 218
\end{styleContentsi}

\begin{styleContentsi}
6.4 Discussion\ \ 219
\end{styleContentsi}

\begin{styleContentsi}
6.4.1 Vowel reduction patterns and highly complex syllable structure\ \ 219
\end{styleContentsi}

\begin{styleContentsi}
6.4.2 Implications for development of syllable structure complexity\ \ 223
\end{styleContentsi}

\begin{styleContentsii}
Chapter 7: Consonant allophony\ \ 227
\end{styleContentsii}

\begin{styleContentsi}
7.1 Introduction and hypotheses\ \ 227
\end{styleContentsi}

\begin{styleContentsi}
7.2 Methodology\ \ 229
\end{styleContentsi}

\begin{styleContentsi}
7.2.1 Patterns considered\ \ 229
\end{styleContentsi}

\begin{styleContentsi}
7.2.2 Coding\ \ 233
\end{styleContentsi}

\begin{styleContentsi}
7.3 Results\ \ 233
\end{styleContentsi}

\begin{styleContentsi}
7.3.1 Distribution of processes in the language sample\ \ 233
\end{styleContentsi}

\begin{styleContentsi}
7.3.2 Processes resulting in articulations associated with Highly Complex category\ \ 236
\end{styleContentsi}

\begin{styleContentsi}
7.3.3 Other processes resulting in assimilation of consonants to vowels\ \ 238
\end{styleContentsi}

\begin{styleContentsi}
7.3.4 Other processes resulting in fortition\ \ 239
\end{styleContentsi}

\begin{styleContentsi}
7.3.5 Processes resulting in articulations associated with Simple category\ \ 240
\end{styleContentsi}

\begin{styleContentsi}
7.3.6 Other processes resulting in lenition or sonorization\ \ 242
\end{styleContentsi}

\begin{styleContentsi}
7.3.7 Summary of results\ \ 243
\end{styleContentsi}

\begin{styleContentsi}
7.4 Discussion\ \ 247
\end{styleContentsi}

\begin{styleContentsi}
7.4.1 Consonant allophony and the development of syllable structure complexity\ \ 247
\end{styleContentsi}

\begin{styleContentsii}
Chapter 8: Highly complex syllable structure: characteristics, development, and stability\ \ 250
\end{styleContentsii}

\begin{styleContentsi}
8.1 Introduction\ \ 250
\end{styleContentsi}

\begin{styleContentsi}
8.2 Syllable structure complexity and morphology\ \ 250
\end{styleContentsi}

\begin{styleContentsi}
8.3 Highly complex syllable structure as a linguistic type\ \ 252
\end{styleContentsi}

\begin{styleContentsi}
8.4 The development of highly complex syllable structure\ \ 255
\end{styleContentsi}

\begin{styleContentsi}
8.4.1 Directionality of syllable structure change\ \ 256
\end{styleContentsi}

\begin{styleContentsi}
8.4.2 Clues from the crosslinguistic patterns\ \ 259
\end{styleContentsi}

\begin{styleContentsi}
8.4.3 Comparisons of related pairs of languages in sample\ \ 261
\end{styleContentsi}

\begin{styleContentsi}
8.4.3.1 Uto-Aztecan: Ute and Tohono O’odham\ \ 261
\end{styleContentsi}

\begin{styleContentsi}
8.4.3.2 Arawakan: Apurinã and Yine\ \ 264
\end{styleContentsi}

\begin{styleContentsi}
8.4.3.4 Atlantic-Congo: Yoruba and Lunda\ \ 265
\end{styleContentsi}

\begin{styleContentsi}
8.4.3.5 Indo-European: Darai and Albanian\ \ 267
\end{styleContentsi}

\begin{styleContentsi}
8.4.3.6 Austronesian: Maori and Lelepa\ \ 269
\end{styleContentsi}

\begin{styleContentsi}
8.4.3.7 Summary of patterns\ \ 271
\end{styleContentsi}

\begin{styleContentsi}
8.4.4 The case of Lezgian\ \ 272
\end{styleContentsi}

\begin{styleContentsi}
8.4.5 Language contact and syllable structure complexity\ \ 274
\end{styleContentsi}

\begin{styleContentsi}
8.4.6 Development of highly complex syllable structure: conclusions and questions\ \ 278
\end{styleContentsi}

\begin{styleContentsi}
8.5 Highly complex syllable structure: a stable and motivated pattern\ \ 280
\end{styleContentsi}

\begin{styleContentsi}
8.5.1 Synchronic stability of highly complex syllable structure\ \ 280
\end{styleContentsi}

\begin{styleContentsi}
8.5.2 Diachronic stability of syllable complexity\ \ 282
\end{styleContentsi}

\begin{styleContentsi}
8.5.3 Phonetic properties of highly complex syllable patterns and long-term stability\ \ 283
\end{styleContentsi}

\begin{styleContentsi}
8.6 Topics for further research\ \ 284
\end{styleContentsi}

\begin{styleContentsii}
References\ \ 287
\end{styleContentsii}

\clearpage\clearpage\subsection[Abbreviations]{\rmfamily Abbreviations}
\begin{styleBody}
1\ \ \ \ first person
\end{styleBody}

\begin{styleBody}
3\ \ \ \ third person
\end{styleBody}

\begin{styleBody}
\textsc{agt}\ \ \ \ agentive
\end{styleBody}

\begin{styleBody}
\textsc{caus}\ \ \ \ causative
\end{styleBody}

\begin{styleBody}
\textsc{cl}\ \ \ \ noun class
\end{styleBody}

\begin{styleBody}
\textsc{cont.p\ \ \ \ }past continuous
\end{styleBody}

\begin{styleBody}
\textsc{dat}\ \ \ \ dative
\end{styleBody}

\begin{styleBody}
\textsc{dem\ \ \ \ }demonstrative
\end{styleBody}

\begin{styleBody}
\textsc{dub}\ \ \ \ dubitive
\end{styleBody}

\begin{styleBody}
\textsc{f}\ \ \ \ feminine
\end{styleBody}

\begin{styleBody}
\textsc{fut}\ \ \ \ future
\end{styleBody}

\begin{styleBody}
\textsc{incl}\ \ \ \ inclusive
\end{styleBody}

\begin{styleBody}
\textsc{inf}\ \ \ \ infinitive
\end{styleBody}

\begin{styleBody}
\textsc{interrog\ \ }interrogative
\end{styleBody}

\begin{styleBody}
\textsc{iter}\ \ \ \ iterative
\end{styleBody}

\begin{styleBody}
\textsc{m}\ \ \ \ masculine
\end{styleBody}

\begin{styleBody}
\textsc{nmlz} \ \ \ \ nominalizer
\end{styleBody}

\begin{styleBody}
\textsc{nom}\ \ \ \ nominative
\end{styleBody}

\begin{styleBody}
\textsc{pl}\ \ \ \ plural
\end{styleBody}

\begin{styleBody}
\textsc{poss}\ \ \ \ possessive
\end{styleBody}

\begin{styleBody}
\textsc{ps}\ \ \ \ predicate specifier
\end{styleBody}

\begin{styleBody}
\textsc{pst}\ \ \ \ past
\end{styleBody}

\begin{styleBody}
\textsc{realis}\ \ \ \ realis
\end{styleBody}

\begin{styleBody}
\textsc{recp}\ \ \ \ reciprocal
\end{styleBody}

\begin{styleBody}
\textsc{refl}\ \ \ \ reflexive
\end{styleBody}

\begin{styleBody}
\textsc{rel}\ \ \ \ relativizer
\end{styleBody}

\begin{styleBody}
\textsc{sg}\ \ \ \ singular
\end{styleBody}

\begin{styleBody}
\textsc{rns}\ \ \ \ translated from the original by Ricardo Napoleão de Souza
\end{styleBody}

\begin{styleBody}
\textsc{sme}\ \ \ \ translated from the original by Shelece Easterday
\end{styleBody}

\begin{styleBody}
\textsc{tz}\ \ \ \ translated from the original by Tim Zingler
\end{styleBody}

\clearpage\subsection[Acknowledgements]{\rmfamily Acknowledgements}
\begin{styleBody}
\ \ This book is a modified version of my Ph.D. dissertation, which was defended in May 2017 at University of New Mexico. While the book features an updated language sample, the findings and interpretations are largely unchanged from the original study. Neither work would have been possible without the mentorship and support of many people along the way.
\end{styleBody}

\begin{styleBody}
\ \ I am particularly grateful to the members of my dissertation committee. My advisor, Caroline Smith, provided helpful and constructive feedback on all aspects of the dissertation and my progress through the Ph.D. Joan Bybee and Ian Maddieson introduced me to language change and phonological typology, respectively, and I have greatly benefited from the inspiring work and mentorship of both. I thank Ioana Chitoran for many thought-provoking discussions of the findings and the next steps in this line of research. Thanks also to Bill Croft.
\end{styleBody}

\begin{styleBody}
\ \ While writing the original dissertation I was fortunate to have the financial support of the Joseph H. Greenberg Endowed Fellowship (2011-2015) and the Russell J. and Dorothy S. Bilinski Fellowship (2016-2017).
\end{styleBody}

\begin{styleBody}
\ \ Since October 2017, I have been employed as a postdoctoral researcher at Laboratoire Dynamique du Langage (UMR5596, CNRS \& Université Lyon 2) in Lyon, France. While revising this book, I have been funded by LABEX ASLAN (ANR-10-LABX-0081) of the Université de Lyon within the French program Investissements d’Avenir. DDL has been a warm, welcoming, and intellectually exciting environment in which to work, and for this I am grateful to my supervisor François Pellegrino, laboratory director Antoine Guillaume, and Anetta Kopecka, the head of the Description, Typologie, et Terrain (DTT) research axis. Special thanks also go to my fellow ‘postdoc ladies’ — Natalia Chousou-Polydouri, Marine Vuillermet, Laetitia de Almeida, Elisa Demuru, and Natasha Aralova — for their friendship and moral support during this stage of our lives and careers, and to DDL alumna Kasia Janic for the use of her magical apartment.
\end{styleBody}

\begin{styleBody}
\ \ Many specialists kindly answered my inquiries about details of the languages and language families discussed in this book, including Doug Marmion (Wutung), Matthew Carter (Bashkir), Netra P. Paudyal (Darai), Jonathan Bobaljik (Itelmen), Andrey Filchenko (Eastern Khanty), Bonny Sands (Hadza), Kirk Miller (Hadza), Logan Sutton (Kiowa-Tanoan), Rosa Vallejos (Cocama-Cocamilla), Sylvia Tufvesson (Semai), Kasia Wojtylak (Murui Huitoto), and Albert Alvarez Gonzalez (Uto-Aztecan). Tim Zingler, Ricardo Napoleão de Souza, Lexie Adams, and Marc Adams helped me with the translation of reference material from the original German, Portuguese, and Spanish.
\end{styleBody}

\begin{styleBody}
\ \ I am grateful to many dear friends for their support over the years that this work was in progress, including Caron Treloar, Lexie Adams, Erin Debenport, Cora McKenna, Francesca Di Garbo, Giorgio Iemmolo, Logan Sutton, Jason Timm, Ricardo Napoleão de Souza, Fabian Armijo, Nick Mathews, Amanda Delgado Galvan, and Sylvia Tufvesson. 
\end{styleBody}

\begin{styleBody}
\ \ Finally, to my family: thank you for everything.
\end{styleBody}

\clearpage\subsection[Chapter 1: Syllables and syllable structure]{\rmfamily Chapter 1: Syllables and syllable structure}
\begin{styleBody}
\ \ A syllable is typically thought of as a unit which speakers use to organize sequences of sounds in their languages. The division of the speech stream into syllables reflects the higher levels of organization which are used in the cognitive processes by which speech is planned and perceived. Syllables are a common unit of abstract linguistic analysis; however, this unit seems to be more concrete and accessible to speakers than other phonological units such as segments. A speaker’s intuition of what is a pronounceable sequence of sounds is strongly influenced by the syllable patterns of the language they speak. Most languages have relatively simple syllable patterns, in which the alternation between relatively closed (consonantal) and relatively open (vocalic) articulations is fairly regular: syllable patterns such as those in the English words \textit{pillow}, \textit{cactus}, and \textit{tree} are crosslinguistically prevalent. Compare these patterns to the examples below (1.1)-(1.5):
\end{styleBody}

\begin{styleBody}
(1.1) \ \ \textbf{Yakima Sahaptin} (\textit{Sahaptian}; USA)
\end{styleBody}

\begin{styleBody}\itshape
ksksa
\end{styleBody}

\begin{styleBody}
‘elephant ear (mushroom)’
\end{styleBody}

\begin{styleBody}
(Hargus \& Beavert 2006: 29)
\end{styleBody}

\begin{styleBody}
(1.2) \ \ \textbf{Georgian} (\textit{Kartvelian}; Georgia)
\end{styleBody}

\begin{styleBody}\itshape
b[27E?]t[361?]s’$\chi $’ali
\end{styleBody}

\begin{styleBody}
‘claw’
\end{styleBody}

\begin{styleBody}
(Butskhrikidze 2002: 204)
\end{styleBody}

\begin{styleBody}
(1.3) \ \ \textbf{Tashlhiyt }(\textit{Afro-Asiatic}; Morocco)
\end{styleBody}

\begin{styleBody}
\textit{ts[2D0?]k[283?]ftst[2D0?]}
\end{styleBody}

\begin{styleBody}
t{}-s[2D0?]{}-k[283?]f{}-t=st[2D0?]
\end{styleBody}

\begin{styleBody}
‘you dried it (\textsc{f})’
\end{styleBody}

\begin{styleBody}
(Ridouane 2008: 332)
\end{styleBody}

\begin{styleBody}
(1.4) \ \ \textbf{Tehuelche} (\textit{Chonan}; Argentina)
\end{styleBody}

\begin{styleBody}\itshape
kt[361?][283?]a[294?][283?]p[283?]kn
\end{styleBody}

\begin{styleBody}
k{}-t[361?][283?]a[294?][283?]p{}-[283?]{}-kn
\end{styleBody}

\begin{styleBody}
\textsc{refl}{}-wash-\textsc{ps-realis}
\end{styleBody}

\begin{styleBody}
‘it is being washed’
\end{styleBody}

\begin{styleBody}
(Fernández Garay \& Hernández 2006: 13)
\end{styleBody}

\begin{styleBody}
(1.5) \ \ \textbf{Itelmen} (\textit{Chukotko-Kamchatkan}; Russia)
\end{styleBody}

\begin{styleBody}
\textit{k[26C?]txuniŋe[294?]n}
\end{styleBody}

\begin{styleBody}
k[26C?]{}-txuni{}-ŋe[294?]n
\end{styleBody}

\begin{styleBody}
‘very dark’
\end{styleBody}

\begin{styleBody}
(Georg \& Volodin 1999: 55)
\end{styleBody}

\begin{styleBody}
To speakers of most languages, the long strings of consonants in these examples are not pronounceable without a great deal of practice, being so different from the relatively simpler patterns that are crosslinguistically prevalent. Yet such patterns are fluently acquired and maintained by native speakers of these languages, and may even be relatively frequent in those languages.
\end{styleBody}

\begin{styleBody}
\ \ Syllable patterns like those illustrated above are typologically rare, occurring in between 5-10\% of the world’s languages. These languages tend to be found in close geographical proximity to one another, with the Pacific Northwest, the Caucasus region, the Atlas Mountains region, Patagonia, and Northeast Asia being particular ‘hotspots’ for such patterns. The accelerating rates of indigenous language obsolescence in those regions mean that such patterns stand to become even rarer in the coming generations.
\end{styleBody}

\begin{styleBody}
\ \ The patterns exemplified above are also famous in the literature for the problems they present to standard models of syllabic and phonological representation. While much effort is made to attempt to fit these patterns into various theoretical frameworks, far less research explores the motivations behind their historical development and maintenance in languages.
\end{styleBody}

\begin{styleBody}
\ \ This book is a typological study exploring the properties of languages with patterns like those above, which I call \textit{highly complex syllable structure}. The studies herein examine a number of phonetic, phonological, and morphological features of these languages. The aims of this study are to establish whether highly complex syllable structure has other linguistic correlates which may suggest a diachronic path (or paths) by which such patterns are likely to evolve.
\end{styleBody}

\begin{styleBody}
\ \ The book is organized as follows: in the following sections, I discuss findings and accounts for crosslinguistic syllable patterns and their implications for highly complex syllable structure, discuss accounts for syllable complexity more generally, and introduce the research questions examined here. In Chapter 2 I discuss considerations in constructing the language sample and propose a definition for highly complex syllable patterns. In Chapter 3 I conduct analyses of syllable structure patterns in the sample. Analyses of segmental and suprasegmental patterns in the sample are presented in Chapters 4 and 5, respectively. Chapter 6 includes analyses of vowel reduction patterns in the sample. In Chapter 7, I examine specific kinds of consonant allophony in the language sample. In Chapter 8, the results are summarized and their implications for the research questions are examined.
\end{styleBody}

\section[1.1 Background]{\rmfamily 1.1 Background}
\section[1.1.1 The syllable]{\rmfamily 1.1.1 The syllable}
\begin{styleBody}
\ \ The syllable is a natural unit of spoken language by which sounds are organized in speech. The hierarchical organization of speech sounds into syllables is said to be “a fundamental property of phonological structure in human language” (Goldstein et al. 2006: 228), and this unit plays a well-established role in linguistic analysis and description. However, the syllable eludes precise definition: research has not yet established clear and consistent correlates for it at the phonetic, physiological or phonological levels (Bell \& Hooper 1978, Laver 1994, Krakow 1999). Much like consonants and vowels, syllables are characterized by distributional, phonetic, and phonological features, of which no single criterion is sufficient for perfectly describing or predicting the trends observed. To take one example of such a criterion, in a review of research on the physiological organization of the syllable, Krakow (1999: 23-34) states that years of research into this topic have yielded “one disappointment after another” and that from an articulatory point of view, the speech stream “simply cannot be divided into discrete, linearly-ordered units the size of the segment or the syllable.” What empirical research has managed to establish with respect to physiological definitions of the syllable is distinct intra- and inter-articulatory patterns for syllable-initial and syllable-final consonants, at least in careful speech. Patterns in the acoustics, phonology, and perception of syllable constituents play an important role in determining and differentiating syllables, but they do not constitute complete or exceptionless definitions of the syllable, either alone or in combination with one another.
\end{styleBody}

\begin{styleBody}
\ \ Nevertheless, the syllable enjoys a well-established role in phonology, proving to be a highly useful unit in linguistic analysis and description. For many languages, it has been demonstrated that stress placement, tone, reduplication, and other phonological and morphological phenomena operate on the domain of the syllable.\footnote{It should also be noted that phonological syllable structure has been argued to be irrelevant or altogether absent in some languages (e.g., Newman 1947 for Nuxalk; Hyman 2011, 2015 for Gokana; Labrune 2012 for Japanese). In these cases it is argued that phonological phenomena can be satisfactorily described by reference to morae, sequences of segments, and word/phrase junctures.} Similarly, the different boundary edges of a syllable are associated with special coarticulatory properties and may serve as environments for allophonic processes. While native speaker intuitions regarding the precise location of syllable boundaries are not always consistent, there is a wealth of evidence that the unit has psychological reality to speakers: e.g., in the existence of syllabary writing systems, word games and secret languages using syllables as target structures, poetry and lyrical song which exploit syllable counts and syllable constituent patterns in a systematic way, and consistent speaker intuitions regarding the number of syllables in a word (Bell \& Hooper 1978, Blevins 1995, Valle\'{ }e et al. 2009).
\end{styleBody}

\begin{styleBody}
\ \ Additional evidence for the syllable as an organizational unit of language is in the observation that those sequences of sounds analyzed as syllables pattern in remarkably similar ways across languages. In fact, strong crosslinguistic tendencies are observed for practically every dimension along which syllable structure can be analyzed. Some of these patterns will be summarized in the following section.
\end{styleBody}

\section[1.1.2 Crosslinguistic patterns in syllable structure]{\rmfamily 1.1.2 Crosslinguistic patterns in syllable structure}
\begin{styleBody}
\ \ Here I describe some of the crosslinguistic patterns of syllable structure that have been observed in the literature. In the following sections I use the descriptive terms \textit{onset}, \textit{nucleus},\textit{ }and \textit{coda }to refer to constituent parts of the syllable: the nucleus consists of the auditory peak of the syllable, typically a vowel; the onset refers to the consonant or group of consonants preceding the nucleus; and the coda refers to the consonant or group of consonants following the nucleus. It is useful to make these distinctions because these constituents have been shown to behave independently of one another in many respects, both within languages and crosslinguistically. In the following sections, the terms are used in a more or less theoretically neutral sense, and often in reference to phonetic realizations, rather than abstract representations, of the syllable. In theoretical models, the phonological constituency of syllables may be posited to take other forms; some of these issues will be discussed in §1.1.3.
\end{styleBody}

\section[1.1.2.1 CV as a universal syllable type]{\rmfamily\bfseries 1.1.2.1 CV as a universal syllable type}
\begin{styleBody}
\ \ One robust pattern in syllable structure typology is the crosslinguistic ubiquity of syllables of the shape CV: a single consonant followed by a vowel.\footnote{In syllable structure analysis, the notations C and V are used for consonant and vowel segments, respectively.} Though it has been claimed that CV syllables are found in all languages, for a few languages it has been posited that this structure does not occur (cf. Breen \& Pensalfini 1999 for Arrernte, Sommer 1969 for the Oykangand dialect of Kunjen, both Australian languages). Such analyses are typically highly abstract and apply only to ‘underlying’ syllable forms: for both Arrernte and Kunjen it has been shown that CV structures do occur in ‘surface’ phonetic forms (Anderson 2000, Sommer 1969, 1981).
\end{styleBody}

\begin{styleBody}
\ \ Due to its crosslinguistic prevalence, the CV structure has been called the universal syllable type and the least marked of all syllable structures (Zec 2007). CV structures are set apart from other syllable types in numerous aspects of their behavior. If only one syllable type occurs in a language, that type will be of the form CV. Such languages are rare, but attested: they include Hawaiian (Maddieson 2011) and Hua (Blevins 1995). CV structures are acquired even before V structures in babbling stages of vocal development and language acquisition (cf. Levelt et al. 2000 for Dutch). The CV structure overwhelmingly predominates in frequency distributions of syllable types within and across languages. In ULSID, a database containing the syllabified lexicons of 17 genealogically and geographically diverse languages, CV syllables account for roughly 54\% of the 250,000 syllables, despite the languages having a wide range of attested syllable patterns (Vall\'{e}e et al. 2009).
\end{styleBody}

\begin{styleBody}
\ \ Due to the above patterns, CV is often interpreted as a universal primitive element of human language. There are challenges to this view: for example, Bell \& Hooper (1978) argue that the characterization of the CV type as inherently ‘unmarked’ is misleading and simplified, as this assumption can be derived from a collection of generalizations regarding other phonological patterns. They argue that the universal status of CV structures can be interpreted as emerging from a conspiracy of other crosslinguistic patterns which include frequent limitations on vowel hiatus and consonant clusters, tendencies toward obligatory consonant-initial or vowel-final word forms, and the fact that the existence of large consonant strings in any word position in a language implies the existence of simple (single C) structures in those positions. As a result of these interacting patterns, it follows that the canonical syllable patterns of any language will include structures of the form CV.
\end{styleBody}

\begin{styleBody}
\ \ Nevertheless, much of the research on motivations behind crosslinguistic trends in syllable patterns returns to the idea of CV as a universal or otherwise privileged syllable type. Some of these proposals will be discussed in the following sections, as other crosslinguistic patterns relating to syllable structure are discussed.
\end{styleBody}

\section[1.1.2.2 Asymmetries in onset and coda patterns]{\rmfamily\bfseries 1.1.2.2 Asymmetries in onset and coda patterns}
\begin{styleBody}
\ \ Many of the typological patterns involving syllables reveal asymmetries in the structure, distribution, and frequency of onsets versus codas. It follows from the crosslinguistic ubiquity of the CV syllable type that all languages have syllables with onsets. By comparison, languages in which syllable codas do not occur are not uncommon: for example, 12.6\% of the languages whose syllable structures were analyzed in the World Atlas of Language Structures Online (WALS) have canonical CV or (C)V structures only. Thus an implicational relationship holds between codas and onsets: if a language has syllables with codas, then it also has syllables with onsets.
\end{styleBody}

\begin{styleBody}
\ \ While the CV shape dominates in frequency distributions within and across languages, its mirror image, the VC structure, is not nearly so freely distributed. Its crosslinguistic lexical frequency distribution is tiny compared to that of CV: only 2.5\% of the syllables in the ULSID database are of the VC type (Valle\'{ }e et al. 2009). The presence of VC shapes in a language generally implies the presence of V, CV, and CVC structures as well (Blevins 1995). These striking differences in distribution indicate that onsets and codas are not equivalent structures.
\end{styleBody}

\begin{styleBody}
\ \ In many languages with single-member codas, consonants in the coda position are restricted to a smaller set of segments than what can be found in onset position. For example, Cocama-Cocamilla has a consonant phoneme inventory of /p t k t[361?]s t[361?][283?] x m n [27E?] w j/. Any of these consonants may function as a syllable onset, but only alveolar nasal /n/ and glides /w j/ occur in coda position (except for under certain structural and prosodic conditions, Vallejos Yopán 2010: 110). Krakow (1999) reports that some classes of segments, such as oral stops, are crosslinguistically disfavored in syllable-final position. Similarly, Clements observes that when both sonorants and obstruents occur in syllable-final position in a language, the set of permissible obstruents tends to be smaller than the set of permissible sonorants (1990: 301). In a crosslinguistic investigation of syllable frequencies in the lexicons of Hawaiian, Rotokas, Pirahã, Eastern Kadazan, and Shipibo, Maddieson \& Precoda (1992) found that CV sequences are relatively unrestricted in their occurrence. Most onset-nucleus combinations in the study occur at rates approximating the values that would be expected from their component segment frequencies. Meanwhile, nucleus-coda combinations are more restricted in their combinatoriality, owing not only to generally smaller sets of allowable consonants in the coda position, but also to restrictions on sequences of particular segments.
\end{styleBody}

\begin{styleBody}
\ \ Both within and across languages, onsets and codas are most frequently simple, consisting of just one consonant. When languages do have tautosyllabic consonant clusters, they are more likely to occur in the onset position (Blevins 2006). In languages that have tautosyllabic clusters in both onset and coda positions, it is often the case that more elaborate structures are permitted for onsets: these tend to be larger, more frequent, and less restricted in their internal patterns than coda clusters (Greenberg 1965/1978, Blevins 2006). There are of course exceptions to these patterns: Dizin, for instance, has a canonical syllable pattern of (C)V(C)(C)(C) (Beachy 2005). However, as will be shown in §3.3.1, such patterns are crosslinguistically less frequent than their mirror images.
\end{styleBody}

\begin{styleBody}
\ \ Diverse accounts have been put forth in the literature to account for asymmetries in onset and coda patterns. A long line of research starting with Sievers (1881) and Jespersen (1904) has argued that the internal organization of the syllable is governed by the phonological principle of sonority, a scalar perceptual property of speech sounds. A typical sonority scale is given in (1.6) with sonority increasing from left to right:
\end{styleBody}

\begin{styleBody}
(1.6)\ \ stop {\textless} fricative {\textless} nasal {\textless} liquid {\textless} glide {\textless} vowel
\end{styleBody}

\begin{styleBody}
In this view, the sonority contour of typical and preferred syllable types rises steeply at the beginning of the syllable and falls less steeply from the nucleus to the end of the syllable (Zwicky 1972, Hooper 1976, Greenberg 1965/1978, Clements 1990). Thus an ideal syllable would consist of a simple onset consisting of a low-sonority sound such as a stop, a vocalic nucleus, and either a coda of high sonority, such as a nasal or a liquid, or no coda at all.
\end{styleBody}

\begin{styleBody}
\ \ Kawasaki-Fukumori (1992) proposes an acoustic-perceptual motivation for certain crosslinguistic syllable patterns, finding that CV sequences are more spectrally dissimilar from one another, and therefore better contrasted, than VC structures. This suggests that onsets are more likely to be correctly perceived by the listener and maintained in languages. In the speech processing literature, it has been found that onsets are more easily identified by listeners than codas (Content et al. 2001) and that codas affect syllable complexity in such a way as to increase the time required for tautosyllabic onset processing (Segui et al. 1991). 
\end{styleBody}

\begin{styleBody}
\ \ Mechanical and temporal constraints on jaw oscillation have been proposed as physiological motivations for the onset-coda asymmetry and predominance of CV patterns observed. In particular, MacNeilage (1998) proposes that CV patterns derive from the earliest forms of human speech, in which open-close alternations of the mouth, simultaneous with phonation, provided a ‘frame’ for articulatory modulation and the emergence of distinct segmental patterns. From an articulatory point of view, the onset-coda asymmetry may reflect differences in intergestural timing between vowels and consonants in onset versus coda position (Byrd 1996a, Browman and Goldstein 1995, Gick et al. 2006, Marin \& Pouplier 2010). This body of research has established that the gestural coordination between onset and nucleus is synchronous, with the production of the consonant and vowel being nearly simultaneous and representing a stable timing relationship. As compared to the asynchronous and more variable timing relationship between nucleus and coda, the onset-nucleus relationship is more stable in the motor control aspects of its production.
\end{styleBody}

\begin{styleBody}
\ \ Finally, from a diachronic point of view, the relatively restricted status of codas may reflect the effects of reductive sound change: consonants in articulatorily weak word-final and syllable-final positions are particularly vulnerable to assimilation, lenition, and elision processes. Such processes can be observed in synchronic allophony and in patterns of historical sound change (Bybee 2015b).
\end{styleBody}

\section[1.1.2.3 Consonant clusters]{\rmfamily\bfseries 1.1.2.3 Consonant clusters}
\begin{styleBody}
\ \ Crosslinguistic patterns in consonant clusters are not limited to the tendency by which onset clusters tend to be larger and less restricted than coda clusters. It has long been observed that some cluster shapes are crosslinguistically more frequent than others. In fact, the phonological shape of clusters has been used, along with cluster size, as a diagnostic for syllable structure complexity. In the classification used by Maddieson (2013a), an onset cluster in which the second member is a liquid or a glide is considered less complex than one in which the second member is a nasal, fricative, or stop.
\end{styleBody}

\begin{styleBody}
\ \ Studies investigating onset and coda clusters have revealed trends in the voicing, place, manner, and sonority of consonant sequences in tautosyllabic clusters. Greenberg (1965/1978) was one of the first large-scale studies of this kind, investigating both the size and specific phonotactic patterns of onset and coda clusters in 104 languages. This study yielded dozens of implicational generalizations. For instance, the presence of a cluster in a language tends to imply the presence of smaller sequences within it; e.g., in English, the onset /sp[279?]/ as in \textit{spring} implies the onsets /sp/ as in \textit{spy} and /p[279?]/ as in \textit{pry}. Greenberg also derived universals regarding phonetic and phonological properties of consonants in sequence: for example, sonorant+voiced obstruent codas tend to imply the occurrence of sonorant+voiceless obstruent codas. Many crosslinguistic studies in a similar vein have followed from this work. In general, such studies tend to be limited in scope to biconsonantal onset patterns. VanDam (2004) is an exception, in that it explores tendencies in cluster size and composition in word-final codas of all sizes from 18 diverse languages. Some crosslinguistic studies of cluster patterns investigate voicing and manner implications regarding patterns of typologically rare structures, such as tautosyllabic sequences of obstruents (Morelli 1999, 2003, Kreitman 2008). However, studies seeking to account for the crosslinguistically most frequent biconsonantal onset patterns — a stop followed by a liquid or a glide, such as /pl/ or /[261?]w/ — are much more common in the literature (Clements 1990, Berent et al. 2008, Berent et al. 2011, Parker 2012, Vennemann 2012). 
\end{styleBody}

\begin{styleBody}
\ \ Many of the latter studies appeal to the notion of sonority in explaining predominant cluster patterns. In fact, it would seem that a sonority model of syllable structure is more often used to explain cluster patterns than it is to explain the onset-coda asymmetries discussed in the preceding section. In this line of reasoning, cluster patterns in which there is an increasing sonority slope towards the nucleus (e.g., a /kl/ onset) are preferred to sonority plateaus (e.g., a /pk/ onset) or reversals (e.g., a /lb/ onset). Implicational universals using various sonority-based scales are often proposed to describe cluster inventory patterns, particularly the C\textsubscript{2} patterns observed in onsets. For example, Morelli (1999) proposes a universal by which the presence of stop-stop onsets in a language implies the presence of stop-fricative onsets. Lennertz \& Berent (2015) predict that stop-nasal onsets are universally preferred to both stop-stop and stop-fricative onsets. Parker (2012) proposes that the presence of biconsonantal onsets in a language implies the presence of a liquid or glide as C\textsubscript{2}. Vennemann (2012) argues that the diachronic simplification of stop-initial biconsonantal onset inventories can be predicted by a six-point sonority scale, in which onset patterns with C\textsubscript{2} furthest to the right on the scale are lost first (1.7).
\end{styleBody}

\begin{styleBody}
(1.7)\ \ glide {\textless} rhotic {\textless} lateral approximant {\textless} nasal {\textless} fricative {\textless} stop
\end{styleBody}

\begin{styleBody}
\ \ There are exceptions to the above generalizations. In a study of 46 diverse languages, it was found that stop-initial biconsonantal onset inventory patterns diverged from the patterns predicted by the scale in (1.7) roughly one-third of the time (Easterday \& Napoleão de Souza 2015). 
\end{styleBody}

\begin{styleBody}
\ \ While a sonority account does capture strong trends in onset patterns, specifically the crosslinguistic predominance of stop-glide and stop-liquid onsets, accounts of syllable patterns appealing to sonority have been criticized for their circularity. Though sonority has been proposed to be correlated with intensity (Gordon 2002, Parker 2002), degree of constriction (Chin 1996, Cser 2003), and manner of articulation (Parker 2011), it lacks a clear and crosslinguistically consistent phonetic definition.\footnote{In this sense, the notion of sonority is much like that of the syllable.} Instead, the notion of sonority is largely derived from phonotactic patterns, which are then explained in terms of sonority. Some have argued that sonority is in fact an epiphenomenon arising from perceptually motivated constraints, and that the only crosslinguistically consistent sonority contrast is the one between obstruents and sonorants (Jany et al. 2007, Henke et al. 2012). Ohala \& Kawasaki-Fukumori (1997) reject the validity of sonority altogether, arguing that it is both circular and too broadly defined to account for the crosslinguistic rarity of sequences such as /pw/ and /dl/ and crosslinguistic prevalence of sequences such as /sk/. They propose that prevalent onset patterns reflect the high ‘survivability’ of certain sequences, which in turn reflect strong modulations — long trajectories in acoustic space — \ in amplitude, periodicity, spectral shape, and fundamental frequency. In this view, sequences such as /ba/ are more strongly modulated than sequences like 
\end{styleBody}

\begin{styleBody}
/ske/ or /ble/, which in turn are more strongly modulated than /pwe/, /pte/, and so on.
\end{styleBody}

\section[1.1.2.4 Nucleus patterns]{\rmfamily\bfseries 1.1.2.4 Nucleus patterns}
\begin{styleBody}
\ \ Crosslinguistic tendencies have also been observed in the patterns of syllable nuclei, which function as the auditory peaks of syllables. The prototypical syllable nucleus consists of a vowel, and indeed there are many languages which allow only vowels in nucleus position. However, there is a range of crosslinguistic variability in the types of segments observed to occur as syllable nuclei. In some languages, liquids or nasals may function as syllabic; e.g. Slovak \textit{krv} [kr[329?]v] ‘blood’ (Zec 2007: 186), and English \textit{button} [b[28C?][294?]n[329?]]. Such patterns are generally well-accepted in the literature: liquids and nasals are vowel-like in some properties of their acoustic structure, so it is clear how such sounds might function as auditory peaks of syllables. More rarely, obstruents are reported to occur as syllable nuclei: e.g. Puget Salish\textit{ sqw[259?][26C?]ps} [sqw[259?][26C?].ps[329?]] ‘cutthroat trout’ (Hoard 1978: 62), Lendu \textit{zz\`{ }zz\'{ }} [zz\`{ }[329?].zz\'{ }[329?]] ‘drink’ (Demolin 2002: 483), Tashlhiyt \textit{tftktstt }[tf[329?].tk[329?].ts[329?]t[2D0?]] ‘you sprained it (\textsc{f})’ (Ridouane 2008: 332). Such cases are often considered problematic, as they involve sounds which are not vowel-like in their acoustic properties and which may even be voiceless. This view discounts the fact that there are many kinds of obstruents with highly salient auditory properties, such as sibilant fricatives and ejective stops.
\end{styleBody}

\begin{styleBody}
\ \ As is the case with consonant clusters, accounts for crosslinguistic patterns of syllabic consonants often appeal to sonority as an explanatory mechanism, with predominant patterns said to reflect a preference for high-sonority syllable nuclei. Along similar lines of reasoning, nucleus patterns in languages are said to follow a sonority-based implicational hierarchy by which the presence of a given sound as a syllable nucleus in a language implies the presence of all more sonorous types of sounds as syllable nuclei (Blevins 1995, Zec 2007). Thus a language with syllabic nasals is predicted to also have syllabic liquids and vowels. In this model, syllabic obstruents are dispreferred and predicted to be the rarest kind of syllabic consonant.
\end{styleBody}

\begin{styleBody}
\ \ A survey of syllabic consonant patterns in 182 diverse languages suggests that the sonority account for syllable nucleus patterns does not capture some important crosslinguistic trends (Bell 1978a). Of the 85 languages with syllabic consonants, 29 had syllabic liquids, 63 had syllabic nasals, and 34 had syllabic obstruents. The patterns considered in this survey include syllabic consonants arising through synchronic processes of vowel reduction, in addition to invariable syllabic consonant patterns, which are more often used to argue for a sonority basis for syllable nucleus patterns. However, the findings suggest that syllabic obstruents are not exceedingly rare, as often claimed, and may in fact be more common than syllabic liquids. A sonority-based implicational hierarchy does not account for a robust minority of the patterns observed in the study: 10/34 (29\%) of the languages with syllabic obstruents do not have syllabic liquids or nasals.
\end{styleBody}

\begin{styleBody}
\ \ As illustrated by the Lendu and Tashlhiyt examples above, in languages with syllabic obstruents, entire words or phrases without vowels may occur. There are many studies which seek to tackle the problem that such languages pose to models of the syllable (e.g., Bagemihl 1991 for Nuxalk, Coleman 2001 for Tashlhiyt). This is despite the fact that words without vowels are easily pronounceable by fluent speakers and may be relatively frequent in the languages in which they occur: for instance, Ridouane (2008: 328f) reports that in Tashlhiyt, 7.9\% of syntactic words in running text are composed entirely of voiceless obstruents. 
\end{styleBody}

\section[1.1.2.5 Syllable structure and morphology]{\rmfamily\bfseries 1.1.2.5 Syllable structure and morphology}
\begin{styleBody}
\ \ It has long been understood that morphological patterns can play an important role in syllable structure complexity. There are many languages in which the largest tautosyllabic consonant clusters arise through inflection or other morphological processes, for example in the coda /kst-s/ in English \textit{texts}. On the basis of such observations, morphologically complex clusters have often been viewed with suspicion in theoretical treatments of the syllable. Comments casting doubt on their status as valid phonological structures can be found throughout the literature examining syllable patterns from both formal theoretical and descriptive typological perspectives: for example, many crosslinguistic studies of consonant clusters, such as Greenberg (1965/1978) and others mentioned above, explicitly exclude morphologically complex clusters from their analyses. 
\end{styleBody}

\begin{styleBody}
\ \ When morphologically derived syllable structures are explicitly addressed in empirical studies of cluster patterns, it tends to be in order to examine how they differ from unambiguously phonological (morpheme-internal) clusters in aspects of their composition, processing, and acquisition. A recent research program has studied patterns of phonotactic (morpheme-internal) and morphonotactic (morphologically complex) consonant clusters (Dressler \& Dziubalska-Kołaczyk 2006). Several studies in this vein have approached the issue by analyzing properties of cluster inventories, finding that morphologically complex clusters are typically larger and more complex (in terms of sonority or alternative properties such as perceptual distance) than those which occur within morphemes (Dressler \& Dziubalska-Kołaczyk 2006, Orzechowska 2012). 
\end{styleBody}

\begin{styleBody}
\ \ Studies of L1 cluster acquisition have revealed earlier production and lower reduction rates for morphologically complex clusters than for morpheme-internal clusters, suggesting that the extra grammatical-semantic function carried by these structures may work in favor of their stability and maintenance, even if the shapes themselves are ‘dispreferred’ (Kamandulyt\.e 2006, Zydorowicz 2010). Morphologically complex clusters with phonotactically ‘dispreferred’ patterns have in fact been proposed to facilitate parsing in speech perception, since they more reliably signal the morphological compositionality of words and thus feed back into the productivity of those morphemes (Hay \& Baayen 2003, Dressler et al. 2010). 
\end{styleBody}

\section[1.1.3 Theoretical models and crosslinguistic patterns of syllable structure]{\rmfamily 1.1.3 Theoretical models and crosslinguistic patterns of syllable structure}
\begin{styleBody}
\ \ The purpose of models of linguistic structure is to provide a framework and context within which to situate, explain, and make predictions about observed language patterns. As a result, models are often heavily influenced by frequent or well-documented crosslinguistic trends. Theoretical models of the syllable reflect many of the crosslinguistic patterns described above. 
\end{styleBody}

\begin{styleBody}
\ \ Many formalist models of the syllable reflect crosslinguistic trends which privilege CV over other patterns. The model of syllable structure proposed in Government Phonology (Kaye et al. 1990) follows in the tradition of generative syntax, in that every element in phonological structure is governed by some other element in a hierarchical fashion and an element may govern at most two constituents. In this model, the syllable element governs the onset and the rime. The rime branches into a nucleus and an optional simple coda. Depending upon the formulation of the model, the onset may branch into two consonants. A more extreme model following from this tradition, the Strict CV approach, posits only onset and nucleus constituents (Lowenstamm 1996, Scheer 2004). Because of the crosslinguistic tendency towards simple or biconsonantal onsets and simple or absent codas, these approaches are sufficient for describing syllable patterns in many languages. Where patterns do not fit into the proposed frame, empty nuclei are posited in order to preserve the underlying structure. Thus onset clusters are assumed to have intervening empty nuclei between the consonants, and simple codas are assumed to be followed by empty nuclei.
\end{styleBody}

\begin{styleBody}
\ \ Common crosslinguistic cluster patterns such as /s/+stop onsets and stop+/s/ codas have been considered problematic in some frameworks, as they represent sonority plateaus or reversals. In order to deal with such issues, it has been proposed that the /s/ in such patterns is not a part of the core syllable, but functions as an extrasyllabic appendix to it (Vaux \& Wolfe 2009, Duanmu 2011). Appendices and extrasyllabic elements are often posited for peripheral members of clusters which belong to separate morphemes. Interestingly, this approach may result in some of the most frequent clusters in a language (e.g., clusters coming about through inflectional markers) being set apart from morphologically simple ones in their phonological representation.
\end{styleBody}

\begin{styleBody}
\ \ In Optimality Theory, syllable patterns are not governed by a rigid model, but are motivated by universal constraints whose relative importance, or ranking, is determined on a language-specific basis (Prince \& Smolensky 1993). In this framework, surface phonetic forms are those which reflect the best possible output, that is, the fewest violations, with respect to the constraint ranking. Crosslinguistic variation in syllable patterns is explained in terms of different rankings of these violable constraints. Many of the constraints reflect common crosslinguistic patterns, e.g. \textsc{Onset}, in which a violation mark is assigned to a syllable without an onset, and *\textsc{Nucleus}/X, in which a violation mark is assigned to syllable nuclei belonging to some sonority class X (e.g., obstruents; McCarthy 2008).
\end{styleBody}

\begin{styleBody}
\ \ In the Articulatory Phonology framework, researchers have developed a coupled oscillator model of syllable structure which is heavily influenced by findings in the motor control literature (Nam \& Saltzman 2003, Goldstein et al. 2006, Nam et al. 2009). In this model, speech gestures are associated with planning routines, or oscillators, which activate the production of that gesture in speech. These oscillators are coupled to one another in one of two stable modes — in-phase or anti-phase — which determine the relative timing of the production of gestures. Gestures coupled in-phase are initiated synchronously, while gestures coupled anti-phase are initiated sequentially. These coupling phases are proposed to correspond to instrumentally established timing relationships observed in the syllable, in which onset gestures are produced synchronously with those of the vowel but coda gestures are timed sequentially after those of the vowel. This model provides a motor control basis for the privileged status of CV in language acquisition and frequency distributions, as well as the distinct timing patterns associated with onsets, codas, and clusters in each of those positions.
\end{styleBody}

\section[1.2 Highly complex syllable structure: typological outlier, theoretical problem]{1.2 Highly complex syllable structure: typological outlier, theoretical problem}
\begin{styleBody}
\ \ Having discussed some of the predominant crosslinguistic trends in syllable patterns, as well as frequent accounts for them, we return to the patterns presented in at the beginning of this chapter (1.8)-(1.12)
\end{styleBody}

\begin{styleBody}
(1.8) \ \ \textbf{Yakima Sahaptin} (\textit{Sahaptian}; USA)
\end{styleBody}

\begin{styleBody}\itshape
ksksa
\end{styleBody}

\begin{styleBody}
‘elephant ear (mushroom)’
\end{styleBody}

\begin{styleBody}
(Hargus \& Beavert 2006: 29)
\end{styleBody}

\begin{styleBody}
(1.9) \ \ \textbf{Georgian} (\textit{Kartvelian}; Georgia)
\end{styleBody}

\begin{styleBody}\itshape
b[27E?]t[361?]s’$\chi $’ali
\end{styleBody}

\begin{styleBody}
‘claw’
\end{styleBody}

\begin{styleBody}
(Butskhrikidze 2002: 204)
\end{styleBody}

\begin{styleBody}
(1.10) \ \ \textbf{Tashlhiyt} (\textit{Afro-Asiatic}; Morocco)
\end{styleBody}

\begin{styleBody}
\textit{ts[2D0?]k[283?]ftst[2D0?]}
\end{styleBody}

\begin{styleBody}
t{}-s[2D0?]{}-k[283?]f{}-t=st[2D0?]
\end{styleBody}

\begin{styleBody}
‘you dried it (\textsc{f})’
\end{styleBody}

\begin{styleBody}
(Ridouane 2008: 332)
\end{styleBody}

\begin{styleBody}
(1.11) \ \ \textbf{Tehuelche} (\textit{Chonan}; Argentina)
\end{styleBody}

\begin{styleBody}\itshape
kt[361?][283?]a[294?][283?]p[283?]kn
\end{styleBody}

\begin{styleBody}
k{}-t[361?][283?]a[294?][283?]p{}-[283?]{}-kn
\end{styleBody}

\begin{styleBody}
\textsc{refl}{}-wash-\textsc{ps-realis}
\end{styleBody}

\begin{styleBody}
‘it is being washed’
\end{styleBody}

\begin{styleBody}
(Fernández Garay \& Hernández 2006: 13)
\end{styleBody}

\begin{styleBody}
(1.12) \ \ \textbf{Itelmen} (\textit{Chukotko-Kamchatkan}; Russia)
\end{styleBody}

\begin{styleBody}
\textit{k[26C?]txuniŋe[294?]n}
\end{styleBody}

\begin{styleBody}
k[26C?]{}-txuni{}-ŋe[294?]n
\end{styleBody}

\begin{styleBody}
‘very dark’
\end{styleBody}

\begin{styleBody}
(Georg \& Volodin 1999: 55)
\end{styleBody}

\begin{styleBody}
In the context of the issues previously discussed, highly complex syllable patterns may be considered problematic in all respects.
\end{styleBody}

\begin{styleBody}
\ \ The syllable patterns in (1.8)-(1.12) are, first of all, extremely large in comparison to the universally privileged CV shape. This fact has been pointed to explicitly in the literature as a reason to consider such patterns invalid: Kaye et al. (1990: 195), in a discussion of syllable patterns with four-consonant codas in Nez Perce, write that “[t]he sheer length of such sequences makes one doubtful of their status as syllable constituents of one and the same syllable.” The example in (1.11) is chosen to illustrate that codas may be much longer than onsets in Tehuelche, which goes against predominant crosslinguistic trends. Further, the word-initial patterns in (1.8) and (1.12) consist entirely of obstruents, which should be strongly dispreferred according to both sonority models (e.g., Clements 1990) and acoustic-perceptual models (Ohala \& Kawasaki-Fukumori 1997) of syllable structure. The word without vowels in (1.10) is typologically rare and implies syllabic obstruents, which are crosslinguistically ‘dispreferred.’ The patterns in (1.10)-(1.12) are further regarded as dubious because their clusters are morphologically complex and therefore perhaps not ‘valid’ phonological structures. All of the patterns above, besides being typologically rare, are theoretically marginalized in that they represent the opposite of the predominant crosslinguistic patterns which models of the syllable seek to capture and describe.
\end{styleBody}

\begin{styleBody}
\ \ When highly complex syllable patterns are explicitly treated in the literature, it tends to be with the purpose of making their patterns fit into prevailing theoretical models. An example of this is Bagemihl’s (1991) analysis of Nuxalk syllable structure. On the basis of reduplication data, Bagemihl analyzes the language as having “relatively ordinary” CRVVC syllable structure,\footnote{Here R stands for ‘resonant,’ corresponding to a sonorant consonant.} in which vowels, liquids, and nasals may function as V nuclei. Segments that do not fit into that syllable frame remain phonologically unsyllabified. Thus a word without sonorants — like \textit{[26C?]$\chi [2B7?]$t[26C?]cx[2B7?]} ‘you spat on me’ — while being fully and fluently pronounceable by speakers, is analyzed as entirely unsyllabified at the phonological level. Similarly, a strict CV approach has been used to account for ‘ghost vowels’ — vowels which alternate with zero — in Mohawk and Polish, both of which have highly complex syllable patterns (Rowicka 1999). However, this has the effect of positing long sequences of simple onsets followed by empty nuclei for the large consonant clusters which occur in those languages, as in Mohawk \textit{khninus} ‘I buy’ or Polish \textit{źdźbło }/[291?]d[361?][291?]bwo/ ‘blade of grass’. These novel phonological analyses are based upon careful consideration of both language-specific patterns and theoretical implications. However, such treatments of highly complex syllable structure have the effect of theoretically ‘normalizing’ these rare syllable patterns: not by taking them at face value as corresponding to possible cognitive representations of language, but by arguing away their unusual properties until they more closely resemble familiar patterns.
\end{styleBody}

\begin{styleBody}
\ \ More problematic are approaches which treat highly complex syllable structure as anomalous or exotic. Such attitudes, as reflected by assumptions about what constitutes possible syllable length and constituency (cf. the quote by Kaye and colleagues above), make it all too easy for researchers to dismiss such patterns as improbable or regard them as statistical aberrations from an established norm. This seems to be more often the case when highly complex syllable patterns occur in underdescribed non-Eurasian languages. It sets a worrisome precedent when the patterns of minority, indigenous, and endangered languages are dismissed in this way. This reinforces a European bias and serves to further marginalize and exoticize languages which are already historically underrepresented in our discipline.
\end{styleBody}

\begin{styleBody}
\ \ Related to this point is the fact that much of the research in linguistics, including syllable structure typology, is influenced by an overrepresentation of data from European languages. A survey of crosslinguistic studies of consonant cluster patterns, for example, revealed an Indo-European bias which ranged from 34\% (Morelli 1999) to 79\% (Vennemann 2012) of the languages in those samples (Easterday \& Napoleão de Souza 2015). In an investigation of the conformity of plosive-initial biconsonantal onset inventories to the predictions of a sonority-based implicational hierarchy in 46 diverse languages, only five of which were Indo-European, it was found that nearly one-third of the languages had patterns diverging from these predictions (ibid.). None of the diverging patterns were found in Indo-European languages, and nearly all were from regions or families which tend to be underrepresented in linguistic research. This suggests that some of the reported norms of syllable structure typology may be heavily biased towards what has been observed in Indo-European and other well-represented families.
\end{styleBody}

\begin{styleBody}
\ \ Other issues which often go unexplored in accounts for crosslinguistic patterns of syllable structure are the influence of processes of language change and the relationship between syllable patterns and other elements of the phonology and the grammar. These issues are of special importance for typologically rare patterns, such as highly complex syllable structure, as they provide a natural explanation for the emergence and maintenance of these purportedly dispreferred patterns. In the following section I briefly discuss some lines of research which situate the issue of syllable structure complexity within holistic typologies of language by relating it to other phonological and grammatical properties.
\end{styleBody}

\section[1.3 Syllable structure complexity: accounts and correlations]{\rmfamily 1.3 Syllable structure complexity: accounts and correlations}
\section[1.3.1 Speech rhythm typologies]{\rmfamily 1.3.1 Speech rhythm typologies}
\begin{styleBody}
\ \ A long line of research in linguistics has sought to characterize and measure rhythmic properties of language which are perceptually and psychologically salient to speakers and play an important role in language acquisition (Cutler \& Mehler 1993). The typology proposed by Pike (1945) distinguished two speech rhythm types: stress-timed languages and syllable-timed languages, with English being a prototypical example of the former and Spanish being a prototypical example of the latter. This typology was later expanded to include a third category of mora timing, for which Japanese is a prototypical example. In its initial formulation, it was postulated that the rhythmic properties of these language types reflect equal timing intervals between those units: between stresses for stress-timed languages, syllables for syllable-timed languages, and morae for mora-timed languages. This ‘isochrony hypothesis’ was eventually instrumentally disconfirmed (Roach 1982). Speech rhythm typologies subsequently shifted their \ focus to phonological holism, relating rhythm types to a confluence of factors involving syllable structure, vowel reduction, vowel length contrasts, and properties of stress placement (Roach 1982, Dauer 1983). In this typology, simple syllable structure is proposed to occur with syllable timing, and complex syllable structure with stress timing. Reduction of vowels in unstressed syllables and variation in lexical stress patterns are additionally proposed to occur with complex syllable structure in stress-timed languages (Auer 1993). The proposed co-occurrences are not meant to be categorical, and as will be discussed in Chapter 5, may reflect the patterns of European languages specifically (Schiering 2007).
\end{styleBody}

\begin{styleBody}
\ \ Proposed measurements of the acoustic properties of speech rhythm have been suggested to relate directly to syllable structure. Metrics developed by Ramus et al. (1999) correspond to the proportion of vocalic intervals and standard deviation of consonantal intervals in speech. In languages with high syllable complexity, a greater standard deviation of consonant intervals and a lower proportion of vocalic intervals is expected, corresponding to both the greater variation in syllable types and the higher probability of consonant sequences in running speech in such languages. When languages are plotted according to these metrics, they fall into groups which largely correspond to traditional rhythm categories of stress timing and syllable timing (but see Wiget et al. 2010 for criticisms of this approach). When these metrics were calculated in a crosslinguistically diverse sample of languages representing various degrees of syllable structure complexity and other phonological properties, it was found that syllable structure complexity is indeed significantly correlated with the expected indices (\textit{p} {\textless} .005), lending empirical validation to the suggested relationship (Easterday et al. 2011). However, the direction of causality behind the relationship is unclear from these findings: while syllable structure contributes heavily to the acoustic-perceptual properties of speech rhythm, it is not clear whether syllable structure necessarily causes or constitutes stress timing. It may instead be that syllable structure is affected by and comes about through the other prosodic and phonological features associated with stress timing, such as vowel reduction.
\end{styleBody}

\section[1.3.2 Other holistic typologies]{\rmfamily 1.3.2 Other holistic typologies}
\begin{styleBody}
\ \ Some holistic typologies which consider syllable complexity attempt to relate the phonology, morphology, syntax, and discourse properties of language to one another. An example of one such ambitious typology is that proposed in various forms by Vladimir Skalička from 1958 to 1979 (Plank 1998). Skalička (1979) proposed five ideal types which languages are supposed to approximate, if not attain: polysynthesis (an idiosyncratic use of the term that does not correspond to modern usage), agglutination, flection, introflection, and isolation. The many phonological and grammatical properties proposed to co-occur in each of these types were meant to be mutually supportive. In two of the types — agglutination and introflection — complex consonant clusters are said to co-occur with rich consonant systems and a high amount of verbal inflection. Other properties of these very specifically-defined classes include a prevalence of vowel harmony and looser fusion between gramemes and the stem in the agglutination type, and root-internal marking in the introflection type. Like many proposed holistic typologies, Skalička’s is largely impressionistic and not based in extensive empirical evidence.
\end{styleBody}

\begin{styleBody}
\ \ A series of empirical studies by Gertraud Fenk-Oczlon and August Fenk have sought to establish correlations between certain grammatical and discourse properties of language and syllable structure specifically. Fenk \& Fenk-Oczlon (1993) tested Menzerath’s Law (paraphrased as “the bigger the whole, the smaller the parts”) and found a significant negative linear correlation between the number of syllables per word and the number of phonemes per syllable, a measure roughly analogous to syllable complexity. Working from the observation that words have more syllables in agglutinating languages, Fenk-Oczlon \& Fenk (2005) established a correspondence between complex syllable structure and a tendency towards prepositions and a low number of grammatical cases on the one hand and simple syllable structure and a tendency to postpositions and a high number of cases on the other. Finally, Fenk-Oczlon \& Fenk (2008) found that high phonological complexity (determined by the number of distinct monosyllables in a language) was correlated with low morphological complexity and high semantic complexity (i.e., high degrees of homonymy and polysemy), as well as rigid word order and idiomatic speech. They explain these results in terms of complexity trade-offs which balance the different sub-systems of language.
\end{styleBody}

\begin{styleBody}
\ \ The results of Shosted (2006) conflict with those of Fenk-Oczlon \& Fenk. This empirical study attempts to test the negative correlation hypothesis, which holds that if one component of language is simplified, then another must be elaborated. Specifically, Shosted considers correlations between syllable structure and inflectional synthesis of the verb in a diversified sample of 32 languages. He finds a slightly positive but statistically insignificant correlation between complexity in the two measures. Shosted’s measure of phonological complexity is not based on measurements of maximal syllable complexity, but instead on the potential number of distinct syllables allowed in each language, a figure which is calculated from the number of phonemic contrasts, canonical syllable patterns, and various phonotactic constraints reported for each language.
\end{styleBody}

\section[1.3.3 Consonantal and vocalic languages]{\rmfamily 1.3.3 Consonantal and vocalic languages}
\begin{styleBody}
\ \ In phonological descriptions and general typological studies, the terms \textit{consonantal }and \textit{vocalic }are sometimes used to describe the holistic phonological character of languages (1.13)-(1.18).
\end{styleBody}

\begin{styleBody}
(1.13) \ \ “In this group, we find on the one hand highly consonantal languages like Kabardian and other Northwest Caucasian languages […], and on the other hand vocalic languages with long morphemes, for example Indonesian and related languages […]”\footnote{Translation TZ.}
\end{styleBody}

\begin{styleBody}
(Skalička 1979: 309)
\end{styleBody}

\begin{styleBody}
(1.14) \ \ “Syntagmatically, all (indigenous) Caucasian idioms can be called ‘consonant-type languages,’ with more consonants in a speech sequence than vowels […] The same term (‘consonantal languages’) can be applied to them paradigmatically as well, all Caucasian languages being notorious for the richness of their consonantal inventories, versus restricted or very restricted vowel systems.” 
\end{styleBody}

\begin{styleBody}
(Chirikba 2008: 43)
\end{styleBody}

\begin{styleBody}
(1.15)\ \ “[Polish] can be described as a ‘consonantal’ language, in two respects: (a) it has a rich system of consonant phonemes […] and (b) it allows heavy consonant clusters …” 
\end{styleBody}

\begin{styleBody}
(Jassem 2003: 103)
\end{styleBody}

\begin{styleBody}
(1.16) \ \ “Slovak is a more consonantal language than German (27 vs. 21) …”
\end{styleBody}

\begin{styleBody}
(Dressler et al. 2015: 56)
\end{styleBody}

\begin{styleBody}
(1.17)\ \ “Since Italian is clearly a less consonantal language than English …” 
\end{styleBody}

\begin{styleBody}
(Dressler \& Dziubalska-Kołaczyk 2006: 263)
\end{styleBody}

\begin{styleTableStyleii}
(1.18)\ \ “Tashlhiyt can be described as a ‘consonantal language.’ […] What makes Tashlhiyt a ‘consonantal language’ \textit{par excellence} is the existence of words composed of consonants only …” 
\end{styleTableStyleii}

\begin{styleTableStyleii}
(Ridouane 2014: 216)
\end{styleTableStyleii}

\begin{styleBody}
\ \ The use of these terms is especially prevalent in Slavic and Caucasian linguistics. In some of those contexts, the terms may refer directly to a holistic phonological typology of Slavic languages developed by Isačenko (1939/1940). In that work, \textit{consonantal} languages are defined as having complex syllable structure, a higher proportion of consonants in the phoneme inventory, the presence of certain consonant contrasts such as secondary palatalization, and fixed or lexically-determined stress. By comparison, \textit{vocalic} languages have simpler syllable structure, lower proportions of consonants in the phoneme inventory, fewer consonant place contrasts, and pitch accent or ‘musical intonation.’ Several of the descriptions above also make reference to the overall size of the consonant phoneme inventory and sequences of consonants in word patterns or the speech stream. The relationship between syllable structure complexity and consonant phoneme inventory size suggested above is an empirically established one: as will be discussed further in Chapter 4, Maddieson (2013a) found a weak but highly significant positive relationship between these features in a set of 484 languages. These findings suggest that the use of the terms consonantal and vocalic is at least to some extent grounded in observable crosslinguistic patterns.
\end{styleBody}

\begin{styleBody}
\ \ Impressionistic descriptions of the phonetic characteristics of languages with highly complex syllable structure are evocative of descriptions of consonantal languages. I present some of these below (1.19)-(1.22).
\end{styleBody}

\begin{styleBody}
(1.19)\ \ \textbf{Kabardian} (\textit{Abkhaz-Adyge}; Russia, Turkey)
\end{styleBody}

\begin{styleTableStyleii}
“On the whole, the vowels have comparatively little prominence, in comparison with the consonants.”
\end{styleTableStyleii}

\begin{styleTableStyleii}
(Kuipers 1960: 24)
\end{styleTableStyleii}

\begin{styleTableStyleii}
“[T]he typical Kabardian pronunciation is imitated most easily if one pronounces the word without vowels other than \textit{a} and with a stress immediately after the initial consonant: the result will show the predominance of consonants over vowels that is typical of Kabardian speech, and the syllabic peaks will be determined automatically by the stress and by the sonority of the sounds in the sequence.” 
\end{styleTableStyleii}

\begin{styleTableStyleii}
(Kuipers 1960: 43)
\end{styleTableStyleii}

\begin{styleBody}
(1.20)\ \ \textbf{Camsá} (isolate; Colombia)
\end{styleBody}

\begin{styleBody}
“Words are pronounced rapidly with vowels practically eliminated word medially. A degree of emphasis is placed on the vowel of the first syllable with the following syllables squeezed together before the stressed syllable.” 
\end{styleBody}

\begin{styleBody}
(Howard 1967: 86-7)
\end{styleBody}

\begin{styleBody}
(1.21)\ \ \textbf{Thompson} (\textit{Salishan}; Canada)
\end{styleBody}

\begin{styleTableStyleii}
“Basic vowel adjustments reflect the general tendency of the language to drop vowels from unstressed syllables wherever possible and to convert to /[259?]/ those vowels that are not dropped. In rapid speech, this tendency is nearly fully realized, so that few tense vowels are heard outside of stressed syllables.”
\end{styleTableStyleii}

\begin{styleBody}
(Thompson \& Thompson 1992: 31)
\end{styleBody}

\begin{styleBody}
(1.22)\ \ \textbf{Itelmen} (\textit{Chukotko-Kamchatkan}; Russia)
\end{styleBody}

\begin{styleBody}
“I suppose it is little exaggeration to say that in the [Itelmen] language there are no vowels, or, perhaps, their vowels are so obscure that it is hardly possible to translate them to European [equivalents].”\footnote{Translation SME.}
\end{styleBody}

\begin{styleBody}
(Volodin 1976: 40-1; quoting V. N. Tyushov)
\end{styleBody}

\begin{styleBody}
\ \ These vivid descriptions of fluent speech in languages with highly complex syllable structure are surely influenced by the stark differences between these phonetic patterns and those of the languages spoken natively by the researchers. However, taken along with observations regarding consonantal languages, as well as findings in the speech rhythm and holistic typology literature, they also suggest a path forward for investigating highly complex syllable structure as a coherent linguistic type characterized by an array of phonetic and phonological features.
\end{styleBody}

\section[1.4 The current study]{\rmfamily 1.4 The current study}
\begin{styleBody}
\ \ The current study is a crosslinguistic investigation of highly complex syllable patterns, their properties, their associations with other linguistic features, and their emergence over time. The two aims of the study are (i) to establish whether languages with highly complex syllable structure constitute a linguistic type, in the sense denoted by the holistic typologies described above, and (ii) to identify possible diachronic paths and natural mechanisms by which these patterns come about in the history of a language. A secondary goal is to ‘de-exoticize’ these rare syllable patterns by considering them at face value as natural language structures rather than as typological and theoretical outliers.
\end{styleBody}

\section[1.4.1 Research questions]{\rmfamily 1.4.1 Research questions}
\begin{styleBody}
\ \ The two broad research questions follow directly from the aims of the study listed above. The first is given in (1.23).
\end{styleBody}

\begin{styleBody}
(1.23) \ \ \textit{Do languages with highly complex syllable structure share other phonetic and phonological characteristics such that this group can be classified as a linguistic type?}
\end{styleBody}

\begin{styleBody}
\ \ This research focus seeks to establish whether highly complex syllable structure is a linguistic type characterized by a convergence of associated phonetic and phonological properties. The properties to be considered follow in part from the findings and proposals in the holistic typologies described above. These include properties of syllable structure, phoneme inventories, suprasegmental patterns, and processes of vowel reduction and consonant allophony (see the following section for a detailed list of considerations). The specific hypotheses regarding the associations between syllable complexity and these properties will be presented with each analysis in upcoming chapters.
\end{styleBody}

\begin{styleBody}
\ \ While the term ‘linguistic type’ is used in the formulation of (1.23), this is not meant in the sense that I expect the results of the analyses to set these languages apart from others in a strict categorical way. As with the holistic language typologies discussed above, it is more likely that phonetic and phonological properties will show a \textit{tendency} to cluster together. If such expectations are borne out in the analyses, they may aid in addressing the second research question:
\end{styleBody}

\begin{styleBody}
(1.24)\textbf{ \ \ }\textit{How does highly complex syllable structure develop over time?}
\end{styleBody}

\begin{styleBody}
\ \ As will become apparent in the following chapters, capturing the development of highly complex syllable structure in real time is not a straightforward endeavor: syllable patterns seem to be remarkably stable and persistent over time and within language families (Napoleão de Souza 2017). Where synchronic and historical accounts based on direct evidence are available, these are useful in approaching the research question in (1.24). Additionally, methods of diachronic typology can be used. This will be discussed further below.
\end{styleBody}

\section[1.4.2 Proposed analyses and framework]{\rmfamily 1.4.2 Proposed analyses and framework}
\begin{styleBody}
\ \ The research questions outlined above are investigated in a sample of 100 languages representing four different categories of syllable complexity and selected to maximize genealogical and geographic diversity. The size and construction of the sample is designed to allow for a maximally systematic investigation of both of the research questions (see Chapter 2 for further detail). For practical reasons, the scope of the book is largely limited to the analysis of phonological and phonetic properties, but in a few cases morphological factors are additionally considered. The analyses are grouped into five coherent studies, each corresponding to a chapter. These are listed below.
\end{styleBody}

\begin{styleBody}
(1.25)\ \ \textbf{Phonological and phonetic properties considered}
\end{styleBody}

\begin{styleBody}\bfseries
Syllable patterns (Chapter 3)
\end{styleBody}

\begin{styleBody}\itshape
Size, location, phonological shape, and morphological complexity of maximal clusters
\end{styleBody}

\begin{styleBody}\itshape
Nucleus patterns, including syllabic consonants
\end{styleBody}

\begin{styleBody}\itshape
Morphological patterns of syllabic consonants
\end{styleBody}

\begin{styleBody}\itshape
Relative prominence of highly complex syllable patterns within languages
\end{styleBody}

\begin{styleBody}\itshape
Phonetic properties of large clusters
\end{styleBody}

\begin{styleBody}\bfseries
Segmental inventories (Chapter 4)
\end{styleBody}

\begin{styleBody}\itshape
Consonant phoneme inventory size
\end{styleBody}

\begin{styleBody}\itshape
Consonant articulations present
\end{styleBody}

\begin{styleBody}\itshape
Vocalic nucleus inventory size
\end{styleBody}

\begin{styleBody}\itshape
Vocalic contrasts present
\end{styleBody}

\begin{styleBody}\bfseries
Suprasegmental properties (Chapter 5)
\end{styleBody}

\begin{styleBody}\itshape
Presence of tone and word stress
\end{styleBody}

\begin{styleBody}\itshape
Predictability of word stress placement
\end{styleBody}

\begin{styleBody}\itshape
Phonological asymmetries between stressed and unstressed syllables
\end{styleBody}

\begin{styleBody}\itshape
Phonetic processes conditioned by stress
\end{styleBody}

\begin{styleBody}\itshape
Phonetic correlates of stress
\end{styleBody}

\begin{styleBody}\bfseries
Vowel reduction (Chapter 6)
\end{styleBody}

\begin{styleBody}\itshape
Presence and prevalence of vowel reduction
\end{styleBody}

\begin{styleBody}\itshape
Affected vowels
\end{styleBody}

\begin{styleBody}\itshape
Conditioning environments
\end{styleBody}

\begin{styleBody}\itshape
Outcomes of vowel reduction and effects on syllable patterns
\end{styleBody}

\begin{styleBody}\bfseries
Consonant allophony (Chapter 7)
\end{styleBody}

\begin{styleBody}\itshape
Presence of specific types of assimilation, lenition, and fortition
\end{styleBody}

\begin{styleBody}\itshape
Conditioning environments
\end{styleBody}

\begin{styleBody}
\ \ The results of these analyses will be used to directly address the research question regarding the establishment of languages with highly complex syllable structure as a linguistic type. While one goal is to quantify associations between syllable structure complexity and specific linguistic features, qualitative patterns in the data will also be considered in this endeavor.
\end{styleBody}

\begin{styleBody}
\ \ Additionally, the results will be used to inform diachronic paths by which highly complex syllable patterns develop, addressing the second research question. Specifically, the methods of diachronic typology — the use of “synchronic variation to dynamicize a typology” (Croft 2003: 272) — will be used. In this method, diachronic processes and paths are inferred, with careful consideration of attested processes and known directionality of language change, from synchronic patterns. This method is especially valuable in the current study, as many of the languages with highly complex syllable structure have little historical documentation. Strong tendencies in the phonetic and phonological properties of languages with highly complex syllable patterns may point to processes of language change which tend to precede, accompany, or follow the development of these structures, hinting at steps in the historical evolution of this linguistic type.
\end{styleBody}

\begin{styleBody}
\ \ Like most typological studies, the analyses in this book rely on written reference materials and are therefore based on standard features of structural linguistic analysis, such as phoneme inventories and phonological processes. However, the interpretations of patterns are informed by a theoretical framework which views the patterns of organization within language as dynamic, interactive, and emergent from usage (Beckner et al. 2009, Bybee 2001, 2010).
\end{styleBody}

\begin{styleBody}
\ \ While I do not have a finely articulated hypothesis regarding the diachronic development of highly complex syllable structure, I enter into these studies with a few ideas and assumptions regarding this issue. Following findings in the speech rhythm literature, I expect that vowel reduction, especially processes resulting in vowel deletion or the development of syllabic consonants, will be highly relevant in the development of these patterns. Since vowel reduction is often associated with unstressed syllables, it is also expected that stress will play an important role. These phenomena may be accompanied by particular processes of consonant allophony, such as palatalization, which over time have the effect of increasing consonant phoneme inventory sizes. Finally, an important aspect of syllable structure development that can be only briefly considered here is the role of morphology. Based upon observations of morphologically complex clusters in languages with highly complex syllable structure, as well as associations posited between syllable complexity and morphological patterns in the literature, I expect that the development of these syllable patterns are often facilitated by a high degree of inflectional or derivational morphology in a language. In a speculative scenario, it is easy to imagine highly complex syllable patterns developing in a highly inflectional affixing language in which stress falls on the root or stem and eventually has segmental effects which include the reduction and eventual deletion of unstressed vowels, resulting in long heteromorphemic consonant sequences at word edges. The plausibility of the various aspects of such a scenario will be explored in upcoming chapters.
\end{styleBody}

\clearpage\subsection[Chapter 2: Language sample]{\rmfamily Chapter 2: Language sample}
\begin{styleBody}
\ \ This chapter describes the language sample used in the study. In §2.1 I discuss general issues of language sampling and specific considerations for sampling in the current study. In §2.2 I examine a previous typology of syllable structure complexity and propose a definition for a category of Highly Complex syllable structure. In §2.3 I discuss the procedure underlying the construction of the language sample. In §2.4 I present the language sample and describe its areal, genealogical, and sociolinguistic features. In §2.5 I briefly discuss the general method of data collection.
\end{styleBody}

\section[2.1 Language sampling]{\rmfamily 2.1 Language sampling}
\begin{styleBody}
\ \ Crosslinguistic comparison is “the fundamental characteristic of typology” (Croft 2003: 6). In order to make general statements about some linguistic property such as syllable complexity, it is necessary to examine the properties of and variation within that feature in a wide variety of languages. Today, linguists have access to a greater array of grammatical descriptions, corpora, and audiovisual materials than ever before. However, for many languages, reference materials are either not available or not descriptive enough for inclusion in most typological studies. Therefore researchers must rely on samples much smaller than the set of the 7,097 languages known to be living today (Simons \& Fennig 2018). Because typology is a data-driven science, the issue of sampling - that is, determining which languages will serve as data sources for addressing the question(s) at hand - is critical in any study. The relative merits of different sampling techniques and methods of controlling for various types of bias have often been the subject of debate in the field (see Bakker 2011 for an overview of the relevant literature). 
\end{styleBody}

\begin{styleBody}
\ \ Before introducing the sample used in the current study, I discuss some of the known sources of bias in typological work. I also discuss the potential effect of these factors on investigating issues of syllable structure or phonology more generally.
\end{styleBody}

\section[2.1.1 Common sources of bias in language sampling]{\rmfamily 2.1.1 Common sources of bias in language sampling}
\begin{styleBody}
\ \ The three most commonly discussed sources of bias in language sampling are \textit{genealogical}, \textit{areal}, and \textit{bibliographic} bias (Bakker 2011). 
\end{styleBody}

\begin{styleBody}
\ \ A typological study may suffer from \textit{genealogical bias} if it includes data from related languages. This presents a potential confound in the interpretation of results, because similar patterns in related languages may not be independent of one another, but instead inherited from a common ancestor. Of all the sources of bias in language sampling, genealogical bias is perhaps the most discussed, and the one most explicitly controlled for. Strategies for minimizing this kind of bias include systematic stratification of the language sample itself at a particular time depth or level of genealogical classification (Bell 1978b, Maddieson 1984, Dryer 1989), or postponement of sampling to post-hoc analysis, when the independence of the feature(s) under study can be determined at each taxonomic level within a language family (Bickel 2008). In practice, though, most typological studies are based on small convenience samples which are heavily skewed towards large, well-known, and well-described language families. Language isolates, which account for up to one-third of known language families (Campbell 2016), and smaller, lesser-known language families often go altogether unrepresented.
\end{styleBody}

\begin{styleBody}
\ \ Another bias which is often considered in constructing language samples is \textit{areal bias}, in which languages spoken in the same geographical and/or cultural area may have influenced one another through prolonged contact. The literature has long noted the existence of linguistic areas, in which languages from more than one family share sets of traits in common with each other but not with other related languages spoken outside the area (Aikhenvald \& Dixon 2001a, Chirikba 2008). Attempts to minimize such bias include consideration of cultural areas in addition to genealogical affiliations in constructing a language sample, with the ideal sample containing no two languages from the same family or area (Perkins 1985). But while traditional linguistic areas are relatively small and geographically delimited (e.g., the Balkan Sprachbund), studies of individual features have revealed even larger areas of linguistic convergence; e.g., North America has a higher prevalence of head-marking agreement strategies as compared to the rest of the world (Dryer 1989). To minimize the effects of areal bias, Dryer proposes a method by which a language sample may be divided into five large continental areas (later refined to six, Dryer 1992), which can be shown to be independent of one another along at least some typological measures.
\end{styleBody}

\begin{styleBody}
\ \ As many as two-thirds of spoken languages do not have reference materials which are thorough enough to be consulted in any but the most basic of typological surveys (Bakker 2011: 106). As a result, virtually every language sample suffers from severe \textit{bibliographic bias}. The best documented languages in the world tend to correspond to those which, for whatever historical reason, have the greatest political, social, and economic power and wide geographic spread (e.g., English, Mandarin Chinese, Spanish, Arabic). Thus we find that bibliographic bias is genealogically and areally skewed, with small, less powerful language families and remote regions particularly underrepresented. The least documented language families in the world, for example, tend to be found in lowland New Guinea and parts of the Amazon region (Hammarström 2010). Similar to, or perhaps a subset of, bibliographic bias is what Moreno Cabrera describes as the \textit{written language bias}. Most of the world’s languages do not have a written or standard form, and references for such languages often describe a specific dialect or ideolect. When languages do have a written or standard form, reference materials often describe that form. A result of this is that typological studies often compare data from “highly heterogeneous sources”: standardized, highly formal registers for written languages, and unstandardized, informal registers of ideolects for unwritten languages (2008: 118).
\end{styleBody}

\begin{styleBody}
\ \ All three of the sources of bias described above — genealogical, areal, and bibliographic bias — may complicate typological studies of syllable structure. As discussed in the introduction, the prevalence of particular consonant cluster patterns in Indo-European languages may skew general perceptions regarding the universality of these patterns. There are also clear areal asymmetries in the global distribution of syllable structure complexity. Languages with Simple syllable structure tend to be spoken near the equator (Maddieson 2013a). Complex syllable structure has been described as a prominent areal feature of specific regions such as the Caucasus (Chirikba 2008) and the Pacific Northwest (Thompson \& Kinkade 1990), though there is evidence that maximal syllable structure is more strongly associated with genealogical affiliation even within these linguistic areas (Napoleão de Souza 2017). Insofar as the global distribution of syllable structure complexity is genealogically and areally skewed, the issue of bibliographic bias is relevant. For example, underdocumented areas such as the New Guinea are known to have a higher than average proportion of languages with canonical (C)V syllable structure (Maddieson 2013a).
\end{styleBody}

\section[2.1.2 Other factors which may influence phonological structure and syllable complexity]{\rmfamily 2.1.2 Other factors which may influence phonological structure and syllable complexity}
\section[2.1.2.1 Population]{\textbf{2.1.2.1 Population}}
\begin{styleBody}
\ \ Speaker population may have an affect on language structure. It has been proposed that rare or linguistically marked structures, such as Object-initial basic word orders, are more likely to arise and persist in small speech communities (Nettle 1999a). Lingua franca which have wide areal spreads and large speaker populations with many second-language speakers may be more vulnerable to simplificatory pressures than languages whose use is limited to small, close-knit communities (Nettle 1999b, Lupyan \& Dale 2010). These proposals suggest that simpler syllable structures may be found in languages with large speaker populations or in situations of heavy language contact. Recent research on emergent phonology in creoles does not necessarily support the latter claim: despite many claims to the contrary, a broad range of complex syllable patterns may occur in these languages (Schramm 2014). Concerning other phonological properties, a positive correlation between phoneme inventory size and speaker population has been noted (Hay \& Bauer 2007). Proposed motivations for this pattern, including founder effects analogous to those found in genetics (Atkinson 2011), are controversial (Bybee 2011, Maddieson et al. 2011, Hunley et al. 2012, \textit{inter alia}).
\end{styleBody}

\section[2.1.2.2 Language endangerment]{\textbf{2.1.2.2 Language endangerment}}
\begin{styleBody}
\ \ Today, language diversity is in precipitous decline due to rapid social, economic, and environmental changes which have global reach. It has been projected that 84\% or more of today’s languages may be lost within the next century (Nettle 1999b: 113-114). There is evidence that language vitality status may influence linguistic structure itself. Languages which are obsolescing (dying) — currently about 13\% of all living languages (Simons \& Fennig 2018) — are known to undergo special kinds of structural change (see Romaine 2010 for a review of research on the structural effects of language obsolescence). In the phonological systems of obsolescing languages, it is common for distinctive contrasts to be leveled and for the regularity of phonological processes to break down. Attested effects of obsolescence on syllable structure include reduction and simplification of syllable margins (Cook 1989, for obsolescing dialects of Chipewyan and Sarcee) and the deletion of entire unstressed syllables in certain word positions (Mithun 1989, for Oklahoma Cayuga).
\end{styleBody}

\section[2.1.2.3 Ecological factors]{\textbf{2.1.2.3 Ecological factors}}
\begin{styleBody}
\ \ There is a growing body of work investigating the effect of ecological factors on language structure. Recent studies have explored the hypothesis that the sound systems of human languages may be adapted to features of the natural environment such as elevation, ambient temperature, density of vegetation, and ambient dessication (Everett 2013, Maddieson \& Coupé 2015, Everett et al. 2016, Everett 2017). Syllable structure has been a subject of particular focus in this research paradigm. A positive correlation between warm climates and the frequency of CV shapes in languages has been proposed to reflect the communication needs of an outdoor lifestyle, in which more sonorous elements of the speech signal may overcome environmental noise and large distances between speakers (Munroe et al. 1996, Fought et al. 2004). Similarly, a negative correlation between density of vegetation and complex syllable structures may reflect the poor transmission qualities of higher frequency sounds (certain consonants and consonant clusters) in densely forested environments (Maddieson \& Coupé 2015).
\end{styleBody}

\section[2.1.3 Specific considerations in the current study]{\rmfamily 2.1.3 Specific considerations in the current study}
\begin{styleBody}
\ \ Bakker states that “the construction of a sample should follow the precise formulation of the research questions one would like to answer on the basis of it” (2011:106). Recall the two broad research questions motivating this study, presented in (2.1)-(2.2):
\end{styleBody}

\begin{styleBody}
(2.1) \ \ \textit{Do languages with highly complex syllable structure share other phonetic and phonological characteristics such that this group can be classified as a linguistic type?}
\end{styleBody}

\begin{styleBody}
(2.2)\textbf{ }\ \ \textit{How does highly complex syllable structure develop over time?}
\end{styleBody}

\begin{styleBody}
\ \ These research questions, and their more specific formulations laid out in Chapter 1, necessitate a carefully constructed language sample. The first seeks to determine whether there are structural features correlated with highly complex syllable structure such that this phenomenon can be considered characteristic of a language type. For the purposes of gaining a thorough understanding of a language type, genealogical and areal diversity are important considerations in the construction of the sample. However, because this study does not aim to quantify the range or overall distribution of highly complex syllable structure, it is not necessary that the sample be genealogically \textit{balanced} in a systematic way. If quantitative analysis reveals crosslinguistically robust patterns of features correlated with syllable structure complexity, these can inform specific predictions about the evolution of highly complex syllable structure, addressing the second research question. Thus it may actually prove useful to have pairs of related languages with different syllable structure complexity in the sample. The effectiveness of the predictions can then be tested and qualitatively evaluated within these pairs, in which the structures are known to derive from the same origin at some point in the past.
\end{styleBody}

\begin{styleBody}
\ \ In order to test for correlations between syllable structure complexity and other structural features, typological bias (Comrie 1989: 12) must be built into the sample. That is, the languages of the sample must be deliberately chosen on the basis of their syllable patterns, so that features of languages with different syllable structure complexity may be compared against one another in a principled way to determine whether the hypothesized correlations exist and are crosslinguistically robust. An important question, then, is how syllable structure complexity is defined in the current study.
\end{styleBody}

\section[2.2 Defining the categories of syllable structure complexity]{\rmfamily 2.2 Defining the categories of syllable structure complexity}
\begin{styleBody}
\ \ Definitions of syllable structure complexity typically consider the size and phonological structure of the onset and coda.\footnote{The size and structure of the nucleus may also be a consideration; cf. Maddieson et al. (2013).} A classification which considers the size, shape, and differential behavior of onsets and codas, as well as dominant crosslinguistic patterns is given in Maddieson (2013a). While broad, this classification captures prevalent crosslinguistic trends in syllable structure and has proven useful in establishing correlations between syllable structure complexity and other features of language structure, such as consonant phoneme inventory size. In a 486-language survey, languages are classified into three categories according to the size and shape of their largest observed onsets and codas: 
\end{styleBody}

\begin{styleBody}
\textbf{Simple:} languages in which the onset is maximally one C, and codas do not occur.
\end{styleBody}

\begin{styleBody}
\textbf{Moderately Complex:} languages in which the onset is maximally two Cs, the second of which is a liquid or a glide; and/or the coda consists of maximally one C.
\end{styleBody}

\begin{styleBody}
\textbf{Complex:} languages in which the maximal onset is two Cs (the second of which is something other than a liquid or a glide) or larger than two Cs; and/or the maximal coda consists of two or more Cs.
\end{styleBody}

\begin{styleBody}
The distribution of the 486 languages in Maddieson's survey according to these three categories can be found in Table 2.1.
\end{styleBody}

\begin{flushleft}
\tablefirsthead{}
\tablehead{}
\tabletail{}
\tablelasttail{}
\begin{supertabular}{m{1.4212599in}m{0.9212598in}m{0.9212598in}}
\hline
{\bfseries Syllable Structure }

{\bfseries Complexity} &
\centering \textbf{\textit{N}}\textbf{ languages} &
\centering\arraybslash{\bfseries Percentage}\\\hline
{\itshape Simple} &
\centering 61 &
\centering\arraybslash 12.6\%\\
{\itshape Moderately Complex} &
\centering 274 &
\centering\arraybslash 56.4\%\\
{\itshape Complex} &
\centering 151 &
\centering\arraybslash 31.1\%\\\hline
\end{supertabular}
\end{flushleft}
\begin{styleBody}
\textbf{Table 2.1.} Distribution of syllable structure complexity in languages of Maddieson (2013a).
\end{styleBody}

\begin{styleBody}
\textit{\ \ }In contrast to the tightly-defined Simple and Moderately Complex categories, the Complex category in Maddieson (2013a) is diverse and open-ended. Languages in this category range from those whose most complex syllable is only a slight expansion on the Moderately Complex types — such as a CVCC shape in which the first consonant of the coda is limited to a liquid or glide — to those having far more complex structures involving up to eight consonants in a tautosyllabic cluster. In order to better understand the internal structure of the Complex category and the distribution of languages within it, I analyzed the size and shape of the maximal syllable structure of these 151 languages. A list of the languages in the sample, along with the references consulted, can be found in Appendix C. The distribution of these languages can be found in Table 2.2, according to the size of their maximal onsets and codas.\footnote{\textrm{Note that there are only 147 languages represented in Table 2.2. Four of the languages in the Complex category in Maddieson (2013a) — Canela-Krahô, Ik, Indonesian, and Yagaria — have been reclassified as having Moderately Complex syllable structure in the current analysis.}}
\end{styleBody}

\begin{flushleft}
\tablefirsthead{}
\tablehead{}
\tabletail{}
\tablelasttail{}
\begin{supertabular}{m{0.84275985in}|m{0.6004598in}|m{0.5997598in}|m{0.6004598in}|m{0.5997598in}|m{0.6011598in}|m{0.6712598in}|m{0.77195984in}|}
{\raggedleft\bfseries Onset →\par}

\raggedleft{\bfseries Coda ↓} &
\centering{\bfseries C} &
\centering{\bfseries CC} &
\centering{\bfseries CCC} &
\centering{\bfseries CCCC} &
\centering{\bfseries CCCCC} &
\centering{\bfseries CCCCCC} &
\centering\arraybslash{\bfseries CCCCCCC}\\\hline
 &
 &
 &
 &
 &
 &
 &
\\\hline
\raggedleft{\bfseries (none)} &
 &
\centering 2 &
\centering 1 &
\centering 1 &
\centering {}- &
\centering {}- &
\centering\arraybslash {}-\\\hhline{-~------}
\raggedleft{\bfseries C} &
 &
\centering 23 &
\centering 9 &
\centering 1 &
\centering {}- &
\centering {}- &
\centering\arraybslash {}-\\\hline
\raggedleft{\bfseries CC} &
\centering 35 &
\centering 21 &
\centering 13 &
\centering 3 &
\centering {}- &
\centering {}- &
\centering\arraybslash {}-\\\hline
\raggedleft{\bfseries CCC} &
\centering 3 &
\centering 5 &
\centering 8 &
\centering 3 &
\centering {}- &
\centering {}- &
\centering\arraybslash {}-\\\hline
\raggedleft{\bfseries CCCC} &
\centering 4 &
\centering 2 &
\centering 4 &
\centering 2 &
\centering {}- &
\centering 1 &
\centering\arraybslash {}-\\\hline
\raggedleft{\bfseries CCCCC} &
\centering 1 &
\centering {}- &
\centering {}- &
\centering 1 &
\centering {}- &
\centering {}- &
\centering\arraybslash 1\\\hline
\raggedleft{\bfseries CCCCCC} &
\centering {}- &
\centering 1 &
\centering {}- &
\centering {}- &
\centering 1 &
\centering {}- &
\centering\arraybslash {}-\\\hline
\raggedleft{\bfseries CCCCCCC} &
\centering {}- &
\centering {}- &
\centering {}- &
\centering {}- &
\centering {}- &
\centering {}- &
\centering\arraybslash {}-\\\hline
\raggedleft{\bfseries CCCCCCCC} &
\centering {}- &
\centering {}- &
\centering {}- &
\centering 1 &
\centering {}- &
\centering {}- &
\centering\arraybslash {}-\\\hline
\end{supertabular}
\end{flushleft}
\begin{styleBody}
\textbf{Table 2.2. }Distribution of languages in Complex syllable structure category in Maddieson (2013a), according to the size of the most complex onset (columns) and coda (rows) structures observed in each language.
\end{styleBody}

\begin{styleBody}
\ \ The visual distribution of the languages in Table 2.2 is striking: most languages cluster toward the upper lefthand corner of the table. Over half of the languages in the Complex category have syllable-marginal clusters of two consonants or fewer; that is, onsets and codas no more complex than a sequence of two obstruents. Though onsets of up to seven consonants and codas of up to eight consonants occur, it is very rare for a language to have more than four consonants at either margin (7/147, 4.8\%).
\end{styleBody}

\begin{styleBody}
\ \ Another interesting property of the distribution in Table 2.2 is that as cluster size increases, the crosslinguistic size asymmetry in onsets and codas appears to level out and then reverse. As discussed in Chapter 1, it is more common for a language to have complex onsets than complex codas. When smaller cluster sizes are considered, this pattern is clear even within the Complex category: languages which have three or more consonants in the onset are more common (50 languages) than languages which have these patterns in the coda (38 languages). However, when larger clusters are considered, we find that there is a reversal of the pattern: it is \textit{less} common for languages to have onsets of four or more consonants (15 languages) than to have codas of the same size (19 languages). Examining the specific patterns in more depth than what is presented in Table 2.2, it seems that this reversal occurs when clusters of three \textit{obstruents} are taken as the cutoff point: 24 languages have these, or more complex clusters, as an onset, and 26 languages have these, or more complex clusters, as a coda.
\end{styleBody}

\begin{styleBody}\bfseries
2.2.1 Long sequences of obstruents: tautosyllabic clusters or syllabic consonants?
\end{styleBody}

\begin{styleBody}
\ \ The above point relates to an interesting feature of attested large onset and coda patterns, also mentioned in Chapter 1, which is that many of these structures do not exhibit the sonority-related sequencing restrictions and contours that are so common of languages with smaller onsets and codas. It is not unusual to observe tautosyllabic clusters consisting entirely of obstruents in languages with large onsets or codas (2.3)-(2.4):
\end{styleBody}

\begin{styleBody}
(2.3) \ \ \textbf{Cocopa} (\textit{Cochimi-Yuman}; USA, Mexico)
\end{styleBody}

\begin{styleBody}
\ \ \textit{p[282?]t[361?][283?][294?]\'{a}[2D0?]w}
\end{styleBody}

\begin{styleBody}
\ \ ‘I hang up several (things)’
\end{styleBody}

\begin{styleBody}
\ \ (Crawford 1966: 36)
\end{styleBody}

\begin{styleBody}
(2.4) \ \ \textbf{Itelmen} (\textit{Chukotko-Kamchatkan}; Russia)
\end{styleBody}

\begin{styleBody}
\ \ \textit{qsa[26C?]txt[361?][283?]}
\end{styleBody}

\begin{styleBody}
\ \ ‘Follow!’
\end{styleBody}

\begin{styleBody}
\ \ (Georg \& Volodin 1999: 44)
\end{styleBody}

\begin{styleBody}
\ \ Thus a common characteristic of the more extreme cases of Complex syllable structure is the presence of long sequences of obstruents. This phenomenon can also be found in Tashlhiyt, a language which is classified as having Complex syllable structure, but which descriptions list as having only simple (Ridouane 2008) or maximally biconsonantal (Puech \& Louali 1999) onsets and codas. Because the language also has syllabic obstruents, it is possible to find words without vowels which consist entirely of voiceless obstruents (2.5a-b):
\end{styleBody}

\begin{styleBody}
(2.5) \ \ \textbf{Tashlhiyt} (\textit{Afro-Asiatic}; Morocco)
\end{styleBody}

\begin{styleBody}
(a)\ \ \textit{tfs$\chi $t}
\end{styleBody}

\begin{styleBody}
‘you cancelled’
\end{styleBody}

\begin{styleBody}
(b)\ \ \textit{tft$\chi $tst[2D0?]}
\end{styleBody}

\begin{styleBody}
‘you rolled it (\textsc{f})’
\end{styleBody}

\begin{styleBody}
\ \ (Ridouane 2002: 95)
\end{styleBody}

\begin{styleBody}
\ \ There is evidence in the literature that word-initial \textit{onset} \textit{clusters} and word-initial \textit{consonant sequences with syllabic obstruents} exhibit different gestural timing patterns between the rightmost consonant and vowel. Goldstein et al. (2007) instrumentally investigated the predictions of a coupled oscillator model of syllable structure for these two different kinds of word-initial sequences. In Georgian, the timing lag between gestures associated with the rightmost consonant and the vowel decreased with larger word-initial consonant sequences, in line with the model’s predictions for onset clusters. Thus word-initial consonant sequences, such as that in \textit{t[361?]s’k’[27E?]iala }‘shiny clean,’ can be interpreted as syllable onsets in Georgian. In Tashlhiyt, no such pattern was found with increasingly larger word-initial consonant sequences; that is, the instrumental evidence suggested that only the consonant immediately preceding the vowel was coupled to it. Thus the initial sequence in \textit{tsmun} ‘\textsc{3fs}{}-\textsc{caus}{}-accompany’ was interpreted as being syllabified as [ts.mun]. These results, expanded and confirmed by Hermes et al. (2011), lend support to the analysis of Tashlhiyt as having simple onsets and nucleus patterns which include syllabic obstruents, rather than large onset clusters.
\end{styleBody}

\begin{styleBody}
\ \ While the results summarized above suggest that onset clusters and word-initial consonant sequences with syllabic consonants are categorically different structures in terms of articulation, the situation is actually more complex than that. The Georgian data in Goldstein et al. (2007) was from two speakers, but only one speaker showed the expected effect. The other speaker had timing patterns which more closely resembled the Tashlhiyt pattern. An examination of this speaker’s productions revealed the regular presence of an ‘epenthetic’ vowel between [k’] and [[27E?]] in \textit{k'[27E?]iali} ‘glitter’ and \textit{t[361?]s’k’[27E?]iala} ‘shiny clean.’ When epenthesis occurred, the decreased timing lag between the rightmost consonant and vowel was not observed; that is, the word-initial sequence of consonants was split by a syllable boundary. The authors note that epenthesis occurred in the speech of both Georgian speakers, but not always in the same forms. However, when epenthesis did occur, it was in very specific phonological environments: when the place of articulation for C\textsubscript{1} was more posterior than that of C\textsubscript{2}. Setting aside the issue of vowel epenthesis for now,\footnote{\textrm{The issue of vowel epenthesis is an important one which has obvious implications for the analysis of syllable structure; this will be discussed further in §3.2.2.}} an interesting finding from Goldstein et al. (2007) is that both timing patterns may occur for the same word-initial sequence in the same language.
\end{styleBody}

\begin{styleBody}
\ \ A related observation regarding large tautosyllabic clusters and syllabic obstruents is that some languages are analyzed as having both in their canonical syllable structure. For example, Crawford (1966) analyzes Cocopa as having large onset clusters, such as [p[282?]t[361?][283?][294?]] in (2.3) above. In addition to these, he proposes that some unstressed syllables can be entirely consonantal, consisting of “an onset only or of an onset and a coda with a predictable ‘murmur’ vowel following the onset as phonetic peak” (1966: 34). For example, \textit{pt[361?][283?]xmuk\'{a}p }‘he embraced her’ is syllabified as [p\textsuperscript{i}.t[361?][283?]x\textsuperscript{a}.mu.k\'{a}p] (1966: 43; quality of ‘murmur’ vowel determined by environment). Patterns of the latter sort occur for specific consonantal combinations in Cocopa, much like the Georgian patterns described above. 
\end{styleBody}

\begin{styleBody}
\ \ There are also a number of languages which are analyzed by one author as having large tautosyllabic clusters, and by another as having simpler syllable margins but also syllabic obstruents. Hoard (1978) gives several examples of languages from the Pacific Northwest which had previously been analyzed as having large onsets and/or codas but might be better thought of as having smaller syllable margins in addition to syllabic stops and affricates (e.g. Quileute, Puget Salish, and Nez Perce). Hoard bases his analyses on impressionistic transcriptions, considering syllables to reflect “the number of audible pulses” in the form which are delineated by “relative separation or detachment” from other segments or groups of segments (1978: 59-60). Similar disagreements of analysis can be found for Piro (Arawakan; Matteson 1965, Lin 1997, Hanson 2010). Perhaps these languages are like Georgian and Cocopa above, in which different timing patterns may be produced by different speakers or for different consonant combinations. Without experimental data it is difficult to say whether one or both analyses are appropriate. What is clear, however, is that it is not uncommon for the same language, when exhibiting long sequences of consonants (especially obstruents) at word margins, to be alternately described as having large tautosyllabic clusters and/or syllabic obstruents. For this reason I argue that it is appropriate to group languages of both types together in the current study.
\end{styleBody}

\begin{styleBody}
\ \ The above approach also has some support in instrumental findings in the literature regarding the articulatory properties of consonantal nuclei. Pouplier \& Beňuš (2011) found that for syllabic liquids in Slovak, the kinematic properties of the consonantal gestures did not undergo consistent changes in nucleus position. The authors conclude that in this respect, syllables with consonantal nuclei behave like consonant clusters. According to other instrumental measures, syllabic liquids exhibit timing and gestural coordination relationships with adjacent consonants which are reminiscent of those found in onset or coda clusters, but still distinct from those found in both non-syllabic consonant clusters and vocalic nuclei. Similarly, Fougeron \& Ridouane (2008) found that /k/ in Tashlhiyt does not undergo consistent acoustic or articulatory changes when syllabic. However, it does exhibit consistent and stable patterns of temporal alignment and coordination relationships with flanking consonants which are, as noted in the discussion of Goldstein et al. (2007) above, similar to those expected for onset-nucleus-coda patterns. Thus the instrumental evidence point to syllabic consonants as phenomena which exhibit some of the coordination properties of vocalic nuclei while maintaining articulatory properties of non-syllabic consonants.
\end{styleBody}

\begin{styleBody}\bfseries
2.2.2 Highly Complex syllable structure: a definition
\end{styleBody}

\begin{styleBody}
\ \ Motivated by all of the observations above — the distribution of languages in Table 2.2, three-obstruent clusters as the locus of the weakening of the onset/coda asymmetry, and the similar patterns observed in languages with large tautosyllabic obstruent clusters and those with syllabic obstruents — I define Highly Complex syllable structure as follows:
\end{styleBody}

\begin{styleBody}
\textbf{Highly Complex}: languages in which the maximal onset or coda consists of three obstruents, or four or more Cs of any kind; and/or in which syllabic obstruents occur, resulting in word-marginal sequences of three or more obstruents.
\end{styleBody}

\begin{styleBody}
\ \ Table 2.3 shows how the 486 languages in Maddieson (2013a) are distributed with the addition of the Highly Complex category as defined above. The definition of the Complex category is adjusted accordingly, and the Simple and Moderately Complex categories are defined as in the previous work.
\end{styleBody}

\begin{flushleft}
\tablefirsthead{}
\tablehead{}
\tabletail{}
\tablelasttail{}
\begin{supertabular}{m{1.4212599in}m{0.9212598in}m{0.9212598in}}
\hline
{\bfseries Syllable Structure }

{\bfseries Complexity} &
\centering \textbf{\textit{N}}\textbf{ languages} &
\centering\arraybslash{\bfseries Percentage}\\\hline
{\itshape Simple} &
\centering 61 &
\centering\arraybslash 12.6\%\\
{\itshape Moderately Complex} &
\centering 278 &
\centering\arraybslash 57.2\%\\
{\itshape Complex} &
\centering 110 &
\centering\arraybslash 22.6\%\\
{\itshape Highly Complex} &
\centering 37 &
\centering\arraybslash 7.6\%\\\hline
\end{supertabular}
\end{flushleft}
\begin{styleBody}
\textbf{Table 2.3.} A reanalysis of the data in Maddieson (2013a), with the added category of Highly Complex as defined in the current work.
\end{styleBody}

\begin{styleBody}
\ \ While the structural divisions between the syllable structure complexity categories are not evenly distributed, they capture predominant crosslinguistic patterns and serve the specific aims of the current study. These categories provide a four-point scale by which syllable structure complexity can be correlated with other structural features. The Highly Complex category is also defined in such a way as to include the extreme end of the syllable structure complexity cline, but not so narrowly as to introduce extreme genealogical and areal bias into this group. This will allow for meaningful examination, both quantitative and qualitative, of the characteristics of this group of languages.
\end{styleBody}

\section[2.3 Constructing the language sample]{\rmfamily 2.3 Constructing the language sample}
\begin{styleBody}
\ \ Three criteria shaped the design of the language sample for this study: (i) the size of the sample must be large enough for meaningful quantitative analysis, but small enough for in-depth qualitative analysis of the languages with highly complex syllable structure; (ii) the proportional representation of the four syllable structure complexity categories must be similar enough to allow for meaningful comparisons between the groups, and (iii) as per the discussion of bias in typological studies in §2.1, the sample should be both genealogically and areally diverse.
\end{styleBody}

\begin{styleBody}
\ \ In addressing (i), a sample of approximately 100 languages is appropriate. While this sample size is moderate for a phonological typology survey, it is comparable to those used in studies such as Bateman (2007) and Bybee \& Easterday (2019), both of which have quantitative and qualitative components. Given the relative crosslinguistic rarity of languages in the Simple and Highly Complex categories (Table 2.3) and the skewed geographical and genealogical distribution of syllable structure complexity, meeting criteria (ii) and (iii) is a more difficult endeavor. In constructing the sample, an attempt was made to strike a reasonable balance between the ideal sample composition and these practical considerations. The resulting sample is described in the following section.
\end{styleBody}

\section[]{\rmfamily }
\section[2.4 Language sample for current study]{\rmfamily 2.4 Language sample for current study}
\begin{styleBody}
\ \ The language sample includes 24 languages in the Simple category, 26 languages in the Moderately Complex category, 25 languages in the Complex category, and 25 languages in the Highly Complex category. Tables 2.4-2.7 list the languages of the sample by syllable structure complexity, geographical macro-area, and genealogical affiliation. A more detailed list which includes ISO 639-3 codes, speaker populations, and language endangerment and development status, can be found in Appendix A.
\end{styleBody}

\begin{flushleft}
\tablefirsthead{}
\tablehead{}
\tabletail{}
\tablelasttail{}
\begin{supertabular}{m{1.2205598in}m{1.6719599in}m{1.6400598in}m{1.6420599in}}
\hline
{\bfseries Region} &
{\bfseries Language} &
{\bfseries Top-level family} &
{\bfseries\itshape Subfamily}\\\hline
{\scshape Africa} &
{\bfseries Hadza} &
(isolate) &
\\\hhline{-~~~}
 &
{\bfseries Southern Grebo} &
Atlantic-Congo &
{\itshape Volta-Congo}\\
 &
{\bfseries Yoruba} &
Atlantic-Congo &
{\itshape Volta-Congo}\\
 &
{\bfseries Ma’di} &
Central Sudanic &
{\itshape Moru-Madi}\\
 &
{\bfseries Southern Bobo Madaré} &
Mande &
{\itshape Western Mande}\\\hline
{\scshape Australia }

{\scshape \& New Guinea} &
{\bfseries Savosavo} &
(isolate) &
\\\hhline{-~~~}
 &
{\bfseries Grass Koiari} &
Koiarian &
{\itshape Koiaric}\\
 &
{\bfseries Rotokas} &
North Bougainville &
{\itshape Rotokas-Askopan}\\
 &
{\bfseries East Kewa} &
Nuclear Trans New Guinea &
{\itshape Enga-Kewa-Huli}\\\hline
{\scshape North }

{\scshape America} &
{\bfseries Towa} &
Kiowa-Tanoan &
\\\hhline{-~~~}
 &
{\bfseries Pinotepa Mixtec} &
Otomanguean &
{\itshape Eastern Otomanguean}\\
 &
{\bfseries Ute} &
Uto-Aztecan &
{\itshape Northern Uto-Aztecan}\\\hline
{\scshape South }

{\scshape America} &
{\bfseries Urarina} &
(isolate) &
\\\hhline{-~~~}
 &
{\bfseries Warao} &
(isolate) &
\\
 &
{\bfseries Apurinã} &
Arawakan &
{\itshape Southern Maipuran}\\
 &
{\bfseries Murui Huitoto} &
Huitotoan &
{\itshape Nuclear Witotoan}\\
 &
{\bfseries Cavineña} &
Pano-Tacanan &
{\itshape Tacanan}\\
 &
{\bfseries Cubeo} &
Tucanoan &
{\itshape Eastern Tucanoan}\\\hline
{\scshape Southeast Asia }

{\scshape \& Oceania} &
{\bfseries Rukai (Budai dialect)} &
Austronesian &
\\\hhline{-~~~}
 &
{\bfseries Maori} &
Austronesian &
{\itshape Malayo-Polynesian}\\
 &
{\bfseries Tukang Besi North} &
Austronesian &
{\itshape Malayo-Polynesian}\\
 &
{\bfseries Saaroa} &
Austronesian &
{\itshape Tsouic}\\
 &
{\bfseries Sichuan Yi} &
Sino-Tibetan &
{\itshape Burmo-Qiangic}\\
 &
{\bfseries Sumi Naga} &
Sino-Tibetan &
{\itshape Kuki-Chin-Naga}\\\hhline{~---}
\end{supertabular}
\end{flushleft}
\begin{styleBody}
\textbf{Table 2.4. }Languages in \textbf{Simple} syllable structure category, by macro-area and genealogical affiliation.
\end{styleBody}

\clearpage\begin{flushleft}
\tablefirsthead{}
\tablehead{}
\tabletail{}
\tablelasttail{}
\begin{supertabular}{m{1.2205598in}m{1.6719599in}m{1.6400598in}m{1.6420599in}}
\hline
{\bfseries Region} &
{\bfseries Language} &
{\bfseries Top-level family} &
{\bfseries\itshape Subfamily}\\\hline
{\scshape Africa} &
{\bfseries Kambaata} &
Afro-Asiatic &
{\itshape Cushitic}\\\hhline{-~~~}
 &
{\bfseries Ewe} &
Atlantic-Congo &
{\itshape Volta-Congo}\\
 &
{\bfseries Fur} &
Furan &
\\
 &
{\bfseries Kanuri} &
Saharan &
{\itshape Western Saharan}\\\hline
{\scshape Australia }

{\scshape \& New Guinea} &
{\bfseries Maybrat} &
Maybrat-Karon &
\\\hhline{-~~~}
 &
{\bfseries Kamasau} &
Nuclear Torricelli &
{\itshape Marienberg}\\
 &
{\bfseries Selepet} &
Nuclear Trans New Guinea &
{\itshape Finisterre-Huon}\\
 &
{\bfseries Alyawarra} &
Pama-Nyungan &
{\itshape Arandic-Thura-Yura}\\\hline
{\scshape Eurasia} &
{\bfseries Kharia} &
Austroasiatic &
{\itshape Munda}\\\hhline{-~~~}
 &
{\bfseries Telugu} &
Dravidian &
{\itshape South Dravidian}\\
 &
{\bfseries Darai} &
Indo-European &
{\itshape Indo-Iranian}\\
 &
{\bfseries Tu} &
Mongolic &
{\itshape Southern Periphery Mongolic}\\
 &
{\bfseries Eastern Khanty} &
Uralic &
{\itshape Khantyic}\\\hline
{\scshape North }

{\scshape America} &
{\bfseries Karok} &
(isolate) &
\\\hhline{-~~~}
 &
{\bfseries North Slavey \newline
(Hare dialect)} &
Athabaskan-Eyak-Tlingit &
{\itshape Athabaskan-Eyak}\\
 &
{\bfseries Kalaallisut} &
Eskimo-Aleut &
{\itshape Eskimo}\\
 &
{\bfseries Choctaw} &
Muskogean &
{\itshape Western Muskogean}\\\hline
{\scshape South }

{\scshape America} &
{\bfseries Carib} &
Cariban &
{\itshape Guianan}\\\hhline{-~~~}
 &
{\bfseries Imbabura Highland Quichua} &
Quechuan &
{\itshape Quechua II}\\
 &
{\bfseries Cocama-Cocamilla} &
Tupian &
{\itshape Maweti-Guarani}\\\hline
{\scshape Southeast Asia }

{\scshape \& Oceania} &
{\bfseries Pacoh} &
Austroasiatic &
{\itshape Katuic}\\\hhline{-~~~}
 &
{\bfseries Paiwan} &
Austronesian &
\\
 &
{\bfseries Kim Mun \newline
(Vietnam dialect)} &
Hmong-Mien &
{\itshape Mienic}\\
 &
{\bfseries Atong} &
Sino-Tibetan &
{\itshape Brahmaputran}\\
 &
{\bfseries Cantonese} &
Sino-Tibetan &
{\itshape Sinitic}\\
 &
{\bfseries Lao} &
Tai-Kadai &
{\itshape Kam-Tai}\\\hhline{~---}
\end{supertabular}
\end{flushleft}
\begin{styleBody}
\textbf{Table 2.5.} Languages in \textbf{Moderately Complex} syllable structure category, by macro-area and genealogical affiliation.
\end{styleBody}

\clearpage\begin{flushleft}
\tablefirsthead{}
\tablehead{}
\tabletail{}
\tablelasttail{}
\begin{supertabular}{m{1.2205598in}m{1.6719599in}m{1.6400598in}m{1.6420599in}}
\hline
{\bfseries\scshape Region} &
{\bfseries Language} &
{\bfseries Top-level family} &
{\bfseries\itshape Subfamily}\\\hline
{\scshape Africa} &
{\bfseries Mpade (Makari dialect)} &
Afro-Asiatic &
{\itshape Chadic}\\\hhline{-~~~}
 &
{\bfseries Jola-Fonyi} &
Atlantic-Congo &
{\itshape North-Central Atlantic}\\
 &
{\bfseries Lunda} &
Atlantic-Congo &
{\itshape Volta-Congo}\\
 &
{\bfseries Dizin (Central dialect)} &
Dizoid &
\\
 &
{\bfseries Gaam} &
Eastern Jebel &
\\\hline
{\scshape Australia }

{\scshape \& New Guinea} &
{\bfseries Mangarrayi} &
Mangarrayi-Maran &
\\\hhline{-~~~}
 &
{\bfseries Nimboran} &
Nimboranic &
\\
 &
{\bfseries Oksapmin} &
Nuclear Trans New Guinea &
{\itshape Asman-Awyu-Ok}\\
 &
{\bfseries Bardi} &
Nyulnyulan &
{\itshape Western Nyulnyulan}\\
 &
{\bfseries Ngarinyin} &
Worrorran &
\\\hline
{\scshape Eurasia} &
{\bfseries Basque (Central dialect)} &
(isolate) &
\\\hhline{-~~~}
 &
{\bfseries Burushaski} &
(isolate) &
\\
 &
{\bfseries Nivkh \newline
(West Sakhalin dialect)} &
(isolate) &
\\
 &
{\bfseries Bashkir} &
Turkic &
{\itshape Common Turkic}\\
 &
{\bfseries Ket} &
Yeniseian &
{\itshape Northern Yeniseian}\\\hline
{\scshape North }

{\scshape America} &
{\bfseries Pech} &
Chibchan &
\\\hhline{-~~~}
 &
{\bfseries Tzeltal \newline
(Aguacatenango dialect)} &
Mayan &
{\itshape Core Mayan}\\
 &
{\bfseries Lakota} &
Siouan &
{\itshape Core Siouan}\\\hline
{\scshape South }

{\scshape America} &
{\bfseries Kadiwéu} &
Guaicuruan &
\\\hhline{-~~~}
 &
{\bfseries Mamaindê} &
Nambiquaran &
{\itshape Nambikwara Complex}\\
 &
{\bfseries Apinayé} &
Nuclear-Macro-Je &
{\itshape Je}\\
 &
{\bfseries Chipaya} &
Uru-Chipaya &
\\\hline
{\scshape Southeast Asia }

{\scshape \& Oceania} &
{\bfseries Koho (Sre dialect)} &
Austroasiatic &
{\itshape Bahnaric}\\\hhline{-~~~}
 &
{\bfseries Lelepa} &
Austronesian &
{\itshape Malayo-Polynesian}\\
 &
{\bfseries Lepcha} &
Sino-Tibetan &
{\itshape Himalayish}\\\hhline{~---}
\end{supertabular}
\end{flushleft}
\begin{styleBody}
\textbf{Table 2.6.} Languages in \textbf{Complex} syllable structure category, by macro-area and genealogical affiliation. 
\end{styleBody}

\begin{flushleft}
\tablefirsthead{}
\tablehead{}
\tabletail{}
\tablelasttail{}
\begin{supertabular}{m{1.2205598in}m{1.6719599in}m{1.6400598in}m{1.6420599in}}
\hline
{\bfseries\scshape Region} &
{\bfseries Language} &
{\bfseries Top-level family} &
{\bfseries\itshape Subfamily}\\\hline
{\scshape Africa} &
{\bfseries Tashlhiyt} &
Afro-Asiatic &
{\itshape Berber}\\\hhline{-~~~}
 &
{\bfseries Doyayo} &
Atlantic-Congo &
{\itshape Volta-Congo}\\
 &
{\bfseries Bench} &
Ta-Ne-Omotic &
\\\hline
{\scshape Australia }

{\scshape \& New Guinea} &
{\bfseries Menya} &
Angan &
{\itshape Nuclear Angan}\\\hhline{-~~~}
 &
{\bfseries Kunjen }

{\bfseries (Oykangand dialect)} &
Pama-Nyungan &
{\itshape Paman}\\
 &
{\bfseries Alamblak} &
Sepik &
{\itshape Sepik Hill}\\
 &
{\bfseries Wutung} &
Sko &
{\itshape Nuclear Skou-Serra-Piore}\\\hline
{\scshape Eurasia} &
{\bfseries Kabardian} &
Abkhaz-Adyge &
{\itshape Circassian}\\\hhline{-~~~}
 &
{\bfseries Itelmen} &
Chukotko-Kamchatkan &
\\
 &
{\bfseries Albanian (Tosk dialect)} &
Indo-European &
{\itshape Albanian}\\
 &
{\bfseries Polish} &
Indo-European &
{\itshape Balto-Slavic}\\
 &
{\bfseries Georgian} &
Kartvelian &
{\itshape Georgian-Zan}\\
 &
{\bfseries Lezgian} &
Nakh-Daghestanian &
{\itshape Daghestanian}\\\hline
{\scshape North }

{\scshape America} &
{\bfseries Passamaquoddy-Maliseet} &
Algic &
{\itshape Algonquian}\\\hhline{-~~~}
 &
{\bfseries Cocopa} &
Cochimi-Yuman &
{\itshape Yuman}\\
 &
{\bfseries Mohawk} &
Iroquoian &
{\itshape Northern Iroquoian}\\
 &
{\bfseries Yakima Sahaptin} &
Sahaptian &
{\itshape Sahaptin}\\
 &
{\bfseries Thompson} &
Salishan &
{\itshape Interior Salish}\\
 &
{\bfseries Tohono O’odham} &
Uto-Aztecan &
{\itshape Southern Uto-Aztecan}\\
 &
{\bfseries Nuu-chah-nulth} &
Wakashan &
{\itshape Southern Wakashan}\\\hline
{\scshape South }

{\scshape America} &
{\bfseries Camsá} &
(isolate) &
\\\hhline{-~~~}
 &
{\bfseries Yine} &
Arawakan &
{\itshape Southern Maipuran}\\
 &
{\bfseries Tehuelche} &
Chonan &
{\itshape Continental Chonan}\\
 &
{\bfseries Qawasqar} &
Kawesqar &
{\itshape North Central Alacalufan}\\\hline
{\scshape Southeast Asia }

{\scshape \& Oceania} &
{\bfseries Semai} &
Austroasiatic &
{\itshape Aslian}\\\hline
\end{supertabular}
\end{flushleft}
\begin{styleBody}
\textbf{Table 2.7.} Languages in \textbf{Highly Complex} syllable structure category, by macro-area and genealogical affiliation.
\end{styleBody}

\section[2.4.1 Areal features of sample]{\rmfamily 2.4.1 Areal features of sample}
\begin{styleBody}
\ \ The 100 languages are roughly evenly distributed among the six geographical macro-areas as defined by Dryer (1989, 1992).\footnote{\textrm{The geographical macro-areas are specifically defined as follows (Dryer 1989: 268; 1992: 83, 133-5). }\textrm{\textbf{Africa}}\textrm{: continent of Africa, including Semitic languages of southwest Asia. }\textrm{\textbf{Australia \& New Guinea}}\textrm{: Australian continent and Melanesia, excluding Austronesian languages of Melanesia. }\textrm{\textbf{Eurasia}}\textrm{: Eurasian landmass, excluding Semitic and languages from families of southeast Asia as defined below, and including the Munda languages of Austroasiatic. }\textrm{\textbf{North America}}\textrm{: North American continent, including languages of Mexico, Mayan and Aztecan languages in Central America, and some branches of Chibchan-Paezan. }\textrm{\textbf{South America}}\textrm{: South American continent, including languages of Central America except Mayan and Aztecan languages, and some Chibchan-Paezan branches. }\textrm{\textbf{Southeast Asia \& Oceania}}\textrm{: Southeast Asian region, including all Sino-Tibetan, Tai-Kadai, Hmong-Mien, and Austroasiatic languages excluding Munda, and Oceania region (Austronesian languages).}} Africa, Australia \& New Guinea, North America, and South America are represented by 17 languages each. Two regions, Eurasia and Southeast Asia \& Oceania, are represented by 16 languages each. See Figure 2.1 for a plotted map of the languages of the sample.
\end{styleBody}

\begin{styleBody}
[MAP]
\end{styleBody}

\begin{styleBody}
\textbf{Figure 2.1.} Geographic distribution of languages in the sample, with colors denoting syllable structure complexity. S=Simple, MC=Moderately Complex, C=Complex, HC=Highly Complex.
\end{styleBody}

\begin{styleBody}
\ \ There are some asymmetries in the areal representation of syllable structure complexity in the sample. In Eurasia, the Simple category is entirely unrepresented. In Southeast Asia \& Oceania, the Highly Complex category is represented by only one language. In North America, the Highly Complex category is relatively overrepresented, accounting for seven of the languages in that region.
\end{styleBody}

\section[2.4.2 Genealogical features of sample]{\rmfamily 2.4.2 Genealogical features of sample}
\begin{styleBody}
\ \ The 100 languages of the sample belong to 74 different language families.\footnote{\textrm{For genealogical affiliations, I use the classifications in Glottolog 3.3 (Hammarström et al. 2018).}} 64 of the language families are represented by one language each; this figure includes nine language isolates and many small and mid-sized families. Another three language families — Arawakan, Pama-Nyungan, and Uto-Aztecan — are represented in the sample by two languages each. All of the language families represented by more than two languages in the sample — Afro-Asiatic, Atlantic-Congo, Austroasiatic, Austronesian, Indo-European, Nuclear Trans New Guinea, and Sino-Tibetan — are within the top ten in the world by size in number of languages (Hammarström et al. 2018). Every attempt was made to maximize the diversity of subfamilies within the top-level families represented by more than one language in the sample. Subfamilies represented by more than one language are Volta-Congo (Atlantic-Congo family, five languages), Malayo-Polynesian (Austronesian family, three languages) and Southern Maipuran (Arawakan family, two languages). There are no pidgins or creoles in the sample.\footnote{There is debate over whether Cocama-Cocamilla may be appropriately classified as a creole. While the language is typically classified as Tupi-Guaraní, it is quite divergent from other languages of the family in aspects of its phonology and morphosyntax, perhaps owing to factors of language contact. See Vallejos Yopán (2010) for a critical discussion of this topic.}
\end{styleBody}

\begin{styleBody}
\ \ Most often, a language family is represented by only one language within a syllable structure complexity category. The Simple category is somewhat less genealogically diverse than the others, with only 19 families represented by the 24 languages.
\end{styleBody}

\begin{styleBody}
\ \ Recall from §2.1.3 that an important feature of the language sample design is the inclusion of pairs of related languages with maximally different syllable structure complexities. This allows for hypotheses about the diachronic development of highly complex syllable structure to be tested at the local level. The sample includes five relevant pairs from five macro-areas: Ute (Simple) and Tohono O’odham (Highly Complex), both Uto-Aztecan from North America; Apurinã (Simple) and Yine (Highly Complex), both Arawakan from South America; Yoruba (Simple) and Lunda (Complex), both Atlantic-Congo from Africa; Darai (Moderately Complex) and Albanian (Highly Complex), both Indo-European from Eurasia; and Maori (Simple) and Lelepa (Complex), both Austronesian from Southeast Asia \& Oceania. Some pairs of languages are more closely related than others.\footnote{The region of Australia \& New Guinea is not represented in these pair comparisons. The only family represented by more than one language in this region is Nuclear Trans New Guinea; however some of its subfamily classifications are controversial or not well demonstrated (Pawley 2005).}
\end{styleBody}

\section[2.4.3 Sociolinguistic features of sample]{\rmfamily 2.4.3 Sociolinguistic features of sample}
\begin{styleBody}
\ \ Speaker population data and language vitality status classifications for the languages of the sample can be found in Appendix A.
\end{styleBody}

\begin{styleBody}
\ \ The L1 speaker populations for the languages of the sample vary widely, ranging from five for Tehuelche (Chonan, Patagonia) and Yakima Sahaptin (Sahaptian, Pacific Northwest) to 74,244,300 for Telugu (Dravidian, Southern India). In general, the languages in the Simple and Moderately Complex categories have larger speaker populations than those in the Complex and Highly Complex categories, though there are many exceptions to this trend. Of the 26 languages with fewer than 1,000 speakers, 18 of them have Complex or Highly Complex syllable structure, and half of the ten languages with fewer than 100 speakers have Highly Complex syllable structure. The prevalence of very smaller speaker populations among languages in these categories is no doubt related to the high concentration of these languages in geographical areas with high rates of language endangerment.
\end{styleBody}

\begin{styleBody}
\ \ The Expanded Graded Intergenerational Disruption Scale (EGIDS) is an assessment of language vitality developed by Lewis \& Simons (2010), following work by Fishman (1991), UNESCO (Brenzinger et al. 2003), and others. It considers many different factors of language use, including rates and means of intergenerational transmission, domains of language use, and official recognition of the language. Using EGIDS as a starting point, Ethnologue (Simons \& Fennig 2018) has developed a coarse-grained estimate of the relative development versus endangerment of languages. By this measure, languages are classified into categories according to their development/vitality status: Institutional, Developing, Vigorous, In Trouble, and Dying.\footnote{\textrm{The language development categories are defined as follows (Simons \& Fennig 2018). }\textrm{\textbf{Institutional: }}\textrm{language has wide use in the home and community and official status at educational, provincial, national, and/or international levels. }\textrm{\textbf{Developing: }}\textrm{language is used in the home, community, and sometimes broader contexts, and in initial stages of developing a system of writing and standardization.}\textrm{\textit{ }}\textrm{\textbf{Vigorous: }}\textrm{language is used in the home and community by speakers of all generations, but has not yet developed a system of graphization or standardization}\textrm{\textit{. }}\textrm{\textbf{In Trouble:}}\textrm{ language is currently in the process of losing intergenerational transmission, with the community shifting to other languages for daily use, but there are still speakers of child-bearing age.}\textrm{\textit{ }}\textrm{\textbf{Dying:}}\textrm{ language has lost intergenerational transmission entirely, all fluent speakers are above child-bearing age.}} Languages with robust vitality are more common in the Simple and Moderately Complex portions of the sample. The Complex and Highly Complex categories have the highest proportion of languages classified as Dying (6/25 languages in each category). It should be noted that in North America, which has perhaps the highest proportion of languages with Highly Complex patterns of all the macro-areas, rates of language endangerment and language loss are extreme. Ethnologue 21 classifies 237/254 (93\%) of the living languages spoken north of the US-Mexico border in North America as In Trouble or Dying (Simons \& Fennig 2018). 
\end{styleBody}

\begin{styleBody}
\ \ The sociolinguistic features of the language sample lead to an interesting observation: highly complex syllable structure is a rare language feature by many different measures, including non-structural ones. As described in Chapter 1, highly complex phonotactic patterns are often treated as anomalies and theoretical outliers, especially when occurring in underdescribed non-European languages. A very small proportion of the world’s languages have structures of this kind. Many of the languages with these structures are found in parts of the world where language endangerment rates are particularly high, and have correspondingly small speaker populations. Thus highly complex syllable structure is a \ marginalized pattern in theoretical, descriptive, historical, and social terms. This is all the more reason to dedicate a typological study to this linguistic feature.
\end{styleBody}

\section[2.5 Data collection]{\rmfamily 2.5 Data collection}
\begin{styleBody}
\ \ The data used for this study was collected from published reference grammars, phonetic and phonological studies, and other relevant language descriptions. In a few cases, an expert researcher on the given language was additionally consulted. Data was collected with the guidance of coding sheets whose questions were designed to address the research questions and hypotheses of each chapter. As the methodology behind the coding of the data differs for each part of the study, it will be discussed separately within each chapter.
\end{styleBody}

\begin{styleBody}
\ \ Because the language references consulted were written in different time periods and reflect a variety of descriptive practices, they use diverse transcription standards. Using the phonetic descriptions of sounds in the language references, all phoneme inventories and phonetic and phonemic transcriptions have been transcribed using International Phonetic Alphabet (IPA) symbols. Where there is some ambiguity in interpreting the phonetic description provided by the source consulted, this has been noted in Appendix B. In the examples given in this chapter and throughout the book, non-IPA symbols have been replaced by their IPA equivalents.
\end{styleBody}

\clearpage\subsection[Chapter 3: Syllable structure patterns in sample]{\rmfamily Chapter 3: Syllable structure patterns in sample}
\begin{styleBody}
\ \ In this chapter properties of syllable structure in the language sample are presented. In §3.1 common topics of research in crosslinguistic studies of syllable structure and specific considerations in the current study are discussed. In §3.2 the methodology and coding strategies are described. In §3.3 the results on onset, coda, and nucleus patterns, as well as morphological constituency patterns in maximal clusters, are presented for the language sample as a whole. In §3.4 a more detailed analysis of the syllable patterns of the languages in the sample with Highly Complex syllable structure is presented. In §3.5 the findings are summarized and related to following chapters.
\end{styleBody}

\section[3.1 Introduction]{\rmfamily 3.1 Introduction}
\section[3.1.1 Crosslinguistic studies of syllable structure]{\rmfamily 3.1.1 Crosslinguistic studies of syllable structure}
\begin{styleBody}
\ \ A common approach to studying syllable structure on a crosslinguistic scale is to compare canonical syllable patterns across languages. This is the range of occurring syllable patterns in a language represented as a sequence of Cs for consonants and Vs for vowels, with parentheses indicating optional components of the onset, nucleus, and coda (e.g., (C)(C)(C)V(C)(C)(C)(C) for English). Many databases of phonological patterns include canonical syllable structure as one of the coded features along with consonant and vowel phoneme inventories; e.g., \ the World Phonotactics Database (Donohue et al. 2013), LAPSyD (Maddieson et al. 2013), and a modified version of the WALS 100-language sample presented in Gordon (2016). Though it is a very general measure, the size and shape of canonical syllable patterns can be used to categorize languages in such a way as to capture predominant global trends (cf. Maddieson 2013a).
\end{styleBody}

\begin{styleBody}
\ \ More often, crosslinguistic studies of syllable patterns are focused on finer-grained aspects of syllable structure, including sub-syllabic constituents. A number of studies investigating the properties of syllable margins have revealed trends regarding the size, voicing, place, manner, and sonority of consonant sequences in the onset and coda, as well as implications regarding the makeup of onset and coda inventories (cf. Greenberg 1965/1978, see Chapter 1). Large-scale studies of syllable margins are typically limited to biconsonantal clusters, as these are crosslinguistically the most frequent cluster type, but larger clusters have been explored as well (VanDam 2004, and a few of the analyses in Greenberg 1965/1978). A few typological studies of cluster patterns focus specifically on implicational relationships in obstruent clusters. For instance, Morelli (1999, 2003) examines implicational relationships in biconsonantal onsets composed of stops and fricatives in 30 languages, while Kreitman (2008) reports implicational relationships among obstruent clusters of mixed sonority and voicing in 62 languages. There are also studies which examine the patterns of simple onsets and codas: Rousset (2004) compares, among other things, the relative proportion of consonants which occur in onset versus coda position in 15 diverse languages. Nucleus patterns have also been the subject of typological investigation, with much of the research emphasizing the sonority-based implications that can be gleaned from global distributions of syllable nuclei patterns (Blevins 1995, Zec 2007). As discussed in §1.2.2.4, Bell (1978a) explores global and areal patterns of syllabic consonants in a sample of 182 diverse languages, deriving implicational generalizations regarding the manner and place of articulation of such sounds. Hoard (1978) is a survey of syllabic obstruent patterns in languages of the Pacific Northwest and Northwest Plateau regions of North America, and includes analyses of data from five diverse languages as well as brief references to others.
\end{styleBody}

\begin{styleBody}
\ \ Relationships among the different constituents of the syllable have also been investigated crosslinguistically. An analysis of syllable frequencies in the lexicons of five languages revealed that simple onsets and nuclei — CV sequences — may combine relatively freely (Maddieson \& Precoda 1992). By contrast, relationships between other pairs of subsyllabic constituents tend to show more restrictions across languages. Onsets and codas have often been treated independently of one another with respect to their internal patterns and relationships to the larger syllable; however, Davis \& Baertsch (2011) show that the coda and the second member of a biconsonantal onset are often restricted to the same subset of consonants in a language. Similarly, Blevins (2006) observes that languages with only open syllables tend to have optional onsets.\textbf{ }Gordon (2006) investigates the wide range of behavior exhibited by languages with respect to issues of syllable weight, which may be dependent upon complex combinations of nucleus and coda patterns, and minimal requirements for root and/or word structures, which may implicate onset patterns in addition to those of the rime.
\end{styleBody}

\begin{styleBody}
\ \ Many researchers have investigated the properties of syllables in contact with one another or within the context of larger domains. Crosslinguistically, syllabification patterns are such that sonority tends to fall from the coda of one syllable to the onset of the next; this observation is supported by evidence from diachronic processes of sound change in various languages (Hooper 1976, Murray \& Vennemann 1983). Syllable margins can also exhibit differing patterns according to their position within larger domains. Côté (2011) observes crosslinguistic asymmetries in coda patterns in final versus medial position in stem, word, and phrasal domains. Similarly, ‘prominent’ environments such as word- and phrase-initial syllables and stressed syllables are often the locus of the highest amount of phonemic contrast and variety in syllable margins in a language, though there are areal exceptions to this global trend. For example, Gasser \& Bowern (2014) note more onset restrictions in word-initial as compared to word-internal position in Australian languages.
\end{styleBody}

\begin{styleBody}
\ \ Crosslinguistic studies also investigate the distribution of syllable types within syllable inventories, lexicons, and words of varying lengths in order to determine frequency patterns and uncover implicational hierarchies. The presence of VC structures in a language, for example, generally implies the presence of V, CV, and CVC structures as well (Blevins 1995). Frequency distributions of syllable types in the lexicons of diverse languages reveal a heavy crosslinguistic dominance of the CV type, despite wide variation in canonical syllable structures (Rousset 2004, Vallée et al. 2009). There is also evidence of a relationship between syllable length and word length. Fenk \& Fenk-Oczlon (1993) tested Menzerath’s Law (paraphrased as ‘the bigger the whole, the smaller the parts’) in a sample of 29 languages and found a significant negative linear correlation between the number of syllables per word and the number of phonemes per syllable.
\end{styleBody}

\begin{styleBody}
\ \ A facet of syllable structure that has received limited crosslinguistic treatment is that of the morphological constituency of clusters at syllable margins. Several of the studies listed above (e.g., Greenberg 1965/1978, Morelli 1999, Kreitman 2008) exclude morphologically complex clusters from analysis on the assumption that these clusters may exhibit different patterns than clusters found within the boundaries of a single morpheme. Dressler \& Dziubalska-Kołaczyk (2006) examine syllable patterns in a sample of four Indo-European languages and propose two types of clusters defined by morphological constituency: phonotactic clusters, which are phonologically motivated and morpheme-internal, and morphonotactic clusters, which come about through morphological processes. They conclude that the latter are often longer and more phonologically marked in comparison to the former (see also Dressler et al. 2010, Orzechowska 2012, and Dressler et al. 2015 for other studies in this vein).
\end{styleBody}

\section[3.1.2 Considerations in the current chapter]{\rmfamily 3.1.2 Considerations in the current chapter}
\begin{styleBody}
\ \ While all of the above issues are important in developing a detailed understanding of crosslinguistic syllable patterns, for practical purposes the current study is limited to exploring just a few of these in depth. In this chapter I investigate issues of syllable structure that are directly pertinent to addressing the main research questions and hypotheses of this study. Specifically, I limit the scope of analysis here to features of syllable structure which have been previously demonstrated or hinted in the literature to be correlated with other linguistic, and especially phonological, features, and those which I hypothesize may reveal clues about the diachronic development of highly complex syllable structure. Additionally, I examine the syllable patterns of the languages in the Highly Complex portion of the sample in greater detail than the other categories, in order to develop a better understanding of the coherence of that group. The features under consideration are listed in (3.1)-(3.2):
\end{styleBody}

\begin{styleBody}
(3.1) \ \ \textbf{Features examined for entire sample}
\end{styleBody}

\begin{styleBody}
(a) \ \ \textit{Canonical syllable structure}
\end{styleBody}

\begin{styleBody}
(b) \ \ \textit{Nucleus patterns}
\end{styleBody}

\begin{styleBody}
(c) \ \ \textit{Morphological constituency of maximal syllable margins and syllabic consonants}
\end{styleBody}

\begin{styleBody}
(3.2) \ \ \textbf{Features examined for languages in the Highly Complex category}
\end{styleBody}

\begin{styleBody}
(a) \ \ \textit{Specific Highly Complex syllable patterns occurring}
\end{styleBody}

\begin{styleBody}
(b) \ \ \textit{Restrictions on Highly Complex syllable patterns}
\end{styleBody}

\begin{styleBody}
(c) \ \ \textit{Relative frequency of Highly Complex patterns within languages}
\end{styleBody}

\begin{styleBody}
(d) \ \ \textit{Phonetic characteristics of Highly Complex clusters}
\end{styleBody}

\begin{styleBody}
\ \ The coding of feature (3.1a), canonical syllable structure, is motivated by previous findings in the research. As mentioned in Chapter 1, positive correlations have been established between syllable structure complexity (defined categorically and holistically with reference to canonical syllable structure) and consonant phoneme inventory size (Maddieson 2013a). A positive relationship has also been established between syllable structure complexity and the number of consonants belonging to certain classes within a language (Maddieson et al. 2013). Gordon (2016) has demonstrated that the trend by which consonant phoneme inventory size increases with syllable complexity occurs on a more fine-grained level as well, when complexity is measured as the combined sum of canonical onset and coda constituents. Along another line of inquiry, Blevins (2006: 336) has described a crosslinguistic tendency by which languages without codas tend to have optional onsets. This suggests a relationship between obligatoriness of constituents and canonical syllable structure patterns which, to my knowledge, has not been investigated in a language sample controlled for syllable structure complexity.
\end{styleBody}

\begin{styleBody}
\ \ Nucleus patterns (3.1b), and syllabic consonant patterns more specifically, have also been suggested in the literature to bear some relation to syllable structure complexity. On the one hand, the holistic phonological typology proposed by Isačenko (1939/1940) predicts that ‘vocalic’ languages — those which have shorter consonant sequences and relatively higher vowel/consonant ratios — are more likely to develop syllabic consonants than ‘consonantal’ languages. This suggests that we might expect a greater prevalence of syllabic consonants in languages of the Simple and Moderately Complex categories. On the other hand, Bell (1978a) notes that syllabic consonants often come about through vowel deletion, often in unstressed syllables, which is also known to be a common source of consonant clusters. Based on that observation, we might expect a higher occurrence of syllabic consonants in languages with Complex or Highly Complex syllable structure. The way that the language sample is constructed allows us to test these predictions directly. My hypothesis, rooted in the discussion of speech rhythm typology in Chapter 1, is that the latter prediction will be borne out (3.3).
\end{styleBody}

\begin{styleBody}
(3.3) \ \ \textit{Languages with more complex syllable structure are more likely to have syllabic consonants.}
\end{styleBody}

\begin{styleBody}
\ \ Feature (3.1c), morphological constituency, is related to both of the issues discussed above. Greenberg (1965/1978: 250) and Dressler \& Dziubalska-Kołaczyk (2006) predict that as the size of a syllable margin increases, so does the probability that it contains morpheme boundaries. Additionally, syllabic consonants are often noted to be largely or entirely restricted to grammatical particles and affixes (Bell 1978a: 159), suggesting additional potential interactions between syllable structure complexity and morphology if syllabic consonants are found to occur more frequently in languages on one end of the syllable complexity scale. In coding for (3.1c) I consider the morphological patterns of the largest syllable margins in each language, as well as the kinds of morphemes in which syllabic consonants occur in the languages which have them. These analyses will provide a rough measure of the relationship between morphology and the syllable patterns of a language, which can then be explored in greater depth at a later point, especially if the results point towards a heavy role of morphology in the development of highly complex syllable structure. My hypotheses regarding morphological constituency follow (3.4a-b):
\end{styleBody}

\begin{styleBody}
(3.4)
\end{styleBody}

\begin{styleBody}
(a)\ \ \textit{As syllable structure complexity increases, so does the likelihood that the largest syllable margin types in a language will be morphologically complex. }
\end{styleBody}

\begin{styleBody}
(b)\ \ \textit{As syllable structure complexity increases, so does the likelihood that syllabic consonants occurring in a language will belong to grammatical elements.}
\end{styleBody}

\begin{styleBody}
\ \ The examination of features (3.2a-d) is intended to develop a more detailed picture of the syllable patterns of the languages in the Highly Complex portion of the sample. The purpose of this portion of the analysis is twofold: to determine whether the languages form a coherent phonological type with respect to their syllable patterns, and to gather information that may be relevant to uncovering how these structures come about over time. The specific syllable patterns falling under the definition of Highly Complex will be examined for each language and compared with those of the other languages of the group to determine how similar they are to one another. In an attempt to characterize the prevalence of Highly Complex syllable patterns within each language, restrictions on the patterns will be examined for each language, and information gathered on the relative frequency of the patterns within the language. Finally, descriptions of the phonetic characteristics of Highly Complex syllable structures will be noted and compared.
\end{styleBody}

\begin{styleBody}
\ \ Apart from those which are aimed at testing the three specific hypotheses listed above, the analyses in this chapter are largely exploratory in nature. The findings here are intended to provide a baseline characterization of the syllable patterns of the four categories of the language sample. Additionally, the in-depth examination of the Highly Complex portion of the sample will serve as a point of reference in which to ground the results of later chapters.
\end{styleBody}

\section[3.2 Methodology ]{\rmfamily 3.2 Methodology }
\section[3.2.1 Patterns considered]{\rmfamily 3.2.1 Patterns considered}
\begin{styleBody}
\ \ As discussed in Chapter 1, interpretations of syllable patterns may vary dramatically according to the theoretical framework which is used. For example, even simple coda consonants and biconsonantal onsets, both crosslinguistically common patterns, are problematic in a strict CV approach, which must posit empty nuclei for phonetic occurrences of these structures (Lowenstamm 1996). Though less restrictive models of the syllable may accept small onset and coda clusters, large clusters and syllabic obstruents are typically considered problematic, with members being relegated to an appendix or left entirely unsyllabified at the level of phonological representation (Vaux \& Wolfe 2009, Bagemihl 1991). In language documentation and description, authors often choose a model which is compatible with the occurring syllable patterns of the language. However, some researchers work within a strict theoretical framework in which the abstract syllable structure has been predetermined, and the patterns of the language are made to fit into the model. This causes obvious complications for crosslinguistic comparison of syllable patterns.
\end{styleBody}

\begin{styleBody}
\ \ Because such a wide range of frameworks is used by researchers in their descriptions of syllable structure, and because the choice of model may have strong implications for the patterns reported, I opt instead to use a definition of canonical syllable structure motivated largely by the invariant phonetic patterns observed at word margins. Here onset clusters include patterns occurring word-initially before a vowel (and, if they differ at all from word-initial patterns, clusters occurring word-internally before a vowel where syllabification as an onset is supported by language-internal evidence). Likewise, coda clusters include patterns occurring word-finally after a vowel (with the same considerations given to word-internal post-vocalic patterns). In this view, all consonants are syllabified as part of an onset or coda, or, in the case of syllabic consonants, as a nucleus. Morphologically complex pre-/post-vocalic sequences, such as the word-initial biconsonantal sequence in (3.5) below, are thus considered as onset or coda patterns, in addition to morpheme-internal sequences.
\end{styleBody}

\begin{styleBody}
(3.5) \ \ \textbf{Tzeltal} (\textit{Mayan}; Mexico)
\end{styleBody}

\begin{styleBody}
/\textbf{s-t[361?][283?]’}uht/
\end{styleBody}

\begin{styleBody}
\textsc{poss.3}{}-stomach
\end{styleBody}

\begin{styleBody}
‘his stomach’
\end{styleBody}

\begin{styleBody}
(Polian 2006: 24)
\end{styleBody}

\begin{styleBody}
Pre-/post-vocalic sequences in phonological words are also considered as syllable margins, even if they belong to separate units syntactically or orthographically (3.6).
\end{styleBody}

\begin{styleBody}
(3.6) \ \ \textbf{Polish} (\textit{Indo-European}; Poland)
\end{styleBody}

\begin{styleBody}
\textit{z pstrągiem}
\end{styleBody}

\begin{styleBody}
\textit{/}\textbf{spstr}oŋ[25F?]em/
\end{styleBody}

\begin{styleBody}
‘with (the) trout’
\end{styleBody}

\begin{styleBody}
(Jassem 2003: 103)
\end{styleBody}

\begin{styleBody}
\ \ Syllable margins reported to occur only in recent loanwords or as a result of variable phonetic processes, such as vowel elision in rapid speech, were not considered in the determination of canonical patterns, or included in the present analysis (the latter issue will be treated extensively in Chapter 6, which examines vowel reduction in the sample). Here ‘canonical’ is used in the sense of syllable patterns which are regular and not reported to occur in variation with other patterns. For instance, /tk/ would not be characterized as a canonical onset in American English, because it is an optional, if for some words ubiquitous (Napoleão de Souza 2019), variant of a form which preserves the original vowel: \textit{tequila }[t[2B0?][259?]k[2B0?]il[259?]] \~{} [t[2B0?]k[2B0?]il[259?]].\footnote{\textrm{Moreover, in this case the much rarer pronunciation of [t[2B0?][259?]k[2B0?]il[259?]] is still considered an acceptable variant by speakers.}} Returning to the issue of abstract models of the syllable versus observed patterns, when syllable margins were described in one way at the phonological level but reported as consistently exhibiting a different pattern at the ‘surface’ level, the latter was taken to be representative of syllable patterns in the language. For example, sequences analyzed as clusters at an abstract level but which were invariably split by phonological epenthesis, as illustrated by the Maybrat example below, were not considered to be canonical clusters (example 3.7; see the following section for a further discussion of epenthesis).
\end{styleBody}

\begin{styleBody}
(3.7) \ \ \textbf{Maybrat }(\textit{Maybrat-Koron}; Indonesia)
\end{styleBody}

\begin{styleBody}\itshape
An epenthetic vowel invariably occurs between two consonants in word-initial position.
\end{styleBody}

\begin{styleBody}
/\textbf{pn}em/
\end{styleBody}

\begin{styleBody}
[\textbf{pan}em]
\end{styleBody}

\begin{styleBody}
‘it is flat’
\end{styleBody}

\begin{styleBody}
(Dol 2007: 35-6)
\end{styleBody}

\begin{styleBody}
\ \ Sounds involving multiple articulations or which consist of phonetic sequences, such as labialized consonants, prenasalized stops, or diphthongs, present complications for determining canonical syllable patterns, which are traditionally expressed in segmental terms. Wherever the author has presented convincing language-internal evidence for the phonological unity of such sounds, I have considered them to be single segments for the purpose of syllable structure coding. The issue of complex segments will be discussed in greater detail in Chapter 4, where I analyze the segmental inventory patterns in the language sample.
\end{styleBody}

\begin{styleBody}
\ \ As described in §2.2, similar patterns observed in languages with large tautosyllabic obstruent clusters and those with syllabic obstruents led me to include the latter in the Highly Complex category, so long as syllabic obstruents were observed to result in word-marginal sequences of three obstruents or longer in those languages. However, syllabic obstruents were not explicitly coded as part of the onset or coda. Instead, Highly Complex patterns involving long tautosyllabic obstruent clusters and those involving syllabic obstruents were coded separately so that these patterns could be disambiguated if necessary in later analyses. For languages with syllabic obstruents, the author’s description of ‘true’ tautosyllabic onset and coda patterns was taken to be canonical. Separately, the largest word-marginal consonant sequences occurring as a result of syllabic obstruents in those languages were coded as such but not considered to be maximal coda or onset structures.
\end{styleBody}

\begin{styleBody}
\ \ It is important to note that while the presence of syllabic obstruents was a partial criterion in determining whether a language has Highly Complex syllable structure, the presence of syllabic nasals and liquids (where liquid is defined as a trill, tap/flap, or lateral approximant) was never used as a diagnostic for membership in this or any category. The reasoning behind this is related to how these patterns are described and analyzed in the language references. Recall that syllabic obstruents and large tautosyllabic obstruent clusters often co-occur in languages, and that it is also common for the same language to be analyzed as having either pattern, to the exclusion of the other, by different authors. This was the case for a number of languages in the sample, including Chipaya, Nimboran, Tashlhiyt, and Yine: all of these languages have been reported in some descriptions to have syllabic obstruents, and in others to have larger obstruent clusters. By comparison, only two languages were described as having syllabic nasals or liquids as an alternative analysis to larger tautosyllabic clusters (Georgian and Yine). This is despite the fact that syllabic nasals and liquids are much more commonly reported in the sample than syllabic obstruents (see §3.3.5). Nasals and liquids do not seem to be as susceptible as obstruents to ambiguous or competing interpretations with respect to syllabicity. This is perhaps not surprising, given the perceptual properties of nasals and liquids. The reported patterns justify separate treatment of syllabic nasals and liquids on one hand, and syllabic obstruents on the other, in the coding of canonical syllable patterns in the sample.
\end{styleBody}

\section[3.2.2 Status of inserted vowels]{\rmfamily 3.2.2 Status of inserted vowels}
\begin{styleBody}
\ \ The Maybrat example in (3.8) brings up another important issue in the analysis of syllable structure, which is the status of inserted vowels. Compare the Maybrat example, reproduced in (8), with the examples from Camsá (3.9a-b).
\end{styleBody}

\begin{styleBody}
(3.8) \ \ \textbf{Maybrat} (\textit{Maybrat-Karon}; Indonesia)
\end{styleBody}

\begin{styleBody}\itshape
An epenthetic vowel invariably occurs between two consonants in word-initial position.
\end{styleBody}

\begin{styleBody}
/\textbf{pn}em/
\end{styleBody}

\begin{styleBody}
[\textbf{pan}em]
\end{styleBody}

\begin{styleBody}
‘it is flat’
\end{styleBody}

\begin{styleBody}
(Dol 2007: 35-6)
\end{styleBody}

\begin{styleBody}
(3.9)\ \ \textbf{Camsá} (isolate; Colombia)
\end{styleBody}

\begin{styleBody}
“A nonphonemic transitional voicoid [[259?]] occurs between stop plus stop […] Initial fricatives […] have optional off-glide before nonfricative consonants at a different place of articulation.”\textit{ }
\end{styleBody}

\begin{styleBody}
(a) \ \ /tkan[268?][272?]e/
\end{styleBody}

\begin{styleBody}
\ \ [t\textbf{\textsuperscript{[259?]}}kan[268?][272?]e]
\end{styleBody}

\begin{styleBody}
\ \ ‘broken’
\end{styleBody}

\begin{styleBody}
(b) \ \ /ft[361?]seŋ[261?]a/
\end{styleBody}

\begin{styleBody}
\ \ [f\textbf{\textsuperscript{u}}t[361?]seŋ[261?]a]
\end{styleBody}

\begin{styleBody}
\ \ ‘black’
\end{styleBody}

\begin{styleBody}
(Howard 1967: 81)
\end{styleBody}

\begin{styleBody}
\ \ There are several important differences between the descriptions of the Maybrat and Camsá examples. In the Maybrat example the vowel is described as epenthetic, while in the Camsá example it is described as a nonphonemic transitional vocoid or an off-glide. In Maybrat, the process is described as invariable, while in Camsá it is described, in the case of the off-glide, as optional. Finally, in Maybrat, the epenthesized vowel is transcribed as [a], while in Camsá it is transcribed as superscripted [\textbf{\textsuperscript{[259?]}}] in one case and [\textbf{\textsuperscript{u}}] in another. These differences in the descriptions and transcription conventions suggest different phenomena. This raises the question of whether the patterns in Maybrat and Camsá should be treated separately for the purposes of syllable structure analysis.
\end{styleBody}

\begin{styleBody}
\ \ In a typological survey of reported inserted vowel patterns, Hall (2006) makes a distinction between epenthetic vowels and intrusive vowels. She argues that the motivations for the two types of inserted vowels are quite different. The function of epenthetic vowels is to provide a nucleus to repair marked or non-occurring syllable structures in a language; that is, it has the effect of producing syllable patterns that are already attested in the language. A textbook example of this kind of process is the adaptation of loanwords from one language into another; e.g. English \textit{technostress} is borrowed into Japanese as \textit{tekunosutoresu }because the consonant clusters and codas in the original form are not part of the sound pattern of the borrowing language (Kay 1995: 69). Hall states that epenthetic vowels tend to be ‘visible’ to phonological processes, behaving like syllable nuclei for the purposes of stress assignment and other processes, and speakers are generally aware of their presence. For example, in Mono, epenthetic vowels occur to ‘repair’ monomoraic lexical words: /[292?]\={i}/ {\textgreater} [\={i}[292?]\={i}] ‘tooth’ (Hall 2006: 6, citing Olson 2003).
\end{styleBody}

\begin{styleBody}
\ \ By contrast, intrusive vowels are not structurally motivated but come about through natural processes of gestural retiming and overlap, and are simply an acoustic effect of these transitions. For example, in Kekchi, an intrusive vowel identical in quality to the preceding vowel may appear within final clusters consisting of a glottal stop followed by a consonant: /po[294?]t/ {\textgreater} [po[294?]ot] ‘blouse’ (Hall 2006: 7, citing Campbell 1974). This is analyzed not as the addition of a new vowel articulation, but as the offset of the vowel gesture carrying over the duration of the glottal stop articulation, which does not require an oral articulation, such that the effects of the vowel can still be heard between the glottal stop and the onset of the following consonant articulation. Hall states that intrusive vowels tend to be ‘invisible’ to such phonological processes, and speakers are typically unaware of their presence, or if they are aware, view them as optional.
\end{styleBody}

\begin{styleBody}
\ \ The properties described above — behavior with respect to phonological processes and speaker awareness — are for Hall the primary means of determining whether an inserted vowel is epenthetic or intrusive. She argues persuasively for the use of speaker intuition in making these determinations, giving several examples of cases in which the phonological behavior of a vocalic element matches the native speaker’s, rather than the fieldworker’s, intuitions about syllabicity. This is supported by growing evidence in the experimental literature that listeners are biased by the timing patterns of their own language when identifying the syllable patterns of another language (Kwon et al. 2017). However, while information regarding native speaker intuition is valuable, it is often unreported. Based on the typological patterns represented in her survey, Hall develops a set of additional features which tend to be associated with epenthesis and intrusion. Epenthetic vowels tend to have the following characteristics: (i) the vowel quality is fixed or copied from a neighboring vowel; (ii) the vowel occurs regardless of speech rate; and (iii) the vowel repairs a structure that is likely to be avoided through other processes in the same language. Intrusive vowels tend to have these characteristics: (i) the vowel quality is neutral, or influenced by the place of articulation of the surrounding consonants; (ii) the vowel is typically found in heterorganic clusters; (iii) the vowel is often optional, has variable duration and voicing, and may disappear as speech rate increases; and (iv) the vowel does not seem to have a repairing function (Hall 2006: 391).\footnote{\textrm{As noted by Browman \& Goldstein (1992a:53), epenthetic vowels may have their origin in the intrusive elements arising from gestural organization. Hall gives several examples of historical cases of vowel epenthesis that may have started with vowel intrusion and then phonologized; e.g., Irish Gaelic }\textrm{\textit{[261?]orm}}\textrm{ {\textgreater} }\textrm{\textit{[261?]or[259?]m}}\textrm{ ‘blue’ (2006: 35).}}
\end{styleBody}

\begin{styleBody}
\ \ Returning to (3.8)-(3.9) above, we see that the Maybrat example falls under Hall’s definition of epenthetic vowels, while the Camsá example falls under her definition of intrusive vowels. Further descriptions by the authors support this view. Dol presents instrumental evidence showing that epenthetic [a] is as prominent as other vowels in the language (2007: 36). Meanwhile, examples given by Howard indicate that the transitional vocoid is not counted for the purposes of stress assignment. This suggests that the Maybrat example [panem] is best analyzed as a CV.CVC structure, while the Camsá examples [t\textbf{\textsuperscript{[259?]}}kan[268?][272?]e] and [f\textbf{\textsuperscript{u}}t[361?]seŋ[261?]a] are best analyzed as having initial onset clusters.
\end{styleBody}

\begin{styleBody}
\ \ The matter of vowel intrusion is relevant to the issue of syllabic consonants as well. Recall the Cocopa example discussed in §2.2, reproduced below as (3.10).
\end{styleBody}

\begin{styleBody}
(3.10)\ \ \textbf{Cocopa} (\textit{Cochimi-Yuman}; USA and Mexico)
\end{styleBody}

\begin{styleBody}
/pt[361?][283?]xmuk\'{a}p/
\end{styleBody}

\begin{styleBody}
[p\textsuperscript{i}.t[361?][283?]x\textsuperscript{a}.mu.k\'{a}p]
\end{styleBody}

\begin{styleBody}
‘he embraced her’
\end{styleBody}

\begin{styleBody}
(Crawford 1966: 43)
\end{styleBody}

\begin{styleBody}
Crawford proposes that some unstressed syllables in Cocopa may be entirely consonantal, consisting only of an onset or onset and coda, but with a predictable ‘murmur’ vowel functioning as a phonetic peak in such cases (1966: 34). What is described as the murmur vowel here has properties of an intrusive vowel: its quality is determined by that of surrounding consonants, and it is transcribed as a superscript, suggesting brevity or an offglide status.
\end{styleBody}

\begin{styleBody}
\ \ Similar descriptions of other languages in the sample suggest an association between intrusive or transitional elements and consonants analyzed as syllabic (e.g., in certain environments in Tashlhiyt; Dell \& Elmedlaoui 2002). Hargus \& Beavert (2006) list several languages in which arguments for consonant (specifically obstruent) syllabicity are rooted in the distribution of ‘epenthetic’ vowels. Additionally, Bell (1978a: 185-6) reports that speakers are unaware of the short vocalic elements associated with syllabic obstruents in Koryak, similar to reports of speaker intuition of intrusive vowels in consonant clusters. Hall notes crosslinguistic associations between vowel intrusion, aspiration (also analyzable as voiceless vowel intrusion), and syllabic consonants, and suggests that all of these phenomena are different acoustic manifestations of fundamentally similar processes of gestural overlap (2006: 413). This issue will be revisited in §3.4.3, when I discuss the phonetic properties of languages in the Highly Complex portion of the sample.
\end{styleBody}

\begin{styleBody}
\ \ In coding for syllable structure patterns in the sample, wherever possible I used the diagnostics proposed in Hall (2006) to determine the status of reported inserted vowels.
\end{styleBody}

\section[3.2.3 Edges of categories]{\rmfamily 3.2.3 Edges of categories}
\begin{styleBody}
\ \ After syllable patterns in the sample were determined with reference to the above criteria, these patterns were recorded and coded for analysis. First, the syllable patterns were used to classify the languages into the four categories of syllable structure complexity as defined in Maddieson (2013a) and §2.2, and reproduced here:
\end{styleBody}

\begin{styleBody}
\textbf{Simple:} languages in which the onset is maximally one C, and codas do not occur.
\end{styleBody}

\begin{styleBody}
\textbf{Moderately Complex:} languages in which the onset is maximally two Cs, the second of which is a liquid or a glide; and/or the coda consists of maximally one C.
\end{styleBody}

\begin{styleBody}
\textbf{Complex:} languages in which the maximal onset is two Cs, the second of which is a C other than a liquid or a glide, or three Cs, so long as all three are not obstruents; and/or the maximal coda consists of two Cs, or three Cs so long as all three are not obstruents.
\end{styleBody}

\begin{styleBody}
\textbf{Highly Complex}: languages in which the maximal onset or coda consists of three obstruents, or four or more Cs of any kind; and/or in which syllabic obstruents occur, resulting in word-marginal sequences of three or more obstruents.
\end{styleBody}

\begin{styleBody}
\ \ A common complication for typological work is that languages sometimes do not fall neatly into the categories defined by the researcher. Like languages themselves, linguistic features are dynamic and constantly changing form, however slowly or subtly. There is often some ambiguity at play when categorizing languages according to some structural feature, because a language may exhibit behavior characteristic of several categories. In this study I took advantage of such ambiguity in constructing the language sample in order to increase genealogical diversity or the representation of syllable structure complexity categories in some regions.
\end{styleBody}

\begin{styleBody}
\ \ The Simple syllable structure pattern is crosslinguistically rare, so in some cases languages were admitted to the category despite having minor exceptions to canonical (C)V patterns. For example, several languages were categorized as having Simple syllable structure despite being reported to have codas in a small handful of lexical items (e.g., five words in Cavineña, Guillaume 2008: 31; two in Saaroa, Pan 2012: 32f). Two of the North American languages included in this category, Towa and Ute, can be argued not to have Simple syllable structure in the strictest definition of the term. In Towa, there are no complex onsets. Simple codas /[283?]/ and /l/ occur in the language, but are rarely produced as such by speakers. For instance, word-final /[283?]/, which always corresponds to Inverse suffix /-[283?]/, is omitted from a phrase-medial noun stem unless it is followed by a vowel-initial pronominal prefix, in which case it is resyllabified as an onset (Yumitani 1998: 22-24). Coda /[283?]/ may occur utterance-finally, but because the language is predominantly verb-final and this suffix attaches to nouns, the frequency of this pattern is extremely rare in natural discourse (Logan Sutton, p.c.).\footnote{In a 244-word text included in Yumitani (1998), there are no examples of phonetic codas in Towa.} In Ute, syllable structure is almost entirely of the shape CV(V), but a recent process devoicing unstressed vowels in the language has resulted in some invariant codas and patterns that could be analyzed as sequences of oral consonants and glottal fricatives (Givón 2011: 20-3, 28). It would appear that Ute has until recently had Simple syllable structure, but is in the process of rapidly developing more complex syllable patterns. Though neither Towa nor Ute present uncontroversial cases of languages with Simple syllable structure, I judge their patterns to be close enough to justify their inclusion as such, thereby increasing the representation of North American languages in this category. 
\end{styleBody}

\begin{styleBody}
\ \ On similar grounds, Eastern Khanty was admitted to the Moderately Complex category. The syllable patterns of this language include occasional coda clusters which are always a result of derivation or inflection in the language. Typically when this happens, vowel epenthesis is employed “robustly and productively” such that most of these sequences are not realized as clusters (Filchenko 2007: 55). However, derived coda clusters with a sonorant preceding a homorganic stop (e.g., \textit{lol-t} ‘crack, dent-\textsc{pl’}) are sometimes retained as such. The author states that the probability of such consonant clusters occurring is extremely low. As a complex coda, this pattern falls under the definition of Complex syllable structure, but since its status in the language is extremely marginal, I place this language into the Moderately Complex category.
\end{styleBody}

\begin{styleBody}
\ \ Yine was admitted to the Highly Complex category despite being a somewhat ambiguous case. Two major descriptions of Yine (Hanson 2010, Matteson 1965) describe the occurrence of biconsonantal and triconsonantal onset clusters in the language. Both describe onset clusters as being relatively unrestricted: “[c]onsonant clusters in Yine show enough range in attested combinations, both word-initially and word-internally, to suggest that there are no sonority-based restrictions imposed on them” (Hanson 2010: 27). Examples of biconsonantal onsets are plentiful and include combinations as varied as /t[361?][283?]k/, /sp/, and /ns/. Matteson (1965: 24) gives few examples of triconsonantal onsets and states that the very low frequency of these shapes had decreased in comparison to a count made a decade previously. However, Hanson, writing 45 years later, writes that “words beginning with three consonants in a sequence are very common” (2010: 27). She describes these clusters as resulting from the affixation of a Class 2 pronominal prefix (/n-/, /p-/, /t-/, /w-/, /h-/) to a stem beginning with a biconsonantal cluster; e.g. /\textbf{p-kn}oya-te/ ‘your tortoise’ (2010: 26). Other examples given include /pc[27E?]/, /nmt[361?][283?]/, and /nt[361?][283?]k/, but there are no explicit examples of three-obstruent clusters listed. Whether this is due to non-occurrence is unclear, and the conflicting descriptions of the frequency of triconsonantal onsets does not help shed light on the issue. Nevertheless, I include this language in the Highly Complex category to increase representation of the category in the South American portion of the sample, acknowledging that it may be a very marginal case.
\end{styleBody}

\section[3.2.4 Coding]{\rmfamily 3.2.4 Coding}
\begin{styleBody}
\ \ The analyses described in §3.2.1-3.2.3 were conducted as part of the process of constructing the language sample (Chapter 2). After the sample was constructed, more specific information on syllable structure was collected for each language, and coded as described below:
\end{styleBody}

\begin{styleBody}
\textbf{Size of maximal onset:} in number of Cs
\end{styleBody}

\begin{styleBody}
\textbf{Size of maximal coda:} in number of Cs
\end{styleBody}

\begin{styleBody}
\textbf{Onset obligatory:} Yes, No
\end{styleBody}

\begin{styleBody}
\textbf{Coda obligatory:} Yes, No
\end{styleBody}

\begin{styleBody}
\textbf{Vocalic nucleus patterns:} Short vowels, Long vowels, Diphthongs, Vowel sequences
\end{styleBody}

\begin{styleBody}
\textbf{Syllabic consonant patterns:} Nasal, Liquid, or Obstruent
\end{styleBody}

\begin{styleBody}
\textbf{Size of maximal word-marginal sequences with syllabic obstruents:} in number of Cs
\end{styleBody}

\begin{styleBody}
\textbf{Predictability of syllabic consonants:} Phonemic, Predictable from word/consonantal context, or Varies with CV sequence
\end{styleBody}

\begin{styleBody}
\textbf{Morphological constituency of maximal syllable margin:} Morpheme-internal, Morphologically Complex, or Both patterns occur
\end{styleBody}

\begin{styleBody}
\textbf{Morphological pattern of syllabic consonants:} Lexical items, Grammatical items, Both
\end{styleBody}

\begin{styleBody}
\ \ Additionally, more detailed information on syllable patterns, restrictions on cluster patterns, syllable type frequency data, and the phonetic characteristics of clusters was gathered for the Highly Complex portion of the sample.
\end{styleBody}

\begin{styleBody}
\ \ An example of the syllable structure coding for Ket, a language with Complex syllable structure, can be found below (3.11).
\end{styleBody}

\begin{styleBody}
(3.11)\ \ \textbf{Ket} (\textit{Yeniseian}; Russia)
\end{styleBody}

\begin{styleBody}
\textbf{Size of maximal onset: }2
\end{styleBody}

\begin{styleBody}
\textbf{Size of maximal coda:} 3
\end{styleBody}

\begin{styleBody}
\textbf{Onset obligatory:} No
\end{styleBody}

\begin{styleBody}
\textbf{Coda obligatory:} No
\end{styleBody}

\begin{styleBody}
\textbf{Vocalic nucleus patterns:} Short vowels
\end{styleBody}

\begin{styleBody}
\textbf{Syllabic consonant patterns:} Nasals
\end{styleBody}

\begin{styleBody}
\textbf{Size of maximal word-marginal sequences with syllabic obstruents:} N/A
\end{styleBody}

\begin{styleBody}
\textbf{Predictability of syllabic consonants:} Predictable from word/consonantal context
\end{styleBody}

\begin{styleBody}
\textbf{Morphological constituency of maximal syllable margin:} Morphologically Complex (Onsets), Both patterns (Codas)
\end{styleBody}

\begin{styleBody}
\textbf{Morphological pattern of syllabic consonants:} Grammatical items
\end{styleBody}

\begin{styleBody}
\ \ All syllable structure coding for the language sample, including illustrative examples and notes on specific onset and coda patterns, can be found in Appendix B.
\end{styleBody}

\section[3.3 Results for language sample]{\rmfamily 3.3 Results for language sample}
\begin{styleBody}
\ \ In this section I present analyses of general patterns of maximal onsets, codas, and syllabic consonants in the language sample. In §3.3.1 I present the maximal onset and coda sizes in the data, and discuss maximal word-marginal patterns in the languages which have syllabic obstruents. In §3.3.2, I briefly examine the relationship between onset and coda complexity. In §3.3.3 the relationship between syllable structure complexity and obligatoriness of syllable margins is investigated. In §3.3.4 and §3.3.5 the patterns of vocalic nuclei and syllabic consonants, respectively, are presented and analyzed with respect to syllable structure complexity. §3.3.6 is a longer subsection which addresses the issue of morphological constituency in maximal syllable margins and syllabic consonants in the data.
\end{styleBody}

\section[3.3.1 Maximal onset and coda sizes]{3.3.1 Maximal onset and coda sizes}
\begin{styleBody}
\ \ The distribution of maximal onset and coda patterns in the sample, by number of consonants, can be found in Figures 3.1-3.2.
\end{styleBody}

\begin{styleBody}
  [Warning: Image ignored] % Unhandled or unsupported graphics:
%\includegraphics[width=3.5154in,height=2.1457in,width=\textwidth]{./ObjectReplacements/Object 3}
 
\end{styleBody}

\begin{styleBody}
\textbf{Figure 3.1.} Maximal onset sizes in sample.
\end{styleBody}

\begin{styleBody}
  [Warning: Image ignored] % Unhandled or unsupported graphics:
%\includegraphics[width=2.8236in,height=2.1457in,width=\textwidth]{./ObjectReplacements/Object 5}
 
\end{styleBody}

\begin{styleBody}
\textbf{Figure 3.2.} Maximal coda sizes in sample.
\end{styleBody}

\begin{styleBody}
\ \ The data presented in Figures 3.1 and 3.2 are for onset and coda patterns determined through the procedure described in §3.2.1. They do not include the largest word-marginal patterns which occur in languages with syllabic obstruents. There are four languages in the sample which are reported to have syllabic obstruents resulting in Highly Complex patterns at word margins.\footnote{Additionally, Tohono O’odham has syllabic obstruents, but only in independent grammatical particles consisting of a single consonant (determiners and conjunctives). These are not reported to be phonologically bound to adjacent words. Therefore I do not include Tohono O’odham in Table 3.1.} These patterns are presented in Table 3.1, along with the reported maximal onset and coda patterns in the languages.
\end{styleBody}

\begin{flushleft}
\tablefirsthead{}
\tablehead{}
\tabletail{}
\tablelasttail{}
\begin{supertabular}{m{1.1858599in}m{0.7143598in}m{0.7150598in}m{1.4962599in}m{1.4969599in}}
\hline
{\bfseries Language} &
\centering{\bfseries Maximal onset} &
\centering{\bfseries Maximal coda} &
{\centering\bfseries Maximal \par}

{\centering\bfseries word-initial \par}

\centering{\bfseries obstruent string} &
{\centering\bfseries Maximal \par}

{\centering\bfseries word-final \par}

\centering\arraybslash{\bfseries obstruent string}\\\hline
Cocopa &
\centering 4 &
\centering 3 &
\centering 5 &
\centering\arraybslash 3\\
Semai &
\centering 2 &
\centering 1 &
\centering 4 &
\centering\arraybslash 1\\
Tashlhiyt &
\centering 1 &
\centering 1 &
\centering{\itshape (words without vowels)} &
\centering\arraybslash{\itshape (word without vowels)}\\
Tehuelche &
\centering 2 &
\centering 3 &
\centering 3 &
\centering\arraybslash {\textgreater}3\\\hline
\end{supertabular}
\end{flushleft}
\begin{styleBody}
\textbf{Table 3.1.} Languages in Highly Complex category with syllabic obstruents. Maximal reported onsets and codas are given in first two columns. The sizes of the maximal word-marginal obstruent strings which occur as the result of syllabic obstruents are given in last two columns.
\end{styleBody}

\begin{styleBody}
\ \ In Tashlhiyt, the size of maximal word-marginal obstruent strings cannot be determined, because there are many examples of words consisting entirely of obstruents in this language (3.12). 
\end{styleBody}

\begin{styleBody}
(3.12)\ \ \textbf{Tashlhiyt} (\textit{Afro-Asiatic}; Morocco)
\end{styleBody}

\begin{styleBody}
\textit{tftktst[2D0?]}
\end{styleBody}

\begin{styleBody}
tf.tk.tst[2D0?]
\end{styleBody}

\begin{styleBody}
‘you took it off (\textsc{f})’
\end{styleBody}

\begin{styleBody}
(Ridouane 2008: 332)
\end{styleBody}

\begin{styleBody}
In Tehuelche, the reference indicates that sequences of up to six consonants may occur word-finally, but the only illustrative example of a pattern this size is given is (3.13).
\end{styleBody}

\begin{styleBody}
(3.13) \ \ \textbf{Tehuelche} (\textit{Chonan}; Argentina)
\end{styleBody}

\begin{styleBody}
\textit{kt[361?][283?]a[294?][283?]p[283?]k’n}
\end{styleBody}

\begin{styleBody}
k.t[361?][283?]a[294?][283?]p.[283?].k’n
\end{styleBody}

\begin{styleBody}
‘it is being washed’
\end{styleBody}

\begin{styleBody}
(Fernández Garay \& Hernández 2006: 13)
\end{styleBody}

\begin{styleBody}
This example includes a nasal which may be syllabic (syllable peaks in CC syllables are not marked by the authors). It is clear from the language description that long obstruent sequences come about when syllabic consonants are strung together, but unclear as to what the upper limit on the size of these is. Word-final sequences of at least four obstruents are attested (3.14).
\end{styleBody}

\begin{styleBody}
(3.14) \ \ \textbf{Tehuelche} (\textit{Chonan}; Argentina)
\end{styleBody}

\begin{styleBody}
\textit{ma[2D0?]le[283?]p[283?]k’}
\end{styleBody}

\begin{styleBody}
ma[2D0?].le[283?]p.[283?].k’
\end{styleBody}

\begin{styleBody}
‘they steal’
\end{styleBody}

\begin{styleBody}
(Fernández Garay \& Hernández 2006: 63)
\end{styleBody}

\section[3.3.2 Relationship between onset and coda complexity ]{\rmfamily 3.3.2 Relationship between onset and coda complexity }
\begin{styleBody}
\ \ Here I present an analysis similar to the one presented in §2.2 for the Complex portion of the Maddieson (2013a) sample. The languages of the sample used in this book are distributed according to their maximal onset and coda patterns.
\end{styleBody}

\begin{flushleft}
\tablefirsthead{}
\tablehead{}
\tabletail{}
\tablelasttail{}
\begin{supertabular}{m{0.6525598in}|m{0.5615598in}|m{0.5622598in}|m{0.5622598in}|m{0.5615598in}|m{0.56295985in}|m{0.5615598in}|m{0.56295985in}|m{0.5615598in}|m{0.5601598in}|}
\hline
\multicolumn{1}{|m{0.6525598in}|}{} &
 &
\multicolumn{8}{m{5.0464597in}|}{\centering{\bfseries Number of Cs in onset}}\\\hline
\multicolumn{1}{|m{0.6525598in}|}{\centering{\bfseries Number of Cs in coda}} &
 &
\centering{\bfseries One} &
\centering{\bfseries Two} &
\centering{\bfseries Three} &
\centering{\bfseries Four} &
\centering{\bfseries Five} &
\centering{\bfseries Six} &
\centering{\bfseries Seven} &
\centering\arraybslash{\bfseries Eight}\\\hline
 &
\centering{\bfseries None} &
\centering 20 &
\centering 1 &
\centering 2 &
\centering 1 &
\centering — &
\centering — &
\centering — &
\centering\arraybslash —\\\hhline{~---------}
 &
\centering{\bfseries One} &
\centering 21 &
\centering 12 &
\centering 5 &
\centering 1 &
\centering — &
\centering — &
\centering — &
\centering\arraybslash —\\\hhline{~---------}
 &
\centering{\bfseries Two} &
\centering 7 &
\centering 5 &
\centering 5 &
\centering — &
\centering — &
\centering — &
\centering — &
\centering\arraybslash —\\\hhline{~---------}
 &
\centering{\bfseries Three} &
\centering 1 &
\centering 4 &
\centering 2 &
\centering 4 &
\centering — &
\centering — &
\centering — &
\centering\arraybslash —\\\hhline{~---------}
 &
\centering{\bfseries Four} &
\centering 3 &
\centering — &
\centering — &
\centering 2 &
\centering — &
\centering — &
\centering — &
\centering\arraybslash —\\\hhline{~---------}
 &
\centering{\bfseries Five} &
\centering — &
\centering — &
\centering — &
\centering — &
\centering 1 &
\centering — &
\centering 1 &
\centering\arraybslash 1\\\hhline{~---------}
 &
\centering{\bfseries Six} &
\centering — &
\centering — &
\centering 1 &
\centering — &
\centering — &
\centering — &
\centering — &
\centering\arraybslash —\\\hhline{~---------}
\end{supertabular}
\end{flushleft}
\begin{styleBody}
\textbf{Table 3.2.} Languages of sample distributed according to maximal onset and coda size.
\end{styleBody}

\begin{styleBody}
\ \ Interestingly, it is common for languages with large clusters at one syllable margin to also exhibit large clusters at the other syllable margin in their canonical patterns. Roughly half of languages in the sample which have a maximal cluster of four or more consonants at one syllable margin will have a similarly large maximal cluster at the other syllable margin. It is striking that \textit{all} languages in the sample with maximal onsets of five or more consonants (Georgian, Itelmen, and Polish) also have maximal codas of five consonants. Meanwhile, the bottom left and top right corners of Table 3.2 are sparsely populated; that is, there are relatively few languages with very large maximal clusters at one syllable margin and very small maximal clusters (or none at all) at the other margin. A similar pattern can be observed in Table 2.2 in §2.2, which used a larger sample of 147 languages. Speaking from a strictly distributional point of view, there is no obvious motivation for this pattern. If we consider onset and coda structures to be independent structures, then we would expect to see the full range of possible variation in their combination crosslinguistically. This point will be revisited in §3.5 and again in Chapter 8.
\end{styleBody}

\section[3.3.3 Syllable structure complexity and obligatoriness of syllable margins]{\rmfamily 3.3.3 Syllable structure complexity and obligatoriness of syllable margins}
\begin{styleBody}
\ \ Blevins notes another crosslinguistic pattern linking onset and coda structures: she observes that languages with only open syllables tend to have optional onsets (2006: 336). In Table 3.3 I examine this relationship in the current language sample. There are some languages in which optional onsets may be reported for the canonical syllable structure, but regular and obligatory consonant epenthesis (usually of a glottal stop or fricative) occurs to produce onsets in all ‘surface’ forms. In the analysis below, I consider such languages as having obligatory onsets.
\end{styleBody}

\begin{flushleft}
\tablefirsthead{}
\tablehead{}
\tabletail{}
\tablelasttail{}
\begin{supertabular}{m{0.84625983in}m{0.84625983in}m{0.84625983in}}
\hline
\centering \textit{N} languages &
\centering Codas occur? &
\centering\arraybslash Onset obligatory?\\\hline
\centering 14 &
\centering Y &
\centering\arraybslash Y\\
\centering 62 &
\centering Y &
\centering\arraybslash N\\
\centering 2 &
\centering N &
\centering\arraybslash Y\\
\centering 22 &
\centering N &
\centering\arraybslash N\\\hline
\end{supertabular}
\end{flushleft}
\begin{styleBody}
\textbf{Table 3.3.} Languages in sample distributed according to occurrence of codas and obligatoriness of onsets.
\end{styleBody}

\begin{styleBody}
\ \ The relationship reported by Blevins is upheld in the current sample. Of the 24 languages with only open syllables (no codas), only two are reported to have obligatory onsets. Obligatory onsets are more common in languages with coda structures (14/76). The two languages in the sample with obligatory onsets and only open syllables are Hadza (Simple) and Yine (Highly Complex). It should also be noted that the Oykangand dialect of Kunjen used here is reported to have obligatory codas. Kunjen is argued to have very marginal onset patterns, with onsets occurring in interjections and sentence-initially in a just a few lexical items (Sommer 1969, 1970, 1981, though see Dixon 1970 for an opposing view).\footnote{According to Sommer’s analysis, Kunjen is a rare example of a language without phonological CV syllables. Another language argued not to have phonological CV syllables is Arrernte, a Pama-Nyungan language of central Australia (cf. Breen \& Pensalfini 1999), though Anderson (2000) reports a canonical surface syllable structure of (C)(C)V(C) for Western Arrernte.} Therefore Kunjen shows a very similar pattern to Hadza and Yine, except that the syllable margins are reversed.
\end{styleBody}

\begin{styleBody}
\ \ The analysis above motivated a more general examination of obligatory syllable margin patterns in the language sample with respect to syllable structure complexity (Table 3.4).
\end{styleBody}

\begin{flushleft}
\tablefirsthead{}
\tablehead{}
\tabletail{}
\tablelasttail{}
\begin{supertabular}{|m{1.2205598in}|m{1.2205598in}|m{1.2205598in}|m{1.2205598in}|m{1.2205598in}|}
\hline
 &
\multicolumn{4}{m{5.1184597in}|}{\centering{\bfseries Syllable Structure Complexity}}\\\hline
 &
{\centering\bfseries Simple\par}

\centering (\textit{N} = 24 lgs) &
{\centering\bfseries Moderately Complex\par}

\centering (\textit{N} = 26 lgs) &
{\centering\bfseries Complex\par}

\centering (\textit{N} = 25 lgs) &
{\centering\bfseries Highly \par}

{\centering\bfseries Complex\par}

\centering\arraybslash (\textit{N} = 25 lgs)\\\hline
{\bfseries Onset obligatory} &
{\centering\itshape Hadza\par}

{\centering\itshape Ute\par}

\centering{\bfseries (2)} &
{\centering\itshape Kambaata\par}

{\centering\itshape Karok\par}

{\centering\itshape Lao\par}

{\centering\itshape Pacoh\par}

\centering{\bfseries (4)} &
{\centering\itshape Koho\par}

{\centering\itshape Lepcha\par}

{\centering\itshape Mangarrayi\par}

\centering{\bfseries (3)} &
{\centering\itshape Bench\par}

{\centering\itshape Nuu-chah-nulth\par}

{\centering\itshape Semai\par}

{\centering\itshape Thompson\par}

{\centering\itshape Tohono O’odham\par}

{\centering\itshape Yakima Sahaptin\par}

{\centering\itshape Yine\par}

\centering\arraybslash{\bfseries (7)}\\\hline
{\bfseries Coda obligatory} &
\centering — &
\centering —  &
\centering —  &
{\centering\itshape Kunjen\par}

\centering\arraybslash{\bfseries (1)}\\\hline
\end{supertabular}
\end{flushleft}
\begin{styleBody}
\textbf{Table 3.4. }Languages in sample with obligatory syllable margins.
\end{styleBody}

\begin{styleBody}
\ \ Obligatory syllable margins are a minor pattern in the language sample, occurring in only 17 languages, but this feature is most common in languages with Highly Complex syllable structure, occurring in roughly one-third (8/25) of those languages. This feature is least common in languages with Simple syllable structure (2/24 languages). The pattern in the Highly Complex category is statistically significant when compared against the patterns in the other three categories combined (p = .03 in Fisher’s exact test). This association between syllable structure complexity and obligatoriness of syllable margins has, to my knowledge, not been previously reported.
\end{styleBody}

\begin{styleBody}
\ \ Obligatory syllable margins are more common in some areas than others: Five of the languages in Table 3.4 are from Southeast Asia \& Oceania. North America is also heavily represented, accounting for another six languages altogether, including four of the languages with obligatory syllable margins from the Highly Complex group. It should further be noted that most of the North American languages with obligatory syllable margins in the Highly Complex group are from the Pacific Northwest (Nuu-chah-nulth, Thompson, and Yakima Sahaptin), so areal factors may be at play. Nevertheless, the Highly Complex pattern in Table 3.4 is not entirely areal in nature, as it includes languages from Africa, Southeast Asia \& Oceania, South America, and Australia \& New Guinea. 
\end{styleBody}

\begin{styleBody}
\ \ Interestingly, the description of Itelmen (Chukotko-Kamchatkan, Highly Complex) suggests that this language, too, had obligatory onsets at some point in its history: morphophonological processes suggest that present-day vowel-initial syllables were at one time initiated by a glottal stop (Georg \& Volodin 1999: 48).
\end{styleBody}

\section[3.3.4 Vocalic nucleus patterns]{\rmfamily 3.3.4 Vocalic nucleus patterns}
\begin{styleBody}
\ \ Vocalic nucleus patterns have until now been excluded from the discussion of syllable patterns, as they are not considered in the definitions of syllable structure complexity used here. However, it is important to note that vocalic nucleus patterns can also exhibit different degrees of complexity. In Table 3.5 I present a very general analysis of these patterns in the sample, showing the distribution of simple and complex vocalic nuclei by syllable structure complexity.
\end{styleBody}

\begin{flushleft}
\tablefirsthead{}
\tablehead{}
\tabletail{}
\tablelasttail{}
\begin{supertabular}{|m{1.7427598in}|m{1.0906599in}|m{1.0906599in}|m{1.0906599in}|m{1.0913599in}|}
\hline
 &
\multicolumn{4}{m{4.59956in}|}{\centering{\bfseries Syllable Structure Complexity}}\\\hline
\raggedleft{\bfseries Languages with:} &
{\centering\bfseries Simple\par}

\centering (\textit{N} = 24 lgs) &
{\centering\bfseries Moderately Complex\par}

\centering (\textit{N} = 26 lgs) &
{\centering\bfseries Complex\par}

\centering (\textit{N} = 25 lgs) &
{\centering\bfseries Highly \par}

{\centering\bfseries Complex\par}

\centering\arraybslash (\textit{N} = 25 lgs)\\\hline
\raggedleft Simple vocalic nuclei only &
\centering 12 &
\centering 9 &
\centering 9 &
\centering\arraybslash 9\\\hline
{\raggedleft Complex vocalic nuclei \par}

\raggedleft{\itshape (long vowels, diphthongs, and/or vowel sequences)} &
\centering 12 &
\centering 17 &
\centering 16 &
\centering\arraybslash 16\\\hline
\end{supertabular}
\end{flushleft}
\begin{styleBody}
\textbf{Table 3.5. }Vocalic nucleus patterns in language sample, by syllable structure complexity.
\end{styleBody}

\begin{styleBody}
\ \ Simple vocalic nuclei — those consisting of a single short vowel — occur in every language. The first row of Table 3.5 shows the number of languages in each complexity category for which this is the only vocalic nucleus pattern occurring. Languages in which complex vocalic nuclei occur in addition to simple vocalic nuclei are shown in the second row. For the sake of simplicity I have collapsed three different kind of complex vocalic nucleus patterns in the analysis here. A language is counted as having long vowels if it has contrastive vowel length, but not if it has predictable vowel lengthening, e.g., a longer variant preceding a voiced coda. Diphthongs and tautosyllabic vowel sequences are difficult to disambiguate from one another, as their analyses by different authors may vary widely; however, vowel sequences reported here as syllable nuclei are those explicitly shown by the author to belong to one syllable, much like a diphthong. That is, this figure does not include cases of hiatus, in which the two vowels in a sequence belong to different syllables.
\end{styleBody}

\begin{styleBody}
\ \ Table 3.5 shows that complex vocalic nuclei are much less likely to occur in languages with Simple syllable structure than in languages from the other categories. This suggests that the potential for more syllable types in languages with more complex syllable structure may be not only a function of larger canonical syllable margins, but also of greater diversity in syllable nucleus patterns. Nevertheless, the analysis above is too coarse to draw strong conclusions about vocalic nucleus patterns and syllable structure complexity. The issue of contrastive vowel length will be treated in greater detail in Chapter 4, along with contrastive nasalization, voicing, and glottalization patterns in the vowel inventories of the sample.
\end{styleBody}

\section[3.3.5 Syllabic consonants]{\rmfamily 3.3.5 Syllabic consonants}
\begin{styleBody}
\ \ In this section I investigate patterns of syllabic consonants in the data. Recall that the previous literature suggests two competing predictions for the relationship between syllable complexity and the presence of syllabic consonants. Isačenko’s (1939/1940) phonological typology predicts that ‘vocalic’ languages, which tend to have simpler syllable structure, will be more likely to develop syllabic consonants, and specifically syllabic sonorants. Meanwhile, Bell (1978a) notes that syllabic consonants, including syllabic obstruents, often come about through vowel reduction processes, which are also known to produce the clusters characteristic of languages with more complex syllable structure. On the basis of the latter observation, in §3.1.2 I formulated a hypothesis that languages with more complex syllable structure are more likely to have syllabic consonant patterns.
\end{styleBody}

\begin{styleBody}
\ \ Here I analyze languages in which the syllabic consonants are reported as invariant patterns. Most often, the syllabicity of these consonants is predictable from the surrounding consonantal and/or word environment, as illustrated by (3.15). Less frequently, syllabic consonants are analyzed as separate phonemes which are contrastive with their non-syllabic counterparts (3.16a-b).
\end{styleBody}

\begin{styleBody}
(3.15)\ \ \textbf{Itelmen} (\textit{Chukotko-Kamchatkan}; Russia)
\end{styleBody}

\begin{styleBody}\itshape
A word-initial alveolar or bilabial nasal stop preceding another consonant is realized as syllabic.
\end{styleBody}

\begin{styleBody}
/\textbf{m}[26C?]im/
\end{styleBody}

\begin{styleBody}
[\textbf{m[329?]}[26C?]im]
\end{styleBody}

\begin{styleBody}
‘blood’
\end{styleBody}

\begin{styleBody}
(Georg \& Volodin 1999: 16)
\end{styleBody}

\begin{styleBody}
(3.16)\ \ \textbf{Ewe} (\textit{Atlantic-Congo}; Ghana, Togo)
\end{styleBody}

\begin{styleBody}
(a) \ \ j[254?]\`{m}[329?]
\end{styleBody}

\begin{styleBody}
\ \ ‘call me’
\end{styleBody}

\begin{styleBody}
(b)\ \ kamp\'{e}
\end{styleBody}

\begin{styleBody}
\ \ ‘scissors’
\end{styleBody}

\begin{styleBody}
(Ameka 1991: 38)
\end{styleBody}

\begin{styleBody}
\ \ Three languages are excluded from the current analysis: Chipaya, Nimboran (both from the Complex category), and Yine (from the Highly Complex category). For all three of these languages, there are conflicting reports regarding the occurrence of syllabic consonants, sometimes from the same author.\footnote{\textrm{There are a few other languages for which there are suggestions of alternate analyses. The dialect of Sahaptin analyzed here, Yakima, is argued not to have syllabic consonants by Hargus \& Beavert (2006) on the basis of distributional and phonological behavior of consonants in sequences. However, it should be noted that Minthorn (2005) argues for syllabic consonants, including obstruents, in the closely related dialect of Umatilla Sahaptin, on the basis of speaker intuition and acoustic analysis. Additionally, one description of Alamblak lists an example of a word consisting entirely of obstruents: }\textrm{\textit{kpt}}\textrm{ ‘basket type’ (SIL 2004: 1); however, no further elaboration is given and obstruents are not included in the description of syllabic consonants in Bruce (1984), so it is unclear whether syllabic obstruents are an issue of debate for this language. Finally, for Itelmen, Volodin (1976: 42) gives transcriptions of lexical items consisting entirely of obstruents (}\textrm{\textit{t[361?][283?]kpt[361?][283?] }}\textrm{‘spoon’). In a later reference, he describes only syllabic sonorants in the language (Georg \& Volodin 1999).}} For example, Matteson gives the following description for Yine, which seems to suggest that consonants in complex onsets both belong to a syllable with a vocalic nucleus and are simultaneously themselves syllabic:
\end{styleBody}

\begin{styleBody}
“We number the consonants of the syllable, beginning with the consonant that immediately precedes the nuclear vowel: +C\textsuperscript{3} +C\textsuperscript{2} +C\textsuperscript{1}V. In the positions of consonants C\textsuperscript{2} and C\textsuperscript{3} occur syllabic allophones of the consonants. Thus the syllable is a complex unit consisting of from one to three syllabic units.” 
\end{styleBody}

\begin{styleBody}
(Matteson 1965: 23)
\end{styleBody}

\begin{styleBody}
Because of the conflicting descriptions of these languages, I opted to exclude them from the current analysis. Georgian and Tashlhiyt also have conflicting descriptions with respect to the occurrence of syllabic consonants, but in both of these cases experimental evidence has been presented to support one analysis over another. The articulatory and acoustic experiments in Ridouane (2008) and Goldstein et al. (2007) support a syllabic consonant analysis for Tashlhiyt, while native speaker intuition reported in Chitoran (1999) does not support an analysis of syllabic sonorants for Georgian. It is interesting to note that all of the languages with conflicting descriptions — those discussed here and the ones mentioned in the footnote — are from the Complex and Highly Complex categories. This recalls the observation noted previously, in which transitions in consonant clusters on the one hand and syllabic consonants on the other may have similar motivations and acoustic manifestations.
\end{styleBody}

\begin{flushleft}
\tablefirsthead{}
\tablehead{}
\tabletail{}
\tablelasttail{}
\begin{supertabular}{|m{1.7427598in}|m{1.0900599in}|m{1.0900599in}|m{1.0900599in}|m{1.0900599in}|}
\hline
 &
\multicolumn{4}{m{4.59646in}|}{\centering{\bfseries Syllable Structure Complexity}}\\\hline
\raggedleft{\bfseries Languages with:} &
{\centering\bfseries Simple\par}

\centering (\textit{N} = 24 lgs) &
{\centering\bfseries Moderately Complex\par}

\centering (\textit{N} = 26 lgs) &
{\centering\bfseries Complex\par}

\centering (\textit{N} = 23 lgs) &
{\centering\bfseries Highly \par}

{\centering\bfseries Complex\par}

\centering\arraybslash (\textit{N} = 24 lgs)\\\hline
\raggedleft{\bfseries Syllabic consonants of any kind} &
\centering{\bfseries 2} &
\centering{\bfseries 6} &
\centering{\bfseries 5} &
\centering\arraybslash{\bfseries 11}\\\hline
\raggedleft{\itshape {}- syllabic nasals} &
\centering 2 &
\centering 5 &
\centering 5 &
\centering\arraybslash 10\\\hline
\raggedleft{\itshape {}- syllabic liquids} &
\centering — &
\centering 2 &
\centering 1 &
\centering\arraybslash 6\\\hline
\raggedleft{\itshape {}- syllabic obstruents} &
\centering — &
\centering — &
\centering — &
\centering\arraybslash 5\\\hline
\raggedleft{\bfseries No syllabic consonants} &
\centering{\bfseries 22} &
\centering{\bfseries 20} &
\centering{\bfseries 18} &
\centering\arraybslash{\bfseries 13}\\\hline
\end{supertabular}
\end{flushleft}
\begin{styleBody}
\textbf{Table 3.6. }Presence of invariant syllabic consonants in language sample, by syllable structure complexity. Chipaya, Nimboran (Complex) and Yine (Highly Complex) excluded.
\end{styleBody}

\begin{styleBody}
\ \ The syllabic consonant patterns reported for the languages of the sample can be found in Table 3.6. There is a steadily increasing trend in the proportion of languages with these patterns as syllable structure complexity increases: only two of languages with Simple syllable structure are reported to have syllabic consonants, compared to 11 of the languages in the Highly Complex category. The trend in the Highly Complex category is statistically significant when compared to the trends in the other three categories combined (p = .01 in Fisher’s exact test). Examining the particular kinds of syllabic consonants represented, the patterns are similar to what is reported in Bell (1978a). Most languages with syllabic consonants have syllabic nasals, and languages with syllabic obstruents are rare. While languages from all four categories have syllabic nasals and most have syllabic liquids, syllabic obstruents are only reported for languages in the Highly Complex category. This is not a remnant of the way the Highly Complex category is defined: recall that languages with syllabic obstruents are categorized as Highly Complex only if these structures participate in word-marginal sequences of three obstruents or more. It is striking that no languages with simpler syllable structure are reported to have syllabic obstruents. Even if the three languages excluded from the previous analysis were included here, the distribution of syllabic obstruents would be among two languages with Complex syllable structure and six languages with Highly Complex syllable structure. 
\end{styleBody}

\begin{styleBody}
\ \ It should also be noted that three of the languages with syllabic obstruents (Cocopa, Semai, Tashlhiyt) are reported to also have both syllabic nasals and syllabic liquids. Tehuelche does not have syllabic liquids. Tohono O’odham is the only language which has syllabic obstruents but not syllabic sonorant consonants. This indicates that the trend in Table 3.6 — by which Highly Complex languages are more likely than any of the other categories to have syllabic consonants \ — is not driven by the inclusion of syllabic obstruents in the definition of that category, or skewed by potential misanalyses which confound syllabic obstruents and large tautosyllabic clusters. Instead, the trend can be obtained from the syllabic nasal and liquid patterns in the sample.
\end{styleBody}

\begin{styleBody}
\ \ While the analyses presented above are for the invariant syllabic consonant patterns observed in the sample, there were also several cases in which syllabic consonants were reported to occur in variation with CV or VC sequences, as illustrated by (3.17)-(3.18).
\end{styleBody}

\begin{styleBody}
(3.17) \ \ \textbf{Sichuan Yi} (\textit{Sino-Tibetan}; China)
\end{styleBody}

\begin{styleBody}
\textit{Nasals and laterals preceding [[268?]] occur in free variation with syllabic consonants.}
\end{styleBody}

\begin{styleBody}
/l[268?]/
\end{styleBody}

\begin{styleBody}
[l[268?]]\~{}[\textbf{l[329?]}]
\end{styleBody}

\begin{styleBody}
(Gerner 2013: 31)
\end{styleBody}

\begin{styleBody}
(3.18) \ \ \textbf{Mamaindê} (\textit{Nambiquaran}, Brazil)
\end{styleBody}

\begin{styleBody}\itshape
When an unstressed vowel is lost resulting in a sequence of nasal plus consonant, a preceding nasal becomes syllabic.
\end{styleBody}

\begin{styleBody}
/[2C8?]johnalat[2B0?]awa/
\end{styleBody}

\begin{styleBody}
[[2C8?]joh\textbf{n[329?]}lat[2B0?]wa]
\end{styleBody}

\begin{styleBody}
‘it is low’
\end{styleBody}

\begin{styleBody}
(Eberhard 2009: 262-3)
\end{styleBody}

\begin{flushleft}
\tablefirsthead{}
\tablehead{}
\tabletail{}
\tablelasttail{}
\begin{supertabular}{|m{1.6531599in}|m{1.1115599in}|m{1.1122599in}|m{1.1122599in}|m{1.1136599in}|}
\hline
 &
\multicolumn{4}{m{4.68596in}|}{\centering{\bfseries Syllable Structure Complexity}}\\\hline
\raggedleft{\bfseries Languages with variable:} &
{\centering\bfseries Simple\par}

\centering (\textit{N} = 1 lg) &
{\centering\bfseries Moderately Complex\par}

\centering (\textit{N} = 3 lgs) &
{\centering\bfseries Complex\par}

\centering (\textit{N} = 2 lgs) &
{\centering\bfseries Highly \par}

{\centering\bfseries Complex\par}

\centering\arraybslash (\textit{N} = 3 lgs)\\\hline
\raggedleft{\itshape Syllabic nasals} &
\centering 1 &
\centering 3 &
\centering 2 &
\centering\arraybslash 3\\\hline
\raggedleft{\itshape Syllabic liquids} &
\centering 1 &
\centering — &
\centering — &
\centering\arraybslash 2\\\hline
\raggedleft{\itshape Syllabic obstruents} &
\centering — &
\centering 1 &
\centering — &
\centering\arraybslash 1\\\hline
\end{supertabular}
\end{flushleft}
\begin{styleBody}
\textbf{Table 3.7.} Distribution of languages in sample with syllabic nasals, liquids, and obstruents occurring in variation with VC or CV structures.
\end{styleBody}

\begin{styleBody}
\ \ Table 3.7 shows the distribution of variable processes producing syllabic consonants in the data. Though the data set is very small, it is interesting that the general distributional pattern is similar to that presented in Table 3.6. The occurrence of syllabic consonants in variation with VC or CV structures is least frequent among languages with Simple syllable structure, and in all categories nasals are the most common syllabic consonant to result. Variable syllabic obstruents occur in two languages. In Paiwan (Moderately Complex), syllabic obstruents may occur when schwa is reduced immediately after a sibilant in rapid speech, and in Kabardian (Highly Complex), they occur as the result of an optional process of high vowel contraction (3.19-3.20):
\end{styleBody}

\begin{styleBody}
(3.19)\ \ \textbf{Paiwan} (\textit{Austronesian}; Taiwan)
\end{styleBody}

\begin{styleBody}
/s[259?]kam/
\end{styleBody}

\begin{styleBody}
[s[259?]kam]\~{}[\textbf{s[329?]}kam]
\end{styleBody}

\begin{styleBody}
‘mattress’
\end{styleBody}

\begin{styleBody}
(Chang 2006: 41)
\end{styleBody}

\begin{styleBody}
(3.20)\ \ \textbf{Kabardian} (\textit{Abkhaz-Adyge}; Russia, Turkey)
\end{styleBody}

\begin{styleBody}
/[26C?]'[259?][292?]/
\end{styleBody}

\begin{styleBody}
[[26C?]’i[292?]]\~{}[[26C?]'\textbf{[292?][329?]}]
\end{styleBody}

\begin{styleBody}
‘old man’
\end{styleBody}

\begin{styleBody}
(Kuipers 1960: 24)
\end{styleBody}

\begin{styleBody}
Kabardian is also the only language in the sample reported to have both invariant syllabic consonants (for sonorants in certain consonant environments) and variable syllabic consonants as a result of synchronic phonetic processes like the one illustrated above.
\end{styleBody}

\begin{styleBody}
\ \ Returning to the hypothesis stated at the beginning of this section, there is evidence that languages with more complex syllable structure are more likely to have syllabic consonant patterns. Specifically, languages with Highly Complex syllable structure are the most likely of all those in the sample to have invariant syllabic consonants, while languages with Simple syllable structure are the least likely. Variable processes resulting in syllabic consonants are also relatively more frequent in languages with non-Simple syllable structure.
\end{styleBody}

\section[3.3.6 Morphological patterns]{\rmfamily 3.3.6 Morphological patterns}
\begin{styleBody}
\ \ In this section I analyze the morphological patterns associated with syllable patterns in the language sample. First, I report the morphological constituency patterns observed in the maximal onset and coda structures in each language. Then I present an analysis of the kinds of morphemes (lexical or grammatical) in which syllabic consonants in the language sample occur. I test the hypotheses formulated in §3.1.2 with respect to these patterns: first, that as syllable structure complexity increases, so does the likelihood that the largest syllable margin types in a language will be morphologically complex; second, that as syllable structure complexity increases, so does the likelihood that syllabic consonants occurring in a language will belong to grammatical elements.
\end{styleBody}

\begin{styleBody}
\ \ Since morphologically complex instances of syllable patterns are often not explicitly described and must be gathered from the examples, it was impractical and in many cases impossible to find morpheme-internal and morphologically complex instances of the same specific consonant sequence in each language, especially for the larger clusters. The patterns analyzed here are for the maximal onset and coda \textit{types}, e.g. CC. For example, the maximal coda in Gaam is two consonants. The word-final patterns shown in the examples below would be taken as evidence that the maximal coda occurs in both morpheme-internal (3.21a) and morphologically complex (3.21b) contexts.
\end{styleBody}

\begin{styleBody}
(3.21) \ \ \textbf{Gaam} (\textit{Eastern Jebel}; Sudan)
\end{styleBody}

\begin{styleBody}
(a)\ \ b\={a}[261?]d[32A?]\`{a}\textbf{rs}
\end{styleBody}

\begin{styleBody}
‘lizard type’
\end{styleBody}

\begin{styleBody}
(b)\ \ [261?][259?]\={u}\textbf{r{}-d[32A?]}
\end{styleBody}

\begin{styleBody}
stomach-\textsc{sg}
\end{styleBody}

\begin{styleBody}
‘stomach’
\end{styleBody}

\begin{styleBody}
(Stirtz 2011: 32, 37)
\end{styleBody}

\begin{styleBody}
\ \ Note also that the definition of \textit{morphologically complex} here refers to sequences derived by any morphological process. That is, sequences derived through reduplication or nonconcatenative processes such as subtractive morphology are also considered to be morphologically complex, even though they don’t involve more than one distinct morpheme.
\end{styleBody}

\begin{styleBody}
\ \ First I test Greenberg’s (1965/1978) prediction in the data: as the size of a syllable margin increases, so does the probability that it contains morpheme boundaries. Figures 3.3 and 3.4 show morphological constituency patterns in maximal onset and coda types in the data.
\end{styleBody}

\begin{styleBody}
  [Warning: Image ignored] % Unhandled or unsupported graphics:
%\includegraphics[width=5.8937in,height=2.0484in,width=\textwidth]{./ObjectReplacements/Object 7}
 
\end{styleBody}

\begin{styleBody}
\textbf{Figure 3.3. }Morphological constituency patterns in maximal onset types in data. For each maximal onset type, figure shows proportion of languages exhibiting the given morphological patterns for that type.
\end{styleBody}

\begin{styleBody}
  [Warning: Image ignored] % Unhandled or unsupported graphics:
%\includegraphics[width=5.8055in,height=2.0484in,width=\textwidth]{./ObjectReplacements/Object 9}
 
\end{styleBody}

\begin{styleBody}
\textbf{Figure 3.4. }Morphological constituency patterns in maximal coda types in data. For each maximal coda type, figure shows proportion of languages exhibiting the given morphological patterns for that type.
\end{styleBody}

\begin{styleBody}
\ \ For both maximal onset and maximal coda patterns in the data, the proportion of languages having these clusters solely in morphologically complex contexts increases with cluster size. However, morphologically complex patterns also occur alongside morpheme-internal patterns in the maximal margins for a number of languages (the “Both patterns” trend in Figures 3.3 and 3.4). When this trend is additionally considered, we find that maximal coda cluster types are generally more likely than maximal onset types to exhibit morphologically complex patterns. We also find that all maximal cluster types of five consonants or larger are found only in morphologically complex contexts.
\end{styleBody}

\begin{styleBody}
\ \ Interestingly, there are some language-internal patterns in the data which go against Greenberg’s prediction. In Lelepa there are biconsonantal onsets showing both morphological patterns, but the only attested triconsonantal onsets are within morphemes (3.22a-c).
\end{styleBody}

\begin{styleBody}
(3.22) \ \ \textbf{Lelepa} (\textit{Austronesian}; Vanuatu)
\end{styleBody}

\begin{styleBody}
(a)\ \ n{}-malo[261?]o
\end{styleBody}

\begin{styleBody}
\textsc{nmlz}{}-darken
\end{styleBody}

\begin{styleBody}
‘darkness’
\end{styleBody}

\begin{styleBody}
(b)\ \ nmal
\end{styleBody}

\begin{styleBody}
‘trunk’
\end{styleBody}

\begin{styleBody}
(c)\ \ psruki
\end{styleBody}

\begin{styleBody}
‘speak’
\end{styleBody}

\begin{styleBody}
(Lacrampe 2014: 107, 207, 42)
\end{styleBody}

\begin{styleBody}
\ \ The analyses presented in Figures 3.3-3.4 test Greenberg’s specific predictions regarding cluster size. However, the hypothesis in (3.4a) is formulated with respect to syllable structure complexity, which is a slightly different question, though we expect to find a similar pattern due to how the categories are defined. In Figures 3.5-3.6 I present the morphological constituency patterns observed by syllable structure complexity category. Note that these figures only include the languages from each category which have complex onsets or complex codas, respectively.
\end{styleBody}

\begin{styleBody}

\end{styleBody}

\begin{styleBody}
\textbf{Figure 3.5. }Morphological constituency patterns in maximal complex onsets, by syllable structure complexity category.\textbf{ }For each category, the figure shows the proportion of languages exhibiting the given morphological patterns in its complex onsets.
\end{styleBody}

\begin{styleBody}

\end{styleBody}

\begin{styleBody}
\textbf{Figure 3.6. }Morphological constituency patterns in maximal complex codas, by syllable structure complexity category.\textbf{ }For each category, the figure shows the proportion of languages exhibiting the given morphological patterns in its complex onsets. Note that the one language with complex codas from the Moderately Complex category — Eastern Khanty — is not included in the figure. Its very marginal complex codas are always morphologically complex.
\end{styleBody}

\begin{styleBody}
\ \ The figures show that as syllable structure complexity increases, both maximal onset and maximal coda clusters are more likely to have morphologically complex patterns, confirming the hypothesis in (3.4a).
\end{styleBody}

\begin{styleBody}
\ \ The patterns in figures 3.5-3.6 are combined in Figure 3.7 in order to show the general trend for morphologically complex patterns in maximal syllable-marginal clusters with respect to syllable structure complexity in the language sample. In this figure onset and coda patterns are collapsed, and the “both contexts” and “only morphologically complex” patterns are combined. For each category, the figure shows the percentage of languages with complex syllable margins for which morphologically complex patterns occur in either or both maximal syllable margins.
\end{styleBody}

\begin{styleBody}

\end{styleBody}

\begin{styleBody}
\textbf{Figure 3.7. }Percentage of languages in each category exhibiting morphologically complex patterns in either or both of its maximal syllable margins.
\end{styleBody}

\begin{styleBody}
\ \ Morphologically complex patterns can be found in the maximal syllable margins of most languages from the Highly Complex category, and this pattern is statistically significant when compared against the patterns in the other two categories combined (p {\textless} .01 in Fisher’s exact test). However, there are six languages in this category for which I could determine only morpheme-internal patterns. In Wutung, the maximal margin is explicitly described as occurring within a few apparently single-morpheme lexical items. In Kunjen and Lezgian, maximal coda and onset clusters, respectively, seem to be limited to single-morpheme lexical items, though the references consulted do not explicitly state this. In Menya, the only morphologically complex instance of a maximal cluster that could be found was in an abstract phonemic transcription for which the phonetic form was unclear. In Passamaquoddy-Maliseet, examples of morphologically complex instances of maximal clusters could not be found, though it seems as though the morphology could produce them. The remaining language, Semai, has syllabic consonants and will be discussed below.
\end{styleBody}

\begin{styleBody}
\ \ Recall that there are four languages in the Highly Complex portion of the sample for which the largest word-marginal obstruent sequences include syllabic consonants. The maximal ‘true’ onset/coda clusters reported for these languages (cf. Table 3.2) were included in the previous analyses in this section, but the maximal word-marginal sequences were not. I present the morphological patterns for these sequences in Table 3.8.
\end{styleBody}

\begin{flushleft}
\tablefirsthead{}
\tablehead{}
\tabletail{}
\tablelasttail{}
\begin{supertabular}{m{0.78095984in}|m{1.2066599in}m{1.2080599in}|m{1.2066599in}m{1.2059599in}}
\hline
{\bfseries Language} &
{\centering\bfseries Maximal \par}

{\centering\bfseries word-initial \par}

\centering{\bfseries obstruent string} &
\centering{\bfseries Morphological pattern} &
{\centering\bfseries Maximal \par}

{\centering\bfseries word-final \par}

\centering{\bfseries obstruent string} &
\centering\arraybslash{\bfseries Morphological pattern}\\\hline
Cocopa &
\centering 5 &
\centering Morph. complex &
\centering 3 &
\centering\arraybslash Morph. complex\\
Semai &
\centering 4 &
\centering Morph. complex &
\centering 1 &
\centering\arraybslash —\\
Tashlhiyt &
\centering (words without vowels) &
\centering Morph. complex &
\centering (words without vowels) &
\centering\arraybslash Morph. complex\\
Tehuelche &
\centering 3 &
\centering Morph. complex &
\centering {\textgreater}3 &
\centering\arraybslash Morph. complex\\\hline
\end{supertabular}
\end{flushleft}
\begin{styleBody}
\textbf{Table 3.8.} Morphological patterns of maximal word-marginal obstruent sequences in languages with syllabic obstruents in Highly Complex category.
\end{styleBody}

\begin{styleBody}
\ \ All of the maximal word-marginal obstruent sequences in the languages in Table 3.8 occur in morphologically complex contexts. For example, in Semai, all maximal word-initial consonant sequences, and indeed all word-initial sequences of more than two consonants, are derived through reduplication processes (3.23).
\end{styleBody}

\begin{styleBody}
(3.23)\ \ \textbf{Semai }(\textit{Austroasiatic}; Malaysia)
\end{styleBody}

\begin{styleBody}
[261?]p.[261?].hup ({\textless} [261?]hup )
\end{styleBody}

\begin{styleBody}
‘irritation on skin (e.g., from bamboo hair)’
\end{styleBody}

\begin{styleBody}
(Sloan 1988: 320; Diffloth 1976a: 256)
\end{styleBody}

\begin{styleBody}
Though maximal word-marginal obstruent string length cannot be determined in Tashlhiyt due to the occurrence of many words consisting entirely of obstruents in this language, the longest such words are morphologically complex (3.24).
\end{styleBody}

\begin{styleBody}
(3.24)\ \ \textbf{Tashlhiyt} (\textit{Afro-Asiatic}; Morocco)
\end{styleBody}

\begin{styleBody}
t{}-s[2D0?]{}-k[283?]f{}-t=st[2D0?]
\end{styleBody}

\begin{styleBody}
ts.sk.[283?]f.tst[2D0?]
\end{styleBody}

\begin{styleBody}
‘you dried it (\textsc{f})’
\end{styleBody}

\begin{styleBody}
(Ridouane 2008: 332; interlinear gloss not provided)
\end{styleBody}

\begin{styleBody}
\ \ Now we turn to the hypothesis in (3.4b): as syllable structure complexity increases, so does the likelihood that syllabic consonants occurring in a language will belong to grammatical elements. This is based partly on Bell’s (1978a) observation that the syllabic consonants in his typological survey were often restricted to grammatical particles and affixes.
\end{styleBody}

\begin{styleBody}
\ \ Only the languages reported in §3.3.5 as having invariant syllabic consonant patterns are included in the analysis here. Additionally, Kabardian is excluded from the present analysis because its precise patterns could not be determined. For each kind of syllabic consonant analyzed (nasal, liquid, and obstruent), I determine whether that type occurs in lexical morphemes, grammatical morphemes, or both. For example, in Bench, syllabic nasals can be found in both lexical and grammatical morphemes (3.25).
\end{styleBody}

\begin{styleBody}
(3.25)\ \ \textbf{Bench} (\textit{Ta-Ne-Omotic}; Ethiopia)
\end{styleBody}

\begin{styleBody}
(a) \ \ [261?][215?]p\textbf{\={m}}
\end{styleBody}

\begin{styleBody}
‘foam’
\end{styleBody}

\begin{styleBody}
(b) \ \ nj\=a[294?]{}-\textbf{\={n}}d
\end{styleBody}

\begin{styleBody}
child-\textsc{pl}
\end{styleBody}

\begin{styleBody}
‘children’
\end{styleBody}

\begin{styleBody}
(Rapold 2006: 107, 111)
\end{styleBody}

\begin{styleBody}
\ \ In Tables 3.9-3.11 I present analyses for the morphological patterns of each kind of syllabic consonant (nasal, liquid, and obstruent) observed in the data.
\end{styleBody}

\begin{flushleft}
\tablefirsthead{}
\tablehead{}
\tabletail{}
\tablelasttail{}
\begin{supertabular}{|m{1.9205599in}|m{1.0400599in}|m{1.0406599in}|m{1.0400599in}|m{1.0406599in}|}
\hline
 &
\multicolumn{4}{m{4.3976603in}|}{\centering{\bfseries Syllable Structure Complexity}}\\\hline
{\raggedleft\bfseries Languages with \par}

\raggedleft{\bfseries syllabic nasals in:} &
{\centering\bfseries Simple\par}

\centering (\textit{N} = 2 lgs) &
{\centering\bfseries Moderately Complex\par}

\centering (\textit{N} = 5 lgs) &
{\centering\bfseries Complex\par}

\centering (\textit{N} = 6 lgs) &
{\centering\bfseries Highly \par}

{\centering\bfseries Complex\par}

\centering\arraybslash (\textit{N} = 9 lgs)\\\hline
\raggedleft{\itshape Lexical morphemes only} &
\centering 1 &
\centering 3 &
\centering 2 &
\centering\arraybslash 1\\\hline
\raggedleft{\itshape Lex. and gram. morphemes} &
\centering — &
\centering 2 &
\centering 2 &
\centering\arraybslash 6\\\hline
\raggedleft{\itshape Grammatical \newline
morphemes only} &
\centering 1 &
\centering — &
\centering 2 &
\centering\arraybslash 2\\\hline
\end{supertabular}
\end{flushleft}
\begin{styleBody}
\textbf{Table 3.9.} Morphological patterns of syllabic nasals in sample, by syllable structure complexity. Kabardian (Highly Complex) is omitted as its pattern could not be determined.
\end{styleBody}

\begin{flushleft}
\tablefirsthead{}
\tablehead{}
\tabletail{}
\tablelasttail{}
\begin{supertabular}{|m{1.9212599in}|m{1.0427599in}|m{1.0434599in}|m{1.0434599in}|m{1.0448599in}|}
\hline
 &
\multicolumn{4}{m{4.41076in}|}{\centering{\bfseries Syllable Structure Complexity}}\\\hline
{\raggedleft\bfseries Languages with \par}

\raggedleft{\bfseries syllabic liquids in:} &
{\centering\bfseries Simple\par}

\centering (\textit{N} = 0 lgs) &
{\centering\bfseries Moderately Complex\par}

\centering (\textit{N} = 2 lgs) &
{\centering\bfseries Complex\par}

\centering (\textit{N} = 1 lg) &
{\centering\bfseries Highly \par}

{\centering\bfseries Complex\par}

\centering\arraybslash (\textit{N} = 5 lgs)\\\hline
\raggedleft{\itshape Lexical morphemes only} &
\centering — &
\centering 2 &
\centering 1 &
\centering\arraybslash 2\\\hline
\raggedleft{\itshape Lex. and gram. morphemes} &
\centering — &
\centering — &
\centering — &
\centering\arraybslash 2\\\hline
\raggedleft{\itshape Grammatical \newline
morphemes only} &
\centering — &
\centering — &
\centering — &
\centering\arraybslash 1\\\hline
\end{supertabular}
\end{flushleft}
\begin{styleBody}
\textbf{Table 3.10.} Morphological patterns of syllabic liquids in sample, by syllable structure complexity. Kabardian (Highly Complex) is omitted as its pattern could not be determined.
\end{styleBody}

\begin{flushleft}
\tablefirsthead{}
\tablehead{}
\tabletail{}
\tablelasttail{}
\begin{supertabular}{|m{1.9205599in}|m{1.0379599in}|m{1.0372599in}|m{1.0379599in}|m{1.0365599in}|}
\hline
 &
\multicolumn{4}{m{4.38596in}|}{\centering{\bfseries Syllable Structure Complexity}}\\\hline
{\raggedleft\bfseries Languages with \par}

\raggedleft{\bfseries syllabic obstruents in:} &
{\centering\bfseries Simple\par}

\centering (\textit{N} = 0 lgs) &
{\centering\bfseries Moderately Complex\par}

\centering (\textit{N} = 0 lgs) &
{\centering\bfseries Complex\par}

\centering (\textit{N} = 0 lgs) &
{\centering\bfseries Highly \par}

{\centering\bfseries Complex\par}

\centering\arraybslash (\textit{N} = 5 lgs)\\\hline
\raggedleft{\itshape Lexical morphemes only} &
\centering — &
\centering — &
\centering — &
\centering\arraybslash —\\\hline
\raggedleft{\itshape Lex. and gram. morphemes} &
\centering — &
\centering — &
\centering — &
\centering\arraybslash 1\\\hline
\raggedleft{\itshape Grammatical \newline
morphemes only} &
\centering — &
\centering — &
\centering — &
\centering\arraybslash 4\\\hline
\end{supertabular}
\end{flushleft}
\begin{styleBody}
\textbf{Table 3.11.} Morphological patterns of syllabic obstruents in sample, by syllable structure complexity.
\end{styleBody}

\begin{styleBody}
\ \ In general, the pattern by which syllabic consonants are found to occur in grammatical morphemes, either exclusively or in addition to lexical morphemes, is the dominant one in the data. This is the case for 15/22 languages with syllabic nasals and all of the languages with syllabic obstruents. And within this very small data set, this trend also appears to increase with syllable structure complexity, suggesting support for the hypothesis. 
\end{styleBody}

\begin{styleBody}
\ \ In fact most of the languages with syllabic consonants in the Highly Complex category have these sounds in grammatical items. In Tehuelche, for example, \textit{all} syllabic consonants correspond to or belong to grammatical morphemes (3.26).
\end{styleBody}

\begin{styleBody}
(3.26) \ \ \textbf{Tehuelche} (\textit{Chonan}; Argentina)
\end{styleBody}

\begin{styleBody}
k.t[361?][283?]a[294?][283?]p.[283?].k’n
\end{styleBody}

\begin{styleBody}
k{}-t[361?][283?]a[294?][283?]p{}-[283?]{}-k’n
\end{styleBody}

\begin{styleBody}
\textsc{refl}{}-wash-\textsc{ps-realis}
\end{styleBody}

\begin{styleBody}
‘it is being washed’
\end{styleBody}

\begin{styleBody}
(Fernández Garay \& Hernández 2006: 13)
\end{styleBody}

\begin{styleBody}
\ \ To summarize, in this section the morphological patterns of maximal onset types, maximal coda types, and syllabic consonant inventories in the sample have been examined. In both cases there is support for the hypotheses in (3.4). Clearly morphology contributes an important role to the development of complex syllable patterns. While this point will be only briefly revisited in the discussion in §3.5, it will be discussed in further detail in Chapter 8.
\end{styleBody}

\section[3.4 Properties of highly complex syllable structure]{\rmfamily 3.4 Properties of highly complex syllable structure}
\begin{styleBody}
\ \ Having described the general patterns of maximal onsets, maximal codas, and syllabic consonants in the data, I now turn to an examination of the properties of syllable structure in the languages in the Highly Complex portion of the sample. In §3.4.1 I give examples of the syllable patterns occurring in each of these languages. In §3.4.2 I attempt to characterize the prevalence of Highly Complex structures within each of the languages by examining restrictions on consonant combinations and reported frequency patterns. In §3.4.3 I present information on the acoustic and perceptual properties of Highly Complex structures.
\end{styleBody}

\section[3.4.1 Examples of Highly Complex syllable patterns in sample]{\rmfamily 3.4.1 Examples of Highly Complex syllable patterns in sample}
\begin{styleBody}
\ \ In order to provide a better picture of what specific syllable patterns occur in the languages of the Highly Complex portion of the sample, I list some representative structures in Table 3.12. The definition of Highly Complex syllable structure includes any onset or coda structure of three obstruents, or of four consonants or more in length. It also includes any word-marginal sequence containing syllabic obstruents such that a sequence of three or more obstruents occurs at a word margin. For each language I give a set of examples for each onset, coda, and/or word-marginal cluster that occurs at each of the following lengths: three consonants, four consonants, and five or more consonants. For Tashlhiyt, I have given some examples of vowelless words in the rightmost column, but have not assigned them to a word margin.
\end{styleBody}

\begin{flushleft}
\tablefirsthead{}
\tablehead{}
\tabletail{}
\tablelasttail{}
\begin{supertabular}{|m{1.0844599in}|m{1.6990598in}|m{1.6983598in}|m{1.6997598in}|}
\hline
 &
\multicolumn{3}{m{5.2546597in}|}{\centering{\bfseries Highly Complex structures}}\\\hline
{\bfseries Language} &
\centering{\bfseries 3-obstruent structures} &
\centering{\bfseries 4-C structures} &
\centering\arraybslash{\bfseries 5-C and larger structures}\\\hline
{\bfseries Alamblak} &
{\fontsize{10pt}{12.0pt}\selectfont\mdseries\upshape \textbf{Onset: }tkb} &
\centering — &
\centering\arraybslash —\\\hline
{\bfseries Bench} &
{\fontsize{10pt}{12.0pt}\selectfont\mdseries\upshape \textbf{Coda:} pst} &
\centering — &
\centering\arraybslash —\\\hline
{\bfseries Menya} &
{\fontsize{10pt}{12.0pt}\selectfont\mdseries\upshape \textbf{Onset: }tpq, ptq} &
\centering — &
\centering\arraybslash —\\\hline
{\bfseries Kabardian} &
{\fontsize{10pt}{12.0pt}\selectfont\mdseries\upshape \textbf{Onset:} zb[263?], p[255?]t, psk’} &
\centering — &
\centering\arraybslash —\\\hline
{\bfseries Lezgian} &
{\fontsize{10pt}{12.0pt}\selectfont\mdseries\upshape \textbf{Onset: }[283?]tk, kst, ktk} &
\centering — &
\centering\arraybslash —\\\hline
{\bfseries Yine} &
{\fontsize{10pt}{12.0pt}\selectfont\mdseries\upshape \textbf{Onset:} (pc[27E?], nt[361?]sp, nt[361?][283?]k)} &
\centering — &
\centering\arraybslash —\\\hline
{\bfseries Camsá} &
{\fontsize{10pt}{12.0pt}\selectfont\mdseries\upshape \textbf{Onset:} stx, st[361?][283?]b, s[283?]t[361?]s} &
{\fontsize{11pt}{13.2pt}\selectfont\mdseries\upshape \textbf{Onset:} [278?]stx} &
\centering\arraybslash —\\\hline
{\bfseries Semai} &
{\fontsize{10pt}{12.0pt}\selectfont\mdseries\upshape \textbf{Word-initial:} st.s} &
{\fontsize{10pt}{12.0pt}\selectfont\mdseries\upshape \textbf{Word-initial:} [261?]p.[261?].h} &
\centering\arraybslash —\\\hline
{\bfseries Nuu-chah-nulth} &
{\fontsize{10pt}{12.0pt}\selectfont\mdseries\upshape \textbf{Coda:} t[361?][283?]tq, kqs, qt[361?][26C?]s, t[127?]t[361?]s} &
{\fontsize{10pt}{12.0pt}\selectfont\mdseries\upshape \textbf{Coda:} mtq[283?], [127?]sq[127?], nkq[127?] } &
\centering\arraybslash —\\\hline
{\bfseries Wutung} &
\centering — &
{\fontsize{10pt}{12.0pt}\selectfont\mdseries\upshape \textbf{Onset}: hmbl } &
\centering\arraybslash —\\\hline
{\bfseries Doyayo} &
\centering — &
{\fontsize{10pt}{12.0pt}\selectfont\mdseries\upshape \textbf{Coda}:\textbf{ }$\beta $lts, [263?]ldz, mnts} &
\centering\arraybslash —\\\hline
{\bfseries Kunjen} &
\centering — &
{\fontsize{10pt}{12.0pt}\selectfont\mdseries\upshape \textbf{Coda: }lbmb, [279?]dnd, j[261?]ŋ[261?]} &
\centering\arraybslash —\\\hline
{\bfseries Passamaquoddy-Maliseet} &
{\fontsize{11pt}{13.2pt}\selectfont\mdseries\upshape \textbf{Onset:} psk, ksp, psk[2B7?]}

{\fontsize{11pt}{13.2pt}\selectfont\mdseries\upshape \textbf{Coda: }psk[2B7?], ksk[2B7?] } &
\centering — &
\centering\arraybslash —\\\hline
{\bfseries Qawasqar} &
{\fontsize{10pt}{12.0pt}\selectfont\mdseries\upshape \textbf{Onset:} qsq, qst, qsk}

{\fontsize{10pt}{12.0pt}\selectfont\mdseries\upshape \textbf{Coda}: qsq } &
{\fontsize{10pt}{12.0pt}\selectfont\mdseries\upshape \textbf{Onset:} qsqj } &
\centering\arraybslash —\\\hline
{\bfseries Tehuelche} &
{\fontsize{10pt}{12.0pt}\selectfont\mdseries\upshape \textbf{Word-initial: }k[283?].x, k[283?].[294?]}

{\fontsize{10pt}{12.0pt}\selectfont\mdseries\upshape \textbf{Coda:} [294?][283?]p } &
{\fontsize{10pt}{12.0pt}\selectfont\mdseries\upshape \textbf{Word-final:} [283?]p.[283?].k’} &
\centering\arraybslash —\\\hline
{\bfseries Albanian} &
{\fontsize{10pt}{12.0pt}\selectfont\mdseries\upshape \textbf{Onset:} skt, p[283?]t}

{\fontsize{10pt}{12.0pt}\selectfont\mdseries\upshape \textbf{Coda:} p[283?]t, kst} &
{\fontsize{10pt}{12.0pt}\selectfont\mdseries\upshape \textbf{Onset:} t[361?][283?]mpl, zmbr} &
\centering\arraybslash —\\\hline
{\bfseries Mohawk} &
{\fontsize{10pt}{12.0pt}\selectfont\mdseries\upshape \textbf{Onset: }ksk, kts, kst, kht }

{\fontsize{10pt}{12.0pt}\selectfont\mdseries\upshape \textbf{Coda:} [294?]ks, [294?]ts, kst} &
{\fontsize{10pt}{12.0pt}\selectfont\mdseries\upshape \textbf{Onset:} shnj khnj } &
\centering\arraybslash —\\\hline
{\bfseries Yakima Sahaptin} &
{\fontsize{10pt}{12.0pt}\selectfont\mdseries\upshape \textbf{Onset: }p[283?]$\chi $, tk[2B7?]s, q’[283?]p}

{\fontsize{10pt}{12.0pt}\selectfont\mdseries\upshape \textbf{Coda:} tks, stk, pt[361?][26C?]’k} &
{\fontsize{10pt}{12.0pt}\selectfont\mdseries\upshape \textbf{Onset: }[283?]t$\chi $n, ksks }

{\fontsize{10pt}{12.0pt}\selectfont\mdseries\upshape \textbf{Coda:} wtk[2B7?][283?], wq’$\chi [283?]$, jlps} &
\centering\arraybslash —\\\hline
{\bfseries Tohono O’odham} &
{\fontsize{10pt}{12.0pt}\selectfont\mdseries\upshape \textbf{Coda:} [261?][282?]p, tpk, bst[361?][283?], psk} &
{\fontsize{10pt}{12.0pt}\selectfont\mdseries\upshape \textbf{Onset: }nd[282?][294?]}

{\fontsize{10pt}{12.0pt}\selectfont\mdseries\upshape \textbf{Coda: }[283?]t[361?][283?]kt[361?][283?], t[361?][283?]spk, [261?][282?]sp} &
\centering\arraybslash —\\\hline
{\bfseries Polish} &
{\fontsize{10pt}{12.0pt}\selectfont\mdseries\upshape \textbf{Onset:} p[283?]t, x[283?]t, tkf[2B2?]}

{\fontsize{10pt}{12.0pt}\selectfont\mdseries\upshape \textbf{Coda:} psk, stf, [283?]t[361?][283?]p} &
{\fontsize{10pt}{12.0pt}\selectfont\mdseries\upshape \textbf{Onset:} pst[283?], fks[283?], vz[261?]l}

{\fontsize{10pt}{12.0pt}\selectfont\mdseries\upshape \textbf{Coda:} [272?]stf, tstf, rstf, pstf} &
{\fontsize{10pt}{12.0pt}\selectfont\mdseries\upshape \textbf{Onset: }spstr}

{\fontsize{10pt}{12.0pt}\selectfont\mdseries\upshape \textbf{Coda: }mpstf}\\\hline
\end{supertabular}
\end{flushleft}
\begin{styleBody}
\textbf{Table 3.12. }Representative sample of Highly Complex patterns occurring in data. (—) indicates that there are no reported patterns of this kind in the given language. The Yine patterns are in parentheses because they are representative triconsonantal clusters for the language but do not contain three obstruents (see discussion in §3.2.3).
\end{styleBody}

\begin{flushleft}
\tablefirsthead{}
\tablehead{}
\tabletail{}
\tablelasttail{}
\begin{supertabular}{|m{1.0844599in}|m{1.6990598in}|m{1.6983598in}|m{1.6997598in}|}
\hline
 &
\multicolumn{3}{m{5.2546597in}|}{\centering{\bfseries Highly Complex structures}}\\\hline
{\bfseries Language} &
\centering{\bfseries 3-obstruent structures} &
\centering{\bfseries 4-C structures} &
\centering\arraybslash{\bfseries 5-C and larger structures}\\\hline
{\bfseries Thompson} &
{\fontsize{10pt}{12.0pt}\selectfont\mdseries\upshape \textbf{Onset:} spt, st[361?]s’k}

{\fontsize{10pt}{12.0pt}\selectfont\mdseries\upshape \textbf{Coda:} x[2B7?]kt, x[2B7?]st[361?]s, pst[361?]s} &
{\fontsize{10pt}{12.0pt}\selectfont\mdseries\upshape \textbf{Coda:} t[361?]sxst[361?]s, jxst[361?]s, [26C?]kst} &
{\fontsize{10pt}{12.0pt}\selectfont\mdseries\upshape \textbf{Coda:} [26C?]qsxtx[2B7?]}\\\hline
{\bfseries Itelmen} &
{\fontsize{10pt}{12.0pt}\selectfont\mdseries\upshape \textbf{Onset:} kth kp'k' [26C?]qz}

{\fontsize{10pt}{12.0pt}\selectfont\mdseries\upshape \textbf{Coda: }p[26C?]h sht} &
{\fontsize{10pt}{12.0pt}\selectfont\mdseries\upshape \textbf{Onset:} ttxn, ksxw, ktxl}

{\fontsize{10pt}{12.0pt}\selectfont\mdseries\upshape \textbf{Coda:} nt[361?][283?]px, mp[26C?]x, [26C?]txt[361?][283?]} &
{\fontsize{10pt}{12.0pt}\selectfont\mdseries\upshape \textbf{Onset:} kp[26C?]kn, tksxqz, kstk’[26C?]kn}

{\fontsize{10pt}{12.0pt}\selectfont\mdseries\upshape \textbf{Coda:} nx[26C?]xt[361?][283?], mstxt[361?][283?]}\\\hline
{\bfseries Georgian} &
{\fontsize{10pt}{12.0pt}\selectfont\mdseries\upshape \textbf{Onset:} t'k'b p't[361?]s'k' psk’ } &
{\fontsize{10pt}{12.0pt}\selectfont\mdseries\upshape \textbf{Onset:} txz$\beta [31E?]$, t[361?]s’q’[27E?]t, brt[361?]s'q{}'}

{\fontsize{10pt}{12.0pt}\selectfont\mdseries\upshape \textbf{Coda:} [27E?]txl, [27E?]t'q'l, nt[361?][283?]xl} &
{\fontsize{10pt}{12.0pt}\selectfont\mdseries\upshape \textbf{Onset:} p’[27E?]t[361?]s’k’$\beta [31E?]$, [261?]$\beta [31E?]$p[27E?]t[361?]sk$\beta [31E?]$n}

{\fontsize{10pt}{12.0pt}\selectfont\mdseries\upshape \textbf{Coda:} nt[361?][283?]xls, [27E?]t[361?]s’q’$\beta [31E?]$s, [27E?]t'k'ls}\\\hline
{\bfseries Cocopa} &
{\fontsize{10pt}{12.0pt}\selectfont\mdseries\upshape \textbf{Onset:} sx[288?], psk[2B7?], xps}

{\fontsize{10pt}{12.0pt}\selectfont\mdseries\upshape \textbf{Coda:} qsk, [282?]sk, xsk} &
{\fontsize{10pt}{12.0pt}\selectfont\mdseries\upshape \textbf{Onset:} [282?]t[361?][283?]x[294?] p[282?]t[361?][283?][294?], p.t[361?][283?]x.m} &
{\fontsize{10pt}{12.0pt}\selectfont\mdseries\upshape \textbf{Word-initial:} pk.[283?]kw}\\\hline
{\bfseries Tashlhiyt} &
{\fontsize{10pt}{12.0pt}\selectfont\mdseries\upshape \textbf{Word-initial:} ts.t}

{\fontsize{10pt}{12.0pt}\selectfont\mdseries\upshape \textbf{Word-final:} k[2B7?]tt, [283?].kd} &
{\fontsize{10pt}{12.0pt}\selectfont\mdseries\upshape \textbf{Word-initial: }ts:$\chi $s}

{\fontsize{10pt}{12.0pt}\selectfont\mdseries\upshape \textbf{Word-final: }stst[2D0?]} &
{\fontsize{10pt}{12.0pt}\selectfont\mdseries\upshape (\textbf{Words without vowels:)}}

{\fontsize{10pt}{12.0pt}\selectfont\mdseries\upshape ts[2D0?]ft$\chi $t, tftktst[2D0?], ts[2D0?]k[283?]ftst[2D0?]}\\\hline
\end{supertabular}
\end{flushleft}
\begin{styleBody}
\textbf{Table 3.12. (cont.) }Representative sample of Highly Complex patterns occurring in data. (—) indicates that there are no reported patterns of this kind in the given language. The Yine patterns are in parentheses because they are representative triconsonantal clusters for the language but do not contain three obstruents (see discussion in §3.2.3).
\end{styleBody}

\begin{styleBody}
\ \ The languages in Table 3.12 are organized so as to highlight several coherent patterns in the data. In the first set of languages (Alamblak, Bench, Menya, Kabardian, Lezgian, Yine, Camsá, Semai, and Nuu-chah-nulth), Highly Complex patterns are limited to one syllable or word margin, usually the onset/initial context. The Highly Complex patterns in these languages are typically limited to triconsonantal clusters, though four-consonant clusters occur in Camsá, Semai, and Nuu-chah-nulth. In the second group of languages (Wutung, Doyayo, and Kunjen), four-consonant clusters occur at one syllable margin, but triconsonantal patterns falling under the definition of Highly Complex (that is, sequences of three obstruents) do not occur. In this group, the CCCC clusters include at least one, but usually two, sonorants. Finally, in the remaining 13 languages, Highly Complex patterns occur in both margins and almost always include clusters of various sizes.
\end{styleBody}

\begin{styleBody}
\ \ It is typically the case in the language sample that if a language has syllable margins of three obstruents, then any larger margins which occur in the language may also include sequences of three or more obstruents. The only apparent exceptions to this trend are four-consonant onsets in Albanian and Mohawk, and four-consonant codas in Georgian. In these cases, the larger clusters always include more than one sonorant, such that sequences of more than two obstruents do not occur, e.g. Georgian coda /[27E?]t’q’l/. In all other languages with both triconsonantal and larger Highly Complex structures, long strings of obstruents are a hallmark characteristic of the larger structures. That is, the patterns in the third group of languages described above are not simply an amalgamation of the patterns from the first and second groups of languages described above. The second group (Wutung, Doyoyo, and Kunjen) represents a minority pattern in that the only Highly Complex structures occurring in these languages do not involve strings of more than two obstruents.
\end{styleBody}

\begin{styleBody}
\ \ It should also be noted that languages with syllabic consonants do not behave as a group apart from the other languages with respect to the distribution of their Highly Complex sequences. Semai patterns with the first group of languages, while Tehuelche, Cocopa, and Tashlhiyt pattern with the third group.
\end{styleBody}

\begin{styleBody}
\ \ Table 3.12 does not provide an exhaustive list of Highly Complex structures for each language; however, for a few languages for which this is a minor pattern, an exhaustive or near-exhaustive list is given. This is the case for Alamblak and Menya. The Highly Complex onsets listed for these languages are not explicitly stated in the references to be the only structures of this sort, but a search of the examples and texts yielded only these patterns. In Bench and Wutung, the single onset given for each language is explicitly stated to be the only one occurring. For other languages, the lists given for larger structures may be exhaustive, but those given for smaller structures may be a tiny representative sample. This is the case for Polish, which has few onsets and codas of five consonants, but a much larger variety of smaller clusters than what is shown here. In the next section, I will discuss issues of the prevalence of Highly Complex syllable patterns in more detail.
\end{styleBody}

\section[3.4.2 Prevalence of Highly Complex syllable patterns within languages]{\rmfamily 3.4.2 Prevalence of Highly Complex syllable patterns within languages}
\begin{styleBody}
\ \ Here I attempt to characterize the prevalence of Highly Complex syllable patterns in the sample. First I examine restrictions on the combinations of consonants occurring in Highly Complex structures in each language. Then I present information on the relative frequency (either quantified or impressionistic) of these patterns as reported in the language descriptions. Together, these measures provide a rough diagnostic for the relative prevalence of the target syllable patterns within the Highly Complex languages of the sample.
\end{styleBody}

\begin{styleBody}
\ \ The analysis of restrictions on consonant combinations presented below is based primarily on the patterns of the smaller Highly Complex structures in each language. This is because the point here is to characterize the prevalence of Highly Complex patterns in general, and not just the maximal patterns occurring in each language. The analysis of restrictions on consonant combinations relies on patterns explicitly reported by the author. In some cases, no explicit description of consonant combinations is given, and I rely on patterns gleaned from the available examples. 
\end{styleBody}

\begin{styleBody}
\ \ For each language, I have classified the Highly Complex patterns which occur into three categories based on their combinatorial restrictions: Severely Restricted, Relatively Restricted, and Relatively Free. Where a language has Highly Complex structures in both margins and the patterns are qualitatively different, I examine the onset and coda separately. In (3.27)-(3.29) I give the definition for each category and illustrative examples from the data. The raw number of potential consonant combinations in a language is, of course, a function of the number of consonants in its phoneme inventory. I have attempted to define these categories so that they do not refer to or depend heavily upon the size of the consonant inventory of the given language.
\end{styleBody}

\begin{styleBody}
(3.27)\ \ \textbf{Severely Restricted: }Just a handful of ({\textless} 5) Highly Complex sequences occur, and/or every member of the sequence has specific restrictions.
\end{styleBody}

\begin{styleBody}
(a) \ \ \textbf{Wutung} (\textit{Sko}; Papua New Guinea)
\end{styleBody}

\begin{styleBody}\itshape
Restrictions on onsets of four consonants:
\end{styleBody}

\begin{styleBody}
Only /hmbl/ occurs.\footnote{The /h/ here appears to be a separate consonant segment and does not represent a modification of the phonation of the following nasal. Marmion (2010: 54) describes it as a segment which can optionally elide preceding sonorant consonants.}
\end{styleBody}

\begin{styleBody}
e.g.\ \ \textit{hmbli[25B?]}
\end{styleBody}

\begin{styleBody}
\ \ \ \ ‘left hand’
\end{styleBody}

\begin{styleBody}
(Marmion 2010: 69)
\end{styleBody}

\begin{styleBody}
(b) \ \ \textbf{Doyayo} (\textit{Atlantic-Congo}; Cameroon)
\end{styleBody}

\begin{styleBody}\itshape
Restrictions on codas of four consonants:
\end{styleBody}

\begin{styleBody}
C\textsubscript{1}: must be /b [261?] m ŋ/ (/b [261?]/ usually realized as [$\beta $ [263?]] in clusters)
\end{styleBody}

\begin{styleBody}
C\textsubscript{2}: must be /l [27E?] n/
\end{styleBody}

\begin{styleBody}
C\textsubscript{3}: must be /d t/
\end{styleBody}

\begin{styleBody}
C\textsubscript{4}: must be /s z/
\end{styleBody}

\begin{styleBody}
Additionally, C\textsubscript{3} and C\textsubscript{4} must match in voicing.
\end{styleBody}

\begin{styleBody}
e.g. \ \ \textit{de$\beta [27E?]$ts}
\end{styleBody}

\begin{styleBody}
\ \ \ \ ‘be cut off for’
\end{styleBody}

\begin{styleBody}
(Wiering \& Wiering 1995: 41-2)
\end{styleBody}

\begin{styleBody}
(3.28)\ \ \textbf{Relatively Restricted: }There are general restrictions on the voicing, place, or manner of some or all members, and/or specific restrictions on one or two (but not all) members.
\end{styleBody}

\begin{styleBody}
(a) \ \ \textbf{Lezgian} (\textit{Nakh-Daghestanian}; Russia, Azerbaijan)
\end{styleBody}

\begin{styleBody}\itshape
Restrictions on onsets of three consonants:
\end{styleBody}

\begin{styleBody}
C\textsubscript{1}: voiceless obstruent
\end{styleBody}

\begin{styleBody}
C\textsubscript{2}: voiceless obstruent or /r/
\end{styleBody}

\begin{styleBody}
C\textsubscript{3}: voiceless obstruent or sonorant
\end{styleBody}

\begin{styleBody}
e.g.\ \ \textit{k[2B0?]sta$\chi $}
\end{styleBody}

\begin{styleBody}
\ \ \ \ ‘spoiled child’
\end{styleBody}

\begin{styleBody}
(Haspelmath 1993: 37)
\end{styleBody}

\begin{styleBody}
(b) \ \ \textbf{Passamaquoddy-Maliseet} (\textit{Algic}; Canada, USA)
\end{styleBody}

\begin{styleBody}\itshape
Restrictions on onsets of three consonants:
\end{styleBody}

\begin{styleBody}
Apart from a few exceptions, triconsonantal onsets or codas are always of the form CsC.
\end{styleBody}

\begin{styleBody}
e.g.\ \ \textit{kspison}
\end{styleBody}

\begin{styleBody}
\ \ \ \ ‘belt’
\end{styleBody}

\begin{styleBody}
(LeSourd 1993: 121)
\end{styleBody}

\begin{styleBody}
(3.29)\ \ \textbf{Relatively Free: }There may be a few abstract restrictions on consonant combinations, and/or combinations are described by author as free or unrestricted.
\end{styleBody}

\begin{styleBody}
(a)\ \ \textbf{Yakima Sahaptin} (\textit{Sahaptian}; USA)
\end{styleBody}

\begin{styleBody}\itshape
Restrictions on codas of three and four consonants:
\end{styleBody}

\begin{styleBody}
Clusters of glottalized or labialized obstruents do not occur.
\end{styleBody}

\begin{styleBody}
e.g. \ \ \textit{$\chi [268?]$p$\chi $p\ \ \ \ \ \ \ \ tawq’$\chi [283?]$}
\end{styleBody}

\begin{styleBody}
\ \ \ \ ‘cottonwood’\ \ \ \ \ \ ‘kerchief, neck scarf’
\end{styleBody}

\begin{styleBody}
(Hargus \& Beavert 2002: 270-1)
\end{styleBody}

\begin{styleBody}
\ \ The distribution of the languages with respect to the three categories above is given in Table 3.13.
\end{styleBody}

\begin{flushleft}
\tablefirsthead{}
\tablehead{}
\tabletail{}
\tablelasttail{}
\begin{supertabular}{m{1.8726599in}m{1.9018599in}m{1.6740599in}}
\hline
{\bfseries Severely Restricted} &
{\bfseries Relatively Restricted } &
{\bfseries Relatively Free}\\\hline
Alamblak

Bench

Doyayo

Kunjen

Menya

{\fontsize{10pt}{12.0pt}\selectfont\mdseries\upshape Qawasqar \textit{(codas)}}

Wutung &
Albanian

Camsá

Georgian

Kabardian

Lezgian

Mohawk

Passamaquoddy-Maliseet

{\fontsize{10pt}{12.0pt}\selectfont\mdseries\upshape Polish\textit{ (codas)}}

{\fontsize{10pt}{12.0pt}\selectfont\mdseries\upshape Qawasqar \textit{(onsets)}}

Semai

Tehuelche

Tohono O’odham &
Cocopa

Itelmen

Nuu-chah-nulth

{\fontsize{10pt}{12.0pt}\selectfont\mdseries\upshape Polish \textit{(onsets)}}

Tashlhiyt

Thompson

Yakima Sahaptin

Yine\\\hline
\end{supertabular}
\end{flushleft}
\begin{styleBody}
\textbf{Table 3.13.} Degree of restriction on consonant combinations in Highly Complex syllable patterns.
\end{styleBody}

\begin{styleBody}
\ \ There are two languages — Polish and Qawasqar — which have different degrees of restriction in their Highly Complex onset and coda patterns. Besides Qawasqar, there are six languages for which all Highly Complex patterns are severely restricted. Interestingly, in only one of these (Doyayo) are the severely restricted patterns associated with specific morphologically complex sequences; in the others, they occur within morphemes. Most often, languages have Highly Complex structures that are relatively restricted in their consonant combinations. Besides Polish and Qawasqar, there are ten languages which have this pattern. Finally, there are seven languages besides Polish which have relatively free consonant combinations in their Highly Complex structures. It is striking that the set of languages with relatively free patterns is similar in size to the set of languages with severely restricted patterns, given the general rarity of languages with Highly Complex syllable structure.
\end{styleBody}

\begin{styleBody}
\ \ Below I present information on the frequency of Highly Complex structures in the languages of the sample. Frequency of syllable patterns is explicitly remarked upon for only 16 of the 25 languages in this category. Most often, reports are impressionistic in nature, but occasionally a researcher provides type frequency data for patterns in the syllable inventory, lexicon, or text. In Table 3.14 I note the nature of the frequency data given for each language. Note that not all of the patterns reported below are strictly Highly Complex patterns; authors often did not make a distinction between different kinds of triconsonantal clusters, for instance.
\end{styleBody}

\begin{flushleft}
\tablefirsthead{}
\tablehead{}
\tabletail{}
\tablelasttail{}
\begin{supertabular}{m{0.77125984in}m{1.0900599in}m{4.32406in}}
\hline
{\bfseries Language} &
{\bfseries Nature of frequency data} &
{\bfseries Reported frequency of HC patterns}\\\hline
{\bfseries Bench} &
{\itshape Impressionistic} &
{\fontsize{10pt}{12.0pt}\selectfont\mdseries\upshape “Syllable patterns ending in [CCC] have thus \textbf{\textit{a very limited actual occurrence}}” (Rapold 2006: 92)}\\\hline
{\bfseries Camsá} &
{\itshape Impressionistic} &
{\fontsize{10pt}{12.0pt}\selectfont\mdseries\upshape “Consonant clusters are very common in Camsá.\textbf{\textit{ }}[…] \textbf{\textit{Clusters of three consonants are not as common }}in the language as clusters of two.” (Howard 1967: 81-4)}\\\hline
{\bfseries Cocopa} &
{\itshape Impressionistic} &
{\fontsize{10pt}{12.0pt}\selectfont\mdseries\upshape “[i]t is \textbf{\textit{quite common}} to find Cocopa words consisting of a single vowel preceded by several consonants.” (Bendixen 1980: 1)}\\\hline
{\bfseries Georgian} &
{\itshape Type frequency in syllable inventory}

{\itshape Type frequency in text} &
\textbf{\textit{28/276 (10\%) }}of onset patterns occurring stem-initially are HC (calculated from data provided in Butskhrikidze 2002: 197-205).

In an excerpt of descriptive prose, \textbf{\textit{24/550 (4.4\%)}} of word-initial patterns and \textbf{\textit{7/559 (1.3\%)}} of word-final patterns consist of three or more consonants (Vogt 1958: 79-80).\\\hline
{\bfseries Kabardian} &
{\itshape Impressionistic} &
“Clusters consist of not more than three, and in the large majority of cases, of two consonants.” (Kuipers 1960: 29)\\\hline
{\bfseries Kunjen} &
{\itshape Type frequency in lexicon} &
{\fontsize{10pt}{12.0pt}\selectfont\mdseries\upshape “VCCCC syllables occur only as the initial syllable of the word, and have been recorded in \textbf{\textit{only twenty words}}.” (Sommer 1969: 35)}\\\hline
{\bfseries Itelmen} &
{\itshape Impressionistic} &
{\fontsize{10pt}{12.0pt}\selectfont\mdseries\upshape “The \textbf{\textit{frequent occurrence}} \textbf{\textit{of complex consonant clusters}}\textbf{ }is one of the most notable traits of Itelmen phonology.” (Georg \& Volodin 1999: 38; translation TZ)}\\\hline
{\bfseries Lezgian} &
{\itshape Impressionistic} &
{\fontsize{10pt}{12.0pt}\selectfont\mdseries\upshape “[w]ord-initial CC- and even \textbf{\textit{CCC- clusters are now common.}}” }

(Haspelmath 1993: 46)\\\hline
\end{supertabular}
\end{flushleft}
\begin{styleBody}
\textbf{Table 3.14.} Reported frequency of Highly Complex syllable patterns. Emphasis my own in all quotations.
\end{styleBody}

\begin{flushleft}
\tablefirsthead{}
\tablehead{}
\tabletail{}
\tablelasttail{}
\begin{supertabular}{m{0.77125984in}m{1.0900599in}m{4.32406in}}
\hline
{\bfseries Language} &
{\bfseries Nature of frequency data} &
{\bfseries Reported frequency of HC patterns}\\\hline
{\bfseries Mohawk} &
{\itshape Type frequency in syllable inventory} &
{\fontsize{10pt}{12.0pt}\selectfont\mdseries\upshape \textbf{6/43 }\textbf{\textit{(14\%)}} of word-initial onsets and\textbf{ 3/25 }\textbf{\textit{(12\%)}} of word-final codas are HC (calculated from data provided in Michelson 1988: 12-13).}\\\hline
{\bfseries Polish} &
{\itshape Type frequency in syllable inventory} &
{\fontsize{10pt}{12.0pt}\selectfont\mdseries\upshape \textbf{\textit{64/426 (15\%)}} of onset patterns occurring word-initially are HC,\textbf{\textit{ 18/141 (13\%)}} of coda patterns occurring word-finally are HC (calculated from data provided in Bargiełowna 1950).}\\\hline
{\bfseries Tashlhiyt} &
{\itshape Type frequency in text} &
{\fontsize{10pt}{12.0pt}\selectfont\mdseries\upshape \textbf{\textit{451/5700 (7.9\%)}} of syntactic words in running text are composed of voiceless obstruents only (Ridouane 2008: 328f).}\\\hline
{\bfseries Thompson} &
{\itshape Impressionistic} &
{\fontsize{10pt}{12.0pt}\selectfont\mdseries\upshape “Sequences of six obstruents are\textbf{\textit{ not uncommon}}.” (Thompson \& Thompson 1992: 25)}\\\hline
{\bfseries Tohono O’odham} &
{\itshape Impressionistic} &
{\fontsize{10pt}{12.0pt}\selectfont\mdseries\upshape Morphological and phonological processes “yield \textbf{\textit{a high frequency}} of complex moras and very intricate syllables” (Hill \& Zepeda 1992: 355)}\\\hline
{\bfseries Wutung} &
{\itshape Type frequency in syllable inventory} &
{\fontsize{10pt}{12.0pt}\selectfont\mdseries\upshape \textbf{\textit{1/40 (2.5\%) }}of\textbf{\textit{ }}onset patterns are HC (calculated from data provided in Marmion 2010).}\\\hline
{\bfseries Yakima Sahaptin} &
{\itshape Type frequency in lexicon} &
{\fontsize{10pt}{12.0pt}\selectfont\mdseries\upshape \textbf{\textit{13/295 (4.4\%)}}\textit{ }of underived nouns and adjectives have onsets of three or four Cs, \textbf{\textit{8/295 (3\%)}} have codas of three or four Cs (calculated from data provided in Hargus \& Beavert 2006).}\\\hline
{\bfseries Yine} &
{\itshape Type frequency in lexicon}

{\itshape Impressionistic} &
{\fontsize{10pt}{12.0pt}\selectfont\mdseries\upshape {\textquotedbl}A little less than one-third of the total number of syllable margins consists of C\textsuperscript{2}C\textsuperscript{1}; \textbf{\textit{not more than one in several hundred, of C}}\textbf{\textit{\textsuperscript{3}}}\textbf{\textit{C}}\textbf{\textit{\textsuperscript{2}}}\textbf{\textit{C}}\textbf{\textit{\textsuperscript{1}}}\textbf{.} The present count of clusters of three consonants shows \textbf{\textit{lower frequency}} than a similar count made ten years ago.” (Matteson 1965: 24)}

“Words beginning with three consonants in sequence are \textbf{\textit{very common}}.” (Hanson 2010: 26)\\\hline
\end{supertabular}
\end{flushleft}
\begin{styleBody}
\textbf{Table 3.14. (cont.)} Reported frequency of Highly Complex syllable patterns. Emphasis my own in all quotations.
\end{styleBody}

\begin{styleBody}
\ \ Comparing the relative frequency patterns in Table 3.14 to the combinatorial restriction patterns in Table 3.13, we find some correspondences between patterns which are not all that surprising. For example, it follows that Wutung, whose Highly Complex syllable patterns are restricted to a single four-consonant onset (3.27a), would also have a very low type frequency of this pattern in its syllable inventory. Similarly, it is expected that Georgian and Polish, both of which have larger clusters and fewer restrictions on consonant combinations, should have a higher type frequency of these patterns in their syllable inventories.\footnote{Mohawk presents an unexpected pattern, in that its cluster patterns are relatively restricted but it has a type frequency of Highly Complex clusters which is on par with that of Georgian and Polish. This is due to the very small consonant phoneme inventory of the language (ten consonants), which limits the overall size of the syllable inventory.} The other kinds of frequency data — type frequency in the lexicon and in running text — also show this correspondence, with higher frequencies typically corresponding to languages with freer consonant combinations in their Highly Complex patterns. It should also be noted that frequency patterns are reported for all but one language with relatively free consonant combinations (Nuu-chah-nulth). Though quantitative type frequency data isn’t given for Cocopa, Itelmen, or Thompson, the authors make a point of mentioning the high frequency and commonplace nature of Highly Complex structures in these languages. 
\end{styleBody}

\begin{styleBody}
\ \ Combining the results of the analyses in this section and §3.4.1, we can identify two extreme patterns in the prevalence of Highly Complex patterns in the data. On one extreme, there is a group of languages for which Highly Complex structures are a minor pattern. These languages have Highly Complex structures at only one syllable/word margin. The structures consist of three or maximally four consonants which are severely restricted in their combination, and have relatively low type frequencies (Table 3.15). On the other extreme, there is a group of languages for which Highly Complex structures are a prevalent pattern. These languages have Highly Complex structures at both syllable/word margins. The structures may be more than four consonants in length, are relatively free in their combination, and have relatively high type frequencies (Table 3.16).
\end{styleBody}

\begin{flushleft}
\tablefirsthead{}
\tablehead{}
\tabletail{}
\tablelasttail{}
\begin{supertabular}{m{1.2677599in}m{1.6330599in}m{1.9872599in}}
\hline
{\bfseries Language} &
{\bfseries Family} &
{\bfseries Region}\\\hline
Alamblak &
{\itshape Sepik} &
{\scshape Australia \& New Guinea}\\
Bench &
{\itshape Ta-Ne-Omotic} &
{\scshape Africa}\\
Doyayo &
{\itshape Atlantic-Congo} &
{\scshape Africa}\\
Kunjen &
{\itshape Pama-Nyungan} &
{\scshape Australia \& New Guinea}\\
Menya &
{\itshape Angan} &
{\scshape Australia \& New Guinea}\\
Wutung &
{\itshape Sko} &
{\scshape Australia \& New Guinea}\\\hline
\end{supertabular}
\end{flushleft}
\begin{styleBody}
\textbf{Table 3.15.} Languages with minor Highly Complex patterns.
\end{styleBody}

\begin{flushleft}
\tablefirsthead{}
\tablehead{}
\tabletail{}
\tablelasttail{}
\begin{supertabular}{m{1.2677599in}m{1.6330599in}m{1.9872599in}}
\hline
{\bfseries Language} &
{\bfseries Family} &
{\bfseries Region}\\\hline
Cocopa &
{\itshape Cochimi-Yuman} &
{\scshape North America}\\
Georgian &
{\itshape Kartvelian} &
{\scshape Eurasia}\\
Itelmen &
{\itshape Chukotko-Kamchatkan} &
{\scshape Eurasia}\\
Polish &
{\itshape Indo-European} &
{\scshape Eurasia}\\
Tashlhiyt &
{\itshape Afro-Asiatic} &
{\scshape Africa}\\
Thompson &
{\itshape Salishan} &
{\scshape North America}\\
Tohono O’odham &
{\itshape Uto-Aztecan} &
{\scshape North America}\\
Yakima Sahaptin &
{\itshape Sahaptian} &
{\scshape North America}\\\hline
\end{supertabular}
\end{flushleft}
\begin{styleBody}
\textbf{Table 3.16.} Languages with prevalent Highly Complex patterns.
\end{styleBody}

\begin{styleBody}
\ \ Over half of the languages in the Highly Complex portion of the sample have syllable patterns which are at one of these extremes. There are different areal distributions for the two groups of languages. The languages with Highly Complex syllable structure as a very minor pattern include all those from the Australia \& New Guinea macro-area, as well as two languages from Africa. The languages with prevalent Highly Complex patterns are spoken in parts of Eurasia, North America, and the Atlas Mountain region of Africa; i.e., regions identified in Chapter 1 as being well-known for their complex syllable patterns. I will return to discussion of these patterns in §3.5.
\end{styleBody}

\clearpage\section[3.4.3 Acoustic and perceptual characteristics]{\rmfamily 3.4.3 Acoustic and perceptual characteristics}
\begin{styleBody}
\ \ Researchers often remark upon the phonetic characteristics of the long tautosyllabic clusters of obstruents which are characteristic of most languages with Highly Complex syllable structure. Descriptions typically note the presence of salient release or aspiration of stops, transitional vocalic elements between consonants at different places or with different manners of articulation, and lengthened consonant articulation for syllabic obstruents. These descriptions are relevant in the establishment of Highly Complex syllable structure as a language type which may have specific acoustic characteristics in addition to abstract phonological characteristics. It is also possible that clues to the development of Highly Complex syllable structure may be found in the acoustic and perceptual properties of these clusters. For example, it has been found that clusters resulting from historically recent processes of vowel syncope may retain traces of the previous vowel in the transitions between consonants (cf. Chitoran \& Babaliyeva 2007 for Lezgian).
\end{styleBody}

\begin{styleBody}
\ \ Descriptions of the acoustic and perceptual characteristics are available for 18/25 of the languages in the Highly Complex portion of the sample. This is somewhat remarkable, given that many of the languages are underdescribed, and that such detailed phonetic descriptions of consonant clusters are not a standard topic for inclusion in language references. These descriptions are presented in Table 3.17.
\end{styleBody}

\begin{flushleft}
\tablefirsthead{}
\tablehead{}
\tabletail{}
\tablelasttail{}
\begin{supertabular}{m{1.1893599in}m{5.14976in}}
\hline
{\bfseries Language} &
\centering\arraybslash{\bfseries Description of phonetic realization of consonant clusters}\\\hline
{\bfseries Alamblak} &
{\fontsize{10pt}{12.0pt}\selectfont\mdseries\upshape “Open transition,” transcribed as [[268?]], varies freely with release in obstruent and other sequences (Bruce 1984: 56-9).}\\\hline
{\bfseries Albanian} &
{\fontsize{10pt}{12.0pt}\selectfont\mdseries\upshape Release between obstruents varies freely with much rarer epenthetic [[259?]] in slow or careful speech (Klippenstein 2010: 24-6).}\\\hline
{\bfseries Camsá} &
{\fontsize{10pt}{12.0pt}\selectfont\mdseries\upshape “Nonphonemic transitional vocoid [[259?]]” occurs between stops or consonant plus nasal at different points of articulation; initial fricatives are lengthened and may have voiceless or voiced off-glide, transcribed as [\textsuperscript{u}] or [\textsuperscript{[259?]}], before a non-fricative consonant at a different place of articulation (Howard 1967: 81).}\\\hline
{\bfseries *Cocopa} &
{\fontsize{10pt}{12.0pt}\selectfont\mdseries\upshape Consonants in some sequences separated by “anaptyctic phonetic vowel” or “indistinct ‘murmur’ vowel” whose quality, transcribed [\textsuperscript{i}], [\textsuperscript{a}], or [\textsuperscript{u}], is determined by surrounding consonants (Crawford 1966: 37-45).}\\\hline
{\bfseries Georgian} &
{\fontsize{10pt}{12.0pt}\selectfont\mdseries\upshape Stops in sequences nearly always released, sometimes with voicing if both are voiced; voiceless stops and affricates have strongly aspirated release; length of interval between C\textsubscript{1} and C\textsubscript{2} release depends on relative place of articulation of the consonants (Chitoran 1999).}\\\hline
{\bfseries Itelmen} &
{\fontsize{10pt}{12.0pt}\selectfont\mdseries\upshape Indeterminant “overtone” transcribed as [[259?]] and described as “extremely short, with an overtone indeterminant in timbre,” occurs in words without vowels and certain consonant combinations (Volodin 1976: 40-1; translation SME).}\\\hline
{\bfseries Kunjen} &
“Brief transitional vocoids” may sometimes be heard between consonants in a cluster. (Sommer 1969: 33)\\\hline
{\bfseries Lezgian} &
Before a voiceless stop or fricative, voiceless stops are always aspirated (Haspelmath 1993: 47); in clusters resulting from historical or synchronic syncope, traces of previous vowel remain audible in stop release and fricative noise (Chitoran \& Babaliyeva 2007).\\\hline
{\bfseries Menya} &
“Non-homorganic consonants are phonetically separated by extremely short vocalic segments which are more and more not being written”; quality of short segments is conditioned by surrounding consonants and vowels (Whitehead 2004: 9, 226).\\\hline
{\bfseries Mohawk} &
Stops are “strongly aspirated” before another (non-identical) consonant (Bonvillain 1973: 28)\\\hline
{\bfseries Nuu-chah-nulth} &
{\fontsize{10pt}{12.0pt}\selectfont\mdseries\upshape The first stop or affricate of a like sequence has “a release typical for such consonants” (Kim 2003: 163-4). Epenthetic [[26A?]] occurs between a nasal and back stop or affricate (Rose 1981: 26-7). Voiceless plain stops are aspirated when they appear in syllable coda clusters (Davidson 2002: 12).}\\\hline
\end{supertabular}
\end{flushleft}
\begin{styleBody}
\textbf{Table 3.17.} Descriptions of acoustic and perceptual characteristics of clusters in languages with Highly Complex syllable structure. Languages omitted due to lack of description are Bench, Doyayo, Kabardian, Passamaquoddy-Maliseet, Polish, Qawasqar, and Wutung. * indicates that reported pattern is for syllables with obstruent nuclei.
\end{styleBody}

\begin{flushleft}
\tablefirsthead{}
\tablehead{}
\tabletail{}
\tablelasttail{}
\begin{supertabular}{m{1.1893599in}m{5.14976in}}
\hline
{\bfseries Language} &
\centering\arraybslash{\bfseries Description of phonetic realization of consonant clusters}\\\hline
{\bfseries *Semai} &
{\fontsize{10pt}{12.0pt}\selectfont\mdseries\upshape Minor syllables consisting of consonants are “clearly heard and perceived as distinct syllables.” (Sloan 1988: 321). Vocalic element in consonantal minor syllable “usually a very short, non-phonemic, epenthetic [[259?]]”, but can vary in quality, and is “optional if the two consonants are easily pronounced without the epenthetic vowel.” (Philips 2007: 2)}\\\hline
{\bfseries *Tashlhiyt} &
{\fontsize{10pt}{12.0pt}\selectfont\mdseries\upshape Short “voiced transitional vocoids” whose quality is predictable by surrounding vowels split consonant sequences when one is voiced (Dell \& Elmedlaoui 2002: 16); Gordon \& Nafi report this for occasional sequences of voiceless consonants (2012: 16). Ridouane (2008: 210) reports that “stop release is obligatory before another stop which is not homorganic with it” and Grice et al. (2015) find that the “vocoid” is not entirely predictable from the voicing properties of surrounding Cs and that its presence is partly conditioned by intonational prominence.}\\\hline
{\bfseries *Tehuelche} &
{\fontsize{10pt}{12.0pt}\selectfont\mdseries\upshape The “accumulation of consonants is made possible by the development of […] \textit{supporting vowels.}” These have “a neutral vowel quality which play the role of lubricator and which corresponds to the neutralization of all other vowels,” and are transcribed as [[259?]] or [[28A?]] depending upon consonantal environment (Fernández Garay \& Hernández 2006: 13; Fernández Garay 1998; translation RNS)}\\\hline
{\bfseries Thompson} &
Plain stops are “somewhat aspirated” before another stop and often before spirants, and strongly aspirated syllable-finally (Thompson \& Thompson 1992: 4). “Laryngeals are usually separated from preceding obstruents by a brief central vowel” whose precise quality is determined by the consonantal environment (1992: 44).\\\hline
{\bfseries Tohono O’odham} &
{\fontsize{10pt}{12.0pt}\selectfont\mdseries\upshape Surface clusters resulting from historical vowel deletion have “very short, voiceless elements”, phonetically transcribed as [\textsuperscript{h}] but which may retain previous vowel quality coloration in the case of high vowels (Hill \& Zepeda 1992: 356). Combinations of voiceless stops “might be considered as separated by a voiceless epenthetic.” (Mason 1950: 81f).}\\\hline
{\bfseries Yakima Sahaptin} &
{\fontsize{10pt}{12.0pt}\selectfont\mdseries\upshape Excrescent [\textsuperscript{[268?]}] is possible in some consonant combinations, such as when a fricative precedes two stops (Hargus \& Beavert 2002); aspiration accompanying voiceless stops has “formant structure that may superficially resemble” that of [[268?]] (2002: 273-4f).}\\\hline
{\bfseries Yine} &
{\fontsize{10pt}{12.0pt}\selectfont\mdseries\upshape “A very salient feature of Yine consonant clusters is the prevalence of an audible interval between the release of the first consonant (C\textsubscript{1}) and the closure of the second consonant (C\textsubscript{2}).” This “intra-cluster release” varies in duration, quality, and voicing, and is never obligatory (Hanson 2010: 28-9). Matteson \& Pike (1958) describe the properties of these “non-phonemic transition vocoids” at length.}\\\hline
\end{supertabular}
\end{flushleft}
\begin{styleBody}
\textbf{Table 3.17. (cont.)} Descriptions of acoustic and perceptual characteristics of clusters in languages with Highly Complex syllable structure. Languages omitted due to lack of description are Bench, Doyayo, Kabardian, Passamaquoddy-Maliseet, Polish, Qawasqar, and Wutung. * indicates that reported pattern is for syllables with obstruent nuclei.
\end{styleBody}

\begin{styleBody}
\ \ Even though many of the descriptions make mention of ‘epenthetic’ vowels, the patterns described above are consistent with those features listed by Hall (2006) as being associated with intrusive vowels. The transitional elements in these clusters are characterized by neutral vowel qualities that may be heavily influenced by surrounding consonants, and may vary in their duration and voicing. In some cases the transitions are described as occurring between specific combinations of consonants with different places of articulation.
\end{styleBody}

\begin{styleBody}
\ \ Most of the references consulted for the above analysis were written by researchers who are not native speakers, and for whom the different timing patterns in consonant sequences in these languages may be especially salient.\footnote{It is interesting to note that for Polish, which has a wealth of available descriptive material written in English by native speakers of Polish, I could find few details on the phonetic characteristics of clusters.} As mentioned in §3.2.2, native speakers are often unaware of the presence of these transitional elements, and when they are aware of them, view them as optional. Menya provides an interesting illustration of this in its writing conventions. The short vocalic elements between non-homorganic consonants in the language are written only sporadically by literate native speakers, and when they are written, there is unsystematic variation in the grapheme used (Whitehead 2004: 9, 226; the quote given in Table 3.17 may also be suggestive of a recent process of vowel reduction). Another piece of evidence for determining the intrusive nature of a vowel is in its ‘invisibility’ to phonological processes. In some cases, explicit descriptions of this are given. For example, the vocalic element transcribed as [[26A?]] that occurs between nasals and back stops or affricates in Nuu-chah-nulth is explicitly described as not being included in the syllable count which determines a vowel lengthening pattern in the language (Rose 1981: 27).
\end{styleBody}

\begin{styleBody}
\ \ The striking similarities in the phonetic descriptions of Highly Complex structures in the 18 languages in Table 3.21 indicate that the languages of this category share more in common than just phonological structure. The presence of transitional elements is a prominent phonetic characteristic of Highly Complex syllable structure. It is also notable that the phonetic descriptions of syllabic obstruents in Cocopa, Semai, Tashlhiyt, and Tehuelche are virtually indistinguishable from the phonetic descriptions of tautosyllabic clusters in the other languages. This recalls Hall’s observation that vowel intrusion and syllabic consonants are motivated by similar processes of gestural overlap, and is further justification for grouping these languages together with the others.
\end{styleBody}

\section[3.5. Discussion]{\rmfamily 3.5. Discussion}
\begin{styleBody}
\ \ As mentioned in §3.1, the studies in this chapter serve two purposes: first, to provide a baseline characterization of syllable patterns in the language sample as a whole; and second, to elucidate in greater detail the specific patterns occurring in languages with Highly Complex syllable structure.
\end{styleBody}

\begin{styleBody}
\ \ In Table 3.18 I summarize the results from §3.3 regarding syllable patterns in the language sample as a whole, and describe how the findings relate to syllable structure complexity. An asterisk (*) indicates that the pattern was found to be statistically significant.
\end{styleBody}

\begin{flushleft}
\tablefirsthead{}
\tablehead{}
\tabletail{}
\tablelasttail{}
\begin{supertabular}{m{2.7129598in}m{3.62616in}}
\hline
{\bfseries Aspect of syllable structure} &
{\bfseries Finding}\\\hline
{\fontsize{10pt}{12.0pt}\selectfont\mdseries\upshape \textbf{1.} \textit{Relationship between maximal onset and coda complexity (§3.3.2)}} &
A large cluster at one margin typically implies a large cluster at the other margin.\\\hline
\textbf{2. *}\textit{Obligatoriness of syllable margins (§3.3.3)} &
Least common in S lgs, most common in HC languages.\\\hline
{\fontsize{10pt}{12.0pt}\selectfont\mdseries\upshape \textbf{3.} \textit{Complex vocalic nuclei (§3.3.4)}} &
Much less likely to occur in S languages.\\\hline
\textbf{4. *}\textit{Presence of syllabic consonants (§3.3.5)} &
Least common in S lgs, most common in HC languages.\\\hline
\textbf{5. }\textit{Morphological constituency patterns (§3.3.6)}

\ \ \ \ \ \textbf{a. *}\textit{of maximal syllable margins}

\textit{\ \ \ \ \ }\textbf{b.} \textit{of syllabic consonants} &
Morphologically complex patterns increase with syllable structure complexity.

More likely to be found in grammatical items as syllable structure increases.\\\hline
\end{supertabular}
\end{flushleft}
\begin{styleBody}
\textbf{Table 3.18. }Summary of findings regarding syllable patterns in language sample as a whole.
\end{styleBody}

\begin{styleBody}
\ \ Some of the analyses presented in §3.3, corresponding to the findings in lines 1-3 of Table 3.18, were exploratory and conducted without any underlying hypotheses. Nevertheless these analyses yielded relevant results with respect to syllable structure complexity.
\end{styleBody}

\begin{styleBody}
\ \ The relationship between maximal onset and coda complexity, in which the presence of large (four or more consonants) sequences at one syllable margin in a language typically implies the presence of large sequences at the other margin, is especially interesting. This is not an expected pattern in terms of probabilistic distribution: if onsets and codas are independent structures, then we would expect to observe a wider range of combinations in maximal onset and coda sizes. However, if syllable structure is viewed not as an entity with abstract phonological motivations, but as a phenomenon reflecting articulatory routines carried out over many generations in the history of a language, this pattern may not be surprising. It is reasonable to imagine, for instance, a scenario in which a strong tendency toward vowel reduction in a language with prefixation and suffixation on stress-carrying stems might result in the eventual complete gestural overlap and deletion of many or most unstressed vowels, yielding long clusters of consonants in both word-marginal contexts. This issue will be discussed further in Chapters 5, 6, and 8.
\end{styleBody}

\begin{styleBody}
\ \ Similarly, the pattern observed with respect to obligatory syllable margins and syllable structure complexity is not necessarily expected. In languages with Simple syllable structure, obligatory onsets significantly limit the size of the syllable inventory. However, this effect on the syllable inventory would be much smaller in languages in the other syllable structure complexity categories. It does not necessarily follow that the Highly Complex category should have a higher rate of obligatory margins than the other categories. Three of the languages with obligatory onsets in the Highly Complex category (Thompson, Tohono O’odham, and Yakima Sahaptin), as well as one language in this group which likely had obligatory onsets historically (Itelmen), also happen to be in the group of languages whose Highly Complex patterns are most prevalent (§3.4.2). Placed in the context of languages in which large consonant clusters are prevalent and the consonant to vowel ratio in speech is presumably higher than average, perhaps the high rate of obligatory syllable margins is not surprising for this category.
\end{styleBody}

\begin{styleBody}
\ \ Greater syllable structure complexity is also associated with a greater likelihood of a language having complex vocalic nuclei (long vowels, diphthongs, or vowel sequences) and/or syllabic consonants. The specific combinatorial restrictions between syllable margins and nuclei are not explored or quantified here. However, the general effect of these patterns could be that as syllable structure complexity increases, so does the range of possible syllable types, and not just as a result of increased consonant combinations. The diversity in syllable margins which provide the basis for the definitions of syllable structure complexity may be accompanied by greater diversity in syllable nuclei as syllable structure complexity increases. This issue will be explored in greater depth in §4.3.
\end{styleBody}

\begin{styleBody}
\ \ The data here confirms the hypothesis that languages with more complex syllable structure are more likely to have syllabic consonants. This pattern is strongly present even when syllabic obstruents are not considered. These results are not in line with the predictions of Isačenko (1939/1940), who observed that ‘vocalic’ languages — a term defined in part by a low consonant to vowel ratio and simpler syllable patterns — are more likely to develop syllabic sonorants. Isačenko’s study is limited to Slavic languages, so it is possible that this group of languages presents an exception to the crosslinguistic patterns. However, the findings of the current study should be considered in the context of Isačenko’s larger point that ‘vocalic’ and ‘consonantal’ languages are characterized not only by different segmental inventory and syllable patterns, but also by different diachronic and synchronic processes of language change. It is known that syllabic consonants often come about through vowel reduction (Bell 1978a), and indeed we observe a set of variable syllabic consonants in the data which come about through vowel reduction. Independently of the observation that vowel reduction is known to create tautosyllabic clusters in some languages, the syllabic consonant patterns here suggest a higher occurrence of vowel reduction processes, both diachronic and synchronic, in languages with more complex syllable structure. This issue will be explored in greater detail in Chapter 6.
\end{styleBody}

\begin{styleBody}
\ \ The hypothesis with respect to morphological patterns in the data was also confirmed. As syllable structure increases, maximal onset and coda clusters are more likely to exhibit morphologically complex patterns, and syllabic consonant inventories are more likely to have members which correspond to grammatical morphemes. These findings point toward a strong influence of morphology in the emergence of more complex, and specifically Highly Complex, syllable structures. The studies here are largely limited to phonological systems and do not allow for an in-depth study of the morphological patterns of the language sample. However, the issue of the role of morphology in the syllable patterns of languages in the Highly Complex group will be revisited in Chapter 8.
\end{styleBody}

\begin{styleBody}
\ \ The analysis of the Highly Complex patterns in §3.4 reveals important patterns within this group of languages that should be considered in the coming chapters. The first is that there are measures by which this is not a coherent group of languages. Analyses of the specific syllable structures occurring in these languages, as well as the restrictions on these structures and their frequency of occurrence, suggests that there are instead several groups. In one group of six languages from Africa and Australia \& New Guinea, Highly Complex syllable structure is an extremely minor pattern, and includes low frequencies of highly restricted structures, often containing several sonorants, at one syllable margin. In another group of eight languages, mostly from Eurasia and North America, Highly Complex syllable structure is a prevalent pattern, and involves high frequencies of long, fairly unrestricted strings of obstruents at both syllable margins. While it wasn’t explicitly discussed in §3.4, these two groups also have different morphological patterns in their maximal syllable structures. Most of the languages with minor Highly Complex patterns have morpheme-internal patterns for their maximal syllable margins (the two exceptions being Bench and Doyayo). By comparison, all of the languages with prevalent Highly Complex patterns have morphologically complex patterns for their maximal syllable/word margins.
\end{styleBody}

\begin{styleBody}
\ \ Thus there are two extreme groups within the Highly Complex category which can be set apart from the rest on the basis of having different sets of coherent behavior in their syllable patterns. The other 11 languages of the sample fall somewhere between these two extremes. In the upcoming studies it will be discussed how languages on the two extremes of the Highly Complex category also exhibit coherent differences in their segmental inventories, stress patterns, and phonological processes, and how the languages in between the extremes behave more like one group or another.
\end{styleBody}

\begin{styleBody}
\ \ The second important finding in §3.4 is that there is another measure by which the languages of the Highly Complex category \textit{are} a coherent group of languages. In all languages for which phonetic properties of clusters were described, Highly Complex clusters were described as having largely similar acoustic and perceptual characteristics. This is true of both languages with large tautosyllabic clusters and those with syllabic obstruents, suggesting that these phenomena are qualitatively similar and/or have similar origins. This point will be revisited in Chapter 8.
\end{styleBody}

\clearpage\subsection[Chapter 4: Phoneme inventories and syllable structure complexity]{\rmfamily Chapter 4: Phoneme inventories and syllable structure complexity}
\begin{styleBody}
\ \ The central research questions of this book seek to (i) establish whether highly complex syllable structure is associated with other phonological features such that it can be identified as a coherent linguistic type, and (ii) use these findings to inform diachronic paths of development for these structures. The purpose of this chapter and the ones that follow is to address these research questions by examining other phonological properties of the language sample as they relate to syllable structure complexity. In this chapter I examine the properties of segment inventories in the language sample. Specifically, I test several hypotheses relating the size and constituency of vowel, and especially consonant, inventories to syllable structure complexity.
\end{styleBody}

\begin{styleBody}
\ \ The chapter is organized as follows. In §4.1 I discuss previous findings in the literature regarding properties of consonant and vowel inventories, and accounts put forth to explain predominant crosslinguistic trends in the patterns observed. I then discuss relevant findings relating the structure of sound inventories to syllable structure complexity, and introduce the hypotheses to be tested in the current study. In §4.2 I describe the methodology behind the data collection and the coding process. In §4.3 I present a brief analysis of vowel inventory properties and test the hypothesis that syllable structure complexity is associated with larger vocalic nucleus inventories. §4.4 is a longer section presenting several different analyses testing hypotheses regarding the size and makeup of consonant inventories with respect to syllable structure complexity. In §4.5 I discuss the results as they relate to highly complex syllable structure, its development, and syllable structure-based phonological typologies more generally.
\end{styleBody}

\section[4.1 Introduction]{\rmfamily 4.1 Introduction}
\begin{styleBody}
\ \ The scope of the current study is limited to examining the properties of segmental phonemes, the more or less discrete units corresponding to contrastive sounds in a language. Suprasegmental properties, including stress and tone, will be considered in Chapter 5. Certain kinds of variation in segments, including vowel reduction and specific kinds of consonant allophony, will be considered in Chapters 6 and 7.
\end{styleBody}

\begin{styleBody}
\ \ A segmental phoneme is an abstract unit corresponding to a set of sounds, usually phonetically similar to one another, which have some functional, cognitive, and/or perceptual equivalence in a language. A language’s phoneme inventory is the group of such units which meaningfully contrast with one another in that language, e.g., /l/ and /b/ in the English pair \textit{leek} and \textit{beak}. This concept is well over a century old and has been modified over the years (e.g., Sapir 1925, Chomsky \& Halle 1968), but is still widely used in both theoretical models and language descriptions. 
\end{styleBody}

\begin{styleBody}
\ \ A disadvantage of phonemic analysis is that it forces a discrete linear analysis on the continuous speech stream. The result of this segmental analysis, some argue, is a theoretical representation which is more reflective of the alphabetic scripts used by many linguists in everyday life than it is of any real property of spoken language (Port 2006, Moreno Cabrera 2008). Noting these problems, some models posit that phonological structures are emergent from more general speech processes. For example, in the Articulatory Phonology framework (Browman \& Goldstein 1992b), syllable patterns emerge from the coordination and phasing of gestures. In Lindblom’s (2000) model, segments and gestures themselves are adaptations to biophysical constraints on perception and production, as well as cognitive processes of memory encoding. Indeed, alternative views such as Articulatory Phonology and exemplar models of language (Bybee 2001) may provide more satisfactory accounts for the fine-grained and gradient nature of sound variation and change. Nevertheless, for all the problematic aspects of phonemic analysis, segment inventories are useful points of comparison in typological studies like the current work. Most language references provide such an analysis at the very minimum, even if no other phonetic or phonological description is given.
\end{styleBody}

\begin{styleBody}
\ \ Phoneme inventories are perhaps the best-researched topic in phonological typology, with numerous large-scale surveys dedicated to their study. Standard works on the typology of phoneme inventories occur as early as the mid-20th century (e.g., Hockett 1955). The Stanford Phonology Archive (Crothers et al. 1979), a project undertaken in connection to the Stanford Universals Project, was the first computerized database of phoneme inventories. Maddieson (1984) drew upon this work in his survey of 317 genealogically balanced languages, the UCLA Phonological Segment Inventory Database (UPSID). Since then, many such large typological surveys of phoneme inventories have been developed, including an expanded version of UPSID (451 languages, Maddieson \& Precoda 1990), the Lyon-Albuquerque Phonological Systems Database (LAPSyD, \~{}700 languages, Maddieson et al. 2013), PHOIBLE (1,672 languages, Moran et al. 2014), and portions of the World Atlas of Language Structures (WALS, \~{}565 languages, Maddieson 2013b, 2013c, inter alia). In addition to these, there have been extensive surveys of phoneme inventories in specific geographical areas (e.g., Michael et al. 2015 for South America, Gasser \& Bowern 2014 for Australia). There are also a great many crosslinguistic studies examining the properties and patterns of specific kinds of sounds, including nasalized vowels (Hajek 2013), non-modal vowels (Gordon 1998), ejectives (Fallon 2002), consonants with secondary palatalization (Hall 2000), affricates (Berns 2013), and post-velar consonants (Sylak-Glassman 2014). 
\end{styleBody}

\section[4.1.1 Crosslinguistic patterns in consonant inventories]{4.1.1 Crosslinguistic patterns in consonant inventories}
\begin{styleBody}
\ \ In articulatory terms, consonants and vowels are distinguished from one another by the degree to which the vocal tract is constricted in their production, with consonants having greater constriction than vowels. Consonants are typically classified by their articulatory characteristics which include phonation, the place of constriction in the vocal tract, and the manner of constriction.
\end{styleBody}

\begin{styleBody}
\ \ Consonant phoneme inventory size is a common point of comparison in crosslinguistic studies of phonological systems. In the 563-language sample in Maddieson (2013b), the languages have an average of 22.7 consonant phonemes, though values range widely from six consonants (in Rotokas) to 122 consonants (in !Xóõ). There are areal patterns to the distribution of consonant inventory size. Small inventories (6-14 consonants) are common in New Guinea and the Amazon region of South America. Large consonant inventories (26 or more consonants) are concentrated in the Pacific Northwest and northern coast of North America, northern Europe, the Caucasus region, and regions of \ Africa.
\end{styleBody}

\begin{styleBody}
\ \ One of the contributions of the 317-language survey in Maddieson (1984) was the establishment of crosslinguistic frequency patterns for consonant phonemes. The following consonants are the 20 most frequently present in the inventories of the language sample (* indicates that dental and alveolar consonants have been pooled).
\end{styleBody}

\begin{styleBody}
(4.1)\ \ /p b *t *d k [261?] [294?] t[361?][283?] f *s [283?] h m *n [272?] ŋ w *l *r j/
\end{styleBody}

\begin{styleBody}
(Maddieson 1984: 12)
\end{styleBody}

\begin{styleBody}
None of the consonants in (4.1) were found to occur in every language in the sample, and some (/[294?] t[361?][283?] f [283?] [272?] r/) were found in fewer than half of the languages. It follows that strict implicational hierarchies derived from these frequency measures do not accurately predict the makeup of observed consonant phoneme inventories, small or large. Nevertheless, all spoken languages have at least several of these consonants, even though there are hundreds of other consonants from which inventories could hypothetically be entirely drawn. The constituency of this set of consonants in terms of numbers of stops, fricatives, nasals, and so on closely resembles the modal makeup of consonant inventories in the sample overall. Lindblom \& Maddieson (1988) note that this set of consonants is nearly identical to that reported for stages of early speech and babbling. Furthermore, in a sample of 32 diverse languages, Gordon finds that there is a strong positive correlation between the frequency of these consonants crosslinguistically (that is, across consonant inventories) and the type frequency of these consonants within languages (2016: 73-4).
\end{styleBody}

\begin{styleBody}
\ \ The convergence of these patterns suggests a phonetic naturalness to the consonants in (4.1) which many researchers have attempted to account for. Stevens (1989) shows that there are configurations of articulators within the vocal tract where the acoustic and auditory properties of the sound produced there are fairly stable with respect to variations in the articulation. He suggests that these regions of acoustic-perceptual stability underly the common distinctions found in phoneme inventories. Maddieson (1996) proposes that phonological patterns are motivated by gestural economy, in that optimal contrastive sounds will involve gestures which are both inherently efficient in their motor requirements and have a high degree of auditory distinctiveness. Other accounts take a more abstract approach. Ohala (1979) observes that consonants in small phoneme inventories may be perceptually close to one another and differ by a minimum rather than maximum of distinctive features. On the basis of this he proposes that consonant phoneme inventories are motivated by a principle of \textit{Maximum Utilization of Available Features }(MUAF). In a similar vein, Clements (2003) proposes that consonant inventories tend towards economy in their constituency; that is, sounds are less likely to occur in a language if their distinctive features are not employed elsewhere in the phoneme inventory, and more likely to occur if all their distinctive features occur elsewhere in the phoneme inventory. The set of consonants in (4.1) is quite coherent in this respect. A typological study of borrowed sounds lends support to these accounts: Maddieson (1985) finds that borrowed sounds are statistically much more likely to fill gaps in the phonological inventory of the recipient language than to alter the basic contrasts of the system.
\end{styleBody}

\begin{styleBody}
\ \ Lindblom \& Maddieson (1988) propose an account for the observed crosslinguistic tendencies which is rooted in properties of articulatory complexity. In their model, consonants are divided into three sets: Set I, basic articulations which often correspond to those in (4.1); Set II, elaborated articulations corresponding to properties described below; and Set III, complex articulations, consisting of combinations of elaborated articulations. Elaborated articulations are defined as those which depart from default modes of phonation and manner (especially airstream mechanism), as well as place articulations which depart from the neutral near-rest positions of active articulators in the vocal tract. A list of these elaborations is reproduced in Table 4.1.
\end{styleBody}

\begin{flushleft}
\tablefirsthead{}
\tablehead{}
\tabletail{}
\tablelasttail{}
\begin{supertabular}{m{2.0344598in}m{1.3136599in}m{1.5573599in}}
\hline
{\bfseries Phonation} &
{\bfseries Manner} &
{\bfseries Place}\\\hline
breathy voice

creaky voice

voiced fricatives/affricates

voiceless sonorants

pre-aspiration

post-aspiration

 &
prenasalization

nasal release

lateral release

ejectives

implosives

clicks &
labiodental

palatoalveolar

retroflex

uvular

pharyngeal

palatalization

labialization

pharyngealization

velarization\\\hline
\end{supertabular}
\end{flushleft}
\begin{styleBody}
\textbf{Table 4.1.} Elaborated consonant articulations, as presented in Lindblom \& Maddieson (1988: 67).
\end{styleBody}

\begin{styleBody}
The model predicts that as consonant inventory sizes increase, languages will include phonemes from Set I (e.g., /k/) until that set is more or less exhausted, at which point Set II consonants (e.g., /k[2B7?]/) and then eventually Set III consonants (e.g., /k[2B7?]’/) may occur. Lindblom \& Maddieson show that this prediction is borne out in the obstruent inventories of the 317-language sample from Maddieson (1984). They suggest that these patterns, at all levels of consonant inventory size, reflect a balance between competing pressures to keep articulatory complexity low while maintaining a sufficient level of perceptual contrast in the system (1988: 72).
\end{styleBody}

\begin{styleBody}
\ \ There are of course many other issues related to consonant inventory patterns which are too numerous to discuss here (e.g., common gaps in stop inventories with respect to place of articulation and voicing). An overview of many such patterns and proposed phonetic accounts for them can be found in Ohala (1983).
\end{styleBody}

\begin{styleBody}
\ \ The issues of consonant phoneme inventory size and elaborated articulations will be revisited in §4.1.3 and §4.1.4. In the following section I discuss reported typological patterns of vowel phoneme inventories.
\end{styleBody}

\section[4.1.2 Crosslinguistic patterns in vowel inventories]{\rmfamily 4.1.2 Crosslinguistic patterns in vowel inventories}
\begin{styleBody}
\ \ As described above, vowels are speech articulations which involve relatively less constriction in the vocal tract than consonants. Vowels are typically classified according to the height and backness of the tongue body and the rounding of the lips, which together constitute vowel ‘quality’. Other articulatory characteristics, such as length, nasalization, and voicing may also be contrastive for these sounds, but only in addition to vowel quality.
\end{styleBody}

\begin{styleBody}
\ \ Perhaps the most common point of typological comparison for vowel phoneme systems is the number and nature of vowel qualities present. Over half of the languages in the 564-language survey in Maddieson (2013c) have five or six vowel qualities present in their phoneme inventories. Like consonant phoneme inventory size, vowel quality inventory size has strong areal patterns with respect to its distribution. Smaller than average systems are common in the Americas, Australia, and isolated smaller regions. Larger than average systems are common in the central belt region of Africa, Southeast Asia, and parts of Eurasia. Areal patterning may also be observed in other vowel features, including contrastive nasalization, which is predominantly concentrated in Western Africa and the Amazon region.
\end{styleBody}

\begin{styleBody}
\ \ The five most common vowel quality phonemes in Maddieson’s (1984) survey are /i a u “o” “e”/ (where quotations indicate that these may not be distinguishable from other vowels in the mid area in the references consulted). Unlike the situation with consonants above, there are many languages which have a triangular system of five vowels corresponding exactly to this set (1984: 136). Generally speaking, there are strong crosslinguistic tendencies relating the size of vowel quality inventories to the vowel qualities observed to occur. For example: for example, 3-vowel systems are most often of the shape /i a u/. 
\end{styleBody}

\begin{styleBody}
\ \ As with crosslinguistic tendencies in consonant inventories, both acoustic/perceptual and articulatory accounts have been put forward to explain the observed patterns in vowel quality inventories. Liljencrants \& Lindblom (1972) test the hypothesis that vowel inventories pattern in such a way as to maximize perceptual distance, measured as a function of formant values. The predictions of their model match very closely the most common vowel quality inventories for vowel systems of three, four, and five vowels, but for larger inventories, there are discrepancies between the model and observed crosslinguistic patterns. The study by Stevens (1989) mentioned above considered vowel systems in addition to consonant systems, and determined /i a u/ to be regions of acoustic/perceptual stability with respect to articulatory variation. Lindblom \& Maddieson (1988) also explored vowel inventory patterns and concluded that as with consonant systems, common vowel system patterns reflect competing pressures of maximization of perceptual contrast and minimization of articulatory complexity. Thus it is not expected that articulatorily ‘complex’ contrasts such as phonation, nasalization, length, and so on would be present in a system of five total vowels, where perceptual distinctiveness can be easily achieved by vowel quality differences alone. Analogous to their model of consonant elaborations, differences in phonation, nasalization, and so on should be expected to occur only in larger systems where vowel quality contrasts have already been exploited. From a diachronic point of view, such contrasts in vowel systems typically imply larger vowel inventories because they come about through sound changes that are systematic across the vowel system or large portions of the vowel system.
\end{styleBody}

\section[4.1.3 Segmental inventories and syllable structure complexity]{\rmfamily 4.1.3 Segmental inventories and syllable structure complexity}
\begin{styleBody}
\ \ The crosslinguistic patterns described above and the proposed accounts for them are limited to phoneme inventories themselves and do not consider possible interactions between phoneme inventories and other aspects of language structure. Yet a number of correlations between phoneme inventories and other phonological structures, most notably syllable structure complexity, have been found.
\end{styleBody}

\begin{styleBody}
\ \ Maddieson (2006) determined that there is a highly significant positive correlation between consonant phoneme inventory size and syllable structure complexity in a sample of roughly 520 languages. In that study, it was found that languages with Simple syllable structure had an average of 19.3, languages with Moderately Complex syllable structure an average of 21.8, and languages with Complex syllable structure an average of 25.7 consonant phonemes (Maddieson 2013a reports similar findings). Within that sample, there are some overlapping geographical distributions of small consonant inventories and simpler syllable structures on the one hand and large consonant inventories and more complex syllable structure on the other hand. The Pacific Northwest region of North America is an example of a region with the latter pattern, and the Amazon Basin is an example of a region with the former pattern. However, Maddieson rejects the idea that the overall correlation was the result of several small-scale patterns, finding the general trend to hold up significantly in all but one of the large geographical regions examined. He concludes that the association between consonant phoneme inventory size and syllable structure complexity is crosslinguistically robust and suggests that “paths of natural historical linguistic change” may be behind this mutually reinforcing pattern of complexity (2006: 118). 
\end{styleBody}

\begin{styleBody}
\ \ Consonant phoneme inventory size is positively correlated with syllable structure complexity when it is measured in non-categorical ways, too. Maddieson (2011) reports a positive correlation between consonant inventory size and Syllable Index values. The Syllable Index is a sum of maximal onset, nucleus, and coda complexity values, closely but not perfectly corresponding to the number of segments in the maximal syllable type. Gordon (2016) plots consonant inventory size against the sum of maximal syllable margins for each language in the modified WALS 100-language sample and finds an increasing, if not stepwise trend. He reports similar results for analyses considering only onset or coda size.
\end{styleBody}

\begin{styleBody}
\ \ There has been limited research into the patterns of specific segment types and syllable structure complexity. Maddieson et al. (2013) report a relationship between segmental complexity in phoneme inventories and syllable structure complexity in the \~{}700-language LAPSyD sample. In that study they consider the number of consonants with one or more elaborated articulations, as defined by Lindblom \& Maddieson (1988) and listed in Table 4.1 above. They find that languages with Complex syllable structure have a mean of 9.6 consonants with elaborated articulations, as compared to means of 6.2 and 4.8 such consonants in the Moderately Complex and Simple categories, respectively. The difference between the Complex pattern and the two other patterns combined was found to be statistically significant.
\end{styleBody}

\begin{styleBody}
\ \ There have also been suggestions of correlations between phoneme inventory properties and other phonological features at smaller scales, within regions or language families. In his holistic phonological typology of Slavic languages, Isačenko (1939/1940) notes that ‘consonantal’ languages are defined by a collection of features, including more complex syllable structure, a larger proportion of consonants in the phoneme inventory, and the presence of contrastive secondary palatalization at various places of articulation. Russian and Polish are prototypical examples of such languages. By comparison, ‘vocalic’ languages have simpler syllable structure, smaller proportions of consonants in the phoneme inventory, and secondary palatalization which is limited to dental consonants or altogether absent. The Ljublana dialect of Slovene exemplifies this type. Chirikba calls all languages of the Caucasus “consonant-type languages”, a term which encompasses a heavy dominance of consonants over vowels in the speech signal, rich consonant inventories, and restricted vowel systems (2008: 43). Chirikba specifically notes the typologically unusual nature of consonant systems in the languages of the region, which include ejectives and richly elaborated sibilant and post-velar articulations.
\end{styleBody}

\begin{styleBody}
\ \ The above observations bring up another relationship worth mentioning, which is that between consonant inventory size and vowel inventory size. While no correlation has ever been established between consonant inventory size and \textit{vowel quality} inventory size (Maddieson 2013c), a positive correlation has been found between consonant inventory size and \textit{total vowel} inventory size (Maddieson 2011). In that study, the ‘total vowel inventory’ includes vowels which contrast in length, nasalization, and phonation properties, as well as diphthongs analyzed as unitary, but does not include tautosyllabic vowel sequences or diphthongs that can be parsed into constituents corresponding to basic vowels in the language.
\end{styleBody}

\section[4.1.4 The current study and hypotheses]{\rmfamily 4.1.4 The current study and hypotheses}
\begin{styleBody}
\ \ The findings described above indicate that the relationship between properties of segment, and especially consonant, inventories and syllable structure complexity is notable at global, regional, and family levels, and thus worthy of further investigation. The hypotheses I introduce and test here build on the previous findings by investigating the relationships between more specific properties of phoneme inventories and syllable structure complexity. Depending on their nature, these findings may help to shed light on those paths of language change suggested by Maddieson (2006) to motivate the observed correlations.
\end{styleBody}

\begin{styleBody}
\ \ The first hypothesis concerns vocalic nucleus inventory size and syllable structure complexity. Recall that in §3.3.4 it was found that complex vocalic nuclei, defined there as long vowels, diphthongs, and/or tautosyllabic vowel sequences, were more frequently present in languages with more complex syllable structure. There is reason to explore this pattern in more depth here. As noted above, a positive correlation between total vowel inventory size and consonant phoneme inventory size has been established in the literature (Maddieson 2011). However, the measure of total vowel inventory size did not include tautosyllabic vowel sequences or diphthongs that can be alternatively analyzed as sequences of segments. Thus it would be interesting to test whether a relationship exists between the \textit{total number of vocalic nuclei} in a language and syllable structure complexity. If such a relationship is found, it would suggest that higher syllable margin diversity is accompanied by higher nucleic diversity in languages with more complex syllable structure, a fact that would have to be considered in any diachronic account of the development of highly complex syllable structure. The hypothesis is as follows (4.2).
\end{styleBody}

\begin{styleBody}
(4.2)\ \ \textit{As syllable structure complexity increases, languages will have larger inventories of vocalic nuclei.}
\end{styleBody}

\begin{styleBody}
\ \ This is the only hypothesis regarding vowel patterns in the sample. The remaining hypotheses are concerned with consonant patterns. The second hypothesis follows the findings of Maddieson (2006), and simply predicts that the previously determined positive association between syllable structure complexity and consonant phoneme inventory size will be upheld when the additional category of Highly Complex syllable structure is included in the analysis. Following observations by Gordon (2016), I also expect that consonant phoneme inventory size will increase with syllable structure complexity when it is measured not just categorically but also as a sum of maximal syllable margins. This hypothesis is given in (4.3).
\end{styleBody}

\begin{styleBody}
(4.3)\ \ \textit{As syllable structure complexity increases, so does the size of consonant phoneme inventories.}
\end{styleBody}

\begin{styleBody}
\ \ The third hypothesis is aimed at quantifying the number of articulatory elaborations present in the consonant inventories of languages with different syllable structure complexity. Maddieson et al. (2013) found a higher mean number of consonants with elaborated articulations in languages with more complex syllable structure. However, the reported findings of that study did not consider whether languages with more complex syllable structure also had more distinct elaborations present in their consonant inventories. That is, the findings do not indicate whether languages with more complex syllable structure have more elaborations in general, or just more consonants sharing the same elaboration. Reported phonological patterns for areas well-known for having complex syllable patterns suggest the presence of more elaborations in their consonant inventories (e.g., ejectives and uvulars in the Caucasus, lateral release and ejectives in the Pacific Northwest). This would also follow indirectly from Lindblom \& Maddieson (1988), who found consonants with combinations of elaborated articulations in languages with large consonant inventories. This leads to the formulation of a third hypothesis (4.4).
\end{styleBody}

\begin{styleBody}
(4.4)\ \ \textit{As syllable structure complexity increases, so does the number of articulatory elaborations present in consonant phoneme inventories.}
\end{styleBody}

\begin{styleBody}
\ \ The final hypothesis relates syllable structure complexity to the occurrence of specific consonant types. This hypothesis is motivated by the observation that there are certain consonants which seem characteristic of languages with more complex syllable structure. Specifically, post-velar and especially uvular consonants, though crosslinguistically rare, are common in regions also famous for complex syllable structure, including the Pacific Northwest, the Caucasus, and the Atlas Mountain region. Similarly, it is my observation that ejectives are often found in languages with complex syllable structure, and often co-occur with uvular consonants in those languages. Based on these observations, a fourth hypothesis is formulated.
\end{styleBody}

\begin{styleBody}
(4.5)\ \ \textit{Languages with differing degrees of syllable structure complexity will exhibit different consonant contrasts in their phoneme inventories.}
\end{styleBody}

\begin{styleBody}
\ \ The data analyses addressing these hypotheses will be presented in §4.3 and §4.4. In the next section I describe the methodology behind the data collection and coding for these analyses.
\end{styleBody}

\section[4.2 Methodology]{\rmfamily 4.2 Methodology}
\section[4.2.1 Patterns considered]{\rmfamily 4.2.1 Patterns considered}
\begin{styleBody}
\ \ In this chapter, only the segmental patterns of the language sample are considered. While most of the analyses here will treat specific articulations that do not constitute segments on their own (i.e., those associated with place, manner, voicing, length, etc.), it must take as a starting point the consonant and vowel phoneme inventories of the language sample. These are understood to be the more or less discrete units which are mostly unpredictable in their distribution and meaningfully contrastive in the native lexicon and grammar of a language. Though consonant and vowel phoneme inventories are reliably reported in most language references, phonemic analysis is not always a straightforward endeavor. Here I discuss some issues that arise in determining phoneme inventory patterns.
\end{styleBody}

\begin{styleBody}
\ \ Phonemic inventories are always the result of an analysis. It is common for there to be slight disagreements regarding the composition of the phoneme inventory in different descriptive materials for the same language. Authors may be writing in different time periods, describing different dialects and/or speech styles, or, in the case of highly endangered languages, working with speakers with varying degrees of proficiency in the language. When sources disagree on just a few elements of the phoneme inventory, I take these factors into consideration. For example, two of the sources on Nuu-chah-nulth, Stonham (1999) and Davidson (2002), list uvular ejectives /q’ q[2B7?]’/ in the consonant phoneme inventory, but a third source (Kim 2003) does not. The former analyses are based primarily on the field notes of Edward Sapir, who worked with the language from 1914-1924. Kim (2003) shows that ejective uvulars have long since merged with pharyngeal /[295?]/ in the present language. I take the more recent analysis to be accurate for the current state of the language. 
\end{styleBody}

\begin{styleBody}
\ \ Of course, the choices made in a phonemic analysis may reflect a number of other factors, including the data available and the author’s own theoretical training and native language biases. When sources present dramatically different phoneme inventories, I accept the source which supports the analysis more thoroughly with illustrative language-internal data. For example, Cunha de Oliveira (2005) presents a consonant phoneme inventory for Apinayé which includes an entire prenasalized consonant series which is not listed in Burgess \& Ham (1968), who take a more abstract approach. Cunha de Oliveira shows that although prenasalized consonants are often in complementary distribution with nasals in the language, there are minimal pairs showing that these sounds are meaningfully contrastive in some environments, an observation that is reinforced by reported native speaker intuition about the forms.
\end{styleBody}

\begin{styleBody}
\ \ Sounds which are limited to recent loanwords, the speech of bilinguals, and certain speech styles were not included in the present study. For instance, in Cocopa, mid front vowel /e/ is described as occurring only in loanwords from Spanish and English, and even then is often replaced by native /i/ (Crawford 1966: 26). In Tzeltal, voiceless labial fricative /f/ and alveolar trill /r/ are reported to occur only in loanwords in the speech of ‘acculturated’ Spanish bilinguals (Kaufman 1971: 13). In Chipaya, glottal stop /[294?]/ occurs in one obsolescing morpheme, \textit{{}-[294?]a}, a declarative suffix formerly used by women to address other women (Cerrón-Palomino 2006: 55-6). In all these and similar cases, the given sounds were omitted from the current analysis.
\end{styleBody}

\begin{styleBody}
\ \ Authors of language descriptions often present ‘marginal’ phonemes — those occurring with very low frequency, highly limited distributions, or in just a few lexical items — in addition to more straightforward ones. Where authors show these to be contrastive in lexical items, I have generally included such phonemes here. Where a marginal phoneme is described as clearly obsolescing or merging with another sound to the point that the contrast is no longer meaningful for most speakers of the variety examined, I have excluded it.
\end{styleBody}

\begin{styleBody}
\ \ As mentioned in §3.2.1, sounds with multiple articulations, such as labialized consonants, affricates, or diphthongs, present obvious complications for a study of this sort. The analysis of a phonetic sequence, such as [mb], as either a sequence of two simple segments or as a single complex segment can in turn affect how the canonical syllable patterns of a language are analyzed.\footnote{Of course, these issues may not prove so problematic in an Articulatory Phonology framework, or any model which does not force a discrete segmental analysis upon sound and syllable patterns.} Because issues such as these may create a potential confound in how we interpret associations between syllable structure complexity and segmental inventories, it is important that competing analyses be carefully evaluated. 
\end{styleBody}

\begin{styleBody}
\ \ Instrumental data can be used to support either a complex segment analysis or a sequential analysis in such a scenario. For example, if a phonetic sequence of homorganic nasal+stop has a durational pattern comparable to that of a simple voiceless stop in a language, this might be taken as evidence for a complex segment analysis (e.g., as shown for Fijian by Maddieson 1989a; though Ladefoged \& Maddieson 1996 note that there is wide crosslinguistic variation in timing patterns in prenasalized consonants). There are a few studies on such issues in languages of the current sample. For example, Chitoran (1998) uses acoustic evidence to argue that Georgian harmonic clusters (e.g., [d[261?]], [t[2B0?]k[2B0?]]) are better analyzed as sequences than as complex segments, as some have claimed. She shows that each member of a harmonic cluster has a release burst, and that the durational properties of these clusters word-internally do not differ significantly from identical sequences found across word boundaries. However, it is generally very rare in the language descriptions consulted here for authors to present acoustic or articulatory evidence supporting one analysis over another in such situations. In the absence of instrumental data, authors often rely on phonological criteria to support their analyses.
\end{styleBody}

\begin{styleBody}
\ \ Sometimes authors base their analyses on distributional data. Erickson argues that phonetic C+[w] structures in Lao are in fact labialized consonants and not onset clusters. He observes that no corresponding C+[j] sequences (or palatalized consonants, for that matter) occur as onsets in the language, and that C+[w] structures are infrequent and limited in their distribution, occurring almost entirely before the low vowel /[251?]/ and never before rounded vowels (2001: 135-8). These facts suggest a historical process by which the consonant in C+[u] sequences may have taken on the rounding of the high back vowel, a crosslinguistically common type of assimilation. Note that in this case, either analysis would put the language in the Moderately Complex category in the current study.
\end{styleBody}

\begin{styleBody}
\ \ Similar criteria are used to posit a series of prenasalized consonants in Tukang Besi. If these structures were considered to be sequences, then they would be the only consonant clusters occurring in the language, which otherwise has canonical (C)V structure. Prenasalized consonants behave as a unit in reduplication processes; that is, words like \textit{karambau} have \textit{kara-karambau} as a reduplicated form, instead of \textit{karam-karambau}. (This argument assumes a syllabification of \textit{ka.ram.bau} in the scenario that [mb] is a sequence and not a complex segment). Additionally, native speakers put syllable breaks before the nasal+C sequences when dividing words into syllables (Donohue 1999: 30-31). 
\end{styleBody}

\begin{styleBody}
\ \ The Tukang Besi evidence is not strictly conclusive. The language could be analyzed as having (C)(C)V syllable structure which has very specific restrictions on C\textsubscript{2} and C\textsubscript{1}. In this case the ambiguous interpretation has important consequences for syllable structure complexity: one interpretation puts the language into the Simple category, while the other puts it into the Complex category. There is just one language in the Complex category, Lunda, which has biconsonantal onset patterns fitting the hypothetical (C)(C)V pattern for Tukang Besi. However, in Lunda other biconsonantal onsets, like C+glide sequences, also occur, and the nasal+C sequences may come about through morphological processes (that is, some are morphologically separable; example 4.6).
\end{styleBody}

\begin{styleBody}
(4.6) \ \ \textbf{Lunda} (\textit{Atlantic-Congo}; Democratic Republic of Congo, Angola, Zambia)
\end{styleBody}

\begin{styleBody}
/ku-n-[292?]ikwila/
\end{styleBody}

\begin{styleBody}
\textsc{inf}{}-\textsc{1.sg}{}-uncover
\end{styleBody}

\begin{styleBody}
[ku.n[292?]i.kwi.la]
\end{styleBody}

\begin{styleBody}
‘to uncover for me’
\end{styleBody}

\begin{styleBody}
(Kawasha 2003: 24)
\end{styleBody}

\begin{styleBody}
In Lunda onsets, nasals may combine with a wide variety of consonants, including all plosives, oral fricatives, /h/, /l/, and /w/. In Tukang Besi, the C in nasal+C structures is always an oral plosive or /s/, though other fricatives and sonorants occur in the language. There is persuasive evidence that nasal+C structures are sequences in Lunda. The nasal+C structures in Tukang Besi do not have much in common with those of Lunda in terms of their behavior. Though the evidence for the unitary status of prenasalized consonants in Tukang Besi is not entirely conclusive, I follow the author’s analysis here, coding these structures as complex segments and classifying the language as having Simple syllable structure. 
\end{styleBody}

\begin{styleBody}
\ \ A similar issue arises in interpreting a phonetic sequence of a mid or low vowel followed by a high offset. This can be analyzed as a diphthong, in which case the entire structure functions as a syllable nucleus, or a sequence of V+glide, in which case the glide is a member of the coda. Competing analyses for such structures can be found in Yakima Sahaptin. Structures represented orthographically as {\textless}ay{\textgreater}, {\textless}aw{\textgreater} {\textless}uy{\textgreater}, and so on, are described as diphthongs by Jansen (2010) and Rigsby \& Rude (1996). However, Hargus \& Beavert (2006) present evidence that the structures ending in the high front articulation may be better analyzed as V+/j/ sequences. Preceding /m/, these structures trigger a vowel epenthesis process that is also conditioned by other sonorant consonants, but not vowels, in the language (4.7).
\end{styleBody}

\begin{styleBody}
(4.7)\ \ \textbf{Yakima Sahaptin} (\textit{Sahaptian}; USA)
\end{styleBody}

\begin{styleBody}
(a)\ \ /t[361?][26C?]’j\'{a}lm/
\end{styleBody}

\begin{styleBody}
\ \ [t[361?][26C?]’j\'{a}l[268?]m]
\end{styleBody}

\begin{styleBody}
\ \ ‘Cle Elum (place name)’
\end{styleBody}

\begin{styleBody}
(b)\ \ /talu\'{ }jm/
\end{styleBody}

\begin{styleBody}
\ \ [talu\'{ }j[268?]m]
\end{styleBody}

\begin{styleBody}
\ \ ‘nail’
\end{styleBody}

\begin{styleBody}
(c)\ \ /naknúwim/
\end{styleBody}

\begin{styleBody}
\ \ [naknúwim]
\end{styleBody}

\begin{styleBody}
\ \ ‘take care of me’
\end{styleBody}

\begin{styleBody}
(Hargus \& Beavert 2006: 28)
\end{styleBody}

\begin{styleBody}
This is presented as evidence that the high front element in these structures behaves as a consonant. Though the /w/ component of such sequences is not reported to trigger the epenthesis process in (4.7), it is reported to pattern with /j/ in other morphophonemic contexts, and Hargus \& Beavert (2006) treat it as a consonant in their analysis. This analysis has the effect of increasing the maximal coda pattern of the language to four, in which all four-consonant codas begin with a glide, e.g., \textit{sajlps} ‘kidney’. However, due to other syllable patterns in the language, it does not affect the syllable structure complexity classification, which in either case is Highly Complex.
\end{styleBody}

\begin{styleBody}
\ \ Kunjen presents an example of a language for which a sequential analysis rather than a complex segment analysis results in patterns which directly affect its syllable structure complexity classification. Sommer rejects a prenasalized stop analysis for structures such as [mb] and [ŋ[261?]] on the basis that reverse sequences occur and all component segments may occur separately (1969: 34). This analysis is what allows Kunjen to be classified as having Highly Complex syllable structure in the current study, as nasal+stop sequences are always present in the four-consonant codas in the language, which are also the only Highly Complex structures occurring (4.8).
\end{styleBody}

\begin{styleBody}
(4.8) \ \ \textbf{Kunjen} (\textit{Pama-Nyungan}, Australia)
\end{styleBody}

\begin{styleBody}
/albmb/
\end{styleBody}

\begin{styleBody}
‘opossum’
\end{styleBody}

\begin{styleBody}
(Sommer 1969: 33)
\end{styleBody}

\begin{styleBody}
\ \ It should be noted that it was generally rare for ambiguous segmental analyses to affect the analysis of syllable structure to the point where a language might be classified in a different syllable structure complexity category. In fact the Tukang Besi and Kunjen examples discussed here are the most potentially problematic cases in the entire language sample.
\end{styleBody}

\section[4.2.2 Coding]{\rmfamily 4.2.2 Coding}
\begin{styleBody}
\ \ After the above criteria were considered and segmental inventories determined, properties of the vowel and consonant inventories were coded as described here.
\end{styleBody}

\begin{styleBody}
\ \ Vowel inventories were coded for all reported contrasts. First, the number of vowel quality distinctions was noted. Every vowel inventory was additionally coded for the presence or absence of contrastive vowel length, nasalization, and other less common contrasts, such as voicing and glottalization. Where such contrasts were present, it was noted whether the contrast was distinctive for all or some vowels. 
\end{styleBody}

\begin{styleBody}
\ \ I also noted the presence of diphthongs and/or tautosyllabic vowel sequences and recorded the number and specific forms of these structures. Because they are so often analyzed as phonological sequences which surface phonetically as diphthongs, diphthongs present complications in establishing vowel phoneme inventory patterns (Maddieson 1984: 133). Recall that the purpose of considering patterns of diphthongs and tautosyllabic vowel sequences here is to establish the size of the full vocalic nucleus inventory for each language in order to test the hypothesis in (4.2). Therefore the diphthongs and tautosyllabic vowel sequences included in the inventories are not necessarily meant to be interpreted as phonologically unitary segments, but as possible nucleus patterns if reported as occurring as such.
\end{styleBody}

\begin{styleBody}
\ \ In (4.9) I illustrate the coding with the vowel phoneme inventory of Pinotepa Mixtec, a language with Simple syllable structure.
\end{styleBody}

\begin{styleBody}
(4.9) \ \ \textbf{Pinotepa Mixtec} (\textit{Otomanguean}; Mexico)
\end{styleBody}

\begin{styleBody}
\textbf{V phoneme inventory:} /i e a o u \~{i} \~{e} \~{a} \~{o} \~{u} i[330?] e[330?] a[330?] o[330?] u[330?] i[330?] e[330?] a[330?] o[330?] u[330?]/
\end{styleBody}

\begin{styleBody}
\textbf{\textit{N}}\textbf{ vowel qualities:} 5
\end{styleBody}

\begin{styleBody}
\textbf{Diphthongs or vowel sequences:} None
\end{styleBody}

\begin{styleBody}
\textbf{Contrastive length:} None
\end{styleBody}

\begin{styleBody}
\textbf{Contrastive nasalization:} All
\end{styleBody}

\begin{styleBody}
\textbf{Other contrasts: }Glottalization (All)
\end{styleBody}

\begin{styleBody}
\ \ For each consonant inventory, the number of non-geminate consonants was recorded. Each consonant inventory was first coded for primary distinctions in voicing, place, and manner of articulation; here I use the term ‘primary’ to refer to those distinctions represented in the standard chart for non-pulmonic consonants in the International Phonetic Alphabet (IPA 2015).\footnote{Note that I use different terminology than IPA in some cases: ‘stop’ instead of ‘plosive’, and ‘palato-alveolar’ instead of ‘postalveolar’.} The presence of a primary voiced/voiceless distinction was noted separately for obstruents and sonorants; voicing had to be the sole distinguishing feature for at least one pair of consonants in order for this distinction to be counted (e.g., /k/ and /[261?]/, /m[325?]/ and /m/). All primary manners of articulation in the inventory were recorded, as were the primary places of articulation for all non-glide consonants. Additionally, I recorded the presence of elaborated articulations related to phonation, manner, and place, as defined by Lindblom \& Maddieson (1988) and listed in Table 4.1 above. Note that there is some overlap in what I take to be primary articulations and the articulations classified as elaborations by Lindblom \& Maddieson (e.g., labiodental, uvular); in the coding such articulations are included in both the place/manner lists and in the list of elaborations.
\end{styleBody}

\begin{styleBody}
\ \ In (4.10) I illustrate the coding with the consonant phoneme inventory of Lepcha, a language with Complex syllable structure.
\end{styleBody}

\begin{styleBody}
(4.10)\ \ \textbf{Lepcha} (\textit{Sino-Tibetan}; Bhutan, India, Nepal)
\end{styleBody}

\begin{styleBody}
\textbf{C phoneme inventory:} 
\end{styleBody}

\begin{styleBody}
/p p[2B0?] b t[32A?] t[32A?][2B0?] d[32A?] [288?] [288?][2B0?] [256?] c c[2B0?] k k[2B0?] [261?] [294?] t[361?]s t[361?]s[2B0?] f v s z [283?] [292?] h m n[32A?] [272?] ŋ r l[32A?] $\beta [31E?]$ j/
\end{styleBody}

\begin{styleBody}
\textbf{\textit{N}}\textbf{ consonant phonemes:} 32
\end{styleBody}

\begin{styleBody}
\textbf{Geminates:} N/A
\end{styleBody}

\begin{styleBody}
\textbf{Voicing contrasts:} Obstruents
\end{styleBody}

\begin{styleBody}
\textbf{Places:} Bilabial, Labiodental, Dental, Alveolar, Palato-alveolar, Retroflex, Palatal, Velar, Glottal
\end{styleBody}

\begin{styleBody}
\textbf{Manners:} Stop, Affricate, Fricative, Nasal, Trill, Central approximant, Lateral approximant
\end{styleBody}

\begin{styleBody}
\textbf{\textit{N}}\textbf{ elaborations:} 5
\end{styleBody}

\begin{styleBody}
\textbf{Elaborations:} Voiced fricatives/affricates, Post-aspiration, Labiodental, Palato-alveolar, Retroflex
\end{styleBody}

\begin{styleBody}
\ \ Dental and alveolar places of articulation are not always reliably distinguished in reference materials (Maddieson 1984: 31-32). Sometimes authors even use the joint label ‘dental/alveolar’ as a cover term for a series of consonants in that general area in the vocal tract. In such cases, I characterize the place of the consonants in question as Dental/Alveolar (4.11).
\end{styleBody}

\begin{styleBody}
(4.11)\ \ \textbf{Southern Grebo} (\textit{Atlantic-Congo}; Liberia)
\end{styleBody}

\begin{styleBody}
\textbf{C phoneme inventory:} /p b t d c [25F?] k [261?] k[361?]p [261?][361?]b f s h m[325?] m n[325?] n [272?] ŋ ŋ[361?]m l[325?] l w[325?] w j/
\end{styleBody}

\begin{styleBody}
\textbf{Places:} Labial-velar, Bilabial, Dental/Alveolar, Palatal, Velar, Glottal
\end{styleBody}

\begin{styleBody}
\ \ The current study considers only non-geminate consonant phonemes. Geminates are not always given the same treatment as consonants of normal length in phonological descriptions, as they often occur in specific morphological contexts. While there are languages in which consonant gemination is contrastive within morphemes, there are many more in which gemination is contrastive at the lexical level but only in morphologically complex contexts. As a result of this, discussions of gemination are often presented in the context of morphophonological processes, and comprehensive lists of geminate consonants may not be given and sometimes must be inferred. As the hypotheses in this chapter are concerned with phonation, place, and manner articulations, issues of consonant gemination are not considered in any depth. However, the reported presence of gemination in consonant inventories is noted in the coding in Appendix B.
\end{styleBody}

\begin{styleBody}
\ \ The phoneme inventory coding for each language in the sample, along with other notes on the consonant and vowel systems, can be found in Appendix B. In the following sections I present the results of the analyses of consonant and vowel inventories. Because only one of the hypotheses in the current chapter relates to vowel inventories, I present this study first.
\end{styleBody}

\section[4.3 Results: Vowel inventories]{\rmfamily 4.3 Results: Vowel inventories}
\begin{styleBody}
\ \ In this section, I describe vowel inventory patterns in the language sample. The purpose of the study here is twofold: first, to test the hypothesis in (4.2) regarding vocalic nucleus inventory size and syllable structure, and second, to explore general features of vowel contrast with respect to syllable structure complexity. For the latter, there are no explicit hypotheses, but any patterns uncovered will be noted in the event that they might help shed light on the development of syllable structure complexity.
\end{styleBody}

\section[4.3.1 Vowel quality inventory size]{\rmfamily 4.3.1 Vowel quality inventory size}
\begin{styleBody}
\ \ The distribution of vowel quality inventory sizes in the languages of the sample can be found in Figure 4.1.
\end{styleBody}

\begin{styleBody}

\end{styleBody}

\begin{styleBody}
\textbf{Figure 4.1. }Languages of sample distributed according to the number of distinctive vowel qualities in their phoneme inventories.
\end{styleBody}

\begin{styleBody}
\ \ The average number of distinctive vowel qualities for languages in the sample is 5.9. The range is 2-13, with the extremes being Kabardian (two vowel qualities) and Eastern Khanty (13 vowel qualities). Over one-third (34) of the languages have systems with five contrastive vowel qualities. The next most common pattern is for languages to have six contrastive vowel qualities. These proportions are nearly identical to those reported for the 564-language sample in Maddieson (2013c).
\end{styleBody}

\begin{styleBody}
\ \ The mean, median, and range of vowel quality inventory sizes for the languages in each category of syllable structure complexity can be found in Table 4.2.
\end{styleBody}

\begin{flushleft}
\tablefirsthead{}
\tablehead{}
\tabletail{}
\tablelasttail{}
\begin{supertabular}{|m{1.2205598in}|m{1.2205598in}|m{1.2205598in}|m{1.2205598in}|m{1.2205598in}|}
\hline
 &
\multicolumn{4}{m{5.1184597in}|}{\centering{\bfseries Syllable Structure Complexity}}\\\hline
{\fontsize{10pt}{12.0pt}\selectfont\mdseries\upshape \textbf{\textit{N}}\textbf{ vowel qualities}} &
{\centering\bfseries Simple\par}

\centering (\textit{N} = 24 lgs) &
{\centering\bfseries Moderately Complex\par}

\centering (\textit{N} = 26 lgs) &
{\centering\bfseries Complex\par}

\centering (\textit{N} = 25 lgs) &
{\centering\bfseries Highly \par}

{\centering\bfseries Complex\par}

\centering\arraybslash (\textit{N} = 25 lgs)\\\hline
{\bfseries Mean} &
\centering 5.8 &
\centering 6.2 &
\centering 6.2 &
\centering\arraybslash 5.3\\\hline
{\bfseries Median} &
\centering 5 &
\centering 6 &
\centering 5 &
\centering\arraybslash 5\\\hline
{\bfseries Range} &
\centering 4-9 &
\centering 3-13 &
\centering 4-10 &
\centering\arraybslash 2-9\\\hline
\end{supertabular}
\end{flushleft}
\begin{styleBody}
\textbf{Table 4.2. }Vowel quality inventory sizes in each syllable structure complexity category.
\end{styleBody}

\begin{styleBody}
\ \ There are no clear trends with respect to mean or median vowel quality inventory size and syllable structure complexity. Statistical analysis shows no significant correlation between vowel quality inventory size and syllable structure complexity, measured either categorically (\textit{r}(100) = -.078, \textit{p} {\textgreater} .05) or as a sum of maximal syllable margin sizes (\textit{r}(100) = -.038, \textit{p} {\textgreater} .05).
\end{styleBody}

\section[4.3.2 Contrastive vowel length]{\rmfamily 4.3.2 Contrastive vowel length}
\begin{styleBody}
\ \ In this section, I examine patterns of contrastive vowel length in the sample. Here I include all languages reported to have contrastive vowel length for some or all vowel qualities. I also include six languages (Ewe, Fur, Kambaata, Maori, Maybrat, and Nimboran) which are described as having tautosyllabic sequences of identical vowels, and two languages (Carib and Selepet) which are described as having diphthongs consisting of identical vowels. In the latter groups of languages, other non-identical vowel sequences or diphthongs are typically present, and phonetically long vowels are often found in morphologically complex contexts. Together, these facts are often used by authors to justify a sequential analysis rather than a contrastive vowel length analysis. I include these languages in the current analysis because the structures in question are reported to be produced as phonetically long vowels which may meaningfully contrast with short vowels (4.12a-b).
\end{styleBody}

\begin{styleBody}
(4.12) \ \ \textbf{Maybrat} (\textit{Maybrat-Karon}; Indonesia)
\end{styleBody}

\begin{styleBody}
(a)\ \ /puut/
\end{styleBody}

\begin{styleBody}
\ \ [pu[2D0?]t]
\end{styleBody}

\begin{styleBody}
\ \ ‘we climb’
\end{styleBody}

\begin{styleBody}
(b)\ \ /put/
\end{styleBody}

\begin{styleBody}
\ \ [put]
\end{styleBody}

\begin{styleBody}
\ \ ‘leech’
\end{styleBody}

\begin{styleBody}
(Dol 2007: 29)
\end{styleBody}

\begin{styleBody}
\ \ The distribution of contrastive vowel length in the languages of the sample according syllable structure complexity can be found in Table 4.3.
\end{styleBody}

\begin{flushleft}
\tablefirsthead{}
\tablehead{}
\tabletail{}
\tablelasttail{}
\begin{supertabular}{|m{1.8240598in}|m{1.0698599in}|m{1.0691599in}|m{1.0698599in}|m{1.0698599in}|}
\hline
 &
\multicolumn{4}{m{4.51496in}|}{\centering{\bfseries Syllable Structure Complexity}}\\\hline
{\bfseries Vowel length} &
{\centering\bfseries Simple\par}

\centering (\textit{N} = 24 lgs) &
{\centering\bfseries Moderately Complex\par}

\centering (\textit{N} = 26 lgs) &
{\centering\bfseries Complex\par}

\centering (\textit{N} = 25 lgs) &
{\centering\bfseries Highly \par}

{\centering\bfseries Complex\par}

\centering\arraybslash (\textit{N} = 25 lgs)\\\hline
{\bfseries Contrastive} &
\centering 8 &
\centering 8 &
\centering 13 &
\centering\arraybslash 11\\\hline
{\bfseries Tautosyllabic sequences or diphthongs of identical Vs} &
\centering 1 &
\centering 6 &
\centering 1 &
\centering\arraybslash —\\\hline
{\bfseries Non-contrastive} &
\centering 16 &
\centering 12 &
\centering 11 &
\centering\arraybslash 14\\\hline
\end{supertabular}
\end{flushleft}
\begin{styleBody}
\textbf{Table 4.3. }Contrastive vowel length in the sample. Note that Maori (in the Simple category) is reported to have contrastive vowel length for one vowel quality, but tautosyllabic sequences of identical vowels for other vowel qualities.\footnote{\textrm{In Maori, nearly all possible combinations of two vowels can be found to occur tautosyllabically in normal speech. Bauer (1999: 524-8) uses this distribution to justify the analysis of all phonetically long vowels as sequences of identical vowels. However, phonetic [a[2D0?]] has a much higher frequency than would be expected if a sequential analysis were accepted, so Bauer analyzes this particular vowel quality as having contrastive length, while [i[2D0?]], [[25B?][2D0?]], etc. are taken to be sequences.}} Therefore the numbers in the Simple column add up to 25.
\end{styleBody}

\begin{styleBody}
\ \ Over half of the languages in the sample (53/100) do not have contrastive vowel length. Vowel length distinctions are somewhat less common in the languages of the Simple category than those of the other categories; however, this trend is not significant in a chi-square test. In terms of geographic distribution, all six macro-areas examined here have four or more languages with vowel length contrasts, with this pattern being most common in North America (12 languages) and least common in Eurasia (four languages).
\end{styleBody}

\begin{styleBody}
\ \ The above analysis does not distinguish between languages which have vowel length contrasts for all vowel qualities and those that have them for only some. In Figure 4.2 below, I present such an analysis, examining only those languages with contrastive vowel length, including those with tautosyllabic sequences or diphthongs of identical vowels.
\end{styleBody}

\begin{styleBody}

\end{styleBody}

\begin{styleBody}
\textbf{Figure 4.2. }Proportion of languages in each syllable structure complexity category which have contrastive vowel length for some or all vowel qualities (VQs).
\end{styleBody}

\begin{styleBody}
\ \ In all categories, languages with vowel length contrasts are generally more likely to have these contrasts for all rather than just some vowel qualities. However, there is an interesting result with respect to vowel length contrasts in the Simple syllable structure category. Although languages with Simple syllable structure are overall less likely to have vowel length contrasts, if they do have a contrast they are also most likely to have this contrast for all vowel qualities. In fact, this pattern is without exception in the language sample.
\end{styleBody}

\begin{styleBody}
\ \ Below I illustrate the prominent patterns in vowel length distinctions with the vowel inventories of two languages: Rotokas, which has Simple syllable structure and length contrasts for all qualities, and Dizin, which has Complex syllable structure but length contrasts only for a subset of vowels (4.13)-(4.14).
\end{styleBody}

\begin{styleBody}
(4.13)\ \ \textbf{Rotokas} (\textit{North Bougainville}; Papua New Guinea)
\end{styleBody}

\begin{styleBody}
\textbf{V phoneme inventory:} /i e a o u i[2D0?] e[2D0?] a[2D0?] o[2D0?] u[2D0?]/
\end{styleBody}

\begin{styleBody}
(4.14)\ \ \textbf{Dizin} (\textit{Dizoid}; Ethiopia)
\end{styleBody}

\begin{styleBody}
\textbf{V phoneme inventory:} /i e [25B?] [268?] [251?] o u i[2D0?] e[2D0?] [251?][2D0?] o[2D0?] u[2D0?]/
\end{styleBody}

\section[4.3.3 Other vowel contrasts]{\rmfamily 4.3.3 Other vowel contrasts}
\begin{styleBody}
\ \ Here I present analyses of other contrastive properties present in the vowel inventories of the language sample, namely nasalization and phonation contrasts. See Table 4.4 for the distribution of languages in the sample with respect to contrastive vowel nasalization.
\end{styleBody}

\begin{flushleft}
\tablefirsthead{}
\tablehead{}
\tabletail{}
\tablelasttail{}
\begin{supertabular}{|m{1.2205598in}|m{1.2205598in}|m{1.2205598in}|m{1.2205598in}|m{1.2205598in}|}
\hline
 &
\multicolumn{4}{m{5.1184597in}|}{\centering{\bfseries Syllable Structure Complexity}}\\\hline
{\bfseries Vowel nasalization} &
{\centering\bfseries Simple\par}

\centering (\textit{N} = 24 lgs) &
{\centering\bfseries Moderately Complex\par}

\centering (\textit{N} = 26 lgs) &
{\centering\bfseries Complex\par}

\centering (\textit{N} = 25 lgs) &
{\centering\bfseries Highly \par}

{\centering\bfseries Complex\par}

\centering\arraybslash (\textit{N} = 25 lgs)\\\hline
{\bfseries Contrastive} &
\centering 9 &
\centering 4 &
\centering 4 &
\centering\arraybslash 4\\\hline
{\bfseries Non-contrastive} &
\centering 15 &
\centering 22 &
\centering 21 &
\centering\arraybslash 21\\\hline
\end{supertabular}
\end{flushleft}
\begin{styleBody}
\textbf{Table 4.4. }Vowel nasalization contrasts in the sample. 
\end{styleBody}

\begin{styleBody}
\ \ Roughly one-fifth of the languages (21/100) have a vowel nasalization contrast for some or all vowel qualities. This feature is much more common in languages with Simple syllable structure, occurring in over a third of languages in that category, as compared to a much smaller proportion of languages in the other categories; this trend is statistically significant in Fisher’s exact test (\textit{p }= .04). Contrastive vowel nasalization is also strongly associated with particular geographic regions in the current sample: all but four of the languages with this feature are found in Africa, North America, and South America. This distribution closely mirrors the areal patterns noted by Hajek (2013) in a 244-language sample. As compared to the analysis of vowel length contrasts in §4.3.2, there is no clear pattern in the sample with respect to the presence of nasalization contrasts for some or all vowel qualities and syllable structure complexity.
\end{styleBody}

\begin{styleBody}
\ \ We now turn to an analysis of phonation contrasts in the vowel inventory data. These are not common, but do occur in six languages in the sample, listed in Table 4.5 by the specific kind of phonation contrast and syllable structure complexity.
\end{styleBody}

\begin{flushleft}
\tablefirsthead{}
\tablehead{}
\tabletail{}
\tablelasttail{}
\begin{supertabular}{|m{1.2205598in}|m{1.2205598in}|m{1.2205598in}|m{1.2205598in}|m{1.2205598in}|}
\hline
 &
\multicolumn{4}{m{5.1184597in}|}{\centering{\bfseries Syllable Structure Complexity}}\\\hline
{\bfseries Other vowel contrasts} &
\centering{\bfseries Simple} &
\centering{\bfseries Moderately Complex} &
\centering{\bfseries Complex} &
{\centering\bfseries Highly \par}

\centering\arraybslash{\bfseries Complex}\\\hline
Contrastive voicing &
\centering{\itshape Ute} &
\centering{\itshape (Kambaata)} &
\centering{\itshape —} &
\centering\arraybslash{\itshape Tohono O’odham}\\\hline
Contrastive glottalization/creaky voice &
{\centering\itshape Pinotepa Mixtec\par}

\centering{\itshape Sichuan Yi} &
\centering{\itshape Pacoh} &
\centering{\itshape Mamaindê} &
\centering\arraybslash{\itshape —}\\\hline
\end{supertabular}
\end{flushleft}
\begin{styleBody}
\textbf{Table 4.5.} Languages in sample with distinctive phonation contrasts in vowel inventories, according to syllable structure complexity.\footnote{\textrm{The phonological status of voiceless vowels in Kambaata is not fully determined: it is not entirely predictable, but neither is it contrastive in the traditional sense (that is, participating in clear minimal pairs, Treis 2008: 20-22).}}
\end{styleBody}

\begin{styleBody}
\ \ In this very small data set, contrastive phonation in vowel inventories is more likely to be found in languages with Simple or Moderately Complex syllable structure. Nearly every language with a phonation contrast in its vowel inventory also has an additional contrast besides vowel quality: either vowel nasalization (two languages) or vowel length (four languages). The only exception to this trend is Sichuan Yi. I illustrate these patterns with the vowel phoneme inventories of Ute and Mamaindê (4.15)-(4.16).
\end{styleBody}

\begin{styleBody}
(4.15)\ \ \textbf{Ute} (\textit{Uto-Aztecan}; USA)
\end{styleBody}

\begin{styleBody}
\textbf{V phoneme inventory:} /i œ a [26F?] u i[2D0?] œ[2D0?] a[2D0?] [26F?][2D0?] u[2D0?] i[325?] œ[325?] a[325?] [26F?][325?] u[325?]/
\end{styleBody}

\begin{styleBody}
(4.16)\ \ \textbf{Mamaindê} (\textit{Nambiquaran}; Brazil)
\end{styleBody}

\begin{styleBody}
\textbf{V phoneme inventory:} /i e a o u \~{i} \~{e} \~{a} \~{o} \~{u} i[330?] e[330?] a[330?] o[330?] u[330?] \~{i}[330?] \~{a}[330?] \~{u}[330?]/
\end{styleBody}

\section[]{\rmfamily }
\section[4.3.4 Diphthongs and vowel sequences]{\rmfamily 4.3.4 Diphthongs and vowel sequences}
\begin{styleBody}
\ \ In this section I analyze other vocalic nucleus patterns in the sample, specifically patterns of diphthongs and tautosyllabic vowel sequences. This analysis excludes the diphthongs or vowel sequences made up of identical vowels that were included in the analysis of contrastive vowel length in §4.3.2. I group together diphthongs and vowel sequences here because the terms are often used interchangeably to refer to the same or very similar tautosyllabic structures, sometimes even within the same language reference. See Table 4.6 for the distribution of languages according to syllable structure complexity and the presence or absence of diphthongs or vowel sequences.
\end{styleBody}

\begin{flushleft}
\tablefirsthead{}
\tablehead{}
\tabletail{}
\tablelasttail{}
\begin{supertabular}{|m{1.2205598in}|m{1.2205598in}|m{1.2205598in}|m{1.2205598in}|m{1.2205598in}|}
\hline
 &
\multicolumn{4}{m{5.1184597in}|}{\centering{\bfseries Syllable Structure Complexity}}\\\hline
{\bfseries Diphthongs or vowel sequences} &
{\centering\bfseries Simple\par}

\centering (\textit{N} = 24 lgs) &
{\centering\bfseries Moderately Complex\par}

\centering (\textit{N} = 26 lgs) &
{\centering\bfseries Complex\par}

\centering (\textit{N} = 25 lgs) &
{\centering\bfseries Highly \par}

{\centering\bfseries Complex\par}

\centering\arraybslash (\textit{N} = 25 lgs)\\\hline
{\bfseries Present} &
\centering 8 &
\centering 14 &
\centering 8 &
\centering\arraybslash 8\\\hline
{\bfseries Absent} &
\centering 16 &
\centering 12 &
\centering 17 &
\centering\arraybslash 17\\\hline
\end{supertabular}
\end{flushleft}
\begin{styleBody}
\textbf{Table 4.6.} Languages of the sample, distributed according to syllable structure complexity and the presence or absence of diphthongs or tautosyllabic vowel sequences.
\end{styleBody}

\begin{styleBody}
\ \ Languages with Moderately Complex syllable structure are much more likely than languages from the other categories to have diphthongs or vowel sequences (just over half of the languages in this category has these patterns, as compared to roughly one-third of the languages in the other categories).
\end{styleBody}

\begin{styleBody}
\ \ There is a very wide range in the size of diphthong and tautosyllabic vowel sequence inventories in the languages of the sample. Extremely large inventories of diphthongs or vowel sequences, as illustrated by the 23-diphthong system of Selepet, are rare (4.17). Selepet has six vowel qualities, and nearly all possible combinations of vowels are reported to occur, either as diphthongs or sequences of identical vowels (which were included in the analyses in §4.3.2). The modal value for the 38 languages in the sample with diphthongs or tautosyllabic vowel sequences is just two such structures, as illustrated for Telugu (4.18).
\end{styleBody}

\begin{styleBody}
(4.17) \ \ \textbf{Selepet }(\textit{Nuclear Trans New Guinea}; Papua New Guinea)
\end{styleBody}

\begin{styleBody}
\textbf{Diphthongs: }/ie ia i[254?] io iu ei eu ai ae ao au [254?]i [254?]e [254?]o [254?]u oi oe ou ui ue ua u[254?] uo/
\end{styleBody}

\begin{styleBody}
(4.18)\ \ \textbf{Telugu} (\textit{Dravidian}; India)
\end{styleBody}

\begin{styleBody}
\textbf{Diphthongs: }/ai au/
\end{styleBody}

\section[4.3.5 Vocalic nucleus inventories and syllable structure complexity]{\rmfamily 4.3.5 Vocalic nucleus inventories and syllable structure complexity}
\begin{styleBody}
\ \ Here I combine the results of the above analyses in order to determine whether there is any positive correlation between the size of vocalic nucleus inventories and syllable structure complexity, as hypothesized in §4.1.4.
\end{styleBody}

\begin{styleBody}
\ \ This hypothesis was motivated by observations in §3.3.4, where it was found that greater syllable structure complexity was associated with a higher likelihood of a language having complex vocalic nuclei, defined there as long vowels, diphthongs, and/or tautosyllabic vowel sequences. It was assumed that the stronger presence of complex vocalic nuclei might correspond to overall larger vocalic nucleus inventories in languages with more complex syllable structure, revealing a relationship between vowel inventories and syllable structure complexity which has not been previously reported. However, the results in §4.3.2-3 show that certain vowel contrasts show strong patterns with respect to syllable structure complexity which may even out this expected effect. While vowel length contrasts are much more frequent in languages with Moderately Complex, Complex, and Highly Complex syllable structure, it is most common to find that vowel length is contrastive for \textit{all} vowel qualities in languages with Simple syllable structure. Meanwhile, contrastive nasalization and phonation are more commonly found in the vowel systems of languages with Simple syllable structure. Here I examine vocalic nucleus inventories to see whether the hypothesized trend is borne out.
\end{styleBody}

\begin{styleBody}
\ \ In this analysis, I include all distinctive vocalic nucleus patterns reported for each language, including all quality, length, nasalization, and phonation contrasts in addition to diphthong and tautosyllabic vowel sequence patterns. For example, the vocalic nucleus inventory of Budai Rukai is given below (4.19).
\end{styleBody}

\begin{styleBody}
(4.19) \ \ \textbf{Budai Rukai} (\textit{Austronesian}; Taiwan)
\end{styleBody}

\begin{styleBody}
\textbf{Vocalic nucleus inventory: }/i [259?] a u i[2D0?] e[2D0?] a[2D0?] u[2D0?] au ai ia ua/
\end{styleBody}

\begin{styleBody}
\ \ In Table 4.7 I present the mean, median, and range values for vocalic nucleus inventory sizes in the language sample.
\end{styleBody}

\begin{flushleft}
\tablefirsthead{}
\tablehead{}
\tabletail{}
\tablelasttail{}
\begin{supertabular}{|m{1.2205598in}|m{1.2205598in}|m{1.2205598in}|m{1.2205598in}|m{1.2205598in}|}
\hline
 &
\multicolumn{4}{m{5.1184597in}|}{\centering{\bfseries Syllable Structure Complexity}}\\\hline
{\bfseries Vocalic nucleus inventory size} &
{\centering\bfseries Simple\par}

\centering (\textit{N} = 24 lgs) &
{\centering\bfseries Moderately Complex\par}

\centering (\textit{N} = 26 lgs) &
{\centering\bfseries Complex\par}

\centering (\textit{N }= 25 lgs) &
{\centering\bfseries Highly \par}

{\centering\bfseries Complex\par}

\centering\arraybslash (\textit{N} = 25 lgs)\\\hline
{\bfseries Mean} &
\centering 12.7 &
\centering 13.3 &
\centering 12.1 &
\centering\arraybslash 10.4\\\hline
{\bfseries Median} &
\centering 11.5 &
\centering 10.5 &
\centering 9 &
\centering\arraybslash 7\\\hline
{\bfseries Range} &
\centering 4-31 &
\centering 3-31 &
\centering 5-35 &
\centering\arraybslash 3-31\\\hline
\end{supertabular}
\end{flushleft}
\begin{styleBody}
\textbf{Table 4.7.} Mean, median, and range values for vocalic nucleus inventory sizes in sample, by syllable structure complexity.
\end{styleBody}

\begin{styleBody}
\ \ Examining the mean and median values for each category of languages, we find that vocalic nucleus inventory size generally decreases as syllable structure complexity increases, a trend which goes against the prediction of the hypothesis. However, due to the great range in vocalic nucleus inventory size observed throughout the sample, there is ultimately no statistically significant correlation between this feature and syllable structure complexity, measured either categorically (\textit{r}(100) = -.116, \textit{p} {\textgreater} .05) or as a sum of maximal syllable margin sizes (\textit{r}(100) = -.126, \textit{p} {\textgreater} .05). 
\end{styleBody}

\section[4.3.6 Summary of vowel patterns in sample]{\rmfamily 4.3.6 Summary of vowel patterns in sample}
\begin{styleBody}
\ \ While vocalic nucleus inventories may have different prototypical characteristics in languages with different syllable patterns, showing different rates and effects of length, nasalization, diphthongs, and other contrasts, their overall size appears to bear no relation to syllable structure complexity. However, the patterns associating specific contrastive properties of vowels with syllable structure complexity are worth noting. I summarize these findings in Table 4.8.
\end{styleBody}

\begin{flushleft}
\tablefirsthead{}
\tablehead{}
\tabletail{}
\tablelasttail{}
\begin{supertabular}{m{3.16916in}m{3.16986in}}
\hline
{\bfseries Positive trends}

(increases with syllable structure complexity) &
{\bfseries Negative trends}

(decreases with syllable structure complexity)\\\hline
Presence of vowel length contrast &
Vowel length contrast in all vowels\\
 &
Presence of vowel nasalization contrast\\
 &
Presence of vowel phonation contrast\\\hline
\end{supertabular}
\end{flushleft}
\begin{styleBody}
\textbf{Table 4.8.} Properties of vowel inventories showing some relationship to syllable structure complexity.
\end{styleBody}

\begin{styleBody}
\ \ There is no obvious reason why the vowel patterns above should bear any direct relationship to syllable structure complexity. Nevertheless, it is important to note these because they may hold information about the history of the languages in which they are spoken. Patterns of phonemic contrast may be a result of the phonologization of historical phonetic processes, which may themselves be relevant to the development of syllable patterns. I will return to this point in the general discussion of results in §4.5.
\end{styleBody}

\section[4.4 Results: Consonant inventories]{\rmfamily 4.4 Results: Consonant inventories}
\section[4.4.1 Consonant phoneme inventory size]{\rmfamily 4.4.1 Consonant phoneme inventory size}
\begin{styleBody}
\ \ Here I present a basic analysis of consonant phoneme inventory sizes in the language sample and test the hypothesis that as syllable structure complexity increases, so does the size of consonant phoneme inventories.
\end{styleBody}

\begin{styleBody}
\ \ A positive correlation between these features has previously been established in Maddieson (2006, 2013a), which both use a three-point system for categorizing syllable structure complexity. The hypothesis predicts that the trend will hold for the four-category system used in the current work. It also predicts that the effect will be found when syllable structure complexity is measured as a sum of maximal syllable margins, as suggested by Gordon (2016) for the modified 100-language WALS sample.
\end{styleBody}

\begin{styleBody}
\ \ In Table 4.9 I present the mean, median, and range values for the consonant phoneme inventory sizes in the language sample, by category of syllable structure complexity. 
\end{styleBody}

\begin{flushleft}
\tablefirsthead{}
\tablehead{}
\tabletail{}
\tablelasttail{}
\begin{supertabular}{|m{1.2205598in}|m{1.2205598in}|m{1.2205598in}|m{1.2205598in}|m{1.2205598in}|}
\hline
 &
\multicolumn{4}{m{5.1184597in}|}{\centering{\bfseries Syllable Structure Complexity}}\\\hline
{\bfseries C phoneme inventory size} &
{\centering\bfseries Simple\par}

\centering (\textit{N} = 24 lgs) &
{\centering\bfseries Moderately Complex\par}

\centering (\textit{N} = 26 lgs) &
{\centering\bfseries Complex\par}

\centering (\textit{N} = 25 lgs) &
{\centering\bfseries Highly \par}

{\centering\bfseries Complex\par}

\centering\arraybslash (\textit{N} = 25 lgs)\\\hline
{\bfseries Mean} &
\centering 20.8 &
\centering 21.7 &
\centering 21.8 &
\centering\arraybslash 26.1\\\hline
{\bfseries Median} &
\centering 17 &
\centering 21.5 &
\centering 21 &
\centering\arraybslash 23\\\hline
{\bfseries Range} &
\centering 6-55 &
\centering 11-32 &
\centering 12-40 &
\centering\arraybslash 10-54\\\hline
\end{supertabular}
\end{flushleft}
\begin{styleBody}
\textbf{Table 4.9. }Mean, median, and range values for non-geminate consonant phoneme inventory sizes in the language sample, by syllable structure complexity.
\end{styleBody}

\begin{styleBody}
\ \ In Table 4.9, both mean and median values for consonant phoneme inventory size increase with syllable structure complexity. Languages in the Highly Complex category have on average about five more consonants than languages in the Simple category. However, there is a wide range of inventory sizes in the language sample as a whole and within each category of syllable structure complexity, such that there is considerable overlap in this feature among the different categories. In fact, the largest inventory in the sample is found in Hadza, a language with Simple syllable structure (4.20), and the third smallest inventory is found in Mohawk, a language with Highly Complex syllable structure (4.21).
\end{styleBody}

\begin{styleBody}
(4.20) \ \ \textbf{Hadza} (isolate; Tanzania)
\end{styleBody}

\begin{styleBody}
\textbf{C phoneme inventory:} /p[2B0?] p b t[2B0?] t d k[2B0?] k [261?] k[2B0?][2B7?] k[2B7?] [261?][2B7?] [294?] p’ k’ k’[2B7?] k[1C0?] k[1C3?] k[1C1?] m n [272?] ŋ ŋ[2B7?] ŋ[325?][1C0?]’ ŋ[1C0?] ŋ[325?][1C3?]’ ŋ[1C3?] ŋ[325?][1C1?]’ ŋ[1C1?] \textsuperscript{m}p[2B0?] \textsuperscript{m}b \textsuperscript{n}t[2B0?] \textsuperscript{n}d \textsuperscript{ŋ}k[2B0?] \textsuperscript{ŋ}[261?] \textsuperscript{n}t[361?]s \textsuperscript{n}d[361?]z \textsuperscript{[272?]}d[361?][292?] t[361?]s d[361?]z t[361?][283?] t[361?][28E?][325?] d[361?][292?] t[361?]s’ t[361?][283?]’ t[361?][28E?][325?]’ f s [26C?] [283?] l j w [266?]/
\end{styleBody}

\begin{styleBody}
\textbf{\textit{N}}\textbf{ consonant phonemes:} 55
\end{styleBody}

\begin{styleBody}
(4.21)\ \ \textbf{Mohawk} (\textit{Iroquoian}; Canada, USA)
\end{styleBody}

\begin{styleBody}
\textbf{C phoneme inventory:} /t k [294?] d[361?][292?] s h n l j w/
\end{styleBody}

\begin{styleBody}
\textbf{\textit{N}}\textbf{ consonant phonemes:} 10
\end{styleBody}

\begin{styleBody}
\ \ Despite this wide range of variation, there is a positive correlation between consonant phoneme inventory size and syllable structure complexity. When syllable structure complexity is measured categorically, the correlation is weakly positive but not quite significant (\textit{r}(100) = .190, \textit{p} = .06). The result is statistically significant when syllable structure complexity is measured as a sum of maximal syllable margins (\textit{r}(100) = .202, \textit{p} = .04).
\end{styleBody}

\begin{styleBody}
\ \ Thus the hypothesis is supported by the patterns in the current study, even though the sample is much smaller than those of previous works which reported similar findings. When Hadza is excluded as an outlier, the correlations between consonant phoneme inventory size and syllable structure complexity become stronger (\textit{r}(99) = .256, \textit{p} = .01 for syllable structure complexity measured categorically, and virtually the same result when it is measured as a sum of maximal syllable margins).
\end{styleBody}

\begin{styleBody}
\ \ In (4.22)-(4.25) I illustrate typical consonant inventory sizes with a language from each category of syllable structure complexity in the sample.
\end{styleBody}

\begin{styleBody}
(4.22)\ \ \textbf{Rukai} (\textit{Austronesian}; Taiwan)
\end{styleBody}

\begin{styleBody}
\textbf{Syllable Structure Complexity Category:} Simple
\end{styleBody}

\begin{styleBody}
\textbf{C phoneme inventory:} /p b t d [256?] k [261?] t[361?]s v $\theta $ ð s m n ŋ r l [26D?] w j/
\end{styleBody}

\begin{styleBody}
\textbf{\textit{N}}\textbf{ consonant phonemes:} 20
\end{styleBody}

\begin{styleBody}
(4.23)\ \ \textbf{Kim Mun} (\textit{Hmong-Mien}; Vietnam)
\end{styleBody}

\begin{styleBody}
\textbf{Syllable Structure Complexity Category:} Moderately Complex
\end{styleBody}

\begin{styleBody}
\textbf{C phoneme inventory: }/p b t d c [25F?] k [261?] f v $\theta $ s h m n [272?] ŋ l [28E?] w j/
\end{styleBody}

\begin{styleBody}
\textbf{\textit{N}}\textbf{ consonant phonemes:} 21
\end{styleBody}

\begin{styleBody}
(4.24)\ \ \textbf{Lunda} (\textit{Atlantic-Congo}; Democratic Republic of Congo, Angola, Zambia)
\end{styleBody}

\begin{styleBody}
\textbf{Syllable Structure Complexity Category:} Complex
\end{styleBody}

\begin{styleBody}
\textbf{C phoneme inventory:} /p b t d k [261?] t[361?][283?] d[361?][292?] f v s z [283?] [292?] h m n [272?] ŋ l w j/
\end{styleBody}

\begin{styleBody}
\textbf{\textit{N}}\textbf{ consonant phonemes:} 22
\end{styleBody}

\begin{styleBody}
(4.25)\ \ \textbf{Tehuelche} (\textit{Chonan}; Argentina)
\end{styleBody}

\begin{styleBody}
\textbf{Syllable Structure Complexity Category:} Highly Complex
\end{styleBody}

\begin{styleBody}
\textbf{C phoneme inventory:} /p b t[32A?] d[32A?] k [261?] q [262?] [294?] p’ t[32A?]’ k’ q’ t[361?][283?] t[361?][283?]’ s [283?] x $\chi $ m n l r w j/
\end{styleBody}

\begin{styleBody}
\textbf{\textit{N}}\textbf{ consonant phonemes:} 25
\end{styleBody}

\begin{styleBody}
\ \ It was previously noted that the association between syllable structure complexity and consonant phoneme inventory size may reflect the overlapping geographical distributions of the two properties. Specifically, Simple syllable structure is most commonly found in equatorial regions, including Africa, New Guinea, and South America. Complex syllable patterns (as defined in that study) are most often found in northern North America, northern Eurasia, and northern Australia. The latter areas, besides Australia, are associated with large consonant inventories, and the former with small consonant inventories. Therefore the global positive correlation between syllable structure complexity and consonant inventory size may be a reflection of specific genealogical or areal trends within these regions. Maddieson (2006) rejected this scenario, finding that the pattern linking syllable structure complexity and consonant phoneme inventory size was significant within most of the geographical regions examined there. This issue is worth investigating here in a language sample with different design features.
\end{styleBody}

\begin{styleBody}
\ \ Below, I examine whether the association between consonant inventory size and syllable structure complexity, with the added category of Highly Complex, can be found within geographical regions in the current sample. This analysis is necessarily impressionistic: because the already moderate sample size is split roughly evenly between six macro-areas, it is difficult to statistically test the patterns. In Figure 4.3, the median consonant inventory sizes are plotted against syllable structure complexity for the languages in each macro-area.
\end{styleBody}

\begin{styleBody}
\textbf{Figure 4.3. }Median consonant phoneme inventory sizes by syllable structure complexity, for each macro-area represented in the sample.
\end{styleBody}

\begin{styleBody}
\ \ The trends in Figure 4.3 are not all linear; this fact may reflect the small sample size for each region (16 or 17 languages each) as much as it does regional trends. However, we do find that in all but one region — Southeast Asia \& Oceania — there are general, if not monotonic, trends by which the median consonant inventory size of languages in the Highly Complex category is higher than that of languages in the Simple category (or Moderately Complex, in the case of Eurasia). Southeast Asia \& Oceania shows an increase in consonant phoneme inventory size from Simple to Complex syllable structure and then a sharp decline in the Highly Complex category, which is represented by only one language in that region (Semai).
\end{styleBody}

\begin{styleStandard}
\ \ While the patterns here are by no means conclusive, they do suggest that the relationship between consonant phoneme inventory size and syllable structure complexity may occur on regional scales in addition to a global scale, even when the Highly Complex category is included. 
\end{styleStandard}
\end{document}
