data 
tables
check all glyphs  

phdthesis in bib
 
  
  *   “will be” > “is” or “are” in reference to material covered in future sections in the book.
 
  *   One proofreader pointed out that I sometimes use “we” when I mean “I”. There are probably many more cases of this in the text, which ought to be corrected.
  
  *   One proofreader asked me to change special terms which were given in italics to terms in quotations. I did so for a few where it was appropriate, but probably there are more throughout the text which other proofreaders did not note. Is there a way to see all italicized items in the document?
  
  crossrefs
  
  bibs
  
  
  
  *   One proofreader pointed out that sometimes the order of phonemic, phonetic, and morphological transcriptions varies in the examples. When writing, I did this intentionally in order to draw focus to the pattern which was relevant in the example. Are there standards that LSP would rather have me follow? If so, I will need to manually go through the examples and change them.
  *   The same proofreader noted that I do not consistently use italics, slashes, and square brackets for orthographic, phonemic, and phonetic representations, respectively. When writing I used these only when necessary to illustrate a process. However, I understand that this can be confusing for the casual reader. Again, is there a standard you would like me to use? If so, I will need to go through each example and modify it accordingly.
  
  
  *   For tables that span more than one page, can we add some indication that the table will be continued (perhaps at the bottom of the page) and also a shortened caption (e.g. Table X.X, continued) at the top of the new page? 
  
  *   In the original manuscript I used the format ChapterNumber.ExampleNumber to refer to examples (e.g., example 6.3). Some of these had not been corrected in the conversion process (meaning the chapter number was still there); proofreaders caught some of them but there are probably many more.
  
  *   Several proofreaders pointed out that in lists of references, semicolons have replaced all but the last of the commas (for example, the list of references at the end of the paragraph in 2.1.2.1). Is this standard?
  
  *   Readers pointed out that some multipage citations were incomplete, for example Facundes (2002: 88-9) instead of Facundes (2002: 88-89). I corrected these where pointed out, but there are probably many more in the text that have not been corrected.
  
  *   There are several times in the prose in which I split a reference over several lines (I'm sorry, this is a bad habit of my writing). So for example, "Maddieson found that .... (1984: 42)". Obviously this causes problems for linking references. Proofreaders pointed out some of these but there are probably more. Will I need to go through these manually and change them?

  
  
