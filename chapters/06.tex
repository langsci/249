\chapter{Vowel reduction and syllable structure complexity}\label{sec:6}
\section{Introduction and hypothesis}\label{sec:6.1}

  In this and the following chapter, I address the research questions of the book by expanding the phonological survey of syllable structure complexity beyond segmental and suprasegmental properties, considering instead the dynamic, ongoing patterns of sound change that occur in languages with different kinds of syllable structure. Specifically, the study in the current chapter investigates the properties of vowel reduction in languages with different syllable structure complexity. The purpose of looking at vowel reduction, in particular, is that it is a known pathway by which complex syllable patterns develop.

  Vowel reduction, and specifically the weakening of vowels in unstressed syllables, has long been proposed to co-occur with complex syllable structure in the rhythm typology literature \citep{Auer1993}. Vowel reduction is also known from historical and comparative evidence to cause changes in canonical syllable structure. For example, in \ili{Ute}, vowel devoicing has very recently created codas and patterns that may be analyzed as complex consonant clusters in what was previously a CV language (\citealt{Givón2011}; see also discussion in \sectref{sec:3.2.3}). Below I illustrate the potential effects of the most extreme form of vowel reduction, vowel deletion, with an example also mentioned in \sectref{sec:5.1.3}. A recent process deleting pretonic high vowels in certain consonantal environments has dramatically changed the canonical syllable patterns of \ili{Lezgian}. Where only simple onsets used to occur, now onsets of two or three consonants are common. The change is ongoing (\ref{ex:6.1}a), but its directionality is apparent when modern invariant forms are compared to conservative standard spelling (\ref{ex:6.1}b).

\ea\label{ex:6.1}
  \etriple{Lezgian}{Nakh-Daghestanian}{Azerbaijan, Russia}
  
\ea\relax  [t͡ʃʰiˈneba] {\textasciitilde} [ˈt͡ʃʰneba]\\
\glt `secretly’

\ex \begin{tabular}[t]{@{}l@{\hspace{4\tabcolsep}}l@{\hspace{4\tabcolsep}}l}
        Standard spelling  & Modern form  & Gloss\\
        \textit{xizan}    &  /χzan/    &  ‘family’\\
        \textit{šutq’unun}  &  /ʃʷtʰq’unun/ &   ‘press out’\\
    \end{tabular}\\
\citep[36--38]{Haspelmath1993}
\z
\z

  Instrumental evidence indicates that this deletion process continues to be high\-ly productive in the language (\citealt{ChitoranBabaliyeva2007}). This acoustic study also shows that vowel deletion does not occur spontaneously, but is the result of an incremental reduction cline. Initial stages of this process include devoicing and reduced duration of the affected vowel before it has reduced to the point that one might consider it to be deleted. Furthermore, there is evidence that this process involves not only reduction of the vowel gesture, but also its overlap with that of the preceding consonant: the articulatory characteristics of the vowel may persist as a secondary articulation of the preceding consonant (note the labialization in the second example in \ref{ex:6.1}b).

  A major goal in the current work is to identify paths by which highly complex syllable structure, and the large consonant clusters associated with it, develop over time. The evidence from \ili{Lezgian}, which itself now has syllable structure that puts it in the Highly Complex category in this study, as well as languages with similar patterns, indicates that vowel reduction is at least one source for the development of complex tautosyllabic consonant clusters in a language. The continued productivity of this particular process in \ili{Lezgian} (\ref{ex:6.1}a) also suggests that vowel reduction patterns may continue to persist even after they have altered the syllable patterns of a language. 

  As noted in \sectref{sec:5.1.3}, there is now additionally a tendency in the language for post-tonic vowels to be deleted in certain consonantal environments \REF{ex:6.2}.

\ea\label{ex:6.2}
  \etriple{Lezgian}{Nakh-Daghestanian}{Azerbaijan, Russia}

diˈdedilaj {\textasciitilde} diˈdedlaj\\
\glt ‘from mother’
\citep[40]{Haspelmath1993}
\z

Haspelmath reports that the process in \REF{ex:6.2} is mostly restricted to inflectional suffixes; though there are other contexts in which it occurs, the precise phonological conditioning is difficult to specify. The processes illustrated by \REF{ex:6.1} and \REF{ex:6.2} share similar properties: both are conditioned by stress and the consonantal environment and both affect predominantly high vowels. An interesting question to consider here is whether these processes share a common motivation. That is, though they are distinct patterns, the similarities in their conditioning and outcomes may reflect an increasing general phonetic tendency towards reduction of high vowels, and/or their overlap with consonantal gestures, in unstressed syllables in the language.

  The observations above suggest several points for further investigation. First of all, because vowel reduction processes may persist in a language after they have altered canonical syllable patterns, it is reasonable to hypothesize that languages with complex syllable structure will be more likely to have ongoing vowel reduction processes than those with simpler syllable structure. The impressionistic descriptions of the phonetic characteristics of languages with highly complex syllable structure in \sectref{sec:1.3.3}, in which unstressed syllables are described as being “squeezed together” and unstressed vowels as being obscure or dropped entirely, suggest that this hypothesis has merit. Second, following the observations regarding the two processes in \ili{Lezgian}, we might expect vowel reduction to also be more prevalent \textit{within} languages with complex syllable patterns. This could manifest in a higher number of distinct vowel reduction patterns within those languages. Finally, it is reasonable to suppose that the generally higher prevalence of vowel reduction in languages with more complex syllable structure may be accompanied by more extreme outcomes of these processes, given that the syllable patterns may have come about through similarly extreme outcomes of vowel reduction at some point in the history of those languages. The analysis in \sectref{sec:5.4.3.1}, in which it was shown that vowel deletion was frequently an outcome of unstressed vowel reduction in languages from the non-Simple categories, would support this idea. With these points in mind, I present the hypothesis for the current chapter in \REF{ex:6.3}.

\ea\label{ex:6.3}
  As syllable structure complexity increases, languages will show stronger effects of vowel reduction, in terms of both prevalence of processes and final outcomes of processes.
\z

  In the study presented here, I test this hypothesis by analyzing the overall prevalence of vowel reduction in the sample. I also analyze the specific characteristics of the affected vowels, conditioning environments, and outcomes associated with vowel reduction in the sample. Trends in these parts of vowel reduction patterns may shed light on the diachronic development of highly complex syllable structure, as well as inform our understanding of synchronic phonetic tendencies in these languages.

  Recall that in \chapref{sec:5} there was an analysis of unstressed vowel reduction which established that these processes occur in a much smaller percentage of languages in the Simple category than in the other three. It is important to note that the following analyses are not limited to vowel reduction conditioned by stress. Since vowel reduction with any kind of conditioning environment may be relevant to the development of syllable structure complexity, the scope of this chapter is much broader than that of the analysis in \sectref{sec:5.4.3}.

\section{ Methodology}\label{sec:6.2}
\subsection{ Patterns considered}\label{sec:6.2.1}

  The term ``vowel reduction'' is most often used in the literature to refer to a process which affects vowel quality, typically in unstressed environments. A prototypical manifestation of this is the movement of unstressed vowels closer to the ``neutral'' central area of the vowel space, e.g. \ili{English} \textit{Rosa’s roses} [\textit{ˈ}ɹoʊzəz \textit{ˈ}ɹoʊzɨz]. \citet[1]{Crosswhite2000}, who defines vowel reduction more narrowly as “[t]he neutralization of two (or more) phonemic vowels when unstressed,” distinguishes two types of vowel reduction in terms of their perceptual outcomes: prominence reduction and contrast enhancement. In prominence-reducing vowel reduction, phonemic contrasts are neutralized to low-sonority vowels, specifically vowels in the mid central region and high vowels (see \ili{English} example above). This has the effect of restricting the entire vowel space in unstressed syllables to a smaller (usually higher and/or more central) region. In contrast-enhancing vowel reduction, vowel contrasts in unstressed syllables are neutralized in such a way as to preserve the peripheral contrasts in the vowel space. For example, in \ili{Luiseño}, mid vowels are raised to their high counterparts in unstressed syllables: \textit{ˈt͡ʃoka} ‘limp, be lame’ > \textit{t͡ʃuˈkat͡ʃkaʃ} ‘limping’ (\citealt{MunroBenson1973}: 19). This has the effect of reducing the five-vowel system /i e a o u/ to a three-vowel system /i a u/ in unstressed syllables. While there are many language-specific studies of vowel reduction (e.g. \citealt{Lindblom1963} for \ili{Swedish}, \citealt{PadgettTabain2005} for \ili{Russian}), large-scale typological studies of the phenomenon are rare and have largely been limited specifically to this issue of neutralization of phonemic contrasts (\citealt{Crosswhite2001,Crosswhite2004,Barnes2006}). Interestingly, a crosslinguistic study of phonetic vowel reduction in 80 languages by \citet{KapatsinskiEtAl2019} found that vowel quality overwhelmingly tends to be centralized, and occasionally raised, as a result of reduction, but did not find synchronic evidence for contrast-enhancement involving both the raising and lowering of vowels to peripheral qualities.

  The current study does not limit itself to phonemic neutralization or reduction of vowel quality in its examination of vowel reduction processes. As indicated by the discussion of the \ili{Lezgian} vowel reduction pattern above, vowel deletion is an incremental process which may involve many different forms of reduction. While the cline to vowel deletion in \ili{English} typically involves vowel quality reduction (e.g. \textit{potato} [pʰoʊˈtʰeɪɾoʊ] > [pʰəˈtʰeɪɾoʊ] > [ˈpʰtʰeɪɾoʊ]), the \ili{Lezgian} example shows that other weakening effects such as devoicing and shortening may be involved in this process (\citealt{ChitoranBabaliyeva2007}). Thus a principled way to approach the current study is to consider any case of vowel weakening to be potentially informative in piecing together the development of highly complex syllable structure.

  The vowel reduction processes examined in the current study encompass any kind of lenition, that is, weakening, of a vowel, in an approach very similar to that taken by \citet{KapatsinskiEtAl2019}. In determining what constitutes a weakening of a vowel, I appeal to phonological models in which sound change is understood in terms of articulatory gestures (\citealt{BrowmanGoldstein1992b,MowreyPagliuca1995}). From this point of view, vowel reduction involves a decrease in the magnitude or duration of vocalic gestures, and/or overlap of vocalic gestures by gestures associated with neighboring sounds, which may have similar outcomes. Thus vowel reduction may involve reduced tongue body displacement, which would produce a change in quality, but it could also involve reduction or loss of the glottalic gesture (devoicing), temporal reduction of the vocalic gestures (shortening), and other effects. We therefore define vowel reduction as any process resulting in 
  (a) the reduction in duration, quality, voicing, or any other property of a vowel, or 
  (b) the vowel no longer having any acoustic manifestation. While the latter phenomenon is typically described as vowel ``deletion,'' it is important to note that gestural overlap with adjacent sounds may result in a vowel no longer being audible, but still having an articulatory trace (cf. \citealt{BrowmanGoldstein1990} for consonant deletion).

  Another complication with traditional uses of the term ``vowel reduction'' is that it may apply to many different kinds of processes, including those limited to specific morphological paradigms and cases of idiosyncratic reduction which occur only in highly frequent words or phrases. In fact both such patterns were frequently found in the language sample (\ref{ex:6.4}--\ref{ex:6.5}).

\ea\label{ex:6.4}
   \etriple{Burushaski}{isolate}{Pakistan}

When the causative prefix is attached to a verb stem, long vowels in the stem frequently shorten, and /ɛ/ tenses and raises to /i/.

\textit{-dɛlʌs} > \textit{ʌdilʌs} ‘make jump’
\citep[1030]{Anderson1997}
\z

\ea\label{ex:6.5}
\etriple{Grass Koiari}{Koiarian}{Papua New Guinea}

The phrase \textit{ego tonitoniva} `very long’ is often produced as \textit{ego tontoniva} in rapid speech.
\citep[7]{Dutton1996}
\z

  Such examples are certainly important in enriching our understanding of vow\-el reduction, attesting to the strong effect that usage-based factors such as analogy, frequency, and automation can have on the sound system of a language, and illustrating the complex intertwining of phonological and morphosyntactic patterns that occur in natural language use and a speaker’s representation of the language \citep{Bybee2001}. However, the analysis of such patterns presents complications. The interpretation of patterns limited to morphological paradigms may be complicated by such factors as inversion or telescoping (\citealt{Vennemann1972,Hyman1975}), in which the chain of developments is obscured or reversed. An example of a synchronic misinterpretation of this sort has been noted for \ili{Lezgian}. \citet{Yu2004} shows that an apparent synchronic process of word-final obstruent voicing in the language is not plausible based on morphophonological and comparative evidence, and that it is more likely that a sequence of processes has devoiced the corresponding word-internal alternants. Similarly, reduction in highly frequent forms, also known as special reduction, has been found in the research to not be entirely comparable to other productive processes of vowel reduction, being more extreme in its effects \citep{BybeeEtAl2016}. For these reasons, vowel reduction patterns limited to specific morphological paradigms and highly frequent forms will be omitted from the current study. Instead, only cases of phonetically or phonologically conditioned vowel reduction are considered, on the basis that these are at least somewhat transparent in their conditioning environments and effects, and productive in the languages for which they have been reported. In \xxref{ex:6.6}{ex:6.9}, I illustrate some of the processes considered in the analysis.

\ea\label{ex:6.6}
\etriple{Pech}{Chibchan}{Honduras}
In rapid speech, vowels in unstressed syllables are sometimes devoiced between voiceless consonants.\\
/sik\`{ĩ}ko/\\\relax
[si̥k\`{ĩ}ko]\\
\glt ‘church’
\citep[18]{Holt1999}
\z

\ea\label{ex:6.7}
\etriple{Kim Mun}{Hmong-Mien}{Vietnam}

Long vowels are shortened and produced with level tone in non-word-final syllables. 
{\begin{multicols}{2}
\ea  /ɡjaːŋ\textsuperscript{35}/

[ɡjaːŋ\textsuperscript{35}]\\
\glt ‘tree’

\ex  /ɡjaːŋ\textsuperscript{35}θɪn\textsuperscript{52}/

[ɡjaŋ\textsuperscript{33}θɪn\textsuperscript{52}]\\
\glt ‘tree trunk’
\citep[117]{Clark2008}
\z\end{multicols}}
\z\pagebreak

\ea\label{ex:6.8}
  \etriple{Alamblak}{Sepik}{Papua New Guinea}

A tense mid front vowel /e/ may be realized as lax [ɛ] in unstressed syllables.

/ˈmetet/

[ˈmetɛt]\\
\glt ‘she is a woman’
\citep[38]{Bruce1984}
\z

\ea\label{ex:6.9}
   \etriple{Karok}{isolate}{USA}

An unaccented word-initial short vowel preceding two consonants may be lost following a pause. 

/iʃpuk/

[ʃpuk]\\
\glt ‘money’
\citep[53]{Bright1957}
\z

  A few types of phonetically- or phonologically-conditioned vowel reduction processes were excluded from the current study. An extremely common type of process in the language sample involved a centralization, laxing, or raising of vowel quality solely as an effect of a specific consonantal environment, as illustrated in \REF{ex:6.10}.

\ea\label{ex:6.10}
\etriple{Maybrat}{Maybrat-Karon}{Indonesia}

A high front vowel /i/ may be realized as high central half-close unrounded vowel [ɪ] when preceding the velar stop /k/.

/manik/

[manɪk]\\
\glt ‘oil’
\citep[15]{Dol2007}
\z

Such processes are not clear examples of vowel reduction, and may be better analyzed as place assimilation of the vowel to the consonant. These were excluded from the present study. In cases in which a reduction in vowel quality was conditioned by another factor in addition to the consonantal environment, such as word position or stress environment, the process was included as a case of vowel reduction. On the other hand, processes involving a reduction in voicing, duration, or other vocalic properties solely as an effect of the consonantal environment were included in the present analysis, but were generally rare.

  As mentioned in \sectref{sec:5.2.1}, where word stress is described as having longer vowel duration as a phonetic correlate, authors sometimes describe the relatively short\-er length of all unstressed syllables as vowel reduction. Such patterns have not been included here as vowel reduction. What has been included are vowel shortening processes in which the reduced length of an unstressed vowel is shorter than what would normally be expected for unstressed vowels, for instance, extra shortening of unstressed vowels in pretonic position as compared to other unstressed positions.

  Finally, processes of vowel harmony which might involve vowel laxing or raising were excluded from the present analysis. Also excluded were cases of vowel deletion conditioned by the presence of an adjacent vowel (i.e. hiatus avoidance), and vowel coalescence or merger.

\subsection{Determining what constitutes a process}\label{sec:6.2.2}

  In any analysis of dynamic processes within or across languages, potential meth\-odological problems arise from issues of how and where to draw the lines which divide the holistic sound pattern of the language into discrete processes. In the current study, it was important to strike a balance between capturing similarities in patterns of vowel reduction and recognizing potentially important differences in those patterns. This sometimes required a reinterpretation of the patterns as they have been reported in the descriptive materials.

  Where more than one vowel was found to reduce in the same way in the same environment, these patterns were grouped together as a single process. Patterns were coded as separate processes when differences in the conditioning environments or outcomes were reported for different affected vowels or groups of vowels. For example, the pattern in \REF{ex:6.11} below was split into two processes due to the slightly different conditioning environments reported for the two affected high vowels. Other aspects of the processes have been ignored here to simplify the exposition.

\ea\label{ex:6.11}
\etriple{Cocama-Cocamilla}{Tupian}{Peru}

\ea  The high back vowel /u/ may be produced as [o] word-finally.

/itimu/

[itimo]\\
\glt ‘liana sp.’

\ex   The high front vowel /i/ may be produced as [e] word-finally following an approximant consonant.

/t͡suwi/

[t͡suwe]\\
\glt ‘tail’

(\citealt{VallejosYopán2010}: 109--110)
\z
\z

  Issues of regularity and speech style were also considered to be important factors in differentiating processes from one another. In the example below \REF{ex:6.12}, a pattern of vowel lengthening in \ili{Doyayo} has been split into two processes based on differences in its regularity in two similar conditioning environments. Patterns like these were not very common in the data.

\ea\label{ex:6.12}
\etriple{Doyayo}{Atlantic-Congo}{Cameroon}

\ea A long vowel is optionally shortened preceding a coda of two or three consonants.

\ex  A long vowel is obligatorily shortened preceding a coda of four consonants.

(\citealt{WieringWiering1994}: 22)
\z
\z

  Some patterns in the language sample involved the optional reduction or deletion of a vowel or group of vowels in some specific conditioning environment. In such cases, the pattern was coded as a single process with two optional outcomes: reduction (in whatever way specified by the source) or deletion, as in the \ili{Lelepa} example below \REF{ex:6.13}.

\ea\label{ex:6.13}
\etriple{Lelepa}{Austronesian}{Vanuatu}

In word-final position following a consonant, high vowels /i u/ and mid back vowel /o/ may be deleted or devoiced.

/nati/

[nati] {\textasciitilde} [nati̥] {\textasciitilde} [nat]\\
\glt ‘banana’

(\citealt{Lacrampe2014}: 15, 64--65)
\z

\subsection{Coding}\label{sec:6.2.3}\largerpage

  As with most typological studies of moderate to large size, the data collection for this study relies on patterns reported in reference grammars and other descriptive materials. Written references, which are often heavily reliant on elicited data, are of course a poor substitute for multi-modal corpora of natural language use. The written word is also a particularly poor medium for the study of speech sound patterns. Typological studies of phonological processes are further complicated by issues of analysis. The direction of a process relies on the analysis of the author and their judgment of what a likely process is, based on the synchronic evidence at hand in the language. Characterizations of variation may be phonetically imprecise and based on impressions rather than instrumental measures, and the degree of detail and patterns attended to may reflect the interest and/or native language biases of the author. In addition to these potential issues, there are many other complications of descriptive and typological work relating to the speech styles and varieties represented in language reference materials, as discussed in \chapref{sec:2}.

  It is expected that the group of vowel reduction processes collected from the sources will reflect some or all of the above problems. However, in a sample of 100 diverse languages ranging from highly endangered languages with a handful of speakers to well-documented languages with institutional status, described by hundreds of researchers from various backgrounds, we expect strong crosslinguistic trends to rise above the “noise“ of the aforementioned complications.

  Each process of vowel reduction was coded for three structural factors: the vowels affected, the properties of the environment reported to cause the reduction, and the outcome of the reduction process. I describe details of the coding procedure here.

  The \textit{affected vowels} were coded according to their phonetic descriptions in the references consulted and the corresponding IPA symbols. Where the affected vowels formed a coherent natural class with respect to the vowel inventory of the language this was also noted (e.g. \textit{long vowels}, \textit{high vowels}, \textit{all vowels}).

  The \textit{conditioning environment} was coded to reflect what phonetic or phonological factors contributed to the occurrence of the process: consonantal environment, word environment, word stress environment, and/or phrase (or utterance) environment. In coding the conditioning environment, sometimes a reinterpretation of the process as reported was required. Where word and phrase/utterance environments were confounded, the phrase/utterance domain was considered to be the conditioning factor, for instance, in cases where an author reported that word-final vowels are reduced at the end of an utterance. Where both word and phrase/utterance environments clearly contributed to the process, then both were coded as conditioning factors, for instance, when word-final vowels are deleted phrase-medially but not phrase-finally. Similarly, when word stress and word position were potentially confounded, then the stress environment was considered the sole conditioning factor. An example of this would be when an author reported that the antepenultimate vowel of a word is reduced preceding a stressed syllable, but the language was also reported to have fixed penultimate word stress.

  The \textit{outcome} was coded according to the nature of the reduction vis-à-vis the phonetic definitions of the affected vowels. Outcomes included reduction in vowel duration, reduction in vowel quality (usually laxing or centralization), devoicing, deletion, and other rarer effects such as tone leveling or glottalization of the vowel. Following the research discussed in \sectref{sec:6.2.1}, and especially the results of \citet{KapatsinskiEtAl2019}, an outcome of vowel raising was considered to be a reduction in vowel quality if it occurred in an unstressed syllable or if it was explicitly described by the author as a process of vowel reduction.

  Insofar as such information was reported, processes were also coded for factors of regularity (e.g. regular or optional application), speech style (e.g. normal, rapid, or casual speech), and sociolinguistic variation (e.g. age of speakers).

  All vowel reduction processes considered in this chapter can be found in Appendix~\ref{sec:Appendix:B}.

\section{Results}\label{sec:6.3}

  Here I present a quantitative analysis of vowel reduction processes occurring in the language sample. These analyses test the hypothesis formulated in \sectref{sec:6.1}: as syllable structure complexity increases, languages will show stronger effects of vowel reduction. In §§\ref{sec:6.3.1}--\ref{sec:6.3.2}, the relative prevalence of vowel reduction processes is examined with respect to syllable structure complexity. In §§\ref{sec:6.3.3}--\ref{sec:6.3.5}, analyses are presented showing how trends in the affected vowels (\sectref{sec:6.3.3}), conditioning environments (\sectref{sec:6.3.4}), and outcomes of vowel reduction processes (\sectref{sec:6.3.5}) differ among languages with different syllable structure complexity. In \sectref{sec:6.3.6} I present a holistic analysis of the reduction processes in the sample. In \sectref{sec:6.3.7} I summarize the patterns observed and discuss how they support the hypothesis.

\subsection{Languages with vowel reduction}\label{sec:6.3.1}\largerpage[2]

  Out of the 100 languages in the sample, 72 were found to have vowel reduction processes as defined in \sectref{sec:6.2.1}. \tabref{tab:6.1} shows the distribution of languages in the sample with respect to syllable structure complexity and the presence or absence of vowel reduction processes.

\begin{table}[H]
\begin{tabular}{lcccc}
\lsptoprule
 & \multicolumn{4}{c}{Syllable structure complexity}\\\cmidrule(lr){2-5}
 \textit{N} languages with: & S & MC & C & HC\\
                            & \textit{N} = 24 & \textit{N} = 26 & \textit{N} = 25 & \textit{N} = 25\\\midrule
 Vowel reduction & 15 & 18 & 19 & 21\\
 No vowel reduction & 9 & 8 & 6 & 4\\
\lspbottomrule
\end{tabular}
\caption{\label{tab:6.1}Languages of the sample distributed according to syllable structure complexity and presence or absence of vowel reduction processes, as reported in sources.}
\end{table}

  In all four categories of syllable structure complexity, languages are more likely than not to be reported to have vowel reduction processes. Here the proportion of languages showing vowel reduction steadily increases with syllable structure complexity. Thus in this initial analysis, we find support for the hypothesis that languages with more complex syllable structure will show stronger effects of vowel reduction. Note that these results differ from the one reported for unstressed vowel reduction in \sectref{sec:5.4.3}, in which the salient pattern was the contrast between the low reduction rate for the Simple category and the roughly similar high reduction rates for the other three categories.

\subsection{Number of distinct vowel reduction processes present}\label{sec:6.3.2}

  Here I analyze the number of distinct vowel reduction processes present in the languages of the sample. The median and range in number of processes for each syllable structure complexity category are presented in \tabref{tab:6.2}.

\begin{table}
\begin{tabularx}{\textwidth}{Qcccc}
\lsptoprule
 & \multicolumn{4}{c}{Syllable structure complexity}\\\cmidrule(lr){2-5}
 \textit{N} distinct vowel reduction processes: & S & MC & C & HC\\
                            & 24 lgs. & 26 lgs. & 25 lgs. & 25 lgs.\\\midrule
 Median & 1 & 1.5 & 2 & 2\\
 Range & 0--5 & 0--5 & 0--7 & 0--7\\
\lspbottomrule
\end{tabularx}
\caption{\label{tab:6.2}Languages of sample distributed according to syllable structure complexity and median and range in number of distinct vowel reduction processes.}
\end{table}

  The trends in \tabref{tab:6.2} indicate that languages with differing syllable structure complexity also differ with respect to the number of distinct vowel reduction processes occurring. Though the trend in the median number of vowel reduction processes is not particularly informative, the languages in the Simple and Moderately Complex categories have a narrower range in number of processes than the languages in the Complex and Highly Complex categories. \figref{fig:6.1} shows the percentage of languages in each category which have zero, one, two, and three or more distinct vowel reduction processes.

  
\begin{figure}
\caption{\label{fig:6.1}Percentage of languages in each syllable structure complexity group with given number of distinct vowel reduction processes.}
\begin{tikzpicture}
\pgfplotstableread{data/fig61.csv}{\table}
    \pgfplotstablegetcolsof{\table}
    \pgfmathtruncatemacro\numberofcols{\pgfplotsretval-1}
            \begin{axis}[easterdaystacked,
                                xticklabels={S,MC,C,HC},
                        ]
            \foreach \i in {1,...,\numberofcols} {
                \addplot+[
                    /pgf/number format/read comma as period, fill
                    ] table [x index={1},y index={\i},x expr=\coordindex] {\table};
                \pgfplotstablegetcolumnnamebyindex{\i}\of{\table}\to{\colname} % Adding column headers to legend
                \addlegendentryexpanded{\colname}
            }
            \end{axis}                                                                           
\end{tikzpicture}
\end{figure}

  As syllable structure complexity increases, so does the number of languages having more than one vowel reduction process operating synchronically. In fact, languages at the far end of the syllable structure complexity scale, those in the Highly Complex category, are much more likely to have two or more vowel reduction processes (18 languages) than to have one or none (seven languages). Statistical tests show there is a significant positive correlation between the number of distinct vowel reduction processes reported per language and syllable structure complexity, measured both categorically ($r(100) = 0.251$, $p = 0.01$) and as a sum of maximal syllable margin sizes ($r(100) = 0.283$, $p = 0.004$).

  The analysis in the previous section revealed that the proportion of languages having any vowel reduction processes increases as syllable structure complexity increases. The results presented here point to a greater prevalence of vowel reduction in languages with more complex syllable structure, in that larger numbers of distinct processes tend to be present in these languages. This lends further support to the hypothesis being tested in this study. The data can also be interpreted as pointing to greater variability in vowel reduction patterns in languages with more complex syllable structure. Recalling the criteria presented in \sectref{sec:6.2.3} for determining what constitutes a distinct process, the trends here reflect a higher degree of variation in affected vowels, conditioning environments, outcomes, and regularity of vowel reduction patterns in languages with more complex syllable structure.

  In \tabref{tab:6.3}, the 178 vowel reduction processes collected from the language sample are distributed according to the syllable structure complexity of the languages in which they occur. In the following sections, trends in the affected sounds, conditioning environments, and outcomes of these processes will be analyzed.\largerpage[-2]
 
\begin{table}
\begin{tabularx}{\textwidth}{Qcccc}
\lsptoprule
 & \multicolumn{4}{c}{Syllable structure complexity}\\\cmidrule(lr){2-5}
                            & S & MC & C & HC\\
                            & 24 lgs. & 26 lgs. & 25 lgs. & 25 lgs.\\\midrule
 \textit{N} reported vowel reduction processes & 24 & 49 & 48 & 57\\
\lspbottomrule
\end{tabularx}
\caption{\label{tab:6.3}Distinct vowel reduction processes in sample, distributed according to the syllable structure complexity of the languages in which they occur.}
\end{table}

\subsection{Affected vowels}\label{sec:6.3.3}

  As described in \sectref{sec:6.2.3}, the vowels affected by each vowel reduction process in the data were coded according to their phonetic descriptions in the references consulted, and where appropriate, according to their natural class with respect to the composition of the language’s vowel inventory. In \tabref{tab:6.4}, the vowel reduction processes in the data are distributed according to the vowels or groups of vowels affected.

\begin{table}
\begin{tabularx}{\textwidth}{Qlccccrr}
\lsptoprule
 & \multicolumn{6}{c}{Syllable structure complexity} & \textit{Total}\\\cmidrule(lr){2-7}
 {Affected vowels} & & S & MC & C & HC &\\
    \textit{N} processes               & & 24  & 49  & 48  & 57& \\\midrule
 {all vowels}                    & & 12 & 10 & 17 & 21   & & \textit{60}\\
 {high vowels}                   & & 7 & 15 & 10 & 7     & & \textit{39}\\
 {low vowel} /a/ {or} /ɑ/        & & -- & 5 & 4 & 2     & & \textit{11}\\
 {mid central vowel} /ə/         & & -- & 3 & -- & 10   & & \textit{13}\\
 {short vowels}                  & & -- & 6 & 5 & 3     & & \textit{14}\\
 {long vowels}                   & & 3 & 5 & 4 & 4      & & \textit{16}\\
 {other}                         & & 2 & 5 & 8 & 10     & & \textit{25}\\
\lspbottomrule
\end{tabularx}
\caption{\label{tab:6.4}Vowel reduction processes in sample, distributed according to affected vowels and syllable structure complexity of languages in which they occur.}
\end{table}

The categories of affected vowels listed in \tabref{tab:6.4} capture the clearest patterns in the data set as a whole. For the sake of simplicity, the category of \textit{high vowels} includes processes which affect all or some high vowels in a language along with processes targeting just one high vowel, such as /i/ or /u/. The \textit{other} category is fairly heterogeneous, and includes processes affecting groups such as non-high vowels, front and high vowels, high and mid vowels, /e/, and so on.

  For languages in all four categories of syllable structure complexity, vowel reduction processes affecting \textit{all vowels} are frequent. In fact, this is the dominant trend for every category except for Moderately Complex. What is more interesting about the data presented in \tabref{tab:6.4} are the secondary and outlier trends with respect to affected vowels and syllable structure complexity. \textit{High vowels} are much less likely to be affected by vowel reduction processes in languages with Highly Complex syllable structure as compared to languages in the other syllable structure complexity categories.\footnote{{The \ili{Lezgian} processes used to illustrate vowel reduction processes in \sectref{sec:6.1} are in fact some of the very few processes targeting high vowels in the Highly Complex category.}} Vowel reduction processes in languages with Simple syllable structure do not target \textit{short vowels} in those languages which have vowel length distinctions, but they do target \textit{long vowels.} This is an unusual trend compared to the patterns in the languages with more complex syllable structure, though it could also be a random effect due to the small number of vowel reduction processes in the Simple category. Finally, perhaps the most interesting feature of the data presented above is that reduction processes in which \textit{schwa} is the sole affected vowel occur almost entirely in languages of the Highly Complex category. Even more strikingly, this is the second most frequent affected vowel category in that group of languages.

  It is important to always bear in mind that any reported phonemic inventory is a product of an author’s analysis. The symbol [ə] is conventionally used as cover symbol for any neutral vowel in the mid central region of the vowel chart \citep[280]{Laver1994}, and phonetic descriptions of such neutral vowels are often impressionistic and unaccompanied by instrumental data in reference materials. To complicate matters, a common outcome of vowel reduction is a vowel produced somewhere in the mid central region. Thus, proving that /ə/ is indeed a contrastive sound, and not a reduced variant of another vowel, is not always a straightforward process in the phonemic analysis of a language. With these caveats in mind, we consider the distribution of languages in the sample which are demonstrated through the analysis of minimal pairs, stress patterns, or other methods to have contrastive /ə/ in their vowel phoneme inventories, with respect to the presence or absence of vowel reduction processes affecting /ə/ specifically (\tabref{tab:6.5}).

\begin{table}
\begin{tabularx}{\textwidth}{Qcccc}
\lsptoprule
 & \multicolumn{4}{c}{Syllable structure complexity}\\\cmidrule(lr){2-5}
  V reduction processes affecting /ə/ & S & MC & C & HC\\
  & \textit{N} = 3 & \textit{N} = 9 &  \textit{N} = 5 & \textit{N} = 8\\\midrule
 Present & -- & 1 & -- & 6\\
 Absent & 3 & 8 & 5 & 2\\
\lspbottomrule
\end{tabularx}
\caption{\label{tab:6.5}Languages in sample reported to have phonemic /ə/, distributed according to syllable structure complexity and presence or absence of vowel reduction processes affecting /ə/ specifically. The trend in Highly Complex languages is highly significant when compared against the combined trend in the Simple, Moderately Complex, and Complex languages ($p<0.001$ in Fisher’s exact test).}
\end{table}

  Contrastive /ə/ is reported in the vowel phoneme inventories of 25 languages in the sample from all four categories of syllable structure complexity, though it is somewhat less frequent in languages with Simple syllable structure. The 13 vowel reduction processes affecting /ə/, as shown in \tabref{tab:6.5} above, are distributed as follows: three occur in \ili{Paiwan} (Austronesian), a language of the Moderately Complex category. The remaining ten processes occur in six diverse languages of the Highly Complex category: \ili{Alamblak} (Sepik), \ili{Albanian} (Indo-European), \ili{Itelmen} (Chukotko-Kamchatkan), \ili{Kabardian} (\ili{Abkhaz}-Adyge), Pas\-sa\-ma\-quod\-dy-Ma\-li\-seet\il{Passamaquoddy-Maliseet} (Algic), and \ili{Thompson} (Salishan). When the trend in the Highly Complex category is cross-tabulated against the trends of the other three categories in \tabref{tab:6.5}, it is found to be statistically signifiant ($p = 0.001$ in Fisher’s exact test).

  The implications of the results reported for affected vowels will be further discussed in \sectref{sec:6.4}. In the next section, the environments conditioning vowel reduction processes in the sample are analyzed.

\subsection{Conditioning environments}\label{sec:6.3.4}

  As described in \sectref{sec:6.2.3}, the conditioning environment of each vowel reduction process in the language sample was coded to reflect whether the consonantal environment, position in the word, position with respect to word stress, and/or position in the phrase/utterance contributed to the occurrence of the process. The results of this analysis can be found in \tabref{tab:6.6}.

\begin{table}
\begin{tabularx}{\textwidth}{Ilccccrr}
\lsptoprule
 & \multicolumn{6}{c}{Syllable structure complexity} & \textit{Total}\\\cmidrule(lr){2-7}
 Conditioning environments & & S & MC & C & HC\\
 \textit{N} processes & & 24 &  49 &  48  &  57  & \\\midrule
 \multicolumn{8}{c}{Single environment} \\\midrule
 {Consonantal} & & 2 & 8 & 6 & 11 & & \textit{27}\\
 {Stress} & & 3 & 4 & 11 & 13 & & \textit{31}\\
 {Word position} & & 3 & 6 & 2 & 1 & & \textit{12}\\
 {Phrase/utterance} & & 1 & 1 & 2 & 2 & & \textit{6}\\
 {Unclear} & & 2 & 2 & 2 & 4 & & \textit{10}\\\midrule
 \multicolumn{8}{c}{Combination of environments} \\\midrule
 {Stress and consonantal} & & 1 & 3 & 8 & 10 & & \textit{22}\\
 {Stress and word position}&  & 3 & 12 & 6 & 8 & & \textit{29}\\
 {Stress and phrase/utterance position} & & 1 & 3 & 3 & 3 & & \textit{10}\\
 {Consonantal and word position and word stress}&  & 2 & 5 & 7 & 3 & &  \textit{17}\\
 {Other combinations} & & 6 & 5 & 1 & 2 & &  \textit{14}\\
\lspbottomrule
\end{tabularx}
\caption{\label{tab:6.6}Conditioning environments of vowel reduction processes in sample.}
\end{table}

  76 of the vowel reduction processes in the language sample are reported to be conditioned by a single aspect of the environment as defined above. For ten processes, not enough information about the process was given to categorize the conditioning environment. The remaining 90 processes are conditioned by a combination of environments, typically \textit{stress} in addition to something else. The most common conditioning environments are \textit{stress} alone (31 processes), the \textit{stress} environment in combination with \textit{word position} (29 processes), and the \textit{consonantal} environment alone (27 processes).

  The main patterns in \tabref{tab:6.6} are summarized in \figref{fig:6.2}, which depicts the overall effect of each of the four environments in conditioning vowel reduction processes in languages with different syllable structure complexity. Here the environments are counted regardless of whether they occur alone or in combination with others in conditioning a process.

  
\begin{figure}
\begin{tikzpicture}
\pgfplotstableread{data/fig62.csv}{\table}
    \pgfplotstablegetcolsof{\table}
    \pgfmathtruncatemacro\numberofcols{\pgfplotsretval-1}
            \begin{axis}[easterdayline,ymax=100]
            \foreach \i in {1,...,\numberofcols} {
                \addplot+ table [x index={1},y index={\i},x expr=\coordindex] {\table};
                \pgfplotstablegetcolumnnamebyindex{\i}\of{\table}\to{\colname} % Adding column headers to legend
                \addlegendentryexpanded{\colname}
            }
            \end{axis}                                                                           
\end{tikzpicture}
\caption{\label{fig:6.2}Relative frequency of environments conditioning vowel reduction processes (expressed as percentage of total processes in each category of syllable structure complexity).}
\end{figure}

\begin{figure}
\begin{tikzpicture}
\pgfplotstableread{data/fig63.csv}{\table}
    \pgfplotstablegetcolsof{\table}
    \pgfmathtruncatemacro\numberofcols{\pgfplotsretval-1}
            \begin{axis}[easterdayline,legend style={font=\footnotesize,anchor=north west,text width=3.75cm,minimum height=.9\baselineskip},ymax=100]
            \foreach \i in {1,...,\numberofcols} {
                \addplot+ table [x index={1},y index={\i},x expr=\coordindex] {\table};
                \pgfplotstablegetcolumnnamebyindex{\i}\of{\table}\to{\colname} % Adding column headers to legend
                \addlegendentryexpanded{\colname}
            }
            \addlegendimage{empty legend}\addlegendentry{}
            \end{axis}                                                                           
\end{tikzpicture}
\caption{\label{fig:6.3}Relative frequency of stress-conditioned vowel reduction processes, expressed as the percentage of all vowel reduction processes from languages which have stress in each category.}
\end{figure}


  The trends in \figref{fig:6.2} show the \textit{stress} environment conditioning higher percentages of vowel reduction processes as syllable structure complexity increases. The \textit{word position} environment shows the opposite trend. The other two environments, \textit{consonantal} and \textit{phrase/utterance position}, have very subtle trends. Chi-square tests show that the \textit{stress} trend is significant ($\chi^2(3, N = 178) = 8.721$, $p = 0.03$) and the \textit{word position} trend is highly significant ($\chi^2(3, N = 178) = 15.986$, $p = 0.001$).\largerpage[-1]

  Recall from the analyses in \sectref{sec:5.4.1} that roughly one-fifth of the languages in the sample are not reported to have word stress. In \figref{fig:6.3} I only include vowel reduction processes from the 79 languages reported to have word stress. The trend shows the percentage of those processes which are conditioned by the stress environment in each category of syllable structure complexity.

  

  The trend in stress-conditioned vowel reduction in \figref{fig:6.3} is similar to the one in \figref{fig:6.2} in that it shows the percentage of such processes rising with syllable structure complexity, in particular setting apart the Simple and Moderately Complex categories from the Complex and Highly Complex categories. In \sectref{sec:5.4.3} it was found that vowel reduction as a segmental effect of stress occurred in a much smaller percentage of languages from the Simple category than from the other three categories: only 7/18 (39\%) of the languages with word stress in the Simple category had unstressed vowel reduction. Here we find a similar pattern: within languages that have word stress, stress conditions a smaller percentage of vowel reduction processes in the Simple and Moderately Complex categories than in the others. This suggests that not only does stress condition vowel reduction in more languages with complex syllable structure, but it also conditions more processes overall in those languages. This is confirmed when examining the ratio of the number of stress-conditioned vowel reduction processes to the number of languages with stress-conditioned vowel reduction in each category (\tabref{tab:6.7}).

\begin{table}
\begin{tabularx}{\textwidth}{Qlccccr}
\lsptoprule
 & \multicolumn{6}{c}{Syllable structure complexity}\\\cmidrule(lr){2-7}
 &  & S & MC & C & HC\\\midrule
 {\textit{N}} V reduction processes conditioned by stress & & 10 & 28 & 35 & 40& \\
 {\textit{N}} lgs. with V reduction conditioned by stress & & 7 & 12 & 15 & 15 & \\
 \textit{ratio} & & \textit{1.4} & \textit{2.3} & \textit{2.3} & \textit{2.7} & \\
\lspbottomrule
\end{tabularx}
\caption{\label{tab:6.7}Ratio of number of stress-conditioned vowel reduction processes to the number of languages with unstressed vowel reduction in each category of syllable structure complexity.}
\end{table}

  The pattern in \tabref{tab:6.7} indicates that in languages with word stress-condi\-tioned vowel reduction, the average number of processes conditioned by word stress, either solely or in addition to other phonological factors, increases with syllable structure complexity.

  The implications of the results reported for conditioning environments will be further discussed in \sectref{sec:6.4}. In the next section, the outcomes of vowel reduction processes in the sample are analyzed.

\subsection{Outcomes}\label{sec:6.3.5}

  Here we examine the reported outcomes of vowel reduction processes in the data. These were reduction in vowel duration, reduction in vowel quality, devoicing, deletion, and a few other effects. The other effects included cases where the vowel forms a syllabic consonant with an adjacent consonant, and two much rarer outcomes, tone leveling and glottalization of the vowel. As with the conditioning environments, sometimes a process involved a combination of outcomes. As described in \sectref{sec:6.2.3}, processes reported to involve the optional reduction or deletion of a vowel or group of vowels in some specific environment were coded as having several outcomes (deletion and whatever other reduction was specified by the author). See \tabref{tab:6.8} for the distribution of processes according to outcome and syllable structure complexity in the data.
  
    The most common outcomes for vowel reduction in the data are \textit{reduction in quality} (60 processes) and \textit{deletion} (43 processes). Processes involving a combination of outcomes are rare, and most of these are of the kind in which a vowel is optionally either reduced or deleted. The major patterns in \tabref{tab:6.8} are shown in \figref{fig:6.4}. Here the outcomes are counted regardless of whether they occur alone or in combination with others.


\begin{table}
\begin{tabularx}{\textwidth}{Qlccccr@{ }r}
\lsptoprule
 & \multicolumn{6}{c}{Syllable structure complexity} & \textit{Total}\\\cmidrule(lr){2-7}
 Outcome of vowel reduction processes & & S & MC & C & HC& \\
 \textit{N} processes & & 24  &  49  &  48  &  57  &  &\\\midrule
 \multicolumn{8}{c}{Single outcome from process}\\\midrule
 {Reduction in duration} & & 2 & 5 & 8 & 9 & & \textit{24}\\
 {Reduction in quality}&  & 4 & 16 & 20 & 20 & & \textit{60}\\
 {Devoicing} & & 7 & 6 & 5 & 6 & & \textit{24}\\
 {Syllabic consonant} & & 1 & 4 & 2 & 3 & & \textit{10}\\
 {Tone leveling or loss} & & -- & -- & 1 & -- & & \textit{1}\\
 {Glottalization of vowel} & & 1 & 1 & -- & -- & & \textit{2}\\
 {Deletion} & & 7 & 15 & 8 & 13 & & \textit{43}\\
 {Unspecified reduction}&  & 1 & -- & -- & 1 & & \textit{2}\\\midrule
 \multicolumn{8}{c}{Several outcomes from process}\\\midrule
 {Reduction or deletion} & & 1 & 1 & 3 & 5 & & \textit{10}\\
 {Other combinations} & & -- & 1 & 1 & -- & & \textit{2}\\
\lspbottomrule
\end{tabularx}
\caption{\label{tab:6.8}Outcomes of vowel reduction processes in sample.}
\end{table}


  
\begin{figure}
\begin{tikzpicture}
\pgfplotstableread{data/fig64.csv}{\table}
    \pgfplotstablegetcolsof{\table}
    \pgfmathtruncatemacro\numberofcols{\pgfplotsretval-1}
            \begin{axis}[easterdayline,ymax=60]
            \foreach \i in {1,...,\numberofcols} {
                \addplot+ table [x index={1},y index={\i},x expr=\coordindex] {\table};
                \pgfplotstablegetcolumnnamebyindex{\i}\of{\table}\to{\colname} % Adding column headers to legend
                \addlegendentryexpanded{\colname}
            }
            \end{axis}                                                                           
\end{tikzpicture}
\caption{\label{fig:6.4}Relative frequency of different outcomes of vowel reduction processes (expressed as percentage of total processes in each category of syllable structure complexity). Here outcomes are counted regardless of whether they occur alone or in combination with other outcomes.}
\end{figure}

  For outcomes of vowel reduction processes, it is the languages with Simple syllable structure which differ markedly in their behavior from the other languages in the sample. Languages in this category are significantly less likely to have processes resulting in \textit{reduction in quality} ($p = 0.04$ in Fisher’s exact test) and significantly more likely to have processes resulting in \textit{devoicing} than languages with more complex syllable structure (also $p = 0.04$ in Fisher’s exact test).

  Another interesting pattern in the data is the predominance of vowel deletion, the frequency of which rises, although not monotonically, with syllable structure complexity. This parallels the trend in vowel deletion established in \sectref{sec:5.4.3.1} for the more limited cases of unstressed vowel reduction. Here deletion is the second most common outcome of vowel reduction in the non-Simple categories. 

  As discussed in \sectref{sec:6.1}, vowel deletion is one of the known diachronic sources of the tautosyllabic consonant clusters which are a defining feature of syllable complexity. The hypothesis tested here predicts a greater prevalence of not only vowel reduction in languages with more complex syllable structure, but also extreme outcomes of vowel reduction, including vowel deletion. The results presented in \tabref{tab:6.8} and \figref{fig:6.4} support this hypothesis, but they do not consider the specific effects of vowel deletion, which can vary dramatically in the structures they produce \xxref{ex:6.14}{ex:6.15}:

\ea\label{ex:6.14}
\etriple{Fur}{Fur}{Sudan}

  In 3-syllable words with the structure (C\textsubscript{1})V\textsubscript{1}C\textsubscript{2}V\textsubscript{2}C\textsubscript{3}V\textsubscript{3}, where C\textsubscript{2} is /l/ or /ɾ/, C\textsubscript{3} is /l/, /ɾ/, or a nasal /m n ɲ ŋ/, and V\textsubscript{1} and V\textsubscript{2} are identical, V\textsubscript{2} may optionally be deleted.

\ea  /tiɾima/

[tiɾima] {\textasciitilde} [tiɾma]\\
\glt ‘sprouted grain’

\ex  /kuɾso/\\
\glt ‘heap of millet ears’

\ex  /jawil/\\
\glt ‘sky’

(\citealt{Jakobi1990}: 27, 29, 60--61; tone left unmarked)
\z
\z

\ea\label{ex:6.15}
\etriple{Albanian}{Indo-European}{Albania, Serbia, Montenegro}

  In rapid speech, mid central vowel /ə/ is optionally deleted when occurring between two consonants, of which C\textsubscript{1} is not /s z ʒ/.

\ea  /dəliɾə/

[dliɾə]\\
\glt ‘pure’

\ex  /məsuesja/

[msuesja]\\
\glt ‘the teacher’

(\citealt{Klippenstein2010}: 21--22, 27)
\z
\z

  In \ili{Fur} (\ref{ex:6.14}a), the optional vowel deletion results in simple codas of the form /l/ or /ɾ/, both of which are invariant structures attested in the canonical syllable patterns of the language (\ref{ex:6.14}b--c). That is, no tautosyllabic clusters or non-canonical patterns are formed as a result of this process. By contrast, in the \ili{Albanian} process, the optional deletion of /ə/ may result in tautosyllabic clusters which are canonical, e.g. /bɾ/ or /ps/ onsets, or non-canonical, e.g. onsets like /dl/ or /ms/ (\ref{ex:6.15}a--b).

  In \tabref{tab:6.9}, the 43 languages in the sample reported to have processes of vowel reduction resulting in changes to syllable patterns are distributed according to the specific structural outcome(s) of these processes. Most of these are processes of vowel deletion, but I also include processes which result in syllabic consonants.

\begin{sidewaystable}\scriptsize
\begin{tabularx}{\textwidth}{IQQQQ}
\lsptoprule
 & \multicolumn{4}{c}{Syllable structure complexity}\\\cmidrule(lr){2-5}
Structure resulting& S & MC & C & HC\\
from vowel deletion & 6 lgs. & 14 lgs. & 10 lgs. & 13 lgs.\\\midrule
{Canonical simple onset} &  & {Alyawarra}, {Carib}, {Cocama-Cocamilla}, {Kamasau} &  & \\
{Canonical simple coda} &  & {Alyawarra}, {Atong}, {Fur}, {Kambaata}, {Paiwan}, {Telugu} & {Bardi}, {Chipaya}, {Ket}, {Lelepa}, {Pech}, {Mamaindê} & {Albanian}, {Kabardian}, {Nuu-chah-nulth}, {Tehuelche}, {Thompson}\\
{Canonical tautosyllabic cluster} &  & {Darai}, {Kamasau} & {Ket}, {Lelepa}, {Pech} & {Albanian}, {Itelmen}, {Lezgian}, {Nuu-chah-nulth}, {Passamaquoddy-Maliseet}, {Qawasqar}, {Tehuelche}, {Thompson}, {Tohono O’odham}\\
{Non-canonical simple coda} & {Saaroa}, {Sumi Naga}, {Tukang Besi} & {Atong}, {Cocama-Cocamilla} & {Lakota}, {Lunda} & {Camsá}\\\tablevspace
{Non-canonical tautosyllabic cluster} & {Southern Grebo}, {Sumi Naga} & {Choctaw}, {Karok}, {Eastern Khanty} &  & {Albanian}, {Nuu-chah-nulth}, {Qawasqar}\\
{Syllabic consonant} & {Sichuan Yi} & {Alyawarra}, {Kamasau}, {Paiwan} & {Mamaindê}, {Oksapmin} & {Doyayo}, {Kabardian}, {Polish}\\
{Syllable deleted} & {Cubeo} &  & {Koho} & {Tehuelche}\\\tablevspace
{Unclear} &  & {Selepet} &  & \\
\lspbottomrule
\end{tabularx}
\caption{\label{tab:6.9}Languages in sample with vowel deletion, distributed according to syllable structure complexity and structural outcome of vowel deletion processes. For some languages, vowel deletion results in several different structural outcomes.}
\end{sidewaystable}

  In most of the languages (26/43) in \tabref{tab:6.9}, vowel deletion processes result in a structure which is attested in the canonical syllable pattern of the language, whether it be a simple onset, simple coda, or tautosyllabic cluster. In 15 languages, vowel deletion results in non-canonical syllable patterns, including otherwise unattested simple codas and tautosyllabic clusters. In nine languages, vowel reduction results in a syllabic consonant. Less commonly, vowel reduction is part of a wider-reaching process which deletes an entire syllable (three languages), or the structural effect of vowel deletion is unclear from the description (one language).

  Arguably, the most extreme effect of vowel deletion is the creation of non-canonical tautosyllabic clusters. Since there were so few instances of this in the data, it is difficult to draw strong conclusions from the distribution of these patterns with respect to syllable structure complexity. The number of languages with vowel deletion producing non-canonical syllable patterns in general (either codas or tautosyllabic clusters) does not increase with syllable structure complexity; if anything, such processes are more strongly associated with languages in the Simple and Moderately Complex categories. However, a notable pattern in \tabref{tab:6.9} is the relatively higher number of languages in the Highly Complex category for which vowel deletion results in tautosyllabic clusters, either canonical or non-canonical. In the Highly Complex category, 9/13 languages have this outcome from vowel deletion, as compared to 11/30 of the languages from the other three categories combined. It is striking that tautosyllabic clusters are an outcome of vowel deletion more often in languages which already have large tautosyllabic clusters. In this respect, there is additional support here for the hypothesis that final outcomes of vowel reduction are more extreme in languages with more complex syllable structure.   

  Additionally, as previously discussed in \sectref{sec:3.3.5}, there is a trend by which vowel reduction processes resulting in syllabic consonants are more characteristic of languages with non-Simple syllable structure. This trend is weak at best, being based on the patterns of just nine languages. However, taken at face value it also lends some support to the hypothesis tested here: vowel reduction resulting in syllabic consonants may alter the syllable patterns of languages in more extreme ways than, say, vowel deletion resulting in canonical simple onsets or codas.

\subsection{Holistic analysis of vowel reduction processes}\label{sec:6.3.6}

  The quantitative analyses presented in §§\ref{sec:6.3.3}--\ref{sec:6.3.5} do not necessarily inform a holistic understanding of the vowel reduction processes in the sample, since they treat the affected vowels, conditioning environments, and outcomes separately. To complement these previous analyses, in \xxref{ex:6.16}{ex:6.19} I summarize the most characteristic kinds of vowel reduction processes which occur in each syllable structure complexity group. This breakdown allows us to examine how the different affected vowels, conditioning environments, and outcomes tend to cluster together into coherent patterns in the sample. The number of processes which fit the general description is given in parentheses. For each syllable structure category I give two prototypical examples of vowel reduction.

\ea\label{ex:6.16}
  Summary of vowel reduction processes in Simple category (\textit{N} = 24 lgs.)
\begin{itemize}
\item Vowels devoiced at word or phrase/utterance margins {(6 lgs.)}
\item High vowels deleted in word- or phrase/utterance-final position {(4 lgs.)}
\item Long vowels shortened/glottalized in various environments {(3 lgs.)}
\item Vowels devoiced in specific consonantal environments {(2 lgs.)}
\item Free variation resulting in quality reduction {(2 lgs.)}
\item Other {(7 lgs.)}
\end{itemize}

\ea\etriple{Apurinã}{Arawakan}{Brazil}

Vowels become devoiced in unstressed word-final position, especially in fast speech.

/tõːˈɡat͡ʃi/

[tõːˈɡat͡ʃi̥]\\
\glt ‘cough’
\citep[60--61]{Facundes2000}

\ex\etriple{Sumi Naga}{Sino-Tibetan}{India}

Word-final high vowels are prone to deletion following a nasal.

/pamú/

[pam{\DejaVuSerif˧}]\\
\glt ‘his older brother’
\citep[369]{Teo2012}
\z
\z

\ea\label{ex:6.17}
  Summary of vowel reduction processes in Moderately Complex category (\textit{N} = 49 lgs.)

\begin{itemize}
\item Vowels, often high or short, deleted in unstressed syllables (13 lgs.)
\item Vowels deleted or devoiced in specific consonantal environments (7 lgs.)
\item High vowels reduced in quality when unstressed and/or at word or phrase/utterance margins (6 lgs.)
\item Long vowels shortened/glottalized in various environments (5 lgs.)
\item Low and mid vowels are reduced in quality when unstressed and/or at word margins (5 lgs.)
\item Other {(13 lgs.)}
\end{itemize}

\ea   \etriple{Karok}{isolate}{USA}

An unaccented word-initial short vowel preceding two consonants may be lost following a pause.

/akvaːt/

[kvaːt]\\
\glt ‘raccoon’
\citep[53]{Bright1957}

\ex  \etriple{Tu}{Mongolic}{China}

High vowels /i u/ are realized as lax in unstressed syllables. 

/t͡ɕawtunˈtu/

[t͡ɕawtʊnˈtu]\\
\glt ‘dream (\textsc{dat})’
\citep[35]{Slater2003}
\z
\z\pagebreak

\ea\label{ex:6.18}
  Summary of vowel reduction processes in Complex category\\(\textit{N} = 48 lgs.)

\begin{itemize}
\item Vowels reduced in quality in unstressed syllables (19 lgs.)
\item Unstressed vowels deleted, often in specific consonantal environments (8 lgs.)
\item Vowels devoiced in environment of voiceless consonants and/or unstressed domain-final environments  {(5 lgs.)}
\item Long vowels shortened in various environments  {(4 lgs.)}
\item All vowels shortened in specific unstressed contexts (3 lgs.)
\item Other (9 lgs.)
\end{itemize}

\ea\etriple{Ngarinyin}{Worrorran}{Australia}

Low central vowel /a/ is realized as [ə] when unstressed.

/ˈbaraˌbara/

[ˈbarəˌbarə]\\
\glt ‘story’
\citep[17--18]{Rumsey1978}

\ex  \etriple{Pech}{Chibchan}{Honduras}
Unstressed vowels are usually deleted between any consonant and a following /ɾ/.\\
/ˈkúhpaɾ\`{ã}/\\\relax
[ˈkúhpɾ\`{ã}]\\
\glt ‘you and I having bought it’
\citep[23]{Holt1999}
\z
\z

\ea\label{ex:6.19}
  Summary of vowel reduction processes in Highly Complex category\\(\textit{N} = 57 lgs.)
  
\begin{itemize}
\item Vowels reduced in quality in unstressed syllables (16 lgs.)
\item Unstressed /ə/ deleted, often in specific consonantal environments {(8 lgs.)}
\item Other unstressed vowels deleted, often in specific consonantal environments (7 lgs.)
\item Unstressed vowels devoiced in specific consonantal and word or phrase/utterance environments (7 lgs.)
\item Long vowels shortened in various environments (5 lgs.)
\item All vowels shortened in specific consonantal environments  {(3 lgs.)}
\item Other (11 lgs.)
\end{itemize}\pagebreak

\ea  \etriple{Thompson}{Salishan}{Canada}

High vowels /i u/ are nearly always realized as [ə] when preceding the main stress, except for when /u/ occurs between two velar consonants. 

/sq’ʷuˈteɬxʷ/

[sq’ʷəˈteɬxʷ]\\
\glt ‘(other) side of the house’

(\citealt{ThompsonThompson1992}: 32)

\ex  \etriple{Kabardian}{Abkhaz-Adyge}{Russia, Turkey)}

Unstressed /ə/ preceding a stressed syllable is often deleted, so long as it does not produce an initial consonant cluster.

/bəsəˈməf’/

[bəsˈməf’]\\
\glt ‘good host’

(\citealt{GordonApplebaum2010}: 42)
\z
\z

  The characteristic patterns of vowel reduction vary widely in the different syllable structure complexity categories. There are two general patterns which occur in all groups of languages: shortening of long vowels in various environments, and vowel devoicing in specific consonantal or domain environments. 

  Some processes are almost entirely unique to languages in a particular category of syllable structure complexity: as noted previously, except for \ili{Paiwan} (Moderately Complex), unstressed /ə/ deletion occurs only in languages with Highly Complex syllable structure. Other general processes may occur with different specifications in languages with different syllable structure complexity. For example, unstressed vowel deletion primarily affects high and short vowels in languages with Moderately Complex syllable structure. In languages with Complex or Highly Complex syllable structure, unstressed vowel deletion tends to affect all unstressed vowels, but is also typically conditioned by the consonantal environment.

  This treatment of vowel reduction processes in the data upholds several of the trends uncovered by the various quantitative analyses. For instance, the domain (word or phrase/utterance) position is a defining property of several of the frequent process types identified for the Simple and Moderately Complex categories. Similarly, the stress environment is prominent in conditioning many of the vowel reduction types in languages with Moderately Complex, Complex, and Highly Complex syllable structure, and a very common outcome of such processes is reduction in vowel quality.

\subsection{Summary of vowel reduction patterns}\label{sec:6.3.7}

  In \sectref{sec:6.1} I formulated a hypothesis based on observations of recent and ongoing processes of vowel reduction causing changes to syllable structure patterns in some languages. The hypothesis predicted that languages with more complex syllable structure would show stronger effects of vowel reduction: specifically, vowel reduction processes were expected to both be more prevalent and have more extreme outcomes as syllable structure complexity increases. These predictions were largely upheld by the analyses in this chapter. The analyses in §§\ref{sec:6.3.1}--\ref{sec:6.3.2} showed that both the percentage of languages with vowel reduction processes and the number of vowel reduction processes per language increases with syllable structure complexity. The analysis of outcomes of vowel reduction processes in \sectref{sec:6.3.5} showed an overall, if not monotonic, increase in vowel deletion rates with syllable structure complexity. Vowel deletion processes resulting in tautosyllabic clusters were found to occur in a higher percentage of languages from the Highly Complex category than from the other categories, and this was by far the most common outcome of vowel deletion in this group of languages.

  The specific properties of vowel reduction found to have positive or negative trends with respect to syllable structure complexity are listed in \tabref{tab:6.10}. Those marked with an asterisk (*) were found to be statistically significant.

\begin{table}
\begin{tabularx}{\textwidth}{QQQ}
\lsptoprule

{Type of property} & Positive trends (increases with syllable structure complexity) & Negative trends (decreases with syllable structure complexity)\\\midrule
{Vowel reduction} & Presence of processes

*Number of distinct processes & \\\tablevspace
{Affected vowels} & Short vowels

*/ə/ & High vowels\\\tablevspace
{Conditioning environments} & *Stress

Number of distinct processes conditioned by stress & *Word position\\\tablevspace
{Outcomes} & *Reduction in quality

Deletion

Deletion resulting in tautosyllabic clusters & *Devoicing\\
\lspbottomrule
\end{tabularx}
\caption{\label{tab:6.10}Properties of vowel reduction associated positively or negatively with syllable structure complexity.}
\end{table}

  As with findings in previous chapters, and as mentioned above already, some of the patterns shown in \tabref{tab:6.10} do not show a gradual trend with respect to syllable structure complexity, instead serving to set apart the Simple and Highly Complex categories from the others.

  In the next section I discuss the implications of these results for our understanding of highly complex syllable structure as a language type and for the development of syllable structure complexity more generally.

\section{Discussion}\label{sec:6.4}
\subsection{Vowel reduction patterns and Highly Complex syllable structure}\label{sec:6.4.1}

  The study of vowel reduction presented here adds several new findings relevant to the first research question of this book, which seeks to establish whether languages with highly complex syllable structure share other phonetic and phonological characteristics in common. The properties of phonetically and phonologically conditioned vowel reduction which are more strongly associated with the Highly Complex category than the others are listed in \REF{ex:6.20}.

\ea\label{ex:6.20}
  Properties of vowel reduction associated with Highly Complex category

\textit{Presence of vowel reduction processes}

\textit{Presence of two or more vowel reduction processes}

\textit{Presence of higher numbers of stress-conditioned vowel reduction processes}

\textit{Absence of vowel reduction processes affecting high vowels}

\textit{Presence of vowel reduction processes affecting /ə/}

\textit{Absence of processes conditioned by word position}

\textit{Presence of vowel deletion}

\textit{Presence of vowel deletion resulting in tautosyllabic clusters}
\z

  As mentioned in previous chapters, the terms ``absence'' and ``presence'' are used here not in a categorical sense. Instead these are meant to correspond to the relative absence or presence of a property in the Highly Complex group as compared to the other syllable structure complexity groups.

  In previous chapters I showed how the segmental and suprasegmental patterns associated with the Highly Complex group were distributed among the languages in that group. The resulting distribution showed that languages in which Highly Complex syllable patterns are more prominent also had more of those associated patterns. In \tabref{tab:6.11} I show a similar breakdown for how the vowel reduction patterns most strongly associated with the Highly Complex portion of the sample are distributed among those languages. The languages are again divided into three groups according to the prominence of their Highly Complex syllable patterns, as established in §§\ref{sec:3.4.1}--\ref{sec:3.4.2}. The vowel reduction properties associated with Highly Complex syllable structure and listed in \REF{ex:6.20} above are given in the columns. A check mark indicates that a language has the expected property; a shaded cell indicates that it does not. 

\begin{table}\small
\begin{tabularx}{\textwidth}{l@{ }CCCCCCC@{ }C}
\lsptoprule
 & \multicolumn{5}{c}{V reduction} & & \multicolumn{2}{c}{Vowel deletion}\\\cmidrule(lr){2-6}\cmidrule(lr){8-9}
 & \rotatebox{90}{\parbox{3cm}{\raggedright \textbf{present}}} & \rotatebox{90}{\parbox{3cm}{\raggedright ${\geq}$2 processes \textbf{present}}}  & \rotatebox{90}{\parbox{3cm}{\raggedright Stress-conditioned \textbf{present}}}  & \rotatebox{90}{\parbox{3cm}{\raggedright affecting only high vowels \textbf{absent}}}  & \rotatebox{90}{\parbox{3cm}{\raggedright affecting  /ə/ \textbf{present}}}  & \rotatebox{90}{\parbox{3cm}{\raggedright Processes conditioned by word position \textbf{absent}}}  & \rotatebox{90}{\parbox{3cm}{\raggedright \textbf{present}}}  & \rotatebox{90}{\parbox{3cm}{\raggedright resulting in tautosyllabic clusters \textbf{present}}}\\\midrule
& \multicolumn{8}{c}{Languages with prevalent Highly Complex patterns}\\\midrule
 \ili{Cocopa} & \ding{51} & \ding{51} & \ding{51} & \ding{51} & \cellcolor{lsLightGray} & \ding{51} & \cellcolor{lsLightGray} & \cellcolor{lsLightGray} \\
 \ili{Georgian} & \cellcolor{lsLightGray} & \cellcolor{lsLightGray} & \cellcolor{lsLightGray} & \ding{51} & \cellcolor{lsLightGray} & \ding{51} & \cellcolor{lsLightGray} & \cellcolor{lsLightGray} \\
 \ili{Itelmen} & \ding{51} & \ding{51} & \cellcolor{lsLightGray} & \ding{51} & \ding{51} & \ding{51} & \ding{51} & \ding{51}\\
 \ili{Polish} & \ding{51} & { \ding{51}} & \ding{51} & \ding{51} & \cellcolor{lsLightGray} & \ding{51} & \cellcolor{lsLightGray} & \cellcolor{lsLightGray} \\
 \ili{Tashlhiyt} & \ding{51} & \cellcolor{lsLightGray} & \cellcolor{lsLightGray} & \ding{51} & \cellcolor{lsLightGray} & \ding{51} & \ding{51} & \cellcolor{lsLightGray} \\
 \ili{Thompson} & \ding{51} & \ding{51} & \ding{51} & \cellcolor{lsLightGray} & \ding{51} & \ding{51} & \ding{51} & \ding{51}\\
 T. O’odham & \ding{51} & \ding{51} & \ding{51} & \cellcolor{lsLightGray} & \cellcolor{lsLightGray} & \cellcolor{lsLightGray} & \ding{51} & \ding{51}\\
 Y. Sahaptin & \ding{51} & { \ding{51}} & \ding{51} & \ding{51} & \cellcolor{lsLightGray} & \ding{51} & \cellcolor{lsLightGray} & \cellcolor{lsLightGray} \\\midrule
& \multicolumn{8}{c}{Languages with intermediate Highly Complex patterns}\\\midrule
 \ili{Albanian} & \ding{51} & \ding{51} & \ding{51} & \ding{51} & \ding{51} & \cellcolor{lsLightGray} & \ding{51} & \ding{51}\\
 \ili{Camsá} & \ding{51} & \ding{51} & \ding{51} & \ding{51} & \cellcolor{lsLightGray} & \cellcolor{lsLightGray} & \ding{51} & \cellcolor{lsLightGray} \\
 \ili{Kabardian} & \ding{51} & \ding{51} & \ding{51} & \ding{51} & \ding{51} & \cellcolor{lsLightGray} & \ding{51} & \cellcolor{lsLightGray} \\
 \ili{Lezgian} & \ding{51} & \ding{51} & \ding{51} & \cellcolor{lsLightGray} & \cellcolor{lsLightGray} & \ding{51} & \ding{51} & \ding{51}\\
 \ili{Mohawk} & \ding{51} & \cellcolor{lsLightGray} & \cellcolor{lsLightGray} & \ding{51} & \cellcolor{lsLightGray} & \ding{51} & \cellcolor{lsLightGray} & \cellcolor{lsLightGray} \\
 Nuuchahnulth & \ding{51} & \ding{51} & \ding{51} & \ding{51} & \cellcolor{lsLightGray} & \cellcolor{lsLightGray} & \ding{51} & \ding{51}\\
 P.-Maliseet & \ding{51} & \ding{51} & \ding{51} & \ding{51} & \ding{51} & \ding{51} & \ding{51} & \ding{51}\\
 \ili{Yine} & \ding{51} & \ding{51} & \ding{51} & \ding{51} & \cellcolor{lsLightGray} & \cellcolor{lsLightGray} & \cellcolor{lsLightGray} & \cellcolor{lsLightGray} \\
 \ili{Qawasqar} & \ding{51} & \ding{51} & \cellcolor{lsLightGray} & \ding{51} & \cellcolor{lsLightGray} & \cellcolor{lsLightGray} & \cellcolor{lsLightGray} & \ding{51}\\
 \ili{Semai} & \ding{51} & \cellcolor{lsLightGray} & \cellcolor{lsLightGray} & \ding{51} & \cellcolor{lsLightGray} & \ding{51} & \cellcolor{lsLightGray} & \cellcolor{lsLightGray} \\
 \ili{Tehuelche} & \ding{51} & \ding{51} & \ding{51} & \ding{51} & \cellcolor{lsLightGray} & \ding{51} & \ding{51} & \ding{51}\\\midrule
& \multicolumn{8}{c}{Languages with minor Highly Complex patterns}\\\midrule
 \ili{Alamblak} & \ding{51} & \ding{51} & \ding{51} & \cellcolor{lsLightGray} & \ding{51} & \ding{51} & \cellcolor{lsLightGray} & \cellcolor{lsLightGray} \\
 \ili{Bench} & \cellcolor{lsLightGray} & \cellcolor{lsLightGray} & \cellcolor{lsLightGray} & \ding{51} & \cellcolor{lsLightGray} & \ding{51} & \cellcolor{lsLightGray} & \cellcolor{lsLightGray} \\
 \ili{Doyayo} & \ding{51} & \ding{51} & \cellcolor{lsLightGray} & \ding{51} & \cellcolor{lsLightGray} & \ding{51} & \cellcolor{lsLightGray} & \cellcolor{lsLightGray} \\
 \ili{Kunjen} & { \ding{51}} & \ding{51} & \ding{51} & \ding{51} & \cellcolor{lsLightGray} & \cellcolor{lsLightGray} & \cellcolor{lsLightGray} & \cellcolor{lsLightGray} \\
 \ili{Menya} & \cellcolor{lsLightGray} & \cellcolor{lsLightGray} & \cellcolor{lsLightGray} & \ding{51} & \cellcolor{lsLightGray} & \ding{51} & \cellcolor{lsLightGray} & \cellcolor{lsLightGray} \\
 \ili{Wutung} & \cellcolor{lsLightGray} & \cellcolor{lsLightGray} & \cellcolor{lsLightGray} & \ding{51} & \cellcolor{lsLightGray} & \ding{51} & \cellcolor{lsLightGray} & \cellcolor{lsLightGray} \\
\lspbottomrule
\end{tabularx}
\caption{\label{tab:6.11}Highly Complex languages, divided into three groups according to the prominence of their Highly Complex patterns. Expected properties are given in columns. A check mark indicates that the given language has the expected property; a shaded cell indicates it does not.}
\end{table}

As in the similar analyses in \sectref{sec:4.5} and \sectref{sec:5.5}, we find that languages which have Highly Complex syllable structure as a prevalent or ``intermediate'' pattern tend to have more of the vowel reduction properties associated with this category than languages which have it as a minor pattern. Like the similar patterns reported for segmental and suprasegmental features, these results lend support to the idea that highly complex syllable structure is a linguistic type which can be defined by a coherent set of prototypical features, in this case, certain types of dynamic vowel reduction patterns.

  As with the previous studies in this book, the results here also point to characteristics which set apart the languages of the Simple category, distinguishing them from languages in the other three categories. These include a strong presence of processes conditioned by word position and processes with an outcome of vowel devoicing, and a relative absence of processes affecting short vowels, processes conditioned by word stress, and processes with an outcome of quality reduction.

  There are several results from the current study of vowel reduction which may prove relevant in addressing the diachronic development of highly complex syllable structure. The hypotheses were motivated by the observation that vowel reduction, specifically in the form of vowel deletion, is a documented source of tautosyllabic consonant clusters. It should therefore be an important contributing factor to the development of the long consonant clusters characteristic of languages with highly complex syllable structure. If increasing syllable structure complexity represents a diachronic cline, then an increase in syllable structure complexity would at some point entail the gradual emergence of previously unattested (non-canonical) tautosyllabic clusters. In the data examined here, evidence of such a scenario is extremely rare as a result of synchronic vowel reduction processes which are salient enough to be reported as productive patterns by authors of language references. Vowel deletion produces new (non-canonical) tautosyllabic clusters in only eight languages of the sample: one with Simple syllable structure (\ili{Southern Grebo}), four with Moderately Complex syllable structure (\ili{Atong}, \ili{Choctaw}, \ili{Karok}, and \ili{Eastern Khanty}) and three with Highly Complex syllable structure (\ili{Albanian}, \ili{Nuu-chah-nulth}, and \ili{Qawasqar}). Of these, detailed distributional and phonetic data on the resulting clusters is available only for \ili{Albanian}. In that language, at least, vowel deletion resulting in tautosyllabic clusters is shown to be quite prevalent, producing dozens of distinct canonical and non-canonical onset sequences (\citealt{Klippenstein2010}; orthographic evidence indicates that non-canonical onset sequences are recent). Nevertheless, the current data set is too small from which to draw strong conclusions regarding general observable patterns of syllable structure emergence (but see further discussion of this point in \chapref{sec:8}). This is an area in which more comprehensive phonetic, distributional, and frequency data would prove extremely informative.

  What is found in the vowel reduction data is evidence of persistent articulatory routines. While vowel deletion is an important process in all syllable structure complexity groups, we find that it is more likely to produce tautosyllabic clusters, either canonical or non-canonical, in the Highly Complex group. That is, vowel deletion is more likely to create clusters in languages which already have a prevalence of consonant clusters. This is in line with the reported findings for \ili{Lezgian}, in which the process of pretonic high vowel deletion continues to persist in the language even after it has altered the syllable structure patterns of the language (\citealt{ChitoranBabaliyeva2007}). This observation is not surprising from a usage-based perspective, in which phonological structure is sensitive to cognitive factors such as frequency effects and analogy \citep{Bybee2001}; that is, a high frequency of complex syllable patterns in a language could facilitate the maintenance and phonologization of novel complex syllable patterns which come about through vowel deletion. It also suggests long-term stability in highly complex syllable structure, a view which is not necessarily afforded by abstract theoretical treatments seeking to account for its problematic nature.

  Several other findings in the data tentatively suggest paths of development for highly complex syllable structure. Vowel deletion targeting /ə/ is present almost exclusively in languages of the Highly Complex portion of the sample. This pattern may point to another persistent articulatory routine. Mid central vowel [ə], or something very much like it, is a common outcome of vowel reduction processes: while the precise outcome of quality reduction isn’t always specified, [ə] was reported to be the specific outcome in 19 of the 64 processes involving quality reduction analyzed here. Similarly, contrastive /ə/ is known to derive historically from reduced vowels in some cases. This seems to be transparently the case for several languages in the sample (e.g. \ili{Pacoh}, \citealt{Alves2000}). The presence of /ə/ deletion almost exclusively in languages of the Highly Complex category could point to a cline in which 
  (1) vowel reduction processes initially affect vowel quality, 
  (2) the reduced vowel quality becomes phonologized, and 
  (3) the reductive tendencies in the language continue to affect the sound that has already been reduced, eventually leading to its deletion.\footnote{{Indeed, this does seem to be the process occurring in American \ili{English} with respect to /ə/ deletion.}} Such a hypothetical path of reduction may then be responsible for the pattern by which high vowels are commonly affected by vowel reduction in the Simple, Moderately Complex, and Complex categories but not the Highly Complex category: processes may have by that stage already affected the quality of high vowels in environments of reduction. Although only speculation here, these possibilities will be explored in greater detail in \chapref{sec:8}.

\subsection{Implications for development of syllable structure complexity}\label{sec:6.4.2}

  We now turn to a discussion of how the results of the study might have implications for the development of syllable structure complexity more generally. A concrete finding from the current study which may bear on this issue pertains to common conditioning environments for vowel reduction. Word position is a highly relevant conditioning environment in languages with simpler syllable structures, while word stress is the strongest conditioning environment in languages with more complex syllable structure. However, an additional important observation is that a robust minority of vowel reduction processes in the sample are not conditioned by stress at all: these include 66 vowel reduction processes (37\% of the total) in 43 languages from all syllable structure complexity categories. Furthermore, the outcomes of such processes may have an effect on syllable structure: seven result in syllabic consonants and 15 result in vowel deletion, some of which create non-canonical syllable patterns (cf. the \ili{Sumi Naga} example in 16b). Interestingly, in one such process in \ili{Paiwan}, vowel deletion occurs in both stressed and unstressed syllables in fast speech. In \REF{ex:6.21}, this process is shown occurring in a stressed syllable and resulting in stress shift and the resyllabification of the adjacent nasal as a canonical coda. A further reduction reduces the stressed vowel and coda to a syllabic consonant.

\ea\label{ex:6.21}
  \etriple{Paiwan}{Austronesian}{Taiwan}

/t͡səˈmədas/

[t͡sʰəˈmədas] {\textasciitilde} [ˈt͡sʰəmdas] {\textasciitilde} [ˈt͡sʰm̩das]\\
\glt ‘a name for male’
\citep[42]{Chang2006}
\z

In \ili{Itelmen}, a similar process of reduction and deletion of /ə/ is reported: e.g. /kəmmanəkit/ > [kəmmanəkɪt] {\textasciitilde} [kɯmmanəkɪt] {\textasciitilde} [kmanəkɪt] ‘I-\textsc{caus}’ (\citealt{GeorgVolodin1999}: 13). Although stress is not marked in that example, \ili{Itelmen} is reported to have fixed initial stress \citep[6]{Bobaljik2006}. Thus this is potentially another case in which reduction and deletion processes target vowels regardless of their stress status, resulting in increased phonotactic complexity. 

  Including \ili{Paiwan} and \ili{Itelmen}, there are 11 languages in the sample from all syllable complexity categories which have both stress and vowel reduction processes, but no vowel reduction processes conditioned by word stress. As discussed in \sectref{sec:5.5.2}, an important consideration in interpreting findings in the speech rhythm literature is the fact that stress systems and syllable patterns may change independently of one another. The presence of the set of vowel reduction patterns discussed above suggests that non-stress-conditioned vowel reduction is an important, if secondary, source by which syllable patterns may be altered over time.

  Though a large majority (18/24) of the languages in the Simple category have word stress, this category is set apart from the rest in that stress is much less likely to condition vowel reduction in these languages. As mentioned in \chapref{sec:5}, this observation has also been made in the speech rhythm literature: syllable-timed languages do not necessarily lack stress, but stress does not have strong segmental effects in those languages \citep{Auer1993}. In the current study it was additionally found that the number of distinct processes conditioned by stress within languages that have stress increases with syllable structure complexity. This finding clarifies one of the more puzzling results from \chapref{sec:5}, where it was found that the percentage of languages with stress-conditioned vowel reduction did not substantially increase across the Moderately Complex, Complex, and Highly Complex categories. In light of the results in this chapter, we can say that while the number of languages with unstressed vowel reduction does not increase across these categories, the effects of stress within those languages does increase. In that sense, at least, there is now stronger evidence for the hypothesis in \chapref{sec:5} regarding the segmental effects of stress, which did not have strong support in the analyses in that chapter: stress conditions a higher number of distinct vowel reduction patterns as syllable structure increases.

  This raises the question of how word stress, specifically, and vowel reduction, more generally, differ in languages with Simple syllable structure. With respect to stress, recall from \chapref{sec:5} that besides segmental effects, there were few properties of stress which set the Simple category apart from the others. Languages in that category were more likely than the others to have pitch as a phonetic correlate of stress and less likely than the others to have intensity as a correlate. However, it is unclear how either of those properties would correspond to lower rates of unstressed vowel reduction in those languages. Turning to the more general question of how vowel reduction differs in languages with Simple syllable structure, one possibility is the presence and effects of tone. In \sectref{sec:5.3} it was found that over half of the languages in the Simple category had tonal contrasts, a proportion that was higher than in the other three categories. Tonal contrasts signal changes in lexical or grammatical meaning, and are typically carried by vowels. It is worth investigating whether the presence of tone makes vowel reduction less likely to occur: in such a scenario, the greater functional load carried by vowels in those languages might make them less susceptible to reduction processes.

  Though a higher proportion of languages with tone do not have vowel reduction processes (14/37, as compared to 15/63 non-tonal languages), this trend is not statistically significant in a chi-square test. A similar analysis of vowel deletion patterns with respect to the presence or absence of tone also yields a non-significant result. Thus it does not seem likely that the relatively higher frequency of tonal systems is a strong motivation for the much lower rates of vowel reduction observed in the Simple category.\footnote{{The analysis of tone presented in \sectref{sec:5.3} does not consider complexity of tonal systems in number of tone contrasts,  nor does it distinguish between ``prototypical'' and restricted tonal systems. It is possible that there could be a relationship between the presence of tone and absence of vowel reduction when such patterns are considered.}}

  The properties most strongly associated with vowel reduction in the languages of the Simple category are those having to do with outcomes: a higher rate of devoicing outcomes and a lower rate of reduced quality outcomes set this category apart from the others. The devoicing outcome bears a relationship to the conditioning environments associated with vowel reduction in the Simple category. Throughout the four syllable structure complexity categories, devoicing is most often conditioned by the word or phrase/utterance environment (22/31 devoicing processes). The higher rate of devoicing in the Simple category is clearly related to the higher rate of domain-conditioned processes in this category: 6/8 of such processes are conditioned by word or phrase/utterance environments, and five of these specify domain-final environments. However, the general motivations behind the higher rate of domain-conditioned processes in the Simple category are not entirely clear.\footnote{{Incidentally, the prominence of vowel devoicing processes in the Simple category is consistent with the findings in \sectref{sec:4.3.3} that phonation contrasts in vowels most often occur in these languages (in a sample of just seven languages).}} It could be that in languages in which all domains are necessarily vowel-final, as in the Simple category, domain-final devoicing will emerge as a more prominent pattern than it would in languages that frequently have consonants in domain-final position.

  Because most of the vowel reduction processes reported in the sources are not accompanied by instrumental data, it is difficult to comment on the relative extremity of different forms of reduction, such as devoicing or reduction in quality, in comparison to vowel deletion. In an Articulatory Phonology model, processes such as vowel reduction come about through an increase in overlap or decrease in magnitude of articulatory gestures \citep{BrowmanGoldstein1992b}. A similar proposal posits that phonetically-conditioned sound change originates in temporal or substantive reduction of gestures (\citealt{MowreyPagliuca1995}). Both models predict that the cline to vowel deletion would necessarily include reduced vowel length, as a result of overlap of surrounding consonantal gestures into the vowel articulation, a temporal reduction of the vocalic gesture itself, or both. The acoustic findings reported by \citet{ChitoranBabaliyeva2007} for \ili{Lezgian} support this: they show that the vowel reduction patterns in that language involve devoicing and decreased vowel duration before eventual deletion. They also show evidence for gestural overlap in that vocalic properties are retained as secondary palatalization or labialization on the preceding vowel.

  The findings of the current study indicate that decreased vowel duration occurs as an outcome of vowel reduction in a roughly equal proportion of processes in each syllable structure complexity category. However, the qualitative analysis of the process types in \sectref{sec:6.3.6} shows that there are important differences between the categories: while shortening of long vowels occurs in all four categories, shortening of all vowels in specific consonantal or unstressed contexts occurs as a strong pattern only in languages with Complex and Highly Complex syllable structure. Since such processes would have a more detrimental effect on short vowels than long vowels from an articulatory point of view, an argument could be made that these are further examples of relatively extreme vowel reduction in languages with more complex syllable structure.

  While there are still unanswered questions regarding particulars of the distribution of vowel reduction properties in the sample, the results of this chapter show that vowel reduction remains relevant, and indeed becomes even more prevalent, as syllable structure complexity increases. Furthermore, the results indicate that though rates of vowel reduction as an effect of stress increase with syllable structure complexity, other sources of vowel reduction are relevant in both the language sample as a whole and in the languages of the Highly Complex category. This suggests that vowel reductive tendencies in general increase with syllable structure complexity. This point will be revisited in \chapref{sec:8}, after presenting a brief study of consonant allophony in \chapref{sec:7}.

