\chapter{Suprasegmental patterns}\label{sec:5}
\section{{Introduction}}\label{sec:5.1}

  In this chapter I describe and present analyses of suprasegmental properties in the language sample. Specifically, the four hypotheses tested here relate the placement, segmental effects, and phonetic correlates of word stress to syllable structure complexity. The distribution of tone in the language sample is also briefly considered in relation to one of these hypotheses.

  The chapter is organized as follows. In \sectref{sec:5.1} I describe general properties of word stress and tone, discuss findings in the literature which relate these to syllable structure complexity, and introduce the hypotheses to be tested in the current study. In \sectref{sec:5.2} I describe the methodology behind the data collection and coding. In \sectref{sec:5.3} I present a brief analysis of the distribution of the presence of tone in the language sample. In \sectref{sec:5.4} I present several analyses to test the hypotheses relating properties of word stress to syllable structure complexity. In \sectref{sec:5.5} I discuss how the results address the main research questions of the book regarding highly complex syllable structure, and how they relate more generally to the evolution of that complexity.

\subsection{Word stress and tone}\label{sec:5.1.1}

  In \chapref{sec:4}, I presented a study of the segmental properties of the language sample. As discussed there, segments are the more or less discrete units which correspond to contrastive consonant and vowel sounds in a language. In this chapter, the focus is instead on suprasegmental properties of the language sample. The term ”suprasegmental” refers to phonological properties of speech which are associated with domains larger than the segment; that is, the syllable, word, or even larger units such as phonological phrases or utterances. In the current study, only two suprasegmental features are considered: word stress and tone. I describe some basic characteristics of these phenomena here.

  Not all languages have word stress. In languages in which it occurs, word stress corresponds to the increased perceptual prominence of a syllable with respect to other syllables in a word. This prominence is acoustically salient and may be accompanied by increased duration, differences in pitch (the perceptual analog of fundamental frequency), higher intensity, and differences in spectral tilt \citep{Gordon2011}. Articulatory properties associated with stress include increased duration of gestures, more extreme articulations (i.e. tighter constrictions for consonants and more open articulations for vowels) and less articulatory overlap between consonantal and vocalic gestures (\citealt{BeckmanEdwards1994,Fougeron1999,DeJongEtAl1993}). Many of the findings on acoustic and articulatory correlates of stress are based on studies of individual languages. Languages vary widely with respect to which phonetic properties cue stress. While \ili{English} uses a combination of duration, intensity, and pitch to signal stress, it is common for languages to rely on just one or two of these cues, or for one to be stronger or more reliable than the others. To illustrate with a language from the current sample: in \ili{Lelepa}, duration, pitch, and intensity are all used to signal stress, but do not necessarily co-occur, and length is noted to be a weaker correlate than the others \citep[58]{Lacrampe2014}.

  Stressed and unstressed syllables may differ in other phonetic and phonological properties as well. Processes conditioned at least in part by the stress environment, such as aspiration in stressed syllables and vowel reduction and flapping in unstressed syllables, may provide allophonic cues to stress. Stressed and unstressed syllables may also show phonological asymmetries. For instance, vowel quality or length contrasts in unstressed syllables may be reduced to a subset of those found in stressed syllables (\citealt{vanderHulst2010}). From a sound change perspective, such asymmetries reflect the phonologization of previous stress-conditioned allophonic processes and may suggest a long history of the effects of word stress in a language.

  While languages with word stress may have both primary and secondary stress patterns, crosslinguistic studies, including the current one, often focus on the properties of primary stress. This is the strongest and most prominent stress in the word. Patterns of primary stress placement vary widely among languages. Sometimes these differences are described in terms of the function of stress. Patterns in which stress predictably falls on the same syllable with respect to a word edge, such as regular penultimate stress patterns, are said to be demarcative or delimitative in their function. That is, these patterns are thought to help the listener identify word boundaries in the speech stream. In other cases, stress may serve a distinctive function by signaling differences in meaning: e.g. \ili{English} \textit{récord} (noun) versus \textit{recórd} (verb). However, it is rare for languages to have stress patterns which are entirely one or the other: most languages with delimitative stress have exceptions to the these patterns, and most languages with distinctive stress do not have many lexical items which are differentiated solely by stress \citep[14-15]{Cruttenden1997}. 

  Stress patterns may also be described in terms of the principles underlying the placement of primary stress. These may include the distance in number of syllables from a specific word or stem edge, the relative weight of syllables, or the structure of tonal sequences, or may be largely unpredictable (\citealt{vanderHulst2010}). Issues of stress placement will be discussed in greater detail in \sectref{sec:5.2.1}.

  There are several large-scale typological surveys of word stress systems. \citet{Hyman1977} examines the placement of stress in fixed stress systems in 306 languages and shows that initial, penultimate, and final position are the most frequent locations used in these systems. In a 400-language survey of phenomena related to syllable weight, \citet{Gordon2006} examines, among other issues, the crosslinguistic distribution of certain weight distinctions for stress placement. \citet{GoedemansvanderHulst2013a,GoedemansvanderHulst2013b} report on word stress placement patterns in a sample of 510 languages. This database, StressTyp, was later updated to include over 750 languages and includes fine-grained classification procedures for stress placement patterns \citep{GoedemansEtAl2017}. Some more general phonological databases also provide information on stress placement (e.g. LAPSyD, \citealt{MaddiesonEtAl2013}).

  Most large crosslinguistic studies and databases of word stress patterns are concerned with issues of stress placement. However, some of the other properties associated with stress have been investigated in smaller typological studies. \citet{Barnes2006} examines the neutralization of contrasts in height, length, and other properties of vowels in unstressed syllables in a diverse array of languages. Similarly, \citet{Crosswhite2001} investigates common neutralization outcomes of unstressed vowel reduction in 40 languages. She finds that two strong crosslinguistic patterns are prominence reduction (centralization, laxing, or raising which reduces the vowel space in unstressed syllables) and vowel peripheralization (neutralization of vowel contrasts to a few peripheral qualities of the vowel space). Fewer crosslinguistic studies have focused on phonetic (that is, not necessarily neutralizing) vowel reduction in unstressed syllables. One such study of these patterns in 81 languages found a predominance of prominence-reducing quality reduction, as well as more minor patterns of vowel raising, unrounding, devoicing, shortening, and deletion \citep{KapatsinskiEtAl2019}. In a diverse sample of 42 languages, \citet{BybeeEtAl1998} report on several properties associated with stress, including predictability of stress placement, lengthening of stressed vowels, unstressed vowel reduction, and consonantal changes conditioned by stress. The relationships between these patterns are used to support a model for the diachronic evolution of unpredictable word stress. This study will be discussed further below.

  The analyses in this chapter are concerned primarily with the properties of word stress. However, tone is additionally considered in several of the analyses. Tone can be defined as the use of pitch to convey lexical or grammatical contrasts. Like word stress, not all languages have tone. Tone is typically described in terms of contrasts in pitch ranges or pitch contours, with each range or contour being meaningfully distinctive; however, specific tones in a language are often additionally associated with other phonetic correlates, including duration and phonation properties such as glottalization \citep[477-81]{Laver1994}. The complexity of tonal systems range from relatively simple, consisting of just two tone distinctions, to much more elaborated (e.g. six or seven tones in \ili{Cantonese}, \citealt{BauerBenedict1997}). There is also wide crosslinguistic variation in the distribution or function of tone within a language. In some languages, nearly all syllables in a word have a lexically or morphologically defined tone. In others, the distribution of tone may be restricted to a single syllable or set of syllables in a word, or there may be limitations on the combinations of tones found in words (see \citealt{Hyman2009} for a discussion of commonly observed restrictions on tonal distribution). 

  Large-scale crosslinguistic studies of tone include \citet{Maddieson2013d}, which surveys the complexity of tonal systems (in number of distinctive tones) in 220 languages and relates the patterns to properties of segment inventories and syllable structure complexity. That study reports a strong geographical component to the distribution of tone languages: they are predominantly found in Africa and Southeast Asia, though they can also be found in parts of New Guinea and the Americas. Tone systems are largely absent in the regions of Australia and most of Eurasia. Other general phonological surveys, such as LAPSyD \citep{MaddiesonEtAl2013} and the World Phonotactics Database \citep{DonohueEtAl2013}, include information on tonal systems.

  Traditionally, linguists have assumed a prosodic typology in which three language types can be identified: stress languages, tone languages, and ”pitch accent” languages. The latter group is regarded as having properties of both stress and tone languages. In practice, the languages described by this term do not form a coherent group with respect to their accentual patterns. \citet{Hyman2009} argues that there are no criteria by which such languages can be defined independently of stress or tone. In his view, languages which are traditionally called ”stress” or ”tone” languages just happen to have more prototypical features with respect to these criteria.

\subsection{Suprasegmentals and syllable structure complexity}\label{sec:5.1.2}

  As discussed in \sectref{sec:1.3.1}, a long-established line of research has related properties of word stress to syllable structure complexity. The typology proposed by \citet{Pike1945} distinguished two speech rhythm types: stress-timed languages and syllable-timed languages. Later refined to include a third category of mora-timing, this typology assumed isochrony, that is, equal timing between stresses, syllables, or morae, depending upon the language type. For example, in stress-timed languages, like \ili{English}, the intervals between stressed syllables were proposed to have roughly equal durations. Syllable-timed languages, like \ili{Spanish}, were proposed to have syllables of roughly equal durations. Isochrony was ultimately disconfirmed \citep{Roach1982}, but related research established that rhythm plays a strong role in speech perception and language acquisition (e.g. \citealt{CutlerMehler1993}). Seeking to characterize measurable properties of speech rhythm, researchers proposed a number of co-occurring phonological features for each rhythm type (\citealt{Dauer1983,Auer1993}). Simple syllable structure was proposed to co-occur with syllable timing, and complex syllable structure with stress timing. Additionally, specific segmental properties and processes were suggested to co-occur with these types: stress timing, for instance, is associated with unstressed vowel reduction, contrastive vowel length, and more variable word stress patterns. Note that in this typology, syllable-timed languages may have word stress (e.g. \ili{Spanish}); it is the different properties and effects of word stress which, in part, set these rhythm types apart from one another.

  A similar holistic phonological typology developed out of the Prague School tradition. \citet{Isačenko1939/1940} proposed a typology of Slavic languages which distinguished between two types: consonantal and vocalic languages. As discussed in previous chapters, these groups of languages were defined according to properties of their phoneme inventories, syllable nuclei, and syllable structure complexity. Additionally, Isačenko considered prosodic features in this classification. Vocalic languages such as \ili{Slovene} are said to be ”polytonic,” characterized by greater distinctions in ”musical intonation” in long syllables, along with simpler syllable structure and lower consonant-to-vowel ratios in the phoneme inventory. By contrast, consonantal languages such as \ili{Russian} are said to be ”monotonic,” characterized by either dynamic or fixed stress systems, complex consonant clusters, and higher consonant-to-vowel ratios in the phoneme inventory (1939/1940: 67--69). Interestingly, the latter classification groups together languages like \ili{Russian}, which has highly unpredictable word stress placement, and \ili{Polish}, which has predominantly fixed placement of word stress.

  More recent research paradigms have attempted to establish the acoustic correlates of speech rhythm. Acoustic metrics corresponding to the proportion of vocalic intervals and standard deviation of consonantal intervals in speech, when plotted against one another, are said to index traditional rhythm categories of stress timing and syllable timing \citep{RamusEtAl1999}. These metrics have been suggested to relate directly to syllable structure complexity. When measured in a crosslinguistically diverse sample of languages carefully controlled for syllable structure complexity, the presence or absence of vowel reduction, and the presence or absence of contrastive vowel length, it was found that syllable structure complexity is indeed significantly correlated with these indices \citep{EasterdayEtAl2011}. However, it should be noted that there is debate in the literature as to the appropriateness and reliability of these and other metrics used to quantify the acoustic properties of speech rhythm \citep{WigetEtAl2010}.

  A few typological studies of moderate size have investigated relationships between stress patterns, syllable structure complexity, and phonological properties and processes. \citet{Auer1993} examines a diverse array of phonological patterns which include syllable structure, stress, vowel harmony, tone, vowel epenthesis, vowel deletion, and consonant assimilation in a sample of 34 diverse languages. He finds a number of correlations and implications between the different measures. In that sample, higher syllable complexity is correlated with a higher presence of word stress and vowel reduction processes. However, the languages show a high degree of variation with respect to most of the measures and do not fall into narrowly defined types. 

  \citet{Schiering2007} examines the distribution of ten phonetic and phonological parameters in a genealogically and geographically diversified sample of 20 languages. In this study, the parameters which most reliably cluster together are phonetic correlates of stress, segmental effects of stress, syllable complexity, and length contrasts. Specifically, languages with a high number of phonetic correlates of word stress are strongly associated with greater segmental effects of stress, more loosely associated with high syllable structure complexity and length contrasts, and negatively associated with the presence of tone and vowel harmony. However, relatively few languages show clusters of all the properties suggested to be prototypical of any rhythm class, suggesting little evidence for discrete rhythm categories. Schiering argues that the evidence instead points to a stress cline, in which gradual increases in the phonetic strength of stress are accompanied by increased segmental effects. He also raises the point that most of the proposed phonological correlates of linguistic rhythm are derived from the patterns of European languages, specifically \ili{English} and \ili{Spanish}. That the expected patterns were not held up in a diversified sample is an important finding.

  \citet{BybeeEtAl1998} explore implicational relationships among the predictability of stress, vowel length as a phonetic correlate of stress, and stress-conditioned processes of vowel reduction and consonant allophony in 42 languages. They hypothesize that as increased vowel duration gradually becomes the primary correlate of predictable word stress, decreased vowel duration becomes an important property of unstressed syllables. As stress, signaled by vowel duration, becomes incrementally stronger in a language, it conditions segmental effects such as vowel quality reduction and consonant allophony. When eventually these effects culminate in vowel deletion, the predictable stress pattern of the language may be disrupted, yielding an unpredictable stress system and an even stronger reliance on duration as a signal for stress, continuing the cycle. The implicational relationships established in the study support this path of development: for example, languages with vowel lengthening also have vowel reduction, which the authors take to imply that vowel reduction becomes a defining property of unstressed syllables before vowel length becomes a defining property of stressed syllables. While the authors do not consider syllable structure, theirs is perhaps the only study of its kind in that it attempts to reconstruct from synchronic typological evidence a diachronic path along which stress systems and concomitant phonetic patterns may develop. Those phonetic patterns, specifically vowel reduction resulting in eventual vowel deletion, are in turn relevant in the development of the consonant clusters associated with syllable structure complexity.

  Comparing the findings of \citet{BybeeEtAl1998} and \citet{Schiering2007} raises some points for further investigation. Bybee and colleagues do not report how their results relate to the syllable structure complexity of the languages examined. However, the diachronic path they propose is clearly relevant to the development of syllable structure complexity. Interestingly, though, Schiering did not find the predictability of stress placement to be reliably correlated with segmental effects of stress or with syllable structure complexity. Instead, he found that the relative strength of stress, in terms of the number of phonetic correlates, was robustly associated with both segmental effects of stress and syllable structure complexity.

  Regarding relationships between tone and syllable structure complexity, some patterns have been noted in the literature. Specifically, \citet{Maddieson2013d} established an inverse relationship between the elaboration of tonal contrasts and syllable structure complexity in a survey of 471 languages. Additionally, languages lacking tone altogether were found to be much more likely to have complex syllable patterns. This concurs with findings by \citet{Auer1993} and \citet{Schiering2007}.

\subsection{The current study and hypotheses}\label{sec:5.1.3}

  The findings discussed above indicate that there are at least some crosslinguistic associations between word stress, its correlates, placement, and segmental effects, and syllable structure complexity. However, none of the studies mentioned above, apart from the acoustic study in \citet{EasterdayEtAl2011}, were conducted in a sample carefully chosen to equally represent differing degrees of syllable structure complexity. Languages with Simple or Highly Complex syllable structure, as defined here, were particularly rare in the samples of \citet{Auer1993}, \citet{BybeeEtAl1998}, and \citet{Schiering2007}, owing to the methods of sample construction used and the relatively lower global frequencies of such languages. It is therefore appropriate to explore these issues in the current study, with the aim of addressing the main research questions of the book. In this chapter I seek to establish the suprasegmental properties, specifically those related to word stress, associated with highly complex syllable structure. In turn, these findings will be used to inform a picture of the diachronic development of these structures.

  The first two hypotheses follow from the research of \citet{BybeeEtAl1998}, as well as observations of ongoing processes in the current sample and findings from \chapref{sec:3}. Bybee and colleagues found that the segmental effects of stress were stronger in languages with less predictable stress patterns, which in turn may come about through segmental effects of stress, namely unstressed vowel reduction and deletion. In the current sample, there are several examples of recent or ongoing processes which support this diachronic path. For example, in \ili{Imbabura Highland Quichua}, the regular penultimate stress pattern has recently been destabilized as a result of the reduction of validator suffixes: \textit{-t͡ʃari-} > \textit{-t͡ʃa-} ‘doubt’ and \textit{-mari-} > \textit{-ma-} ‘emphatic firsthand information.’ \citet{Cole1982} reports that words with the short forms of these suffixes usually carry word-final stress \REF{ex:5.1}.

\ea\label{ex:5.1}
  \textbf{Imbabura Highland Quichua} (\textit{Quechuan}; Ecuador)

ʃamunɡaˈt͡ʃa

ʃamu-n-ɡa-t͡ʃa

come-3-\textsc{fut-dub}\\
\glt ‘perhaps he will come’
\citep[209]{Cole1982}
\z

  The process in \ili{Imbabura Highland Quichua} may illustrate an early stage of the diachronic path proposed in \citet{BybeeEtAl1998}, as it is limited to a (presumably frequent) set of grammatical constructions and the stress pattern of the language is still largely predictable. A process which may illustrate later stages of this diachronic path can be found in \ili{Lezgian}. In this language, stress placement was until recently largely predictable within the stem, typically falling on the second syllable therein. It has recently become more unpredictable with recent and ongoing processes of unstressed vowel deletion. The most productive such process involves the deletion of high vowels which follow voiceless obstruents in pretonic syllables \citep[36]{Haspelmath1993}. The history of this process is long enough that it is reflected in the standard spelling for some lexical items. An example of a word which still shows variation in pronunciation is given in \REF{ex:5.2}.

\ea\label{ex:5.2}
  \textbf{Lezgian} (\textit{Nakh-Daghestanian}; Azerbaijan, Russia)

t͡ʃʰiˈneba {\textasciitilde} ˈt͡ʃʰneba\\
\glt ‘secretly’
\citep[38]{Haspelmath1993}
\z

As a result of this process, the stress placement, which was already somewhat unpredictable in that it was associated with position in the stem and not the word, has become even more unpredictable. Additionally, there is now a tendency in the language for post-tonic vowels to be deleted in certain consonantal environments. Haspelmath reports that this process is mostly restricted to inflectional suffixes, but there are a few cases where it seems to be more general \REF{ex:5.3}.

\ea\label{ex:5.3}
  \textbf{Lezgian} (\textit{Nakh-Daghestanian}; Azerbaijan, Russia)

diˈdedilaj {\textasciitilde} diˈdedlaj\\
\glt ‘from mother’
\citep[40]{Haspelmath1993}
\z

  While the process in \REF{ex:5.3} does not currently affect word-final syllables, it is reasonable to imagine that the reductive processes conditioned by stress may continue to spread to other similar environments in the language, creating more complex syllable patterns as they phonologize. A natural outcome of such a scenario in any language, assuming it has affixation and/or polysyllabic stems, would be large maximal onset and coda structures. This would nicely account for the strong pattern in canonical syllable shapes discussed in \sectref{sec:3.3.2} in which languages with a large maximal consonant cluster at one syllable margin tend to have a similarly large maximal cluster in the other syllable margin.

  With these points in mind, I formulate the following two hypotheses for the current study \xxref{ex:5.4}{ex:5.5}.

\ea\label{ex:5.4}
  H\textsubscript{1}: As syllable structure complexity increases, so does the proportion of languages in which the placement of word stress is unpredictable.
\z
  
\ea\label{ex:5.5}
  H\textsubscript{2}: As syllable structure complexity increases, word stress has stronger phonetic and phonological effects in languages.
\z

  Any associations between unpredictable stress and the strength of its segmental effects must be accounted for by some specific property of stress in the language. \citet{BybeeEtAl1998} propose that these effects arise when vowel duration gradually becomes a phonetic correlate of stress. \citet{Schiering2007}, on the other hand, finds stronger segmental effects of stress in languages with more co-occurring phonetic correlates of stress. These observations lead to formulate two additional hypotheses \xxref{ex:5.6}{ex:5.7}.

\ea\label{ex:5.6}
  H\textsubscript{3}: As syllable structure complexity increases, so does the likelihood that vowel duration is used as a phonetic correlate of word stress.
\z

\ea\label{ex:5.7}
  H\textsubscript{4}: As syllable structure complexity increases, word stress will be signaled by an increasing number of phonetic correlates.
\z

  These hypotheses will be tested in upcoming sections. It should be mentioned at the outset that prominence and accentual and tonal phenomena occurring at higher levels of phonological organization may contribute to stress patterns, segmental processes, and articulatory coordination in important ways (e.g. \citealt{FougeronKeating1997}). However, the description of such patterns in standard language references is often impressionistic, inconsistent, or altogether absent. While the phenomena considered here are limited to stress and tonal patterns within the word, it is important to acknowledge that there are many additional factors which might affect the patterns observed.

\section{Methodology}\label{sec:5.2}
\subsection{Patterns considered}\label{sec:5.2.1}

  In the current study, only primary stress patterns are considered. Furthermore, it is the dominant patterns which are considered and coded here; that is, patterns for which there may be exceptions in a handful of words or grammatical constructions, but which are not obscured by these exceptions. After excluding such minor deviations, the stress pattern of each language was characterized according to the organizing principles underlying stress assignment. In doing so, it was useful to first make a distinction between stress patterns dependent upon phonological structure, on the one hand, and those dependent upon morphological or lexical structure, on the other hand.

   Within languages in which stress placement depends upon phonological factors, several different kinds of phonological factors may determine stress assignment. In a fixed stress pattern, stress always falls on the same syllable of a word in relation to a word boundary: stress may regularly fall on the initial syllable of a word, for example, or the antepenultimate syllable of a word. This pattern occurs regardless of what kind of morpheme (root/stem or affix) that syllable happens to belong to. For example, in \ili{Cocama-Cocamilla}, stress falls predictably on the penultimate syllable of each word. Thus stress shifts as additional morphemes are added to a word (\ref{ex:5.8}a-c).

\ea\label{ex:5.8}
  \textbf{Cocama-Cocamilla} (\textit{Tupian}; Peru)

\ea   ˈkaɾi\\
\glt ‘drag’

\ex   kaɾiˈtaka\\
\glt ‘limp dragging a foot’

\ex   kaɾikaɾiˈtaka\\
\glt ‘limp jumping on a foot’

(\citealt{VallejosYopán2010}: 121)
\z
\z

  As mentioned above, in most languages, some exceptions to the dominant stress patterns are reported. To illustrate this with \ili{Cocama-Cocamilla}, when one of a small group of morphemes, including the relativizer /-n/, are affixed to a stem, the stress may shift to the final syllable (\ref{ex:5.9}c; \ref{ex:5.9}a--b show the typical penultimate pattern).

\ea\label{ex:5.9}
  \textbf{Cocama-Cocamilla} (\textit{Tupian}; Peru)
\ea   ˈmuna\\
  muna\\
steal

\ex  muˈnaɾi\\
  muna-ɾi\\
steal-\textsc{prog}

\ex  munaˈɾin\\
  muna-ɾi-n\\
steal-\textsc{prog}-\textsc{rel}

(\citealt{VallejosYopán2010}: 122)
\z
\z

  Another stress pattern which is not morphologically or lexically conditioned but predictable from the phonological structure of a word is a weight-sensitive system. In such a system, stress falls on a heavy syllable, usually defined as one having a long vowel or a coda, or occasionally specific vowel qualities \citep{Gordon2006}. Stress patterns which are sensitive to syllable weight are often additionally oriented towards one of the word edges (\citealt{GoedemansvanderHulst2013b}). For example, in \ili{Kabardian}, stress falls on the final syllable of the word if it is heavy. If the final syllable is light, then stress falls on the penultimate syllable instead (\ref{ex:5.10}a--b).

\ea\label{ex:5.10}
  \textbf{Kabardian} (\textit{\ili{Abkhaz}-Adyge}; Russia, Turkey)

\ea   lɐˈʒaː\\
\glt ‘work (\textsc{pst} \textsc{interrog})’

\ex  ˈməʃɐ\\
\glt ‘bear’

(\citealt{GordonApplebaum2010}: 38)
\z
\z

  There are some other less common scenarios in which stress placement is dependent upon phonological factors. In \ili{Southern Bobo Madaré}, which is reported to have both word stress and tone, the tonal pattern of the word determines the stress placement. Stress falls on the first of two identical tones in disyllabic words (\ref{ex:5.11}a), and in phonological words with one or more high tones, stress falls on the first high tone (\ref{ex:5.11}b).

\ea\label{ex:5.11}
  \textbf{Southern Bobo Madaré} (\textit{Mande}; Burkina Faso)

\ea  \textsf{ˈ}ba\textsf{\={} }ɾa\textsf{\={} }\\
\glt ‘work’

\todo{check diacritics}
\ex  nĩ \`{} mĩ \`{} sa\`{} ˈla\'{} lo\`{} \\
\glt ‘boy’
\citep[110]{Morse1976}
\z
\z

  In languages in which the dominant stress pattern is determined by morphological factors, phonological factors may still play a role. For example, in \ili{Tehuelche}, stress always falls on the initial syllable of a stem. In words without prefixation, stress is word-initial, but in words with prefixation, it is not. While stress is not predictable from the phonological form of the word, it is predictable within the stem itself (\ref{ex:5.12}a-c).

\ea\label{ex:5.12}
  \textbf{Tehuelche} (\textit{Chonan}; Argentina)

\ea  ˈqampen\\
\glt ‘sheep’

\ex  ˈjeʃemk’en

jeʃem-k’en

spring-\textsc{nmlz}\\
\glt ‘spring’

\ex   ʔoˈmaːnk

ʔo-maː-n-k

\textsc{nmlz}-kill-\textsc{nmlz-m}\\
\glt ‘assassin’

(\citealt{FernándezGaray1998}: 107-8)
\z
\z

Similarly, syllable weight may factor into stress placement in languages in which stress is always associated with a root or stem, producing a phonologically predictable stress pattern within the root/stem (e.g. in \ili{Mamaindê} in the current sample).

  Other languages have stress placement which is morphologically conditioned, but much less predictable. In \ili{Yakima Sahaptin}, all words carry one main stress. All roots have an unpredictable lexically-determined stress, such that there are near-minimal pairs for stress in the language (\ref{ex:5.13}a-c). \citet{HargusBeavert2005} report that there are statistical preferences for stress placement in roots: it tends to fall in heavy syllables, to be trochaic when syllable weight is not a factor, and to have right directionality within the root. However, besides many exceptions to these patterns within roots themselves, there are additional complicating factors in stress assignment owing to affixation. Some affixes do not alter the stress pattern of the word (\ref{ex:5.13}d), but nearly half of the affixes in the language carry stress and cause stress to shift from or within the root. When stressed affixes are attached to a root, stress is preferentially assigned to a stressed suffix over a stressed prefix or root, and to a stressed prefix over a root (\ref{ex:5.13}e-f). Additionally, there are some suffixes which do not attract stress to themselves but which shift it to another position within the root (\ref{ex:5.13}g).

\ea\label{ex:5.13}
  \textbf{Yakima Sahaptin} (\textit{Sahaptian}; USA)

\ea  ˈwjanawi-\\
\glt ‘arrive’

\ex  aˈnawi-\\
\glt ‘be hungry’

\ex  kʷ’ajaˈwi\\
\glt ‘mountain lion’

\ex  paˈp’ɨχʃa

pa-ˈp’ɨχʃa

3.\textsc{pl.nom}-remember\\
\glt ‘they remember’

\ex  ˈpapap’ɨχʃa

ˈpapa-ˈp’ɨχʃa

\textsc{recp}-remember\\
\glt ‘they remember each other’

\ex  pɨtjaˈɬa

ˈpɨtja-ˈɬa

spear-\textsc{agt}\\
\glt ‘spearer’

\ex  at͡ɬ’aˈwiɬam ‘beggar’ (< aˈt͡ɬ’awi- ‘ask, beg for, request’)

(\citealt{HargusBeavert2005}: 66-7, 77, 92)
\z
\z

  In some languages, the stress pattern is unpredictable in a different way: it is highly variable in general. \citet{Marmion2010} describes stress in \ili{Wutung} as being present in words of two syllables or more, but being neither phonemic nor predictable. The stress pattern of a word may vary freely between speakers and within the same speaker (\ref{ex:5.14}a). Nevertheless, stress is perceptually salient and the variable location of its placement may have a strong effect on the realization of certain sequences (\ref{ex:5.14}b).

\ea\label{ex:5.14}
  \textbf{Wutung} (\textit{Sko}; Papua New Guinea)

\ea  /hlapã /

[ˈhlapã ] {\textasciitilde} [hlaˈpã ]\\
\glt ‘night’

\ex  /huwɵ/

[huˈwɵ] {\textasciitilde} [ˈhuːɵ]\\
\glt ‘stomach’

(\citealt{Marmion2010}: 57, 91)
\z
\z

  As the examples above show, there is a great deal of crosslinguistic variation in stress placement patterns. The stress patterns in the language sample must be categorized in a principled way in order to address the hypothesis in \REF{ex:5.4}. In operationalizing stress predictability, I follow \citet{Schiering2007}, who in turn follows an earlier version of \citet{GoedemansvanderHulst2013a}. In this system, three types of stress placement systems are distinguished: fixed stress location, weight-sensitive stress placement, and morphologically or lexically conditioned stress placement. On this scale, fixed stress systems have the highest predictability and morphologically or lexically conditioned stress systems have the lowest predictability. In testing the hypothesis in \REF{ex:5.4}, the few stress placement patterns which do not fit into these categories, such as the tone-conditioned system in \ili{Southern Bobo Madaré} \REF{ex:5.11} and the variable system in \ili{Wutung} \REF{ex:5.14}, are excluded. However, these systems are included in other parts of the study, including the analyses of phonetic correlates of stress and stress-conditioned phonetic processes.

  In order to address the hypothesis in \REF{ex:5.5} regarding the segmental effects of stress, phonetic processes reported to be conditioned by stress were collected. The processes considered include vowel reduction in unstressed syllables and consonant allophony in stressed or unstressed syllables. A reported process of regular vowel lengthening in stressed syllables was considered to be a phonetic correlate of stress (see below) and was not coded as a stress-conditioned process. Like segmental inventories, allophonic processes are always the result of analyses. In order to avoid some of the better-known pitfalls of synchronic phonological analysis (see \sectref{sec:6.2} for more discussion of this point), I have limited the processes examined here to those conditioned solely by the phonological environment. That is, processes described as occurring within specific morphological or morphophonemic environments have been excluded.

  Vowel reduction processes here include any process by which the vowel is deleted or reduced in quality, duration, or voicing. Other less common effects such as the loss of tonal contrasts have been considered as well. A reduction in vowel quality typically includes centralization, laxing, or raising towards a more ‘neutral’ or less sonorant vowel quality. However, in some languages, vowel reduction may involve a neutralization of contrasts which maximizes peripheral contrasts (\citealt{Crosswhite2001}; see discussion in \sectref{sec:6.2}). I have included raising as reduction when it is explicitly described as such in the reference, but not lowering. This is consistent with the results of \citet{KapatsinskiEtAl2019}, who only found evidence for the raising component of peripheralization at the phonetic level. In some languages for which vowel duration is a correlate of word stress, the relatively shorter vowel duration in all unstressed syllables is reported as vowel reduction. I have not included such cases here, but have included processes in which the reduced length of a vowel is shorter than what would normally be expected for unstressed vowels: for instance, extra shortening of unstressed vowels in pretonic position. I have included all vowel reduction processes in unstressed syllables regardless of whether stress is the sole conditioning environment; that is, the processes include those occurring in unstressed syllables but requiring additional conditioning factors such as word position or consonantal environment. Additional details on the collection and coding of unstressed vowel reduction processes can be found in \sectref{sec:6.2}, where I describe the methodology behind a more general study of vowel reduction in the language sample. Illustrative examples of some of the vowel reduction processes considered in the current chapter can be found in \xxref{ex:5.15}{ex:5.17}.

\ea\label{ex:5.15}
  \textbf{Bardi} (\textit{Nyulnyulan}; Australia)

Short vowels are reduced in quality in unstressed syllables.

/\textsf{ˈ}ɡamaɖa/

[\textsf{ˈ}kamɜɖa]\\
\glt ‘mother’s mother’
\citep[88-90]{Bowern2012}
\z

\ea\label{ex:5.16}
  \textbf{Apurinã} (\textit{Arawakan}; Brazil)

Vowels become devoiced in unstressed word-final position, especially in fast speech.

/mapoˈɰat͡sa/

[mapoˈɰat͡sḁ]\\
\glt ‘caterpillar’
\citep[60-61]{Facundes2000}
\z

\ea\label{ex:5.17}
  \textbf{Choctaw} (\textit{Muskogean}; USA)

A word-initial unstressed high front vowel /i/ may be deleted before a sequence of /s/ or /ʃ/ and another consonant.

/iskitiːˈnih/

[iskitiːˈnih] {\textasciitilde} [skitiːˈnih]\\
\glt ‘it’s small’
\citep[19]{Broadwell2006}
\z

  Processes of consonant allophony conditioned by word stress were also considered in the current study. I did not limit the data collection to specific kinds of processes, but included any phonetic process affecting consonants which was reported to be conditioned by the stress environment, either alone or in addition to other conditioning factors. Processes affecting consonants in unstressed syllables often include the voicing, flapping, or spirantization of stops, but less common processes such as spirantization of affricates, debuccalization, and deletion of consonants also occur. See \xxref{ex:5.18}{ex:5.21} for examples of some of the patterns recorded.

\ea\label{ex:5.18}
  \textbf{Pinotepa Mixtec} (\textit{Otomanguean}; Mexico)

Plosives /t͡ʃ k kʷ/ in post-tonic syllables are voiced on occasion.

/ˈtʃikaɾa\={} /

[ˈtʃiɡaɾa\={} ]\\
\glt ‘he is walking’
\citep[5]{Bradley1970}
\z

\ea\label{ex:5.19}
  \textbf{Tukang Besi} (\textit{Austronesian}; Indonesia)

A voiced velar stop /ɡ/ may lenite to [ɣ] between unstressed vowels.

/n̪oɡɯˈɡɯd̪ɯ/

[n̪oɣɯˈɡɯd̪ɯ]\\
\glt ‘they make noise’
\citep[27]{Donohue1999}
\z

\ea\label{ex:5.20}
  \textbf{Cubeo} (\textit{Tucanoan}; Colombia)

Voiceless stops /p k/ may be realized as [h] directly following a stressed syllable.

/ˈhapuɾabi/

[ˈhahuɾabi]\\
\glt ‘he is heard’
\citep[123]{Chacon2012}
\z

\todo{check diacritics}
\ea\label{ex:5.21}
  \textbf{Pech} (\textit{Chibchan}; Honduras)

In rapid speech, glottal fricative /h/ is often deleted following a stressed vowel and preceding an unstressed vowel.

/ˈka\`{} hã /

[ˈkã \`{} ː]\\
\glt ‘town’
\citep[24]{Holt1999}
\z

  Processes affecting consonants in stressed syllables were less frequently reported. These processes often include aspiration or affrication of stops, glide strengthening, and lengthening; less frequent patterns include devoicing and place assimilation \xxref{ex:5.22}{ex:5.24}.

\ea\label{ex:5.22}
  \textbf{Maori} (\textit{Austronesian}; New Zealand)

A voiceless velar stop may be affricated preceding /a/ at the onset of a stressed syllable.

/ˈkaɾaŋa/

[ˈk͡xaɾaŋa]\\
\glt ‘call’
\citep[521-522]{Bauer1999}
\z

\ea\label{ex:5.23}
  \textbf{Tu} (\textit{Mongolic}; China)

Palatal glide /j/ may be spirantized word-initially and in syllables which are stressed.

/ˈja/

[ˈʝa]\\
\glt ‘what’
\citep[31-32]{Slater2003}
\z

\ea\label{ex:5.24}
  \textbf{Nivkh} (isolate; Russia)

In stressed syllables, consonants followed by front vowels /i e/ may acquire secondary palatalization.

/ˈkʰeq/

[ˈkʰʲeq]\\
\glt ‘fox’
\citep[23]{Shiraishi2006}
\z

  As discussed above, segmental effects of stress in a language may become phonologized, which over time may result in phonological differences between unstressed and stressed syllables. In particular, unstressed syllables may show a limited range of contrasts in consonants, consonant combinations, vowel qualities, vowel length, and tone (\citealt{vanderHulst2010}). In addition to considering phonetic processes conditioned by stress, I have coded languages for stress-related phonological asymmetries. In some cases, the asymmetry between stressed and unstressed syllables is limited to just one contrastive feature. For example, in \ili{Burushaski}, vowel length contrasts are limited to stressed syllables in underived lexical items \citep[1028]{Anderson1997}. However, in some languages there are dramatic phonological differences between stressed and unstressed syllables. Such systems are common in some of the language families of Southeast Asia, where words often have a sesquisyllabic pattern: a stressed main syllable preceded by a presyllable which is unstressed and highly limited in its phonological composition (\citealt{Matisoff1973,Michaud2012}; similar patterns may be found in some language families of Mesoamerica, including Otomanguean). For example, presyllables in \ili{Koho} are limited to three general shapes: a sequence of an unaspirated, unimploded obstruent, /ə/, and an optional liquid or nasal coda; the sequence /ʔa/; or a syllabic nasal. By contrast, main syllables show the full range of consonant and vowel contrasts, and may have tautosyllabic consonant clusters. See (\ref{ex:5.25}a-c) for examples.

\ea\label{ex:5.25}
  \textbf{Koho} (\textit{Austroasiatic}; Vietnam)

\ea  sənˈdjaŋ\\
\glt ‘steep side of a valley’

\ex  ʔaˈsuh\\
\glt ‘to blow on a fire’

\ex  m̩ˈpoŋ\\
\glt ‘door’

(\citealt{Olsen2014}: 32, 46, 48)
\z
\z

  As discussed in \sectref{sec:5.1.2}, there may be associations between specific phonetic correlates of stress and the extent to which stress has segmental effects in a language (\citealt{BybeeEtAl1998,Schiering2007}). This observation motivated the hypotheses in \xxref{ex:5.6}{ex:5.7}. In coding for this study, three phonetic correlates of stress -- vowel duration, pitch, and intensity -- were noted wherever explicitly described in language references. In older language references, in particular, phonetic descriptions of stress are often impressionistic, if they are included at all. More recent works sometimes give instrumental evidence for the phonetic correlates of stress. In coding for the phonetic correlates of stress, I differentiate between reports which were impressionistic and those which were based on instrumental measurements. Where sources disagree on phonetic correlates of stress, I give preference to descriptions based on instrumental measurements, if available.

  As mentioned above, relationships between the presence and complexity of tonal systems and syllable structure complexity have been established in the literature \citep{Maddieson2013d}. Because none of the hypotheses in the current chapter are directly related to tone, I do not present a detailed analysis of these patterns here. Degrees of complexity within tonal systems (cf. \citealt{Maddieson2013d}), for example, are not considered. I also do not distinguish between ``prototypical'' tonal systems, in which most syllables bear tone and tonal combinations are relatively free, and systems in which tonal patterns are relatively more restricted (cf. \citealt{Hyman2009}). However, it is important to consider the presence of tone in the languages of the sample. Since tone makes use of pitch contrasts and pitch is often a correlate of word stress, there is the potential that word stress may manifest in phonetically different ways depending on the presence or absence of a tonal system \citep{Gordon2011}. This in turn could be reflected in any associations observed between phonetic correlates of stress and syllable structure complexity in the sample. 

\subsection{Coding}\label{sec:5.2.2}

  The information gathered on word stress and tonal patterns in the sample was coded as follows. First, the presence or absence of tone and word stress were noted. If a language was noted as having word stress, the dominant stress placement pattern was coded as one of the following: Fixed, meaning stress falls in a predictable location with respect to word boundaries; Weight-Sensitive, meaning stress placement is sensitive to factors such as vowel length, presence of a coda, and/or vowel quality but can be determined from the phonological, and not morphological, structure of a word; and Morphologically or Lexically Conditioned, in which a description of the stress pattern must refer to the morphology. As mentioned above, this classification is meant to correspond to a three-point scale representing the predictability of stress placement (cf. \citealt{Schiering2007}). 

  The presence of phonetic processes reported to be conditioned by stress were coded, and the type of process noted. These include Vowel Reduction, Consonant Allophony in Unstressed Syllables, and Consonant Allophony in Stressed Syllables. Differences in the phonological properties of stressed and unstressed syllables were also noted and coded as: Vowel Quality Contrasts, Vowel Length Contrasts, Consonant Contrasts, Tonal Contrasts, and Other. Phonetic correlates of stress were coded as Vowel Duration, Pitch, and Intensity. Here the category Pitch includes both level pitch (usually higher than in unstressed syllables) and pitch contours associated with stressed syllables. Each reported correlate was additionally coded for whether it was based on impressionistic observations or instrumental evidence.

  In \REF{ex:5.26} I illustrate the coding for \ili{Kadiwéu}, a language with Complex syllable structure.

\ea\label{ex:5.26}
  \textbf{Kadiwéu} (\textit{Guaicuruan}; Brazil)

\textbf{Tone:} No

\textbf{Word stress:} Yes

\textbf{Stress placement:} Weight-Sensitive

\textbf{Phonetic processes conditioned by stress:} Consonant Allophony in Unstressed Syllables

\textbf{Differences in phonological properties of stressed and unstressed syllables:} (None)

\textbf{Phonetic correlates of stress:} Intensity (impressionistic)
\z

The coding for each language in the sample can be found in Appendix B.

  In the following sections I present analyses to test the hypotheses in \xxref{ex:5.4}{ex:5.7}. Because the analysis of tone is brief, I present it first before moving on to the study of word stress in the sample.

\section{Results: Tone}\label{sec:5.3}

  The distribution of tone in the languages of the sample with respect to syllable structure complexity can be found in \tabref{tab:5.1}.

\begin{table}
\begin{tabularx}{\textwidth}{Qcccc}
\lsptoprule
 & \multicolumn{4}{c}{Syllable structure complexity}\\\cmidrule(lr){2-5}
  & S & MC & C & HC\\
  \textit{N} languages with tone    & \textit{N} = 24 & \textit{N} = 26 & \textit{N} = 25 & \textit{N} = 25\\\midrule
 Present & 13 & 9 & 9 & 6\\
 Absent or not reported & 11 & 17 & 16 & 19\\
\lspbottomrule
\end{tabularx}
\caption{\label{tab:5.1}Languages of sample distributed according to presence of tone.}
\end{table}

  Tone is present in 37 languages of the sample. The proportion of languages reported to have tone decreases with syllable structure complexity: more than half of the languages in the Simple category have tone, while only one-fifth of the languages in the Highly Complex category do. This finding is in line with the findings of \citet{Maddieson2013d}. As observed in that work, there is a strong areal component to the distribution of tone: 21 of the languages with tone in the current study are located in the macro-areas of Africa and Southeast Asia \& Oceania. Within these regions, languages of all syllable structure complexity types can be found to have tone.

  The issue of tone will be revisited in \sectref{sec:5.4.1}, then further in \sectref{sec:5.4.5} in an analysis of phonetic correlates of stress.

\section{Results: Stress}\label{sec:5.4}

  The analyses in this section examine properties of stress in the language sample. This section is organized as follows. \sectref{sec:5.4.1} presents a general description of the presence of stress in the sample. Stress placement patterns are analyzed in \sectref{sec:5.4.2}. Phonetic processes conditioned by stress are examined in \sectref{sec:5.4.3} and subsections therein. Phonological asymmetries between stressed and unstressed syllables are analyzed in \sectref{sec:5.4.4}. In \sectref{sec:5.4.5} the phonetic correlates of stress in the sample are examined. The results of the analyses of word stress properties are summarized in \sectref{sec:5.4.6}.

\subsection{Presence of word stress and syllable structure complexity}\label{sec:5.4.1}

  The distribution of word stress in the languages of the sample can be found in \tabref{tab:5.2}. In this analysis I have excluded two languages: for \ili{Southern Grebo} and \ili{Qawasqar}, there are conflicting reports regarding the presence or absence of stress.

\begin{table}
\begin{tabularx}{\textwidth}{Qcccc}
\lsptoprule
 & \multicolumn{4}{c}{Syllable structure complexity}\\\cmidrule(lr){2-5}
  & S & MC & C & HC\\
   \textit{N} languages with word stress  & \textit{N} = 23 & \textit{N} = 26 & \textit{N} = 25 & \textit{N} = 24\\\midrule
 Present & 18 & 19 & 21 & 21\\
 Absent or not reported & 5 & 7 & 4 & 3\\
\lspbottomrule
\end{tabularx}
\caption{\label{tab:5.2}Languages of sample distributed according to presence of word stress. Southern Grebo (Simple category) and Qawasqar (Highly Complex category) have been excluded.}
\end{table}

  Most of the languages in the sample (79/100) are reported to have word stress. While there is a moderate increase in this property from the Moderately Complex (19/26 languages, or 73\%) to the Highly Complex category (21/24, or 88\%), this trend is not monotonic as the Simple category has an intermediate pattern.

  In \tabref{tab:5.3}, I combine the patterns from Tables \ref{tab:5.1} and \ref{tab:5.2} to show how both tone and word stress are distributed in the languages of the sample.

\begin{table}
\begin{tabularx}{\textwidth}{Qcccc}
\lsptoprule
 & \multicolumn{4}{c}{Syllable structure complexity}\\\cmidrule(lr){2-5}
  & S & MC & C & HC\\
   \textit{N} languages with & \textit{N} = 23 & \textit{N} = 26 & \textit{N} = 25 & \textit{N} = 24\\\midrule
 Word stress only & 11 & 15 & 15 & 17\\
 Tone only & 5 & 5 & 3 & 2\\
 Both tone and word stress & 7 & 4 & 6 & 4\\
 Neither & -- & 2 & 1 & 1\\
\lspbottomrule
\end{tabularx}
\caption{\label{tab:5.3}Languages of sample distributed according to presence of word stress and/or tone. Southern Grebo (Simple category), and Qawasqar (Highly Complex category) have been excluded.}
\end{table}

  The pattern in the first row in \tabref{tab:5.3} indicates that the percentage of languages with word stress but no tonal contrasts increases with syllable structure complexity (48\% of languages in the Simple category versus 71\% of languages in the Highly Complex category). This result could reflect the fact that there is little geographic overlap between macro-areas where tonal systems are common and those where more complex syllable patterns are common.

  There are four languages in the sample for which neither word stress nor tone are reported to be present: \ili{Kalaallisut}, \ili{Kharia}, \ili{Oksapmin}, and \ili{Tashlhiyt}. In the case of \ili{Kalaallisut}, instrumental evidence has been presented to support the analysis of the language as having no stress \citep{Jacobsen2000}. Similarly, instrumental evidence has been used to show that what is often analyzed as word stress in \ili{Tashlhiyt} is actually an effect of phrase-level accentual patterns \citep{RoettgerEtAl2015}.

\subsection{Stress assignment}\label{sec:5.4.2}

  In this section, I present an analysis addressing the hypothesis, formulated in \REF{ex:5.6}, that as syllable structure complexity increases, so does the proportion of languages in which stress placement is unpredictable.

  The distribution of the languages with word stress in the sample according to their dominant stress placement patterns can be found in \tabref{tab:5.4}. In this analysis, one language (\ili{Menya}) has been excluded because the descriptions of stress patterns were too minimal to allow for classification. Thus the current analysis includes 78 languages.

\begin{table}
\begin{tabularx}{\textwidth}{Qcccc}
\lsptoprule
& \multicolumn{4}{c}{Syllable structure complexity}\\\cmidrule(lr){2-5}
& S & MC & C & HC\\
   Word stress placement pattern & \textit{N} = 18 & \textit{N} = 19 & \textit{N} = 21 & \textit{N} = 20\\\midrule
 Fixed & 7 & 10 & 12 & 8\\
 Weight-sensitive & 4 & 2 & 2 & 4\\
 Morphologically or Lexically Conditioned & 6 & 4 & 6 & 7\\
 Variable or other & 1 & 3 & 1 & 1\\
\lspbottomrule
\end{tabularx}
\caption{\label{tab:5.4}Languages of sample with word stress distributed according to their dominant stress placement patterns. Menya (Highly Complex category) has been excluded.}
\end{table}

  The patterns classified as ``Variable or other'' in \tabref{tab:5.4} are those whose stress placement is determined by phonological factors other than location with respect to the word edge or weight, or whose stress patterns may vary widely according to speaker or situational context, like the examples in \REF{ex:5.11} and \REF{ex:5.14} in \sectref{sec:5.2.1}. Excluding these languages, I plot the patterns of the remaining 72 languages in \figref{fig:5.1}. This figure shows the percentage of languages in each category of syllable structure complexity having the given stress placement pattern.

\begin{figure}
\caption{\label{fig:5.1}Percentage of languages exhibiting each of the given stress placement patterns, by syllable structure complexity.}
\begin{tikzpicture}
\pgfplotstableread{data/fig51.csv}{\table}
    \pgfplotstablegetcolsof{\table}
    \pgfmathtruncatemacro\numberofcols{\pgfplotsretval-1}
            \begin{axis}[easterdaystacked,reverse stacked plots=false,
                                xticklabels={S,MC,C,HC},legend style={cells={align=left}}
                        ]
            \foreach \i in {1,...,\numberofcols} {
                \addplot+[
                    /pgf/number format/read comma as period, fill
                    ] table [x index={1},y index={\i},x expr=\coordindex] {\table};
                \pgfplotstablegetcolumnnamebyindex{\i}\of{\table}\to{\colname} % Adding column headers to legend
            }
            \legend{Fixed,Weight-sensitive,Morphologically or\\Lexically Conditioned}
            \end{axis}                                                                           
\end{tikzpicture}
\end{figure}

  Recall that the three categories for stress placement employed here can be used as measures for the predictability of stress placement, with fixed systems being most predictable and morphologically or lexically conditioned systems being least predictable. Interpreting the patterns in \figref{fig:5.1}, we find that the percentage of languages with the least predictable -- that is, morphologically or lexically conditioned -- stress placement mildly increases from the Moderately Complex to the Highly Complex category. However, the linear trend is again broken by the pattern in the Simple category, which shows a pattern virtually identical to that of the Highly Complex category. Thus, while the hypothesis in \REF{ex:5.4} is perhaps weakly supported within the set of languages with non-Simple syllable structure, we do not find general support for the hypothesis.

  Recall the discussion of patterns of stress placement determined by morphological factors in \sectref{sec:5.2.1}. In some languages, stress is morphologically determined but predictable within the morphological domain; for example, in \ili{Tehuelche}, stress always falls on the initial syllable in the stem \REF{ex:5.12}. In other languages, stress placement may be lexically determined and/or sensitive to morphological factors, but these factors are so complex that stress is largely unpredictable; this is the case for stress in \ili{Yakima Sahaptin} \REF{ex:5.13}. There are also languages in which morphologically conditioned stress placement is intermediate between these extremes. For example, in \ili{Choctaw}, accent is predictable for all underived verbs and deverbal nouns, but unpredictable in underived nouns and some other contexts \citep{Broadwell2006}. In \tabref{tab:5.5}, I have distributed the 22 languages with morphologically or lexically conditioned stress by the predictability of those systems. Systems like that of \ili{Tehuelche} are classified as having stress which is predictable within the stem. Systems like that of \ili{Yakima Sahaptin} are classified as having stress which is unpredictable. Systems somewhere in between, like that of \ili{Choctaw}, are given an intermediate classification. The classification of languages into these categories is impressionistic to some extent.

\begin{table}
\begin{tabularx}{\textwidth}{Qcccc}
\lsptoprule
& \multicolumn{4}{c}{Syllable structure complexity}\\\cmidrule(lr){2-5}
& S & MC & C & HC\\
   Languages with morphologically or lexically conditioned stress placement & \textit{N} = 6 & \textit{N} = 4 & \textit{N} = 6 & \textit{N} = 7\\\midrule
 Predictable within stem & 2 & 2 & 2 & 2\\
 Intermediate & 3 & 1 & 2 & 2\\
 Unpredictable & 1 & 1 & 2 & 3\\
\lspbottomrule
\end{tabularx}
\caption{\label{tab:5.5}Languages with morphologically or lexically conditioned word stress patterns, distributed according to predictability of those patterns and syllable structure complexity.}
\end{table}

  Examining morphologically or lexically conditioned stress systems in more detail in \tabref{tab:5.5}, there may be some additional weak evidence for the hypothesis. Unpredictable morphologically or lexically conditioned systems are more common in languages from the Complex and Highly Complex categories. However, the small sample size makes it difficult to draw strong conclusions from these patterns.

  While we do not find strong support for the hypothesis in \REF{ex:5.4}, there are suggestions of associations between unpredictable word stress and highly complex syllable structure. This point will be revisited in \sectref{sec:5.4.6}, after other phonetic and phonological properties of stress in the sample have been examined.

\subsection{Phonetic processes conditioned by word stress}\label{sec:5.4.3}

  In this section, processes conditioned by word stress in the language sample are analyzed as a first step in testing the hypothesis that as syllable structure complexity increases, word stress has stronger phonetic and phonological effects in languages.

  In this analysis, I include only the 79 languages in the sample which are reported to have word stress. In \figref{fig:5.2}, I show the percentage of languages in each category which have unstressed vowel reduction, processes affecting consonants in unstressed syllables, and processes affecting consonants in stressed syllables.

\begin{figure}
\caption{\label{fig:5.2} Percentage of languages with word stress in each category of syllable structure complexity exhibiting stress-conditioned vowel reduction or consonant processes.}
\begin{tikzpicture}
\pgfplotstableread{data/fig52.csv}{\table}
    \pgfplotstablegetcolsof{\table}
    \pgfmathtruncatemacro\numberofcols{\pgfplotsretval-1}
            \begin{axis}[easterdayline]
            \foreach \i in {1,...,\numberofcols} {
                \addplot+ table [x index={1},y index={\i},x expr=\coordindex] {\table};
                \pgfplotstablegetcolumnnamebyindex{\i}\of{\table}\to{\colname} % Adding column headers to legend
                \addlegendentryexpanded{\colname}
            }
            \end{axis}                                                                           
\end{tikzpicture}
\end{figure}

  The patterns in \figref{fig:5.2} indicate that, as syllable structure complexity increases, languages are more likely to have phonetic vowel reduction processes as an effect of word stress. In particular, this pattern shows that languages with Simple syllable structure are much less likely than languages from the other three categories to have unstressed vowel reduction. When the pattern in the Simple category is cross-tabulated against those for the other three categories combined, the result is statistically significant: χ\textsuperscript{2}(1, \textit{N} = 79) = 5.298, \textit{p} = 0.02. By comparison, the trend in processes affecting consonants in unstressed syllables shows an erratic, but overall decreasing trend with respect to syllable structure complexity. Processes affecting consonants in stressed syllables are generally rare in the sample and show a level trend. 

  Thus we find that while the trend in unstressed vowel reduction processes follows the pattern predicted by the hypothesis, the trends in the consonant-affecting processes do not. These results prompt a more detailed analysis of both vowel reduction processes and consonant processes conditioned by stress in the sample.

\subsubsection{{Unstressed} {vowel} {reduction}}\label{sec:5.4.3.1}

  General vowel reduction patterns in the sample will be examined in greater detail in \chapref{sec:6}, so I present here just a brief analysis of unstressed vowel reduction in the 49 languages of the sample for which it is reported. Here only the outcomes of these processes are analyzed. Specifically, I consider outcomes involving reduction in vowel duration, reduction in vowel quality, vowel devoicing, and vowel deletion. A language may have several unstressed vowel reduction processes yielding different outcomes; each such process and outcome has been included in the analysis here. The results are shown in \figref{fig:5.3}. For each category of syllable structure complexity, the percentage of languages with unstressed vowel reduction resulting in the given outcome is shown.

\begin{figure}
\begin{tikzpicture}
\pgfplotstableread{data/fig53.csv}{\table}
    \pgfplotstablegetcolsof{\table}
    \pgfmathtruncatemacro\numberofcols{\pgfplotsretval-1}
            \begin{axis}[easterdayline]
            \foreach \i in {1,...,\numberofcols} {
                \addplot+ table [x index={1},y index={\i},x expr=\coordindex] {\table};
                \pgfplotstablegetcolumnnamebyindex{\i}\of{\table}\to{\colname} % Adding column headers to legend
                \addlegendentryexpanded{\colname}
            }
            \end{axis}                                                                           
\end{tikzpicture}
\caption{\label{fig:5.3} Percentage of languages with unstressed vowel reduction having the given outcome of vowel reduction in each category of syllable structure complexity.}
\end{figure}

  There are three interesting patterns in \figref{fig:5.3}. The first is that languages in the  Moderately Complex, Complex, and Highly Complex categories are much more likely than those in the Simple category to have reduction in duration and quality as outcomes of unstressed vowel reduction. Second, devoicing and deletion outcomes show broadly level trends with respect to syllable structure complexity. Finally, the Simple category is additionally set apart from the others in that all four outcomes of unstressed vowel reduction are roughly equally represented in those languages. That is, languages in the Simple category (which number only seven in this analysis), unlike those in other categories, do not show any strong tendencies in the outcomes of unstressed vowel reduction.

\subsubsection{{Processes} {affecting} {consonants} {in} {unstressed} {syllables}}\label{sec:5.4.3.2}

  We now turn to an examination of processes affecting consonants in unstressed syllables. In the current sample, these processes occur in 22 languages and form a heterogeneous group, with most of the process types occurring in just one or two languages. In \tabref{tab:5.6}, I list the more frequent processes separately and group together the minor trends under the label of ``Other.''\footnote{{Flapping is not a frequent process, but because it is often described as a prototypical process for consonants in unstressed syllables, it is given a separate row in the table.}} In examining the table, note that a language may have more than one process affecting consonants in unstressed syllables; therefore the numbers going down the columns may add up to more than the totals in the column headings.

\begin{table}
\begin{tabularx}{\textwidth}{Qccccr}
\lsptoprule
& \multicolumn{4}{c}{Syllable structure complexity}\\\cmidrule(lr){2-5}
& S & MC & C & HC & Total\\
   Processes affecting consonants in unstressed syllables & \textit{N} = 7 & \textit{N} = 4 & \textit{N} = 8 & \textit{N} = 3\\\midrule
 Deletion & 1 & -- & 3 & 1 & \textit{5}\\
 Voicing & 1 & 1 & 1 & 1 & \textit{4}\\
 Spirantization & 2 & 1 & -- & -- & \textit{3}\\
 Flapping & -- & 1 & 1 & -- & \textit{2}\\
 Other & 7 & 5 & 4 & 2 & \textit{18}\\
\lspbottomrule
\end{tabularx}
\caption{\label{tab:5.6}Processes affecting consonants in unstressed syllables in sample, by syllable structure complexity.}
\end{table}

  Because the data set consists of only 22 languages and the total number of languages with each kind of process is so small, it is difficult to draw conclusions from the patterns in \tabref{tab:5.6}. However, one interesting pattern is that languages with simpler syllable structure seem to be associated with not only a higher number but also a higher diversity of processes affecting consonants in unstressed syllables. Seven languages in the Simple category have such processes. In this group, the more common processes of deletion, voicing, and spirantization occur. However, there are also seven different kinds of ‘Other’ processes represented in this group: devoicing, aspiration, lengthening, glottalization, debuccalization, secondary palatalization, and change in place of articulation.

\subsubsection{{Processes} {affecting} {consonants} {in} {stressed} {syllables}}\label{sec:5.4.3.3}

  Processes affecting consonants in stressed syllables are less common than those affecting consonants in unstressed syllables.\footnote{{Ian Maddieson (p.c.) points out that this lower number could be an artifact of analysis, in which consonant realization in stressed syllables may be more likely to be taken as the basic form of the phoneme.}} In the current sample, 15 languages were reported to have such patterns. The processes examined here form quite coherent groups: all but six of the processes can be classified as glide strengthening, lengthening, aspiration, or affrication. See \tabref{tab:5.7} for their distribution in the sample. Note again that a language may have more than one process affecting consonants in stressed syllables; therefore the numbers going down the columns may add up to more than the totals in the column headings.

\begin{table}
\begin{tabularx}{\textwidth}{Qccccr}
\lsptoprule
& \multicolumn{4}{c}{Syllable structure complexity}\\\cmidrule(lr){2-5}
& S & MC & C & HC & Total\\
   Processes affecting consonants in stressed syllables & \textit{N} = 4 & \textit{N} = 4 & \textit{N} = 4 & \textit{N} = 3\\\midrule
 Glide strengthening & 3 & 2 & 1 & -- & \textit{6}\\
 Lengthening & 4 & 1 & -- & -- & \textit{5}\\
 Aspiration & 2 & 1 & 1 & -- & \textit{4}\\
 Affrication & 1 & 1 & -- & -- & \textit{2}\\
 Other & -- & -- & 3 & 3 & \textit{6}\\
\lspbottomrule
\end{tabularx}
\caption{\label{tab:5.7}Processes affecting consonants in stressed syllables in sample, by syllable structure complexity.}
\end{table}

  As in the previous analysis, it is difficult to draw strong conclusions from such a small data set. Here I note some apparent patterns. First, the Simple category is associated with generally higher rates of the most common processes affecting consonants in stressed syllables (glide strengthening, aspiration, lengthening, and affrication). Some of the languages in the Simple category have more than one such process: for example, \ili{Pinotepa Mixtec} is reported to have both aspiration and glide strengthening in stressed syllables. Second, the minor (``Other'') trends in the current analysis are found in languages with more complex syllable structure: these are as varied as palatalization (\ili{Nivkh}, Complex), voicing and implosion (\ili{Mamaindê}, Complex), ``more fortis'' articulation (\ili{Kunjen}, Highly Complex), labialization (\ili{Thompson}, Highly Complex), and devoicing (\ili{Tohono O’odham}, Highly Complex). These results are in contrast to the results for unstressed processes affecting consonants, in which a greater variety of processes was observed in languages in the Simple and Moderately Complex categories.

\subsubsection{{Implicational} {relationships} {between} {phonetic} {processes} {conditioned} {by} {stress}}\label{sec:5.4.3.4}

  Because the number of languages with unstressed vowel reduction is higher in the sample than the number of languages with processes affecting consonants in unstressed syllables, which in turn is higher than the number of languages with processes affecting consonants in stressed syllables, we might expect to find implicational relationships among some of these processes. That is, it might be the case that the presence of one kind of stress-conditioned phonetic process in a language implies the presence of another kind of process. Any such implications might shed light on the diachronic development of segmental effects of word stress.

  In \tabref{tab:5.8} I present the distribution of languages with word stress in the sample according to the presence or absence of unstressed vowel reduction and stress-conditioned processes affecting consonants. Here, processes affecting consonants in stressed syllables and those affecting consonants in unstressed syllables have been collapsed.

\begin{table}
\begin{tabularx}{.75\textwidth}{lXSSX}
\lsptoprule
 {Unstressed V reduction} & \multicolumn{4}{c}{Stress-conditioned C processes}\\\cmidrule(lr){2-5}
  & & {Present} & {Absent} &\\\midrule
 {Present} & ~ & 19 & 30 & ~\\
 {Absent} & & 9 & 21 &\\
\lspbottomrule
\end{tabularx}
\caption{\label{tab:5.8}Languages with word stress, distributed according to presence or absence of unstressed vowel reduction and stress-conditioned processes affecting consonants.}
\end{table}

  The trend in \tabref{tab:5.8} is not significant in a chi-square test. Therefore, while most of the languages with stress-conditioned consonant allophony have unstressed vowel reduction, the pattern is not a strong one. This finding is inconsistent with that of \citet{BybeeEtAl1998}, who established an implicational universal by which the presence of consonant changes conditioned by stress in a language implies the presence of vowel reduction in unstressed syllables. It is important to keep in mind that the aforementioned study examined a specific subset of specific consonant changes, and had a very different language sample composition from that of the present survey, in which languages with Simple and Highly Complex syllable patterns are overrepresented. However, even excluding the patterns from languages in those categories, the implication between stress-conditioned consonant allophony and unstressed vowel reduction is not significant in Fisher’s exact test.

   In \tabref{tab:5.9}, the languages with word stress in the sample are distributed according to the presence or absence of processes affecting consonants in unstressed and stressed environments, respectively.

\begin{table}
\begin{tabularx}{.75\textwidth}{lXSSX}
\lsptoprule
 Stressed C processes & \multicolumn{4}{c}{Unstressed C processes}\\\cmidrule(lr){2-5}
             &  & {Present} & {Absent} &\\\midrule
 {Present}   & ~& 9 & 6   & ~\\
 {Absent}    &  & 13 & 51 & ~\\
\lspbottomrule
\end{tabularx}
\caption{\label{tab:5.9}Languages with word stress, distributed according to presence or absence of processes affecting consonants in unstressed and stressed environments.}
\end{table}

  The distribution in \tabref{tab:5.9} indicates that the presence of processes affecting consonants in stressed environments tends to imply the presence of processes affecting consonants in unstressed environments: this is true of 9/15 languages with consonant processes in stressed environments. Though this trend is not universal, it is significant (\textit{p} = .004 in Fisher’s exact test). Given the kinds of processes observed in the data sets in \sectref{sec:5.4.3.2} and \sectref{sec:5.4.3.3}, this trend could very well be reflective of the pattern by which processes of weakening are crosslinguistically more frequent than processes of strengthening (\citealt{BybeeEasterday2019,Bybee2015b,Lavoie2015}).

\subsection{Phonological properties of stressed and unstressed syllables}\label{sec:5.4.4}

  Another way to approach the hypothesis regarding the segmental effects of word stress is to examine asymmetries in the phonological properties of stressed and unstressed syllables. These patterns may reflect the phonologization of stress-conditioned phonetic processes, such as those described in the previous sections, and may indicate that word stress has a long history of segmental effects in a language.

  The phonological differences in stressed and unstressed syllables considered here are differences in vowel quality contrasts, vowel length contrasts, tonal contrasts, and consonant contrasts. The contrasts examined here are not necessarily categorical: authors may report exceptions in a few lexical items or describe the pattern as an overwhelming tendency. For example, the phoneme /ə/ in \ili{East Kewa} is described as occurring “most often in an unstressed position only” (\citealt{FranklinFranklin1978}: 19). Included in the definition of vowel quality contrasts here are regular phonologized processes of vowel reduction which have the effect of neutralizing vowel quality contrasts: for example, when some or all vowels are realized as /ə/ in unstressed syllables. Regular unstressed vowel reduction processes with such dramatic neutralizing effects on quality were quite rare in the sample, being reported for only two languages: \ili{Thompson} and \ili{Tohono O’odham}. Therefore, there is very little overlap between the reduced vowel quality contrasts examined here and the phonetic unstressed vowel reduction processes reported in \sectref{sec:5.4.3.1}.

  Relatively few languages within the sample were reported to have phonological differences between stressed and unstressed syllables, as I have defined them here: in total, only twelve languages had such patterns. I show their distribution with respect to syllable structure complexity in \tabref{tab:5.10}.

\begin{table}
\begin{tabularx}{\textwidth}{Qcccc}
\lsptoprule
& \multicolumn{4}{c}{Syllable structure complexity}\\\cmidrule(lr){2-5}
& S & MC & C & HC\\
   Languages with phonological differences between stressed and unstressed syllables & \textit{N} = 18 & \textit{N} = 18 & \textit{N} = 21 & \textit{N} = 21\\\midrule
 Present & 3 & 3 & 3 & 3\\
 Absent or not reported & 15 & 15 & 18 & 18\\
\lspbottomrule
\end{tabularx}
\caption{\label{tab:5.10}Languages with word stress exhibiting phonological differences between stressed and unstressed syllables.}
\end{table}

  The pattern in \tabref{tab:5.10} does not support the hypothesis that languages with more complex syllable structure are more likely to show segmental effects of word stress. If anything, the Simple and Moderately Complex categories have a slightly higher rate than the other categories for this pattern.

  As mentioned in \sectref{sec:5.2.1}, languages may show varying degrees of phonological differences between stressed and unstressed syllables. In \tabref{tab:5.11} I list the languages reported to have phonological differences between stressed and unstressed syllables according to the number of phonological properties (vowel quality contrasts, consonant contrasts, etc.) for which that difference occurs.

\begin{table}
\begin{tabularx}{\textwidth}{QCCCC}
\lsptoprule
& \multicolumn{4}{c}{Syllable structure complexity}\\\cmidrule(lr){2-5}
& S & MC & C & HC\\
   Phonological differences between stressed and unstressed syllables & \textit{N} = 3 & \textit{N} = 3 & \textit{N} = 3 & \textit{N} = 3\\\midrule

 1 property & { \textit{Cavineña}}

{\textit{East Kewa}}

 \textit{Murui Huitoto} & \textit{Eastern Khanty} & { \textit{Bardi}}

 \textit{Burushaski} & \textit{Thompson}\\
 2 properties & \textit{--} & \textit{--} & \textit{--} & { \textit{Semai}}

 \textit{Tohono O’odham}\\
 3 or more properties & \textit{--} & { \textit{Lao}}

 \textit{Pacoh} & \textit{Koho} & \textit{--}\\
\lspbottomrule
\end{tabularx}
\caption{\label{tab:5.11}Number of phonological differences between stressed and unstressed syllables in the sample, by syllable structure complexity.}
\end{table}

  While all three of the languages in the Simple category in \tabref{tab:5.11} have just one property each, there otherwise does not appear to be a trend, within this very small data set, by which the number of phonological differences between stressed and unstressed syllables increases incrementally with syllable structure complexity. The languages with the most phonological differences between stressed and unstressed syllables are from the Moderately Complex and Complex categories. All three of these -- \ili{Lao}, \ili{Pacoh}, and \ili{Koho} -- are spoken in the Southeast Asia \& Oceania macro-area and are described as having sesquisyllabic word patterns.\footnote{{\ili{Sumi Naga} (Simple category), which is not reported to have word stress and is therefore excluded from these analyses, has some sesquisyllabic root and word patterns in which the minor syllable is limited in its vowel quality, consonant, and tone distinctions (\citealt{Teo2009}: 61-64; \citeyear{Teo2012}: 371-372).}} 

  The specific phonological differences between stressed and unstressed syllables observed in these languages can be found in \tabref{tab:5.12}. Note that because languages may have more than one such difference, the numbers going down the columns may add up to more than the totals in the column headings.

\begin{table}
\begin{tabularx}{\textwidth}{Qcccc}
\lsptoprule
& \multicolumn{4}{c}{Syllable structure complexity}\\\cmidrule(lr){2-5}
& S & MC & C & HC\\
   Phonological differences between stressed and unstressed syllables & \textit{N} = 3 & \textit{N} = 3 & \textit{N} = 3 & \textit{N} = 3\\\midrule
 Vowel quality contrasts & 1 & 3 & 1 & 3\\
 Vowel length contrasts & 1 & 2 & 3 & 2\\
 Tonal contrasts & 1 & 1 & 1 & --\\
 Consonant contrasts & -- & 2 & 1 & --\\
\lspbottomrule
\end{tabularx}
\caption{\label{tab:5.12}Phonological differences between stressed and unstressed syllables in the sample, by syllable structure complexity.}
\end{table}

  Again, it is difficult to draw conclusions about patterns from such a small data set. The data suggests that differences in vowel length contrasts between stressed and unstressed syllables are less common in languages with Simple syllable structure, but recall from the analysis in \sectref{sec:4.3.2} that vowel length contrasts are rarer in this group of languages in general.

\subsection{Phonetic correlates of stress}\label{sec:5.4.5}

  In this section, I analyze the phonetic correlates of stress reported for languages of the sample with word stress. Specifically, I test the hypotheses formulated in \REF{ex:5.6} and \REF{ex:5.7} which relate phonetic correlates of word stress to syllable structure complexity.

  The phonetic correlates of stress examined here are vowel duration, pitch, and intensity. Altogether, phonetic correlates could be determined for 60 languages, roughly three-fourths of the languages reported to have word stress. In \tabref{tab:5.13} I show the number of languages from each syllable structure complexity category which are reported to have each correlate of word stress. Note that languages may have more than one phonetic correlate of stress, so the numbers going down the columns are not expected to add up to the totals in the column headers.

\begin{table}
\begin{tabularx}{\textwidth}{Qcccc}
\lsptoprule
& \multicolumn{4}{c}{Syllable structure complexity}\\\cmidrule(lr){2-5}
& S & MC & C & HC\\
   Phonetic correlates of word stress & \textit{N} = 16 & \textit{N} = 12 & \textit{N} = 16 & \textit{N} = 16\\\midrule
 Vowel Duration & 9 & 6 & 7 & 12\\
 Pitch & 13 & 9 & 8 & 8\\
 Intensity & 8 & 9 & 12 & 11\\
\lspbottomrule
\end{tabularx}
\caption{\label{tab:5.13}Reported correlates (impressionistic or instrumentally confirmed) of word stress in languages of the sample, by syllable structure complexity. 18 languages with word stress have been excluded here because phonetic correlates of stress are not described. One additional language (Ngarinyin) has also been omitted, but is reported to have \textit{decreased} duration as a correlate of stress for one vowel, /a/.}
\end{table}

  In order to better illustrate the trends in \tabref{tab:5.13}, in \figref{fig:5.4} I plot the percentage of languages in each syllable structure category reported to have each correlate of word stress.

\begin{figure}
\begin{tikzpicture}
\pgfplotstableread{data/fig54.csv}{\table}
    \pgfplotstablegetcolsof{\table}
    \pgfmathtruncatemacro\numberofcols{\pgfplotsretval-1}
            \begin{axis}[easterdayline]
            \foreach \i in {1,...,\numberofcols} {
                \addplot+ table [x index={1},y index={\i},x expr=\coordindex] {\table};
                \pgfplotstablegetcolumnnamebyindex{\i}\of{\table}\to{\colname} % Adding column headers to legend
                \addlegendentryexpanded{\colname}
            }
            \end{axis}                                                                           
\end{tikzpicture}
\caption{\label{fig:5.4} Percentage of languages which are reported to use given phonetic correlate of word stress, by syllable structure complexity.}
\end{figure}

  We find some weak support for the hypothesis that vowel duration as a phonetic correlate of stress is more common in languages with more complex syllable structure. In the Highly Complex category, three-fourths of the languages with word stress have this property. However, rather than being gradual, this trend sets the Highly Complex category apart from the other three, in which roughly half of the languages use vowel duration as a correlate of stress. The percentage of languages in which pitch is reported to signal word stress decreases with syllable structure complexity. Intensity as a phonetic correlate of word stress is reported much less often for languages in the Simple category than the others. Only the trend in pitch is statistically significant, and that is when the patterns in the Simple and Moderately Complex categories are combined and cross-tabulated against those for the other two categories combined (χ\textsuperscript{2}(1, \textit{N} = 60) = 4.434, \textit{p} = 0.04).

  These results are somewhat surprising in light of previous findings in this and the previous chapter. In \sectref{sec:5.3} it was found that tonal contrasts are more frequently found in languages of the Simple category than in the others. Since tonal contrasts are signaled by pitch and tone is most frequently found in the Simple category, we might expect pitch to be used as a phonetic correlate of stress less frequently in this category than the others. Similarly, in \sectref{sec:4.3.2}, it was found that vowel length contrasts are more common in languages outside the Simple category. Since vowel length contrasts are common in languages of the Highly Complex category, we might expect vowel duration to be used less frequently as a phonetic correlate of stress. The trends observed in \figref{fig:5.4} go against both of these predictions. An analysis of the phonetic correlates of stress within languages with tone and vowel length contrasts shows that such assumptions about contrasts and phonetic correlates of stress are not entirely justified. Of the 19 languages with tonal contrasts for which the phonetic correlates of stress are described, 11 use pitch to signal stress. Of the 24 languages with contrastive vowel length for which phonetic correlates of stress are described, 15 use vowel duration to signal stress. 

  The distributions in \tabref{tab:5.13} and \figref{fig:5.4} show how each individual correlate of word stress patterns with respect to syllable structure complexity. We now turn to a test of the hypothesis in \REF{ex:5.7} which predicts that increasing syllable structure complexity will be accompanied by an increased number of phonetic correlates of word stress. In other words, we expect that the proportion of languages with two or three of the correlates examined here will increase across the four syllable structure complexity categories. The observed distribution can be found in \figref{fig:5.5}.

\begin{figure}
\begin{tikzpicture}
\pgfplotstableread{data/fig55.csv}{\table}
    \pgfplotstablegetcolsof{\table}
    \pgfmathtruncatemacro\numberofcols{\pgfplotsretval-1}
            \begin{axis}[easterdaystacked,reverse stacked plots=false,
                                xticklabels={S,MC,C,HC},
                        ]
            \foreach \i in {1,...,\numberofcols} {
                \addplot+[
                    /pgf/number format/read comma as period, fill
                    ] table [x index={1},y index={\i},x expr=\coordindex] {\table};
                \pgfplotstablegetcolumnnamebyindex{\i}\of{\table}\to{\colname} % Adding column headers to legend
                \addlegendentryexpanded{\colname}
            }
            \end{axis}                                                                           
\end{tikzpicture}
\caption{\label{fig:5.5} Percentage of languages exhibiting given number of phonetic correlate of word stress in each syllable structure complexity category.}
\end{figure}

  While the rate of co-occurrence of all three phonetic correlates of stress is slightly higher in the Highly Complex portion of the sample than in the other categories, there are no obvious trends in \figref{fig:5.5}. Therefore we do not find strong support for the hypothesis.

  The patterns described above are for all reported phonetic correlates of stress, regardless of whether they are based on impressions or instrumental evidence. Instrumental evidence for phonetic correlates of stress was reported for only 17 languages in the sample, which are unevenly distributed among the syllable complexity categories. Within this very small data set, the trends in the pitch and intensity correlates are similar to those presented in \figref{fig:5.4}. However, vowel duration as an instrumentally-confirmed correlate shows a level trend across the syllable complexity categories, further limiting the extent to which there is concrete evidence for the hypothesis in \REF{ex:5.6}.

\subsection{Summary of word stress patterns}\label{sec:5.4.6}

  Four hypotheses were formulated in \sectref{sec:5.1.4} with respect to word stress and syllable structure complexity. The first was that the proportion of languages with unpredictable word stress placement would increase with syllable structure complexity. The results of the analysis in \sectref{sec:5.4.2} did not confirm this on a broad scale: morphologically or lexically conditioned word stress did not show a trend with respect to syllable structure complexity. However, a finer-grained analysis of patterns within morphologically or lexically conditioned stress systems indicated that the most unpredictable patterns within that group are more commonly found in languages of the Complex and Highly Complex categories. A second hypothesis predicted that word stress would have stronger segmental effects in languages as syllable structure complexity increased. The analyses in \sectref{sec:5.4.3}-4 provided mixed support for this hypothesis. Processes of unstressed vowel reduction, and outcomes of these processes resulting in reduction in quality and deletion, were much more common in languages with Moderately Complex, Complex, and Highly Complex syllable structure than those with Simple syllable structure. However, trends in stress-conditioned processes affecting consonants showed either decreasing or level rates with respect to syllable structure complexity. Likewise, an examination of phonological differences between stressed and unstressed syllables did not yield support for the hypothesis. Finally, a study of the phonetic correlates of word stress in \sectref{sec:5.4.5} tested the hypotheses that increasing syllable structure complexity would be accompanied by increasing use of vowel duration as a correlate of stress, as well as an increased number of phonetic correlates of stress. One of these hypotheses was weakly supported in the sample: the proportion of languages in which vowel duration signals stress is much higher in the Highly Complex category than in the others. However, no relationship was found between the number of phonetic correlates of stress and syllable structure complexity.

  The specific properties of word stress and tone found to have positive or negative trends with respect to syllable structure complexity are listed in \tabref{tab:5.14}. Italicized font indicates that the trend is based on a small data set (fewer than ten languages), and an asterisk indicates that the result was found to be statistically significant.

\begin{table}
\begin{tabularx}{\textwidth}{QQQ}
\lsptoprule
{Type of property}  & {Positive trends} (increases with syllable structure complexity)  & {Negative trends} (decreases with syllable structure complexity)\\\midrule
{Presence of stress and tone} & Word stress and no tone & Tone\\
{Stress placement} & \textit{Unpredictable morphological or lexical conditioning}\\
{Segmental effects of word stress} & *Unstressed vowel reduction

Reduction in vowel quality

Vowel deletion

\textit{Asymmetry in vowel length contrasts in} 

     \textit{stressed and unstressed syllables} & C processes in unstressed syllables

\textit{Diversity in unstressed C processes}\\
{Phonetic correlates of stress} & Vowel duration

Intensity & *Pitch\\
\lspbottomrule
\end{tabularx}
\caption{\label{tab:5.14}Properties of word stress associated positively or negatively with syllable structure complexity.}
\end{table}

  As discussed above, some of the patterns shown in \tabref{tab:5.14} do not have an incremental trend with respect to syllable structure complexity. For instance, the trends in tone, unstressed vowel reduction, and reduced quality or deletion outcomes of vowel reduction are not gradual, but set the Simple category apart from the other three categories, which all have similar patterns with respect to these properties. Similarly, a more frequent use of vowel duration as a correlate of stress sets the Highly Complex category apart from the others.

  For the most part, the results of the analyses in this chapter lend only weak support to the hypotheses. However, some of the unexpected patterns in the data, such as the opposing patterns of unstressed vowel reduction and stress-conditioned consonant allophony with respect to syllable structure complexity, indicate that stress may nevertheless play an important and complex role in the diachronic development of syllable structure. In light of the results here, it is necessary to rethink the hypotheses and the relationships between word stress, its effects, and syllable structure complexity. These issues will be explored in the following section.

\section{Discussion}\label{sec:5.5}
\subsection{Suprasegmental patterns and Highly Complex syllable structure}\label{sec:5.5.1}

  Having conducted analyses to test hypotheses relating suprasegmental properties to syllable structure complexity, we return to the main research questions of the book. While there was mixed support for the hypotheses tested here, the analyses revealed that there are suprasegmental patterns more strongly associated with the Highly Complex category than the other categories. In \REF{ex:5.27} I list the suprasegmental patterns which are most characteristic of languages of the Highly Complex category. I exclude minor patterns which were determined on the basis of data from ten or fewer languages.

\ea\label{ex:5.27}
  \textbf{Suprasegmental patterns associated with Highly Complex category}

\textit{Presence of stress and absence of tone}

\textit{Absence of stress-conditioned processes affecting consonants}

\textit{Presence of vowel duration as a phonetic correlate of stress}
\z

  As mentioned in the discussion of segmental patterns in \sectref{sec:4.5.1}, the terms ‘absence’ and ‘presence’ are not used here in a categorical sense. Instead these are meant to correspond to the relative absence or presence of a property in the Highly Complex group as compared to the other syllable structure complexity groups.

  In \sectref{sec:4.5.1} I showed how the segmental patterns associated with the Highly Complex group were distributed among the languages in that group. The resulting distribution showed that languages in which Highly Complex syllable patterns are more prominent also had more of those associated segmental patterns. In \tabref{tab:5.15} I illustrate how the suprasegmental patterns in \REF{ex:5.27} are distributed among the languages. Although it was found to be associated strongly with all of the non-Simple categories, I also include the presence of unstressed vowel reduction in the table to illustrate its distribution. The languages are again divided into three groups according to the prominence of their Highly Complex syllable patterns, as established in \sectref{sec:3.4.1}-2. The suprasegmental properties associated with Highly Complex syllable structure and listed in \REF{ex:5.27} above are given in the columns. A check mark indicates that a language has the expected property; a shaded cell indicates that it does not. 

\begin{table}
\begin{tabularx}{\textwidth}{lCCCCC}
\lsptoprule
 & Word stress \textbf{present} & {Tone} \textbf{absent} & Stress-conditioned C allophony \textbf{absent} & Unstressed V reduction \textbf{present} & {V duration  as correlate of stress} \textbf{present}\\\midrule
\multicolumn{6}{c}{{Languages with prevalent Highly Complex patterns}}\\\midrule
 \ili{Cocopa} & \ding{51} & \ding{51} & \cellcolor{lsLightGray} & \ding{51} & \ding{51}\\
 \ili{Georgian} & \ding{51} & \ding{51} & { \ding{51}} & \cellcolor{lsLightGray} & \cellcolor{lsLightGray}  \textit{(nr)}\\
 \ili{Itelmen} & \ding{51} & \ding{51} & { \ding{51}} & \cellcolor{lsLightGray} & \cellcolor{lsLightGray} \\
 \ili{Polish} & \ding{51} & \ding{51} & \ding{51} & \ding{51} & \ding{51}\\
 \ili{Tashlhiyt} & \cellcolor{lsLightGray} & \ding{51} & {} & \cellcolor{lsLightGray} & \cellcolor{lsLightGray} \\
 \ili{Thompson} & \ding{51} & \ding{51} & {} & \ding{51} & \ding{51}\\
 \ili{Tohono O’odham} & \ding{51} & \ding{51} & {} & \ding{51} & \ding{51}\\
 \ili{Yakima Sahaptin} & \ding{51} & \ding{51} & \ding{51} & \ding{51} & \cellcolor{lsLightGray} \\\midrule
 \multicolumn{6}{c}{{Languages with intermediate Highly Complex patterns}}\\\midrule
 \ili{Albanian} & \ding{51} & \ding{51} & \ding{51} & \ding{51} & \ding{51}\\
 \ili{Camsá} & \ding{51} & \ding{51} & \ding{51} & \ding{51} & \ding{51}\\
 \ili{Kabardian} & \ding{51} & \ding{51} & \ding{51} & \ding{51} & \ding{51}\\
 \ili{Lezgian} & \ding{51} & \ding{51} & \ding{51} & \ding{51} & \ding{51}\\
 \ili{Mohawk} & \ding{51} & \cellcolor{lsLightGray} & \ding{51} & \cellcolor{lsLightGray} & \cellcolor{lsLightGray} \\
 \ili{Nuu-chah-nulth} & \ding{51} & \ding{51} & \ding{51} & \ding{51} & \ding{51}\\
 \ili{Passamaquoddy-Maliseet} & \ding{51} & \cellcolor{lsLightGray} & \ding{51} & \ding{51} & \ding{51}\\
 \ili{Yine} & \ding{51} & \ding{51} & \ding{51} & \ding{51} & \cellcolor{lsLightGray} \\
 \ili{Qawasqar} & \cellcolor{lsLightGray}  (unclear) & \ding{51} & \cellcolor{lsLightGray} & \cellcolor{lsLightGray} & \cellcolor{lsLightGray} \\
 \ili{Semai} & \ding{51} & \ding{51} & \ding{51} & \cellcolor{lsLightGray} & \cellcolor{lsLightGray} \textit{(nr)}\\
 \ili{Tehuelche} & \ding{51} & \ding{51} & \ding{51} & \ding{51} & \cellcolor{lsLightGray} \textit{(nr)}\\\midrule
\multicolumn{6}{c}{{Languages with minor Highly Complex patterns}}\\\midrule
 \ili{Alamblak} & \ding{51} & \ding{51} & \ding{51} & \ding{51} & \textit{(nr)}\\
 \ili{Bench} & \cellcolor{lsLightGray} & \cellcolor{lsLightGray} & \cellcolor{lsLightGray} & \cellcolor{lsLightGray} & \cellcolor{lsLightGray} \\
 \ili{Doyayo} & \cellcolor{lsLightGray} & \cellcolor{lsLightGray} & \cellcolor{lsLightGray} & \cellcolor{lsLightGray} & \cellcolor{lsLightGray} \\
 \ili{Kunjen} & \ding{51} & \ding{51} & \cellcolor{lsLightGray} & \cellcolor{lsLightGray} & \ding{51}\\
 \ili{Menya} & \ding{51} & \cellcolor{lsLightGray} & \ding{51} & \cellcolor{lsLightGray} & \cellcolor{lsLightGray} \textit{(nr)}\\
 \ili{Wutung} & \ding{51} & \cellcolor{lsLightGray} & \cellcolor{lsLightGray} & \cellcolor{lsLightGray} & \ding{51}\\
\lspbottomrule
\end{tabularx}
\caption{\label{tab:5.15}Highly Complex languages, divided into three groups according to the prominence of their Highly Complex patterns. Expected suprasegmental properties are given in columns. A check mark indicates that the given language has the expected property; a shaded cell indicates it does not. Note that for Qawasqar it is unclear whether the language has word stress. \textit{(nr)} indicates that phonetic correlates of stress were not reported for the language.}
\end{table}

  The pattern in \tabref{tab:5.15} indicates that the predictions are largely upheld. Languages which have Highly Complex syllable structure as a prevalent or especially as an intermediate pattern tend to have more of the suprasegmental properties associated with Highly Complex syllable structure than languages which have Highly Complex syllable structure as a minor pattern.

  Despite the pattern in \tabref{tab:5.15}, the results in the current study suggest that highly complex syllable structure is not as reliably associated with suprasegmental features as it is with segmental features. With that in mind, we revisit the second research question of the book regarding the evolution of highly complex syllable structure. Some of the strongest patterns in the word stress data examined here serve to set apart the Simple category from the other three categories of syllable structure complexity: most notably, these include patterns in the presence of unstressed vowel reduction (and associated outcomes of vowel quality reduction and deletion). Therefore it is difficult to relate these results to the development of Highly Complex patterns specifically. This is further complicated by the fact that the hypotheses tested here were weakly supported, or not supported at all. This prompts us to reexamine some of the assumptions underlying these hypotheses.

  The hypotheses were rooted in findings from previous studies which related properties of stress to specific effects of stress independently of syllable structure complexity (\citealt{BybeeEtAl1998,Schiering2007}), though the relevance of these findings to syllable structure complexity seems clear enough. However, as noted above, the size and construction of the language samples used in the previous studies and the current one are quite different. It would be fruitful, then, to examine some of those associations established in the previous studies to see if they hold in the current sample, without reference to syllable structure complexity.

  Analyses of the relationships between predictability of stress placement, vowel duration as a correlate of stress, and the various segmental effects of stress -- associations noted by \citet{BybeeEtAl1998} -- in the current sample yield two statistically significant patterns. Languages with morphologically or lexically conditioned stress placement are more likely to have stress-conditioned allophonic variation in consonants (χ\textsuperscript{2}(1, \textit{N} = 79) = 5.282, \textit{p} = .02). Languages with morphologically or lexically conditioned stress placement are also likely to have vowel duration as a phonetic correlate of stress (χ\textsuperscript{2}(1, \textit{N} = 60) = 4.239, \textit{p} = .04)

  Other associations -- i.e. between stress predictability and unstressed vowel reduction, segmental effects of stress and vowel duration as a correlate of stress, and so on -- were not found to be statistically significant. This could be an effect of the composition of the current sample, in which the representation of syllable patterns which are relatively rare crosslinguistically is artificially high. For example, vowel reduction was found to be much rarer in languages of the Simple category, which represent 24/100 languages here. Another factor affecting the results could be differences in the information available from the sources consulted. Bybee and colleagues state that very few of the references consulted in that study made mention of the phonetic correlates of stress (\citeyear{BybeeEtAl1998}: 278). They had to rely instead upon descriptions of reported vowel lengthening processes in order to determine whether vowel duration was a correlate of stress. Many of the references consulted in the current study have been written in the last 20 years, during which time the reporting of phonetic correlates of stress has become standard procedure in language description.

  \citet{Schiering2007} found a strong positive association between the strength of stress in number of phonetic correlates and the extent to which stress has segmental effects in the languages of his sample. I conducted a similar analysis with the data here, calculating the correlation between the number of phonetic correlates reported to signal stress and the number of types of stress-conditioned segmental processes occurring, the three possibilities being unstressed vowel reduction, consonant allophony in unstressed syllables, and consonant allophony in stressed syllables. This analysis revealed a moderate but significant positive correlation between the phonetic strength of stress and the extent of the segmental effects of stress (\textit{r}(60) = .334, \textit{p} = .009).

  We find that a few strong associations between properties of stress and its segmental effects occur in the current study, replicating results from previous studies. It is puzzling then, given other associations established or proposed in the literature between these features and syllable structure complexity, that we did not find strong patterns linking properties of stress and syllable structure complexity in the current sample. This suggests that the role of stress in the development of highly complex syllable structure is more subtle than originally expected.

\subsection{Word stress and the development of syllable structure patterns}\label{sec:5.5.2}

  Though the patterns established here relating word stress properties to syllable structure complexity are unexpected and difficult to interpret, it is nevertheless important to attempt to relate them to the development of syllable structure patterns more generally.

  The pattern by which languages in the Simple category are less likely to have unstressed vowel reduction is expected from a diachronic point of view. If such patterns become prevalent enough in a language to become phonologized, this could eventually lead to a language developing more complex syllable patterns, at which point it would no longer belong to the Simple category. A related point is that languages with Simple syllable structure show highly disparate outcomes with respect to both unstressed vowel reduction and stress-conditioned consonant allophony. By comparison, languages with more complex syllable patterns consistently show two particularly strong outcomes with respect to unstressed vowel reduction: deletion and reduction in vowel quality. These trends could be interpreted as indicative of relative phonologization of stress-conditioned allophonic processes. That is, the more consistent outcomes of unstressed vowel reduction in languages with non-Simple syllable structure could point to a longer history of segmental effects of stress in those languages. We would expect this to be the case for languages in the Moderately Complex, Complex, and Highly Complex categories more so than languages in the Simple category. 

  The decrease in the number of languages having stress-conditioned consonant allophony with respect to syllable structure complexity was quite unexpected. This pattern is especially interesting in light of the findings in \chapref{sec:4}, in which certain consonant articulations were found to be associated with different ends of the syllable structure complexity cline. One diachronic interpretation of this pattern could be that consonant articulations associated with more complex syllable structure have their origin in stress-conditioned allophonic processes in languages with simpler syllable structure. While parallel processes of vowel reduction occur, making the syllable structure more complex, these consonantal processes may phonologize and eventually result in new contrastive phonemes. However, such a speculative scenario is difficult to examine in the current data set. None of the stress-conditioned processes examined here result in uvulars or ejectives. Processes resulting in the other articulations associated with high syllable complexity are relatively rare: out of more than 50 stress-conditioned consonant processes collected here, only five result in palato-alveolar and/or affricate articulations. These do all occur in languages from the Simple and Moderately Complex categories.

  A few possibilities come to mind for why languages in the Highly Complex category do not show the highest rates of stress-conditioned segmental processes. One is that there \textit{are} higher rates of vowel reduction in these languages, but the coarse-grained analyses in this chapter did not capture this fact. The analysis here considered only the presence or absence of unstressed vowel reduction patterns, but not the number of such patterns in each of the languages. This issue will be explored in further depth in \chapref{sec:6}, which presents a detailed analysis of all phonetic vowel reduction patterns in the language sample. Another possibility is that in languages of the Highly Complex category, segmental effects of stress have already operated in the languages for long periods of time and had dramatic effects on the phonology. In such a scenario, pre- and post-tonic vowels will have been largely reduced, leaving few vowels outside of stems to be affected by phonetic unstressed vowel reduction. Likewise, the absence or highly reduced nature of unstressed vowels would make consonant allophony in unstressed syllables unlikely. In other words, languages with Highly Complex syllable patterns may not show extreme segmental effects of stress because these processes have essentially progressed to completion within these languages. Such a scenario is, however, extremely speculative and not likely given the phonological facts of most of the languages in this group. While there are a few languages in which most unstressed vowels are highly reduced (e.g. \ili{Thompson}), there are many more in which this is not the case.

  A simpler, more plausible, and more satisfactory explanation for the patterns observed in this chapter is that word stress simply does not have the universally strong effect on syllable structure development that it was thought to have when the hypotheses of this study were formulated. Concurrent with that is the observation that there are many ways in which stress systems and syllable patterns may change independently of one another. Schiering notes the following issue in positing motivations for speech rhythm types:

\begin{quote}
“[P]roblems translating these observations to crosslinguistic data from a world-wide sample arise because at each step of the diachronic scenario for each phonological parameter of linguistic rhythm, multiple evolutionary scenarios may in principle be at work.” 
\citep[353]{Schiering2007}
\end{quote}

  Schiering gives several examples of how unpredictable word stress placement patterns may come about independently of vowel reduction. For example, in \ili{Turkish}, unpredictable stress patterns can be found in loanwords and in a grammaticalized construction in which the phrasal stress pattern has been reanalyzed as an irregular word stress pattern. In the current sample, there are similar patterns in which irregular stress patterns have been introduced by the recent grammaticalization of formerly independent words which retain their original stress patterns (e.g. in \ili{Imbabura Highland Quichua}, example 5.1). \citet{BybeeEtAl1998} also present several historically attested alternative paths by which stress placement patterns may change independently of vowel reduction. By the same token, processes of vowel reduction which have the effect of altering syllable patterns do not have to be conditioned by stress. For example, in \ili{Nkore-Kiga}, a language which does not have word stress, high vowels may be deleted in certain consonantal contexts word-medially \citep[202-205]{Taylor1985}. And consonant allophony, of course, may be conditioned by many other phonological factors besides stress.

  The findings here indicate that the properties and effects of word stress are just a few components of the “phonetics-phonology constellation” \citep[354]{Schiering2007} characterizing highly complex syllable structure. In fact, in comparing the results of this chapter to those of the previous chapter, an important finding might be that general properties of gestural organization in speech could be just as relevant as the effects of word stress in the development of highly complex syllable structure. This point will be reconsidered in the following chapters, which examine more generally the properties of vowel reduction and certain kinds of consonant allophony in the sample as they relate to syllable structure complexity.

