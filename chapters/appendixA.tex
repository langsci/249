\subsection{Appendix A: Language sample}

Information on the language sample used in the study is listed in Tables A1-A4.



\textbf{\textsc{Key to reading table:}}



\textbf{ISO 693-3:} ISO 693-3 code for language used in survey.



\textbf{Language:} Dialect is given in parentheses where relevant.



\textbf{Syllable Structure:} 



\textbf{\textit{S} }= Simple 



\textbf{\textit{MC} }= Moderately Complex



\textbf{\textit{C} }= Complex



\textbf{\textit{HC} }= Highly Complex



\textbf{Macro-area:} Following Dryer (1989: 268; 1992: 83, 133-5). 



\textbf{\textit{Africa =}} continent of Africa, including Semitic languages of southwest Asia. 



\textbf{\textit{Australia \& New Guinea =}} Australian continent and Melanesia, excluding Austronesian languages of Melanesia. 



\textbf{\textit{Eurasia =}} Eurasian landmass, excluding Semitic and languages from families of southeast Asia as defined below, and including the Munda languages of Austro-Asiatic. 



\textbf{\textit{North America =} }North American continent, including languages of Mexico, Mayan and Aztecan languages in Central America, and some branches of Chibchan-Paezan. 



\textbf{\textit{South America =}} South American continent, including languages of Central America except Mayan and Aztecan languages, and some Chibchan-Paezan branches. 



\textbf{\textit{Southeast Asia \& Oceania =} }Southeast Asian region, including all Sino-Tibetan, Tai-Kadai, Hmong-Mien, and Austro-Asiatic languages excluding Munda, and Oceania region (Austronesian languages).



\textbf{Top-level family} and \textbf{Subfamily:} Following genealogical classifications listed in Glottolog 3.3 \citep{HammarströmEtAl2018}.



\textbf{Speaker Population:} L1 speaker population figure for language (or specific dialect) given in Ethnologue 21 (\citealt{SimonsFennig2018}). An asterisk indicates that another source was used for population estimate; these can be found beneath the table.



\textbf{Date:} Date given in Ethnologue 21 (\citealt{SimonsFennig2018}) for speaker population figure.



\textbf{Vitality Status:} Following Ethnologue 21 (\citealt{SimonsFennig2018}). 



\textbf{\textit{Institutional =}} language has wide use in the home and community and official status at educational, provincial, national, and/or international levels. 



\textbf{\textit{Developing =}} language is used in the home, community, and sometimes broader contexts, and in initial stages of developing a system of writing and standardization. 



\textbf{\textit{Vigorous =}} language is used in the home and community by speakers of all generations, but has not yet developed a system of graphization or standardization. 



\textbf{\textit{In Trouble =}} language is currently in the process of losing intergenerational transmission, with the community shifting to other languages for daily use, but there are still speakers of child-bearing age. 



\textbf{\textit{Dying =}} language has lost intergenerational transmission entirely, and all fluent speakers are above child-bearing age.



\begin{tabularx}{\textwidth}{XXXXXXXXX}
\lsptoprule

 \textbf{ISO 639-3} & \textbf{Language} & \textbf{Syllable Structure} & \textbf{Macro-area} & \textbf{Top-level family} & \textbf{Subfamily} & \raggedleft \textbf{Speaker Population} & \textbf{Date} & { \textbf{Vitality}}

 \textbf{Status}\\
 hts & {\textbf{Hadza}} & S & Africa & {(isolate)} &  & \raggedleft 950 & 2013 & In Trouble\\
 grj & {\textbf{Southern Grebo}} & S & Africa & {Atlantic-Congo} & {\textit{Volta-Congo}} & \raggedleft 65,000 & 2012 & Vigorous\\
 yor & {\textbf{Yoruba}} & S & Africa & {Atlantic-Congo} & {\textit{Volta-Congo}} & \raggedleft 19,043,700 & 1993 & Institutional\\
 mhi & {\textbf{Ma’di}} & S & Africa & {Central Sudanic} & {\textit{Moru-Madi}} & \raggedleft 293,000 & 2014 & Developing\\
 bbo & {\textbf{Southern Bobo Madaré}} & S & Africa & {Mande} & {\textit{Western Mande}} & \raggedleft 181,000 & 2009 & Developing\\
 svs & {\textbf{Savosavo}} & S & Aus \& New Guinea & {(isolate)} &  & \raggedleft 2,420 & 1999 & Vigorous\\
 kbk & {\textbf{Grass Koiari}} & S & Aus \& New Guinea & {Koiarian} & {\textit{Koiaric}} & \raggedleft 1,700 & 2000 & Vigorous\\
 roo & {\textbf{Rotokas}} & S & Aus \& New Guinea & {North Bougainville} & {\textit{\ili{Rotokas}-Askopan}} & \raggedleft 4,320 & 1981 & Developing\\
 kjs & {\textbf{East Kewa}} & S & Aus \& New Guinea & {Nuclear Trans New Guinea} & {\textit{Enga-Kewa-Huli}} & \raggedleft 45,000 & 2000 & Developing\\
 tow & {\textbf{Towa}} & S & N America & {Kiowa-Tanoan} &  & \raggedleft 1,790 & 2007 & In Trouble\\
 mio & {\textbf{Pinotepa Mixtec}} & S & N America & {Otomanguean} & {\textit{Eastern Otomanguean}} & \raggedleft 20,000 & 1990 & Vigorous\\
 ute & {\textbf{Ute}} & S & N America & {Uto-Aztecan} & {\textit{Northern Uto-Aztecan}} & \raggedleft 920 & 2007 & In Trouble\\
 ura & {\textbf{Urarina}} & S & S America & {(isolate)} &  & \raggedleft 3,000 & 2002 & Developing\\
 wba & {\textbf{Warao}} & S & S America & {(isolate)} &  & \raggedleft 28,100 & 2007 & Vigorous\\
 apu & {\textbf{Apurinã}} & S & S America & {Arawakan} & {\textit{Southern Maipuran}} & \raggedleft 2,870 & 2006 & In Trouble\\
 huu & {\textbf{Murui Huitoto}} & S & S America & {Huitotoan} & {\textit{Nuclear Witotoan}} & \raggedleft 2,000 & 2016 & In Trouble\\
 cav & {\textbf{Cavineña}} & S & S America & {Pano-Tacanan} & {\textit{Tacanan}} & \raggedleft 600 & 2011 & In Trouble\\
 cub & {\textbf{Cubeo}} & S & S America & {Tucanoan} & {\textit{Eastern Tucanoan}} & \raggedleft 6,260 & 2008 & Institutional\\
 dru & {\textbf{\ili{Rukai} (Budai dialect)}} & S & SE Asia \& Oceania & {Austronesian} &  & \raggedleft 10,500 & 2002 & Developing\\
 mri & {\textbf{Maori}} & S & SE Asia \& Oceania & {Austronesian} & {\textit{Malayo-Polynesian}} & \raggedleft 158,640 & 2013 & In Trouble\\
 khc & {\textbf{\ili{Tukang Besi} North}} & S & SE Asia \& Oceania & {Austronesian} & {\textit{Malayo-Polynesian}} & \raggedleft 120,000 & 1995 & Vigorous\\
 sxr & {\textbf{Saaroa}} & S & SE Asia \& Oceania & {Austronesian} & {\textit{Tsouic}} & \raggedleft 10 & 2012 & Dying\\
 iii & {\textbf{Sichuan Yi}} & S & SE Asia \& Oceania & {Sino-Tibetan} & {\textit{Burmo-Qiangic}} & \raggedleft 2,000,000 & 2004 & Institutional\\
\lspbottomrule
\end{tabularx}
\textbf{Table A1.} Portion of language sample with Simple syllable structure.


 
\begin{tabularx}{\textwidth}{XXXXXXXXX}
\lsptoprule

 \textbf{ISO 639-3} & \textbf{Language} & \textbf{Syllable Structure} & \textbf{Macro-area} & \textbf{Top-level family} & \textbf{Subfamily} & \raggedleft \textbf{Speaker Population} & \textbf{Date} & { \textbf{Vitality}}

 \textbf{Status}\\
 ktb & {\textbf{Kambaata}} & MC & Africa & {Afro-Asiatic} & {\textit{Cushitic}} & \raggedleft 743,000 & 2007 & Institutional\\
 ewe & {\textbf{Ewe}} & MC & Africa & {Atlantic-Congo} & {\textit{Volta-Congo}} & \raggedleft 4,184,000 & 2013 & Institutional\\
 fvr & {\textbf{Fur}} & MC & Africa & {Furan} &  & \raggedleft 745,800 & 2004 & Developing\\
 knc & {\textbf{Kanuri}} & MC & Africa & {Saharan} & {\textit{Western Saharan}} & \raggedleft 3,290,500 & 1985 & Institutional\\
 ayz & {\textbf{Maybrat}} & MC & Aus \& New Guinea & {\ili{Maybrat}-Karon} &  & \raggedleft 20,000 & 1987 & Developing\\
 kms & {\textbf{Kamasau}} & MC & Aus \& New Guinea & {Nuclear Torricelli} & {\textit{Marienberg}} & \raggedleft 960 & 2003 & In Trouble\\
 spl & {\textbf{Selepet}} & MC & Aus \& New Guinea & {Nuclear Trans New Guinea} & {\textit{Finisterre-Huon}} & \raggedleft 7,000 & 1988 & Developing\\
 aly & {\textbf{Alyawarra}} & MC & Aus \& New Guinea & {Pama-Nyungan} & {\textit{Arandic-Thura-Yura}} & \raggedleft 1,660 & 2006 & Developing\\
 khr & {\textbf{Kharia}} & MC & Eurasia & {Austroasiatic} & {\textit{Mundaic}} & \raggedleft 241,580 & 2001 & Developing\\
 tel & {\textbf{Telugu}} & MC & Eurasia & {Dravidian} & {\textit{South Dravidian}} & \raggedleft 74,244,300 & 2001 & Institutional\\
 dry & {\textbf{Darai}} & MC & Eurasia & {Indo-European} & {\textit{Indo-Iranian}} & \raggedleft 11,700 & 2011 & In Trouble\\
 mjg & {\textbf{Tu}} & MC & Eurasia & {Mongolic} & {\textit{Southern Periphery Mongolic}} & \raggedleft 152,000 & 2000 & In Trouble\\
 kca & {\textbf{Eastern Khanty}} & MC & Eurasia & {Uralic} & {\textit{Khantyic}} & \raggedleft 2,000 & 2007 & In Trouble\\
 kyh & {\textbf{Karok}} & MC & N America & {(isolate)} &  & \raggedleft 12 & 2007 & Dying\\
 scs & {\textbf{\ili{North Slavey} (Hare dialect)}} & MC & N America & {Athabaskan-Eyak-Tlingit} & {\textit{Athabaskan-Eyak}} & \raggedleft 710 & 2007 & In Trouble\\
 kal & {\textbf{Kalaallisut}} & MC & N America & {Eskimo-Aleut} & {\textit{Eskimo}} & \raggedleft 44,000 & 2007 & Institutional\\
 cho & {\textbf{Choctaw}} & MC & N America & {Muskogean} & {\textit{Western Muskogean}} & \raggedleft 10,400 & 2010 & In Trouble\\
 car & {\textbf{Carib}} & MC & S America & {Cariban} & {\textit{Guianan}} & \raggedleft 7,358 & 2001 & In Trouble\\
 qvi & {\textbf{Imbabura Highland Quechua}} & MC & S America & {Quechuan} & {\textit{Quechua II}} & \raggedleft 150,000 & 2007 & Developing\\
 cod & {\textbf{Cocama-Cocamilla}} & MC & S America & {Tupian} & {\textit{Maweti-Guarani}} & \raggedleft 250 & 2007 & Dying\\
 pac & {\textbf{Pacoh}} & MC & SE Asia \& Oceania & {Austroasiatic} & {\textit{Katuic}} & \raggedleft 32,500 & 2002 & In Trouble\\
 pwn & {\textbf{Paiwan}} & MC & SE Asia \& Oceania & {Austronesian} &  & \raggedleft 66,100 & 2002 & Developing\\
 mji & {\textbf{\ili{Kim Mun} (Vietnam dialect)}} & MC & SE Asia \& Oceania & {Hmong-Mien} & {\textit{Mienic}} & \raggedleft 374,500 & 2000 & Vigorous\\
 aot & {\textbf{Atong}} & MC & SE Asia \& Oceania & {Sino-Tibetan} & {\textit{Brahmaputran}} & \raggedleft 10,000 & (no date) & In Trouble\\
 yue & {\textbf{Cantonese}} & MC & SE Asia \& Oceania & {Sino-Tibetan} & {\textit{Sinitic}} & \raggedleft 62,967,910 & 2013 & Institutional\\
 lao & {\textbf{Lao}} & MC & SE Asia \& Oceania & {Tai-Kadai} & {\textit{Kam-Tai}} & \raggedleft 3,253,700 & 2005 & Institutional\\
\lspbottomrule
\end{tabularx}
\textbf{Table A2.} Portion of language sample with Moderately Complex syllable structure.


 
\begin{tabularx}{\textwidth}{XXXXXXXXX}
\lsptoprule

 \textbf{ISO 639-3} & \textbf{Language} & \textbf{Syllable Structure} & \textbf{Macro-area} & \textbf{Top-level family} & \textbf{Subfamily} & \raggedleft \textbf{Speaker Population} & \textbf{Date} & { \textbf{Vitality}}

 \textbf{Status}\\
 mpi & {\textbf{\ili{Mpade} (Makari dialect)}} & C & Africa & {Afro-Asiatic} & {\textit{Chadic}} & \raggedleft 16,000 & 2004 & In Trouble\\
 dyo & {\textbf{Jola-Fonyi}} & C & Africa & {Atlantic-Congo} & {\textit{North-Central Atlantic}} & \raggedleft 397,100 & (no date) & Developing\\
 lun & {\textbf{Lunda}} & C & Africa & {Atlantic-Congo} & {\textit{Volta-Congo}} & \raggedleft 403,000 & 2010 & Institutional\\
 mdx & {\textbf{\ili{Dizin} (Central dialect)}} & C & Africa & {Dizoid} &  & \raggedleft 33,900 & 2010 & Institutional\\
 tbi & {\textbf{Gaam}} & C & Africa & {Eastern Jebel} &  & \raggedleft 67,200 & 2000 & Vigorous\\
 mpc & {\textbf{Mangarrayi}} & C & Aus \& New Guinea & {\ili{Mangarrayi}-Maran} &  & \raggedleft 12 & 2006 & Dying\\
 nir & {\textbf{Nimboran}} & C & Aus \& New Guinea & {Nimboranic} &  & \raggedleft 2,000 & 1987 & Dying\\
 opm & {\textbf{Oksapmin}} & C & Aus \& New Guinea & {Nuclear Trans New Guinea} & {\textit{Asman-Awyu-Ok}} & \raggedleft 8,000 & 1991 & Developing\\
 bcj & {\textbf{Bardi}} & C & Aus \& New Guinea & {Nyulnyulan} & {\textit{Western Nyulnyulan}} & \raggedleft 160 & 2006 & Dying\\
 ung & {\textbf{Ngarinyin}} & C & Aus \& New Guinea & {Worrorran} &  & \raggedleft 57 & 2006 & In Trouble\\
 bsk & {\textbf{Burushaski}} & C & Eurasia & {(isolate)} &  & \raggedleft 96,800 & 2004 & Vigorous\\
 eus & {\textbf{Basque}} & C & Eurasia & {(isolate)} &  & \raggedleft 545,800 & 2012 & Institutional\\
 niv & {\textbf{\ili{Nivkh} (West Sakhalin dialect)}} & C & Eurasia & {(isolate)} &  & \raggedleft 15* & 2014 & Dying\\
 bak & {\textbf{Bashkir}} & C & Eurasia & {Turkic} & {\textit{Common Turkic}} & \raggedleft 1,245,990 & 2010 & Institutional\\
 ket & {\textbf{Ket}} & C & Eurasia & {Yeniseian} & {\textit{Northern Yeniseian}} & \raggedleft 2010 & 2010 & Dying\\
 pay & {\textbf{Pech}} & C & N America & {Chibchan} &  & \raggedleft 990 & 1993 & Dying\\
 tzh & {\textbf{Tzeltal}} & C & N America & {Mayan} & {\textit{Core Mayan}} & \raggedleft 372,000 & 2000 & Developing\\
 lkt & {\textbf{Lakota}} & C & N America & {Siouan} & {\textit{Core Siouan}} & \raggedleft 2,200 & 1997 & In Trouble\\
 kbc & {\textbf{Kadiwéu}} & C & S America & {Guaicuruan} &  & \raggedleft 1,590 & 2006 & In Trouble\\
 wmd & {\textbf{Mamaindê}} & C & S America & {Nambiquaran} & {\textit{Nambikwara Complex}} & \raggedleft 330 & 2007 & In Trouble\\
 apn & {\textbf{Apinayé}} & C & S America & {Nuclear-Macro-Je} & {\textit{Je}} & \raggedleft 1,260 & 2003 & Developing\\
 cap & {\textbf{Chipaya}} & C & S America & {Uru-Chipaya} &  & \raggedleft 1,200 & 1995 & Developing\\
 kpm & {\textbf{Koho}} & C & SE Asia \& Oceania & {Austroasiatic} & {\textit{Bahnaric}} & \raggedleft 166,000 & 2009 & Developing\\
 lpa & {\textbf{Lelepa}} & C & SE Asia \& Oceania & {Austronesian} & {\textit{Malayo-Polynesian}} & \raggedleft 400 & 1989 & Vigorous\\
 lep & {\textbf{Lepcha}} & C & SE Asia \& Oceania & {Sino-Tibetan} & {\textit{Himalayish}} & \raggedleft 69,800 & 2001 & Vigorous\\
\lspbottomrule
\end{tabularx}
\textbf{Table A3.} Portion of language sample with Complex syllable structure.


 
\begin{tabularx}{\textwidth}{XXXXXXXXX}
\lsptoprule

 \textbf{ISO 639-3} & \textbf{Language} & \textbf{Syllable Structure} & \textbf{Macro-area} & \textbf{Top-level family} & \textbf{Subfamily} & \raggedleft \textbf{Speaker Population} & \textbf{Date} & { \textbf{Vitality}}

 \textbf{Status}\\
 shi & {\textbf{Tashlhiyt}} & HC & Africa & {Afro-Asiatic} & {\textit{Berber}} & \raggedleft 3,896,000 & 2004 & Developing\\
 dow & {\textbf{Doyayo}} & HC & Africa & {Atlantic-Congo} & {\textit{Volta-Congo}} & \raggedleft 18,000 & 1985 & Developing\\
 bcq & {\textbf{Bench}} & HC & Africa & {Ta-Ne-Omotic} &  & \raggedleft 348,000 & 2007 & Institutional\\
 mcr & {\textbf{Menya}} & HC & Aus \& New Guinea & {Angan} & {\textit{Nuclear Angan}} & \raggedleft 20,000 & 1998 & Developing\\
 kjn & {\textbf{Kunjen}} & HC & Aus \& New Guinea & {Pama-Nyungan} & {\textit{Paman}} & \raggedleft 20 & 1991 & Dying\\
 amp & {\textbf{Alamblak}} & HC & Aus \& New Guinea & {Sepik} & {\textit{Sepik Hill}} & \raggedleft 1,530 & 2000 & Developing\\
 wut & {\textbf{Wutung}} & HC & Aus \& New Guinea & {Sko} & {\textit{Nuclear Skou-Serra-Piore}} & \raggedleft 900 & 2003 & Vigorous\\
 kbd & {\textbf{Kabardian}} & HC & Eurasia & {\ili{Abkhaz}-Adyge} & {\textit{Circassian}} & \raggedleft 1,628,500 & 2010 & Developing\\
 itl & {\textbf{Itelmen}} & HC & Eurasia & {Chukotko-Kamchatkan} &  & \raggedleft 80 & 2010 & Dying\\
 als & {\textbf{\ili{Albanian} (Tosk dialect)}} & HC & Eurasia & {Indo-European} & {\textit{Albanian}} & \raggedleft 1,841,400 & 2012 & Institutional\\
 pol & {\textbf{Polish}} & HC & Eurasia & {Indo-European} & {\textit{Balto-Slavic}} & \raggedleft 40,248,740 & 2013 & Institutional\\
 kat & {\textbf{Georgian}} & HC & Eurasia & {Kartvelian} & {\textit{\ili{Georgian}-Zan}} & \raggedleft 4,347,320 & 1993 & Institutional\\
 lez & {\textbf{Lezgian}} & HC & Eurasia & {Nakh-Daghestanian} & {\textit{Daghestanian}} & \raggedleft 616,760 & 2010 & Institutional\\
 pqm & {\textbf{Passamaquoddy-Maliseet}} & HC & N America & {Algic} & {\textit{Algonquian}} & \raggedleft 590 & 2011 & In Trouble\\
 coc & {\textbf{Cocopa}} & HC & N America & {Cochimi-Yuman} & {\textit{Yuman}} & \raggedleft 350 & 1998 & In Trouble\\
 moh & {\textbf{Mohawk}} & HC & N America & {Iroquoian} & {\textit{Northern Iroquoian}} & \raggedleft 3,540 & 1999 & In Trouble\\
 yak & {\textbf{Yakama Sahaptin}} & HC & N America & {Sahaptian} & {\textit{Sahaptin}} & \raggedleft 5** & 2006 & Dying\\
 thp & {\textbf{Thompson}} & HC & N America & {Salishan} & {\textit{Interior Salish}} & \raggedleft 130 & 2014 & In Trouble\\
 ood & {\textbf{Tohono O’odham}} & HC & N America & {Uto-Aztecan} & {\textit{Southern Uto-Aztecan}} & \raggedleft 14,094 & 2007 & In Toruble\\
 nuk & {\textbf{Nuu-chah-nulth}} & HC & N America & {Wakashan} & {\textit{Southern Wakashan}} & \raggedleft 130 & 2014 & Dying\\
 kbh & {\textbf{Camsá}} & HC & S America & {(isolate)} &  & \raggedleft 4,000 & 2008 & Developing\\
 pib & {\textbf{Yine}} & HC & S America & {Arawakan} & {\textit{Southern Maipuran}} & \raggedleft 4,000 & 2000 & Developing\\
 teh & {\textbf{Tehuelche}} & HC & S America & {Chonan} & {\textit{Continental Chonan}} & \raggedleft 5*** & 2012 & Dying\\
 alc & {\textbf{Qawasqar}} & HC & S America & {Kawesqar} & {\textit{North Central Alacalufan}} & \raggedleft 12 & 2006 & Dying\\
 sea & {\textbf{Semai}} & HC & SE Asia \& Oceania & {Austroasiatic} & {\textit{Aslian}} & \raggedleft 10,000 & 2007 & Institutional\\
\lspbottomrule
\end{tabularx}
\textbf{Table A4.} Portion of language sample with Highly Complex syllable structure.



* Population figure from \citet{BotmaShiraishi2014}.



** Population figure from \citet{HargusBeavert2006}.



*** Population figure from \textit{aoNEK FILMS} \REF{ex:key:2012}, includes semi-speakers.
 
