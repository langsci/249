\chapter{Language sample}
Information on the language sample used in the study is listed in Tables~\ref{tab:A.1}--\ref{tab:A.4}.

\section*{Key to reading tables}
\begin{description}[leftmargin=*]
\item[(ISO) ISO 693-3:] ISO 693-3 code for language used in survey.
\item[Language:] Dialect is given in parentheses where relevant.
\item[(MA) Macro-area:] Following \textcites[268]{Dryer1989}[83, 133--135]{Dryer1992}.
    {\sloppy\begin{description}[labelindent=1cm,itemindent=*]
    \item[(A) Africa:] continent of Africa, including Semitic languages of southwest Asia. 
    \item[(ANG) Australia \& New Guinea:] Australian continent and Melanesia, excluding Austronesian languages of Melanesia. 
    \item[(EA) Eurasia:] Eurasian landmass, excluding Semitic and languages from families of southeast Asia as defined below, and including the Munda languages of Austro-Asiatic. 
    \item[(NA) North America:] North American continent, including languages of Mexico, Mayan and Aztecan languages in Central America, and some branches of Chibchan-Paezan. 
    \item[(SA) South America:] South American continent, including languages of Central America except Mayan and Aztecan languages, and some Chibchan-Paezan branches. 
    \item[(SEA) Southeast Asia \& Oceania:] Southeast Asian region, including all Sino-Tibetan, Tai-Kadai, Hmong-Mien, and Austro-Asiatic languages excluding Munda, and Oceania region (Austronesian languages).
    \end{description}}
\item[(Tlf) Top-level family and Subfamily:] Following genealogical classifications listed in Glottolog 3.3 \citep{HammarströmEtAl2018}.
\item[(Pop) Speaker Population:] L1 speaker population figure for language (or specific dialect) given in Ethnologue 21 (\citealt{SimonsFennig2018}). An asterisk indicates that another source was used for population estimate; these can be found beneath the table.
\item[Date:] Date given in Ethnologue 21 (\citealt{SimonsFennig2018}) for speaker population figure.
\item[(Vit) Vitality Status:] Following Ethnologue 21 (\citealt{SimonsFennig2018}). 
    \begin{description}[labelindent=1cm]
    \item[(I) Institutional:] language has wide use in the home and community and official status at educational, provincial, national, and/or international levels. 
    \item[(D) Developing:] language is used in the home, community, and sometimes broader contexts, and in initial stages of developing a system of writing and standardization. 
    \item[(V) Vigorous:] language is used in the home and community by speakers of all generations, but has not yet developed a system of graphization or standardization. 
    \item[(T) In trouble:] language is currently in the process of losing intergenerational transmission, with the community shifting to other languages for daily use, but there are still speakers of child-bearing age. 
    \item[(†) Dying:] language has lost intergenerational transmission entirely, and all fluent speakers are above child-bearing age.
    \end{description}
\end{description}

% {\tiny\begin{longtable}{llllllrlc}
% \lsptoprule {ISO} & {Language} & {S} & {MA} & {Top-level family} & {Subfamily} &  {Pop} & {Date} & {Vit}\\\midrule\endfirsthead%
% \midrule {ISO} & {Language} & {S} & {MA} & {Top-level family} & {Subfamily} &  {Pop} & {Date} & {Vit}\\\midrule\endhead%
% \endfoot\lspbottomrule\endlastfoot
\begin{table}\footnotesize
\caption{Portion of language sample with Simple syllable structure.\label{tab:A.1}}
\begin{tabularx}{\textwidth}{lQlQQrlc}
\lsptoprule {ISO} & {Language}  & {MA} & {Tlf} & {Subfamily} & \multicolumn{1}{c}{Pop} & {Date} & {Vit}\\\midrule
 hts & {{Hadza}} &  A & {(isolate)} &  &  950 & 2013 & T\\
 grj & {{Southern Grebo}} &  A & {Atlantic-Congo} & {\textit{Volta-Congo}} &  65,000 & 2012 & V\\
 yor & {{Yoruba}} &  A & {Atlantic-Congo} & {\textit{Volta-Congo}} &  19,043,700 & 1993 & I\\
 mhi & {{Ma’di}} &  A & {Central Sudanic} & {\textit{Moru-Madi}} &  293,000 & 2014 & D\\
 bbo & {{Southern Bobo Madaré}} &  A & {Mande} & {\textit{Western Mande}} &  181,000 & 2009 & D\\
 svs & {{Savosavo}} &  ANG & {(isolate)} &  &  2,420 & 1999 & V\\
 kbk & {{Grass Koiari}} &  ANG & {Koiarian} & {\textit{Koiaric}} &  1,700 & 2000 & V\\
 roo & {{Rotokas}} &  ANG & {North Bougainville} & {\textit{Rotokas-Askopan}} &  4,320 & 1981 & D\\
 kjs & {{East Kewa}} &  ANG & {Nuclear Trans New Guinea} & {\textit{Enga-Kewa-Huli}} &  45,000 & 2000 & D\\
 tow & {{Towa}} &  NA & {Kiowa-Tanoan} &  &  1,790 & 2007 & T\\
 mio & {{Pinotepa Mixtec}} &  NA & {Oto\-mang\-uean} & {\textit{Eastern Otomanguean}} &  20,000 & 1990 & V\\
 ute & {{Ute}} &  NA & {Uto-Aztecan} & {\textit{Northern Uto-Aztecan}} &  920 & 2007 & T\\
 ura & {{Urarina}} &  SA & {(isolate)} &  &  3,000 & 2002 & D\\
 wba & {{Warao}} &  SA & {(isolate)} &  &  28,100 & 2007 & V\\
 apu & {{Apurinã}} &  SA & {Arawakan} & {\textit{Southern Maipuran}} &  2,870 & 2006 & T\\
 huu & {{Murui Huitoto}} &  SA & {Huitotoan} & {\textit{Nuclear Witotoan}} &  2,000 & 2016 & T\\
 cav & {{Cavineña}} &  SA & {Pano-Tacanan} & {\textit{Tacanan}} &  600 & 2011 & T\\
 cub & {{Cubeo}} &  SA & {Tucanoan} & {\textit{Eastern Tucanoan}} &  6,260 & 2008 & I\\
 dru & {{Rukai (Budai dialect)}} &  SEA & {Austronesian} &  &  10,500 & 2002 & D\\
 mri & {{Maori}} &  SEA  & {Austronesian} & {\textit{Malayo-Polynesian}} &  158,640 & 2013 & T\\
 khc & {{Tukang Besi North}} &  SEA  & {Austronesian} & {\textit{Malayo-Polynesian}} &  120,000 & 1995 & V\\
 sxr & {{Saaroa}} &  SEA  & {Austronesian} & {\textit{Tsouic}} &  10 & 2012 & †\\
 iii & {{Sichuan Yi}} &  SEA & {Sino-Tibetan} & {\textit{Burmo-Qiangic}} &  2,000,000 & 2004 & I\\
 \lspbottomrule
% \end{longtable}}
\end{tabularx}
\end{table}


 
\begin{table}\footnotesize
\begin{tabularx}{\textwidth}{lQlQQrlc}
\lsptoprule {ISO} & {Language} & {MA} & {Tlf} & {Subfamily} & \multicolumn{1}{c}{Pop} & {Date} & {Vit}\\\midrule
 ktb & {{Kambaata}} &  A & {Afro-Asiatic} & {\textit{Cushitic}} &  743,000 & 2007 & I\\
 ewe & {{Ewe}} &  A & {Atlantic-Congo} & {\textit{Volta-Congo}} &  4,184,000 & 2013 & I\\
 fvr & {{Fur}} &  A & {Furan} &  &  745,800 & 2004 & D\\
 knc & {{Kanuri}} &  A & {Saharan} & {\textit{Western Saharan}} &  3,290,500 & 1985 & I\\
 ayz & {{Maybrat}} &  ANG & {Maybrat-Karon} &  &  20,000 & 1987 & D\\
 kms & {{Kamasau}} &  ANG & {Nuclear Torricelli} & {\textit{Marienberg}} &  960 & 2003 & T\\
 spl & {{Selepet}} &  ANG & {Nuclear Trans New Guinea} & {\textit{Finisterre-Huon}} &  7,000 & 1988 & D\\
 aly & {{Alyawarra}} &  ANG & {Pama-Nyungan} & {\textit{Arandic-Thura-Yura}} &  1,660 & 2006 & D\\
 khr & {{Kharia}} &  EA & {Austroasiatic} & {\textit{Mundaic}} &  241,580 & 2001 & D\\
 tel & {{Telugu}} &  EA & {Dravidian} & {\textit{South Dravidian}} &  74,244,300 & 2001 & I\\
 dry & {{Darai}} &  EA & {Indo-Europ.} & {\textit{Indo-Iranian}} &  11,700 & 2011 & T\\
 mjg & {{Tu}} &  EA & {Mongolic} & {\textit{Southern Periphery Mongolic}} &  152,000 & 2000 & T\\
 kca & {{Eastern Khanty}} &  EA & {Uralic} & {\textit{Khantyic}} &  2,000 & 2007 & T\\
 kyh & {{Karok}} &  NA & {(isolate)} &  &  12 & 2007 & †\\
 scs & {{North Slavey (Hare dialect)}} &  NA & {Athabaskan-Eyak-Tlingit} & {\textit{Athabaskan-Eyak}} &  710 & 2007 & T\\
 kal & {{Kalaallisut}} &  NA & {Eskimo-Aleut} & {\textit{Eskimo}} &  44,000 & 2007 & I\\
 cho & {{Choctaw}} &  NA & {Muskogean} & {\textit{Western Muskogean}} &  10,400 & 2010 & T\\
 car & {{Carib}} &  SA & {Cariban} & {\textit{Guianan}} &  7,358 & 2001 & T\\
 qvi & {{Imbabura Highland Quechua}} &  SA & {Quechuan} & {\textit{Quechua II}} &  150,000 & 2007 & D\\
 cod & {{Cocama-Cocamilla}} &  SA & {Tupian} & {\textit{Maweti-Guarani}} &  250 & 2007 & †\\
 pac & {{Pacoh}} &  SEA  & {Austroasiatic} & {\textit{Katuic}} &  32,500 & 2002 & T\\
 pwn & {{Paiwan}} &  SEA  & {Austronesian} &  &  66,100 & 2002 & D\\
 mji & {{Kim Mun\footnote{(Vietnam dialect)}}} &  SEA  & {Hmong-Mien} & {\textit{Mienic}} &  374,500 & 2000 & V\\
 aot & {{Atong}} &  SEA  & {Sino-Tibetan} & {\textit{Brahmaputran}} &  10,000 & n.d. & T\\
 yue & {{Cantonese}} &  SEA  & {Sino-Tibetan} & {\textit{Sinitic}} &  62,967,910 & 2013 & I\\
 lao & {{Lao}} &  SEA  & {Tai-Kadai} & {\textit{Kam-Tai}} &  3,253,700 & 2005 & I\\
\lspbottomrule
\end{tabularx}
\caption{Portion of language sample with Moderately Complex syllable structure.\label{tab:A.2}}
\end{table}


 
\begin{table}\footnotesize
\begin{tabularx}{\textwidth}{lQlQQrlc}
\lsptoprule {ISO} & {Language} & {MA} & {Tlf} & {Subfamily} & \multicolumn{1}{c}{Pop} & {Date} & {Vit}\\\midrule
 mpi & {{Mpade (Makari dialect)}} &  A & {Afro-Asiatic} & {\textit{Chadic}} &  16,000 & 2004 & T\\
 dyo & {{Jola-Fonyi}} &  A & {Atlantic-Congo} & {\textit{North-Central Atlantic}} &  397,100 & n.d. & D\\
 lun & {{Lunda}} &  A & {Atlantic-Congo} & {\textit{Volta-Congo}} &  403,000 & 2010 & I\\
 mdx & {{Dizin (Central dialect)}} &  A & {Dizoid} &  &  33,900 & 2010 & I\\
 tbi & {{Gaam}} &  A & {Eastern Jebel} &  &  67,200 & 2000 & V\\
 mpc & {{Mangarrayi}} &  ANG & {Mangarrayi-Maran} &  &  12 & 2006 & †\\
 nir & {{Nimboran}} &  ANG & {Nimboranic} &  &  2,000 & 1987 & †\\
 opm & {{Oksapmin}} &  ANG & {Nuclear Trans New Guinea} & {\textit{Asman-Awyu-Ok}} &  8,000 & 1991 & D\\
 bcj & {{Bardi}} &  ANG & {Nyulnyulan} & {\textit{Western Nyulnyulan}} &  160 & 2006 & †\\
 ung & {{Ngarinyin}} &  ANG & {Worrorran} &  &  57 & 2006 & T\\
 bsk & {{Burushaski}} &  EA & {(isolate)} &  &  96,800 & 2004 & V\\
 eus & {{Basque}} &  EA & {(isolate)} &  &  545,800 & 2012 & I\\
 niv & {{Nivkh (West Sakhalin dialect)}} &  EA & {(isolate)} &  &  15\footnote{Population figure from \citet{BotmaShiraishi2014}.} & 2014 & †\\
 bak & {{Bashkir}} &  EA & {Turkic} & {\textit{Common Turkic}} &  1,245,990 & 2010 & I\\
 ket & {{Ket}} &  EA & {Yeniseian} & {\textit{Northern Yeniseian}} &  2010 & 2010 & †\\
 pay & {{Pech}} &  NA & {Chibchan} &  &  990 & 1993 & †\\
 tzh & {{Tzeltal}} &  NA & {Mayan} & {\textit{Core Mayan}} &  372,000 & 2000 & D\\
 lkt & {{Lakota}} &  NA & {Siouan} & {\textit{Core Siouan}} &  2,200 & 1997 & T\\
 kbc & {{Kadiwéu}} &  SA & {Guaicuruan} &  &  1,590 & 2006 & T\\
 wmd & {{Mamaindê}} &  SA & {Nambiquaran} & {\textit{Nambikwara Complex}} &  330 & 2007 & T\\
 apn & {{Apinayé}} &  SA & {Nuclear-Macro-Je} & {\textit{Je}} &  1,260 & 2003 & D\\
 cap & {{Chipaya}} &  SA & {Uru-Chipaya} &  &  1,200 & 1995 & D\\
 kpm & {{Koho}} &  SEA  & {Austroasiatic} & {\textit{Bahnaric}} &  166,000 & 2009 & D\\
 lpa & {{Lelepa}} &  SEA  & {Austronesian} & {\textit{Malayo-Polynesian}} &  400 & 1989 & V\\
 lep & {{Lepcha}} &  SEA  & {Sino-Tibetan} & {\textit{Himalayish}} &  69,800 & 2001 & V\\
\lspbottomrule
\end{tabularx}
\caption{Portion of language sample with Complex syllable structure.\label{tab:A.3}}
\end{table}


 
\begin{table}\footnotesize
\begin{tabularx}{\textwidth}{lQlQQrlc}
\lsptoprule {ISO} & {Language} & {MA} & {Tlf} & {Subfamily} & \multicolumn{1}{c}{Pop} & {Date} & {Vit}\\\midrule
 shi & {{Tashlhiyt}} &  A & {Afro-Asiatic} & {\textit{Berber}} &  3,896,000 & 2004 & D\\
 dow & {{Doyayo}} &  A & {Atlantic-Congo} & {\textit{Volta-Congo}} &  18,000 & 1985 & D\\
 bcq & {{Bench}} &  A & {Ta-Ne-Omotic} &  &  348,000 & 2007 & I\\
 mcr & {{Menya}} &  ANG & {Angan} & {\textit{Nuclear Angan}} &  20,000 & 1998 & D\\
 kjn & {{Kunjen}} &  ANG & {Pama-Nyungan} & {\textit{Paman}} &  20 & 1991 & †\\
 amp & {{Alamblak}} &  ANG & {Sepik} & {\textit{Sepik Hill}} &  1,530 & 2000 & D\\
 wut & {{Wutung}} &  ANG & {Sko} & {\textit{Nuclear Skou-Serra-Piore}} &  900 & 2003 & V\\
 kbd & {{Kabardian}} &  EA & {Abkhaz-Adyge} & {\textit{Circassian}} &  1,628,500 & 2010 & D\\
 itl & {{Itelmen}} &  EA & {Chukotko-Kamchatkan} &  &  80 & 2010 & †\\
 als & {{Albanian\footnote{(Tosk dialect)}}} &  EA & {Indo-Europ.} & {\textit{Albanian}} &  1,841,400 & 2012 & I\\
 pol & {{Polish}} &  EA & {Indo-Europ.} & {\textit{Balto-Slavic}} &  40,248,740 & 2013 & I\\
 kat & {{Georgian}} &  EA & {Kartvelian} & {\textit{Georgian-Zan}} &  4,347,320 & 1993 & I\\
 lez & {{Lezgian}} &  EA & {Nakh-Daghestanian} & {\textit{Daghestanian}} &  616,760 & 2010 & I\\
 pqm & {{P.-Maliseet}} &  NA & {Algic} & {\textit{Algonquian}} &  590 & 2011 & T\\
 coc & {{Cocopa}} &  NA & {Cochimi-Yuman} & {\textit{Yuman}} &  350 & 1998 & T\\
 moh & {{Mohawk}} &  NA & {Iroquoian} & {\textit{Northern Iroquoian}} &  3,540 & 1999 & T\\
 yak & {{Yakama Sahaptin}} &  NA & {Sahaptian} & {\textit{Sahaptin}} &  5\footnote{Population figure from \citet{HargusBeavert2006}.} & 2006 & †\\
 thp & {{Thompson}} &  NA & {Salishan} & {\textit{Interior Salish}} &  130 & 2014 & T\\
 ood & {{Tohono O’odham}} &  NA & {Uto-Aztecan} & {\textit{Southern Uto-Aztecan}} &  14,094 & 2007 & T\\
 nuk & {{Nuu-chah-nulth}} &  NA & {Wakashan} & {\textit{Southern Wakashan}} &  130 & 2014 & †\\
 kbh & {{Camsá}} &  SA & {(isolate)} &  &  4,000 & 2008 & D\\
 pib & {{Yine}} &  SA & {Arawakan} & {\textit{Southern Maipuran}} &  4,000 & 2000 & D\\
 teh & {{Tehuelche}} &  SA & {Chonan} & {\textit{Continental Chonan}} &  5\footnote{Population figure from \textit{aoNEK FILMS} \REF{ex:key:2012}, includes semi-speakers.} & 2012 & †\\
 alc & {{Qawasqar}} &  SA & {Kawesqar} & {\textit{North Central Alacalufan}} &  12 & 2006 & †\\
 sea & {{Semai}} &  SEA  & {Austroasiatic} & {\textit{Aslian}} &  10,000 & 2007 & I\\
\lspbottomrule
\end{tabularx}
\caption{Portion of language sample with Highly Complex syllable structure.\label{tab:A.4}}
\end{table}
