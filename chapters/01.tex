\chapter{Syllables and syllable structure}\label{sec:1}

  A syllable is typically thought of as a unit which speakers use to organize sequences of sounds in their languages. The division of the speech stream into syllables reflects the higher levels of organization which are used in the cognitive processes by which speech is planned and perceived. Syllables are a common unit of abstract linguistic analysis; however, this unit seems to be more concrete and accessible to speakers than other phonological units such as segments. A speaker’s intuition of what is a pronounceable sequence of sounds is strongly influenced by the syllable patterns of the language they speak. Most languages have relatively simple syllable patterns, in which the alternation between relatively closed (consonantal) and relatively open (vocalic) articulations is fairly regular: syllable patterns such as those in the \ili{English} words \textit{pillow}, \textit{cactus}, and \textit{tree} are crosslinguistically prevalent. Compare these patterns to the examples below \xxref{ex:1.1}{ex:1.5}:

\ea\label{ex:1.1}
\etriple{Yakima} \textbf{Sahaptin}{Sahaptian}{USA}

\textit{ksksa}\\
\glt ‘elephant ear (mushroom)’

(\citealt{HargusBeavert2006}: 29)
\z

\ea\label{ex:1.2}
\etriple{Georgian}{Kartvelian}{Georgia}

\textit{bɾt͡s’χ’ali}\\
\glt ‘claw’
\citep[204]{Butskhrikidze2002}
\z

\ea\label{ex:1.3}
\etriple{Tashlhiyt}{Afro-Asiatic}{Morocco}

\textit{tsːkʃftstː}

t-sː-kʃf-t=stː\\
\glt ‘you dried it (\textsc{f})’
\citep[332]{Ridouane2008}
\z

\ea\label{ex:1.4}
\etriple{Tehuelche}{Chonan}{Argentina}

\textit{kt͡ʃaʔʃpʃkn}

k-t͡ʃaʔʃp-ʃ-kn

\textsc{refl}-wash-\textsc{ps-realis}\\
\glt ‘it is being washed’

(\citealt{FernándezGarayHernández2006}: 13)
\z

\ea\label{ex:1.5}
\etriple{Itelmen}{Chukotko-Kamchatkan}{Russia}

\textit{kɬtxuniŋeʔn}

kɬ-txuni-ŋeʔn\\
\glt ‘very dark’

(\citealt{GeorgVolodin1999}: 55)
\z

To speakers of most languages, the long strings of consonants in these examples are not pronounceable without a great deal of practice, being so different from the relatively simpler patterns that are crosslinguistically prevalent. Yet such patterns are fluently acquired and maintained by native speakers of these languages, and may even be relatively frequent in those languages.

  Syllable patterns like those illustrated above are typologically rare, occurring in between 5-10\% of the world’s languages. These languages tend to be found in close geographical proximity to one another, with the Pacific Northwest, the Caucasus region, the Atlas Mountains region, Patagonia, and Northeast Asia being particular ‘hotspots’ for such patterns. The accelerating rates of indigenous language obsolescence in those regions mean that such patterns stand to become even rarer in the coming generations.

  The patterns exemplified above are also famous in the literature for the problems they present to standard models of syllabic and phonological representation. While much effort is made to attempt to fit these patterns into various theoretical frameworks, far less research explores the motivations behind their historical development and maintenance in languages.

  This book is a typological study exploring the properties of languages with patterns like those above, which I call \textit{highly complex syllable structure}. The studies herein examine a number of phonetic, phonological, and morphological features of these languages. The aims of this study are to establish whether highly complex syllable structure has other linguistic correlates which may suggest a diachronic path (or paths) by which such patterns are likely to evolve.

  The book is organized as follows: in the following sections, I discuss findings and accounts for crosslinguistic syllable patterns and their implications for highly complex syllable structure, discuss accounts for syllable complexity more generally, and introduce the research questions examined here. In \chapref{sec:2}, I discuss considerations in constructing the language sample and propose a definition for highly complex syllable patterns. In \chapref{sec:3}, I conduct analyses of syllable structure patterns in the sample. Analyses of segmental and suprasegmental patterns in the sample are presented in Chapters 4 and 5, respectively. \chapref{sec:6} includes analyses of vowel reduction patterns in the sample. In \chapref{sec:7}, I examine specific kinds of consonant allophony in the language sample. In \chapref{sec:8}, the results are summarized and their implications for the research questions are examined.

\section{Background}\label{sec:1.1}
\subsection{The syllable}\label{sec:1.1.1}

  The syllable is a natural unit of spoken language by which sounds are organized in speech. The hierarchical organization of speech sounds into syllables is said to be “a fundamental property of phonological structure in human language” (\citealt{GoldsteinEtAl2006}: 228), and this unit plays a well-established role in linguistic analysis and description. However, the syllable eludes precise definition: research has not yet established clear and consistent correlates for it at the phonetic, physiological or phonological levels (\citealt{BellHooper1978,Laver1994,Krakow1999}). Much like consonants and vowels, syllables are characterized by distributional, phonetic, and phonological features, of which no single criterion is sufficient for perfectly describing or predicting the trends observed. To take one example of such a criterion, in a review of research on the physiological organization of the syllable, \citet[23-34]{Krakow1999} states that years of research into this topic have yielded “one disappointment after another” and that from an articulatory point of view, the speech stream “simply cannot be divided into discrete, linearly-ordered units the size of the segment or the syllable.” What empirical research has managed to establish with respect to physiological definitions of the syllable is distinct intra- and inter-articulatory patterns for syllable-initial and syllable-final consonants, at least in careful speech. Patterns in the acoustics, phonology, and perception of syllable constituents play an important role in determining and differentiating syllables, but they do not constitute complete or exceptionless definitions of the syllable, either alone or in combination with one another.

  Nevertheless, the syllable enjoys a well-established role in phonology, proving to be a highly useful unit in linguistic analysis and description. For many languages, it has been demonstrated that stress placement, tone, reduplication, and other phonological and morphological phenomena operate on the domain of the syllable.\footnote{{It should also be noted that phonological syllable structure has been argued to be irrelevant or altogether absent in some languages (e.g. \citealt{Newman1947} for \ili{Nuxalk}; \citealt{Hyman2011,Hyman2015} for \ili{Gokana}; \citealt{Labrune2012} for \ili{Japanese}). In these cases it is argued that phonological phenomena can be satisfactorily described by reference to morae, sequences of segments, and word/phrase junctures.}} Similarly, the different boundary edges of a syllable are associated with special coarticulatory properties and may serve as environments for allophonic processes. While native speaker intuitions regarding the precise location of syllable boundaries are not always consistent, there is a wealth of evidence that the unit has psychological reality to speakers: e.g. in the existence of syllabary writing systems, word games and secret languages using syllables as target structures, poetry and lyrical song which exploit syllable counts and syllable constituent patterns in a systematic way, and consistent speaker intuitions regarding the number of syllables in a word (\citealt{BellHooper1978,Blevins1995,ValléeEtAl2009}).

  Additional evidence for the syllable as an organizational unit of language lies in the observation that those sequences of sounds analyzed as syllables pattern in remarkably similar ways across languages. In fact, strong crosslinguistic tendencies are observed for practically every dimension along which syllable structure can be analyzed. Some of these patterns will be summarized in the following section.

\subsection{Crosslinguistic patterns in syllable structure}\label{sec:1.1.2}

  Here I describe some of the crosslinguistic patterns of syllable structure that have been observed in the literature. In the following sections I use the descriptive terms \textit{onset}, \textit{nucleus}, and \textit{coda} to refer to constituent parts of the syllable: the nucleus consists of the auditory peak of the syllable, typically a vowel; the onset refers to the consonant or group of consonants preceding the nucleus; and the coda refers to the consonant or group of consonants following the nucleus. It is useful to make these distinctions because these constituents have been shown to behave independently of one another in many respects, both within languages and crosslinguistically. In the following sections, the terms are used in a more or less theoretically neutral sense, and often in reference to phonetic realizations, rather than abstract representations, of the syllable. In theoretical models, the phonological constituency of syllables may be posited to take other forms; some of these issues are discussed in \sectref{sec:1.1.3}.

\subsubsection{{CV} {as} {a} {universal} {syllable} {type}}\label{sec:1.1.2.1}

 One robust pattern in syllable structure typology is the crosslinguistic ubiquity of syllables of the shape CV: a single consonant followed by a vowel.\footnote{{In syllable structure analysis, the notations C and V are used for consonant and vowel segments, respectively.}} Though it has been claimed that CV syllables are found in all languages, for a few languages it has been posited that this structure does not occur (cf. \citealt{BreenPensalfini1999} for \ili{Arrernte}, \citealt{Sommer1969} for the Oykangand dialect of \ili{Kunjen}, both Australian languages). Such analyses are typically highly abstract and apply only to ``underlying'' syllable forms: for both \ili{Arrernte} and \ili{Kunjen} it has been shown that CV structures do occur in ``surface'' phonetic forms (\citealt{Anderson2000,Sommer1969,Sommer1981}).

  Due to its crosslinguistic prevalence, the CV structure has been called the universal syllable type and the least marked of all syllable structures \citep{Zec2007}. CV structures are set apart from other syllable types in numerous aspects of their behavior. If only one syllable type occurs in a language, that type will be of the form CV. Such languages are rare, but attested: they include \ili{Hawaiian} \citep{Maddieson2011} and \ili{Hua} \citep{Blevins1995}. CV structures are acquired even before V structures in babbling stages of vocal development and language acquisition (cf. \citealt{LeveltEtAl2000} for \ili{Dutch}). The CV structure overwhelmingly predominates in frequency distributions of syllable types within and across languages. In ULSID, a database containing the syllabified lexicons of 17 genealogically and geographically diverse languages, CV syllables account for roughly 54\% of the 250,000 syllables, despite the languages having a wide range of attested syllable patterns (\citealt{ValléeEtAl2009}).

  Due to the above patterns, CV is often interpreted as a universal primitive element of human language. There are challenges to this view: for example, \citet{BellHooper1978} argue that the characterization of the CV type as inherently ``unmarked'' is misleading and simplified, as this assumption can be derived from a collection of generalizations regarding other phonological patterns. They argue that the universal status of CV structures can be interpreted as emerging from a conspiracy of other crosslinguistic patterns which include frequent limitations on vowel hiatus and consonant clusters, tendencies toward obligatory consonant-initial or vowel-final word forms, and the fact that the existence of large consonant strings in any word position in a language implies the existence of simple (single C) structures in those positions. As a result of these interacting patterns, it follows that the canonical syllable patterns of any language will include structures of the form CV.

  Nevertheless, much of the research on motivations behind crosslinguistic\linebreak trends in syllable patterns returns to the idea of CV as a universal or otherwise privileged syllable type. Some of these proposals will be discussed in the following sections, as other crosslinguistic patterns relating to syllable structure are discussed.

\subsubsection{{Asymmetries} {in} {onset} {and} {coda} {patterns}}\label{sec:1.1.2.2}

  Many of the typological patterns involving syllables reveal asymmetries in the structure, distribution, and frequency of onsets versus codas. It follows from the crosslinguistic ubiquity of the CV syllable type that all languages have syllables with onsets. By comparison, languages in which syllable codas do not occur are not uncommon: for example, 12.6\% of the languages whose syllable structures were analyzed in the World Atlas of Language Structures Online (WALS) have canonical CV or (C)V structures only. Thus an implicational relationship holds between codas and onsets: if a language has syllables with codas, then it also has syllables with onsets.

  While the CV shape dominates in frequency distributions within and across languages, its mirror image, the VC structure, is not nearly so freely distributed. Its crosslinguistic lexical frequency distribution is tiny compared to that of CV: only 2.5\% of the syllables in the ULSID database are of the VC type (\citealt{ValléeEtAl2009}). The presence of VC shapes in a language generally implies the presence of V, CV, and CVC structures as well \citep{Blevins1995}. These striking differences in distribution indicate that onsets and codas are not equivalent structures.

  In many languages with single-member codas, consonants in the coda position are restricted to a smaller set of segments than what can be found in onset position. For example, \ili{Cocama-Cocamilla} has a consonant phoneme inventory of /p t k t͡s t͡ʃ x m n ɾ w j/. Any of these consonants may function as a syllable onset, but only the alveolar nasal /n/ and the glides /w j/ occur in coda position (except for under certain structural and prosodic conditions,  \citealt{VallejosYopán2010}: 110). \citet{Krakow1999} reports that some classes of segments, such as oral stops, are crosslinguistically disfavored in syllable-final position. Similarly, \citet[301]{Clements1990} observes that when both sonorants and obstruents occur in syllable-final position in a language, the set of permissible obstruents tends to be smaller than the set of permissible sonorants. In a crosslinguistic investigation of syllable frequencies in the lexicons of \ili{Hawaiian}, \ili{Rotokas}, Pirahã, \ili{Eastern Kadazan}, and \ili{Shipibo}, \citet{MaddiesonPrecoda1992} found that CV sequences are relatively unrestricted in their occurrence. Most onset-nucleus combinations in the study occur at rates approximating the values that would be expected from their component segment frequencies. Meanwhile, nucleus-coda combinations are more restricted in their combinatoriality, owing not only to generally smaller sets of allowable consonants in the coda position, but also to restrictions on sequences of particular segments.

  Both within and across languages, onsets and codas are most frequently simple, consisting of just one consonant. When languages do have tautosyllabic consonant clusters, they are more likely to occur in the onset position \citep{Blevins2006}. In languages that have tautosyllabic clusters in both onset and coda positions, it is often the case that more elaborate structures are permitted for onsets: these tend to be larger, more frequent, and less restricted in their internal patterns than coda clusters (\citealt{Greenberg19651978}, \citealt{Blevins2006}). There are of course exceptions to these patterns: \ili{Dizin}, for instance, has a canonical syllable pattern of (C)V(C)(C)(C) \citep{Beachy2005}. However, as will be shown in \sectref{sec:3.3.1}, such patterns are crosslinguistically less frequent than their mirror images.

  Diverse accounts have been put forth in the literature to account for asymmetries in onset and coda patterns. A long line of research starting with \citet{Sievers1881} and \citet{Jespersen1904} has argued that the internal organization of the syllable is governed by the phonological principle of sonority, a scalar perceptual property of speech sounds. A typical sonority scale is given in \REF{ex:1.6} with sonority increasing from left to right:

\ea\label{ex:1.6}
  stop < fricative < nasal < liquid < glide < vowel
\z

In this view, the sonority contour of typical and preferred syllable types rises steeply at the beginning of the syllable and falls less steeply from the nucleus to the end of the syllable (\citealt{Zwicky1972,Hooper1976,Greenberg19651978,Clements1990}). Thus an ideal syllable would consist of a simple onset consisting of a low-sonority sound such as a stop, a vocalic nucleus, and either a coda of high sonority, such as a nasal or a liquid, or no coda at all.

  \citet{Kawasaki-Fukumori1992} proposes an acoustic-perceptual motivation for certain crosslinguistic syllable patterns, finding that CV sequences are more spectrally dissimilar from one another, and therefore better contrasted, than VC structures. This suggests that onsets are more likely to be correctly perceived by the listener and maintained in languages. In the speech processing literature, it has been found that onsets are more easily identified by listeners than codas \citep{ContentEtAl2001} and that codas affect syllable complexity in such a way as to increase the time required for tautosyllabic onset processing \citep{SeguiEtAl1991}. 

  Mechanical and temporal constraints on jaw oscillation have been proposed as physiological motivations for the onset-coda asymmetry and predominance of CV patterns observed. In particular, \citet{MacNeilage1998} proposes that CV patterns derive from the earliest forms of human speech, in which open-close alternations of the mouth, simultaneous with phonation, provided a ‘frame’ for articulatory modulation and the emergence of distinct segmental patterns. From an articulatory point of view, the onset-coda asymmetry may reflect differences in intergestural timing between vowels and consonants in onset versus coda position (\citealt{Byrd1996a,BrowmanGoldstein1995,GickEtAl2006,MarinPouplier2010}). This body of research has established that the gestural coordination between onset and nucleus is synchronous, with the production of the consonant and vowel being nearly simultaneous and representing a stable timing relationship. As compared to the asynchronous and more variable timing relationship between nucleus and coda, the onset-nucleus relationship is more stable in the motor control aspects of its production.

  Finally, from a diachronic point of view, the relatively restricted status of codas may reflect the effects of reductive sound change: consonants in articulatorily weak word-final and syllable-final positions are particularly vulnerable to assimilation, lenition, and elision processes. Such processes can be observed in synchronic allophony and in patterns of historical sound change \citep{Bybee2015b}.

\subsubsection{{Consonant} {clusters}}\label{sec:1.1.2.3}

  Crosslinguistic patterns in consonant clusters are not limited to the tendency by which onset clusters tend to be larger and less restricted than coda clusters. It has long been observed that some cluster shapes are crosslinguistically more frequent than others. In fact, the phonological shape of clusters has been used, along with cluster size, as a diagnostic for syllable structure complexity. In the classification used by \citet{Maddieson2013a}, an onset cluster in which the second member is a liquid or a glide is considered less complex than one in which the second member is a nasal, fricative, or stop.

  Studies investigating onset and coda clusters have revealed trends in the voicing, place, manner, and sonority of consonant sequences in tautosyllabic clusters. \citet{Greenberg19651978} was one of the first large-scale studies of this kind, investigating both the size and specific phonotactic patterns of onset and coda clusters in 104 languages. This study yielded dozens of implicational generalizations. For instance, the presence of a cluster in a language tends to imply the presence of smaller sequences within it; e.g. in \ili{English}, the onset /spɹ/ as in \textit{spring} implies the onsets /sp/ as in \textit{spy} and /pɹ/ as in \textit{pry}. Greenberg also derived universals regarding phonetic and phonological properties of consonants in sequence: for example, sonorant+voiced obstruent codas tend to imply the occurrence of sonorant+voiceless obstruent codas. Many crosslinguistic studies in a similar vein have followed from this work. In general, such studies tend to be limited in scope to biconsonantal onset patterns. \citet{VanDam2004} is an exception, in that it explores tendencies in cluster size and composition in word-final codas of all sizes from 18 diverse languages. Some crosslinguistic studies of cluster patterns investigate voicing and manner implications regarding patterns of typologically rare structures, such as tautosyllabic sequences of obstruents (\citealt{Morelli1999,Morelli2003,Kreitman2008}). However, studies seeking to account for the crosslinguistically most frequent biconsonantal onset patterns -- a stop followed by a liquid or a glide, such as /pl/ or /ɡw/ -- are much more common in the literature (\citealt{Clements1990,BerentEtAl2008,BerentEtAl2011,Parker2012,Vennemann2012}). 

  Many of the latter studies appeal to the notion of sonority in explaining predominant cluster patterns. In fact, it would seem that a sonority model of syllable structure is more often used to explain cluster patterns than it is to explain the onset-coda asymmetries discussed in the preceding section. In this line of reasoning, cluster patterns in which there is an increasing sonority slope towards the nucleus (e.g. a /kl/ onset) are preferred to sonority plateaus (e.g. a /pk/ onset) or reversals (e.g. a /lb/ onset). Implicational universals using various sonority-based scales are often proposed to describe cluster inventory patterns, particularly the C\textsubscript{2} patterns observed in onsets. For example, \citet{Morelli1999} proposes a universal by which the presence of stop-stop onsets in a language implies the presence of stop-fricative onsets. \citet{LennertzBerent2015} predict that stop-nasal onsets are universally preferred to both stop-stop and stop-fricative onsets. \citet{Parker2012} proposes that the presence of biconsonantal onsets in a language implies the presence of a liquid or glide as C\textsubscript{2}. \citet{Vennemann2012} argues that the diachronic simplification of stop-initial biconsonantal onset inventories can be predicted by a six-point sonority scale, in which onset patterns with C\textsubscript{2} furthest to the right on the scale are lost first \REF{ex:1.7}.

\ea\label{ex:1.7}
  glide < rhotic < lateral approximant < nasal < fricative < stop
\z

  There are exceptions to the above generalizations. In a study of 46 diverse languages, it was found that stop-initial biconsonantal onset inventory patterns diverged from the patterns predicted by the scale in \REF{ex:1.7} roughly one-third of the time (\citealt{EasterdayNapoleãodeSouza2015}). 

  While a sonority account does capture strong trends in onset patterns, specifically the crosslinguistic predominance of stop-glide and stop-liquid onsets, accounts of syllable patterns appealing to sonority have been criticized for their circularity. Though sonority has been proposed to be correlated with intensity (\citealt{Gordon2002,Parker2002}), degree of constriction (\citealt{Chin1996,Cser2003}), and manner of articulation \citep{Parker2011}, it lacks a clear and crosslinguistically consistent phonetic definition.\footnote{{In this sense, the notion of sonority is much like that of the syllable.}} Instead, the notion of sonority is largely derived from phonotactic patterns, which are then explained in terms of sonority. Some have argued that sonority is in fact an epiphenomenon arising from perceptually motivated constraints, and that the only crosslinguistically consistent sonority contrast is the one between obstruents and sonorants (\citealt{JanyEtAl2007,HenkeEtAl2012}). \citet{OhalaKawasaki-Fukumori1997} reject the validity of sonority altogether, arguing that it is both circular and too broadly defined to account for the crosslinguistic rarity of sequences such as /pw/ and /dl/ and crosslinguistic prevalence of sequences such as /sk/. They propose that prevalent onset patterns reflect the high ``survivability'' of certain sequences, which in turn reflect strong modulations -- long trajectories in acoustic space --  in amplitude, periodicity, spectral shape, and fundamental frequency. In this view, sequences such as /ba/ are more strongly modulated than sequences like /ske/ or /ble/, which in turn are more strongly modulated than /pwe/, /pte/, and so on.

\subsubsection{{Nucleus} {patterns}}\label{sec:1.1.2.4}

  Crosslinguistic tendencies have also been observed in the patterns of syllable nuclei, which function as the auditory peaks of syllables. The prototypical syllable nucleus consists of a vowel, and indeed there are many languages which allow only vowels in nucleus position. However, there is a range of crosslinguistic variability in the types of segments observed to occur as syllable nuclei. In some languages, liquids or nasals may function as syllabic; e.g. \ili{Slovak} \textit{krv} [kr̩v] ‘blood’ \citep[186]{Zec2007}, and \ili{English} \textit{button} [bʌʔn̩]. Such patterns are generally well-accepted in the literature: liquids and nasals are vowel-like in some properties of their acoustic structure, so it is clear how such sounds might function as auditory peaks of syllables. More rarely, obstruents are reported to occur as syllable nuclei: e.g. \ili{Puget Salish} \textit{sqwəɬps} [sqwəɬ.ps̩] ‘cutthroat trout’ \citep[62]{Hoard1978}, \ili{Lendu} \textit{zz\`{} zz\'{} } [zz\`{} ̩.zz\'{} ̩] ‘drink’\todo{check diacritics}
  \citep[483]{Demolin2002}, \ili{Tashlhiyt} \textit{tftktstt} [tf̩.tk̩.ts̩tː] ‘you sprained it (\textsc{f})’ \citep[332]{Ridouane2008}. Such cases are often considered problematic, as they involve sounds which are not vowel-like in their acoustic properties and which may even be voiceless. This view discounts the fact that there are many kinds of obstruents with highly salient auditory properties, such as sibilant fricatives and ejective stops.

  As is the case with consonant clusters, accounts for crosslinguistic patterns of syllabic consonants often appeal to sonority as an explanatory mechanism, with predominant patterns said to reflect a preference for high-sonority syllable nuclei. Along similar lines of reasoning, nucleus patterns in languages are said to follow a sonority-based implicational hierarchy by which the presence of a given sound as a syllable nucleus in a language implies the presence of all more sonorous types of sounds as syllable nuclei (\citealt{Blevins1995,Zec2007}). Thus a language with syllabic nasals is predicted to also have syllabic liquids and vowels. In this model, syllabic obstruents are dispreferred and predicted to be the rarest kind of syllabic consonant.

  A survey of syllabic consonant patterns in 182 diverse languages suggests that the sonority account for syllable nucleus patterns does not capture some important crosslinguistic trends \citep{Bell1978a}. Of the 85 languages with syllabic consonants, 29 had syllabic liquids, 63 had syllabic nasals, and 34 had syllabic obstruents. The patterns considered in this survey include syllabic consonants arising through synchronic processes of vowel reduction, in addition to invariable syllabic consonant patterns, which are more often used to argue for a sonority basis for syllable nucleus patterns. However, the findings suggest that syllabic obstruents are not exceedingly rare, as often claimed, and may in fact be more common than syllabic liquids. A sonority-based implicational hierarchy fails to  account for a robust minority of the patterns observed in the study: 10/34 (29\%) of the languages with syllabic obstruents do not have syllabic liquids or nasals.

  As illustrated by the \ili{Lendu} and \ili{Tashlhiyt} examples above, in languages with syllabic obstruents, entire words or phrases without vowels may occur. There are many studies which seek to tackle the problem that such languages pose to models of the syllable (e.g. \citealt{Bagemihl1991} for \ili{Nuxalk}, \citealt{Coleman2001} for \ili{Tashlhiyt}). This is despite the fact that words without vowels are easily pronounceable by fluent speakers and may be relatively frequent in the languages in which they occur: for instance, \citet[328f]{Ridouane2008} reports that in \ili{Tashlhiyt}, 7.9\% of syntactic words in running text are composed entirely of voiceless obstruents. 

\subsubsection{{Syllable} {structure} {and} {morphology}}\label{sec:1.1.2.5}

  It has long been understood that morphological patterns can play an important role in syllable structure complexity. There are many languages in which the largest tautosyllabic consonant clusters arise through inflection or other morphological processes, for example in the coda /kst-s/ in \ili{English} \textit{texts}. On the basis of such observations, morphologically complex clusters have often been viewed with suspicion in theoretical treatments of the syllable. Comments casting doubt on their status as valid phonological structures can be found throughout the literature examining syllable patterns from both formal theoretical and descriptive typological perspectives: for example, many crosslinguistic studies of consonant clusters, such as \citet{Greenberg19651978} and others mentioned above, explicitly exclude morphologically complex clusters from their analyses. 

  When morphologically derived syllable structures are explicitly addressed in empirical studies of cluster patterns, it tends to be in order to examine how they differ from unambiguously phonological (morpheme-internal) clusters in aspects of their composition, processing, and acquisition. A recent research program has studied patterns of phonotactic (morpheme-internal) and morphonotactic (morphologically complex) consonant clusters (\citealt{DresslerDziubalska-Kołaczyk2006}). Several studies in this vein have approached the issue by analyzing properties of cluster inventories, finding that morphologically complex clusters are typically larger and more complex (in terms of sonority or alternative properties such as Net Auditory Distance) than those which occur within morphemes (\citealt{DresslerDziubalska-Kołaczyk2006}, \citealt{Orzechowska2012}). 

  Studies of L1 cluster acquisition have revealed earlier production and lower reduction rates for morphologically complex clusters than for morpheme-internal clusters, suggesting that the extra grammatical-semantic function carried by these structures may work in favor of their stability and maintenance, even if the shapes themselves are ``dispreferred'' (\citealt{Kamandulyte2006,Zydorowicz2010}). Morphologically complex clusters with phonotactically dispreferred patterns have in fact been proposed to facilitate parsing in speech perception, since they more reliably signal the morphological compositionality of words and thus feed back into the productivity of those morphemes (\citealt{HayBaayen2003,DresslerEtAl2010}). 

\subsection{Theoretical models and crosslinguistic patterns of syllable structure}\label{sec:1.1.3}

  The purpose of models of linguistic structure is to provide a framework and context within which to situate, explain, and make predictions about observed language patterns. As a result, models are often heavily influenced by frequent or well-documented crosslinguistic trends. Theoretical models of the syllable reflect many of the crosslinguistic patterns described above. 

  Many formalist models of the syllable reflect crosslinguistic trends which privilege CV over other patterns. The model of syllable structure proposed in Government Phonology \citep{KayeEtAl1990} follows in the tradition of generative syntax, in that every element in phonological structure is governed by some other element in a hierarchical fashion and an element may govern at most two constituents. In this model, the syllable element governs the onset and the rime. The rime branches into a nucleus and an optional simple coda. Depending upon the formulation of the model, the onset may branch into two consonants. A more extreme model following from this tradition, the Strict CV approach, posits only onset and nucleus constituents (\citealt{Lowenstamm1996,Scheer2004}). Because of the crosslinguistic tendency towards simple or biconsonantal onsets and simple or absent codas, these approaches are sufficient for describing syllable patterns in many languages. Where patterns do not fit into the proposed frame, empty nuclei are posited in order to preserve the underlying structure. Thus onset clusters are assumed to have intervening empty nuclei between the consonants, and simple codas are assumed to be followed by empty nuclei.

  Common crosslinguistic cluster patterns such as /s/+stop onsets and stop+/s/ codas have been considered problematic in some frameworks, as they represent sonority plateaus or reversals. In order to deal with such issues, it has been proposed that the /s/ in such patterns is not a part of the core syllable, but functions as an extrasyllabic appendix to it (\citealt{VauxWolfe2009,Duanmu2011}). Appendices and extrasyllabic elements are often posited for peripheral members of clusters which belong to separate morphemes. Interestingly, this approach may result in some of the most frequent clusters in a language (e.g. clusters coming about through inflectional markers) being set apart from morphologically simple ones in their phonological representation.

  In Optimality Theory, syllable patterns are not governed by a rigid model, but are motivated by universal constraints whose relative importance, or ranking, is determined on a language-specific basis (\citealt{PrinceSmolensky1993}). In this framework, surface phonetic forms are those which reflect the best possible output, that is, the fewest violations with respect to the constraint ranking. Crosslinguistic variation in syllable patterns is explained in terms of different rankings of these violable constraints. Many of the constraints reflect common crosslinguistic patterns, e.g. \textsc{Onset}, in which a violation mark is assigned to a syllable without an onset, and *\textsc{Nucleus}/X, in which a violation mark is assigned to syllable nuclei belonging to some sonority class X (e.g. obstruents; \citealt{McCarthy2008}).

  In the Articulatory Phonology framework, researchers have developed a coupled oscillator model of syllable structure which is heavily influenced by findings in the motor control literature (\citealt{NamSaltzman2003,GoldsteinEtAl2006,NamEtAl2009}). In this model, speech gestures are associated with planning routines, or oscillators, which activate the production of that gesture in speech. These oscillators are coupled to one another in one of two stable modes -- in-phase or anti-phase -- which determine the relative timing of the production of gestures. Gestures coupled in-phase are initiated synchronously, while gestures coupled anti-phase are initiated sequentially. These coupling phases are proposed to correspond to instrumentally established timing relationships observed in the syllable, in which onset gestures are produced synchronously with those of the vowel but coda gestures are timed sequentially after those of the vowel. This model provides a motor control basis for the privileged status of CV in language acquisition and frequency distributions, as well as the distinct timing patterns associated with onsets, codas, and clusters in each of those positions.

\section{Highly complex syllable structure: Typological outlier, theoretical problem}\label{sec:1.2}

  Having discussed some of the predominant crosslinguistic trends in syllable patterns, as well as frequent accounts for them, we return to the patterns presented in at the beginning of this chapter \xxref{ex:1.8}{ex:1.12}

\ea\label{ex:1.8}
\etriple{Yakima} \textbf{Sahaptin}{Sahaptian}{USA}

\textit{ksksa}\\
\glt ‘elephant ear (mushroom)’

(\citealt{HargusBeavert2006}: 29)
\z

\ea\label{ex:1.9}
\etriple{Georgian}{Kartvelian}{Georgia}

\textit{bɾt͡s’χ’ali}\\
\glt ‘claw’
\citep[204]{Butskhrikidze2002}
\z

\ea\label{ex:1.10}
\etriple{Tashlhiyt}{Afro-Asiatic}{Morocco}

\textit{tsːkʃftstː}

t-sː-kʃf-t=stː\\
\glt ‘you dried it (\textsc{f})’
\citep[332]{Ridouane2008}
\z

\ea\label{ex:1.11}
\etriple{Tehuelche}{Chonan}{Argentina}

\textit{kt͡ʃaʔʃpʃkn}

k-t͡ʃaʔʃp-ʃ-kn

\textsc{refl}-wash-\textsc{ps-realis}\\
\glt ‘it is being washed’

(\citealt{FernándezGarayHernández2006}: 13)
\z

\ea\label{ex:1.12}
\etriple{Itelmen}{Chukotko-Kamchatkan}{Russia}

\textit{kɬtxuniŋeʔn}

kɬ-txuni-ŋeʔn\\
\glt ‘very dark’

(\citealt{GeorgVolodin1999}: 55)
\z

In the context of the issues previously discussed, highly complex syllable patterns may be considered problematic in all respects.

  The syllable patterns in \xxref{ex:1.8}{ex:1.12} are, first of all, extremely large in comparison to the universally privileged CV shape. This fact has been pointed to explicitly in the literature as a reason to consider such patterns invalid: \citet[195]{KayeEtAl1990}, in a discussion of syllable patterns with four-consonant codas in \ili{Nez Perce}, write that “[t]he sheer length of such sequences makes one doubtful of their status as syllable constituents of one and the same syllable.” The example in \REF{ex:1.11} is chosen to illustrate that codas may be much longer than onsets in \ili{Tehuelche}, which goes against predominant crosslinguistic trends. Furthermore, the word-initial patterns in \REF{ex:1.8} and \REF{ex:1.12} consist entirely of obstruents, which should be strongly dispreferred according to both sonority models (e.g. \citealt{Clements1990}) and acoustic-perceptual models (\citealt{OhalaKawasaki-Fukumori1997}) of syllable structure. The word without vowels in \REF{ex:1.10} is typologically rare and implies syllabic obstruents, which are crosslinguistically ``dispreferred''. The patterns in \REF{ex:1.10}-\REF{ex:1.12} are further regarded as dubious because their clusters are morphologically complex and therefore perhaps not valid phonological structures. All of the patterns above, besides being typologically rare, are theoretically marginalized in that they represent the opposite of the predominant crosslinguistic patterns which models of the syllable seek to capture and describe.

  When highly complex syllable patterns are explicitly treated in the literature, it tends to be with the purpose of making their patterns fit into prevailing theoretical models. An example of this is \citegen{Bagemihl1991} analysis of \ili{Nuxalk} syllable structure. On the basis of reduplication data, Bagemihl analyzes the language as having “relatively ordinary” CRVVC syllable structure,\footnote{{Here R stands for ‘resonant,’ corresponding to a sonorant consonant.}} in which vowels, liquids, and nasals may function as V nuclei. Segments that do not fit into that syllable frame remain phonologically unsyllabified. Thus a word without sonorants -- like \textit{ɬχʷtɬcxʷ} ‘you spat on me’ -- while being fully and fluently pronounceable by speakers, is analyzed as entirely unsyllabified at the phonological level. Similarly, a strict CV approach has been used to account for ``ghost vowels'' -- vowels which alternate with zero -- in \ili{Mohawk} and \ili{Polish}, both of which have highly complex syllable patterns \citep{Rowicka1999}. However, this has the effect of positing long sequences of simple onsets followed by empty nuclei for the large consonant clusters which occur in those languages, as in \ili{Mohawk} \textit{khninus} ‘I buy’ or \ili{Polish} \textit{źdźbło} (/ʑd͡ʑbwo/) ‘blade of grass’. These novel phonological analyses are based upon careful consideration of both language-specific patterns and theoretical implications. However, such treatments of highly complex syllable structure have the effect of theoretically ``normalizing'' these rare syllable patterns: not by taking them at face value as corresponding to possible cognitive representations of language, but by arguing away their unusual properties until they more closely resemble familiar patterns.

  More problematic are approaches which treat highly complex syllable structure as anomalous or exotic. Such attitudes, as reflected by assumptions about what constitutes possible syllable length and constituency (cf. the quote by Kaye and colleagues above), make it all too easy for researchers to dismiss such patterns as improbable or regard them as statistical aberrations from an established norm. This seems to be more often the case when highly complex syllable patterns occur in underdescribed non-Eurasian languages. It sets a worrisome precedent when the patterns of minority, indigenous, and endangered languages are dismissed in this way. This reinforces a European bias and serves to further marginalize and exoticize languages which are already historically underrepresented in our discipline.

  Related to this point is the fact that much of the research in linguistics, including syllable structure typology, is influenced by an overrepresentation of data from European languages. A survey of crosslinguistic studies of consonant cluster patterns, for example, revealed an Indo-European bias which ranged from 34\% \citep{Morelli1999} to 79\% \citep{Vennemann2012} of the languages in those samples (\citealt{EasterdayNapoleãodeSouza2015}). In an investigation of the conformity of plosive-initial biconsonantal onset inventories to the predictions of a sonority-based implicational hierarchy in 46 diverse languages, only five of which were Indo-European, it was found that nearly one-third of the languages had patterns diverging from these predictions (Ibid.). None of the diverging patterns were found in Indo-European languages, and nearly all were from regions or families which tend to be underrepresented in linguistic research. This suggests that some of the reported norms of syllable structure typology may be heavily biased towards what has been observed in Indo-European and other well-represented families.

  Other issues which often go unexplored in accounts for crosslinguistic patterns of syllable structure are the influence of processes of language change and the relationship between syllable patterns and other elements of the phonology and the grammar. These issues are of special importance for typologically rare patterns, such as highly complex syllable structure, as they provide a natural explanation for the emergence and maintenance of these purportedly dispreferred patterns. In the following section I briefly discuss some lines of research which situate the issue of syllable structure complexity within holistic typologies of language by relating it to other phonological and grammatical properties.

\section{Syllable structure complexity: Accounts and correlations}\label{sec:1.3}
\subsection{Speech rhythm typologies}\label{sec:1.3.1}

  A long line of research in linguistics has sought to characterize and measure rhythmic properties of language which are perceptually and psychologically sa\-lient to speakers and play an important role in language acquisition (\citealt{CutlerMehler1993}). The typology proposed by \citet{Pike1945} distinguished two speech rhythm types: stress-timed languages and syllable-timed languages, with \ili{English} being a prototypical example of the former and \ili{Spanish} being a prototypical example of the latter. This typology was later expanded to include a third category of mora timing, for which \ili{Japanese} is a prototypical example. In its initial formulation, it was postulated that the rhythmic properties of these language types reflect equal timing intervals between those units: between stresses for stress-timed languages, syllables for syllable-timed languages, and morae for mora-timed languages. This ``isochrony hypothesis'' was eventually instrumentally disconfirmed \citep{Roach1982}. Speech rhythm typologies subsequently shifted their  focus to phonological holism, relating rhythm types to a confluence of factors involving syllable structure, vowel reduction, vowel length contrasts, and properties of stress placement (\citealt{Roach1982,Dauer1983}). In this typology, simple syllable structure is proposed to occur with syllable timing, and complex syllable structure with stress timing. Reduction of vowels in unstressed syllables and variation in lexical stress patterns are additionally proposed to occur with complex syllable structure in stress-timed languages \citep{Auer1993}. The proposed co-occurrences are not meant to be categorical, and as will be discussed in \chapref{sec:5}, may reflect the patterns of European languages specifically \citep{Schiering2007}.

  Proposed measurements of the acoustic properties of speech rhythm have been suggested to relate directly to syllable structure. Metrics developed by \citet{RamusEtAl1999} correspond to the proportion of vocalic intervals and standard deviation of consonantal intervals in speech. In languages with high syllable complexity, a greater standard deviation of consonant intervals and a lower proportion of vocalic intervals is expected, corresponding to both the greater variation in syllable types and the higher probability of consonant sequences in running speech in such languages. When languages are plotted according to these metrics, they fall into groups which largely correspond to traditional rhythm categories of stress timing and syllable timing (but see \citealt{WigetEtAl2010} for criticism of this approach). When these metrics were calculated in a crosslinguistically diverse sample of languages representing various degrees of syllable structure complexity and other phonological properties, it was found that syllable structure complexity is indeed significantly correlated with the expected indices (\textit{p} < .005), lending empirical validation to the suggested relationship \citep{EasterdayEtAl2011}. However, the direction of causality behind the relationship is unclear from these findings: while syllable structure contributes heavily to the acoustic-perceptual properties of speech rhythm, it is not clear whether syllable structure necessarily causes or constitutes stress timing. It may instead be that syllable structure is affected by and comes about through the other prosodic and phonological features associated with stress timing, such as vowel reduction.

\subsection{Other holistic typologies}\label{sec:1.3.2}

 Some holistic typologies which consider syllable complexity attempt to relate the phonology, morphology, syntax, and discourse properties of language to one another. An example of one such ambitious typology is that proposed in various forms by Vladimir Skalička from 1958 to 1979 \citep{Plank1998}. \citet{Skalička1979} proposed five ideal types which languages are supposed to approximate, if not attain: polysynthesis (an idiosyncratic use of the term that does not correspond to modern usage), agglutination, flection, introflection, and isolation. The many phonological and grammatical properties proposed to co-occur in each of these types were meant to be mutually supportive. In two of the types -- agglutination and introflection -- complex consonant clusters are said to co-occur with rich consonant systems and a high degree of verbal inflection. Other properties of these very specifically-defined classes include a prevalence of vowel harmony and looser fusion between gramemes and the stem in the agglutination type, and root-internal marking in the introflection type. Like many proposed holistic typologies, Skalička’s is largely impressionistic and not based in extensive empirical evidence.

  A series of empirical studies by Gertraud Fenk-Oczlon and August Fenk have sought to establish correlations between certain grammatical and discourse properties of language and syllable structure specifically. \citet{FenkFenkOczlon1993} tested Menzerath’s Law (paraphrased as “the bigger the whole, the smaller the parts”) and found a significant negative linear correlation between the number of syllables per word and the number of phonemes per syllable, a measure roughly analogous to syllable complexity. Working from the observation that words have more syllables in agglutinating languages, \citet{FenkOczlonFenk2005} established a correspondence between complex syllable structure and a tendency towards prepositions and a low number of grammatical cases on the one hand and simple syllable structure and a tendency to postpositions and a high number of cases on the other. Finally, \citet{FenkOczlonFenk2008} found that high phonological complexity (determined by the number of distinct monosyllables in a language) was correlated with low morphological complexity and high semantic complexity (i.e. high degrees of homonymy and polysemy), as well as rigid word order and idiomatic speech. They explain these results in terms of complexity trade-offs which balance the different subsystems of language.

  The results of \citet{Shosted2006} conflict with those of Fenk-Oczlon \& Fenk. This empirical study attempts to test the negative correlation hypothesis, which holds that if one component of language is simplified, then another must be elaborated. Specifically, Shosted considers correlations between syllable structure and inflectional synthesis of the verb in a diversified sample of 32 languages. He finds a slightly positive but statistically insignificant correlation between complexity in the two measures. Shosted’s measure of phonological complexity is not based on measurements of maximal syllable complexity, but instead on the potential number of distinct syllables allowed in each language, a figure which is calculated from the number of phonemic contrasts, canonical syllable patterns, and various phonotactic constraints reported for each language.

\subsection{Consonantal and vocalic languages}\label{sec:1.3.3}

  In phonological descriptions and general typological studies, the terms \textit{consonantal} and \textit{vocalic} are sometimes used to describe the holistic phonological character of languages \xxref{ex:1.13}{ex:1.18}.

\ea\label{ex:1.13}
   “In this group, we find on the one hand highly consonantal languages like \ili{Kabardian} and other Northwest Caucasian languages […], and on the other hand vocalic languages with long morphemes, for example \ili{Indonesian} and related languages […]”\footnote{{Translation TZ.}}
\citep[309]{Skalička1979}
\z

\ea\label{ex:1.14}
   “Syntagmatically, all (indigenous) Caucasian idioms can be called ‘consonant-type languages,’ with more consonants in a speech sequence than vowels […] The same term (‘consonantal languages’) can be applied to them paradigmatically as well, all Caucasian languages being notorious for the richness of their consonantal inventories, versus restricted or very restricted vowel systems.” 
\citep[43]{Chirikba2008}
\z

\ea\label{ex:1.15}
  “[\ili{Polish}] can be described as a ‘consonantal’ language, in two respects: 
  (a) it has a rich system of consonant phonemes […] and (b) it allows heavy consonant clusters …” 
\citep[103]{Jassem2003}
\z

\ea\label{ex:1.16}
   “\ili{Slovak} is a more consonantal language than \ili{German} (27 vs. 21) …” 
(\citealt{DresslerEtAl2015}: 56)
\z

\ea\label{ex:1.17}
  “Since \ili{Italian} is clearly a less consonantal language than \ili{English} …” 
(\citealt{DresslerDziubalska-Kołaczyk2006}: 263)
\z

\ea\label{ex:1.18}
  “\ili{Tashlhiyt} can be described as a ‘consonantal language.’ […] What makes \ili{Tashlhiyt} a ‘consonantal language’ \textit{par excellence} is the existence of words composed of consonants only …” 
\citep[216]{Ridouane2014}
\z

	The use of these terms is especially prevalent in Slavic and Caucasian linguistics. In some of those contexts, the terms may refer directly to a holistic phonological typology of Slavic languages developed by \citet{Isačenko1939/1940}. In that work, ``consonantal languages'' are defined as having complex syllable structure, a higher proportion of consonants in the phoneme inventory, the presence of certain consonant contrasts such as secondary palatalization, and fixed or lexically-determined stress. By comparison, ``vocalic languages'' have simpler syllable structure, lower proportions of consonants in the phoneme inventory, fewer consonant place contrasts, and pitch accent or ‘musical intonation.’ Several of the descriptions above also make reference to the overall size of the consonant phoneme inventory and sequences of consonants in word patterns or the speech stream. The relationship between syllable structure complexity and consonant phoneme inventory size suggested above is an empirically established one: as will be discussed further in \chapref{sec:4}, \citet{Maddieson2013a} found a weak but highly significant positive relationship between these features in a set of 484 languages. These findings suggest that the use of the terms consonantal and vocalic is at least to some extent grounded in observable crosslinguistic patterns.

  Impressionistic descriptions of the phonetic characteristics of languages with highly complex syllable structure are evocative of descriptions of consonantal languages. I present some of these below \xxref{ex:1.19}{ex:1.22}.

\ea\label{ex:1.19}
  \textbf{Kabardian} (\textit{\ili{Abkhaz}-Adyge}; Russia, Turkey)

“On the whole, the vowels have comparatively little prominence, in comparison with the consonants.”
\citep[24]{Kuipers1960}

“[T]he typical \ili{Kabardian} pronunciation is imitated most easily if one pronounces the word without vowels other than \textit{a} and with a stress immediately after the initial consonant: the result will show the predominance of consonants over vowels that is typical of \ili{Kabardian} speech, and the syllabic peaks will be determined automatically by the stress and by the sonority of the sounds in the sequence.” 
\citep[43]{Kuipers1960}
\z

\ea\label{ex:1.20}
  \textbf{Camsá} (isolate; Colombia)
“Words are pronounced rapidly with vowels practically eliminated word medially. A degree of emphasis is placed on the vowel of the first syllable with the following syllables squeezed together before the stressed syllable.” 
\citep[86-7]{Howard1967}
\z

\ea\label{ex:1.21}
  \textbf{Thompson} (\textit{Salishan}; Canada)

“Basic vowel adjustments reflect the general tendency of the language to drop vowels from unstressed syllables wherever possible and to convert to /ə/ those vowels that are not dropped. In rapid speech, this tendency is nearly fully realized, so that few tense vowels are heard outside of stressed syllables.” 
(\citealt{ThompsonThompson1992}: 31)
\z

\ea\label{ex:1.22}
  \textbf{Itelmen} (\textit{Chukotko-Kamchatkan}; Russia)

“I suppose it is little exaggeration to say that in the [\ili{Itelmen}] language there are no vowels, or, perhaps, their vowels are so obscure that it is hardly possible to translate them to European [equivalents].”\footnote{{Translation SME.}} 
(\citealt{Volodin1976}: 40-1; quoting V. N. Tyushov)
\z

  These vivid descriptions of fluent speech in languages with highly complex syllable structure are surely influenced by the stark differences between these phonetic patterns and those of the languages spoken natively by the researchers. However, taken along with observations regarding consonantal languages, as well as findings in the speech rhythm and holistic typology literature, they also suggest a path forward for investigating highly complex syllable structure as a coherent linguistic type characterized by an array of phonetic and phonological features.

\section{The current study}\label{sec:1.4}

  The current study is a crosslinguistic investigation of highly complex syllable patterns, their properties, their associations with other linguistic features, and their emergence over time. The two aims of the study are (i) to establish whether languages with highly complex syllable structure constitute a linguistic type, in the sense denoted by the holistic typologies described above, and (ii) to identify possible diachronic paths and natural mechanisms by which these patterns come about in the history of a language. A secondary goal is to ``de-exoticize'' these rare syllable patterns by considering them at face value as natural language structures rather than as typological and theoretical outliers.

\subsection{Research questions}\label{sec:1.4.1}

  The two broad research questions follow directly from the aims of the study listed above. The first is given in \REF{ex:1.23}.

\ea\label{ex:1.23}
   \textit{Do languages with highly complex syllable structure share other phonetic and phonological characteristics such that this group can be classified as a linguistic type?}
\z

  This research focus seeks to establish whether highly complex syllable structure is a linguistic type characterized by a convergence of associated phonetic and phonological properties. The properties to be considered follow in part from the findings and proposals in the holistic typologies described above. These include properties of syllable structure, phoneme inventories, suprasegmental patterns, and processes of vowel reduction and consonant allophony (see the following section for a detailed list of considerations). The specific hypotheses regarding the associations between syllable complexity and these properties will be presented with each analysis in upcoming chapters.

  While the term ``linguistic type'' is used in the formulation of \REF{ex:1.23}, this is not meant in the sense that I expect the results of the analyses to set these languages apart from others in a strict categorical way. As with the holistic language typologies discussed above, it is more likely that phonetic and phonological properties will show a tendency to cluster together. If such expectations are borne out in the analyses, they may aid in addressing the second research question:

\ea\label{ex:1.24}
   \textit{How does highly complex syllable structure develop over time?}
\z

  As will become apparent in the following chapters, capturing the development of highly complex syllable structure in real time is not a straightforward endeavor: syllable patterns seem to be remarkably stable and persistent over time and within language families (\citealt{NapoleãodeSouza2017}). Where synchronic and historical accounts based on direct evidence are available, these are useful in approaching the research question in \REF{ex:1.24}. Additionally, methods of diachronic typology can be used. This will be discussed further below.

\subsection{Proposed analyses and framework}\label{sec:1.4.2}

  The research questions outlined above are investigated in a sample of 100 languages representing four different categories of syllable complexity and selected to maximize genealogical and geographic diversity. The size and construction of the sample is designed to allow for a maximally systematic investigation of both research questions (see \chapref{sec:2} for further detail). For practical reasons, the scope of the book is largely limited to the analysis of phonological and phonetic properties, but in a few cases morphological factors are additionally considered. The analyses are grouped into five coherent studies, each corresponding to a chapter. These are listed below.

\ea\label{ex:1.25}
  \textbf{Phonological and phonetic properties considered}

\textbf{Syllable patterns (\chapref{sec:3})}

\textit{Size, location, phonological shape, and morphological complexity of maximal clusters}

\textit{Nucleus patterns, including syllabic consonants}

\textit{Morphological patterns of syllabic consonants}

\textit{Relative prominence of highly complex syllable patterns within languages}

\textit{Phonetic properties of large clusters}

\textbf{Segmental inventories (\chapref{sec:4})}

\textit{Consonant phoneme inventory size}

\textit{Consonant articulations present}

\textit{Vocalic nucleus inventory size}

\textit{Vocalic contrasts present}

\textbf{Suprasegmental properties (\chapref{sec:5})}

\textit{Presence of tone and word stress}

\textit{Predictability of word stress placement}

\textit{Phonological asymmetries between stressed and unstressed syllables}

\textit{Phonetic processes conditioned by stress}

\textit{Phonetic correlates of stress}

\textbf{Vowel reduction (\chapref{sec:6})}

\textit{Presence and prevalence of vowel reduction}

\textit{Affected vowels}

\textit{Conditioning environments}

\textit{Outcomes of vowel reduction and effects on syllable patterns}

\textbf{Consonant allophony (\chapref{sec:7})}

\textit{Presence of specific types of assimilation, lenition, and fortition}

\textit{Conditioning environments}
\z

  The results of these analyses are used to directly address the research question regarding the establishment of languages with highly complex syllable structure as a linguistic type. While one goal is to quantify associations between syllable structure complexity and specific linguistic features, qualitative patterns in the data will also be considered in this endeavor.

  Additionally, the results will be used to inform diachronic paths by which highly complex syllable patterns develop, addressing the second research question. Specifically, the methods of diachronic typology -- the use of “synchronic variation to dynamicize a typology” \citep[272]{Croft2003} -- are used. In this method, diachronic processes and paths are inferred, with careful consideration of attested processes and known directionality of language change, from synchronic patterns. This method is especially valuable in the current study, as many of the languages with highly complex syllable structure have little historical documentation. Strong tendencies in the phonetic and phonological properties of languages with highly complex syllable patterns may point to processes of language change which tend to precede, accompany, or follow the development of these structures, hinting at steps in the historical evolution of this linguistic type.

  Like most typological studies, the analyses in this book rely on written reference materials and are therefore based on standard features of structural linguistic analysis, such as phoneme inventories and phonological processes. However, the interpretations of patterns are informed by a theoretical framework which views the patterns of organization within language as dynamic, interactive, and emergent from usage (\citealt{BecknerEtAl2009,Bybee2001,Bybee2010}).

  While I do not have a finely articulated hypothesis regarding the diachronic development of highly complex syllable structure, I enter into these studies with a few ideas and assumptions regarding this issue. Following findings in the speech rhythm literature, I expect that vowel reduction, especially processes resulting in vowel deletion or the development of syllabic consonants, will be highly relevant in the development of these patterns. Since vowel reduction is often associated with unstressed syllables, it is also expected that stress will play an important role. These phenomena may be accompanied by particular processes of consonant allophony, such as palatalization, which over time have the effect of increasing consonant phoneme inventory sizes. 
  
  Finally, an important aspect of syllable structure development that can be only briefly considered here is the role of morphology. Based upon observations of morphologically complex clusters in languages with highly complex syllable structure, as well as associations posited between syllable complexity and morphological patterns in the literature, I expect that the development of these syllable patterns is often facilitated by a high degree of inflectional or derivational morphology in a language. In a speculative scenario, it is easy to imagine highly complex syllable patterns developing in a highly inflectional affixing language in which stress falls on the root or stem and eventually has segmental effects which include the reduction and eventual deletion of unstressed vowels. This may result in long heteromorphemic consonant sequences at word boundaries. The plausibility of the various aspects of such a scenario will be explored in subsequent chapters.

