\addchap{\lsAbbreviationsTitle\ and conventions} 
\begin{multicols}{2}
\begin{tabbing}
\textsc{interrog}\hspace{1em}\= translated from the original\kill
\#\_            \> Word-initial\\
\_\#             \> Word-final\\
1                 \> first person\\
3                 \> third person\\
\textsc{agt}      \> agentive\\
C                 \> Complex\\
\textsc{caus}     \> causative\\
Cd                \> Coda\\
\textsc{cl}       \> noun class\\
\textsc{cont.p}   \> past continuous\\
\textsc{dat}      \> dative\\
\textsc{dem}      \> demonstrative\\
\textsc{dub}      \> dubitive\\
\textsc{f}        \> feminine\\
\textsc{fut}      \> future\\
HC                \> Highly complex\\
\textsc{incl}     \> inclusive\\
\textsc{inf}      \> infinitive\\
\textsc{interrog} \> interrogative\\
\textsc{iter}     \> iterative\\
\textsc{m}        \> masculine\\
MC                \> Moderately complex\\
\textsc{nmlz}     \> nominalizer\\
\textsc{nom}      \> nominative\\
O                 \> Onset\\
\textsc{pl}       \> plural\\
\textsc{poss}     \> possessive\\
\textsc{ps}       \> predicate specifier\\
\textsc{pst}      \> past\\
\textsc{realis}   \> realis\\
\textsc{recp}     \> reciprocal\\
\textsc{refl}     \> reflexive\\
\textsc{rel}      \> relativizer\\
S                 \> Simple\\
\textsc{sg}       \> singular\\
RNS      \> translated from the original\\ \> by Ricardo Napoleão de Souza\\
SME      \> translated from the original\\ \> by Shelece Easterday\\
TZ      \> translated from the original\\ \> by Tim Zingler\\
V-less wds       \> Words without vowels
\end{tabbing}
\end{multicols}\addvspace{2\baselineskip}

\noindent In this text, some terms referring to syllable structure complexity may appear in both uppercase (e.g. Simple, Highly Complex) and lowercase (e.g. simple, highly complex). When uppercase is used, the term refers to a syllable structure complexity category as strictly defined in \sectref{sec:2.2}, or to the group of languages in the sample whose syllable patterns belong to that category. This convention is most often used in the presentation and discussion of the quantitative analyses in the book. When lowercase is used, it indicates a more general reference to relative syllable structure complexity. In this usage, the primary object of study is referred to as highly complex syllable structure.
