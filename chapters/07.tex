\chapter{Consonant allophony}\label{sec:7}
\section{Introduction and hypotheses}\label{sec:7.1}

  In this chapter, I explore ongoing patterns of consonant allophony in the language sample which may shed light on some of the associations between syllable structure complexity and segmental and suprasegmental properties observed in preceding chapters. While the purpose of the study of vowel reduction in \chapref{sec:6} was in part to observe directly the processes which cause syllable patterns to become more complex, the study in this chapter approaches the issue of the development of syllable structure complexity more obliquely. 

  Recall that in \chapref{sec:4}, several segmental correlates of highly complex syllable structure were established. Specifically, palato-alveolar, uvular, ejective, and affricate articulations were found to be most frequent in languages in the Highly Complex category. As discussed in \sectref{sec:4.5.4}, these articulations are often observed to come about through processes of assimilation (especially of consonants to vowels) and fortition. By contrast, the articulations associated with the Simple category (prenasalization and flaps/taps) are often observed to come about through processes of lenition and sonorization. This observation brings up the question of whether the segmental properties associated with the Highly Complex category, and the sound change processes they imply, precede, follow, or accompany the development of complex syllable patterns in those languages, which are themselves directly caused by vowel reduction.

  In \sectref{sec:5.4.3}, it was found that for languages with word stress, the percentage of languages with unstressed vowel reduction increased with syllable structure complexity, particularly when languages in the Simple category are compared against those of the other three categories. By comparison, the trends in stress-conditioned consonant allophony showed erratic (in the case of unstressed syllables) or level (in the case of stressed syllables) trends with respect to syllable structure complexity. These trends, first shown in \figref{fig:5.2}, are reproduced below (\figref{fig:7.1}).

  
\begin{figure}
\begin{tikzpicture}
\pgfplotstableread{data/fig71.csv}{\table}
    \pgfplotstablegetcolsof{\table}
    \pgfmathtruncatemacro\numberofcols{\pgfplotsretval-1}
            \begin{axis}[easterdayline]
            \foreach \i in {1,...,\numberofcols} {
                \addplot+ table [x index={1},y index={\i},x expr=\coordindex] {\table};
                \pgfplotstablegetcolumnnamebyindex{\i}\of{\table}\to{\colname} % Adding column headers to legend
                \addlegendentryexpanded{\colname}
            }
            \end{axis}                                                                           
\end{tikzpicture}
\caption{\label{fig:7.1}Percentage of languages with word stress in each category of syllable structure complexity exhibiting stress-conditioned vowel reduction or consonant processes.}
\end{figure}

As discussed in \chapref{sec:5}, the findings with respect to stress-conditioned consonant processes were unexpected, in that it follows from the speech rhythm literature and related work (\citealt{BybeeEtAl1998,Schiering2007}) that segmental effects of stress in general will \textit{increase} with syllable structure complexity. While that was the case with unstressed vowel reduction (see also \sectref{sec:6.3.4}), it was not so for consonant allophony in stressed or unstressed syllables.

  While it proved to be an unexpected result in \chapref{sec:5}, the patterns of stress-conditioned consonant allophony may provide valuable information for formulating hypotheses regarding when the segmental properties associated with the Highly Complex category develop in relation to the development of the syllable patterns themselves. If the syllable structure complexity scale represents a diachronic cline, then the patterns in \figref{fig:7.1} suggest that in languages with word stress, consonant allophony may be equally as prevalent as vowel reduction in early stages of syllable structure change. This suggests that we might expect to find allophonic processes resulting in the articulations associated with the Highly Complex category more often in languages with simpler syllable structure. As these processes become more prevalent and regular over the history of a language, they may phonologize and become part of the segment inventory of the language, ceasing to be productive. If accompanied or followed by vowel reduction, this diachronic scenario could result in languages with complex syllable structure being more likely to have those specific consonant articulations as contrastive phonemes.

  The scenario above is speculative and, moreover, based solely on the findings for stress-conditioned segmental processes discussed in \chapref{sec:5}. However, it motivates a hypothesis which is testable in the language sample \REF{ex:7.1}.

\ea\label{ex:7.1}
  H\textsubscript{1}: As syllable structure complexity decreases, allophonic processes resulting in palato-alveolars, uvulars, ejectives, and affricates will become more prevalent.
\z

This hypothesis predicts that allophonic processes resulting in articulations associated with Highly Complex syllable structure will be most prevalent in languages of the Simple category.

  As discussed in \chapref{sec:4}, the articulations associated with the Highly Complex category typically come about through processes of assimilation of consonants to vowels and fortition. Along the same lines of reasoning that motivate the hypothesis in \REF{ex:7.1}, we might predict assimilatory and strengthening processes in general to be more prevalent in languages with simpler syllable structure. This motivates a second hypothesis \REF{ex:7.2}.

\ea\label{ex:7.2}
  H\textsubscript{2}: As syllable structure complexity decreases, allophonic processes of assimilation of consonants to vowels and fortition will become more prevalent.
\z

  \ili{Even} if these hypotheses are borne out in the data, they may not support the diachronic scenario described above. After all, the trend in \figref{fig:7.1} shows the percentage of languages with stress-conditioned consonant allophony in general decreasing with syllable structure complexity. The patterns predicted by \xxref{ex:7.1}{ex:7.2}, if borne out, must be disambiguated in some way from other patterns of consonant allophony in order to be taken as support for any diachronic path. Therefore I predict that allophonic processes resulting in the articulations associated with the Simple category -- prenasalization and flaps/taps -- will show a different trend with respect to syllable structure complexity, perhaps remaining level or increasing with syllable structure complexity. Similarly, I predict that allophonic processes resulting in lenition or sonorization more generally will follow a similar pattern, either remaining level or increasing in prominence with increasing syllable structure complexity. Since my predictions for the processes associated with the Simple category are not specific, I do not formulate hypotheses for them here. However, these patterns will be considered in the analyses that follow.

  Note that although the hypotheses here are motivated by patterns observed in \chapref{sec:5} for stress-conditioned consonant allophony, the analyses here consider all relevant processes of consonant allophony and not just those conditioned by the stress environment.

\section{Methodology}\label{sec:7.2}
\subsection{Patterns considered}\label{sec:7.2.1}

  As with the study of vowel reduction in \chapref{sec:6}, only phonetically or phonologically conditioned processes affecting consonants are considered here. I proceed here with the same disclaimers regarding author biases and judgments about the details and directionality of the processes reported in language references.

  In order to test the first hypothesis, allophonic processes resulting in the consonant articulations found to be positively associated with the Highly Complex category were considered. These include \textit{palato-alveolar}, \textit{uvular}, \textit{ejective}, and \textit{affricate} outcomes. The \textit{uvular} category includes articulations described as post-velar or back velar. I did not consider processes producing the articulations found to be associated with the Highly Complex category on the basis of fewer than ten data points; that is, processes resulting in \textit{lateral fricatives}, \textit{lateral affricates}, and \textit{pharyngeals} were not considered. However, there were almost no examples of such processes in the language sample. Allophonic processes resulting in ejectives were also not found in the language sample. Some examples of allophonic processes resulting in articulations associated with Highly Complex syllable structure can be found in \xxref{ex:7.3}{ex:7.5}.

\ea\label{ex:7.3}
  \textbf{Chipaya} (\textit{Uru-Chipaya}; Bolivia)

Dental fricative /s̪/ is realized as [ʃ] when occurring between high vowels.

/s̪qis̪i/

[s̪qiʃi]\\
\glt ‘leather’
(\citealt{Cerrón-Palomino2006}: 48-9)
\z

\ea\label{ex:7.4}
  \textbf{Bashkir} (\textit{Turkic}; Russia)

Voiceless velar fricative /x/ is optionally realized as post-velar [χ] in words with only back vowels.

/xɑfɑ/

[χɑfɑ]\\
\glt ‘worry’
\citep[11]{Poppe1964}
\z

\ea\label{ex:7.5}
  \textbf{Semai} (\textit{Austroasiatic}; Malaysia)

Voiceless palatal stop /c/ is slightly affricated [c͡ç] in syllable onsets.

/mɑcɔːt/

[mɑc͡çɔːt]\\
\glt ‘small’
\citep[5]{Philips2007}
\z

  In order to test the second hypothesis, allophonic processes resulting in consonant assimilation to an adjacent vowel were considered. Only processes conditioned by adjacent vowels which have the effect of causing the consonant to become more like the vowel in articulatory terms (as denoted by articulatory descriptions or IPA transcriptions) were considered here: e.g. a velar consonant produced with labialization adjacent to rounded vowels. I limited the processes examined here to those involving \textit{palatalization}, \textit{labialization}, or \textit{velarization}. The term \textit{palatalization} here includes any process resulting in a consonant moving closer to the palatal region, except for those resulting in palato-alveolars, which are considered in the group of processes described above. This category includes processes resulting in fronting of velars or uvulars, secondary palatalization, and the production of palatal consonants. See \xxref{ex:7.6}{ex:7.8} for examples.

\ea\label{ex:7.6}
  \textbf{Lepcha} (\textit{Sino-Tibetan}; Bhutan, India, Nepal)

Velar stops /k ɡ/ are slightly palatalized [kʲ ɡʲ] before front vowels /i e/.

/kit/

[kʲit]\\
\glt ‘snatch’
\citep[21]{Plaisier2007}
\z

\ea\label{ex:7.7}
  \textbf{Karok} (isolate; USA)

Voiceless velar fricative /x/ is labialized [xʷ] after a back (rounded) vowel.

/θuxxaθ/

[θuxʷxʷaθ]\\
\glt ‘mother’s sister’
\citep[8]{Bright1957}
\z

\ea\label{ex:7.8}
  \textbf{Lezgian} (\textit{Nakh-Daghestanian}; Azerbaijan, Russia)

Lateral approximant /l/ is velarized [ɫ] syllable-finally following a back vowel.

/pʰtul/

[pʰtuɫ]\\
\glt ‘grandchild’
(\citealt{Haspelmath1993}: 35, 37)
\z

  Additionally, processes of fortition were considered in addressing the hypothesis in \REF{ex:7.2}. In defining fortition, I follow \citet{Bybee2015b} and \citet{BybeeEasterday2019} in considering fortition to be an increase in the magnitude of a gesture. Here, I apply that definition only to changes in manner of articulation, and include processes which result in a consonant being produced with greater constriction relative to its original articulation (again as denoted by an explicit articulatory description or the change in articulation implied by the IPA transcriptions used). Though gemination and consonant lengthening are commonly described as fortition and involve an increase in the duration of gestures, I have excluded them from the present analysis because the segmental analyses in \chapref{sec:4} did not consider consonant length. The processes here fall into two categories: \textit{glide strengthening}, in which a glide becomes more constricted in its articulation, and other \textit{increased constriction}, in which any other kind of consonant becomes more constricted in its articulation. While I did not include processes involving the total assimilation of a consonant to an adjacent consonant, other kinds of increases in constriction which involved consonantal conditioning were included. However, this was a minor conditioning environment in comparison to vocalic and domain environments. See \xxref{ex:7.9}{ex:7.11} for examples.

\ea\label{ex:7.9}
  \textbf{Cocama-Cocamilla} (\textit{Tupian}; Peru)

Palatal approximant /j/ is optionally realized as voiced alveolar fricative [z] intervocalically.

/pijaki/

[pizaki]\\
\glt ‘toucan’
(\citealt{VallejosYopán2010}: 99-100)
\z

\ea\label{ex:7.10}
  \textbf{East Kewa} (\textit{Nuclear Trans New Guinea}; Papua New Guinea)

Velar and labial fricatives /x ɸ/ occur as affricates utterance-initially.

/xaa/

[kxaa]\\
\glt ‘smell’
(\citealt{FranklinFranklin1978}: 24)
\z

\ea\label{ex:7.11}
  \textbf{Albanian} (\textit{Indo-European}; Albania, Serbia, Montenegro)

Fricatives /f v θ ð/ have occasional stop allophones word-finally and before consonants.

/kafʃon/

[kap̪ʃon]\\
\glt ‘it bites’
\citep[16]{Newmark1957}
\z

  The processes resulting in the consonant articulations strongly associated with the Simple category which are considered here include \textit{prenasalization} and \textit{flap/tap consonants}. See \xxref{ex:7.12}{ex:7.13} for examples of the processes collected.

\ea\label{ex:7.12}
  \textbf{Darai} (\textit{Indo-European}; Nepal)

Intervocalically, voiced bilabial stop /b/ may be realized as [mb].

/kabo/

[kambo]\\
\glt ‘house post’
(\citealt{KotapishKotapish1973}: 27)
\z

\ea\label{ex:7.13}
  \textbf{Kadiwéu} (\textit{Guaicuruan}; Brazil)

The voiced alveolar stop /d/ is realized as a tap [ɾ] intervocalically in rapid speech.

/d͡ʒit͡ʃiditike/

[d͡ʒit͡ʃiɾitike]\\
\glt ‘I swing it’
\citep[16]{Sandalo1997}
\z

  Finally, I considered several specific types of lenition in this study. I take lenition to be an articulatory weakening; that is, a decrease in the magnitude or duration of a gesture (\citealt{BrowmanGoldstein1992b,MowreyPagliuca1995,BybeeEasterday2019}). The processes considered here are prototypical types of lenition or sonorization, processes in which a consonant becomes more vowel-like in its articulation: \textit{obstruent voicing, spirantization of stops/affricates}, \textit{debuccalization}, and \textit{consonants becoming glides or vowels}. Spirantization is defined here as any process which involves a stop or affricate becoming a fricative. Debuccalization involves the loss of the oral constriction of a consonant. Examples of such processes are given in \xxref{ex:7.14}{ex:7.17}.

\ea\label{ex:7.14}
  \textbf{Mohawk} (\textit{Iroquoian}; Canada, USA)

Alveolar fricative /s/ is voiced word-initially preceding a vowel and intervocalically.

/onisela/

[onizela]\\
\glt ‘shelf’
\citep[30-1]{Bonvillain1973}
\z

\ea\label{ex:7.15}
  \textbf{Pech} (\textit{Chibchan}; Honduras)

Voiced bilabial stop /b/ is realized as fricative [β] intervocalically.

/tibiebiska/

[tiβieβiska]\\
\glt ‘type of grass’
\citep[16]{Holt1999}
\z

\ea\label{ex:7.16}
  \textbf{Towa} (\textit{Kiowa-Tanoan}; USA)

A voiceless alveolar fricative /s/ is realized as a glottal [h] syllable-initially in a syllable carrying low tone, especially among younger speakers.

/sõ/

[hõ]

pronominal prefix
\citep[13]{Yumitani1998}
\z

\ea\label{ex:7.17}
  \textbf{Gaam} (\textit{Eastern Jebel}; Sudan)

Voiced stops /b ɟ/ are weakened to approximants intervocalically.

/kaɟan/

[kajan]\\
\glt ‘bring-3.\textsc{sg.nom.cont.p}’
\citep[24-5]{Stirtz2011}
\z

  As with the vowel reduction study in \chapref{sec:6}, patterns of consonant allophony were considered to be one process if a sound or group of sounds was affected in the same way in the same conditioning environment. Patterns which were similar but differed along any of those parameters were coded as separate processes.

\subsection{Coding}\label{sec:7.2.2}

  The types of consonant allophony examined here are to varying extents defined by what sounds are affected and how. For this reason, the conditioning environment is the only aspect of the processes coded in detail. Conditioning environments were coded for the presence of four factors: segmental environment, stress environment, domain (word/phrase/utterance) environment, and free variation.

\section{Results}\label{sec:7.3}
\subsection{Distribution of processes in the language sample}\label{sec:7.3.1}

  In total, 288 allophonic processes fitting the descriptions of the process types given in \sectref{sec:7.2} were collected and analyzed. \tabref{tab:7.1} shows how these processes are distributed in the language sample. 

\begin{table}
\begin{tabularx}{\textwidth}{Qcccc}
\lsptoprule
 & \multicolumn{4}{c}{Syllable structure complexity}\\\cmidrule(lr){2-5}
& S & MC & C & HC\\
& \textit{N} = 24 & \textit{N} = 26 & \textit{N} = 25 & \textit{N} = 25\\\midrule
 \textit{N} languages with process types considered here & 18 & 22 & 22 & 24\\
 \textit{N} processes collected & 69 & 77 & 84 & 58\\
 ratio processes/lg & 3.8 & 3.5 & 3.8 & 2.4\\
\lspbottomrule
\end{tabularx}
\caption{\label{tab:7.1}Distribution of allophonic consonant processes considered in the current study among categories of syllable structure complexity.}
\end{table}

Most of the languages in the sample (86/100) had at least one of the process types examined here. The proportion of languages reported to have these specific processes increases with syllable structure complexity, contrary to the pattern established for stress-conditioned consonant allophony in \chapref{sec:5} and shown in \figref{fig:7.1}. However, we can also see from the table that languages in the Simple, Moderately Complex, and Complex categories have more processes per language than those in the Highly Complex category. Below I present several analyses in order to directly address the hypotheses in \REF{ex:7.1} and \REF{ex:7.2} regarding the rates of allophonic processes resulting in articulations associated with the Highly Complex category, and assimilation and fortition more generally.

  As a first test of these hypotheses, in \figref{fig:7.2} I show the percentage of languages in each category of syllable structure complexity which have allophonic processes resulting in (i) articulations associated with the Highly Complex category (palato-alveolar, uvular/back velar, ejective, affricate), (ii) other assimilation of consonants to vowels (palatalization/fronting, labialization, and velarization), and (iii) fortition (glide strengthening or increased constriction).

\begin{figure}
\begin{tikzpicture}
\pgfplotstableread{data/fig72.csv}{\table}
    \pgfplotstablegetcolsof{\table}
    \pgfmathtruncatemacro\numberofcols{\pgfplotsretval-1}
            \begin{axis}[easterdayline]
            \foreach \i in {1,...,\numberofcols} {
                \addplot+ table [x index={1},y index={\i},x expr=\coordindex] {\table};
                \pgfplotstablegetcolumnnamebyindex{\i}\of{\table}\to{\colname} % Adding column headers to legend
                \addlegendentryexpanded{\colname}
            }
            \end{axis}                                                                           
\end{tikzpicture}
\caption{\label{fig:7.2}Percentage of languages in each category which have allophonic processes resulting in articulations associated with the Highly Complex category, fortition, or other assimilation of consonants to vowels.}
\end{figure}

  The pattern in the figure shows support for the hypothesis in \REF{ex:7.1}: as syllable structure complexity increases, the percentage of languages with allophonic processes resulting in the articulations associated with Highly Complex syllable structure moderately decreases. There is only mixed support for the hypothesis in \REF{ex:7.2}. While the percentage of languages with allophonic fortition processes slightly decreases with increasing syllable structure complexity, the trend in assimilation processes is level.

  As discussed above, additional support for the hypotheses may be found if allophonic processes resulting in articulations associated with the Simple category, and processes of lenition or sonorization more generally, exhibit a different pattern altogether with respect to syllable structure complexity. In \figref{fig:7.3}, I show the percentage of languages in each category of syllable structure complexity which have consonant allophony resulting in (i) articulations associated with the Simple category (prenasalization and flapping), and (ii) lenition or sonorization (obstruent voicing, spirantization, debuccalization, and consonants becoming glides or vowels).

\begin{figure}
\begin{tikzpicture}
\pgfplotstableread{data/fig73.csv}{\table}
    \pgfplotstablegetcolsof{\table}
    \pgfmathtruncatemacro\numberofcols{\pgfplotsretval-1}
            \begin{axis}[easterdayline]
            \foreach \i in {1,...,\numberofcols} {
                \addplot+ table [x index={1},y index={\i},x expr=\coordindex] {\table};
                \pgfplotstablegetcolumnnamebyindex{\i}\of{\table}\to{\colname} % Adding column headers to legend
                \addlegendentryexpanded{\colname}
            }
            \end{axis}                                                                           
\end{tikzpicture}
\caption{\label{fig:7.3}Percentage of languages in each category which have allophonic processes resulting in articulations associated with the Simple category or lenition/sonorization.}
\end{figure}
  Although the lenition/sonorization pattern shows an overall increase, neither of the trends in \figref{fig:7.3} shows a linear trend with respect to syllable structure complexity. But since the patterns differ markedly from the trend for Highly Complex-associated articulations in \figref{fig:7.2}, they can be taken as lending some tentative support to the hypotheses.

  The analyses thus far have only considered the presence of processes within languages, and not the individual processes themselves. In the following figures, I show how the process types pattern in terms of the percentage of the total number of allophonic processes they represent in each syllable structure complexity category.

\begin{figure}
\begin{tikzpicture}
\pgfplotstableread{data/fig74.csv}{\table}
    \pgfplotstablegetcolsof{\table}
    \pgfmathtruncatemacro\numberofcols{\pgfplotsretval-1}
            \begin{axis}[easterdayline]
            \foreach \i in {1,...,\numberofcols} {
                \addplot+ table [x index={1},y index={\i},x expr=\coordindex] {\table};
                \pgfplotstablegetcolumnnamebyindex{\i}\of{\table}\to{\colname} % Adding column headers to legend
                \addlegendentryexpanded{\colname}
            }
            \end{axis}                                                                           
\end{tikzpicture}
\caption{\label{fig:7.4}Percentage of processes in each category which result in articulations associated with the Highly Complex category, fortition, or other assimilation of consonants to vowels.}
\end{figure}

  The analysis in \figref{fig:7.4} shows that allophonic processes yielding articulations associated with Highly Complex syllable structure are more prevalent in languages with Simple syllable structure than the others. The trends with respect to fortition and assimilation, however, show little variation with respect to syllable structure complexity (note that the figure is scaled from 0-50\%).

  Examining the processes resulting in articulations associated with the Simple category (\figref{fig:7.5}), we find that these are infrequent in all four syllable structure complexity categories. Processes of lenition or sonorization are more frequent in general, and also increase in frequency with syllable structure complexity.

\begin{figure}
\begin{tikzpicture}
\pgfplotstableread{data/fig75.csv}{\table}
    \pgfplotstablegetcolsof{\table}
    \pgfmathtruncatemacro\numberofcols{\pgfplotsretval-1}
            \begin{axis}[easterdayline]
            \foreach \i in {1,...,\numberofcols} {
                \addplot+ table [x index={1},y index={\i},x expr=\coordindex] {\table};
                \pgfplotstablegetcolumnnamebyindex{\i}\of{\table}\to{\colname} % Adding column headers to legend
                \addlegendentryexpanded{\colname}
            }
            \end{axis}                                                                           
\end{tikzpicture}
\caption{\label{fig:7.5}Percentage of processes in each category which result in articulations associated with the Simple category or lenition/sonorization.}
\end{figure}

  In sum, the analyses in this section show that allophonic processes producing articulations associated with the segmental inventories of languages with Highly Complex syllable structure are most frequent in languages from the Simple category. Fortition shows a similar but less robust pattern, particularly when the presence or absence of processes is considered. Processes of lenition or sonorization are most frequent in the Complex and Highly Complex categories. Finally, assimilation of consonants to vowels and processes producing articulations associated with the segmental inventories of languages in the Simple categories do not show robust trends with respect to syllable structure complexity.

  While the results in this section provide some direct support for the first hypothesis \REF{ex:7.1} and some mixed support for the second hypothesis \REF{ex:7.2}, they are based on very abstract, general analyses of process types. In the following sections, each process type will be examined in more fine-grained detail.

\subsection{Processes resulting in articulations associated with the Highly Complex category}\label{sec:7.3.2}

  In this section, I examine allophonic processes in the language sample which result in the articulations most strongly associated with the Highly Complex category: palato-alveolar, uvular, and affricate. Recall that no allophonic processes resulting in ejective articulations were found in the language sample.

  In \figref{fig:7.6}, I plot the percentage of languages in each syllable structure complexity category which have processes resulting in palato-alveolar, uvular, or affricate articulations.

\begin{figure}
\begin{tikzpicture}
\pgfplotstableread{data/fig76.csv}{\table}
    \pgfplotstablegetcolsof{\table}
    \pgfmathtruncatemacro\numberofcols{\pgfplotsretval-1}
            \begin{axis}[easterdayline]
            \foreach \i in {1,...,\numberofcols} {
                \addplot+ table [x index={1},y index={\i},x expr=\coordindex] {\table};
                \pgfplotstablegetcolumnnamebyindex{\i}\of{\table}\to{\colname} % Adding column headers to legend
                \addlegendentryexpanded{\colname}
            }
            \end{axis}                                                                           
\end{tikzpicture}
\caption{\label{fig:7.6}Percentage of languages in each syllable structure complexity category with allophonic processes resulting in articulations associated with the Highly Complex category.}
\end{figure}
  We find that the percentage of languages with processes resulting in affricates is much higher for languages in the Simple category than the other three, although this trend is not linear. The trend in processes resulting in palato-alveolar articulations shows a moderate but steady decrease with syllable structure complexity. Processes producing uvulars are generally infrequent, occurring in only seven languages in the sample, and this trend is different from the others, peaking in the Complex category. Thus it would seem that the processes producing pa\-la\-to-alveolars and affricates drive the trend by which this group of processes is found more frequently in languages with simpler syllable structure.

  In \tabref{tab:7.2}, I examine the conditioning environments for the processes resulting in palato-alveolars, uvulars, and affricates. In order to simplify the presentation, I do not break the processes down by syllable structure complexity.

\begin{table}
\begin{tabularx}{\textwidth}{Qcccc}
\lsptoprule
 & \multicolumn{3}{c}{Allophonic processes yielding:} \\\cmidrule(lr){2-4}
& Palato-alveolar & Uvular & Affricate & Total for group\\
Conditioning environment & \textit{N} = 32 & \textit{N} = 9 & \textit{N} = 36 & \textit{N} = 57\\\midrule
 Segmental & 26 \textit{(81\%)} & 9 \textit{(100\%)} & 17 \textit{(47\%)} & 46 \textit{(81\%)}\\
 Domain & 4 \textit{(13\%)} & -- & 6 \textit{(17\%)} & 7 \textit{(12\%)}\\
 Stress & 2 \textit{(6\%)} & -- & 6 \textit{(17\%)} & 6 \textit{(11\%)}\\
 Free variation & 3 \textit{(9\%)} & -- & 11 \textit{(31\%)} & 13 \textit{(23\%)}\\
\lspbottomrule
\end{tabularx}
\caption{\label{tab:7.2}Conditioning environments for allophonic processes producing palato-alveolars, uvulars, and affricates. A process may have more than one conditioning environment. The total figures for the entire group reflect the fact that several processes have palato-alveolar affricate outcomes.}
\end{table}

  We see that segmental factors are by far the strongest conditioning environment for this group of processes. An examination of the specific segmental conditioning environments reveals they are almost always vocalic, suggesting a high degree of assimilation of consonants to vowels.  In particular, processes resulting in palato-alveolar and affricate outcomes are typically conditioned by high and/or front vowels, while those with uvular outcomes are typically conditioned by low and/or back vowels. What additionally sets the processes producing pa\-la\-to-alveolar and affricate articulations apart from those producing uvular articulations is a greater variety of conditioning environments in the former two groups. In the group of processes which produce affricates, the effect of the segmental environment is somewhat weaker, while stress and free variation play a stronger role.

\subsection{Other processes resulting in assimilation of consonants to vowels}\label{sec:7.3.3}

  Here, I examine more closely the allophonic processes resulting in palatalization, labialization, and velarization in the sample. In \figref{fig:7.7}, I show the percentage of languages in each category which have such processes.

\begin{figure}
\begin{tikzpicture}
\pgfplotstableread{data/fig77.csv}{\table}
    \pgfplotstablegetcolsof{\table}
    \pgfmathtruncatemacro\numberofcols{\pgfplotsretval-1}
            \begin{axis}[easterdayline]
            \foreach \i in {1,...,\numberofcols} {
                \addplot+ table [x index={1},y index={\i},x expr=\coordindex] {\table};
                \pgfplotstablegetcolumnnamebyindex{\i}\of{\table}\to{\colname} % Adding column headers to legend
                \addlegendentryexpanded{\colname}
            }
            \end{axis}                                                                           
\end{tikzpicture}
\caption{\label{fig:7.7}Percentage of languages in each syllable structure complexity category with allophonic processes resulting in articulations associated with assimilation of consonants to vowels.}
\end{figure}
  Recall that in the analyses shown in Figures \ref{fig:7.2} and \ref{fig:7.4}, the assimilation processes did not show the expected trend with respect to syllable structure complexity; that is, these processes were not more prevalent in languages with simpler syllable structure. Here we find that labialization and velarization, besides being relatively infrequent in the language sample, do not show any strong patterns with respect to syllable structure complexity. On the other hand, the palatalization trend in \figref{fig:7.7} is similar to the one in \figref{fig:7.6} for palato-alveolars. This is unsurprising, given that these are very similar process types.

  In \tabref{tab:7.3}, I show the conditioning patterns for these process types. Again, in order to simplify the presentation, I do not break down the processes by syllable structure complexity.

\begin{table}
\begin{tabularx}{\textwidth}{Qcccc}
\lsptoprule
 & \multicolumn{3}{c}{Allophonic processes yielding:} \\\cmidrule(lr){2-4}
 Conditioning & Palatalization & Labialization & Velarization & Total\\
 environment  &                &               &              & for group\\
 & \textit{N} = 25 & \textit{N} = 11 & \textit{N} = 3 & \textit{N} = 39\\\midrule
 Segmental & 25 \textit{(100\%)} & 11 \textit{(100\%)} & 3 \textit{(100\%)} & 39 \textit{(100\%)}\\
 Domain & 2 \textit{(8\%)} & 1 \textit{(9\%)} & 1 \textit{(33\%)} & 4 \textit{(10\%)}\\
 Stress & 3 \textit{(12\%)} & 1 \textit{(9\%)} & -- & 4 \textit{(10\%)}\\
 Free variation & -- & -- & -- & --\\
\lspbottomrule
\end{tabularx}
\caption{\label{tab:7.3}Conditioning environments for allophonic processes resulting in palatalization, labialization, and velarization. A process may have more than one conditioning environment.}
\end{table}

  Because of how these processes have been defined (as particular types of assimilation of consonants to vowels), all have segmental conditioning. Unsurprisingly, palatalization is typically conditioned by high and/or front vowels, labialization by rounded vowels, and velarization by back vowels. Additionally, the domain and stress environments play very minor roles in conditioning these processes (the latter being stronger in palatalization processes).

\subsection{Other processes resulting in fortition}\label{sec:7.3.4}

  Here, I examine processes resulting in glide strengthening and other increased constriction. The percentage of languages having these process types in each syllable structure category is given in \figref{fig:7.8}.

\begin{figure}
\begin{tikzpicture}
\pgfplotstableread{data/fig78.csv}{\table}
    \pgfplotstablegetcolsof{\table}
    \pgfmathtruncatemacro\numberofcols{\pgfplotsretval-1}
            \begin{axis}[easterdayline]
            \foreach \i in {1,...,\numberofcols} {
                \addplot+ table [x index={1},y index={\i},x expr=\coordindex] {\table};
                \pgfplotstablegetcolumnnamebyindex{\i}\of{\table}\to{\colname} % Adding column headers to legend
                \addlegendentryexpanded{\colname}
            }
            \end{axis}                                                                           
\end{tikzpicture}
\caption{\label{fig:7.8}Percentage of languages in each syllable structure complexity category with allophonic processes resulting in articulations associated with strengthening.}
\end{figure}

  Recall from the trends in Figures \ref{fig:7.2} and \ref{fig:7.4} that the prevalence of fortition processes in general was found to slightly decrease with syllable structure complexity. Judging from the pattern in \figref{fig:7.8}, it would appear that the trend was driven by the other increased constriction processes rather than by glide strengthening.

  In \tabref{tab:7.4}, I show how the fortition processes pattern with respect to conditioning environments.

\begin{table}
\begin{tabularx}{\textwidth}{Qccc}
\lsptoprule
 & \multicolumn{2}{c}{Allophonic processes yielding} \\\cmidrule(lr){2-3}
Conditioning & Glide & Other increased  & Total for group\\
environment  & strengthening & constriction \\
 & \textit{N} = 30 & \textit{N} = 34 & \textit{N} = 63 \\\midrule
 Segmental & 17 \textit{(57\%)} & 20 \textit{(59\%)} & 37 \textit{(59\%)}\\
 Domain & 13 \textit{(43\%)} & 12 \textit{(35\%)} & 25 \textit{(40\%)}\\
 Stress & 6 \textit{(20\%)} & -- & 6 \textit{(10\%)}\\
 Free variation & 3 \textit{(10\%)} & 7 \textit{(21\%)} & 10 \textit{(16\%)}\\
\lspbottomrule
\end{tabularx}
\caption{\label{tab:7.4}Conditioning environments for allophonic processes resulting in glide strengthening and other increased constriction. A process may have more than one conditioning environment.}
\end{table}

  We find a very different pattern in this group of processes than in the previous groups examined. While the segmental environment, and particularly a high and/or front vowel, is still the strongest conditioning factor for both process types, the domain environment is involved in conditioning 40\% of the processes. Examining the specific environments, this is most often word-initial or syllable-initial position. Stress is only a conditioning factor for glide strengthening processes.

\subsection{Processes resulting in articulations associated with the Simple category}\label{sec:7.3.5}

  The percentage of languages in each category which have allophonic processes resulting in articulations associated with the Simple category -- flaps/taps or prenasalization -- can be found in \figref{fig:7.9}.

\begin{figure}
\begin{tikzpicture}
\pgfplotstableread{data/fig79.csv}{\table}
    \pgfplotstablegetcolsof{\table}
    \pgfmathtruncatemacro\numberofcols{\pgfplotsretval-1}
            \begin{axis}[easterdayline]
            \foreach \i in {1,...,\numberofcols} {
                \addplot+ table [x index={1},y index={\i},x expr=\coordindex] {\table};
                \pgfplotstablegetcolumnnamebyindex{\i}\of{\table}\to{\colname} % Adding column headers to legend
                \addlegendentryexpanded{\colname}
            }
            \end{axis}                                                                           
\end{tikzpicture}
\caption{\label{fig:7.9}Percentage of languages in each syllable structure complexity category with allophonic processes resulting in articulations associated with lower syllable complexity.}
\end{figure}

  Recall from the analysis in \figref{fig:7.5} that this group of processes showed an erratic pattern with respect to syllable structure complexity. The pattern here which most closely resembles the one for the group of processes as a whole is the flap/tap process type. There were only two languages with processes resulting in prenasalization in the entire sample.

  In \tabref{tab:7.5}, I show how these processes pattern with respect to conditioning environments.

\begin{table}
\begin{tabularx}{\textwidth}{Qccc}
\lsptoprule
 & \multicolumn{2}{c}{Allophonic processes yielding} \\\cmidrule(lr){2-3}
Conditioning environment & Flap/tap & Prenasalization & Total for group\\
 & \textit{N} = 2 & \textit{N} = 25 & \textit{N} = 27 \\\midrule
 Segmental & 18 \textit{(72\%)} & 2 \textit{(100\%)} & 20 \textit{(74\%)}\\
 Domain & 5 \textit{(20\%)} & 1 \textit{(50\%)} & 6 \textit{(22\%)}\\
 Stress & 2 \textit{(8\%)} & -- & 2 \textit{(7\%)}\\
 Free variation & 3 \textit{(12\%)} & -- & 3 \textit{(11\%)}\\
\lspbottomrule
\end{tabularx}
\caption{\label{tab:7.5}Conditioning environments for allophonic processes resulting in prenasalization and flaps/taps. A process may have more than one conditioning environment.}
\end{table}

  The segmental environment is the strongest conditioning factor in this group of processes. In particular, the intervocalic environment conditions most of the processes producing flaps/taps (14/18 processes with segmental conditioning). There is also a secondary effect of domain environment for both types. In these cases, the environment is nearly always word-medial or word-final. Additionally, the two flap/tap processes conditioned by stress occur specifically in unstressed environments.

\subsection{Other processes resulting in lenition or sonorization}\label{sec:7.3.6}

  Other processes resulting in lenition or sonorization, specifically those with outcomes of obstruent voicing, spirantization, debuccalization, or consonants becoming glides or vowels, are examined here. See \figref{fig:7.10} for the percentage of languages in each syllable structure complexity category which have such processes.

\begin{figure}
\begin{tikzpicture}
\pgfplotstableread{data/fig710.csv}{\table}
    \pgfplotstablegetcolsof{\table}
    \pgfmathtruncatemacro\numberofcols{\pgfplotsretval-1}
            \begin{axis}[easterdayline]
            \foreach \i in {1,...,\numberofcols} {
                \addplot+ table [x index={1},y index={\i},x expr=\coordindex] {\table};
                \pgfplotstablegetcolumnnamebyindex{\i}\of{\table}\to{\colname} % Adding column headers to legend
                \addlegendentryexpanded{\colname}
            }
            \end{axis}                                                                           
\end{tikzpicture}
\caption{\label{fig:7.10}Percentage of languages in each syllable structure complexity category with allophonic processes resulting in lenition or sonorization.}
\end{figure}
  Recall that lenition/sonorization processes as a group did not show a coherent trend with respect to syllable structure complexity in Figures \ref{fig:7.3} and \ref{fig:7.5}. In the analysis here, we find that both obstruent voicing and spirantization processes show generally rising, although not linear, trends. These two patterns are strikingly similar to one another. Interestingly, in the sample there is very little overlap in the processes themselves (i.e., processes which have outcomes of both spirantization and obstruent voicing), and only a third of the languages with either of these processes have both of them.

  Processes by which a consonant becomes a glide or vowel do not show a strong trend with respect to syllable structure complexity. Debuccalization is relatively rare in the sample and also does not have a strong pattern with respect to syllable structure complexity.

  The conditioning environments for the various lenition and sonorization processes can be found in \tabref{tab:7.6}.

\begin{table}
\resizebox{\textwidth}{!}{\begin{tabular}{lccccc}
\lsptoprule
 & \multicolumn{4}{c}{Allophonic processes yielding} \\\cmidrule(lr){2-5}
 Conditioning  & Obstruent & Spiranti- & Debuccali- & C > glide, V & Total\\
 environment   & voicing   & zation    & zation     &              & for group\\
 & \textit{N} = 42 & \textit{N} = 39 & \textit{N} = 8 & \textit{N} = 23 & \textit{N} = 111\\\midrule
 Segmental & 33 \textit{(79\%)} & 25 \textit{(64\%)} & 2 \textit{(25\%)} & 18 \textit{(78\%)} & 77 \textit{(69\%)}\\
 Domain & 7 \textit{(17\%)} & 6 \textit{(15\%)} & 3 \textit{(38\%)} & 5 \textit{(22\%)} & 21 \textit{(19\%)}\\
 Stress & 6 \textit{(14\%)} & 1 \textit{(3\%)} & 1 \textit{(13\%)} & -- & 8 \textit{(7\%)}\\
 Free variation & 2 \textit{(5\%)} & 9 \textit{(23\%)} & 2 \textit{(25\%)} & 1 \textit{(4\%)} & 14 \textit{(13\%)}\\
\lspbottomrule
\end{tabular}}
\caption{\label{tab:7.6}Conditioning environments for allophonic processes resulting in spirantization, debuccalization, and consonants becoming glides or vowels.}
\end{table}

  The segmental environment is the strongest conditioning factor for obstruent voicing, spirantization, and consonants becoming glides or vowels. The intervocalic environment is particularly strong for spirantization, where it conditions 17/25 processes with segmental conditioning. As was also the case for the processes in \sectref{sec:7.3.5}, domain environments, typically syllable- or word-final position, are relatively prominent conditioning environments for this group of processes. In fact, for debuccalization processes, the domain environment is the most common conditioning factor.

  In the following section I summarize the trends found for individual process types as they relate to syllable structure complexity.

\subsection{Summary of results} \label{sec:7.3.7}

  The analyses in \sectref{sec:7.3.2}--\ref{sec:7.3.6} indicate that there are some associations between the allophonic processes examined here and syllable structure complexity. I list these associations, and prominent conditioning environments that were identified, in \tabref{tab:7.7} below. I exclude processes resulting in uvulars, velarization, prenasalization, and debuccalization, as these occurred infrequently in the sample (in ten or fewer languages). Recall that for most process types, the segmental environment was the most important conditioning factor. If a process type has another prominent factor conditioning 20\% of processes or more, I list that environment in the third column of the table.

\begin{sidewaystable}
\begin{tabularx}{\textwidth}{lQQQ}
\lsptoprule
 & Outcome of process & Frequency with respect to syllable structure complexity & Prominent non-segmental conditioning environments\\\midrule
HC-associated articulations & \textit{Palato-alveolar} & decreasing & --\\
 & \textit{Affricate} & decreasing & free variation\\\tablevspace
Assimilation of C to V & \textit{Palatalization} & decreasing & --\\
 & \textit{Labialization} & (none) & --\\\tablevspace
Fortition & \textit{Glide strengthening} & (none) & domain-initial, stress\\
& \textit{Other increased constriction} & decreasing & domain-initial, free variation\\\tablevspace
S-associated articulations & \textit{Flap/tap} & (none) & domain-medial or -final\\\tablevspace
Lenition and sonorization & \textit{Obstruent voicing} & increasing & --\\
& \textit{Spirantization} & increasing & free variation\\
& \textit{Consonant} > \textit{vowel or glide} & (none) & domain-final\\
\lspbottomrule
\end{tabularx}
\caption{\label{tab:7.7}Associations between allophonic processes, syllable structure complexity, and prominent non-segmental conditioning environments.}
\end{sidewaystable}

  There are four process types which become more prevalent in the languages of the sample as syllable structure complexity decreases. Two of these process types have outcomes which result in articulations associated with the segmental inventories of languages in the Highly Complex category: \textit{palato-alveolars} and \textit{affricates}. The other two process types -- those resulting in \textit{palatalization} and \textit{other increased constriction} -- are forms of assimilation of consonants to vowels and fortition, respectively. These kinds of sound changes are known to be common sources of the Highly Complex-associated articulations, as discussed in \sectref{sec:4.5.4}. Thus we find that all of the frequent process types examined here which are more prevalent in languages with simpler syllable structure have outcomes which are associated in some way with segmental properties of highly complex syllable structure. This finding is in line with the hypotheses of this chapter. However, within those three larger groups of processes, there are also two frequent process types which do not show a trend with respect to syllable structure complexity: \textit{labialization} and \textit{glide strengthening}. Thus the pattern is not universal.

  There are two frequent process types which show increasing trends with syllable structure complexity: \textit{obstruent voicing} and \textit{spirantization}. Neither of these process types, strictly speaking, produces articulations associated with segmental inventories of the Simple category. However, they are both forms of lenition or sonorization, a family of sound changes relevant to the development of flaps/taps and prenasalized consonants.\footnote{{As noted in \sectref{sec:4.5.3.1}, the relatively rare process of prenasalization may itself sometimes be a strategy for maintaining voicing in obstruent consonants.}} The only frequent process type producing Simple-associated articulations, \textit{flaps/taps}, does not show a trend with respect to syllable structure complexity, nor do processes by which \textit{consonants become glides or vowels}. These results lend some further support to the hypothesis, as they show that there are different distributions of allophonic outcomes associated with segmental properties of the Highly Complex category on the one hand and the Simple category on the other.

  There are also some associations in the data between process types and secondary conditioning environments. As shown in the previous sections, most of the process types are primarily conditioned by segmental environments. However, there are some interesting patterns with respect to the secondary conditioning factors. For processes producing Highly Complex-associated articulations, assimilation of consonants to vowels, and fortition, other prevalent conditioning environments include domain-initial position (for \textit{glide strengthening} and \textit{other increased constriction}), free variation (for \textit{affrication} and \textit{other increased constriction}), and stress (for \textit{glide strengthening}). For processes producing Simple-associated articulations and lenition or sonorization, other prevalent conditioning factors include domain-medial or -final position (for \textit{flap/tap} and \textit{consonants becoming glides or vowels}) and free variation (for \textit{spirantization}). 

  The major divide in process types is accompanied by differences in conditioning factors. However, these non-segmental conditioning factors are frequent for only three of the process types that show a strong trend with respect to syllable structure complexity: other increased constriction, affrication, and spirantization, and for the latter two the prevalent secondary conditioning factor is free variation. Thus, it is difficult to relate these conditioning results directly to syllable structure complexity.

  As the previous analyses show, stress is not a strong conditioning factor for the allophonic processes examined here. In fact, it is prevalent for only one process type, glide strengthening. This is very different from the results obtained in \chapref{sec:6} for vowel reduction. To illustrate, in \figref{fig:7.11}, I show the percentage of allophonic consonant processes in each syllable structure complexity category which are conditioned by stress.

  
\begin{figure}
\begin{tikzpicture}
\pgfplotstableread{data/fig711.csv}{\table}
    \pgfplotstablegetcolsof{\table}
    \pgfmathtruncatemacro\numberofcols{\pgfplotsretval-1}
            \begin{axis}[easterdayline,legend style={font=\footnotesize,anchor=north west,text width=3.75cm,minimum height=.9\baselineskip},]
            \foreach \i in {1,...,\numberofcols} {
                \addplot+ table [x index={1},y index={\i},x expr=\coordindex] {\table};
                \pgfplotstablegetcolumnnamebyindex{\i}\of{\table}\to{\colname} % Adding column headers to legend
                \addlegendentryexpanded{\colname}
            }
            \addlegendimage{empty legend}\addlegendentry{}
            \end{axis}                                                                           
\end{tikzpicture}
\caption{\label{fig:7.11}Percentage of languages with consonant allophony process types examined here conditioned by word stress.}
\end{figure}

  Note that the pattern in \figref{fig:7.11} is subtler than the analogous one in \figref{fig:7.1} (also that the scale has been reduced to 0-40\% for \figref{fig:7.11}). That is because the pattern in \figref{fig:7.1} includes languages with \textit{any} reported processes of stress-conditioned consonant allophony in the sample, including the very common processes of aspiration and consonant lengthening. The consonant processes considered here are limited to those relevant to the hypotheses, and stress is evidently not a strong conditioning environment for them.

  In \tabref{tab:7.8}, I show the number of processes conditioned by stress in each category, and the number of languages with stress-conditioned processes. In the bottom row I give the ratio of processes to languages.

\begin{table}
\begin{tabularx}{\textwidth}{Qcccc}
\lsptoprule
 & \multicolumn{4}{c}{Syllable structure complexity}\\\cmidrule(lr){2-5}
 & S & MC & C & HC\\\midrule
 \textit{N} {allophonic C processes conditioned by stress} & 10 & 4 & 9 & 1\\
 \textit{N} {lgs. with allophonic C processes conditioned by stress} & 4 & 2 & 6 & 1\\
 ratio & 2.5 & 2 & 1.5 & 1\\
\lspbottomrule
\end{tabularx}
\caption{\label{tab:7.8}Ratio of number of stress-conditioned vowel reduction processes to the number of languages with unstressed vowel reduction in each category of syllable structure complexity.}
\end{table}

	Within the small set of languages with stress-conditioned consonant processes of the types examined here, the average number of such processes per language decreases with syllable structure complexity. That is, stress conditions more consonant processes in languages with simpler syllable structure. This result is essentially a reversal of the pattern found in \sectref{sec:6.3.4} for stress-conditioned vowel reduction. In that analysis, it was found that in languages with unstressed vowel reduction, the average number of such processes increased with syllable structure complexity.

  In \tabref{tab:7.9}, I show the distribution of stress-conditioned processes by process type and syllable structure complexity. Note that processes producing uvulars, velarization, prenasalization, and glides or vowels from consonants were not reported to be conditioned by stress in the sample.

\begin{table}
\begin{tabular}{lcccc}
\lsptoprule
 & \multicolumn{4}{c}{Syllable structure complexity}\\\cmidrule(lr){2-5}
 Process type & S & MC & C & HC\\
             & \textit{N} = 10 & \textit{N} = 4 & \textit{N} = 9 & \textit{N} = 1\\\midrule
 Affrication & 4 & 2 & -- & --\\
 Palato-alveolar & 2 & -- & 1 & --\\
 Palatalization & 1 & -- & 1 & --\\
 Labialization & -- & -- & -- & 1\\
 Glide strengthening & 3 & 1 & 3 & --\\
 Other increased constriction & -- & -- & 1 & --\\
 Voiced obstruent & 2 & -- & 3 & --\\
 Flap/tap & -- & 1 & 1 & --\\
 Spirantization & 1 & -- & -- & --\\
 Debuccalization & 1 & -- & -- & --\\
\lspbottomrule
\end{tabular}
\caption{\label{tab:7.9}Processes of consonant allophony conditioned by stress. Some processes have several outcomes (e.g. palato-alveolar and affricate).}
\end{table}

  Besides having more stress-conditioned consonant processes per language, the Simple category is also associated with the greatest diversity in outcomes from such processes. For example, in \ili{Pinotepa Mixtec} stress-conditioned consonant allophony results in palato-alveolar affricates, glide strengthening, and obstruent voicing.

\section{Discussion}\label{sec:7.4}

  In \sectref{sec:7.1}, two hypotheses were formulated regarding expected patterns of consonant allophony in the sample. Following from observations of stress-conditioned consonant allophony in \chapref{sec:5}, it was thought that the articulations found to be associated with the segmental inventories of languages in the Highly Complex category in \chapref{sec:4} might have their origin in allophonic processes at some earlier stage in the languages, perhaps even before the complex syllable patterns developed. It was therefore hypothesized that allophonic processes resulting in these articulations, or associated sound changes such as assimilation of consonants to vowels and fortition, would be most prevalent in languages with simpler syllable structures. By contrast, it was expected that allophonic processes producing articulations established in \chapref{sec:4} to be associated with Simple syllable structure, and other kinds of lenition or sonorization, would show a different trend, perhaps increasing in prevalence with syllable structure complexity or remaining constant across the categories.

  The results here point to some support for the hypotheses, but also a more complex situation than what was predicted. Four of the nine process types examined here which have outcomes of Highly Complex-associated articulations, general assimilation of consonants to vowels, or fortition were found to have the predicted pattern. Two had trends with no apparent relationship to syllable structure complexity, and three were either very infrequent or unattested in the language sample. Of the six process types with outcomes of Simple-associated articulations, lenition, or sonorization, two had increasing trends with syllable structure complexity, two had trends with no apparent relationship to syllable structure complexity, and two were infrequent in the language sample.

  Loose associations between conditioning environments and the patterns observed lend further support to the hypotheses by indicating that these patterns may not be random, but instead have coherent motivations. High and/or front vowels, domain-initial environments, and in one case stress, are relevant conditioning factors for processes resulting in Highly Complex-associated articulations and related sound changes. Intervocalic, domain-medial, and domain-final environments are important in conditioning processes resulting in Simple-as\-so\-ci\-at\-ed articulations and related sound changes. 

  I discuss some possible diachronic implications of these patterns in the following section.

\subsection{Consonant allophony and the development of syllable structure complexity} \label{sec:7.4.1}

  In \sectref{sec:7.1}, it was stated that if the hypotheses regarding consonant allophony were borne out in the data, this might illuminate elements of the diachronic path by which highly complex syllable structure develops. Specifically, it was thought that comparing these patterns against the vowel reduction patterns established in \chapref{sec:6} might reveal information about the relative order of certain processes of change in the development of this language type.

  The results reveal a greater prevalence of some process types associated with the segmental properties of the Highly Complex category as syllable structure complexity decreases. Palato-alveolars, affricates, palatalization, and increased constriction are most commonly outcomes of allophonic processes in languages of the Simple category. If we liberally assume that all of these allophonic processes are heading towards phonemicization and that all of these languages are heading towards higher syllable complexity, that would indicate that some of the articulations associated with the Highly Complex category may start to develop quite some time before complex syllable patterns develop out of vowel deletion. An additional scenario is suggested by the opposing patterns with respect to prevalence of stress-conditioned consonant allophony and stress-conditioned vowel reduction. Stress conditions more consonant processes in languages with simpler syllable structure and more vowel reduction processes in languages with complex syllable structure. If syllable structure complexity is a diachronic cline, the stress-conditioned patterns could indicate that in early processes of syllable structure change, stress has stronger effects on consonants than on vowels. Additionally, both of these scenarios could suggest that the processes affecting consonants develop concurrently with the initial stages of vowel reduction in languages. In this scenario, we might expect a large amount of overlap between languages with the relevant consonant allophony and languages with phonetic vowel reduction.

  I tested this scenario in the current language sample. \tabref{tab:7.10} shows the languages in the Simple category reported to have vowel reduction cross-tabulated against those reported to have consonant processes producing the relevant articulations (that is, those found here to bear a relationship to syllable structure complexity: palato-alveolars, affricates, palatalization, and other increased constriction). \tabref{tab:7.11} shows the same analysis for stress-conditioned processes only.

\begin{table}
\begin{tabular}{lcc}
\lsptoprule
 {V reduction processes} & \multicolumn{2}{c}{Relevant C allophony}\\\cmidrule(lr){2-3}
                 & {Present} & {Absent}\\\midrule
 {Present} & 12 & 3\\
 {Absent} & 5 & 4\\
\lspbottomrule
\end{tabular}
\caption{\label{tab:7.10}Languages of Simple category, distributed according to presence or absence of vowel reduction processes and consonant allophony resulting in palato-alveolars, affricates, palatalization, or increased constriction.}
\end{table}




\begin{table}
\begin{tabularx}{\textwidth}{Qcc}
\lsptoprule
Stress-conditioned V reduction processes & \multicolumn{2}{c}{Stress-conditioned relevant C allophony}\\\cmidrule(lr){2-3}
 & {Present} & {Absent}\\\midrule
 {Present} & 2 & 5\\
 {Absent} & 1 & 10\\
\lspbottomrule
\end{tabularx}
\caption{\label{tab:7.11}Languages of Simple category with word stress, distributed according to presence or absence of stress-conditioned vowel reduction processes and consonant allophony resulting in palato-alveolars, affricates, palatalization, or increased constriction.}
\end{table}

  Of the two analyses, the pattern in \tabref{tab:7.10} comes closer to approximating the relationship we would expect if syllable structure-changing vowel reduction develops concurrently with consonant allophony producing Highly Complex-associated articulations. Though the presence of vowel reduction tends to imply the presence of the relevant consonant processes (12/15 languages with vowel reduction) and vice versa (12/17 languages with consonant allophony), the pattern in the small data set in \tabref{tab:7.10} is not statistically significant in Fisher’s exact test. Because the number of languages with stress-conditioned processes is so small in the Simple category (occurring in only 8/18 languages with word stress), it is difficult to determine whether stress has stronger effects on consonants or vowels. However, the pattern in \tabref{tab:7.11} shows that for the processes targeted here, stress is more likely to condition vowel reduction than consonant allophony. Thus there is not strong evidence for a scenario by which stress-conditioned vowel reduction is preceded by stress-conditioned consonant allophony in the early stages of syllable structure change.

  Of course, the above scenario is grossly overgeneralized. We should not expect all processes resulting in specific kinds of consonant allophony to be part of a larger process of syllable structure change in a language, least of all processes of palatalization and affrication, which are extremely prevalent crosslinguistically (\citealt{Bhat1978,Bateman2007}). Furthermore, the analyses above indicate that stress is a much less relevant conditioning factor for the consonant processes under examination here than it is for vowel reduction (and recall from the discussion in \sectref{sec:6.4.2} that the role of stress in vowel reduction is not universal). Nevertheless, the tendencies in the data with respect to these processes, and the differences between their distributions and distributions of prototypical lenition and sonorization processes, make it tempting to draw connections between consonant allophony and syllable structure complexity. \ili{Even} if the relationship is entirely coincidental, it is very interesting that the group of processes which are more prevalent at the simpler end of the syllable structure complexity scale tend to result in some of the segmental articulations associated with more complex syllable structure.

  Two of the frequent process types resulting in Simple-associated articulations and lenition/sonorization show a rising trend with respect to syllable structure complexity, while two others show no clear trend. An interesting question is why the outcomes of such processes -- in particular, prenasalization and flapping -- correspond to segmental contrasts more often in languages with simpler syllable structure. One clue to this could be in the conditioning environments. Intervocalic conditioning environments were found almost exclusively in processes of these types. Again, in an oversimplified scenario, this may indicate that such patterns, though prevalent in all categories, are more likely to phonologize in languages in which intervocalic environments are more consistently present, as they would be in languages with simpler syllable structure.

  While the analyses in this chapter do not contribute any definitive results directly linking consonant allophony to the development of highly complex syllable structure, they perhaps provide a few helpful clues. These results will be revisited in the next chapter, in which the results of the studies in Chapters 3-7 are summarized and given a diachronic interpretation.

